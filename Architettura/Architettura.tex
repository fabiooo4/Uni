\documentclass[a4paper]{article}

\usepackage[utf8]{inputenc}
\usepackage[T1]{fontenc}
\usepackage{textcomp}
\usepackage[italian]{babel}
\usepackage{amsmath, amssymb}
\usepackage{amsfonts}
\usepackage{mdframed}
\usepackage{ntheorem}
\usepackage{xcolor}
\usepackage{graphicx}
\graphicspath{{./figures/}}

\usepackage{import}
\usepackage{pdfpages}
\usepackage{transparent}
\usepackage{xcolor}

% Useful definitions frame
\theoremstyle{break}
\theoremheaderfont{\bfseries}
\newmdtheoremenv[%
	linecolor=gray,leftmargin=0,%
	rightmargin=0,
	innertopmargin=8pt,%
	ntheorem]{define}{Definizioni utili}[section]

% Example frame
\theoremstyle{break}
\theoremheaderfont{\bfseries}
\newmdtheoremenv[%
	linecolor=gray,leftmargin=0,%
	rightmargin=0,
	innertopmargin=8pt,%
	ntheorem]{example}{Esempio}[section]

% Important definition frame
\theoremstyle{break}
\theoremheaderfont{\bfseries}
\newmdtheoremenv[%
	linecolor=gray,leftmargin=0,%
	rightmargin=0,
	backgroundcolor=gray!40,%
	innertopmargin=8pt,%
	ntheorem]{definition}{Definizione}[section]

% Exercise frame
\theoremstyle{break}
\theoremheaderfont{\bfseries}
\newmdtheoremenv[%
	linecolor=gray,leftmargin=0,%
	rightmargin=0,
	innertopmargin=8pt,%
	ntheorem]{exercise}{Esercizio}[section]


% figure support
\usepackage{import}
\usepackage{xifthen}
\pdfminorversion=7
\usepackage{pdfpages}
\usepackage{transparent}
\newcommand{\incfig}[1]{%
	\def\svgwidth{\columnwidth}
	\import{./figures/}{#1.pdf_tex}
}

\pdfsuppresswarningpagegroup=1

\begin{document}
\begin{titlepage}
	\begin{center}
		\vspace*{1cm}

		\Huge
		\textbf{Probabilità e Statistica\\Esercizi}

		\vspace{0.5cm}
		\LARGE
		UniVR - Dipartimento di Informatica

		\vspace{1.5cm}

		\textbf{Fabio Irimie}

		\vfill


		\vspace{0.8cm}


		2° Semestre 2023/2024

	\end{center}
\end{titlepage}


\tableofcontents
\pagebreak

\section{Numeri razionali a virgola mobile}
Gli standard della virgola mobile sono: IEEE 754/85 e IEEE 754/19. Questo standard
è stato rivisto molte volte e ora viene usato da tutte le codifiche per i numeri in
virgola mobile.

Il numero viene separato in due parti: Mantissa (\( M \)) e una base (\(b\))
con un esponente ($e$).
\[
	N = +- M * b^{+-E}
\]
Questo permette di dividere il numero in modo da poter scegliere quanti bit dedicare
alla mantissa e quanti all'esponente.

Ci sono 2 problemi però:
\begin{itemize}
	\item bisogna scegliere la base in cui fare la codifica (base 2)
	\item divisione bit tra M e E (23 M, 8 E, 1 S)
	\item rappresentazione univoca (1.0...)
	\item bisogna trovare un modo per rappresentare gli errori
\end{itemize}

Un numero in base 10 si può rappresentare in più modi>
\(120_{10} = 12*10^{1} = 120 * 10^{0} = 1.20 * 10^{2}\)

Se la mantissa e la base sono in base 2 le operazioni tra numeri sono agevolate.

\(0110 * 2 = 1100\) è uno shift a sinistra in binario.

\(1010/2 = 0101\) è uno shift a destra in binario.

\subsection{Divisione tra bit con mantissa e base diversa}
Un numero è rappresentabile in 2 modi:
\begin{itemize}
	\item 32 bit (singola precisione / float)
	\item 64 bit (doppia precisione / double)
\end{itemize}

Prendiamo in considerazione 32 bit, ora dobbiamo decidere quanti bit dedicare alla
mantissa e alla base

$2^{+-E}$\\
$|E| = 4 bit = 2^{+7}$\\
$5 bit = 2^{+15}$\\
$6 bit = 2^{+31}$\\
$7 bit = 2^{+63}$\\
$8 bit = 2^{+127}$

L'impatto dei bit sull'esponente è doppiamente esponenziale, quindi cresce tantissimo.
Tra tutti i bit a disposizione ne dedichiamo 8 all'esponente, $32 - 8 = 24$ bit rimanenti,
quindi 23 bit vengono assegnati alla mantissa e 1 bit viene assegnato al segno.
\label{fig:disegno1}

Per la rappresentazione univoca la mantissa si codifica in virgola fissa.
Cioè si parte da una mantissa con un punto fisso e dividendo o moltiplicando (shift) si
può spostare la virgola per arrivare alla forma 1.00000... e questa forma è la
rappresentazione univoca. Questa operazioe si chiama normalizzazione e visto che la
rappresentazione è sempre la stessa l'1. non viene rappresentato, quindi
viene inserito nella mantissa solo tutto ciò che viene dopo l'1. .
\label{fig:disegno2}

Se lavorassimo con un esponente in complemento a due ci sarebbe il seguente problema:
$00000000000...0 = 1*2^{0}=1$

Allora si è deciso di codificare l'esponente in Eccesso 127. Quindi per rappresentare
lo zero si usa come esponente il minore numero possibile: $1*2^{127} = 0$
Per codificare i numeri si somma 127 al numero desiderato e visto che i numeri possibili
ora vanno da -127 a +127 se codifichiamo il risultato in modulo avremo dei numeri
da 0 a 256.

\begin{example}
	$1\:01110111\:0110...0$\\
	$M = -(1+1/4+1/8)*2 = -(11/8)*2^{E}$\\
	$E = (1+2+4+16+32+64)-127=119-127=-8$\\
	$N = -11/8 * 2^{-8}$
\end{example}
\begin{exercise}
	Codifica $+(4+\frac{1}{2}+\frac{1}{16})*2^{+34}$
\end{exercise}

\begin{itemize}
	\item $0\:00000000\:0...0 = +0$
	\item $1\:00000000\:0...0 = -0$
\end{itemize}

Quando l'esponente è tutto 1 e la mantissa tutta 0 allora equivale a $infinito$
+ o - in base al primo bit. Se invece la mantissa è diversa da 0 con esponente tutti 1
allora rappresenta un errore NaN.

Somma:\\
\label{es2}

\end{document}
