\documentclass[a4paper]{article}

\usepackage[utf8]{inputenc}
\usepackage[T1]{fontenc}
\usepackage{textcomp}
\usepackage[italian]{babel}
\usepackage{amsmath, amssymb}
\usepackage{amsfonts}
\usepackage{mdframed}
\usepackage{float}
\usepackage{ntheorem}
\usepackage{xcolor}
\usepackage{graphicx}
\usepackage{tikz}
\graphicspath{{./figures/}}

\usepackage{ntheorem}
\newtheorem{theorem}{Teorema}

\usepackage{import}
\usepackage{pdfpages}
\usepackage{transparent}
\usepackage{xcolor}

% Useful definitions frame
\theoremstyle{break}
\theoremheaderfont{\bfseries}
\newmdtheoremenv[%
	linecolor=gray,leftmargin=0,%
	rightmargin=0,
	innertopmargin=8pt,%
	innerbottommargin=8pt,
	ntheorem]{define}{Definizioni utili}[section]

% Example frame
\theoremstyle{break}
\theoremheaderfont{\bfseries}
\newmdtheoremenv[%
	linecolor=gray,leftmargin=0,%
	rightmargin=0,
	innertopmargin=8pt,%
	innerbottommargin=8pt,
	ntheorem]{example}{Esempio}[section]

% Important definition frame
\theoremstyle{break}
\theoremheaderfont{\bfseries}
\newmdtheoremenv[%
	linecolor=gray,leftmargin=0,%
	rightmargin=0,
	backgroundcolor=gray!40,%
	innertopmargin=8pt,%
	innerbottommargin=8pt,
	ntheorem]{definition}{Definizione}[section]

% Exercise frame
\theoremstyle{break}
\theoremheaderfont{\bfseries}
\newmdtheoremenv[%
	linecolor=gray,leftmargin=0,%
	rightmargin=0,
	innertopmargin=8pt,%
	innerbottommargin=8pt,
	ntheorem]{exercise}{Esercizio}[section]

% figure support
\usepackage{import}
\usepackage{xifthen}
\pdfminorversion=7
\usepackage{pdfpages}
\usepackage{transparent}
\newcommand{\incfig}[1]{%
	\def\svgwidth{\columnwidth}
	\import{./figures/}{#1.pdf_tex}
}

\pdfsuppresswarningpagegroup=1

\begin{document}
\begin{titlepage}
	\begin{center}
		\vspace*{1cm}

		\Huge
		\textbf{Probabilità e Statistica\\Esercizi}

		\vspace{0.5cm}
		\LARGE
		UniVR - Dipartimento di Informatica

		\vspace{1.5cm}

		\textbf{Fabio Irimie}

		\vfill


		\vspace{0.8cm}


		2° Semestre 2023/2024

	\end{center}
\end{titlepage}


\tableofcontents
\pagebreak
\section{Introduzione}
L'informatica è nata per la risoluzione i problemi di calcolo, in particolare
quelli di calcolo numerico. Per questo motivo i primi computer erano macchine
che eseguivano operazioni aritmetiche. Per risolvere questi problemi si usano
degli algoritmi che sono una sequenza di istruzioni semplici che portano poi
a risolvere problemi di complessità variabile. Anche gli algoritmi hanno una
complessità che deve essere adeguata alla risoluzione del problema.

\subsection{Hardware}
Un algoritmo deve essere trasformato in un processo di calcolo automatico,
quindi deve essere implementato tramite hardware. Ci sono due tipi di hardware:
\begin{itemize}
	\item \textbf{Embedded} che è un hardware dedicato ad un singolo compito.
	      Ad esempio il microonde.
	\item \textbf{General purpose} non si sa l'utilizzo finale, quindi ha
	      funzionalità generali ampliate dal software installato. L'hardware
	      general purpose è programmabile attraverso il software. Un esempio
	      è il PC.
\end{itemize}

In base al tipo di hardware l'algoritmo viene implementato in diversi modi:
\begin{itemize}
	\item \textbf{Algoritmo} \( \to  \) \textbf{Software}: Tramite un linguaggio di programmazione
	\item \textbf{Algoritmo} \( \to  \)  \textbf{Hardware embedded}: Tramite linguaggi di basso livello
	      come C, Assembly o il sistema operativo.
	\item \textbf{Algoritmo} \( \to  \)  \textbf{Hardware}: Tramite sintesi logica
\end{itemize}

\subsection{Campionamento dei dati}
Ogni cosa nel mondo è rappresentabile da funzioni continue nel tempo \( f(t) \),
ma con risorse finite è impossibile rappresentare infiniti dati, bisogna quindi
campionarli.

\begin{figure}[h]
	\label{fig:f(t)}
	\centering
	\begin{tikzpicture}[scale=0.6, domain=0:10]
		\coordinate (A) at (0,4);
		\coordinate (B) at (1,4);
		\coordinate (C) at (2,2);
		\coordinate (D) at (3,4);
		\coordinate (E) at (4,1);
		\coordinate (F) at (5,3);
		\coordinate (G) at (6,2);
		\coordinate (H) at (7,4);
		\coordinate (I) at (8,3);
		\coordinate (J) at (9,2);
		\coordinate (K) at (10,5);

		\draw [->] (0,0) -- (10,0) node[right] {$t$};
		\draw [->] (0,0) -- (0,5) node[above] {$f(t)$};

		\draw [gray!50, ultra thin] (0,0) grid (10,5);
		\draw [blue, ultra thick] plot [smooth, tension=1] coordinates { (A) (B) (C) (D) (E) (F) (G) (H) (I) (J) (K) };
		\draw [red, thick ] (A) -- (B) -- (C) -- (D) -- (E) -- (F) -- (G) -- (H) -- (I) -- (J) -- (K);

		\draw [fill] (A) circle [radius=0.1];
		\draw [fill] (B) circle [radius=0.1];
		\draw [fill] (C) circle [radius=0.1];
		\draw [fill] (D) circle [radius=0.1];
		\draw [fill] (E) circle [radius=0.1];
		\draw [fill] (F) circle [radius=0.1];
		\draw [fill] (G) circle [radius=0.1];
		\draw [fill] (H) circle [radius=0.1];
		\draw [fill] (I) circle [radius=0.1];
		\draw [fill] (J) circle [radius=0.1];
		\draw [fill] (K) circle [radius=0.1];

		\draw (0, -0.2) -- (1, -0.2) node[below, xshift=-10] {\( \Delta t \) };
	\end{tikzpicture}
	\caption{Funzione casuale continua nel tempo}
\end{figure}
Per campionare la funzione nella figura \ref{fig:f(t)} bisogna scegliere un intervallo di tempo \( \Delta t \) e prendere
un valore della funzione ogni \( \Delta t \). In questo caso le linee
verticali rappresentano il \textbf{campionamento}, mentre quelle orizzontali
reppresentano la \textbf{discretizzazione o quantizzazione}.
La linea rossa è una spezzata approssimata della funzione continua, infatti
per il teorema di Shannon:

\begin{theorem}
	Deciso il grado di errore da voler compiere, esistono una precisa frequenza di
	campionamento e un intervallo di discretizzazione che garantiscono
	quell'errore.
\end{theorem}
Il sistema di calcolo è ora diventato digitale, cioè elabora i segnali numerici
in ingresso per produrre segnali numerici in uscita.

\begin{figure}[h]
	\centering
	\begin{tikzpicture}
		\node[draw, text=red, align=center] (Realtà fisica) at (0,0) {Realtà\\
			fisica};
		\node[draw, align=center] (Campionamento e discretizzazione) at (2,-1.5) {Campionamento e\\
			discretizzazione};
		\node[draw, text=blue, align=center] (Codifica) at (4,0) {Codifica};
		\node[draw, align=center] (Sistema digitale) at (6,-1.5) {Sistema\\
			digitale};
		\node[draw, text=blue, align=center] (Decodifica) at (8,0) {Decodifica};
		\node[draw, text=red, align=center] (Informazioni) at (9,-1.5) {Informazioni};

		\draw[->,draw] (Realtà fisica) to (Campionamento e discretizzazione);
		\draw[->,draw] (Campionamento e discretizzazione) to (Codifica);
		\draw[->,draw] (Codifica) to (Sistema digitale);
		\draw[->,draw] (Sistema digitale) to (Decodifica);
		\draw[->,draw] (Decodifica) to (Informazioni);
	\end{tikzpicture}
	\caption{Dalla realtà fisica al sistema digitale}
\end{figure}


\section{Sistemi di codifica}
Ogni sistema digitale lavora in base binaria, quindi entrano \( N \)  bit
ed escono \( M \)  bit. I bit in uscita devono essere codificati per
realizzare delle informazioni. Ci sono 2 tipi di informazioni:

\begin{itemize}
	\item \textbf{Informazioni intelleggibili}: sono già chiare agli esseri umani,
	      come un testo scritto.
	\item \textbf{Informazioni non intelleggibili}: hanno bisogno di macchine
	      per essere riprodotte, come le casse per l'audio.
\end{itemize}

\subsection{Codifica di informazioni non numeriche}
Ogni informazione deve avere un codice univoco in modo che il sistema
digitale non possa sbagliare a decodificarla. Date \( M \)  informazioni si
ricavano \( n = log_2{(M)} \)  codici disponibili per rappresentarle.

\begin{example}
	Con \( M=7 \) informazioni:
	\begin{itemize}
		\item \( n=log_2{(7)} \approx 3\; bit \)
		\item \( 2^3=8 \) codici disponibili
	\end{itemize}
\end{example}

\subsection{Numeri interi assoluti}
I numeri interi assoluti rappresentano solo i valori da \( 0 \) a \( 2^n-1 \),
dove \( n \) è il numero di bit disponibile.

La codifica da base decimale a base binaria prende il nome di \textbf{codifica
	a modulo}

\begin{example}
	\label{ex:57modulo}
	Si deve convertire il numero \( 57_{10} \) in base binaria
	\begin{center}
		\( n=log_2{(57)} = 6 \) bit (minimi)\\
		\( \sum_{i=1}^{n-1} 2^n-1 = 63 \) (codici massimi)
	\end{center}
	Si eseguono i seguenti passaggi:
	\begin{enumerate}
		\item Si sottraggono le potenze di 2 partendo da \( n-1 \).
		      \begin{itemize}
			      \item Se la potenza \( 2^i \) è minore o uguale del numero,
			            allora si moltiplica per 1.
			      \item Se la potenza \( 2^i \) è maggiore del numero,
			            allora si moltiplica per 0.
		      \end{itemize}
		\item Le sottrazioni continuano fino a quando si giunge a 0.
	\end{enumerate}
	\(57_{10}-{\textcolor{cyan}{1}}*2^{\textcolor{red}{5}}
		=25_{10}-{\textcolor{cyan}{1}}*2^{\textcolor{red}{4}}
		=9_{10}-{\textcolor{cyan}{1}}*2^{\textcolor{red}{3}}
		=1_{10}-{\textcolor{cyan}{0}}*2^{\textcolor{red}{2}}
		=1_{10}-{\textcolor{cyan}{0}}*2^{\textcolor{red}{1}}
		=1_{10}-{\textcolor{cyan}{1}}*2^{\textcolor{red}{0}}\)
\end{example}

\subsection{Numeri interi relativi}
La codifica più ovvia per i numeri interi relativi è la codifica a
\textbf{modulo + segno}. Tuttavia rappresenta varie problematiche, per cui
si preferisce usare la codifica in \textbf{complemento a 2}.

\subsubsection{Codifica a modulo + segno}
\begin{center}
	Intervallo: \( -2^{n-1} \le N \le 2^{n-1}-1 \)
\end{center}
Il segno si rappresenta con un bit, 0 per il positivo e 1 per il negativo.
Il bit più significativo è il bit del segno, mentre i bit meno significativi
rappresentano il modulo.

\begin{center}
	\begin{tikzpicture}
		\draw[draw] (0, 0) rectangle (2,1) node[pos=.5, align=center] {1 bit:\\
				segno \( \pm \) };
		\draw[draw] (2, 0) rectangle (7,1) node[pos=.5, align=center] {7 bit: modulo};
	\end{tikzpicture}
\end{center}
Considerando l'esempio \ref{ex:57modulo} si hanno le seguenti rappresentazioni:

\begin{center}
	\( +57_{10}=\textbf{0}|111001_2 \)\\
	\( -57_{10}=\textbf{1}|111001_2 \)
\end{center}
Sorge però un problema quando si vuole rappresentare il valore \( 0_{10} \),
che in binario risulterebbe:

\begin{center}
	\( +0_{10}=\textbf{0}|000000_2 \)\\
	\( -0_{10}=\textbf{1}|000000_2 \)
\end{center}
Inoltre le somme che passano dal positivo al negativo e viceversa risultano errate.

\subsubsection{Codifica in complemento a 2}
\begin{center}
	Intervallo: \( -2^{n-1} \le N \le 2^{n-1}-1 \)
\end{center}
La codifica in complemento a 2 rimuove tutti i problemi della codifica in modulo
+ segno. Questa codifica infatti rende le somme molto più semplici. La somma facile
infatti è l'obiettivo di questa codifica e parte dell'idea di trovare la
codifica di -1, pertanto si cerca di formulare \( -1+1=0 \).

\begin{center}
	\begin{tabular}{ c|c }
		Obiettivo            & Risultato      \\
		\hline                                \\
		\( ????_2 \) \( + \) & \( 1111_2 + \) \\
		\( 0001_2 = \)       & \( 0001_2 = \) \\ [2ex]
		\hline                                \\
		\( 0000_2 = \)       & \( 0000_2 \)   \\
	\end{tabular}
\end{center}

Se si considera il numero di bit \( n=4 \), allora l'intervallo di valori è
\( -2^3 \le N \le 2^3-1 \):

\begin{center}
	\begin{tabular}{c|c}
		\( 0_{10} = 0000_{2}\) & \( -1_{10} = 1111_{2}\) \\
		\( 1_{10} = 0001_{2}\) & \( -2_{10} = 1110_{2}\) \\
		\( 2_{10} = 0010_{2}\) & \( -3_{10} = 1101_{2}\) \\
		\( 3_{10} = 0011_{2}\) & \( -4_{10} = 1100_{2}\) \\
		\( 4_{10} = 0100_{2}\) & \( -5_{10} = 1011_{2}\) \\
		\( 5_{10} = 0101_{2}\) & \( -6_{10} = 1010_{2}\) \\
		\( 6_{10} = 0110_{2}\) & \( -7_{10} = 1001_{2}\) \\
		\( 7_{10} = 0111_{2}\) & \( -8_{10} = 1000_{2}\) \\
	\end{tabular}
\end{center}
I valori nel complemento a 2 ciclano, quindi se si somma 1 a 7 si ottiene -8.

\begin{example}
	Sottrazione con il complemento a 2: \( 43-17=25 \)
	\[
		n=7 \; bit
	\]
	\begin{enumerate}
		\item Per prima cosa si prende il valore assoluto del numero negativo
		      \( 17_{10} \) e si converte in binario.
		      \begin{center}
			      \( 17_{10}=0010001_{2} \)
		      \end{center}
		\item Si inverte il numero trovato.
		      \begin{center}
			      \( !(0010001_2) = 1101110_2 = -18_{10} \)
		      \end{center}
		\item Si somma 1 al numero trovato.
		      \begin{center}
			      \begin{tabular}{l}
				      \( 1101110\; + \) \\
				      \( 0000001 = \)   \\
				      \hline
				      \( 1101111 \)
			      \end{tabular}\\
			      \( 1101111_2 = -17_{10} \)
		      \end{center}
		\item Si somma il numero trovato al numero positivo.
		      \begin{center}
			      \begin{tabular}{l}
				      \( 0010001\; + \) \\
				      \( 1101111 = \)   \\
				      \hline
				      \( \textbf{1}0011010 \)
			      \end{tabular}
		      \end{center}
		\item Il risultato ottenuto è: \[
			      \textbf{1}0011010
		      \] Si osserva che c'è un bit in più rispetto a quelli disponibili (quello
		      in grassetto),
		      vuol dire che risulta in overflow\footnote{Indica il "traboccamento",
			      cioè se viene superato il limite massimo l'overfflow è un errore,
			      non perchè sia sbagliata la somma, ma perchè il risultato non è codificabile
			      con il numero di bit disponibili}, quindi si scarta il bit più significativo e
		      si ottiene:\[
			      0011010_2 = 26_{10}
		      \] che è il risultato corretto.
	\end{enumerate}
\end{example}

\paragraph{Estensione del numero con il complemento a 2}
\begin{itemize}
	\item Se un numero è \textbf{positivo} va esteso con gli \( \textbf{0} \)
	      \begin{center}
		      \begin{tabular}{l|l}                                        \\
			      \( +57_{10}+ \)  & \( 0111001_2\;+ \)        \\
			      \( +7_{10}\;= \) & \( \textbf{000}1001_2= \) \\ \\
			      \hline                                       \\
			      \( +64_{10} \)   & \( 1000010_2 \)
		      \end{tabular}
	      \end{center}
	\item Se un numero è \textbf{negativo} va esteso con gli \( \textbf{1} \)
	      \begin{center}
		      \begin{tabular}{l|l}                                        \\
			      \( +57_{10}+ \)  & \( 0111001_2\;+ \)        \\
			      \( -7_{10}\;= \) & \( \textbf{111}1001_2= \) \\ \\
			      \hline                                       \\
			      \( +50_{10} \)   & \( 10110010_2 \)
		      \end{tabular}
	      \end{center}
\end{itemize}

\section{Numeri razionali}
I numeri razionali sono composti da una parte intera e una parte frazionaria.
Si possono codificare in 2 modi:
\begin{itemize}
	\item \textbf{Virgola fissa}(fixed point): viene usata maggiormente nei
	      sistemi embedded quando si sa a priori il numero più grande e la
	      precisione che si vuole ottenere
	\item \textbf{Virgola mobile}(floating point): viene usata maggiormente
	      nei sistemi general purpose.
\end{itemize}

\subsection{Codifica in virgola fissa}
\begin{example}
	Si hanno a disposizione 8 bit: 4 per la parte intera e 4 per la parte frazionaria.
	Vogliamo decodificare il numero \( 0110.1011_2 \):
	\Large\[
		\underbrace{\stackrel{2^{3}}{0}\;\stackrel{2^{2}}{1}\;\stackrel{2^{1}}{1}\;\stackrel{2^{0}}{1}}_{+6} .
		\underbrace{\stackrel{2^{-1}}{1}\;\stackrel{2^{-2}}{0}\;\stackrel{2^{-3}}{1}\;\stackrel{2^{-4}}{1}}_{\frac{1}{2}+\frac{1}{8}+\frac{1}{16}}
	\]
	\normalsize\[
		+6 + \frac{1}{2}+\frac{1}{8}+\frac{1}{16}= 6+\frac{11}{16} = \frac{107}{16} = 6.6875
	\]
\end{example}
Se si vuole codificare un numero da decimale a binario bisogna tenere in considerazione
che non è certo che il numero sia razionale anche in base 2, quindi bisogna
approssimare per rappresentarlo.

\begin{example}
	\label{ex:virgolaFissaDecBin}
	Prendiamo in considerazione \( +4 +\frac{3}{5} \), in questo caso bisogna andare
	"a tentoni" e trovare la rappresentazione binaria che approssima con il minor
	errore possibile.
	\[
		4_{10} = 0100_2
	\]
	\[
		0.1001 = \frac{9}{10} \Delta \frac{3}{80}
	\]
	\[
		0.0111 = \frac{7}{16} \Delta -\frac{4}{80}
	\]
	\[
		0.0110 = \frac{3}{8} \Delta \frac{9}{40}
	\]
	\[
		\underline{0.1010 = \frac{5}{8} \Delta -\frac{1}{40}}
	\]
	\( \Delta \) rappresenta l'errore, quindi la rappresentazione più vicina è
	\( 0100.1010_2 \). Però non è stato rappresentato \( \frac{3}{5} \), ma
	\( \frac{1}{2}+\frac{1}{16}=\frac{9}{16} \).
\end{example}
Questo metodo è pesante perchè bisogna controllare più alternative.

\subsubsection{Errore percentuale}
Bisogna decidere se calcolarlo rispetto alla parte intera o a quella frazionaria.
Nel seguente esempio viene calcolato l'errore percentuale rispetto alla parte
frazionaria dell'esempio \ref{ex:virgolaFissaDecBin}.
\begin{example}
	\[
		\frac{1}{40} : \frac{3}{5} = \frac{1}{40} * \frac{5}{3} = \frac{1}{24} \approx 0.052\%
	\]
\end{example}
Il massimo errore che si può fare è l'overflow.

\subsection{Codifica in virgola mobile}
Gli standard della virgola mobile sono: IEEE 754. Questo standard
è stato rivisto molte volte e ora viene usato da tutte le codifiche per i numeri in
virgola mobile.\\
Il numero viene separato in 3 parti:
\begin{itemize}
	\item \textbf{M}: Mantissa
	\item \textbf{B}: Base 2
	\item \textbf{e}: Esponente
\end{itemize}
La struttura del numero è quindi:
\[
	N = \pm \cdot B^{\pm e}
\]
Questo permette di dividere il numero in modo da poter scegliere quanti bit dedicare
alla mantissa e quanti all'esponente. Si riscontrano però i seguenti problemi:
\begin{itemize}
	\item Bisogna scegliere la base in cui fare la codifica \(\to\)  base 2
	\item Bisogna scegliere la divisione di bit tra \emph{segno}, \emph{mantissa} e \emph{esponente} \( \to \)   \( 1\; S \), \( 23\; M \), \( 8\; e \)
	\item La rappresentazione deve essere univoca \( \to \)  \( 1.\; \ldots_2 \)
	\item Bisogna trovare un modo per rappresentare gli errori
\end{itemize}
Se la mantissa e la base sono in base 2 la moltiplicazione e la
divisione sono agevolate tramite l'utilizzo dello \emph{shift}.

\begin{itemize}
	\item \(0110 \cdot  2 = 1100\) è uno shift a sinistra in binario.
	      \begin{center}
		      \Large
		      \begin{tikzpicture}
			      \node[align=left] (Prima1) at (0,0) {0};
			      \node[align=left] (Prima2) at (0.2,0) {1};
			      \node[align=left] (Prima3) at (0.4,0) {1};
			      \node[align=left] (Prima4) at (0.6,0) {0};

			      \node[align=left] (Dopo1) at (0,-1) {1};
			      \node[align=left] (Dopo2) at (0.2,-1) {1};
			      \node[align=left] (Dopo3) at (0.4,-1) {0};
			      \node[align=left] (Dopo4) at (0.6,-1) {0};

			      \draw[->, draw] (Prima2) to (Dopo1);
			      \draw[->, draw] (Prima3) to (Dopo2);
			      \draw[->, draw] (Prima4) to (Dopo3);
		      \end{tikzpicture}
	      \end{center}

	\item \(1010/2 = 0101\) è uno shift a destra in binario.
	      \begin{center}
		      \Large
		      \begin{tikzpicture}
			      \node[align=left] (Prima1) at (0,0) {1};
			      \node[align=left] (Prima2) at (0.2,0) {0};
			      \node[align=left] (Prima3) at (0.4,0) {1};
			      \node[align=left] (Prima4) at (0.6,0) {0};

			      \node[align=left] (Dopo1) at (0,-1) {0};
			      \node[align=left] (Dopo2) at (0.2,-1) {1};
			      \node[align=left] (Dopo3) at (0.4,-1) {0};
			      \node[align=left] (Dopo4) at (0.6,-1) {1};

			      \draw[->, draw] (Prima1) to (Dopo2);
			      \draw[->, draw] (Prima2) to (Dopo3);
			      \draw[->, draw] (Prima3) to (Dopo4);
		      \end{tikzpicture}
	      \end{center}

\end{itemize}


\subsubsection{Divisione di bit tra segno, mantissa ed esponente}
Un numero è rappresentabile in 2 modi:
\begin{itemize}
	\item Singola precisione 32 bit \( \to  \) float
	\item Doppia precisione 64 bit \( \to  \) double
\end{itemize}

Prendiamo in considerazione 32 bit, ora dobbiamo decidere quanti bit dedicare
alla mantissa e all'esponente.
\[
	2^{\pm e}
\]
\begin{center}
	$|e| = 4 bit = 2^{+7}$\\
	$5 bit = 2^{+15}$\\
	$6 bit = 2^{+31}$\\
	$7 bit = 2^{+63}$\\
	$8 bit = 2^{+127}$
\end{center}
L'impatto dei bit sull'esponente è doppiamente esponenziale, quindi cresce tantissimo.

\begin{itemize}
	\item \textbf{8 bit} all'esponente, quindi l'esponente
	      può assumere valori da \( -127 \) a \( +127 \).
	\item \textbf{23 bit} alla mantissa, quindi la mantissa
	      può assumere valori da \( 0 \) a \( 2^{23}-1 \)
	\item \textbf{1 bit} al segno.
\end{itemize}

\begin{center}
	\begin{tikzpicture}
		\draw[draw] (0, 0) rectangle (1.6,1) node[pos=.5, align=center] {1 bit:\\
				segno \( \pm \) };
		\draw[draw] (1.6, 0) rectangle (5,1) node[pos=.5, align=center] {8 bit: esponente};
		\draw[draw] (5, 0) rectangle (10,1) node[pos=.5, align=center] {23 bit: mantissa};
	\end{tikzpicture}
\end{center}
Per la rappresentazione univoca la mantissa si codifica in virgola fissa.
Cioè si parte da una mantissa con un \textbf{punto fisso} e dividendo o moltiplicando (shift) si
può spostare la virgola per arrivare alla forma \textbf{1.00000...} e questa forma è la
rappresentazione univoca.

Questa operazioe si chiama \textbf{normalizzazione} e visto che la
rappresentazione è sempre la stessa l'\emph{1.} non viene rappresentato, quindi
viene inserito nella mantissa solo tutto ciò che viene dopo.
\begin{figure}[H]
	\begin{center}
		\( 11111111 \) \( \pm \infty \)\\
		\( 11111110 \) \( +127 \) \\
		\( \ldots \)\\
		\( 00000000 \) \( \pm 0 \)\\
		\( \ldots \)\\
		\( 00000001 \) \( -126 \) \\
		\( 00000000 \) \( -127 \)
	\end{center}
	\caption{Range dell'esponente}
\end{figure}
Si è deciso di codificare l'esponente in \textbf{Eccesso 127}. Quindi per
rappresentare lo zero si usa come esponente il minore numero possibile:
$1 \cdot 2^{-127} = 0$. Per codificare i numeri si somma 127 al numero desiderato
e visto che i numeri possibili ora vanno da -127 a +127 se codifichiamo
il risultato in modulo avremo dei numeri da 0 a 256.

\begin{figure}[H]
	\begin{example}
		Si vuole decodificare il seguente numero:
		\[1\:01110111\:0110...0\]
		\[M = -(1+\frac{1}{4}+\frac{1}{4})*2 = -(\frac{11}{8})*2^{e}\]
		\[E = (1+2+4+16+32+64)-127=119-127=-8\]
		\[N = -\frac{11}{8} * 2^{-8}\]
	\end{example}
\end{figure}

\begin{figure}[H]
	\begin{example}
		Codifica $+(4+\frac{1}{2}+\frac{1}{16})*2^{+34}$
       \begin{enumerate}
        \item Sappiamo già che il numero è positivo quindi:
            \[
            S=0
            \] 
            \item Calcoliamo la mantissa:
                \[
                    4+\frac{1}{2}+\frac{1}{16}= \underbrace{100}_{4_{10}}.
                    \underbrace{10010 \ldots 0}_{\frac{1}{2}+\frac{1}{16}}
                \] 
                \item La mantissa va normalizzata moltiplicando per 4:
                \[
                    100.10010 \ldots 0 * 2^{+2} = 1.0010010 \ldots 0
                \]
                \[
                M = 0010010 \ldots 0
                \] 
            \item Calcoliamo l'esponente:
       \end{enumerate} 
	\end{example}
\end{figure}

\begin{itemize}
	\item $0\:00000000\:0...0 = +0$
	\item $1\:00000000\:0...0 = -0$
\end{itemize}

Quando l'esponente è tutto 1 e la mantissa tutta 0 allora equivale a $infinito$
+ o - in base al primo bit. Se invece la mantissa è diversa da 0 con esponente tutti 1
allora rappresenta un errore NaN.

Somma:\\
\label{es2}


%Lezione 4
\section{Modelli}
Per un progetto bisogna creare un \textbf{modello} che rappresenti il sistema.
Boole ha cercato di rappresentare tutte le algebre. Lo ha fatto attraverso
una quintupla: \( <B^{n}, \cdot, + ,\{0,1\}> \)
\label{D1}
\begin{itemize}
    \item \( B^n \) è l'insieme di valori
    \item \( \{0,1\} \) è l'alfabeto (sistema binario)
        \item \( \cdot  \) e \( + \) sono 2 operatori
\end{itemize}
Bool garantisce che si può creare qualsiasi funzione utilizzando soltanto i
2 operatori:
\[
    f(B^n) \to B^m
\] 
\begin{example}
    Si vuole creare un modello con 2 bit in entrata e 1 in uscita:
    \[
    n=2 \; m=1
    \] 
    \( O=1 \leftrightarrow A=B \)\\
    \( f(B^2) \to B \) 
    \label{D2}
    Per mappare i valori in ingresso con quelli di uscita si usa una
    \textbf{tabella di verità}:
\begin{center}
	\begin{tabular}{c|c|c}
		\( A \) & \( B \) & \( O \) \\
		\hline
		0            & 0           & 1                       \\
		0            & 1           & 0                       \\
		1            & 0           & 0                       \\
		1            & 1           & 1                       \\
	\end{tabular}
\end{center}
Chiamiamo mintermine un punto dello spazio booleano in ingresso in cui la
funzione vale 1. Il maxtermine è il contrario.
L'insieme di mintermini \( \{m_0\footnote{\( m_n \): n è il valore in modulo
del relativo numero binario, \( m \) sta per modulo. \( m_3 = 11_2 \) }, m_3\} \) si chiama \textbf{ON-SET} 
L'insieme dei maxtermini \( \{m_1, m_2\} \) si chiama \textbf{OFF-SET}. Basta
uno dei due insiemi (ON-SET, OFF-SET) per definire la funzione.
\[
m_3 = A \cdot B
\] 
Dire che \( m_3 \) è il prodotto delle due variabili è un modo corretto per
rappresentarlo.
\[
 m_0 = \bar{A} \cdot \bar{B} 
\] 
Per rappresentare il mintermine basta fare il prodotto delle variabili se
valgono 1 o delle variabili negate se valgono 0.

Per rappresentare la funzione si può usare la somma dei mintermini:
\[
O = m_0 + m_3 = \bar{A} \cdot \bar{B} + A \cdot B = 0
\] 
Questa rappresentazione viene detta: \emph{Espressione in somma di prodotti} 
\begin{theorem}
   Dato un ON-SET c'è sempre una sola espressione in somma di prodotti che lo
   rappresenti.
\end{theorem}
\end{example}

\subsection{Tabelle di verità}
\subsubsection{Operatore prodotto}
\begin{center}
	\begin{tabular}{c|c|c}
		\( A \) & \( B \) & \( O \) \\
		\hline
		0            & 0           & 0                       \\
		0            & 1           & 0                       \\
		1            & 0           & 0                       \\
		1            & 1           & 1                       \\
	\end{tabular}
\end{center}

\subsubsection{Operatore somma}
\begin{center}
	\begin{tabular}{c|c|c}
		\( A \) & \( B \) & \( O \) \\
		\hline
		0            & 0           & 0                       \\
		0            & 1           & 1                       \\
		1            & 0           & 1                       \\
		1            & 1           & 1                       \\
	\end{tabular}
\end{center}

\subsubsection{Operatore negazione}
\begin{center}
	\begin{tabular}{c|c}
		\( A \) & \( O \) \\
		\hline
		0            & 1 \\
		1            & 0 \\
	\end{tabular}
\end{center}

\section{Transistor}
È un "comando di accensione" che permette di accendere o spegnere un circuito.
\subsection{Transistor CMOS}
\label{D3}
\label{D4}
\label{D5}

\section{Espressione in somma di prodotti}
\label{D6}
I circuiti devono spesso tenere conto di alcune specifiche da ottimizzare:
\begin{itemize}
    \item \textbf{Area}: minor numero di porte logiche
    \item \textbf{Latency}: più porte logiche si attraversano più sarà il ritardo
    \item \textbf{Power}: più porte logiche si attraversano più sarà
            il consumo
    \item \textbf{Safety}: più porte logiche si attraversano più sarà
            la probabilità di errore
\end{itemize}
\end{document}
