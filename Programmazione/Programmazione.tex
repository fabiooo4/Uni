\documentclass[a4paper]{article}

\usepackage[utf8]{inputenc}
\usepackage[T1]{fontenc}
\usepackage{textcomp}
\usepackage[italian]{babel}
\usepackage{amsmath, amssymb}
\usepackage{amsfonts}
\usepackage{mdframed}
\usepackage{ntheorem}
\usepackage{float}
\usepackage{listings}
\usepackage{xcolor}
\usepackage{graphicx}
\graphicspath{{./figures/}}

\usepackage{tikz}
\usetikzlibrary{shapes.geometric, shapes.misc, arrows}

% Code blocks
\definecolor{codegreen}{rgb}{0,0.6,0}
\definecolor{codegray}{rgb}{0.5,0.5,0.5}
\definecolor{codepurple}{rgb}{0.58,0,0.82}
\definecolor{backcolour}{rgb}{0.95,0.95,0.95}

\lstdefinestyle{mystyle}{
	backgroundcolor=\color{backcolour},
	commentstyle=\color{codegreen},
	keywordstyle=\color{magenta},
	numberstyle=\tiny\color{codegray},
	stringstyle=\color{codepurple},
	basicstyle=\ttfamily\footnotesize,
	breakatwhitespace=false,
	breaklines=true,
	captionpos=b,
	keepspaces=true,
	numbers=left,
	numbersep=5pt,
	showspaces=false,
	showstringspaces=false,
	showtabs=false,
	tabsize=2
}

\lstset{style=mystyle}

% Flow chart styles
\tikzstyle{startstop} = [ellipse, rounded corners, minimum width=3cm, minimum height=1cm,text centered, draw=black]
\tikzstyle{input} = [chamfered rectangle, chamfered rectangle corners=north west, chamfered rectangle xsep=2cm, minimum width=3cm, minimum height=1cm, text centered, draw=black]
\tikzstyle{output} = [chamfered rectangle, chamfered rectangle corners=south east, chamfered rectangle xsep=2cm, minimum width=3cm, minimum height=1cm, text centered, draw=black]
\tikzstyle{process} = [rectangle, minimum width=3cm, minimum height=1cm, text centered, draw=black]
\tikzstyle{decision} = [diamond, minimum width=3cm, minimum height=1cm, text centered, draw=black]
\tikzstyle{arrow} = [thick,->,>=stealth]

% Code blocks
\usepackage{listings}
\usepackage{color}

\definecolor{dkgreen}{rgb}{0,0.6,0}
\definecolor{gray}{rgb}{0.5,0.5,0.5}
\definecolor{mauve}{rgb}{0.58,0,0.82}

\lstset{frame=tb,
	aboveskip=3mm,
	belowskip=3mm,
	showstringspaces=false,
	columns=flexible,
	basicstyle={\small\ttfamily},
	numbers=none,
	numberstyle=\tiny\color{gray},
	keywordstyle=\color{blue},
	commentstyle=\color{dkgreen},
	stringstyle=\color{mauve},
	breaklines=true,
	breakatwhitespace=true,
	tabsize=3
}

\usepackage{import}
\usepackage{pdfpages}
\usepackage{transparent}
\usepackage{xcolor}

% Useful definitions frame
\theoremstyle{break}
\theoremheaderfont{\bfseries}
\newmdtheoremenv[%
	linecolor=gray,leftmargin=0,%
	rightmargin=0,
	innertopmargin=8pt,%
	ntheorem]{define}{Definizioni utili}[section]

% Example frame
\theoremstyle{break}
\theoremheaderfont{\bfseries}
\newmdtheoremenv[%
	linecolor=gray,leftmargin=0,%
	rightmargin=0,
	innertopmargin=8pt,%
	ntheorem]{example}{Esempio}[section]

% Important definition frame
\theoremstyle{break}
\theoremheaderfont{\bfseries}
\newmdtheoremenv[%
	linecolor=gray,leftmargin=0,%
	rightmargin=0,
	backgroundcolor=gray!40,%
	innertopmargin=8pt,%
	ntheorem]{definition}{Definizione}[section]

% Exercise frame
\theoremstyle{break}
\theoremheaderfont{\bfseries}
\newmdtheoremenv[%
	linecolor=gray,leftmargin=0,%
	rightmargin=0,
	innertopmargin=8pt,%
	ntheorem]{exercise}{Esercizio}[section]


% figure support
% Inkscape figures
\newcommand{\incfig}[2][1]{%
	\def\svgwidth{#1\columnwidth}
	\import{./figures/}{#2.pdf_tex}
}

\pdfsuppresswarningpagegroup=1


\begin{document}
\begin{titlepage}
	\begin{center}
		\vspace*{1cm}

		\Huge
		\textbf{Probabilità e Statistica\\Esercizi}

		\vspace{0.5cm}
		\LARGE
		UniVR - Dipartimento di Informatica

		\vspace{1.5cm}

		\textbf{Fabio Irimie}

		\vfill


		\vspace{0.8cm}


		2° Semestre 2023/2024

	\end{center}
\end{titlepage}


\tableofcontents
\pagebreak

\section{Introduzione}
L'informatica è la scienza che si occupa della rappresentazione (estrarre
le informazioni rilevanti: quali, cosa, quante) ed elaborazione
(come si arriva passo dopo passo a risolvere un problema) dell'informazione. È una scienza con un approccio rigoroso e formale.

\subsection{Algoritmo}
Dopo aver raccolto le informazioni del problema che si vuole risolvere
si procede con la costruzione di un algoritmo che risolva il problema.
Questo algoritmo viene scritto utilizzando un linguaggio di programmazione
(C, Java, \ldots).

\begin{definition}
	Dato un esecutore e un problema l'algoritmo è una sequenza finita e
	ordinata di istruzioni elementari, comprensibili dall'esecutore,
	che risolvono il problema dato.
\end{definition}

\subsubsection{Proprietà}
\begin{itemize}
	\item \textbf{Sequenza}: l'ordine è importante,
	\item \textbf{Finita}: deve terminare,
	\item \textbf{Deterministico}: deve essere preciso,
	\item \textbf{Corretto}: deve risolvere il problema,
	\item \textbf{Completo}: deve considerare tutti
	      gli aspetti del problema,
	\item \textbf{Generale}: deve risolvere tutti i casi del problema,
	\item \textbf{Efficienza}
	      \begin{itemize}
		      \item \textbf{Tempo}: deve terminare in un tempo ragionevole,
		      \item \textbf{Spazio}: deve utilizzare una quantità di memoria
		            ragionevole.
	      \end{itemize}
\end{itemize}

\subsection{Linguaggio pseudo-formale}
Il linguaggio si divide in:
\begin{itemize}
	\item \textbf{Vocabolario}: insieme di simboli che possono essere usati,
	\item \textbf{Sintassi}: regole per la costruzione di frasi,
	\item \textbf{Semantica}: significato delle frasi corrette.
\end{itemize}

\section{Diagrammi di flusso}
I seguenti simboli vengono utilizzati per la costruzione di un diagramma
di flusso che rappresenta un algoritmo:
\begin{itemize}
	\item \textbf{Inizio/fine}: rappresenta l'inizio e la fine dell'algoritmo,
	\item \textbf{Istruzione}: rappresenta un'istruzione,
	\item \textbf{Condizione}: rappresenta una condizione,
	\item \textbf{Sottoprogramma}: rappresenta un sottoprogramma.
\end{itemize}

\begin{figure}[H]
	\centering
	\incfig[0.3]{simbolidiagramma}
	\caption{Simboli del diagramma di flusso}
	\label{fig:simbolidiagramma}
\end{figure}

\section{Esercizi con diagrammi di flusso}
Dato un numero dare in output il doppio.
\begin{figure}[H]
	\centering
	\incfig[0.3]{numeroDoppio}
	\caption{Diagramma di flusso per il doppio}
	\label{fig:doppio}
\end{figure}

Dati 2 numeri dire il maggiore.
\begin{figure}[H]
	\centering
	\incfig[1]{maggiore3numeri}
	\caption{Diagramma di flusso per il doppio}
	\label{fig:maggiore3numeri}
\end{figure}
\subsubsection{Stampa tutti i numeri naturali fino a n}
\( \forall i \in [o,n] \) stampo a video i:
\begin{figure}
	\includegraphics{numeriNaturali}
\end{figure}
Questo è un ciclo a condizione iniziale, cioè appena la condizione diventa falsa
si esce dal ciclo.

\subsubsection{Trovare il più grande tra n numeri}
\begin{center}
	\scalebox{0.8}{
		\begin{tikzpicture}[node distance=2cm]
			\node (start) [startstop] {Inizio};
			\node (out1) [output, below of=start] {Quanti voti?};
			\node (in1) [input, below of=out1] {n};
			\node (proc1) [process, below of=in1] {i = 1};
			\node (in2) [input, below of=proc1] {voto};
			\node (proc2) [process, below of=in2] {max = voto};
			\node (dec1) [decision, below of=proc2, yshift=-0.5cm] {i $<$ n};
			\node (out2) [output, right of=in2, xshift=3cm] {max};
			\node (stop) [startstop, below of=out2] {Fine};
			\node (proc3) [process, below of=dec1, yshift=-0.5cm] {i = i + 1};
			\node (in3) [input, below of=proc3] {voto};
			\node (dec2) [decision, below of=in3, yshift=-0.5cm] {voto $>$ max};
			\node (proc4) [process, right of=dec2, xshift=2cm] {max = voto};

			\draw [arrow] (start) -- (out1);
			\draw [arrow] (out1) -- (in1);
			\draw [arrow] (in1) -- (proc1);
			\draw [arrow] (proc1) -- (in2);
			\draw [arrow] (in2) -- (proc2);
			\draw [arrow] (proc2) -- (dec1);
			\draw [arrow] (dec1) -- node[anchor=east] {Vero} (proc3);
			\draw [arrow] (proc3) -- (in3);
			\draw [arrow] (in3) -- (dec2);
			\draw [arrow] (dec2) -- node[anchor=south] {Vero} (proc4);
			\draw [arrow] (dec2) -- node[anchor=south] {Falso} ++(-3,0) |- (dec1);
			\draw [arrow] (proc4) |- (dec1);
			\draw [arrow] (dec1) -- node[anchor=west] {Falso} (out2);
			\draw [arrow] (out2) -- (stop);
		\end{tikzpicture}
	}
\end{center}

\subsection{Ciclo guidato da sentinella}
È un ciclo che termina quando viene inserito un valore particolare.
\subsubsection{Trova il voto più alto tra un insieme di voti di cardinalità a priori
	non specificata (terminare quando viene inserito -1)}
\begin{center}
	\scalebox{0.8}{
		\begin{tikzpicture}[node distance=2cm]
			\node (inizio) [startstop] {Inizio};
			\node (voto1) [input, below of=inizio] {voto};
			\node (sentinella) [decision, below of=voto1] {voto è -1?};
			\node (fine1) [startstop, right of=sentinella, xshift=2cm] {Fine};
			\node (maxvoto) [process, below of=sentinella, yshift=-0.5cm] {max = voto};
			\node (votonon1) [decision, below of=maxvoto, yshift=-0.5cm] {voto \( \neq \)  -1?};
			\node (voto2) [input, below of=votonon1, yshift=-0.5cm] {voto};
			\node (votomaggioremax) [decision, below of=voto2, yshift=-0.5cm] {voto \( > \) max?};
			\node (maxvoto2) [process, below of=votomaggioremax, yshift=-0.5cm] {max = voto};
			\node (max) [output, right of=maxvoto, xshift=3cm] {max};
			\node (fine2) [startstop, below of=max] {Fine};

			\draw [arrow] (inizio) -- (voto1);
			\draw [arrow] (voto1) -- (sentinella);
			\draw [arrow] (sentinella) -- node[anchor=south] {Vero} (fine1);
			\draw [arrow] (sentinella) -- node[anchor=east] {Falso} (maxvoto);
			\draw [arrow] (maxvoto) -- (votonon1);
			\draw [arrow] (votonon1) -- node[anchor=east] {Vero} (voto2);
			\draw [arrow] (votonon1) -- node[anchor=south] {Falso} (max);
			\draw [arrow] (voto2) -- (votomaggioremax);
			\draw [arrow] (votomaggioremax) -- node[anchor=east] {Vero} (maxvoto2);
			\draw [arrow] (votomaggioremax) -- node[anchor=south] {Falso} ++(-3,0) |- (votonon1);
			\draw [arrow] (max) -- (fine2);
			\draw [arrow] (votomaggioremax) -- ++(3,0) |- (votonon1);
		\end{tikzpicture}
	}
\end{center}

\section{Linguaggio di programmazione C}
È un linguaggio di programmazione:
\begin{itemize}
	\item \textbf{General purpose}: può essere utilizzato per risolvere
	      qualsiasi problema.
	\item \textbf{Dichiarativo}: si dichiarano le variabili e si
	      specifica cosa fare con esse.
	\item \textbf{Imperativo}: si specifica come risolvere il problema.
	\item \textbf{Case sensitive}: distingue tra maiuscole e minuscole.
	\item \textbf{Compilato}: il codice sorgente viene tradotto in codice macchina.
\end{itemize}

\subsection{Catena di programmazione}
\begin{enumerate}
	\item \textbf{Editing}: si scrive il programma con un editor
	\item \textbf{Pre-processing}(Pre-processore): inclusione di librerie.
	\item \textbf{Compilazione}(Compilatore): traduzione il programma C in
	      codice oggetto comprensibile dal calcolatore. Vengono segnalati
	      errori (statici) e se non ci sono errori viene generato un file eseguibile.
	\item \textbf{Linking}: collegamento il codice oggetto di programma con
	      le librerie.
	\item \textbf{Loading}: il programma va caricato nella RAM prima dell'esecuzione
	\item \textbf{Esecuzione}: il programma viene eseguito.
\end{enumerate}
Un esempio di codice in C è il seguente:
\begin{lstlisting}[language=C]
#include <stdio.h>

int main() {
		printf("Hello World!");
		return 0;
	}
\end{lstlisting}

\subsection{Dichiarazione di variabili}
Le variabili sono di 2 tipi:
\begin{itemize}
	\item \textbf{Built-in}: sono già presenti nel linguaggio.
	\item \textbf{User-defined}: definite dall'utente.
\end{itemize}

\paragraph{Tipo:} L'insieme dei valori ammissibili più le operazioni ammesse sull'insieme
\begin{itemize}
	\item Intero:
	      \begin{itemize}
		      \item \texttt{int} 16bit in complemento a 2
		      \item \texttt{long} 32bit in complemento a 2
	      \end{itemize}
	\item Reale:
	      \begin{itemize}
		      \item \texttt{float} 32bit in virgola mobile
		      \item \texttt{double} 64bit in virgola mobile
	      \end{itemize}
	\item Carattere: \texttt{char} 8bit ASCII
\end{itemize}
Il codice è diviso in:
\begin{itemize}
	\item \textbf{Parte dichiarativa}: dichiarazione delle variabili
	\item \textbf{Parte esecutiva}: istruzioni che vengono eseguite
\end{itemize}
Il seguente codice è un esempio che mostra la separazione delle 2 parti:
\begin{lstlisting}[language=C]
   #include <stdio.h> 

   int main() {
       //Parte dichiarativa
       int lato;
       int perimetro, area;
       char t;
       float media;

       //Parte esecutiva
       lato = 4;
       perimetro = lato * 4;

       return 0;
   }
\end{lstlisting}
\subsubsection{Operatori}
In C le operazioni disponibili sono:
\begin{itemize}
	\item \textbf{+} somma
	\item \textbf{-} sottrazione
	\item \textbf{*} moltiplicazione
	\item \textbf{/} divisione
	\item \textbf{\%} resto della divisione intera
\end{itemize}
Un esempio di codice che usa gli operatori è il seguente:
\begin{lstlisting}[language=C]
    #include <stdio.h>

    int main() {
        int a, b;
        float media;
        float bonus;

        a = 25;
        b = 26;
        media = (a + b) / 2; // 25.0
        return 0;
    }
\end{lstlisting}
Con \( a \) e \( b \) interi la divisione è fatta per intero, quindi il risultato è 25
che poi viene assegnato alla variabile float \emph{media}, quindi il risultato è 25.0.
Per avere il risultato corretto bisogna dichiarare a e b come float. Un'altra opzione
sarebbe di rendere float il 2 del divisore, in questo modo il risultato sarebbe:
\( (a+b)=2.0=25.5 \)

\begin{define}
	\begin{lstlisting}[language=C]
t = 'a'; // Assegnato il carattere a alla variabile t
t = a; // Assegnato il valore della variabile a alla variabile t
	\end{lstlisting}
\end{define}

\subsection{Output (Stampa a video standard)}
Per stampare a video si utilizza la funzione \texttt{printf()}:
\begin{lstlisting}[language=C]
    printf("Hello World!\n"); // \n va a capo
    printf("Hello \t world"); // \t tabulazione (4 spazi)
\end{lstlisting}
Per stampare il valore delle variabili:
\begin{lstlisting}[language=C]
    printf("Il valore di a e' %d\n", a); // %d indica un intero
    printf("Il valore di a e' %d e di b e' %d\n", a, b);

    // %c indica un carattere
    // %f indica un float
    // %g indica un double
\end{lstlisting}

\subsection{Input da tastiera}
Per leggere da tastiera si utilizza la funzione \texttt{scanf()}:
\begin{lstlisting}[language=C]
    scanf("%d", &a); // &a indica l'indirizzo di memoria di a
    scanf("%d %d", &a, &b);
\end{lstlisting}
\end{document}
