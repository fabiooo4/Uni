\documentclass[a4paper]{article}

\usepackage[utf8]{inputenc}
\usepackage[T1]{fontenc}
\usepackage{textcomp}
\usepackage[italian]{babel}
\usepackage{amsmath, amssymb}
\usepackage{amsfonts}
\usepackage{mdframed}
\usepackage{ntheorem}
\usepackage{xcolor}
\usepackage{graphicx}
\graphicspath{{./figures/}}

% Code blocks
\usepackage{listings}
\usepackage{color}

\definecolor{dkgreen}{rgb}{0,0.6,0}
\definecolor{gray}{rgb}{0.5,0.5,0.5}
\definecolor{mauve}{rgb}{0.58,0,0.82}

\lstset{frame=tb,
	aboveskip=3mm,
	belowskip=3mm,
	showstringspaces=false,
	columns=flexible,
	basicstyle={\small\ttfamily},
	numbers=none,
	numberstyle=\tiny\color{gray},
	keywordstyle=\color{blue},
	commentstyle=\color{dkgreen},
	stringstyle=\color{mauve},
	breaklines=true,
	breakatwhitespace=true,
	tabsize=3
}

\usepackage{import}
\usepackage{pdfpages}
\usepackage{transparent}
\usepackage{xcolor}

% Useful definitions frame
\theoremstyle{break}
\theoremheaderfont{\bfseries}
\newmdtheoremenv[%
	linecolor=gray,leftmargin=0,%
	rightmargin=0,
	innertopmargin=8pt,%
	ntheorem]{define}{Definizioni utili}[section]

% Example frame
\theoremstyle{break}
\theoremheaderfont{\bfseries}
\newmdtheoremenv[%
	linecolor=gray,leftmargin=0,%
	rightmargin=0,
	innertopmargin=8pt,%
	ntheorem]{example}{Esempio}[section]

% Important definition frame
\theoremstyle{break}
\theoremheaderfont{\bfseries}
\newmdtheoremenv[%
	linecolor=gray,leftmargin=0,%
	rightmargin=0,
	backgroundcolor=gray!40,%
	innertopmargin=8pt,%
	ntheorem]{definition}{Definizione}[section]

% Exercise frame
\theoremstyle{break}
\theoremheaderfont{\bfseries}
\newmdtheoremenv[%
	linecolor=gray,leftmargin=0,%
	rightmargin=0,
	innertopmargin=8pt,%
	ntheorem]{exercise}{Esercizio}[section]


% figure support
% Inkscape figures
\newcommand{\incfig}[2][1]{%
	\def\svgwidth{#1\columnwidth}
	\import{./figures/}{#2.pdf_tex}
}

\pdfsuppresswarningpagegroup=1


\begin{document}
\begin{titlepage}
	\begin{center}
		\vspace*{1cm}

		\Huge
		\textbf{Analisi 1}

		\vspace{0.5cm}
		\LARGE
		UniVR - Dipartimento di Informatica

		\vspace{1.5cm}

		\textbf{Fabio Irimie}

		\vfill


		\vspace{0.8cm}

    Corso di Giacomo Canevari

		1° Semestre 2023/2024

	\end{center}
\end{titlepage}


\tableofcontents
\pagebreak

\section{Introduzione}
L'informatica è la scienza che si occupa della rappresentazione (estrarre
le informazioni rilevanti: quali, cosa, quante) ed elaborazione
(come si arriva passo dopo passo a risolvere un problema) dell'informazione. È una scienza con un approccio rigoroso e formale.

\subsection{Algoritmo}
Dopo aver raccolto le informazioni del problema che si vuole risolvere
si procede con la costruzione di un algoritmo che risolva il problema.
Questo algoritmo viene scritto utilizzando un linguaggio di programmazione
(C, Java, \ldots).

\begin{definition}
	Dato un esecutore e un problema l'algoritmo è una sequenza finita e
	ordinata di istruzioni elementari, comprensibili dall'esecutore,
	che risolvono il problema dato.
\end{definition}

\subsubsection{Proprietà}
\begin{itemize}
	\item \textbf{Sequenza}: l'ordine è importante,
	\item \textbf{Finita}: deve terminare,
	\item \textbf{Deterministico}: deve essere preciso,
	\item \textbf{Corretto}: deve risolvere il problema,
	\item \textbf{Completo}: deve considerare tutti
	      gli aspetti del problema,
	\item \textbf{Generale}: deve risolvere tutti i casi del problema,
	\item \textbf{Efficienza}
	      \begin{itemize}
		      \item \textbf{Tempo}: deve terminare in un tempo ragionevole,
		      \item \textbf{Spazio}: deve utilizzare una quantità di memoria
		            ragionevole.
	      \end{itemize}
\end{itemize}

\subsection{Linguaggio pseudo-formale}
Il linguaggio si divide in:
\begin{itemize}
	\item \textbf{Vocabolario}: insieme di simboli che possono essere usati,
	\item \textbf{Sintassi}: regole per la costruzione di frasi,
	\item \textbf{Semantica}: significato delle frasi corrette.
\end{itemize}

\section{Diagrammi di flusso}
I seguenti simboli vengono utilizzati per la costruzione di un diagramma
di flusso che rappresenta un algoritmo:
\begin{itemize}
	\item \textbf{Inizio/fine}: rappresenta l'inizio e la fine dell'algoritmo,
	\item \textbf{Istruzione}: rappresenta un'istruzione,
	\item \textbf{Condizione}: rappresenta una condizione,
	\item \textbf{Sottoprogramma}: rappresenta un sottoprogramma.
\end{itemize}

\begin{figure}[ht]
	\centering
	\incfig[0.3]{simbolidiagramma}
	\caption{Simboli del diagramma di flusso}
	\label{fig:simbolidiagramma}
\end{figure}

Dato un numero dare in output il doppio.
\begin{figure}[ht]
	\centering
	\incfig[0.3]{numeroDoppio}
	\caption{Diagramma di flusso per il doppio}
	\label{fig:doppio}
\end{figure}

Dati 2 numeri dire il maggiore.
\begin{figure}[ht]
	\centering
	\incfig[0.3]{maggiore3numeri}
	\caption{Diagramma di flusso per il doppio}
	\label{fig:maggiore3numeri}
\end{figure}

\end{document}
