\documentclass[a4paper]{article}

\usepackage[utf8]{inputenc}
\usepackage[T1]{fontenc}
\usepackage{textcomp}
\usepackage[italian]{babel}
\usepackage{amsmath, amssymb}
\usepackage{booktabs,xltabular}
\usepackage{amsfonts}
\usepackage{cancel}
\usepackage{mdframed}
\usepackage{makecell}
\usepackage{float}
\usepackage{xcolor}
\usepackage{listings}
\usepackage{graphicx}
\usepackage{tikz}
\usetikzlibrary{shapes, arrows, automata, petri, decorations.pathreplacing, positioning, calc}
\usepackage{circuitikz}
\usepackage[label=corner]{karnaugh-map}
\graphicspath{{./figures/}}

\usepackage{ntheorem}
\newtheorem{theorem}{Teorema}

\usepackage{import}
\usepackage{pdfpages}
\usepackage{transparent}
\usepackage{xcolor}

\usepackage{hyperref}
\hypersetup{
    colorlinks=false,
}

% Code blocks
\definecolor{codegreen}{rgb}{0,0.6,0}
\definecolor{codegray}{rgb}{0.5,0.5,0.5}
\definecolor{codepurple}{rgb}{0.58,0,0.82}
\definecolor{backcolour}{rgb}{0.95,0.95,0.95}

\lstdefinestyle{mystyle}{
	backgroundcolor=\color{backcolour},
	commentstyle=\color{codegreen},
	keywordstyle=\color{magenta},
	numberstyle=\tiny\color{codegray},
	stringstyle=\color{codepurple},
	basicstyle=\ttfamily\footnotesize,
	breakatwhitespace=false,
	breaklines=true,
	captionpos=b,
	keepspaces=true,
	numbers=left,
	numbersep=5pt,
	showspaces=false,
	showstringspaces=false,
	showtabs=false,
	tabsize=2
}

\lstset{style=mystyle}

\usepackage{color}

\definecolor{dkgreen}{rgb}{0,0.6,0}
\definecolor{gray}{rgb}{0.5,0.5,0.5}
\definecolor{mauve}{rgb}{0.58,0,0.82}

\lstset{frame=tb,
	aboveskip=3mm,
	belowskip=3mm,
	showstringspaces=false,
	columns=flexible,
	basicstyle={\small\ttfamily},
	numbers=none,
	numberstyle=\tiny\color{gray},
	keywordstyle=\color{blue},
	commentstyle=\color{dkgreen},
	stringstyle=\color{mauve},
	breaklines=true,
	breakatwhitespace=true,
	tabsize=3
}

\usepackage{import}
\usepackage{pdfpages}
\usepackage{transparent}
\usepackage{xcolor}



% Useful definitions frame
\theoremstyle{break}
\theoremheaderfont{\bfseries}
\newmdtheoremenv[%
	linecolor=gray,leftmargin=0,%
	rightmargin=0,
	innertopmargin=8pt,%
	innerbottommargin=8pt,
	ntheorem]{define}{Definizioni utili}[section]

% Example frame
\theoremstyle{break}
\theoremheaderfont{\bfseries}
\newmdtheoremenv[%
	linecolor=gray,leftmargin=0,%
	rightmargin=0,
	innertopmargin=8pt,%
	innerbottommargin=8pt,
	ntheorem]{example}{Esempio}[section]

% Important definition frame
\theoremstyle{break}
\theoremheaderfont{\bfseries}
\newmdtheoremenv[%
	linecolor=gray,leftmargin=0,%
	rightmargin=0,
	backgroundcolor=gray!40,%
	innertopmargin=8pt,%
	innerbottommargin=8pt,
	ntheorem]{definition}{Definizione}[section]

% Exercise frame
\theoremstyle{break}
\theoremheaderfont{\bfseries}
\newmdtheoremenv[%
	linecolor=gray,leftmargin=0,%
	rightmargin=0,
	innertopmargin=8pt,%
	innerbottommargin=8pt,
	ntheorem]{exercise}{Esercizio}[section]

% figure support
\usepackage{import}
\usepackage{xifthen}
\pdfminorversion=7
\usepackage{pdfpages}
\usepackage{transparent}
\newcommand{\incfig}[1]{%
	\def\svgwidth{\columnwidth}
	\import{./figures/}{#1.pdf_tex}
}

% FSM tikz
\tikzset{
    place/.style={
        circle,
        thick,
        draw=black,
        minimum size=6mm,
    },
        state/.style={
        circle,
        thick,
        draw=blue!75,
        fill=blue!20,
        minimum size=6mm,
    },
}

\pdfsuppresswarningpagegroup=1

\begin{document}
\begin{center}
  \vspace*{1cm}

  \Huge
  \textbf{Architettura degli elaboratori}
  \LARGE
  \textbf{Esercitazione}

  \vspace{0.5cm}
  \LARGE
  UniVR - Dipartimento di Informatica

  \vspace{1.5cm}

  \textbf{Fabio Irimie}

  \vfill


  \vspace{0.8cm}


  2° Semestre 2023/2024

\end{center}

\pagebreak
\tableofcontents
\pagebreak

\section{Microoperazioni}
I seguenti esercizi fanno riferimento all'architettura con 1 bus.
\subsection{Esercizio 1 (Fetch)}
\begin{exercise}
  Scrivere le microistruzioni per effettuare il fetch di un'istruzione, commentando ogni 
  passo. 
  \begin{center}
    \texttt{Fetch}
  \end{center}

  \begin{enumerate}
    \item [F]
      \begin{enumerate}
        \item[1.] \( PC_{out},\; MAR_{in},\; READ,\; SELECT_4,\; ADD,\; Z_{in} \) 

          \noindent Estraggo dal program counter l'indirizzo dell'istruzione da eseguire e
          lo metto nel MAR per ottenere l'istruzione. Successivamente metto imposto
          \( SELECT_4 \) che seleziona la costante \( 4 \) dal multiplexer per sommarla
          al program counter che si trova già nella ALU, questo equivale ad andare 
          all'istruzione successiva, cioè PC + 1 word. Il risultato della somma
          viene salvato nel registro \( Z \).

        \item[2.] \( Z_{out},\; PC_{in},\; WMFC \) 

          \noindent Estraggo l'indirizzo dell'istruzione successiva dall'registro \( Z \) 
          e lo inserisco nel program counter mentre aspetto che la funzione di lettura
          del \( MAR \) venga completata.

        \item[3.] \( MDR_{out},\; IR_{in} \) 

          \noindent Una volta che viene letto il dato all'indirizzo inserito nel \( MAR \)
          il risultato viene messo in \( MDR \) e successivamente trasferito nell'\( IR \)
          completando così il fetch dell'istruzione.
      \end{enumerate}
  \end{enumerate}
\end{exercise}

\subsection{Esercizio 2 (Inc \%reg)}
\begin{exercise}
  Descrivere le microistruzioni relative alla seguente istruzione: 
  \begin{center}
    \texttt{INC \%EAX}
  \end{center}

  \begin{enumerate}
    \item [F]
      \begin{enumerate}
        \item [1.] \( PC_{out},\; MAR_{in},\; READ,\; SELECT_4,\; ADD,\; Z_{in} \) 
        \item [2.] \( Z_{out},\; PC_{in},\; WMFC \) 
        \item [3.] \( MDR_{out},\; IR_{in}, \) 
      \end{enumerate}

    \item [DE]
      \begin{enumerate}
        \item [4.] \( EAX_{out},\; SELECT_0,\; CB,\; ADD,\; Z_{in} \) 
        \item [5.] \( Z_{out},\; EAX_{in},\; END \) 
      \end{enumerate}
  \end{enumerate}
\end{exercise}

\subsection{Esercizio 3 (Inc var)}
\vspace{2cm}
\begin{exercise}
  Descrivere le microistruzioni relative a questa istruzione:
  \begin{center}
    \texttt{INC variable}
  \end{center}
  Si assume che la variabile 'variable' sia costituita da un indirizzo immediato, direttamente codificato nell'istruzione ($IR\_imm\_field\__{out}$).

  \begin{enumerate}
    \item [F]
      \begin{enumerate}
        \item [1.] \( PC_{out},; MAR_{in},\; SELECT_4,\; ADD,\; Z_{in} \) 
        \item [2.] \( Z_{out},\; PC_{in},\; WMFC \) 
        \item [3.] \( MDR_{out},\; IR_{in} \) 
      \end{enumerate}
    \item [DE]
      \begin{enumerate}
        \item [4.] \( IR\_imm\_field_{out},\; MAR_{in},\; READ,\; WMFC \)
        \item [5.] \( MDR_{out},\; SELECT_0,\; CB,\; ADD,\; Z_{in} \) 
        \item [6.] \( Z_{out},\; MDR_{in},\; WRITE,\; WMFC,\; END \) 
      \end{enumerate}
  \end{enumerate}
\end{exercise}

\subsection{Esercizio 4 (CALL rel)}
\begin{exercise}
  Si descrivano le microistruzioni relative alla seguente istruzione:
  \begin{center}
    \texttt{CALL etichetta}
  \end{center}
  Assunzioni:
  \begin{itemize}
    \item 
      La flag 'etichetta' è costituita da un indirizzo immediato, direttamente codificato 
      nell'istruzione;
    \item salto relativo, in quanto l'indirizzo di salto è specificato tramite 
      etichetta simbolica.
  \end{itemize}

  \begin{enumerate}
    \item [F]
      \begin{enumerate}
        \item [1.] \( PC_{out},\; MAR_{in},\; READ,\; SELECT_4,\; ADD,\; Z_{in} \) 
        \item [2.] \( Z_{out},\; PC_{in},\; WMFC \) 
        \item [3.] \( MDR_{out},\; IR_{in} \) 
      \end{enumerate}
    \item [DE]
      \begin{enumerate}
        \item [4.] \( ESP_{out},\; SELECT_4,\; SUB,\; Z_{in} \)
        \item [5.] \( Z_{out},\; MAR_{in},\; ESP_{in}\) 
        \item [6.] \( PC_{out},\; MDR_{in},\; WRITE,\; V_{in} \) 
        \item [7.] \( IR\_imm\_field_{out},\; SELECT_V,\; ADD,\; Z_{in},\; WMFC \) 
        \item [8.] \( Z_{out},\; PC_{in},\; END \) 
      \end{enumerate}
  \end{enumerate}
\end{exercise}

\subsection{Esercizio 5 (CALL abs)}
\begin{exercise}
  Descrivere le microistruzioni della seguente istruzione:
  \begin{center}
    \texttt{CALL (\%EAX, \%EBX)}
  \end{center}

  Assunzioni:
  \begin{itemize}
    \item 
      L'indirizzo a cui saltare è \%eax + \%ebx (indirizzamento indiretto) 
    \item 
      Salto assoluto.
  \end{itemize}

  \begin{enumerate}
    \item [F]
      \begin{enumerate}
        \item [1.] \( PC_{out},\; MAR_{in},\; READ,\; SELECT_4,\; ADD,\; Z_{in} \) 
        \item [2.] \( Z_{out},\; PC_{in},\; WMFC \) 
        \item [3.] \( MDR_{out},\; IR_{in} \) 
      \end{enumerate}
    \item [DE]
      \begin{enumerate}
        \item [4.] \( ESP_{out},\; SELECT_4,\; SUB,\; Z_{in} \)  
        \item [5.] \( Z_{out},\; MAR_{in},\; ESP_{in} \) 
        \item [6.] \( PC_{out},\; MDR_{in},\; WRITE \) 
        \item [7.] \( EAX_{out},\; V_{in},\; WMFC \) 
        \item [8.] \( EBX_{out},\; SELECT_V,\; ADD,\; Z_{in} \) 
        \item [9.] \( Z_{out},\; PC_{in},\; END \) 
      \end{enumerate}
  \end{enumerate}
\end{exercise}

\subsection{Esercizio 6 (RET)}
\begin{exercise}
  Descrivere le microistruzioni della seguente istruzione:
  \begin{center}
    \texttt{RET}
  \end{center}
  (ritorno di una funzione).

  \begin{enumerate}
    \item [F]
      \begin{enumerate}
        \item [1.] \( PC_{out},\; MAR_{in},\; READ,\; SELECT_4,\; ADD,\; Z_{in} \) 
        \item [2.] \( Z_{out},\; PC_{in},\; WMFC \) 
        \item [3.] \( MDR_{out},\; IR_{in} \) 
      \end{enumerate}
    \item [DE]
      \begin{enumerate}
        \item [4.] \( ESP_{out},\; MAR_{in},\; READ,\; SELECT_4,\; ADD,\; Z_{in} \) 
        \item [5.] \( Z_{out},\; ESP_{in},\; WMFC \) 
        \item [6.] \( MDR_{out},\; PC_{in},\; END \) 
      \end{enumerate}
  \end{enumerate}
\end{exercise}

\subsection{Esercizio 7 (JMP)}
\begin{exercise}
  Descrivere le microistruzioni della seguente istruzione:
  \begin{center}
    \texttt{JMP (\%eax)}
  \end{center}
  (Assumo che il salto sia relativo al PC)
  \begin{enumerate}
    \item [F]
      \begin{enumerate}
        \item [1.] \( PC_{out},\; MAR_{in},\; READ,\; SELECT_4,\; ADD,\; Z_{in} \) 
        \item [2.] \( Z_{out},\; PC_{in},\; WMFC \) 
        \item [3.] \( MDR_{out},\; IR_{in} \) 
      \end{enumerate}
    \item [DE]
      \begin{enumerate}
        \item [4.] \( EAX_{out},\; MAR_{in},\; READ \) 
        \item [5.] \( WMFC,\; PC_{out},\; V_{in} \) 
        \item [6.] \( MDR_{out},\; SELECT_V,\; ADD,\; Z_{in} \) 
        \item [7.] \( Z_{out},\; PC_{in},\; END \) 
      \end{enumerate}
  \end{enumerate}
\end{exercise}

\subsection{Esercizio 8 (JZ)}
\begin{exercise}
  Descrivere le microistruzioni della seguente istruzione:
  \begin{center}
    \texttt{JZ (\%eax)}
  \end{center}
  (Assumo che il salto sia relativo al PC)
  \begin{enumerate}
    \item [F]
      \begin{enumerate}
        \item [1.] \( PC_{out},\; MAR_{in},\; READ,\; SELECT_4,\; ADD,\; Z_{in} \) 
        \item [2.] \( Z_{out},\; PC_{in},\; WMFC \) 
        \item [3.] \( MDR_{out},\; IR_{in} \) 
      \end{enumerate}
    \item [DE]
      \begin{enumerate}
        \item [4.] \( if\,(!ZERO)\,END,\; EAX_{out},\; MAR_{in},\; READ \) 
        \item [5.] \( WMFC,\; PC_{out},\; V_{in} \) 
        \item [6.] \( MDR_{out},\; SELECT_V,\; ADD,\; Z_{in} \) 
        \item [7.] \( Z_{out},\; PC_{in},\; END \) 
      \end{enumerate}
  \end{enumerate}
\end{exercise}

\subsection{Esercizio 9 (Call abs)}
\begin{exercise}
  Elencare e commentare le micro istruzioni relative alla completa esecuzione
  (caricamento, decodifica, esecuzione) della seguente istruzione assembler: 
  \begin{center}
    \texttt{CALL (\%eax, \%ebx)}
  \end{center}
  dove:
  \begin{itemize}
    \item 
      si assume che l'indirizzo a cui saltare sia memorizzato nella locazione di memoria
      puntata dal valore \%eax+\%ebx (indirizzamento indiretto);
    \item 
      si assume salto assoluto (salvo diversamente specificato);
    \item 
      si assume l'utilizzo di architettura Intel dove lo stack cresce verso indirizzi
      di memoria inferiori rispetto ed \%ESP punta alla cima occupata dallo stack.
  \end{itemize}
  
  \begin{enumerate}
    \item [F]
      \begin{enumerate}
        \item[1.] \( PC_{out},\; MAR_{in},\; READ,\; SELECT_4,\; ADD,\; Z_{in} \) 

          \noindent Estraggo dal program counter l'indirizzo dell'istruzione da eseguire e
          lo metto nel MAR per ottenere l'istruzione. Successivamente metto imposto
          \( SELECT_4 \) che seleziona la costante \( 4 \) dal multiplexer per sommarla
          al program counter che si trova già nella ALU, questo equivale ad andare 
          all'istruzione successiva, cioè PC + 1 word. Il risultato della somma
          viene salvato nel registro \( Z \).

        \item[2.] \( Z_{out},\; PC_{in},\; WMFC \) 

          \noindent Estraggo l'indirizzo dell'istruzione successiva dall'registro \( Z \) 
          e lo inserisco nel program counter mentre aspetto che la funzione di lettura
          del \( MAR \) venga completata.

        \item[3.] \( MDR_{out},\; IR_{in} \) 

          \noindent Una volta che viene letto il dato all'indirizzo inserito nel \( MAR \)
          il risultato viene messo in \( MDR \) e successivamente trasferito nell'\( IR \)
          completando così il fetch dell'istruzione.
      \end{enumerate}
    \item [DE]
      \begin{enumerate}
        \item [4.] \( ESP_{out},\; SELECT_4,\; SUB,\; Z_{in} \)  

          \noindent Sottraggo 4 (1 word) dallo stack pointer in modo da creare spazio
          nello stack per inserire il program counter dell'istruzione successiva da
          eseguire.
        \item [5.] \( Z_{out},\; MAR_{in},\; ESP_{in} \) 

          \noindent Estraggo il risultato della somma e lo inserisco all'interno del
          registro ESP e nel MAR in modo da poter scrivere il PC in quell'indirizzo.
        \item [6.] \( PC_{out},\; MDR_{in},\; WRITE \) 

          \noindent Ottengo il program counter e lo inserisco nel MDR per poter richiamare
          la funzione WRITE che scrive il dato presente in MDR all'interno dell'indirizzo
          presente nel registro MAR. Fino a questo punto è stata eseguita una push del
          program counter nello stack.
        \item [7.] \( EAX_{out},\; V_{in},\; WMFC \) 

          \noindent Metto il valore del registro EAX nel registro V per poter sommare.
          In questa istruzione richiamo la funzione WMFC per aspettare che la funzione
          di memoria venga eseguita.
        \item [8.] \( EBX_{out},\; SELECT_V,\; ADD,\; Z_{in} \) 

          \noindent Inserisco EBX nel bus e seleziono il registro V per la somma mettendo
          il risultato di EAX+EBX nel registro Z
        \item [9.] \( Z_{out},\; PC_{in},\; END \) 

          \noindent Estraggo il risultato dal registro Z, che sarebbe l'indirizzo a cui
          effettuare il salto, e lo inserisco direttamente nel program counter visto che
          questo è stato specificato come salto assoluto.
      \end{enumerate}
  \end{enumerate}
\end{exercise}

\subsection{Esercizio 10 (XCHG)}
\begin{exercise}
  Elencare e commentare le micro istruzioni relative alla completa esecuzione (caricamento, decodifica, esecuzione) della seguente istruzione assembler: 

  \begin{center}
    \texttt{XCHG variabile, \%eax}
  \end{center}

  \begin{itemize}
    \item 
      si assume che la variabile sia costituita da un indirizzo immediato direttamente
      codificato dentro l'istruzione (con il segnale IR\_imm\_field disponibile).
  \end{itemize}

  \begin{enumerate}
    \item [F]
      \begin{enumerate}
        \item[1.] \( PC_{out},\; MAR_{in},\; READ,\; SELECT_4,\; ADD,\; Z_{in} \) 

          \noindent Estraggo dal program counter l'indirizzo dell'istruzione da eseguire e
          lo metto nel MAR per ottenere l'istruzione. Successivamente metto imposto
          \( SELECT_4 \) che seleziona la costante \( 4 \) dal multiplexer per sommarla
          al program counter che si trova già nella ALU, questo equivale ad andare 
          all'istruzione successiva, cioè PC + 1 word. Il risultato della somma
          viene salvato nel registro \( Z \).

        \item[2.] \( Z_{out},\; PC_{in},\; WMFC \) 

          \noindent Estraggo l'indirizzo dell'istruzione successiva dall'registro \( Z \) 
          e lo inserisco nel program counter mentre aspetto che la funzione di lettura
          del \( MAR \) venga completata.

        \item[3.] \( MDR_{out},\; IR_{in} \) 

          \noindent Una volta che viene letto il dato all'indirizzo inserito nel \( MAR \)
          il risultato viene messo in \( MDR \) e successivamente trasferito nell'\( IR \)
          completando così il fetch dell'istruzione.
      \end{enumerate}
    \item [D]
      \begin{enumerate}
        \item [4.] \( IR\_imm\_field_{out},\; MAR_{in},\; READ \) 

          \noindent Viene preso l'indirizzo della variabile e viene inserito nel MAR
          per poter leggere il valore contenuto.
        \item [5.] \( EAX_{out},\; SELECT_0,\; ADD,\; Z_{in},\; WMFC \)

          \noindent Il valore di EAX viene sommato a 0 per poter salvare il valore
          all'interno del registro Z della ALU, in modo da poter effettuare lo scambio.
          Infine si aspetta che la funzione di lettura venga completata.
      \end{enumerate}
    \item [E]
      \begin{enumerate}
        \item [6.] \( MDR_{out},\; EAX_{in} \) 

          \noindent Viene estratto il valore della variabile dal registro MDR e viene
          inserito nel registro EAX.
        \item [7.] \( Z_{out},\; MDR_{in},\; WRITE \) 

          \noindent Viene preso il vecchio valore di EAX e viene messo in MDR per poterlo
          scrivere all'indirizzo della variabile che si trova ancora nel MAR.
        \item [8.] \( WMFC,\; END \) 

          \noindent Si aspetta che la funzione di scrittura venga completata e viene
          interrotta l'istruzione.
      \end{enumerate}
  \end{enumerate}
\end{exercise}

\end{document}
