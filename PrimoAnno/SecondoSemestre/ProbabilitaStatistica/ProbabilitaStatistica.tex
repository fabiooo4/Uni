\documentclass[a4paper]{article}

\usepackage[utf8]{inputenc}
\usepackage[T1]{fontenc}
\usepackage{textcomp}
\usepackage[italian]{babel}
\usepackage{amsmath, amssymb}
\usepackage[makeroom]{cancel}
\usepackage{amsfonts}
\usepackage{mdframed}
\usepackage{xcolor}
\usepackage{float}
\usepackage{tikz}
\usepackage{pgfplots}
\pgfplotsset{width=10cm,compat=1.9}
\usetikzlibrary{pgfplots.fillbetween, pgfplots.statistics}
\pgfplotsset{compat=newest, ticks=none}
\usepackage{graphicx}
\graphicspath{{./figures/}}

\usepackage{import}
\usepackage{pdfpages}
\usepackage{transparent}
\usepackage{xcolor}

\usepackage{hyperref}
\hypersetup{
    colorlinks=false,
}

\usepackage{ntheorem}
\newtheorem{theorem}{Teorema}

% Useful definitions frame
\theoremstyle{break}
\theoremheaderfont{\bfseries}
\newmdtheoremenv[%
	linecolor=gray,leftmargin=0,%
	rightmargin=0,
	innertopmargin=8pt,%
	ntheorem]{define}{Definizioni utili}[section]

% Example frame
\theoremstyle{break}
\theoremheaderfont{\bfseries}
\newmdtheoremenv[%
	linecolor=gray,leftmargin=0,%
	rightmargin=0,
	innertopmargin=8pt,%
	ntheorem]{example}{Esempio}[section]

% Important definition frame
\theoremstyle{break}
\theoremheaderfont{\bfseries}
\newmdtheoremenv[%
	linecolor=gray,leftmargin=0,%
	rightmargin=0,
	backgroundcolor=gray!40,%
	innertopmargin=8pt,%
	ntheorem]{definition}{Definizione}[section]

% Exercise frame
\theoremstyle{break}
\theoremheaderfont{\bfseries}
\newmdtheoremenv[%
	linecolor=gray,leftmargin=0,%
	rightmargin=0,
	innertopmargin=8pt,%
	ntheorem]{exercise}{Esercizio}[section]


% figure support
\usepackage{import}
\usepackage{xifthen}
\pdfminorversion=7
\usepackage{pdfpages}
\usepackage{transparent}
\newcommand{\incfig}[1]{%
	\def\svgwidth{\columnwidth}
	\import{./figures/}{#1.pdf_tex}
}

\pdfsuppresswarningpagegroup=1

\begin{document}
\begin{titlepage}
	\begin{center}
		\vspace*{1cm}

		\Huge
		\textbf{Probabilità e Statistica\\Esercizi}

		\vspace{0.5cm}
		\LARGE
		UniVR - Dipartimento di Informatica

		\vspace{1.5cm}

		\textbf{Fabio Irimie}

		\vfill


		\vspace{0.8cm}


		2° Semestre 2023/2024

	\end{center}
\end{titlepage}


\tableofcontents
\pagebreak

% Libro: S.M ROSS, Probabilità e Statistica per l'ingegneria e le scienze Apogeo, 2015 (ed. 3)

\section{Cos'è la probabilità e la statistica?}
La statistica è una scienza che si occupa di raccogliere, organizzare, analizzare
e interpretare i dati. Nella statistica si cerca di estrapolare informazioni da
esperimenti \textbf{aleatori} (esperimenti che non si possono ripetere esattamente allo stesso
modo) e di prendere decisioni basate su queste informazioni. Ogni esperimento aleatorio
ha bisogno di un \textbf{modello probabilistico} che ne descriva le caratteristiche principali.

\subsection{Popolazione, variabili e campione}
\begin{itemize}
	\item \textbf{Popolazione}: tutti i possibili oggetti di un'indagine statistica
	\item \textbf{Individuo}: un singolo oggetto della popolazione
	\item \textbf{Variabile}: una qualsiasi caratteristica di un individuo della
	      popolazione soggetta a possibili variazioni da individuo a individuo; è l'oggetto
	      di interesse in uno studio
	\item \textbf{Range della variabile}: \( R_x \) è l'insieme di tutti i possibili
	      valori che la variabile \( x \) può assumere
	\item \textbf{Campione}: un sottoinsieme rappresentativo della popolazione
	      composto dalle variabili relative ad un sottoinsieme di individui
	\item \textbf{Realizzazione del campione di dimension \( n \)}: (post esperimento)
	      le osservazioni del campione:\[
		      \underline{x} = (\tilde{x}_1, \ldots, \tilde{x}_n)
	      \]
	\item \textbf{Range dei dati}: \( \mathcal{R}_{\underline{x}} \) i valori che la
	      variabile può assumere tra il minimo e il massimo
\end{itemize}

\subsection{Parametro e Stima}
\begin{itemize}
	\item \textbf{Parametro}:una misura che descrive una proprietà dell'intera popolazione
	\item \textbf{Stima}: una misura che descrive una proprietà del campione e che
	      fornisce informazioni sul parametro
\end{itemize}

\subsection{Variabili}
Le variabili possono essere di diverso tipo:
\begin{itemize}
	\item \textbf{Variabili qualitative nominali}:
	      \begin{itemize}
		      \item \textbf{Ordinali}: possono essere ordinate
		      \item \textbf{Non ordinali}: non possono essere ordinate
	      \end{itemize}
	      I valori che assumono si definiscono anche \textbf{modalità}
	\item \textbf{Variabili quantitative}:
	      Sono valori numerici e si distinguono in:
	      \begin{itemize}
		      \item \textbf{Aleatorie continue}: derivano da processi di misura e assumono
		            i loro range (valori che possono assumere). Sono sottoinsiemi reali
		      \item \textbf{Aleatorie discrete}: derivano da processi di conteggio e
		            assumono valori interi
	      \end{itemize}
\end{itemize}


\section{Statistica descrittiva}
Consiste nella raccolta, organizzazione, rappresentazione e analisi dei \textbf{dati}.

\subsection{Strumenti di sintesi}
\subsubsection{Tabelle di frequenza}
Sono tabelle di frequenze di individui con una certa caratteristica o aventi una caratteristica
appartenente ad un certo intervallo.
\begin{itemize}
	\item \textbf{Frequenza assoluta}: conteggio del numero di individui
	\item \textbf{Frequenza relativa}: percentuale del numero di individui
	\item \textbf{Frequenza cumulativa}: conteggio o percentuale del numero di individui
	      fino ad un certo punto
\end{itemize}

\subsubsection{Distribuzioni}
Sono rappresentazioni del modo in cui diverse \textbf{modalità} si distribuiscono tra
gli individui di una popolazione.
\begin{itemize}
	\item \textbf{Caso discreto}: \( f \): valore variabile \( \to  \) frequenza relativa
	\item \textbf{Caso continuo o numerabile}: \( f \): intervallo di valori variabile \( \to  \) frequenza relativa
\end{itemize}

\subsubsection{Distribuzioni cumulative}
Sono distribuzioni che rappresentano la frequenza cumulativa di una variabile. Possono essere:
\begin{itemize}
	\item \textbf{Caso discreto}: \( f \): valore variabile \( \to  \) frequenza cumulaiva
	      relativa
	\item \textbf{Caso continuo o numerabile}: \( f \): intervallo \( \to  \) frequenza
	      cumulativa relativa
\end{itemize}

\subsubsection{Grafici}
Sono rappresentazioni grafiche delle distribuzioni. Possono essere:
\begin{itemize}
	\item \textbf{Istogrammi}: è costituito da rettangoli, insistenti sulle classi
	      della partizione, attigui le cui aree sono confrontabili con le probabilità.
	      \[
		      \text{area rettangolo} \quad i = h_i \cdot |\pi _i| \approx P_X(\pi ) \approx f_i
	      \]
	      \[
		      h_i = \frac{f_i}{|\pi _i|} \quad \text{per ogni} \; i \in I
	      \]
	      L'area del rettangolo che insiste sulla classe \( \pi_i  \) della partizione è pari
	      alla frequenza relativa della classe, quindi l'area torale è 1.
	\item \textbf{Diagrammi a barre}: rappresentano le frequenze di una variabile. Le barre
	      sono separate e la loro altezza è proporzionale alla frequenza
	\item \textbf{Diagrammi a torta}: rappresentano le frequenze relative di una variabile
	\item \textbf{Boxplot}: rappresentano le frequenze di una variabile
	\item \textbf{Poligono di frequenza (ogiva)}: è un grafico a linee continue che ha
	      sull'asse delle ordinate le frequenze cumulative. Questo tipo di grafici è il più comune
	      per rappresentare le frequenze cumulative.
\end{itemize}


\section{Frequenze}
Siano \( \underline{x} = (\tilde{x}_1, \ldots, \tilde{x}_n) \) una realizzazione del
campione di dimensione \( n \) e \( \mathcal{R}_{\underline{x}} \) il range dei dati.
Si dice \textbf{partizione} di \( \mathcal{R}_{\underline{x}} \):
\[
	\pi = \{\pi_i\}_{i \in I}
\]
La \textbf{classe i-esima} è l'elemento i-esimo della partizione

\subsection{Frequenze campionarie}
\subsubsection{Frequenza assoluta}
Si dice \textbf{frequenza assoluta} \( n_i \) per ogni \( i \in I \) il numero di
osservazioni che appartengono a \( \pi_i  \), cioè:
\[
	n_i = card(\tilde{x}_j \in \pi_i, \quad j = 1, \ldots, n) \quad \text{(cardinalità)}
\]
\[
	0 \le n_i \le n, \; \text{per ogni}\; i \in I \quad e \quad \sum_{i \in I} n_i = n
\]

\subsubsection{Frequenza relativa}
Si dice \textbf{frequenza relativa} \( f_i \) per ogni \( i \in I \) la percentuale delle
osservazioni che appartengono a \( \pi _i \), cioè:
\[
	f_i = \frac{n_i}{n}
\]
\[
	0 \le f_i \le 1, \; \text{per ogni}\; i \in I \quad e \quad \sum_{i \in I} f_i = 1
\]

\subsection{Frequenze cumulative}
\subsubsection{Frequenza cumulativa assoluta}
Si dice \textbf{frequenza cumulativa assoluta} \( N_i \) il numero di osservazioni che
appartengono alle classi \( \pi_h \), con \( h \le i \), cioè:
\[
	N_i = \sum_{h=1}^{i} n_h
\]
\[
	0 \le N_i \le n, \; \text{per ogni}\; i \in I \quad e \quad N_i \le N_j, \; i<j
\]

\subsubsection{Frequenza cumulativa relativa}
Si dice \textbf{frequenza cumulativa relativa} \( F_i \) della i-esima classe la somma
delle frequenze relative delle classi \( \pi _h \), con \( h \le i \), cioè:
\[
	F_i = \sum_{h=1}^{i} f_h = \frac{1}{n}N_i = \frac{1}{n}N_{i-1} + f_i
\]
\[
	0 \le F_i \le 1, \; \text{per ogni}\; i \in I \quad e \quad F_i \le F_j, \; i<j
\]

\section{Statistica descrittiva}
\subsection{Indici statistici}
Sono misure quantitative che fornicono informazioni sulla distribuzione di una certa
caratteristica.
\subsubsection{Indici di posizione o centralità}
Forniscono informazioni del valore attorno al quale si posizionano i dati. Consentono
di valutare l'ordine di grandezza della variabile aleatoria e aiutano a "localizzare"
la distribuzione. Sono espressi nella stessa unità di misura della variabile.

\vspace{1em}
\noindent Sia \( \underline{x} = (\tilde{x_1}, \ldots, \tilde{x_n}) \) un campione di dimensione \( n \).
\begin{itemize}
	\item \textbf{Media campionaria}: è il valore medio dei dati (baricentro dei dati):
	      \[
		      \overline{x} = \frac{1}{n} \sum_{j=1}^{n} \tilde{x}_j
	      \]
	\item \textbf{Moda campionaria}: \( m \) , valore che si ripete più frequentemente. Ci possono
	      essere più valori modali.

	      \vspace{1em}
	      \noindent Sia \( \underline{y} = (y_1, \ldots, y_n) \) il campione
	      ordinato (\( y_i \in \{\tilde{x_1}, \ldots, \tilde{x_n}\}  \) e \( y_i \le y_{i+1} \) )
	\item \textbf{Mediana campionaria}:\( M \): è il valore centrale del campione, una
	      volta ordinato.
	      \[
		      M = \begin{cases}
			      y_{\frac{n+1}{2}}                                & \text{se } n \text{ è dispari} \\
			      \frac{1}{2}(y_{\frac{n}{2}} + y_{\frac{n}{2}+1}) & \text{se } n \text{ è pari}
		      \end{cases}
	      \]
\end{itemize}

\subsubsection{Indici di dispersione}
Forniscono informazioni su quanto i dati si disperdono attorno ad un valore centrale. Sono:
\begin{itemize}
	\item \textbf{Range}: differenza tra il massimo e il minimo valore:
	      \[
		      r = \underset{j \in \{1, \ldots, n\} }{\max} \tilde{x}_j - \underset{j \in \{1, \ldots, n\} }{\min} \tilde{x}_j
	      \]
	\item \textbf{Scarto Quadratico Medio campionario}: misura la dispersione dei dati attorno alla media
	      \[
		      s'^2 = \frac{1}{n} \sum_{j=1}^{n} (\tilde{x_j} - \bar{x})^2
	      \]
	\item \textbf{Varianza campionaria}: misura la dispersione dei dati attorno alla media
	      \[
		      s^2 = \frac{1}{n-1} \sum_{j=1}^{n} (\tilde{x_j} - \bar{x})^2
	      \]
	\item \textbf{Deviazione standard campionaria}: misura la distanza dei dati attorno alla media
	      \[
		      s = \sqrt{s^2} = \sqrt{\frac{1}{n-1} \sum_{j=1}^{n} (\tilde{x_j} - \bar{x})^2}
	      \]
	      Per interpretare la deviazione standard si possono definire
	      \textbf{valori usuali} di una variabile i valori del campione compresi tra:
	      \begin{itemize}
		      \item \textbf{Minimo valore "usuale"}: media campionaria - 2 deviazioni standard
		      \item \textbf{Massimo valore "usuale"}: media campionaria + 2 deviazioni standard
	      \end{itemize}

\end{itemize}

\subsubsection{Indici di forma}
Sia \( \underline{x} = (\tilde{x}_1, \ldots, \tilde{x}_n) \) un campione di dimensione
\( n \).
\begin{itemize}
	\item \textbf{Asimmetria / Skewness}: misura la simmetria della distribuzione
	      \[
		      \gamma_1 = \frac{1}{n-1} \sum_{j=1}^{n} \left( \frac{\tilde{x}_j - \bar{x}}{s} \right)^3
	      \]
	      \begin{itemize}
		      \item \( \gamma_1 > 0 \): distribuzione asimmetrica a destra (con coda più lunga
		            a destra)
		      \item \( \gamma_1 < 0 \): distribuzione asimmetrica a sinistra (con coda più
		            lunga a sinistra)
		      \item \( \gamma_1 = 0 \): distribuzione simmetrica
	      \end{itemize}
	\item \textbf{Curtosi}: misura la "appuntitura" della distribuzione
	      \[
		      \gamma_2 = \frac{1}{n-1} \sum_{j=1}^{n} \left( \frac{\tilde{x}_j - \bar{x}}{s} \right)^4
	      \]
	      \begin{itemize}
		      \item \( \gamma_2 = 3 \): curtosi della normale standard, (variabile di riferimento)
		      \item \( \gamma_2 > 3 \): ci sono meno valori agli estremi di quanto aspettato, e
		            di conseguenza si ha una minore dispersione dei dati. In tal caso la distribuzione
		            risulta abbastanza appuntita
		      \item \( \gamma_2 < 3 \): ci sono più valori agli estremi di quanto aspettato, e
		            di conseguenza si ha una maggiore dispersione dei dati. In tal caso la distribuzione
		            risulta piatta
	      \end{itemize}
\end{itemize}

\subsubsection{Indici di posizione relativi}
Rappresentano indici di posizione, ma non centrali, bensì indici di posizionamento
relativo.
\begin{itemize}
	\item \textbf{Percentili}: Se \( p \) è un numero tra \( 0 \) e \( 100 \),
	      il \textbf{percentile di ordine p} (o \( p \)-esimo percentile, se \( p \) è intero)
	      è il dato che delimita il primo \( p\% \) dei dati (ordinati) dai rimanenti dati.
	\item \textbf{Quartili}: Valori che separano i dati in quattro parti, una volta ordinati.
	      \begin{figure}[H]
		      \centering
		      \begin{tikzpicture}
			      \draw[thick, ->] (0,0) -- (12,0);
			      \foreach \x in {1,2,3,4,5,6,7,8,9,10,11}
			      \draw (\x cm,3pt) -- (\x cm,-3pt);

			      \draw (0,-0.2) -- ++(0,-0.3) -- ++(2.95,0) node[midway, below]
			      {Primo Quartile} -- ++(0,0.3);

			      \draw (3,-0.2) -- ++(0,-0.3) -- ++(2.95,0) node[midway, below]
			      {Secondo Quartile} -- ++(0,0.3);

			      \draw (6,-0.2) -- ++(0,-0.3) -- ++(2.95,0) node[midway, below]
			      {Terzo Quartile} -- ++(0,0.3);

			      \draw (9,-0.2) -- ++(0,-0.3) -- ++(3,0) node[midway, below]
			      {Quarto Quartile} -- ++(0,0.3);

		      \end{tikzpicture}
	      \end{figure}
	      \[
		      \underline{x} = (\tilde{x_1}, \ldots, \tilde{x_n}) \quad \text{campione di dimensione } n
	      \]
	      \[
		      \underline{y} = (y_1, \ldots, y_n) \quad \text{campione ordinato}
	      \]
	      Il primo quartile è il valore che separa il \( 25\% \) inferiore dal \( 75\% \) superiore
	      dei dati.
	      \[
		      Q_1 = \begin{cases}
			      \frac{y_\frac{n}{4} + y_\frac{n}{4}+1}{2} & \frac{n}{4} \text{  intero}    \\
			      y_{\lceil \frac{n}{4} \rceil}             & \frac{n}{4} \text{ non intero}
		      \end{cases}
	      \]
	      Il secondo quartile è il 50-esimo percentile, ovvero la mediana. È il valore che separa
	      il \( 50\% \) inferiore dal \( 50\% \) superiore dei dati.
	      \[
		      Q_2 = M = \begin{cases}
			      \frac{y_{\frac{n}{2}} + y_{\frac{n}{2}+1}}{2} & \frac{n}{2} \text{  intero}    \\
			      y_{\lceil \frac{n}{2} \rceil}                 & \frac{n}{2} \text{ non intero}
		      \end{cases}
	      \]
	      Il terzo quartil è il 75-esimo percentile, ovvero il valore che separa il \( 75\% \)
	      inferiore dal \( 25\% \) superiore dei dati.
	      \[
		      Q_3 = \begin{cases}
			      \frac{y_{\frac{3n}{4}} + y_{\frac{3n}{4}+1}}{2} & \frac{3n}{4} \text{  intero}    \\
			      y_{\lceil \frac{3n}{4} \rceil}                  & \frac{3n}{4} \text{ non intero}
		      \end{cases}
	      \]
	      Lo scarto (o distanza interquartile) è la differenza tra il terzo e il primo quartile:
	      \[
		      IR = Q_3 - Q_1
	      \]
\end{itemize}

\subsubsection{Box-Plot}
Fornisce informazioni sulla forma della distribuzione:
\begin{figure}[H]
	\centering
	\begin{tikzpicture}
		\begin{axis}[
				y=1.5cm,
				x=1.5cm,
				axis lines=left,
				axis line style={-},
				xmin=0,
				xmax=5,
				ymin=0,
				ymax=0.001,
				xtick={1,2,3,4},
				xticklabels={, \( Q_1 \), \( M \), \( Q_3 \),},
				ytick=\empty,
				xlabel={},
				ylabel={},
				xlabel style={below right},
				ylabel style={above left},
				x label style={at={(axis description cs:0.5,-0.1)},anchor=north},
				y label style={at={(axis description cs:-0.1,0.5)},anchor=south},
				title={Box-Plot},
				title style={at={(axis cs:2.5,1.5)}},
				clip=false
			]
			\addplot[boxplot prepared={
						lower whisker=0.5,
						lower quartile=1,
						median=2.5,
						upper quartile=3,
						upper whisker=4
					}, color=black] coordinates {};

			% Draw dashed lines from whiskers and quartiles to x axis
			\draw[dashed] (axis cs:0.5,0.6) -- (axis cs:0.5,0) node[below, scale=0.7] {Minimo};
			\draw[dashed] (axis cs:1,0.6) -- (axis cs:1,0) node[below, scale=0.7] {$Q_1$};
			\draw[dashed] (axis cs:2.5,0.6) -- (axis cs:2.5,0) node[below, scale=0.7] {Mediana};
			\draw[dashed] (axis cs:3,0.6) -- (axis cs:3,0) node[below, scale=0.7] {$Q_3$};
			\draw[dashed] (axis cs:4,0.6) -- (axis cs:4,0) node[below, scale=0.7] {Massimo};

			\draw (axis cs:0.5,-0.3) -- ++(0,-0.15) -- (axis cs:1,-0.45) node[midway, below] {\( 25\% \) } -- ++(0,0.15);
			\draw (axis cs:1,-0.3) -- ++(0,-0.15) -- (axis cs:2.5,-0.45) node[midway, below] {\( 25\% \) } -- ++(0,0.15);
			\draw (axis cs:2.5,-0.3) -- ++(0,-0.15) -- (axis cs:3,-0.45) node[midway, below] {\( 25\% \) } -- ++(0,0.15);
			\draw (axis cs:3,-0.3) -- ++(0,-0.15) -- (axis cs:4,-0.45) node[midway, below] {\( 25\% \) } -- ++(0,0.15);
		\end{axis}
	\end{tikzpicture}
\end{figure}

\subsection{Outliers}
\begin{figure}[H]
	\begin{definition}
		Gli \textbf{Outliers} sono valori estremi, insolitamente grandi o piccoli, rispetto
		al resto dei dati. La loro presenza potrebbe distorcere i risultati dell'analisi, e
		richiede pertanto un'analisi più accurata.
		\[
			x \le Q_1 - 1.5 \cdot IR \quad \text{oppure} \quad x \ge Q_3 + 1.5 \cdot IR
		\]
	\end{definition}
\end{figure}

\subsubsection{Outliers deboli}
Si dicono outliers deboli:
\[
	Q_1 - 3 \cdot IR < x \le Q_1 - 1.5 \cdot IR
\]
\begin{center}
	oppure
\end{center}
\[
	Q_3 + 1.5 \cdot IR < x \le Q_3 + 3 \cdot IR
\]

\subsubsection{Outliers forti}
Si dicono outliers forti:
\[
	x \le Q_1 - 3 \cdot IR
\]
\begin{center}
	oppure
\end{center}
\[
	x \ge Q_3 + 3 \cdot IR
\]
\section{Statistica descrittiva bivariata}
La statistica descrittiva bivariata si occupa di studiare la relazione tra due variabili.

\subsection{Relazione tra 2 variabili}
\begin{itemize}
	\item \textbf{Correlazione}: Associazione \textbf{lineare} tra 2 variabili. La forza
	      dell'associazione è data dal \textbf{coefficiente di correlazione}.
	\item \textbf{Regressione}: dipendenza di una variabile (dipendente) da
	      un’altra variabile (indipendente)
\end{itemize}

\subsubsection{Correlazione}
Sia \( (\underline{x}, \underline{y}) = ((\tilde{x}_1, \tilde{y}_1), \ldots (\tilde{x}_n, \tilde{y}_n)) \)
un campione di dimensione \( n \) di due misure \( x \) ed \( y \), con medie campionarie
\( \bar{x} \) e \( \bar{y} \), deviazioni standard campionarie \( (s_x, s_y) \).
\[
	\bar{x} = \frac{1}{n} \sum_{i=1}^{n} \tilde{x}_i
\]
\[
	\bar{y} = \frac{1}{n} \sum_{i=1}^{n} \tilde{y}_i
\]
\[
	s_x = \sqrt{\frac{1}{n-1} \sum_{i=1}^{n} (\tilde{x}_i - \bar{x})^2}
\]
\[
	s_y = \sqrt{\frac{1}{n-1} \sum_{i=1}^{n} (\tilde{y}_i - \bar{y})^2}
\]
Il coefficiente di correlazione campionario è definito come:
\[
	\rho_n \stackrel{\Delta}{=} \frac{\sum_{i=1}^{n} (\tilde{x}_i - \bar{x})
		(\tilde{y}_i - \bar{y})}{\sqrt{\sum_{i=1}^{n} (\tilde{x}_i - \bar{x})^2
			\sum_{i=1}^{n} (\tilde{y}_i - \bar{y})^2}}
\]
Il risultato sarà un numero compreso tra \( -1 \) e \( 1 \):
\[
	| \rho_n | \le 1
\]
Questo indice misura il grado di dipendenza lineare tra le due variabili.

\begin{table}[H]
	\centering
	\begin{tabular}{|c|c|}
		\hline
		\( | \rho_n | \)      & Grado di correlazione tra \( \underline{x}\;e\;\underline{y} \) \\
		\hline
		\( \rho_n = -1 \)     & massima correlazione lineare inversa                            \\
		\( -1 < \rho_n < 0 \) & correlazione inversa                                            \\
		\( \rho_n = 0 \)      & assenza di correlazione                                         \\
		\( 0 < \rho_n < 1 \)  & correlazione diretta                                            \\
		\( \rho_n = 1 \)      & massima correlazione lineare diretta                            \\
		\hline
	\end{tabular}
\end{table}
Sono indici qualitativi:
\begin{table}[H]
	\centering
	\begin{tabular}{|c|c|}
		\hline
		\( \rho_n \)                    & Grado di correlazione tra \( \underline{x}\;e\;\underline{y} \) \\
		\hline
		\( | \rho_n | \le 0.5 \)        & scarsa correlazione                                             \\
		\( 0.5 < | \rho_n | \le 0.75 \) & correlazione moderata                                           \\
		\( 0.75 < | \rho_n | \le 0.9 \) & correlazione buona                                              \\
		\( | \rho_n | > 0.9 \)          & correlazione molto buona                                        \\
		\hline
	\end{tabular}
\end{table}

\subsubsection{Regressione}
La regressione lineare è un modello matematico che cerca di esprimere una variabile.
Per ipotesi riteniamo che due variabili siano legate da una relazione del tipo \( y = g(x) \)
\begin{enumerate}
	\item I dati accoppiati \( (x,y) \) costituiscono un campione di dati quantitativi
	\item Dallo scatter plot possiamo ipotizzare che nella \textbf{popolazione} ci sia una
	      relazione lineare del tipo:
	      \[
		      y_i = \beta_0 + \beta_1 x_i + \varepsilon_i
	      \]
	      dove \( \varepsilon_i \) è l'errore casuale, con distribuzione a campana
	      \begin{figure}[H]
		      \centering
		      \begin{tikzpicture}[scale=0.7]
			      \begin{axis}[
					      xlabel={x},
					      ylabel={Densità},
				      ]
				      \addplot[domain=-3:3, samples=100, color=red]{1/(sqrt(2*pi))*exp(-x^2/2)};
			      \end{axis}
		      \end{tikzpicture}
	      \end{figure}
	\item Cerchiamo di individuare l'equazione della \textbf{curba di regressione relativa
		      del campione}:
	      \[
		      \hat{y}_i = a + bx_i
	      \]
\end{enumerate}

\subsubsection{Determinazione dei coefficienti della retta di regressione}
L'obiettivo è quello di determinare i coefficienti \( a \) e \( b \) in modo ottimale,
affinchè la retta di regressione \( \hat{y}_i = a + bx_i \) sia il più possibile vicina
ai punti \( (x_i, y_i) \) del campione.

\vspace{1em}
\noindent Si determina quindi l'equazione generica della curva interpolante stimando i parametri in
modo da rendere \textbf{minima} la distanza al quadrato dei punti osservati dalla curva.
\[
	\sum_{i=1}^{n} (y_i - \hat{y}_i)^2 = \sum_{i=1}^{n} (y_i - a - bx_i)^2
\]
Equazioni normali:
\[
	\begin{cases}
		\sum_{i=1}^{n} y_i = na + b \sum_{i=1}^{n} x_i \\
		\sum_{i=1}^{n} x_i y_i = a \sum_{i=1}^{n} x_i + b \sum_{i=1}^{n} x_i^2
	\end{cases}
\]

\subsection{Riassunto}
\begin{itemize}
	\item Dato \textbf{un campione}: abbiamo determinato una \textbf{stima} di alcuni
	      parametri (media, deviazione standard, varianza, quartili, ...), una stima della
	      distribuzione (frequqnze relative) con grafici (istogramma [frequenza relativa],
	      diagrammi [area = frequenza relativa], boxplot [quartili, outliers])
	\item Dati \textbf{due campioni}: abbiamo determinato una \textbf{stima} di alcuni
	      parametri (media, deviazione standard, varianza, quartili, ...) ed una stima della
	      distribuzione (frequenze relative) con grafici (scatter plot, retta di regressione,
	      coefficiente di correlazione) e abbiamo fatto un \textbf{confronto}.

	      Abbiamo determinato una stima della \textbf{correlazione} e la retta di regressione lineare.
	      \[
		      \rho_n = \text{coeff. di correlazione} \quad \rho_n \approx 1
	      \]
\end{itemize}

\noindent Per capire se le informazioni tratte dal campione sono statisticamente significative
si fa riferimento alla \textbf{statistica inferenziale}. Ma bisogna essere ingrado di
parlare di probabilità e di distribuzioni teoriche (modelli probabilistici).

\section{Probabilità}
La probabilità di un evento \( A \in \mathcal{A} \) rappresenta una misura di quanto
ci si aspetta che si verifichi l'evento \( A \).

\vspace{1em}
\noindent Calcolare le probabilità non significa "prevedere il futuro", ma trovare come
distribuire un maggiore o minore \textbf{grado di fiducia} tra i vari possibili modi
in cui si potrà presentare un certo fenomeno aleatorio.


\begin{figure}[H]
	\begin{define}
		L'\textbf{ipotesi dei modelli} è lo spazio dei campioni \textbf{finito} \( \Leftrightarrow
		card(\Omega) = n < \infty\)

		\textbf{Eventi equiprobabili}:
		\[
			P(\omega_i) = P(\omega_j), \quad i,j \in \{1, \ldots, n\}
		\]
	\end{define}
\end{figure}

La probabilità di un evento \( A \in \mathcal{A} \) si calcola come:
\[
	P(A) = \frac{\text{casi favorevoli ad \( A \) }}{\text{casi possibili}} = \frac{card(A)}{card(\Omega)}
\]

\subsection{Esperimenti aleatori}
Un fenomeno \textbf{casuale}, o aleatorio, è un fenomeno \textbf{osservabile}, ma non
prevedibile. Cioè conoscendo i dati iniziali e le leggi, non possiamo prevederne il
risultato. Ciò che invece possiamo conoscere è l'insieme di tutti i possibili risultati.
\begin{itemize}
	\item \textbf{Fenomeno deterministico}: Dati + Leggi = Conoscenza
	\item \textbf{Fenomeno non deterministico}: Dati + Leggi = Non Conoscenza
\end{itemize}

\noindent Alcuni esempi di esperimenti sono:
\begin{itemize}
	\item Consideriamo tre figli di una stessa coppia. Controlliamo il sesso dei tre.
	\item Lancio un dado.
	      Controllo il numero che esce.
	\item Lancio 2 dadi.
	      Controllo i numeri che escono.
	\item Considero i piselli so che possono avere il baccello verde o giallo e il fiore
	      bianco o viola.
	      Ne estraggo uno a caso. Che caratteristiche ha?
	\item Sono ad un call center.
	      Conto il numero di telefonate
	      che arrivano in un intervallo di tempo
	\item Misuro all’altezza di un uomo
	      di 40 anni italiano
\end{itemize}

\subsection{Spazio campionario ed eventi}
È l'insieme di tutti i possibili risultati di un esperimento casuale:
\[
	\Omega = \{\omega_1, \omega_2, \ldots, \omega_n\}
\]
Uno dei possibili risultati dell'esperimento si chiama \textbf{Evento elementare}:
\[
	\{\omega_i\},\quad i = 1, \ldots, n
\]
L'\textbf{Evento} è un sottoinsieme dello spazio campione \( A \subset \Omega \) in cui
sono contenuti alcuni dei possibili eventi elementari, quelli favorevoli all'evento
considerato.
\subsection{Esperimenti}
\subsubsection{Esperimento 1: Lancio di un dado}
Prendiamo in considerazione il lancio di un dado:
\[
	\text{Lo spazio dei campioni è: } \quad \Omega = \{1, 2, 3, 4, 5, 6\}
\]
I possibili eventi sono:
\begin{enumerate}
	\item[A =] Il risultato del lancio è 1
	\item[B =] Il risultato del lancio è dispari
	\item[C =] Il risultato del lancio è maggiore di 4
	\item[D =] Il risultato del lancio è dispari non maggiore di 4
	\item[E =] Il risultato del lancio è pari
	\item[F =] Il risultato del lancio è 7
	\item[G =] Il risultato del lancio è tra 1 e 6
\end{enumerate}
\[
	A = \{1\} \quad B = \{1, 3, 5\}
\]
\[
	C = \{1, 2, 3, 4\} \quad D = \{1, 3, 5\} \bigcap \{1, 2, 3, 4\} = B \bigcap C = \{1,3\}
\]
\[
	E = \{2, 4, 6\} = \Omega \setminus B = \overline{\{1, 3, 5\} } = \overline{\tilde{B}}
\]
\[
	F = \{7\} = \overline{\Omega} = \emptyset
\]
\[
	G = \{1, 2, 3, 4, 5, 6\} = \Omega
\]

\subsubsection{Esperimento 2: Lancio di 2 dadi}
Prendiamo in considerazione il lancio di 2 dadi:
\[
	\Omega_2 = \{1, 2, 3, 4, 5, 6\} \times \{1, 2, 3, 4, 5, 6\} = \{(1,1), (1,2), \ldots, (6,6)\}
\]
L'evento: A = Esce almeno un 6 è:
\[
	A = \{(6,1), (6,2), \ldots, (6,6), (1,6), (2,6), \ldots, (5,6)\}
\]
\subsubsection{Esperimento 3: Sesso dei nascituri}
Consideriamo 3 figli di una stessa coppia. Controlliamo il sesso dei tre.

Se considero una \textbf{singola nascita} lo spazio dei campioni è:
\[
	\Omega = \{M, F\}
\]
Quindi si hanno due possibili eventi elementari:
\[
	\{\text{M}\}, \{\text{F}\}
\]
Se invece considero \textbf{tre nascite} lo spazio dei campioni è:
\[
	\Omega_3 = \{(\omega_1, \omega_2, \omega_3) \quad \omega_i \in \Omega\}
\]
quindi è costituito da tutte le \textbf{terne} ordinate di maschi e femmine.
\begin{table}[H]
	\centering
	\begin{tabular}{|c|c|c|}
		\hline
		1° Figlio & 2° Figlio & 3° Figlio \\
		\hline
		M         & M         & M         \\
		M         & M         & F         \\
		M         & F         & M         \\
		M         & F         & F         \\
		F         & M         & M         \\
		F         & M         & F         \\
		F         & F         & M         \\
		F         & F         & F         \\
		\hline
	\end{tabular}
\end{table}
\noindent Ogni terna rappresenta un \textbf{evento elementare}.

\subsubsection{Caratteristiche degli esperimenti 1-3}
\begin{itemize}
	\item Lo spazio dei campioni è finito
	\item Gli eventi sono tutte le parti di \( \Omega \), cioè tutti i possibili sottoinsiemi
	      di \( \Omega \)
\end{itemize}

\subsubsection{Esperimento 4: Tempo di attesa}
Sono ad un call center e conto il numero di telefonate che arrivano in un intervallo di tempo.

\noindent Lo spazio dei campioni è:
\[
	\Omega = \{0, 1, 2, 3, \ldots\}
\]

\textbf{Caratteristiche}:
\begin{itemize}
	\item Lo spazio dei campioni è \textbf{infinito numerabile}
	\item Gli eventi sono tutte le parti di \( \Omega \), cioè tutti i possibili sottoinsiemi
	      di \( \Omega \)
\end{itemize}

\subsubsection{Esperimento 5: Misure}
Misuro l’altezza di un uomo di 40 anni italiano.

\noindent Lo spazio dei campioni è:
\[
	\Omega \subseteq \mathbb{R}
\]

\textbf{Caratteristiche}:
\begin{itemize}
	\item Lo spazio dei campioni è un sottoinsieme di \( \mathbb{R} \), quindi è
	      \textbf{infinito non numerabile}
	\item Gli eventi sono tutti i sottointervalli di \( \mathbb{R} \), le loro unioni e le
	      loro intersezioni
\end{itemize}

\subsubsection{Tipi di esperimenti}
Gli esperimenti possono essere di diversi tipi:
\begin{itemize}
	\item \textbf{Misure di conteggio}
	\item \textbf{Misure continue}
\end{itemize}

\subsubsection{Tipi di eventi}
\begin{itemize}
	\item \textbf{Evento aleatorio}:\\
	      È un sottoinsieme dello spazio campionario, cioè \( A \subset \Omega \),
	      ad esempio il lancio di un dado un evento aleatorio potrebbe essere:
	      "esce un numero pari"
	\item \textbf{Evento elementare}:\\
	      È un evento che contiene un solo elemento, cioè \( A = \{\omega\} \), ad esempio il
	      lancio di un dado ha come eventi elementari:
	      "esce 1", "esce 2", "esce 3", "esce 4", "esce 5", "esce 6"
	\item \textbf{Eventi complementari}:\\
	      Sono eventi che si escludono a vicenda, ad esempio nel lancio di un dado:
	      E = "esce un numero pari" e \( \overline{E} \) = "esce un numero dispari"
	      sono eventi complementari
	\item \textbf{Eventi incompatibili}:\\
	      Sono eventi che non possono verificarsi contemporaneamente, ad esempio nel lancio di un dado:
	      E = "esce un numero pari" e F = "esce il numero 6" sono eventi incompatibili
	\item \textbf{Eventi compatibili}:\\
	      Sono eventi che possono verificarsi contemporaneamente, ad esempio nel lancio di un dado:
	      E = "esce un numero pari" e F = "esce il numero 2" sono eventi compatibili
	\item \textbf{Evento certo}:\\
	      È un evento che si verifica sempre, cioè \( A = \Omega \), ad esempio il lancio di
	      un dato ha sempre un risultato certo.
	\item \textbf{Evento impossibile}:\\
	      È un evento che non si verifica mai, cioè \( A = \emptyset \), ad esempio il lancio di
	      un dato non può avere come risultato 7.
\end{itemize}

\subsection{Spazio campionario e insieme degli eventi}
\begin{figure}[H]
	\begin{definition}
		Lo \textbf{spazio dei campioni \( \Omega \) } è l'insieme di tutti i possibili esiti
		(risultati). La cardinalità di uno spazio dei campioni può esssere finita, infinita
		numerabile e infinita non numerabile.
		\[
			\Omega = \{\omega_1, \omega_2, \ldots, \omega_n\} \quad \text{oppure} \quad
			\Omega \subseteq \mathbb{R}
		\]
	\end{definition}
\end{figure}
\begin{figure}[H]
	\begin{definition}
		L'\textbf{insieme degli eventi \( \mathcal{A} \) } è un insieme \textbf{finito} di
		parti di \( \Omega \) tali che sia un'algebra, cioè tale che:
		\begin{enumerate}
			\item[$A_1$.] \( \Omega \in \mathcal{A} \)
			\item[$A_2$.] Unione di eventi è un evento
			      \[
				      A,B \in \mathcal{A} \Rightarrow A \cup B \in \mathcal{A}
			      \]
			\item[$A_3$.] se \( A,B \in \mathcal{A} \), allora \( A \setminus B \in \mathcal{A}  \)
			      \[
				      \Downarrow
			      \]
			      \[
				      A \in \mathcal{A} \Rightarrow A^c = \Omega \setminus A \in \mathcal{A}
			      \]
		\end{enumerate}
		L'insieme degli eventi rappresenta tutti gli eventi che ci \textbf{interessati}
		rispetto all'esperimento preso in considerazione, e che \textbf{ben descrivono
			l'esperimento} stesso.
	\end{definition}
\end{figure}

\begin{figure}[H]
	\begin{definition}
		\( \sigma  \)-algebra \( \mathcal{F} \) è un insieme qualsiasi \( \mathcal{F} \) di
		parti di \( \Omega \) tali che:
		\begin{enumerate}
			\item[$A_1$.] \( \Omega \in \mathcal{F} \)
			\item[$A_2^\sigma$.] sia \( \{A_n\}_{n \in \mathbb{N}}\) con \(A_n \in \mathcal{F}\),
			      allora \(\bigcup_{n \in \mathbb{N}} A_n \in \mathcal{F}\)
			\item[$A_3^\sigma$.] se \( A,B \in \mathcal{F} \), allora \( A \setminus B \in \mathcal{F}\)
			      Diremo \textbf{evento} ogni sottoinsieme \( A \in \mathcal{F} \).
			      \[
				      \Downarrow
			      \]
			      \[
				      A \in \mathcal{F} \Rightarrow A^c = \Omega \setminus A \in \mathcal{F}
			      \]
		\end{enumerate}
	\end{definition}
\end{figure}

\subsubsection{Esempi}
\begin{figure}[H]
	\begin{example}
		Lancio il dado e controllo che numero esce
		\[
			\mathcal{A} = \mathcal{P}(\Omega) =
		\]
		\[
			= \Big\{ \{1\} ,\{2\} ,\{3\}, \{4\}, \{5\}, \{6\},
		\]
		\[
			\{1,2\}, \{1,3\}, \{1,4\}, \{1,5\}, \{1,6\},
		\]
		\[
			\{2,3\}, \{2,4\}, \{2,5\}, \{2,6\},
		\]
		\[
			\ldots
		\]
		\[
			\{1,2,3\}, \{1,2,4\}, \{1,2,5\}, \{1,2,6\},\ldots
		\]
		\[
			\{1,2,3,4\}, \{1,2,3,5\}, \{1,2,3,6\},\ldots
		\]
		\[
			\{1,2,3,4,5\}, \{1,2,3,4,6\},\ldots
		\]
		\[
			\{1,2,3,4,5,6\} = \Omega, \emptyset \Big\}
		\]
	\end{example}
\end{figure}

\subsection{Probabilità degli esperimenti 1-2}
\subsubsection{Esperimento 1: Lancio di un dado}
\[
	\Omega_1 = \{1, 2, 3, 4, 5, 6\}
\]
\[
	A = \; \text{esce almeno un 6}
\]
\[
	P(\{i\} ) = \frac{\text{casi favorevoli}}{\text{casi possibili}} = \frac{card(\{i\} )}
	{card(\Omega)} = \frac{1}{6}, \quad i = 1, \ldots, 6
\]

\subsubsection{Esperimento 2: Lancio di 2 dadi}
\[
	\Omega_2 = \{1, 2, 3, 4, 5, 6\} \times \{1, 2, 3, 4, 5, 6\} = \{(1,1), (1,2), \ldots, (6,6)\}
\]
\[
	A = \; \text{esce almeno un 6}
\]
\[
	P(A) = \frac{\text{casi favorevoli ad \( A \) }}{\text{casi possibili}} = \frac{card(A)}{card(\Omega)}
	= \frac{11}{36}
\]
\subsection{Definizione frequentista di probabilità}
\begin{figure}[H]
	\begin{define}
		L'ipotesi dei modelli deve essere ripetibile all'esperimento, quindi bisogna avere
		tante prove ripetute (nelle stesse condizioni) ed indipendenti
	\end{define}
\end{figure}

La probabilità di un evento \( A \in \mathcal{A} \), fatte \( n \) prove:
\[
	P(A) = \frac{\text{numero di occorrenze di \( A \) }}{n} = f_n(A)
\]
Si basa sulla \textbf{legge empirica del caso} che sintetizza una regolarità osservabile
sperimentalmente.

\subsection{Definizione soggettiva di probabilità}
È la misura del grado di fiducia che un individuo \textbf{coerente} assegna al verificarsi
di un dato evento in base alle sue \textbf{conoscenze}

Probabilità di un evento \( A \in \mathcal{A} \):
\[
	P(A) = \frac{\text{posta}}{\text{{vincita}}} = \frac{P}{V}
\]
In breve, "se ci credo, pago"

\section{Assiomi di Kolmogorov}
L’impostazione assiomatica permette a Kolmogorov di non esplicitare esattamente come
valutare la probabilità (lasciando quindi la libertà di seguire l’approccio più adatto al caso in
esame), ma di limitarsi solo a indicare quali sono le regole formali che una misura di
probabilità deve soddisfare per poter essere dichiarata tale.

\subsection{Assiomi}
\subsubsection{Caso finito}
\begin{figure}[H]
	\begin{definition}
		\[
			(\Omega, \mathcal{A}, P)
		\]
		\begin{enumerate}
			\item[$P_1.$] \( P(\Omega) = 1 \)
			\item[$P_2.$] sia \( A,B \in \mathcal{A} \) disgiunti, t.c
			      \[
				      A \cap B = \emptyset
			      \]
			      allora
			      \[
				      P(A \cup B) = P(A) + P(B)
			      \]
			      (additività finita)
		\end{enumerate}
	\end{definition}
\end{figure}

\subsubsection{Caso generale}
\begin{figure}[H]
	\begin{definition}
		\[
			(\Omega, \mathcal{A}, P)
		\]
		\begin{enumerate}
			\item[$P_1.$] \( P(\Omega) = 1 \)
			\item[$P_2^\sigma.$] sia \( \{A_n\}_n, A_n \in \mathcal{F}  \) disgiunti t.c.
			      \[
				      A_i \cap A_j = \emptyset, \quad i \neq j
			      \]
			      allora
			      \[
				      P\left( \bigcup_{i=1}^{\infty} A_i \right) = \sum_{i=1}^{\infty} P(A_i)
			      \]
			      (\( \sigma  \)-additività)
		\end{enumerate}
	\end{definition}
\end{figure}

\section{Modelli probabilistici}
\subsection{Regola per l'unione e l'intersezione di eventi}
Sia \( (\Omega, \mathcal{A}, P) \) uno spazio di probabilità. Siano \( A,B \in \mathcal{A} \)
due eventi. Allora:
\begin{itemize}
  \item \textbf{L'unione di \( A \) e \( B \)}:
    \[
    P(A \cup B) = \begin{cases}
      P(A) + P(B) & \text{se \( A \) e \( B \) sono disgiunti} \\
      P(A) + P(B) - P(A \cap B) & \text{caso generale}
    \end{cases}
    \] 
  \item \textbf{L'intersezione di \( A \) e \( B \)}:
    \[
    P(A \cap B) = \begin{cases}
      P(A)P(B) & \text{se \( A \) e \( B \) sono indipendenti} \\
      P(A|B)P(B) = P(A)P(B|A) & \text{caso generale}
    \end{cases}
    \] 
\end{itemize}

\subsection{Modello equiprobabile}
L'equiprobabilità è un modello in cui tutti gli eventi elementari hanno la \textbf{stessa
	probabilità} di verificarsi. Ciò implica che lo spazio dei campioni deve essere \textbf{finito}.

\begin{figure}[H]
	\begin{definition}[Proprietà della probabilità]
		\begin{enumerate}
			\item[$P_1$.] \( P(\Omega) = 1 \)
			\item[$P_2$.] Se \( A,B \in \mathcal{P}(\Omega) \) tale che:
			      \[
				      A \cap B = \emptyset,\; \text{allora}
			      \]
			      \[
				      P(A \cup B) = P(A) + P(B)
			      \]
		\end{enumerate}
	\end{definition}
\end{figure}
\[
	\Downarrow
\]
\begin{figure}[H]
	\begin{definition}[Probabilità uniforme]
		\[
			1 = P(\Omega) = P(\{1,2,3,4,5,6\} ) = P(\cup_i \{\omega_i\} ) =
		\]
		\[
			= \sum_i P(\{\omega_i\} ) = card(\Omega) \cdot P(\{\omega_i\} )
		\]
		da cui:
		\[
			P(\{\omega_i\} ) = \frac{1}{card(\Omega)}, \; \text{per ogni} \; \omega_i \in \Omega
		\]
	\end{definition}
\end{figure}

\begin{figure}[H]
	\begin{definition}[Modello equiprobabile o uniforme]
		È lo spazio di probabilità \( (\Omega, \mathcal{A}, P) \) tale che:
		\begin{enumerate}
			\item[$M_1$.] \( \Omega \) è finito, cioè la cardinalità di \( \Omega \)
			      ( \( card(\Omega) \in \mathbb{N} \) ) è tale che:
			      \[
				      card(\Omega) < \infty
			      \]
			\item[$M_2$.] \( \mathcal{A} = \mathcal{P}(\Omega) \) è l'insieme delle parti
			      di \( \Omega \)
			\item[$M_3$.] per ogni \( \omega \in \Omega \)
			      \[
				      P(\{\omega\} ) = \text{costante}
			      \]
		\end{enumerate}
	\end{definition}
\end{figure}

\subsection{Spazio di probabilità}
\subsubsection{Caso finito}
\begin{figure}[H]
	\begin{definition}[Spazio di probabilità]
		Dato lo spazio dei campioni \( \Omega = \{\omega_1, \ldots, \omega_n\}  \) e
		l'insieme degli eventi \( \mathcal{A} \), la probabilità \( P \) è definita come:
		\begin{enumerate}
			\item[$P_1$.] \( P(\Omega) = 1 \)
			\item[$P_2$.] Se \( A,B \in \mathcal{A} \), disgiunti, tale che:
			      \[
				      A \cap B = \emptyset,\; \text{allora}
			      \]
			      \[
				      P(A \cup B) = P(A) + P(B)
			      \]
			      (additività finita)
		\end{enumerate}
	\end{definition}
\end{figure}

\subsubsection{Caso generale}
\begin{figure}[H]
	\begin{definition}[Spazio di probabilità]
		Dato lo spazio dei campioni \( \Omega = \{\omega_1, \ldots, \omega_n\}  \),
		oppure \( \Omega \subseteq \mathbb{R} \) e
		la \( \sigma \)-algebra \( \mathcal{F} \), la probabilità \( P \) è definita come:
		\begin{enumerate}
			\item[$P_1$.] \( P(\Omega) = 1 \)
			\item[$P_2^\sigma$.] sia \( \{A_n\}_n, A_n \in \mathcal{F}  \), disgiunti, tale che:
			      \[
				      A_i \cap A_j = \emptyset, \quad \text{per}\; i \neq j;\; \text{allora}
			      \]
			      \[
				      P\left( \bigcup_{i=1}^{\infty} A_i \right) = \sum_{i=1}^{\infty} P(A_i)
			      \]
			      (\( \sigma  \)-additività)
		\end{enumerate}
	\end{definition}
\end{figure}

\subsection{Indipendenza di eventi}
Sia \( (\Omega, \mathcal{A}, P) \) uno spazio di probabilità. Siano \( A,B \in \mathcal{A} \)
due eventi. Allora \( A \) e \( B \) si dicono \textbf{indipendenti} se:
\[
	P(A \cap B) = P(A) \cdot P(B)
\]
In breve due eventi si dicono indipendenti se l'occorrenza di uno dei due non influenza
la probabilità di occorrenza dell'altro.

\subsubsection{Proposizione}
Siano \( A,B \in \mathcal{A} \) due eventi indipendenti. Allora:
\[
  P(A|B) = \frac{P(A \cap B)}{P(B)} = \frac{P(A) \cdot P(B)}{P(B)} = P(A)
\] 
L'intersezione:
\[
   P(A \cap B) = \begin{cases}
    P(A)P(B) & \text{se \( A \) e \( B \) sono indipendenti} \\
    P(A|B)P(B) = P(A)P(B|A) & \text{caso generale} 
   \end{cases}
\] 

\subsection{Probabilità condizionata}
Sia \( (\Omega, \mathcal{A}, P) \) uno spazio di probabilità \textbf{uniforme}. Siano \( A,B \in \mathcal{A} \)
due eventi. Prendiamo per ipotesi che \( P(B) > 0 \), quindi \( B \) è accaduto, cioè
\( B \) diventa un evento certo \( \Rightarrow prob(B) = 1 \). Allora la relazione tra
la probabilità \( P \) e la probabilità \( P_B \) sullo spazio ristretto a \( B \) è:
\[
	P_B(A) = \frac{card_E(A)}{card(\Omega_B)} = \frac{card(A \cap B)}{card(B)} =
	\frac{card(A \cap B)}{card(B)} \cdot \frac{card(\Omega)}{card(\Omega)} =
	\frac{P(A \cap B)}{P(B)}
\]
Quindi:
\begin{figure}[H]
	\begin{definition}
		La \textbf{Probabilità condizionata a \( B \)} per ogni \( A \in \Omega \) è:
		\[
			P(A|B) = \frac{P(A \cap B)}{P(B)}
		\]
	\end{definition}
\end{figure}

\subsubsection{Esempio}
\begin{example}
	Consideriamo l'esperimento che consiste nel lancio di un dado equo. Consideriamo i due
	eventi:
	\[
		A = \text{"esce un numero"}\;>3 = \{4,5,6\}
	\]
	\[
		B = \text{"esce un numero pari"} = \{2,4,6\}
	\]
	Si lanci il dado una volta:
	\begin{enumerate}
		\item[B1.] Calcolare la probabilità dell'evento \( A \)
		\item[B2.] Lanciato il dado una persona guarda il risultato e afferma che è uscito un
		      numero pari, cioè è accaduto l'evento \( B \). Sapendo questa informazione, come
		      diventa la probabilità dell'evento \( A \)?
	\end{enumerate}
	Il dado è equo, quindi si utilizza il modello equiprobabile.

	\vspace{1em}
	\noindent \textbf{B1. Calcolare \( P(A) \) }:\\
	Lo spazio dei campioni è:
	\[
		\Omega = \{1,2,3,4,5,6\}
	\]
	La probabilità di \( A \) è:
	\[
		P(\{\omega\} ) = \frac{1}{card(\Omega)} = \frac{1}{6}, \quad \omega \in \Omega
	\]
	\[
		P(A) = \frac{card(A)}{card(\Omega)} = \frac{3}{6} = \frac{1}{2}
	\]

	\vspace{1em}
	\noindent \textbf{B2. Calcolare \( P(A|B) \) }: probabilità dell'evento \( A \) sapendo che sia
	occorso \( B \) \\
	Lo spazio dei campioni è:
	\[
		\Omega_B = \{2,4,6\}
	\]
	La probabilità di \( A \) sapendo che sia occorso \( B \) è:
	\[
		P_B(\{\omega\} ) = \frac{1}{card(\Omega_B)} = \frac{1}{3}, \quad \omega \in \Omega_B
	\]
	\[
		P_B(A) = \frac{card(A)}{card(\Omega_B)} = \frac{2}{3}
	\]
	Condizionare ad un evento significa costruire un nuovo spazio di probabilità ristretto
	all'evento condizionante:
	\[
		\text{Spazio dei campioni:}\; \Omega \to \Omega_B = B
	\]
	\[
		\text{Misura di probabilità:}\; P \to P_B:P_B(\Omega_B) = \frac{card(\Omega_B)}{card(\Omega_B)}
		= 1 = P(\Omega)
	\]
	La relazione tra \( P \) e \( P_B \):
	\[
		P_B(A) = \frac{card_B(A)}{card(\Omega_B)} = \frac{card(A \cap B)}{card(B)}
		= \frac{card(A \cap B)}{card(\Omega)}\frac{card(\Omega)}{card(B)} =\frac{P(A \cap B)}{P(B)}
	\]
\end{example}

\subsubsection{Proposizione}
\( P_B(\cdot) = P(\cdot|B) \) è una misura di \textbf{probabilità} su \( (\Omega, \mathcal{A}) \)
tale che:
\[
	P_B(A) = P(A|B) = \frac{P(A \cap B)}{P(B)} \quad A \cap B \neq \emptyset
\]
\[
	P_B(A) = P(A|B) = \frac{P(A \cap B)}{P(B)} = 0 \quad A \cap B = \emptyset
\]
\[
	P_B(\Omega) = P(\Omega|B) = \frac{P(\Omega \cap B)}{P(B)} = 1
\]
\[
	P_B(B) = P(B|B) = \frac{P(B \cap B)}{P(B)} = 1
\]
L'ultima proprietà è coerente con l'ipotesi che \( B \) è occorso e quindi è un evento
certo.

\subsubsection{Come determinare la probabilità condizionata}
Sia \( (\Omega, \mathcal{A}, P) \) uno spazio di probabilità \textbf{uniforme} e sia
\( B \subset \Omega \) tale che \( P(B) > 0 \). Condizionare ad un evento \( B \)
significa costruire un nuovo spazio di probabilità ristretto all'evento condizionante \( B \):
\( (\Omega_B, \mathcal{F}_B, P_B) \)
\[
	\text{Spazio dei campioni:}\; \Omega \to \Omega_B = B
\]
\[
	\text{Misura di probabilità:}\; P \to P_B:P_B(B) = 1
\]
Si utilizza lo spazio di partenza \( (\Omega, \mathcal{A}, P) \) su cui definisco la misura
condizionata, per ogni \( A \subseteq \Omega \):
\[
	P_B(A) = P(A|B) = \frac{P(A \cap B)}{P(B)}
\]

\subsection{Probabilità a Priori e a Posteriori (Formula di Bayes)}
Sia \( (\Omega, \mathcal{A}, P) \) uno spazio di probabilità. Si dice \textbf{partizione
di \( \Omega \)} un insieme di eventi \( \{E_j\}_{1 \le j \le n} \) tali che:
\begin{enumerate}
  \item sono disgiunti:
    \[
      i,j \in I \quad \text{tale che} \quad i \neq j,\; E_i \cap E_j = \emptyset
    \] 
  \item per ogni \( i \in I \):
    \[
    P(E_i) > 0
    \] 
  \item ricoprono tutto lo spazio:
    \[
    \bigcup_{j = 1}^n E_j = \Omega
    \]
\end{enumerate}

\subsubsection{Teorema delle probabilità totali}
Sia \( (\Omega, \mathcal{A}, P) \) uno spazio di probabilità e sia \( \{E_j\}_{1 \le j \le n} \)
una partizione di \( \Omega \). Sia \( B \in \mathcal{A} \) un evento. Allora:
\[
  P(B) = \sum_{j = 1}^n P(B|E_j)P(E_j)
\] 
\subsubsection{Dimostrazione}
\[
  B = B \cap \Omega = B \cap \bigcup_{j=1}^n E_j = \bigcup_{j=1}^n (B \cap E_j)
\] 
\[
  P(B) = P\left( \bigcup_{j=1}^n (B \cap E_j) \right) \sum_{j=1}^n P(B \cap E_j) =
  \sum_{j=1}^n P(B|E_j)P(E_j)
\] 

\subsubsection{Teorema di Bayes}
\begin{figure}[H]
  \begin{definition}
    Sia \( \{E_i\}_{i \in I}  \) una partizione di \( \Omega \) (finita o numerabile) e sia
    \( P(A) > 0 \). Allora per un generico elemento della partizione \( E_n \), con
    \( n \in I \), si ha:
    \[
      P(E_n|A) = \frac{P(A|E_n)P(E_n)}{\sum_{i \in I} P(A|E_i)P(E_i)}
    \] 
  \end{definition}
\end{figure}
\subsubsection{Dimostrazione}
\[
  P(E_n|A) = \frac{P(A \cap E_n)}{P(A)} = \frac{P(A|E_n)P(E_n)}{P(A)} = \frac{P(A|E_n)P(E_n)}{
  \sum_{i \in I} P(A|E_i)P(E_i)}
\] 

\subsection{Probabilità a priori e a posteriori}
\subsubsection{Probabilità a priori}
\( P(A) \): è una probabilità da determinare senza altre informazioni. Non si hanno
informaizoni sufficienti per determinarla.

\subsubsection{Probabilità a posteriori}
\( P(A|E_n) \): si hanno strumenti ed informazioni sufficienti per determinare queste
probabilità. Se si conoscono delle altre informazioni, quindi se si restringe lo spazio
campionario, si riescono a determinare.

\subsection{Esercizi}
\subsubsection{Esercizio 1}
Una popolazione si compone per un \( 40\% \)  di fumatori \((F)\) e per il restante 
\( 60\% \) di non fumatori \( (N) \). Si sa che il \( 25\% \)  dei fumatori e il 
\( 7\% \)  dei non fumatori ha una malattia respiratoria cronica \( (M) \) .
\begin{enumerate}
  \item Calcolare la probabilità che un individuo scelto a caso sia effetto dalla malattia
    respiratoria 
  \item Se l’individuo scelto è affetto dalla malattia, calcolare la probabilità che
    sia un fumatore. 
\end{enumerate}

\[
\Omega = \{F,F^c\} \quad \text{partizione}
\] 
\[
M = \text{"Malattia respiratoria"}
\] 
\[
P(M|F) = 0.25 \quad P(M|F^c) = 0.07
\] 
\[
P(M) = P(M|F)P(F) + P(M|F^c)P(F^c) = 0.25 \cdot 0.4 + 0.07 \cdot 0.6 = 0.14
\] 
\[
  P(F|M) \stackrel{\text{Bayes}}{=} \frac{P(M|F)P(F)}{P(M)} = \frac{0.25 \cdot 0.4}{0.14} = 0.70
\] 

\section{Variabili aleatorie discrete}
Sia \( (\Omega, \mathcal{F}, P) \) uno spazio di probabilità che modellizza un certo
esperimento. Una \textbf{variabile aleatoria} è una funzione che associa un numero ad
ogni esito possibile dell'esperimento. Formalmente:
\[
X: \Omega \to \mathbb{R}
\] 
Sia \( \mathcal{B} \) l'insieme degli eventi su \( \mathbb{R} \) (intervalli, unione ed 
intersezione di intervalli), per ogni \( A \subset \mathbb{R} \):
\[
  X^{-1}(A) \in \mathcal{F}
\] 
è l'inversa di \( X \) su \( A \).

\vspace{1em}
\noindent Le variabili aleatorie possono essere:
\begin{itemize}
  \item \textbf{Discrete}: se assumono un numero finito o numerabile di valori. Ad esempio:
    \[
    X \leadsto \{0,1,2,3,4,5,6\}
    \] 
  \item \textbf{Continue}: se assumono valori in un intervallo reale. Ad esempio:
    \[
    X \leadsto \mathbb{R}
    \]
\end{itemize}

\subsection{Distribuzione di una variabile aleatoria}
Sia \( (\Omega, \mathcal{F}, P) \) uno spazio di probabilità che modella un certo
esperimento e sia \( X: \Omega \to \mathbb{R} \) una variabile aleatoria.
Sia \( \mathcal{B} \) l'insieme degli eventi su \( \mathbb{R} \) (intervalli, unione ed
intersezione di intervalli). 

\vspace{1em}
\noindent La probabilità di un evento nella funzione \( X \) prendendo in considerazione
il lancio di un dado si indica come:
\[
P(1) = P(X = 1) = P(\{\omega \in \Omega: X(\omega) = 1\} ) = \frac{1}{6}
\] 
L'insieme \( \{P(1), P(2), \ldots, P(6)\} \) rappresenta le densità di probabilità,
chiamato anche \textbf{distribuzione di \( X \) }

\subsection{Variabili aleatorie discrete}
Sia \( (\Omega, \mathcal{B}, P) \) uno spazio di probabillità, si definisce \textbf{
  variabile aleatoria discreta
}:
\[
X : \Omega \to R_X = \{x_1, x_2, \ldots, x_n\} 
\] 
con \( card(R_X) \) al più numerabile.

\begin{figure}[H]
  \begin{definition}
    Distribuzione o legge di \( X \) 
    \[
      P_X(\{x_i\} ) = P(X^{-1}(\{x_i\} )) = P(X=x_i) \quad \text{per ogni} \; x_i,\; i = 1, \ldots, n
    \] 
    La legge di una variabile aleatoria discreta è dunque caratterizzata da una
    \textbf{funzione di probabilità}
  \end{definition}
\end{figure}

\begin{example}
  Prendiamo in considerazione il lancio di 2 dadi: lancio i 2 dadi e ne sommo i valori.
  \[
  \Omega = \{1,2,3,4,5,6\} \times \{1,2,3,4,5,6\}
  \] 
  \[
    \mathcal{F} = \mathcal{P}(\Omega)
  \] 
  \[
  P(\omega) = \frac{1}{36}
  \] 
  \[
  X : \Omega \to \{2,3,4,5,6,7,8,9,10,11,12\} \subset \mathbb{N}
  \] 
  \[
    (x,y) \to x+y
  \] 
  La legge di \( X \) è:
  \begin{table}[H]
    \centering
    \begin{tabular}{|c|l|c|}
      \hline
      \( x_i \) & \( X^{-1}(x_i) \) & \( P_X(x_i) \) \\
      \hline
      2 & \(\{(1,1)\} \) & \( 1/36 \) \\
      3 & \(\{(1,2), (2,1)\} \) & \( 2/36 \) \\
      4 & \(\{(1,3), (2,2), (3,1)\} \) & \( 3/36 \) \\
      5 & \(\{(1,4), (2,3), (3,2), (4,1)\} \) & \( 4/36 \) \\
      6 & \(\{(1,5), (2,4), (3,3), (4,2), (5,1)\} \) & \( 5/36 \) \\
      7 & \(\{(1,6), (2,5), (3,4), (4,3), (5,2), (6,1)\} \) & \( 6/36 \) \\
      8 & \(\{(2,6), (3,5), (4,4), (5,3), (6,2)\} \) & \( 5/36 \) \\
      9 & \(\{(3,6), (4,5), (5,4), (6,3)\} \) & \( 4/36 \) \\
      10 & \(\{(4,6), (5,5), (6,4)\} \) & \( 3/36 \) \\
      11 & \(\{(5,6), (6,5)\} \) & \( 2/36 \) \\
      12 & \(\{(6,6)\} \) & \( 1/36 \) \\
      \hline
    \end{tabular}
  \end{table}

  \noindent La rappresentazione grafica della distribuzione di \( X \) è:
  \begin{figure}[H]
    \centering
    \begin{tikzpicture}
      \begin{axis}[
        ybar,
        enlargelimits=0.15,
        legend style={at={(0.5,-0.15)},
        anchor=north,legend columns=-1},
        ylabel={\( P_X(x_i) \)},
        xlabel={\( x_i \)},
        symbolic x coords={2,3,4,5,6,7,8,9,10,11,12},
        xtick=data,
        ]
        \addplot coordinates {(2,0.0278) (3,0.0556) (4,0.0833) (5,0.1111) (6,0.1389) (7,0.1667) (8,0.1389) (9,0.1111) (10,0.0833) (11,0.0556) (12,0.0278)};
      \end{axis}
    \end{tikzpicture}
    \caption{Distribuzione di \( X \)}
  \end{figure}
  cioè una distrubuzione triangolare.
\end{example}

\begin{figure}[H]
  \begin{definition}
    Il \textbf{Valore atteso} \( \mathbb{E}[X] \) è definito come:
    \[
      \mathbb{E}[X] = \sum_{i=1}^n x_i P_X(x_i)
    \] 
    Si tratta di una media pesata dei valori assunti dalla variabile aleatoria con i
    pesi dati dalle probabilità che la variabile assuma quei valori.
  \end{definition}
\end{figure}

\begin{figure}[H]
  \begin{definition}
    La \textbf{Varianza} di \( X \) è definita come:
    \[
      Var(X) = \mathbb{E}[(X - \mathbb{E}[X])^2] = \sum_{i=1}^n (x_i - \mathbb{E}[X])^2 P_X(x_i)
    \] 
  \end{definition}
\end{figure}

\begin{example}
  Prendiamo in considerazione il lancio di una moneta:
  \[
    \Omega = \{T,C\} \quad \mathcal{F} = \mathcal{P}(\Omega)
  \] 
  Si vuole calcolare swe esce testa, quindi \( X \) è la variabile aleatoria che
  rappresenta il lancio della moneta:
  \[
  X : \Omega \to \{0,1\} \quad t.c.\;\; X(T) = 1, \quad X(C) = 0
  \] 
  Questa variabile viene chiamata \textbf{variabile aleatoria di Bernoulli} ed è una
  variabile \textbf{dicotomica} (solo 2 valori).
  \begin{itemize}
    \item \textbf{Moneta equa}:
      \[
      P_1(\{T\} ) = P_1(\{C\} ) = \frac{1}{2}
      \] 
      \[
      \Downarrow
      \] 
      \[
      P_X(1) = P_1(X=1) = P_1(\{T\} ) = \frac{1}{2}
      \] 
      \[
      P_X(0) = \frac{1}{2}
      \] 
    \item \textbf{Moneta non equa}:
      \[
      P_2(T) = \frac{1}{3} \quad P_2(C) = \frac{2}{3}
      \] 
      \[
      P_X(1) = P_2(X=1) = P_2(\{T\} ) = \frac{1}{3}
      \] 
      \[
      P_X(0) = 1 - P_X(1) = \frac{2}{3}
      \] 
  \end{itemize}
\end{example}

\section{Schema di Bernoulli}
\subsection{Prove dicotomiche ripetute ed indipendenti}
Sia \( (\Omega, \mathcal{L}, P) \) uno spazio di probabilità che modella un certo
esperimento. Considero \( A \in \mathcal{L} \) un successo e supponendo che
\[
P(A) = p \in (0,1)
\] 
ripeto un esperimento in modo indipendente

\subsection{Conteggio del successo in ciascuna prova}
\begin{figure}[H]
  \begin{definition}[MOLTO IMPORTANTE]
    \[
      X \sim \mathcal{B}(p) \quad \text{X è una variabile di Bernoulli}
    \] 
    \[
    p \equiv \text{probabilità di successo}
    \] 

    \[ X: \Omega \to \{0,1\} \subset \mathbb{N}  \]
    \[
    \text{successo} \; \to 1
    \] 
    \[
    \text{insuccesso} \; \to 0
    \] 
    La legge di \( X \) è:
    \[
    p_x(1) = P(X=1) = P(\text{"successo"}) = p
    \] 
    \[
    p_x(0) = P(X=0) = P(\text{"insuccesso"}) = 1-p_x(1) = 1-p
    \] 

    \vspace{1em}
    \noindent Il valore atteso è:
    \[
      \mathbb{E}[X] = p
    \] 
    cioè il successo.

    \vspace{1em}
    \noindent La varianza si calcola come:
    \[
      Var(X) = p(1-p)
    \] 
  \end{definition}
\end{figure}

\subsection{Conteggio del successo in cuascuna delle n volte }
Si ripete l'esperimento dicotomico \( n \) volte in condizioni di indipendenza (lancio
la moneta \( n \)  volte). Chiamo \( X_1, X_2, \ldots, X_n \) i risultati degli \( n \)
lanci.
\[
X_i \sim \mathcal{B}(p) \quad \text{per ogni} \; i = 1, \ldots, n
\] 
Si è interessati all'evento:
\[
A = \text{"numero di successi ottenuti negli \( n \) lanci"} = \text{"numero di teste"}
\] 
La variabile che conta il numero di successi è:
\[
X = X_1 + X_2 + \ldots + X_n \equiv \text{"conta il numero di successi"}
\] 
\[
  X \sim \mathcal{B}(n,p) \quad \text{\textbf{variabile binomiale}}
\] 
La variabile binomiale conta il numero di successi con:
\[
n \equiv \text{numero di prove}
\] 
\[
p \equiv \text{probabilità di successo in ogni prova}
\] 
I valori che potrà assumere la variabile binomiale sono:
\[
X \leadsto \{0,1,2, \ldots, n\}
\] 

\subsubsection{Modello per il conteggio del successo}
\textbf{Esito prova i-esima}:
\[
X_1 \sim \mathcal{B}(p) \quad P_X(x_i) = P^x_i(1-p)^{1-x_i}
\] 
perchè:
\[
P_x(0) = P^0(1-p)^1 = (1-p) \quad \text{"insuccesso"}
\] 
\[
P_x(1) = P^1(1-p)^0 = p \quad \text{"successo"}
\] 

\vspace{1em}
\noindent \textbf{Esito delle \( n \) prove}: Vettore aleatorio \( X_1, X_2, \ldots, X_n \) che
assume valori:
\[
  (x_1, x_2, \ldots, x_n) \quad x_j \in \{0,1\}
\] 

\vspace{1em}
\noindent \textbf{Prove indipendenti} (Distribuzione della variabile binomiale):
\[
P(X_1 = x_1, X_2 = x_2, \ldots, X_n = x_n) = P(X_1 = x_1)P(X_2 = x_2) \ldots P(X_n = x_n)
\] 
\[
= P^{x_1}(1-p)^{1-x_1}P^{x_2}(1-p)^{1-x_2} \ldots P^{x_n}(1-p)^{1-x_n}
\] 
\[
= p^{\sum_{i=1}^n x_i}(1-p)^{n - \sum_{i=1}^n x_i}
\] 
dove \( \sum_{i=1}^n x_i = k \) è il numero di successi ottenuti nei \( n \) lanci.

\[
P_X(X=n)\begin{pmatrix} n \\ k \end{pmatrix} = \frac{n!}{k!(n-k)!} \quad \text{\textbf{coefficiente
binomiale}}
\] 
\[ \begin{pmatrix} n \\ k \end{pmatrix} \cdot p^k(1-p)^{n-k} \] è la probabilità di ottenere
\( k \) successi in \( n \) prove.

\subsection{Processi stocastici (prove indipendenti ripetute infinite volte)}
Il \textbf{processo} consiste nel ripetere infinite volte un esperimento dicotomico,
infatti si dice \textbf{processo di Bernoulli} la variabile \( \{X_n\}_{n \in \mathbb{N}}
\) che fa riferimento ad una famiglia di variabili aleatorie indipendenti e identicamente
distribuite:
\[
X_i \sim B(p)
\] 
Il \textbf{processo binomiale} \( \{Z_n\}_{n \in \mathbb{N}}  \) fa riferimento ad una
famiglia di variabili aleatorie ad incrementi indipendenti che modella il numero di 
successi in \( n \) prove all'aumentare del numero:
\[
Z_0 = 0 \quad Z_n = \sum_{i=1}^n X_i \sim B(n,p)
\] 

\subsection{La variabile geometrica}
\begin{figure}[H]
  \begin{definition}
    È il numero della prova in cui si ha il \textbf{primo successo}
    \[
      T \sim G(p)
    \] 
    Il range, sapendo che a priori non si sa quando accade il primo successo, è:
    \[
      \mathcal{R}_T = 1,2,3, \ldots
    \] 
    Legge:
    \[
      p_T(k) = p(1-p)^{k-1} \quad \text{per ogni} \; k = 1,2,3, \ldots
    \] 
    Eventi equivalenti:
    \[
      \{T = k\} = \{Z_{k-1}\} \cap \{X_k = 1\}
    \] 
    Se il primo successo si ha nella prova \( k \)-esima, allora fino alla prova
    \( k - 1 \)  non si hanno successi \( \{Z_{k-1} = 0\} \)  e nella prova 
    \(  k \)-esima si ha un successo \( \{X_k = 1\} \).


    \vspace{1em}
    \noindent Il valore atteso è:
    \[
      \mathbb{E}[T] = \frac{1}{p}
    \] 
    La varianza si calcola come:
    \[
    var(T) = \frac{1-p}{p^2}
    \] 
  \end{definition}
\end{figure}
Più la variabile è vicina a 1, più è probabile che il successo avvenga al primo tentativo.

\subsection{Eventi rari: conteggio dei successi per un numero infinito di esperimenti}
Supponiamo che il numero dei successi \( n \) cresca all'infinito e che la probabilità
\( p_n \) di successa vada a zero, ma in modo tale che:
\[
np_n \to \lambda
\] 
Sia dunque \( Z_n \sim B(n,p_n) \) 

\begin{figure}[H]
  \begin{definition}[Poisson]
    Conta il numero di eventi che avvengono in un intervallo finito di tempo \( [0,t] \) 
    \textbf{Poisson:}
    \[
    Y \sim P(\lambda)
    \] 
    \textbf{Range:}
    \[
      \mathcal{R} = \mathbb{N}
    \] 
    \textbf{Legge:}
    \[
      p_y(k) = \frac{\lambda^k}{k!}e^{-\lambda}
    \] 
  \end{definition}
\end{figure}

\begin{figure}[H]
  \begin{definition}
    \textbf{Binomiale converge alla Poisson}
    \[
    \lim_{n \to \infty, p_n \to 0} B(n,p_n) = P(\lambda)
    \] 
  \end{definition}
\end{figure}

\section{La variabile di Poisson}
La variabile di Poisson di parametro \( \lambda > 0 \) modella il conteggio di successo
in tempi continui.
\begin{figure}[H]
  \begin{definition}
    \textbf{Variabile aleatoria di Poisson:}
    \[
    Y \sim P(\lambda)
    \] 
    \textbf{Range:}
    \[
      \mathcal{R} = \mathbb{N}
    \] 
    \textbf{Legge:} per ogni \( k = 0, 1, \ldots \) 
    \[
      p_y(k) = \frac{\lambda^k}{k!}e^{-\lambda}
    \] 
    Il valore atteso è:
    \[
      \mathbb{E}[Y] = \lambda
    \] 
    La varianza è:
    \[
    var(Y) = \lambda
    \] 
  \end{definition}
\end{figure}
\noindent Poisson approssima la binomiale quando \( n \) è grande e \( p \) è piccolo.

\subsection{Eventi rari: il processo di Poisson}
Il processo di Poisson modellizza il conteggio dei successi in intervalli di tempi
sotto l'ipotesi che la probabilità di un successo in uno specifico intervallo di tempo
piccolo è molto piccola:
\[
np_n = \lambda t
\] 
Il \textbf{processo di Poisson} \( \{Y_t\}_{t \in \mathbb{R}_+}  \) fa riferimento ad
una famiglia di variabili aleatorie ad incrementi indipendenti che modella il
numero di successi in un intervallo \( [0,t] \) tale che \( Y_0 = 0 \) e
\[
Y_t \sim P(\lambda t), \; \forall t \ge 0
\] 
\( \lambda \) rappresenta un numero medio di successi atteso per unità di tempo:
\[
  \mathbb{E}[Y_t] = \lambda t \quad \Rightarrow \quad \frac{\mathbb{E}[Y_t]}{t} = \lambda
\] 

\subsubsection{Conteggio del tempo del primo successo}
Osservando un esperimento modellato da un processo di Poisson \( \{Y_t\}_t  \) di 
parametro \( \lambda \). Si vuole modellare il tempo in cui si ha il primo successo:
\[
  \text{Tempo del primo successo:}\; T \;\text{con}\; \mathcal{R}_T = \mathbb{R}_+
\] 
Legge di \( T \):
\[
  F_T(t)= 1-e^{-\lambda t}, \quad f_T(t) = \lambda e^{-\lambda t}, \quad t \in \mathbb{R}_+
\] 
Esponenziale di parametro \( \lambda \quad \) \( T \sim exp(\lambda) \) 

\section{Variabili aleatorie continue}
Sia \( (\Omega, \mathcal{F}, P) \) uno spazio di probabilità, la variabile aleatoria
continua viene indicata come:
\[
X: \Omega \to R_X \subset \mathbb{R}
\] 
La legge di \( X \) (o distribuzione):
\[
  P_X(A) = P(X^{-1}(A)) = P(X \in A) \quad \text{per ogni} \; A \subset \mathcal{B}
\] 
dove \( \mathcal{B} \) è l'insieme degli eventi su \( \mathbb{R} \) (intervalli, unione
ed intersezione di intervalli).

\vspace{1em}
\noindent La legge di una variabile aleatoria continua è caratterizzata da una
\textbf{densità di probabilità}:
\[
f_X: \mathbb{R} \to \mathbb{R}
\] 
\[
P_X(A) = \int_A f_X(x)dx
\] 

\textbf{Proprietà di \( f_X \) }:
\[
0 \le f_X(x) \le 1 \quad x \in \mathbb{R}
\] 
\[
\int_{-\infty}^{+\infty} f_X(x)dx = 1
\] 

\section{Statistica inferenziale}
La probabilità e la statistica sono due approcci complementari allo stesso problema che
descrive fenomeni non deterministici. La statistica inferenziale introduce un modello
probabilistico per descrivere la popolazione tramite una variabile aleatoria:
\[
X \sim \mathcal{N}(\mu, \sigma^2) \quad \mu, \sigma^2 \; \text{incogniti}
\] 
Gli individui sono le "realizzazioni" della \( X \).

\begin{example}
  \( X =\) "altezze degli italiani maschi di 40 anni"
  \[
  X \sim \mathcal{N}(\mu, \sigma^2)
  \] 
  Prendo \( X_1 \sim X \leadsto\) Mario, \( X_1(10) = x_1 \) (altezza di mario)
  \[
    \text{Campione: } \underline{X} = (X_1, X_2, \ldots, X_n) \quad X_i \sim X \sim \text{ Popolazione}
  \] 
  Le \( X_i \) sono indipendenti
  \[
  \text{Dati: } X_1(\omega) = x_1, X_2(\omega) = x_2, \ldots, X_n(\omega) = x_n
  \] 
  \[
    \text{Dati: } \underline{x} = (x_1, x_2, \ldots, x_n)
  \] 
  una possibile realizzazione del campione aleatorio \( (X_1, X_2, \ldots, X_n) \)
\end{example}

Si parte da una teoria, cioè un modello con \textbf{parametri liberi} e si aggiungono dei
dati sperimentali. Ci si chiede se si può fare delle inferenze su parametri.
\begin{itemize}
  \item \textbf{Design}: stabilire gli esperimenti
  \item \textbf{Analisi}: analizzare i dati
\end{itemize}

\subsection{Distribuzione esponenziale}
È una variabile aleatoria continua:
\[
f_X(x) = \begin{cases}
  \lambda e^{-\lambda x} & x \ge 0 \\
  0 & x < 0
\end{cases}
\] 
\[
  P(X \in [a,b]) = \int_a^b f_X(x) dx
\] 
\[
f_X(x) = P(x \le x) = \begin{cases}
  1-e^{-\lambda x} & x \ge 0 \\
  0 & x < 0
\end{cases}
\] 
\[
  P(X \le x) = \int_{0}^x f_X(t) dt
\] 
\[
  \mathbb{E}[X] = \frac{1}{\lambda} \quad \mathbb{E}[X] = \int_{0}^{\infty} x f_X(x) dx
\] 
\[
  Var(X) = \frac{1}{\lambda^2}
\] 
La successiva è una \textbf{funzione di soppravivenza}, cioè fornisce la probabilità che
un certo evento non si verifichi entro un certo tempo \( x \) :
\[
  S(x) = 1-f(x) = e^{-\lambda x}
\] 
La seguente è la "perdita di memoria" della variabile aleatoria esponenziale:
\[
 P(X > x+x' | X>x) = P(X > x')
\] 

\begin{example}
  Il tempo (in ore) necessario per riparare un macchinario è una variabile aleatoria
  esponenziale di parametro \( \lambda = 1 \) 
  \[
  X = \text{ Tempo necessario per la riparazione } \;\; X \sim \mathcal{E}(\lambda)
  \] 
  \[
  f_X(x) = \begin{cases}
    e^{-x} & x \ge 0 \\
    0 & x < 0
  \end{cases}
  \] 
  \[
  \mathbb{E}[X] = \frac{1}{\lambda} = 1
  \] 
  \[
  Var(X) = \frac{1}{\lambda^2} = 1
  \] 
  \begin{enumerate}
    \item Qual'è la probabilità che la riparazione superi le 2 ore?
      \[
        P(X >2) = \int_2^{+\infty} e^{-x} dx = e^{-2} \approx 0.135
      \] 
    \item Qual'è la probabilità che la riparazione richieda almeno 3 ore, sapendo che
      ne richiede più di 2?
      \[
        P(X \ge 3 \;|\; X> \ge 2) \underset{\text{assenza di memoria}}{=} P(X \ge 1)
      \] 
  \end{enumerate}
\end{example}

% Inizia il Luigione

\section{Tipi di stima}
Esistono 2 tipi di stima:
\begin{itemize}
  \item \textbf{Stima puntuale}
  \item \textbf{Stima intervallare}
\end{itemize}

\subsection{Stima intervallare}
Si ottiene un intervallo di valori plausibili per il parametro \( \theta \) e a
ciascuno di questi intervalli si associa un livello di confidenza che \( \theta \) vi
appartenga.

\subsection{Stima puntuale}
Si ottiene un singolo valore come stima per il valore di \( \theta \)

\subsubsection{Definizione}
\begin{figure}[H]
  \begin{definition}
    Un \textbf{campione} (o campione casuele/aleatorio) di ampiezza \( n \) è una
    collezione di \( n \) variabili aleatorie \( X_1, \ldots, X_n \) \textbf{indipendenti}
    e tutte con la stessa distribuzione \( f(x; \theta) \), dove \( \theta \) è un
    parametro incognito.
    
    \vspace{1em}
    \noindent Una \textbf{statistica} è una variabile aleatoria ottenuta come funzione
    del campione, cioè:
    \[
    T = T(X_1, \ldots, X_n)
    \] 
  \end{definition}
\end{figure}

\begin{example}
  Sia \( X_1, \ldots, X_n \) un campione casuale. Le variabili aleatorie:
  \begin{itemize}
    \item \( \overline{X} = \frac{1}{n} \sum_{i=1}^n X_i \) (media campionaria)

    \item \( S^2 = \frac{1}{n-1} \sum_{i=1}^n (X_i - \hat{X})^2 \) (varianza campionaria)

    \item \( V = \min\{X_1, \ldots, X_n\} \) (minimo del campione)

    \item \( W = \max\{X_1, \ldots, X_n\} \) (massimo del campione)
  \end{itemize}
  sono tutte statistiche.
\end{example}

\subsubsection{Stimatori}
Uno \textbf{stimatore} (o statistica sufficienta) è una qualunque statistica
\( T \) che non dipende dal parametro
\( \theta \) e viene usata per stimare \( \theta \).

\begin{example}
  La media campionaria \( \overline{X} \) viene tipicamente usata come stimatore del valor
  medio di una popolazione. La varianza campionaria \( S^2 \) viene tipicamente usata
  come stimatore della varianza di una popolazione.
\end{example}

\noindent Una \textbf{Stima} è il \textbf{valore numerico} assunto dallo stimatore sui dati
osservati \( x_1, \ldots, x_n \), cioè \( \hat{\theta} = T(x_1,\ldots,x_n) \) 

\vspace{1em}
\noindent \( T \) si chiama stimatore quando:
\[
  \mathbb{P}_{\theta}(x_1, \ldots, x_n \; | \; T(x_1, \ldots, x_n) = \theta)
\] 

\subsubsection{Stimatori di massima verosomiglianza}
Sia \( X_1, \ldots, X_n \) un campione casuale estratto da una popolazione con densità
\( f(x;\theta) \) (discreta o continua), che dipende dal parametro incognito \( \theta \).

\noindent Se interpretiamo:
\[
f(x_1, \ldots, x_n; \theta)
\] 
come la verosomiglianza (o plausibilità) che si realizzi la n-upla di dati
\( x_1, \ldots, x_n \) quando \( \theta \) è il vero valore del parametro, sempra
ragionevole adottare come stima di \( \theta \) il valore che rende massima la la
verosomiglianza per i dati osservati. In altri termini la stima di massima verosomiglianza
\( \overline{\theta}_{mv} \) è definita come il valore di \( \theta \) che rende
massima la funzione \( f(x_1, \ldots, x_n; \theta) \) quando i dati osservati sono
\( x_1, \ldots, x_n \) 

\vspace{1em}
\noindent In breve si "sposta" \( \theta \) in modo che sia sul massimo dei valori possibili.
Si salva questo valore di \( \theta \) e si chiama \( \overline{\theta} \):
\[
\overline{\theta} = \text{argmax}_{\theta} f(x_1, \ldots, x_n; \theta)
\] 

\begin{example}
  Consideriamo gli esperimenti \( X_1, \ldots, X_n \) distribuiti indipendentemente
  \[
  X_i \quad f(x,\theta)
  \] 
  \[
  f(x_1, \ldots, x_n; \theta) = \prod_{i=1}^n f(x_i; \theta)
  \] 
  \[
    f'(\theta) = \frac{d}{d\theta} f(\hat{x}_1, \ldots, \hat{x}_n; \theta) = 0
  \] 
  \[
    f'(\theta) = 0 \quad \hat{\theta} \text{ t.c.}
  \] 
  \[
    \frac{d}{d\theta} f(x_1, \ldots, x_n; \theta) = f'(x) \text{ (è una statistica)}
  \] 
  \[
    \hat{\theta} \text{ t.c. } \frac{d}{d \theta} f(x_1, \ldots, x_n; \hat{\theta}) = 0
  \] 
  \[
    \Theta = g(X_1, \ldots, X_n) \text{ stimatore}
  \] 
\end{example}

\begin{exercise}
  Stimatore di massima verosomiglianza per il parametro di una Bernoulli:

  \vspace{1em}
  \noindent Sia \( X_1, \ldots, X_n \) un campione casuela estratto da una popolazione
  con densità (discreta) di Bernoulli di parametro \( q \), dove \( q \in (0,1) \) è
  incognito. Dal momento che la densità marginale delle \( X_i \) è data da
  \[
  f(x;q) = q^x(1-q)^{1-x} \quad x \in \{0,1\}
  \] 
  la densità congiunta (e quindi la verosomiglianza) è data da:
  \[
  f(x_1, \ldots, x_n; q) = q^{\sum_{i=1}^n x_i}(1-q)^{n - \sum_{i=1}^n x_i}
  \] 
  La verosomiglianza corrispondente risulta quindi pari a:
  \[
  log(f(x_1, \ldots, x_n; q)) = \left( \sum_{i=1}^n x_i \right) log(q) + \left( n - \sum_{i=1}^n x_i \right) log(1-q)
  \] 
  Derivando rispetto a \( q \) otteniamo:
  \[
  \frac{d}{dq} log(f(x_1, \ldots, x_n; q)) = \frac{\sum_{i=1}^n x_i}{q} - \frac{n - \sum_{i=1}^n x_i}{1-q}
  \] 
  Risolvendo \( \frac{d}{dq} log(f(x_1, \ldots, x_n; q)) = 0 \) rispetto alla variabile \( q \),
  si ottiene \[ \hat{q} = \frac{1}{n} \sum_{i=1}^n x_i \] che è un punto di massimo
  dal momento che:
  \[
  \frac{d^2}{dq^2} log(f(x_1, \ldots, x_n; q)) \le 0
  \] 
  Pertanto lo stimatore di massima verosomiglianza per il parametro \( q \) di una
  distribuzione di Bernoulli è \( Q_{mv} = \overline{X} \) (stimatore di massima verosomiglianza).

  \vspace{1em}
  Supponiamo di aver osservato il seguente campione di dati:
  \[
  1\;0\;0\;0\;0\;0\;1\;0\;0\;1
  \] 
  Allora otteniamo la stima di massima verosomiglianza
  \[
    \hat{q} = \frac{1}{10}(1+0+0+0+0+0+1+0+0+1) = \frac{3}{10} = 0.3
  \] 
  (i dati sono stati generati con \(\mathcal{B}(0.25))\) che sarebbe il calore vero di \( q \))
\end{exercise}

\subsubsection{Bontà di uno stimatore}
Sia \( X_1, \ldots, X_n \) un campione casuale estratto da una popolazione nota eccetto
che per un parametro ignoto \( \theta \).

\noindent Come scegliamo uno stimatore \( T = T(X_1, \ldots, X_n) \) per \( \theta \)?

\noindent Come ne valutiamo la bontà?

\vspace{1em}
\noindent Vogliamo minimizzare la deviazione dal valore reale del parametro e per farlo ci
basiamo sui valori di \( \mathbb{E}(T) \) e \( Var(T) \).

\begin{figure}[H]
  \begin{definition}[Bias]
    Sia \( T = T(X_1, \ldots, X_n) \) uno stimatore del parametro \( \theta \). Allora
    \( b(T) = \mathbb{E}(T) - \theta \) è detto \textbf{bias} (o distorsione) di \( T \)
    come stimatore di \( \theta \).

    \noindent Se esso è nullo, diciamo che \( T \) è uno stimatore corretto (o non distorto)
    di \( \theta \).
  \end{definition}
\end{figure}

\begin{example}
  Sia \( X_1, \ldots, X_n \) un campione casuale estratto da una popolazione con media
  \( \mu \). La media campionaria \( \overline{X} \) è uno stimatore corretto per il valor
  medio \( \mu \). Infatti, calcoliamo:
  \[
    b(\overline{X}) = \mathbb{E}(\overline{X}) - \mu = \frac{1}{n} \sum_{i=1}^n \underbrace{\mathbb{E}(X_i)}_{\text{\( \mu \) per ogni \( i \) }} - \mu =
    \frac{1}{n} \cdot n \mu - \mu = \mu - \mu = 0
  \] 
\end{example}

\begin{definition}[Errore quadratico medio]
  Sia \( T = T(X_1, \ldots, X_n) \) uno stimatore del parametro \( \theta \). Allora
  \( MSE(T) = \mathbb{E}[(T- \theta)^2] \) è detto errore quadratico medio dello
  stimatore \( T \).

  \vspace{1em}
  \noindent Possiamo esplicitare:
  \[
    MSE(T) = \mathbb{E}[(T-\theta)^2] = \mathbb{E}(T^2) - 2\theta \mathbb{E}(T) + \theta^2 =
  \] 
  \[
    = \underbrace{\mathbb{E}(T^2) - \mathbb{E}(T)^2}_{= Var(T)} + \underbrace{\mathbb{E}(T)^2 - 2 \theta \mathbb{E}(T) + \theta^2}_{= (\mathbb{E}(T)-\theta)^2} =
  \] 
  \[
    = Var(T) + b(T)^2
  \] 
  \begin{itemize}
    \item \( Var(T) \) Misura la dispersione/variabilità dello stimatore e quindi la sua
      \textbf{precisione}.
    \item \( b(T)^2 \) Misura l'\textbf{accuratezza} dello stimatore.
  \end{itemize}

  \noindent Uno stimatore buono (o \textbf{efficiente}) controlla sia la varianza che la distorsione.
  Se \( T \) è corretto, allora \( MSE(T) = Var(T) \).

  \vspace{1em}
  \noindent Uno stimatore corretto, ma con varianza molto elevata può avere un errore
  quadratico medio maggiore di uno stimatore distorto, ma con varianza inferiore
\end{definition}

\begin{example}
  Lo stimatore di massima verosomiglianza \( S^2_{mv} \) per la varianza di una popolazione
  normale \textbf{non} è corretto, mentre la varianza campionaria lo è.

  \vspace{1em}
  \noindent Risulta però: \( MSE(S^2_{mv}) \le MSE(S^2) \).
\end{example}

\noindent Lo scopo della statistica è quello di trovare tra gli stimatori corretti (o
asintoticamente corretti) quelli che hanno errore quadratico medio minore.


\subsection{Stima intervallare}
Sia \( X_1, \ldots, X_n \) un campione casuale estratto da una popolazione della quale
ci interessa stimare un parametro \( \theta \). Di solito non ci si può aspettare che la
stima puntuale ottenuta valutando lo stimatore sui dati osservati sia il valore reale
di \( \theta \), ma che ci sia vicino. Perciò, rispetto ad un valore puntuale, è
preferibile produrre produrre un intervallo per il quale abbiamo una certa fiducia che il
parametro ci appartenga.

\begin{definition}
  Sia \( X_1, \ldots, X_n \) un campione casuale estratto da una popolazione della quale
  ci interessa stimare un parametro \( \theta \). Siano:
  \[
  L_1 = L_1(X_1, \ldots, X_2) \quad \text{e} \quad L_2 = L_2(X_1, \ldots, X_n)
  \] 
  due statistiche, che non dipendono da \( \theta \), tali che:
  \[
  \mathbb{P}(L_1 \le \theta \le L_2) = 1 - \alpha
  \] 
  dove \( \alpha \in (0,1) \). L'intervallo \( (L_1, L_2) \) si chiama \textbf{stimatore
  intervallare} del parametro \( \theta \). Siano:
  \[
    \hat{l}_1 = L_1(x_1, \ldots, x_n) \quad \text{e} \quad \hat{l}_2 = L_2(x_1, \ldots, x_n)
  \] 
  i valori assunti dalle statistiche \( L_1 \) e \( L_2 \) sui dati osservati \( x_1, \ldots, x_n \).
  Allora \( \hat{l}_1, \hat{l}_2 \) è l'\textbf{intervallo di confidenza} di livello \( 1- \alpha \) 
  (oppure, al \( (1-\alpha)100\% \)) per il parametro \( \theta \).
\end{definition}

\end{document}
