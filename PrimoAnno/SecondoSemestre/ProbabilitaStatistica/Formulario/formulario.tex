\documentclass[a4paper]{article}

\usepackage[utf8]{inputenc}
\usepackage[T1]{fontenc}
\usepackage{textcomp}
\usepackage[italian]{babel}
\usepackage{amsmath, amssymb}
\usepackage{amsfonts}
\usepackage{mdframed}
\usepackage{ntheorem}
\usepackage{xcolor}
\usepackage{float}
\usepackage{graphicx}
\graphicspath{{./figures/}}

\usepackage{import}
\usepackage{pdfpages}
\usepackage{transparent}
\usepackage{xcolor}

% Useful definitions frame
\theoremstyle{break}
\theoremheaderfont{\bfseries}
\newmdtheoremenv[%
linecolor=gray,leftmargin=0,%
rightmargin=0,
innertopmargin=8pt,%
ntheorem]{define}{Definizioni utili}[section]

% Example frame
\theoremstyle{break}
\theoremheaderfont{\bfseries}
\newmdtheoremenv[%
linecolor=gray,leftmargin=0,%
rightmargin=0,
innertopmargin=8pt,%
ntheorem]{example}{Esempio}[section]

% Important definition frame
\theoremstyle{break}
\theoremheaderfont{\bfseries}
\newmdtheoremenv[%
linecolor=gray,leftmargin=0,%
rightmargin=0,
backgroundcolor=gray!40,%
innertopmargin=8pt,%
ntheorem]{definition}{Definizione}[section]

% Exercise frame
\theoremstyle{break}
\theoremheaderfont{\bfseries}
\newmdtheoremenv[%
linecolor=gray,leftmargin=0,%
rightmargin=0,
innertopmargin=8pt,%
ntheorem]{exercise}{Esercizio}[section]


% figure support
\usepackage{import}
\usepackage{xifthen}
\pdfminorversion=7
\usepackage{pdfpages}
\usepackage{transparent}
\newcommand{\incfig}[1]{%
  \def\svgwidth{\columnwidth}
  \import{./figures/}{#1.pdf_tex}
}

\pdfsuppresswarningpagegroup=1

\begin{document}
\section{Parametri}
\begin{itemize}
  \item \textbf{Media}:
    \[
      \mu = \frac{1}{n} \sum_{i=1}^{n} x_{i}
    \]
  \item \textbf{Varianza}:
    \[
      \sigma^{2} = \frac{1}{n} \sum_{i=1}^{n} (x_{i} - \mu)^{2}
    \]
  \item \textbf{Deviazione standard}:
    \[
      \sigma = \sqrt{\sigma^{2}}
    \]
  \item \textbf{Covarianza}:
    \[
      \text{Cov}(X, Y) = \frac{1}{n} \sum_{i=1}^{n} (x_{i} - \mu_{x}) \cdot (y_{i} - \mu_{y})
    \]
  \item \textbf{Coefficiente di correlazione}:
    \[
      \rho = \frac{\text{Cov}(X, Y)}{\sigma_{x} \cdot \sigma_{y}}
    \]
  \item \textbf{Covarianza campionaria}:
    \[
      \text{Cov}(X, Y) = \frac{1}{n-1} \sum_{i=1}^{n} (x_{i} - \bar{x}) \cdot (y_{i} - \bar{y})
    \]
  \item \textbf{Coefficiente di correlazione campionario}:
    \[
      r = \frac{\text{Cov}(X, Y)}{s_{x} \cdot s_{y}}
    \]
  \item \textbf{Correlazione lineare}:
    \[
      Y = a + b \cdot X
    \]
    \[
      b = \frac{\text{Cov}(X, Y)}{\sigma^{2}_{x}}
    \]
    \[
      a = \mu_{y} - b \cdot \mu_{x}
    \]
\end{itemize}
\section{Probabilità di eventi}
\[
P(A \cup B) = P(A) + P(B) - P(A \cap B)
\] 
\[
P(A \cap B) = P(A) \cdot P(B)
\] 

\section{Variabili aleatorie}
\subsection{Bernoulli}
\[
  X \sim \text{Bern}(p)
\] 
\[
P(X = 1) = p \quad P(X = 0) = 1 - p
\] 
\begin{itemize}
  \item \textbf{Legge}:
    \[
      P(X = x) = \binom{n}{x} \cdot p^{x} \cdot (1-p)^{n-x}
    \] 
  \item \textbf{Valore atteso}:
    \[
      E(X) = p
    \]
  \item \textbf{Varianza}:
    \[
      \text{Var}(X) = p \cdot (1-p)
    \]
\end{itemize}

\subsubsection{Coefficiente binomiale}
\[
  \binom{n}{k} = \frac{n!}{k!(n-k)!}
\] 

\subsection{Normale}
\[
  X \sim \mathcal{N}(\mu, \sigma^{2})
\]
\subsubsection{Standardizzazione}
\[
  Z = \frac{X - \mu}{\sigma}
\]

\subsection{Poission}
\[
  Y \sim \text{Po}(\lambda)
\] 
\begin{itemize}
  \item \textbf{Legge}:
    \[
      p_y(k) = \frac{\lambda^{k} \cdot e^{-\lambda}}{k!}
    \] 
    oppure
    \[
      P(X = x) = \frac{\lambda^{x} \cdot e^{-\lambda}}{x!}
    \] 
  \item \textbf{Valore atteso}:
    \[
      E(Y) = \lambda
    \]
  \item \textbf{Varianza}:
    \[
      \text{Var}(Y) = \lambda
    \]
\end{itemize}

\section{Statistica inferenziale}
\subsection{Correttezza di uno stimatore}
Uno stimatore è corretto quando il bias è nullo. Un bias rappresenta la differenza tra il valore atteso dello stimatore e il valore del parametro da stimare. Quindi uno stimatore è corretto se:
\[
  b(\hat{\theta}) = E(\hat{\theta}) - \theta = 0
\]
dove $\hat{\theta}$ è lo stimatore e $\theta$ è il parametro da stimare.

\subsection{Intervalli di confidenza}
\begin{itemize}
  \item \textbf{Intervallo superiore}:
    \[
      L_{sup} = \hat{\theta} + z_{\frac{\alpha}{2}} \cdot \frac{\sigma}{\sqrt{n}}
    \]
  \item \textbf{Intervallo inferiore}:
    \[
      L_{inf} = \hat{\theta} - z_{\frac{\alpha}{2}} \cdot \frac{\sigma}{\sqrt{n}}
    \]
\end{itemize}

\end{document}
