\documentclass[a4paper]{article}

\usepackage[utf8]{inputenc}
\usepackage[T1]{fontenc}
\usepackage{textcomp}
\usepackage[italian]{babel}
\usepackage{amsmath, amssymb}
\usepackage{booktabs,xltabular}
\usepackage{amsfonts}
\usepackage{cancel}
\usepackage{mdframed}
\usepackage{makecell}
\usepackage{float}
\usepackage{xcolor}
\usepackage{listings}
\usepackage{graphicx}
\usepackage{tikz}
\usetikzlibrary{shapes, arrows, automata, petri, decorations.pathreplacing, positioning, calc}
\usepackage{circuitikz}
\usepackage[label=corner]{karnaugh-map}
\graphicspath{{./figures/}}

\usepackage{ntheorem}
\newtheorem{theorem}{Teorema}

\usepackage{import}
\usepackage{pdfpages}
\usepackage{transparent}
\usepackage{xcolor}

\usepackage{hyperref}
\hypersetup{
    colorlinks=false,
}

% Code blocks
\definecolor{codegreen}{rgb}{0,0.6,0}
\definecolor{codegray}{rgb}{0.5,0.5,0.5}
\definecolor{codepurple}{rgb}{0.58,0,0.82}
\definecolor{backcolour}{rgb}{0.95,0.95,0.95}

\lstdefinestyle{mystyle}{
	backgroundcolor=\color{backcolour},
	commentstyle=\color{codegreen},
	keywordstyle=\color{magenta},
	numberstyle=\tiny\color{codegray},
	stringstyle=\color{codepurple},
	basicstyle=\ttfamily\footnotesize,
	breakatwhitespace=false,
	breaklines=true,
	captionpos=b,
	keepspaces=true,
	numbers=left,
	numbersep=5pt,
	showspaces=false,
	showstringspaces=false,
	showtabs=false,
	tabsize=2
}

\lstset{style=mystyle}

\usepackage{color}

\definecolor{dkgreen}{rgb}{0,0.6,0}
\definecolor{gray}{rgb}{0.5,0.5,0.5}
\definecolor{mauve}{rgb}{0.58,0,0.82}

\lstset{frame=tb,
	aboveskip=3mm,
	belowskip=3mm,
	showstringspaces=false,
	columns=flexible,
	basicstyle={\small\ttfamily},
	numbers=none,
	numberstyle=\tiny\color{gray},
	keywordstyle=\color{blue},
	commentstyle=\color{dkgreen},
	stringstyle=\color{mauve},
	breaklines=true,
	breakatwhitespace=true,
	tabsize=3
}

\usepackage{import}
\usepackage{pdfpages}
\usepackage{transparent}
\usepackage{xcolor}



% Useful definitions frame
\theoremstyle{break}
\theoremheaderfont{\bfseries}
\newmdtheoremenv[%
	linecolor=gray,leftmargin=0,%
	rightmargin=0,
	innertopmargin=8pt,%
	innerbottommargin=8pt,
	ntheorem]{define}{Definizioni utili}[section]

% Example frame
\theoremstyle{break}
\theoremheaderfont{\bfseries}
\newmdtheoremenv[%
	linecolor=gray,leftmargin=0,%
	rightmargin=0,
	innertopmargin=8pt,%
	innerbottommargin=8pt,
	ntheorem]{example}{Esempio}[section]

% Important definition frame
\theoremstyle{break}
\theoremheaderfont{\bfseries}
\newmdtheoremenv[%
	linecolor=gray,leftmargin=0,%
	rightmargin=0,
	backgroundcolor=gray!40,%
	innertopmargin=8pt,%
	innerbottommargin=8pt,
	ntheorem]{definition}{Definizione}[section]

% Exercise frame
\theoremstyle{break}
\theoremheaderfont{\bfseries}
\newmdtheoremenv[%
	linecolor=gray,leftmargin=0,%
	rightmargin=0,
	innertopmargin=8pt,%
	innerbottommargin=8pt,
	ntheorem]{exercise}{Esercizio}[section]

% figure support
\usepackage{import}
\usepackage{xifthen}
\pdfminorversion=7
\usepackage{pdfpages}
\usepackage{transparent}
\newcommand{\incfig}[1]{%
	\def\svgwidth{\columnwidth}
	\import{./figures/}{#1.pdf_tex}
}

% FSM tikz
\tikzset{
    place/.style={
        circle,
        thick,
        draw=black,
        minimum size=6mm,
    },
        state/.style={
        circle,
        thick,
        draw=blue!75,
        fill=blue!20,
        minimum size=6mm,
    },
}

\pdfsuppresswarningpagegroup=1

\begin{document}

\begin{titlepage}
	\begin{center}
		\vspace*{1cm}

		\Huge
		\textbf{Probabilità e Statistica\\Esercizi}

		\vspace{0.5cm}
		\LARGE
		UniVR - Dipartimento di Informatica

		\vspace{1.5cm}

		\textbf{Fabio Irimie}

		\vfill


		\vspace{0.8cm}


		2° Semestre 2023/2024

	\end{center}
\end{titlepage}


\tableofcontents
\pagebreak

% Info:
% Esame: 4 esercizi. Si possono portare appunti, solo scritti a mano e non troppi.
% Libro: Segnali e sistemi (Giuseppe Ricci & Maria Elena Valeher)
% Libro2: Signals & systems second edition (Alan V. Oppenheim & Alan S. Willsky)
\section{Concetti base}
Un sistema è formato da \textbf{segnali trasmessi}, un'esempio di segnale
è la voce che usiamo per comunicare tra di noi. Il sistema prende le informazioni
ricevute dal segnale e le rielabora. 

\noindent 
Degli esempi di sistema sono:
\begin{itemize}
  \item Microfono-Casse
  \item Freno della macchina
\end{itemize}

\subsection{Tipi di segnali}
I segnali possono essere di due tipi:

\begin{itemize}
  \item \textbf{Segnali a tempo continuo}: Segnali che hanno infiniti punti per ogni
    infinitesimo di tempo.
    \begin{figure}[H]
      \centering
      \begin{tikzpicture}
        \draw[->] (-1,0) -- (5,0) node[right] {$t$};
        \draw[->] (0,-1.5) -- (0,1.5) node[above] {$x(t)$};
        \draw[domain=0:4.5,smooth,variable=\x,blue] plot ({\x},{sin(\x*5 r)});
      \end{tikzpicture}
      \caption{Esempio di segnale a tempo continuo}
    \end{figure}
  \item \textbf{Segnali a tempo discreto}: Segnali che hanno un numero finito di punti
    per ogni intervallo di tempo.
    \begin{figure}[H]
      \centering
      \begin{tikzpicture}
        \draw[->] (-1,0) -- (5,0) node[right] {$k$};
        \draw[->] (0,-1.5) -- (0,1.5) node[above] {$x(k)$};
        \draw[domain=0:4.5,smooth,variable=\x,white] plot ({\x},{sin(\x*5 r)});
        \foreach \x in {0,0.2,...,4.5}
        \draw[blue,thick] (\x,0) -- (\x,{sin(\x*5 r)});
      \end{tikzpicture}
      \caption{Esempio di segnale a tempo discreto}
    \end{figure}

\end{itemize}

\noindent
Per elaborare i dati attraverso un computer bisogna convertire un segnale continuo
in uno discreto, questo processo è chiamato \textbf{campionamento} e non è
\textbf{distruttivo}, cioè si può tornare indietro al segnale originale.
\begin{figure}[H]
  \centering
  \begin{tikzpicture}
    \draw[->] (-1,0) -- (5,0) node[right] {$t$};
    \draw[->] (0,-1.5) -- (0,1.5) node[above] {$x(t)$};
    \draw[domain=0:4.5,smooth,variable=\x] plot ({\x},{sin(\x*5 r)});
    \draw[domain=0:4.5,smooth,variable=\x,blue, thick] plot[ycomb, mark=*, mark size=1.5pt] ({\x},{sin(\x*5 r)});
  \end{tikzpicture}
  \caption{Esempio di campionamento}
\end{figure}

\noindent
Una volta campionato il segnale si deve \textbf{quantizzare}, ovvero approssimare
il valore del segnale a un valore discreto, questa operazione è \textbf{parzialmente
distruttiva}, cioè si può tornare indietro al segnale originale perdendo alcune
informazioni.
\begin{figure}[H]
  \centering
  \begin{tikzpicture}
    \draw[->] (-1,0) -- (5,0) node[right] {$t$};
    \draw[->] (0,-1.5) -- (0,1.5) node[above] {$x(t)$};
    \draw[domain=0:4.5,smooth,variable=\x,blue] plot[const plot] ({\x},{sin(\x*5 r)});
  \end{tikzpicture}
  \caption{Esempio di quantizzazione}
\end{figure}

Infine si fa \textbf{encoding}, ovvero
si codifica il segnale per poterlo adattare ad un altro tipo di segnale, questo
processo è \textbf{completamente distruttivo}.

\vspace{1em}
\noindent
I segnali possono essere di dimensioni diverse, ad esempio:
\begin{itemize}
  \item L'andamento di una borsa è un segnale a 1 dimensione.
  \item Una foto in bianco e nero è un segnale a 2 dimensioni \( (x,y) \).
  \item Una foto colorata è un segnale multidimensionale \( (x,y)^3 \) per
    rappresentare ogni colore (R,G,B).
\end{itemize}

\subsection{Rappresentazione dei sistemi}
Un sistema lo rappresentiamo con un blocco, dove all'ingresso mettiamo il segnale
in ingresso e all'uscita il segnale in uscita.
\begin{figure}[H]
  \centering
  \begin{tikzpicture}
    \node[draw,minimum width=1cm,minimum height=1cm] (A) at (0,0) {\( \Sigma \) };
    \draw[->] (-1,0) -- (A) node[midway,above left] {$\stackrel{u(t)}{\text{Ingresso}}$};
    \draw[->] (A) -- (1,0) node[midway,above right] {$\stackrel{v(t)}{\text{Uscita}}$};
  \end{tikzpicture}
  \caption{Rappresentazione di un sistema}
\end{figure}

\noindent
L'output di un sistema può essere rielaborato per essere inserito nuovamente come
input in un altro sistema, ad esempio:

\begin{figure}[H]
  \centering
  \begin{tikzpicture}
    \node[draw,minimum width=1cm,minimum height=1cm] (A) at (0,0) {\( \Sigma \) };
    \draw[->] (-1,0) -- (A) node[midway,above left] {$\stackrel{u(t)}{\text{Ingresso}}$};
    \draw[->] (A) -- (1,0) node[midway,above right] {$\stackrel{v(t)}{\text{Uscita}}$};

    \node[draw,minimum width=1cm,minimum height=1cm] (B) at (0,-1.5) {\( G \) };
    \draw[<-] (-0.75,0) |- (B) node[midway,above left] {};
    \draw[<-] (B) -| (0.75,0) node[midway,above right] {};
  \end{tikzpicture}
  \caption{Rappresentazione di due sistemi in cascata}
\end{figure}

\section{Notazioni}
Tutti i segnali sono indicati con la lettera minuscola, ad esempio:
\[
  \underbrace{f}_{segnale} \quad \underbrace{f(t)}_{\text{segnale a tempo continuo}}
\] 
Oppure si utilizzano delle notazioni standard:
\begin{enumerate}
  \item \( t,\;\tau,\;t_i \): tempo continuo
  \item \( k \): tempo discreto
\end{enumerate}

\noindent
In questo corso si considerano solo segnali continui o discreti monodimensionali
non negativi e solo sistemi \textbf{LTI} (Lineari e Tempo Invarianti):
\begin{enumerate}
  \item \textbf{Lineare}: Vale la \textbf{sovrapposizione degli effetti}, cioè se \( v_1(t) \)
    è l'uscita del sistema per \( u_1(t) \) e \( v_2(t) \) è l'uscita del sistema
    per \( u_2(t) \) allora \( v_1(t) + v_2(t) \) è l'uscita del sistema per
    \( u_1(t) + u_2(t) \).
  \item \textbf{Tempo Invariante}: A prescindere dal punto di tempo in cui si
    applica il segnale, l'uscita del sistema è sempre la stessa.
    \begin{figure}[H]
      \centering
      \begin{tikzpicture}
        \draw[->] (-0.2,0) -- (6,0) node[right] {$t$};
        \draw[->] (0,-0.2) -- (0,2.2) node[above] {$u(t)$};
        \draw[blue, domain=0:6, smooth] plot ({\x},{sin((cos(\x r) * \x - 3)/2 r)+1});

        \node[draw,minimum width=1cm,minimum height=1cm] (A) at (3,-2) {\( \Sigma \) };
        \draw[<-] (A) -- ++(0,1) node[midway,above left] {};
        \draw[->] (A) -- ++(0,-1) node[midway,above right] {};
        \node[below left, blue] at (0,0) {$t_0$};

        \draw[->] (-0.2,-5) -- (6,-5) node[right] {$t$};
        \draw[->] (0,-5.2) -- (0,-3.2) node[above] {$v(t)$};
        \draw[red, domain=0:6, samples=100, smooth, yshift=-5cm] plot ({\x},{sin((cos(2*\x r) * \x - 3)/2 r)+1});
        \node[below left, red] at (0,-5) {$t_0$};
      \end{tikzpicture}
    \end{figure}
    \[
      t_1 = t_0 + t_n
    \] 
    \begin{figure}[H]
      \centering
      \begin{tikzpicture}
        \draw[->] (-0.2,0) -- (6,0) node[right] {$t$};
        \draw[->] (0,-0.2) -- (0,2.2) node[above] {$u(t)$};
        \node[below left] at (0,0) {$t_0$};
        \draw[blue, domain=0:5, smooth] plot ({\x + 1},{sin((cos(\x r) * \x - 3)/2 r)+1});
        \node[below, blue] at (1,0) {$t_1$};

        \node[draw,minimum width=1cm,minimum height=1cm] (A) at (3,-2) {\( \Sigma \) };
        \draw[<-] (A) -- ++(0,1) node[midway,above left] {};
        \draw[->] (A) -- ++(0,-1) node[midway,above right] {};
        \draw[blue] (0.1,0.1) -- ++(0,0.2) -- ++(0.9,0) node[midway, above, blue] {$t_n$} -- ++(0,-0.2);

        \draw[->] (-0.2,-5) -- (6,-5) node[right] {$t$};
        \draw[->] (0,-5.2) -- (0,-3.2) node[above] {$v(t)$};
        \draw[red, domain=0:5, samples=100, smooth, yshift=-5cm] plot ({\x + 1},{sin((cos(2*\x r) * \x - 3)/2 r)+1});
        \node[below, red] at (1,-5) {$t_1$};
        \draw[red] (0.1,-4.9) -- ++(0,0.2) -- ++(0.9,0) node[midway, above, red] {$t_n$} -- ++(0,-0.2);
      \end{tikzpicture}
      \caption{Esempio di invarianza nel tempo}
    \end{figure}
\end{enumerate}

\noindent
I sistemi vengono rappresentati con lettere maiuscole greche o non.

\section{Sistemi}
\subsection{Approccio classico}
Questo approccio prevede di avere un \textbf{evento fisico} (circuito, molla, ecc...) e per
questo evento bisogna definire un \textbf{modello} del sistema. Questo si può fare attraverso
degli strumenti grafici o matematici. Come strumenti matematici si usano:
\begin{enumerate}
  \item \textbf{Continuo}: 
    \begin{enumerate}
      \item Equazioni differenziali
      \item Trasformate di Laplace
      \item Trasformate di Fourier
    \end{enumerate}
  \item \textbf{Discreto}: 
    \begin{enumerate}
      \item Equazioni alle differenze
      \item Trasformate Z
    \end{enumerate}
\end{enumerate}

\noindent Una volta modellato l'evento fisico si può fare un'analisi del sistema
e ciò permette di descrivere la \textbf{stabilità} e le \textbf{proprietà} del sistema.

\noindent
L'ultima fase è quella di \textbf{sintesi}, cioè la fase di correzione del sistema
per far si che risulti stabile.

\subsection{Approccio moderno}
L'approccio moderno ha solo un blocco per rappresentare gli stati:
\begin{figure}[H]
  \centering
  \begin{tikzpicture}
    \node[draw,minimum width=1cm,minimum height=1cm] (A) at (0,0) {Stati};
    \draw[->] (-1,0) -- (A) node[midway,above left] {Ev. Fisico};
  \end{tikzpicture}
  \caption{Rappresentazione di un sistema con l'approccio moderno}
\end{figure}

\subsection{Obsolescenza}
L'obsolescenza è il numero di anni che un sistema può durare. I sistemi che
verranno studiati sono quelli che si trovano nella sezione di comportamento lineare,
cioè i sistemi che non cambiano nel tempo.
\begin{figure}[H]
  \centering
  \begin{tikzpicture}
    % Define constants
    \def\A{1}
    \def\C{1}
    \def\xone{-2}
    \def\xtwo{2}

    % Left quadratic part: A(x - x1)^2
    \draw[blue, thick, domain=-4:\xone, samples=100] 
      plot (\x, {\A*(\x - \xone)^2 + \C});

    % Middle constant part: C
    \draw[blue, thick] 
      plot[domain=\xone:\xtwo] (\x, {\C});

    % Right quadratic part: A(x - x2)^2
    \draw[blue, thick, domain=\xtwo:4, samples=100] 
      plot (\x, {\A*(\x - \xtwo)^2 + \C});

    % Axes
    \draw[->] (-4.2,0) -- (4.5,0) node[right] {$x$};
    \draw[->] (-4,-0.5) -- (-4,5) node[above] {$y$};

    % Labels
    \node[above] at (0, 1) {Comportamento lineare};
  \end{tikzpicture}
  \caption{Sezione di comportamento lineare}
\end{figure}

\noindent
Un'esempio è una molla che si deforma in base alla forza applicata, quando essa
si deforma assume un comportamento plastico e quindi non lineare,
mentre quando non si deforma assume un comportamento elastico e quindi lineare.

\subsection{Causalità}
La causalità è l'input del sistema e l'effetto è l'output che produce, quindi
la causa precede sempre l'effetto. Non esiste un sistema causale che abbia
l'output prima dell'input.
\begin{figure}[H]
  \centering
  \begin{tikzpicture}
    % Define constants
    \def\A{-0.4}      % Amplitude
    \def\lambda{0.2}  % Damping factor
    \def\omega{-1.1}     % Angular frequency
    \def\phi{-6.9}       % Phase shift
    \def\offset{-6.7}         % X Offset


    \draw[->] (-0.5,0) -- (6,0) node[right] {$t$};
    \draw[->] (0,-0.2) -- (0,1.5) node[above] {$u(t)$};

    \draw[domain=0:5.9,smooth,tension=0.8,blue] plot ({\x}, {2.2*\x*exp(-0.8*\x)})
      node[above right] {$u(t)$};

    \draw[domain=1:5.9,smooth,tension=0.8,red] plot ({\x}, 
      {\A*exp(-\lambda*(\x + \offset)) * cos((\omega * (\x + \offset) + \phi) r) + 1})
      node[above] {$v(t)$};

    \node[blue] at (0,0) [below left] {$t_0$};
    \node[red] at (1,0) [below] {$t_1$};

    \node at (3,-1) {$\color{blue} t_0 \color{black} < \color{red} t_1$};
  \end{tikzpicture}
  \caption{Esempio di causalità}
\end{figure}

\subsection{Stabilità}
Un sistema è stabile se, a seguito di un'oscillazione, ritorna al suo stato
di equilibrio e il sistema si ferma. Un sistema è instabile se, a seguito di un'oscillazione,
si allontana dal suo stato di equilibrio.
\begin{figure}[H]
  \centering
  \begin{tikzpicture}
    % Axis
    \draw[->] (-0.2,0) -- (4,0) node[right] {$t$};
    \draw[->] (0,-0.2) -- (0,3) node[above] {$y$};

    \draw[red, domain=-2:1, samples=100, smooth] plot ({\x+2},{exp(\x)});
  \end{tikzpicture}
  \caption{Sistema instabile}
\end{figure}
\begin{figure}[H]
  \centering
  \begin{tikzpicture}
    % Axis
    \draw[->] (-0.2,0) -- (4,0) node[right] {$t$};
    \draw[->] (0,-0.7) -- (0,2) node[above] {$y$};

    \draw[red, domain=0.001:3.9, samples=100, smooth] plot ({\x},{sin(5*\x r)/(3*\x)});
  \end{tikzpicture}
  \caption{Sistema stabile}
\end{figure}

\subsubsection{Stabilità BIBO (Bounded Input Bounded Output)}
Se il segnale di ingresso è limitato in ampiezza allora il segnale di uscita
è limitato in ampiezza.
\[
\exists M > 0,\; | u(t) | < M \; \forall t \in \mathbb{R}
\] 
\[
\Downarrow
\] 
\[
  \exists N > 0,\; | v(t) | < N \; \forall t \in \mathbb{R}
\] 
\[
  \text{con} \; M,N \in \mathbb{R}\; \text{non per forza uguali}
\] 

\begin{figure}[H]
  \centering
  \begin{tikzpicture}[]
    % Input graph
    \draw[->] (-0.2,0) -- (5,0) node[right] {$t$};
    \draw[->] (0,-2) -- (0,2) node[above] {$u(t)$};
    \draw[blue, domain=0.001:4.8, samples=100, smooth] plot ({\x},{sin(\x*\x r)/(0.5*\x)});
    \draw[blue, dashed] (4.8,1.7) -- (0,1.7) node[left] {$M$};
    \draw[blue, dashed] (4.8,-0.92) -- (0,-0.92) node[left] {$-M$};

    % System
    \node[draw,minimum width=1cm,minimum height=1cm] (A) at (2.5,-3) {\( \Sigma \) };
    \draw[<-] (A) -- ++(0,1) node[midway,above left] {};
    \draw[->] (A) -- ++(0,-1) node[midway,above right] {};

    % Output graph
    \draw[->] (0,-8) -- (0,-4) node[above] {$v(t)$};
    \draw[->] (-0.2,-6) -- (5,-6) node[right] {$t$};
    \draw[red, domain=0.001:4.8, samples=200, smooth] plot ({\x},{sin(\x*\x*\x*0.5 r)/(0.7*\x) - 6});
    \draw[red, dashed] (4.8,-5) -- (0,-5) node[left] {$N$};
    \draw[red, dashed] (4.8,-6.65) -- (0,-6.65) node[left] {$-N$};
  \end{tikzpicture}
  \caption{Esempio di sistema stabile BIBO}
\end{figure}

\subsubsection{Stabilità Asintotica}
Se il segnale di ingresso si annulla allora il segnale di uscita si annulla.
\[
\lim_{t \to \infty} v(t) = 0 \;\; \forall r \; \text{di} \; u(t),\; t \in \mathbb{R}
\] 
\begin{figure}[H]
  \centering
  \begin{tikzpicture}[]
    % Input graph
    \draw[->] (-0.2,0) -- (5,0) node[right] {$t$};
    \draw[->] (0,-2) -- (0,2) node[above] {$u(t)$};
    \draw[blue, domain=0.001:4.8, samples=100, smooth] plot ({\x},{(sin((\x*5)/2 r))/(1.3*\x)});
    \node[above, blue, scale=0.8, yshift=0.2cm] at (5,0) {$\lim_{t \to \infty} = 0$};

    % System
    \node[draw,minimum width=1cm,minimum height=1cm] (A) at (2.5,-3) {\( \Sigma \) };
    \draw[<-] (A) -- ++(0,1) node[midway,above left] {};
    \draw[->] (A) -- ++(0,-1) node[midway,above right] {};

    % Output graph
    \draw[->] (0,-8) -- (0,-4) node[above] {$v(t)$};
    \draw[->] (-0.2,-6) -- (5,-6) node[right] {$t$};
    \draw[red, domain=0.001:4.8, smooth] plot ({\x},{sin(\x*5 r)/(2.5*\x) - 6});
    \node[above, red, scale=0.8, yshift=0.2cm] at (5,-6) {$\lim_{t \to \infty} = 0$};
  \end{tikzpicture}
  \caption{Esempio di sistema stabile asintotico}
\end{figure}

\noindent
La stabilità asintotica implica la stabilità BIBO, ma non viceversa.

\section{Modello di segnali}
Un segnale si può scrivere nel seguente modo:
\[
\alpha \in \mathbb{C}
\] 
\[
  l \in \mathbb{R}
\] 

\[
  y(t) = \sum_{i} \sum_{j} c_{ij} \cdot \color{blue} \underbrace{l^{\alpha t}}_{\text{Parte esponenziale}}
  \color{black} \cdot \color{green} \underbrace{\frac{t^l}{l!}}_{\text{Parte polinomiale}}
\] 
\begin{figure}[H]
  \centering
  \begin{tikzpicture}
    \draw[->] (-3,0) -- (3,0) node[right] {$t$};
    \draw[->] (0,-0.2) -- (0,3) node[above] {$y(t)$};
    \draw[domain=-3:1.1,smooth,variable=\x,blue] plot ({\x},{exp(\x)}) node[above right, scale=0.9] {$l^{\alpha t} \;\; (\alpha > 0)$};
    \draw[domain=-1.1:3,smooth,variable=\x,cyan] plot ({\x},{exp(-\x)}) node[above right, scale=0.9] {$l^{\alpha t} \;\; (\alpha < 0)$};
    \draw[domain=0:3,smooth,variable=\x,green] plot ({\x},{\x}) node[above right, scale=0.9] {$t$};

    \draw[->] (0,-1) -- ++(0,-1) node[midway, right] {Risultato};

    \draw[->,yshift=1cm] (-3.2,-6) -- (3,-6) node[right] {$t$};
    \draw[->,yshift=1cm] (-3,-6.2) -- (-3,-4) node[above] {$y(t)$};
    \draw[domain=0:5.9,smooth,tension=0.8,red,yshift=-5cm,xshift=-3cm] plot ({\x}, {2.2*\x*exp(-0.8*\x)});
  \end{tikzpicture}
  \caption{Esempo di segnale}
\end{figure}

\noindent
Ad esempio con \( l = 1 \):
\[
  y(t) = \sum_{i} \sum_{j} c_{ij} \cdot l^{\alpha t} \cdot \frac{t^1}{1!} =
  \sum_{i} \sum_{j} c_{ij} \cdot l^{\alpha t} \cdot t
\] 

\noindent
Con \( \alpha < 0 \) il sistema è stabile perchè l'esponenziale tende a 0.

\vspace{1em}
\noindent 
Con \( l = 2 \):
\[
  y(t) = \sum_{i} \sum_{j} c_{ij} \cdot l^{\alpha t} \cdot \frac{t^2}{2!} =
  \sum_{i} \sum_{j} c_{ij} \cdot l^{\alpha t} \cdot \frac{t^2}{2}
\]
ecc...

\end{document}
