\documentclass[a4paper]{article}
\usepackage{import}
\usepackage[utf8]{inputenc}
\usepackage[T1]{fontenc}
\usepackage{textcomp}
\usepackage[italian]{babel}
\usepackage{amsmath, amssymb}
\usepackage{booktabs,xltabular}
\usepackage{amsfonts}
\usepackage{subcaption}
\usepackage{amsthm}
\usepackage{cancel}
\usepackage{mdframed}
\usepackage{makecell}
\usepackage{float}
\usepackage{xcolor}
\usepackage{listings}
\usepackage{gensymb}
\usepackage{graphicx}
\usepackage{bodeplot}
\usepackage{physics}
\usepackage{tikz}
\usetikzlibrary{shapes, arrows, automata, petri, decorations.markings, decorations.pathreplacing, positioning, calc, quotes}
\usepackage{circuitikz}
\usepackage[label=corner]{karnaugh-map}
\graphicspath{{./figures/}}

% Set default font to sans-serif
\renewcommand{\familydefault}{\sfdefault} 
\usepackage{eulervm}

\usepackage{forest}

\usepackage{mathtools}
\DeclarePairedDelimiter\ceil{\lceil}{\rceil}
\DeclarePairedDelimiter\floor{\lfloor}{\rfloor}

% \usepackage{ntheorem}

\usepackage{import}
\usepackage{pdfpages}
\usepackage{transparent}
\usepackage{xcolor}

\usepackage{hyperref}
\hypersetup{
    colorlinks=false,
}

% Code blocks
\definecolor{codegreen}{rgb}{0,0.6,0}
\definecolor{codegray}{rgb}{0.5,0.5,0.5}
\definecolor{codepurple}{rgb}{0.58,0,0.82}
\definecolor{backcolour}{rgb}{0.95,0.95,0.95}

\lstdefinestyle{mystyle}{
	backgroundcolor=\color{backcolour},
	commentstyle=\color{codegreen},
	keywordstyle=\color{magenta},
	numberstyle=\tiny\color{codegray},
	stringstyle=\color{codepurple},
	basicstyle=\ttfamily\footnotesize,
	breakatwhitespace=false,
	breaklines=true,
	captionpos=b,
	keepspaces=true,
	numbers=left,
	numbersep=5pt,
	showspaces=false,
	showstringspaces=false,
	showtabs=false,
	tabsize=2
}

\lstset{style=mystyle}

\usepackage{color}
\usepackage{import}
\usepackage{pdfpages}
\usepackage{transparent}
\usepackage{xcolor}

% Example frame
\theoremstyle{definition}
\newmdtheoremenv[%
	linecolor=gray,leftmargin=0,%
	rightmargin=0,
	innertopmargin=8pt,%
	innerbottommargin=8pt,
	ntheorem]{example}{Esempio}[section]

% Important definition frame
\theoremstyle{definition}
\newmdtheoremenv[%
	linecolor=gray,leftmargin=0,%
	rightmargin=0,
	backgroundcolor=gray!40,%
	innertopmargin=8pt,%
	innerbottommargin=8pt,
	ntheorem]{definition}{Definizione}[section]

% Exercise frame
\theoremstyle{definition}
\newmdtheoremenv[%
	linecolor=gray,leftmargin=0,%
	rightmargin=0,
	innertopmargin=8pt,%
	innerbottommargin=8pt,
	ntheorem]{exercise}{Esercizio}[section]

% Theorem frame
\theoremstyle{definition}
\newmdtheoremenv[%
  linecolor=gray,leftmargin=0,%
  rightmargin=0,
  innertopmargin=8pt,%
  innerbottommargin=8pt,
  ntheorem]{theorem}{Teorema}[section]

\theoremstyle{definition}
\newmdtheoremenv[%
  linecolor=white,leftmargin=0,%
  rightmargin=0,
  innertopmargin=8pt,%
  innerbottommargin=8pt,
  ntheorem]{define}{Definizione utile}[section]

% figure support
\usepackage{import}
\usepackage{xifthen}
\pdfminorversion=7
\usepackage{pdfpages}
\usepackage{transparent}
\newcommand{\incfig}[1]{%
	\def\svgwidth{\columnwidth}
	\import{./figures/}{#1.pdf_tex}
}

% FSM tikz
\tikzset{
    place/.style={
        circle,
        thick,
        draw=black,
        minimum size=6mm,
    },
        state/.style={
        circle,
        thick,
        draw=black,
        fill=white,
        minimum size=6mm,
    },
}

\pdfsuppresswarningpagegroup=1

\usepackage{pgfplots}
\pgfplotsset{compat=1.18,width=10cm}

% Save plots as pdf and reuse them without compiling every time
\usetikzlibrary{external}
\tikzexternalize[prefix=figures/tikz/, optimize=false]


\pgfplotsset{width=7cm}

\begin{document}

\begin{titlepage}
	\begin{center}
		\vspace*{1cm}

		\Huge
		\textbf{Probabilità e Statistica\\Esercizi}

		\vspace{0.5cm}
		\LARGE
		UniVR - Dipartimento di Informatica

		\vspace{1.5cm}

		\textbf{Fabio Irimie}

		\vfill


		\vspace{0.8cm}


		2° Semestre 2023/2024

	\end{center}
\end{titlepage}


\tableofcontents
\pagebreak

% Info:
% Esame: 4 esercizi. Si possono portare appunti, solo scritti a mano e non troppi.
% Libro: Segnali e sistemi (Giuseppe Ricci & Maria Elena Valeher)
% Libro2: Signals & systems second edition (Alan V. Oppenheim & Alan S. Willsky)
\section{Concetti base}
Un sistema è formato da \textbf{segnali trasmessi}, un'esempio di segnale
è la voce che usiamo per comunicare tra di noi. Il sistema prende le informazioni
ricevute dal segnale e le rielabora. 

\noindent 
Degli esempi di sistema sono:
\begin{itemize}
  \item Microfono-Casse
  \item Freno della macchina
\end{itemize}

\subsection{Tipi di segnali}
I segnali possono essere di due tipi:

\begin{itemize}
  \item \textbf{Segnali a tempo continuo}: Segnali che hanno infiniti punti per ogni
    infinitesimo di tempo.
    \begin{figure}[H]
      \centering
      \begin{tikzpicture}
        \draw[->] (-1,0) -- (5,0) node[right] {$t$};
        \draw[->] (0,-1.5) -- (0,1.5) node[above] {$x(t)$};
        \draw[domain=0:4.5,smooth,samples=100,variable=\x,blue] plot ({\x},{sin(\x*5 r)});
      \end{tikzpicture}
      \caption{Esempio di segnale a tempo continuo}
    \end{figure}
  \item \textbf{Segnali a tempo discreto}: Segnali che hanno un numero finito di punti
    per ogni intervallo di tempo.
    \begin{figure}[H]
      \centering
      \begin{tikzpicture}
        \draw[->] (-1,0) -- (5,0) node[right] {$k$};
        \draw[->] (0,-1.5) -- (0,1.5) node[above] {$x(k)$};
        \draw[domain=0:4.5,smooth,variable=\x,white] plot ({\x},{sin(\x*5 r)});
        \foreach \x in {0,0.2,...,4.5}
       \draw[blue,thick] (\x,0) -- (\x,{sin(\x*5 r)});
      \end{tikzpicture}
      \caption{Esempio di segnale a tempo discreto}
    \end{figure}

\end{itemize}

\noindent
Per elaborare i dati attraverso un computer bisogna convertire un segnale continuo
in uno discreto, questo processo è chiamato \textbf{campionamento} e non è
\textbf{distruttivo}, cioè si può tornare indietro al segnale originale.
\begin{figure}[H]
  \centering
  \begin{tikzpicture}
    \draw[->] (-1,0) -- (5,0) node[right] {$t$};
    \draw[->] (0,-1.5) -- (0,1.5) node[above] {$x(t)$};
    \draw[domain=0:4.5,smooth,variable=\x] plot ({\x},{sin(\x*5 r)});
    \draw[domain=0:4.5,smooth,variable=\x,blue, thick] plot[ycomb, mark=*, mark size=1.5pt] ({\x},{sin(\x*5 r)});
  \end{tikzpicture}
  \caption{Esempio di campionamento}
\end{figure}

\noindent
Una volta campionato il segnale si deve \textbf{quantizzare}, ovvero approssimare
il valore del segnale a un valore discreto, questa operazione è \textbf{parzialmente
distruttiva}, cioè si può tornare indietro al segnale originale perdendo alcune
informazioni.
\begin{figure}[H]
  \centering
  \begin{tikzpicture}
    \draw[->] (-1,0) -- (5,0) node[right] {$t$};
    \draw[->] (0,-1.5) -- (0,1.5) node[above] {$x(t)$};
    \draw[domain=0:4.5,smooth,variable=\x,blue] plot[const plot] ({\x},{sin(\x*5 r)});
  \end{tikzpicture}
  \caption{Esempio di quantizzazione}
\end{figure}

Infine si fa \textbf{encoding}, ovvero
si codifica il segnale per poterlo adattare ad un altro tipo di segnale, questo
processo è \textbf{completamente distruttivo}.

\vspace{1em}
\noindent
I segnali possono essere di dimensioni diverse, ad esempio:
\begin{itemize}
  \item L'andamento di una borsa è un segnale a 1 dimensione.
  \item Una foto in bianco e nero è un segnale a 2 dimensioni \( (x,y) \).
  \item Una foto colorata è un segnale multidimensionale \( (x,y)^3 \) per
    rappresentare ogni colore (R,G,B).
\end{itemize}

\subsection{Rappresentazione dei sistemi}
Un sistema lo rappresentiamo con un blocco, dove all'ingresso mettiamo il segnale
in ingresso e all'uscita il segnale in uscita.
\begin{figure}[H]
  \centering
  \begin{tikzpicture}
    \node[draw,minimum width=1cm,minimum height=1cm] (A) at (0,0) {\( \Sigma \) };
    \draw[->] (-1,0) -- (A) node[midway,above left] {$\stackrel{u(t)}{\text{Ingresso}}$};
    \draw[->] (A) -- (1,0) node[midway,above right] {$\stackrel{v(t)}{\text{Uscita}}$};
  \end{tikzpicture}
  \caption{Rappresentazione di un sistema}
\end{figure}

\noindent
L'output di un sistema può essere rielaborato per essere inserito nuovamente come
input in un altro sistema, ad esempio:

\begin{figure}[H]
  \centering
  \begin{tikzpicture}
    \node[draw,minimum width=1cm,minimum height=1cm] (A) at (0,0) {\( \Sigma \) };
    \draw[->] (-1,0) -- (A) node[midway,above left] {$\stackrel{u(t)}{\text{Ingresso}}$};
    \draw[->] (A) -- (1,0) node[midway,above right] {$\stackrel{v(t)}{\text{Uscita}}$};

    \node[draw,minimum width=1cm,minimum height=1cm] (B) at (0,-1.5) {\( G \) };
    \draw[<-] (-0.75,0) |- (B) node[midway,above left] {};
    \draw[<-] (B) -| (0.75,0) node[midway,above right] {};
  \end{tikzpicture}
  \caption{Rappresentazione di due sistemi in cascata}
\end{figure}

\section{Notazioni}
Tutti i segnali sono indicati con la lettera minuscola, ad esempio:
\[
  \underbrace{f}_{segnale} \quad \underbrace{f(t)}_{\text{segnale a tempo continuo}}
\] 
Oppure si utilizzano delle notazioni standard:
\begin{enumerate}
  \item \( t,\;\tau,\;t_i \): tempo continuo
  \item \( k \): tempo discreto
\end{enumerate}

\noindent
In questo corso si considerano solo segnali continui o discreti monodimensionali
non negativi e solo sistemi \textbf{LTI} (Lineari e Tempo Invarianti):
\begin{enumerate}
  \item \textbf{Lineare}: Vale la \textbf{sovrapposizione degli effetti}, cioè se \( v_1(t) \)
    è l'uscita del sistema per \( u_1(t) \) e \( v_2(t) \) è l'uscita del sistema
    per \( u_2(t) \) allora \( v_1(t) + v_2(t) \) è l'uscita del sistema per
    \( u_1(t) + u_2(t) \).
  \item \textbf{Tempo Invariante}: A prescindere dal punto di tempo in cui si
    applica il segnale, l'uscita del sistema è sempre la stessa.
    \begin{figure}[H]
      \centering
      \begin{tikzpicture}
        \draw[->] (-0.2,0) -- (6,0) node[right] {$t$};
        \draw[->] (0,-0.2) -- (0,2.2) node[above] {$u(t)$};
        \draw[blue, domain=0:6, smooth] plot ({\x},{sin((cos(\x r) * \x - 3)/2 r)+1});

        \node[draw,minimum width=1cm,minimum height=1cm] (A) at (3,-2) {\( \Sigma \) };
        \draw[<-] (A) -- ++(0,1) node[midway,above left] {};
        \draw[->] (A) -- ++(0,-1) node[midway,above right] {};
        \node[below left, blue] at (0,0) {$t_0$};

        \draw[->] (-0.2,-5) -- (6,-5) node[right] {$t$};
        \draw[->] (0,-5.2) -- (0,-3.2) node[above] {$v(t)$};
        \draw[red, domain=0:6, samples=100, smooth, yshift=-5cm] plot ({\x},{sin((cos(2*\x r) * \x - 3)/2 r)+1});
        \node[below left, red] at (0,-5) {$t_0$};
      \end{tikzpicture}
    \end{figure}
    \[
      t_1 = t_0 + t_n
    \] 
    \begin{figure}[H]
      \centering
      \begin{tikzpicture}
        \draw[->] (-0.2,0) -- (6,0) node[right] {$t$};
        \draw[->] (0,-0.2) -- (0,2.2) node[above] {$u(t)$};
        \node[below left] at (0,0) {$t_0$};
        \draw[blue, domain=0:5, smooth] plot ({\x + 1},{sin((cos(\x r) * \x - 3)/2 r)+1});
        \node[below, blue] at (1,0) {$t_1$};

        \node[draw,minimum width=1cm,minimum height=1cm] (A) at (3,-2) {\( \Sigma \) };
        \draw[<-] (A) -- ++(0,1) node[midway,above left] {};
        \draw[->] (A) -- ++(0,-1) node[midway,above right] {};
        \draw[blue] (0.1,0.1) -- ++(0,0.2) -- ++(0.9,0) node[midway, above, blue] {$t_n$} -- ++(0,-0.2);

        \draw[->] (-0.2,-5) -- (6,-5) node[right] {$t$};
        \draw[->] (0,-5.2) -- (0,-3.2) node[above] {$v(t)$};
        \draw[red, domain=0:5, samples=100, smooth, yshift=-5cm] plot ({\x + 1},{sin((cos(2*\x r) * \x - 3)/2 r)+1});
        \node[below, red] at (1,-5) {$t_1$};
        \draw[red] (0.1,-4.9) -- ++(0,0.2) -- ++(0.9,0) node[midway, above, red] {$t_n$} -- ++(0,-0.2);
      \end{tikzpicture}
      \caption{Esempio di invarianza nel tempo}
    \end{figure}
\end{enumerate}

\noindent
I sistemi vengono rappresentati con lettere maiuscole greche o non.

\section{Sistemi}
\subsection{Approccio classico}
Questo approccio prevede di avere un \textbf{evento fisico} (circuito, molla, ecc...) e per
questo evento bisogna definire un \textbf{modello} del sistema. Questo si può fare attraverso
degli strumenti grafici o matematici. Come strumenti matematici si usano:
\begin{enumerate}
  \item \textbf{Continuo}: 
    \begin{enumerate}
      \item Equazioni differenziali
      \item Trasformate di Laplace
      \item Trasformate di Fourier
    \end{enumerate}
  \item \textbf{Discreto}: 
    \begin{enumerate}
      \item Equazioni alle differenze
      \item Trasformate Z
    \end{enumerate}
\end{enumerate}

\noindent Una volta modellato l'evento fisico si può fare un'analisi del sistema
e ciò permette di descrivere la \textbf{stabilità} e le \textbf{proprietà} del sistema.

\noindent
L'ultima fase è quella di \textbf{sintesi}, cioè la fase di correzione del sistema
per far si che risulti stabile.

\subsection{Approccio moderno}
L'approccio moderno ha solo un blocco per rappresentare gli stati:
\begin{figure}[H]
  \centering
  \begin{tikzpicture}
    \node[draw,minimum width=1cm,minimum height=1cm] (A) at (0,0) {Stati};
    \draw[->] (-1,0) -- (A) node[midway,above left] {Ev. Fisico};
  \end{tikzpicture}
  \caption{Rappresentazione di un sistema con l'approccio moderno}
\end{figure}

\subsection{Obsolescenza}
L'obsolescenza è il numero di anni che un sistema può durare. I sistemi che
verranno studiati sono quelli che si trovano nella sezione di comportamento lineare,
cioè i sistemi che non cambiano nel tempo.
\begin{figure}[H]
  \centering
  \begin{tikzpicture}
    % Define constants
    \def\A{1}
    \def\C{1}
    \def\xone{-2}
    \def\xtwo{2}

    % Left quadratic part: A(x - x1)^2
    \draw[blue, thick, domain=-4:\xone, samples=100] 
      plot (\x, {\A*(\x - \xone)^2 + \C});

    % Middle constant part: C
    \draw[blue, thick] 
      plot[domain=\xone:\xtwo] (\x, {\C});

    % Right quadratic part: A(x - x2)^2
    \draw[blue, thick, domain=\xtwo:4, samples=100] 
      plot (\x, {\A*(\x - \xtwo)^2 + \C});

    % Axes
    \draw[->] (-4.2,0) -- (4.5,0) node[right] {$x$};
    \draw[->] (-4,-0.5) -- (-4,5) node[above] {$y$};

    % Labels
    \node[above] at (0, 1) {Comportamento lineare};
  \end{tikzpicture}
  \caption{Sezione di comportamento lineare}
\end{figure}

\noindent
Un'esempio è una molla che si deforma in base alla forza applicata, quando essa
si deforma assume un comportamento plastico e quindi non lineare,
mentre quando non si deforma assume un comportamento elastico e quindi lineare.

\subsection{Causalità}
La causalità è l'input del sistema e l'effetto è l'output che produce, quindi
la causa precede sempre l'effetto. Non esiste un sistema causale che abbia
l'output prima dell'input.
\begin{figure}[H]
  \centering
  \begin{tikzpicture}
    % Define constants
    \def\A{-0.4}      % Amplitude
    \def\lambda{0.2}  % Damping factor
    \def\omega{-1.1}     % Angular frequency
    \def\phi{-6.9}       % Phase shift
    \def\offset{-6.7}         % X Offset


    \draw[->] (-0.5,0) -- (6,0) node[right] {$t$};
    \draw[->] (0,-0.2) -- (0,1.5) node[above] {$u(t)$};

    \draw[domain=0:5.9,smooth,tension=0.8,blue] plot ({\x}, {2.2*\x*exp(-0.8*\x)})
      node[above right] {$u(t)$};

    \draw[domain=1:5.9,smooth,tension=0.8,red] plot ({\x}, 
      {\A*exp(-\lambda*(\x + \offset)) * cos((\omega * (\x + \offset) + \phi) r) + 1})
      node[above] {$v(t)$};

    \node[blue] at (0,0) [below left] {$t_0$};
    \node[red] at (1,0) [below] {$t_1$};

    \node at (3,-1) {$\color{blue} t_0 \color{black} < \color{red} t_1$};
  \end{tikzpicture}
  \caption{Esempio di causalità}
\end{figure}

\subsection{Stabilità}
Un sistema è stabile se, a seguito di un'oscillazione, ritorna al suo stato
di equilibrio e il sistema si ferma. Un sistema è instabile se, a seguito di un'oscillazione,
si allontana dal suo stato di equilibrio.
\begin{figure}[H]
  \centering
  \begin{tikzpicture}
    % Axis
    \draw[->] (-0.2,0) -- (4,0) node[right] {$t$};
    \draw[->] (0,-0.2) -- (0,3) node[above] {$y$};

    \draw[red, domain=-2:1, samples=100, smooth] plot ({\x+2},{exp(\x)});
  \end{tikzpicture}
  \caption{Sistema instabile}
\end{figure}
\begin{figure}[H]
  \centering
  \begin{tikzpicture}
    % Axis
    \draw[->] (-0.2,0) -- (4,0) node[right] {$t$};
    \draw[->] (0,-0.7) -- (0,2) node[above] {$y$};

    \draw[red, domain=0.001:3.9, samples=100, smooth] plot ({\x},{sin(5*\x r)/(3*\x)});
  \end{tikzpicture}
  \caption{Sistema stabile}
\end{figure}

\subsubsection{Stabilità BIBO (Bounded Input Bounded Output)}
Se il segnale di ingresso è limitato in ampiezza allora il segnale di uscita
è limitato in ampiezza.
\[
\exists M > 0,\; | u(t) | < M \; \forall t \in \mathbb{R}
\] 
\[
\Downarrow
\] 
\[
  \exists N > 0,\; | v(t) | < N \; \forall t \in \mathbb{R}
\] 
\[
  \text{con} \; M,N \in \mathbb{R}\; \text{non per forza uguali}
\] 

\begin{figure}[H]
  \centering
  \begin{tikzpicture}[]
    % Input graph
    \draw[->] (-0.2,0) -- (5,0) node[right] {$t$};
    \draw[->] (0,-2) -- (0,2) node[above] {$u(t)$};
    \draw[blue, domain=0.001:4.8, samples=100, smooth] plot ({\x},{sin(\x*\x r)/(0.5*\x)});
    \draw[blue, dashed] (4.8,1.7) -- (0,1.7) node[left] {$M$};
    \draw[blue, dashed] (4.8,-0.92) -- (0,-0.92) node[left] {$-M$};

    % System
    \node[draw,minimum width=1cm,minimum height=1cm] (A) at (2.5,-3) {\( \Sigma \) };
    \draw[<-] (A) -- ++(0,1) node[midway,above left] {};
    \draw[->] (A) -- ++(0,-1) node[midway,above right] {};

    % Output graph
    \draw[->] (0,-8) -- (0,-4) node[above] {$v(t)$};
    \draw[->] (-0.2,-6) -- (5,-6) node[right] {$t$};
    \draw[red, domain=0.001:4.8, samples=200, smooth] plot ({\x},{sin(\x*\x*\x*0.5 r)/(0.7*\x) - 6});
    \draw[red, dashed] (4.8,-5) -- (0,-5) node[left] {$N$};
    \draw[red, dashed] (4.8,-6.65) -- (0,-6.65) node[left] {$-N$};
  \end{tikzpicture}
  \caption{Esempio di sistema stabile BIBO}
\end{figure}

\subsubsection{Stabilità Asintotica}
Se il segnale di ingresso si annulla allora il segnale di uscita si annulla.
\[
\lim_{t \to \infty} v(t) = 0 \;\; \forall r \; \text{di} \; u(t),\; t \in \mathbb{R}
\] 
\begin{figure}[H]
  \centering
  \begin{tikzpicture}[]
    % Input graph
    \draw[->] (-0.2,0) -- (5,0) node[right] {$t$};
    \draw[->] (0,-2) -- (0,2) node[above] {$u(t)$};
    \draw[blue, domain=0.001:4.8, samples=100, smooth] plot ({\x},{(sin((\x*5)/2 r))/(1.3*\x)});
    \node[above, blue, scale=0.8, yshift=0.2cm] at (5,0) {$\lim_{t \to \infty} = 0$};

    % System
    \node[draw,minimum width=1cm,minimum height=1cm] (A) at (2.5,-3) {\( \Sigma \) };
    \draw[<-] (A) -- ++(0,1) node[midway,above left] {};
    \draw[->] (A) -- ++(0,-1) node[midway,above right] {};

    % Output graph
    \draw[->] (0,-8) -- (0,-4) node[above] {$v(t)$};
    \draw[->] (-0.2,-6) -- (5,-6) node[right] {$t$};
    \draw[red, domain=0.001:4.8, smooth] plot ({\x},{sin(\x*5 r)/(2.5*\x) - 6});
    \node[above, red, scale=0.8, yshift=0.2cm] at (5,-6) {$\lim_{t \to \infty} = 0$};
  \end{tikzpicture}
  \caption{Esempio di sistema stabile asintotico}
\end{figure}

\noindent
La stabilità asintotica implica la stabilità BIBO, ma non viceversa.

\section{Modello di segnali}
Un segnale si può scrivere nel seguente modo:
\[
\alpha \in \mathbb{C}
\] 
\[
  t \in \mathbb{R}
\] 
\[
l \in \mathbb{Z}
\] 

\[
  y(t) = \sum_{i} \sum_{j} c_{ij} \cdot \color{blue} \underbrace{e^{\alpha t}}_{\text{Parte esponenziale}}
  \color{black} \cdot \color{green} \underbrace{\frac{t^l}{l!}}_{\text{Parte polinomiale}}
\] 
\begin{figure}[H]
  \centering
  \begin{tikzpicture}
    \draw[->] (-3,0) -- (3,0) node[right] {$t$};
    \draw[->] (0,-0.2) -- (0,3) node[above] {$y(t)$};
    \draw[domain=-3:1.1,smooth,variable=\x,blue] plot ({\x},{exp(\x)}) node[above right, scale=0.9] {$e^{\alpha t} \;\; (\alpha > 0)$};
    \draw[domain=-1.1:3,smooth,variable=\x,cyan] plot ({\x},{exp(-\x)}) node[above right, scale=0.9] {$e^{\alpha t} \;\; (\alpha < 0)$};
    \draw[domain=0:3,smooth,variable=\x,green] plot ({\x},{\x}) node[above right, scale=0.9] {$t$};

    \draw[->] (0,-1) -- ++(0,-1) node[midway, right] {Risultato};

    \draw[->,yshift=1cm] (-3.2,-6) -- (3,-6) node[right] {$t$};
    \draw[->,yshift=1cm] (-3,-6.2) -- (-3,-4) node[above] {$y(t)$};
    \draw[domain=0:5.9,smooth,tension=0.8,red,yshift=-5cm,xshift=-3cm] plot ({\x}, {2.2*\x*exp(-0.8*\x)});
  \end{tikzpicture}
  \caption{Esempo di segnale}
\end{figure}

\noindent
Ad esempio con \( l = 1 \):
\[
  y(t) = \sum_{i} \sum_{j} c_{ij} \cdot e^{\alpha t} \cdot \frac{t^1}{1!} =
  \sum_{i} \sum_{j} c_{ij} \cdot e^{\alpha t} \cdot t
\] 

\noindent
Con \( \alpha < 0 \) il sistema è stabile perchè l'esponenziale tende a 0.

\vspace{1em}
\noindent 
Con \( l = 2 \):
\[
  y(t) = \sum_{i} \sum_{j} c_{ij} \cdot e^{\alpha t} \cdot \frac{t^2}{2!} =
  \sum_{i} \sum_{j} c_{ij} \cdot e^{\alpha t} \cdot \frac{t^2}{2}
\]
ecc...

\vspace{1em}
\noindent
Siccome \( \alpha \in \mathbb{C} \) si può riscrivere come:
\[
\alpha = \lambda + j \omega
\] 
\[
  \lambda \text{ è la parte reale}
\] 
\[
  j \omega \text{ è la parte immaginaria}
\] 
Quindi il segnale diventa:
\[
  y(t) = \sum_{i} \sum_{j} c_{ij} \cdot e^{\lambda t} \cdot e^{j \omega t} \cdot \frac{t^l}{l!}
\] 
Utilizzando la forma trigonometrica dei numeri complessi si ha che:
\[
  e^{j \omega} = \cos(\omega t) + j \sin(\omega t)
\] 
\[
  e^{(\lambda + j \omega)} = e^{\alpha t} = \rho (cos(\omega t) + j \sin(\omega t))
\] 
Per le formule di Eulero che dice:
\[
  cos\theta = \frac{e^{j \theta} + e^{-j \theta}}{2}
\] 
\[
  sin\theta = \frac{e^{j \theta} - e^{-j \theta}}{2j}
\]
\begin{figure}[H]
  \begin{definition}[Complesso cognugato]
    A ogni numero complesso è associato un cognugato che ha la stessa parte reale,
    ma parte immaginaria opposta.
  \end{definition}
\end{figure}
\label{07-10-D1}
\[
  \color{blue} S = \rho (cos(\theta) + j \sin(\theta)) \color{black}
\] 
\[
  \color{red} \bar{S} = \rho (cos(-\theta) + j \sin(-\theta)) \color{black}
\]
\section{Funzioni in C}
\subsection{Funzione a variabili complesse}
\begin{figure}[H]
  \begin{definition}[Funzione a variabile complessa]
    Una funzione a variabile complessa è una funzione che ha come dominio un insieme
    di numeri complessi e come codominio un insieme di numeri complessi.
  \end{definition}
\end{figure}

\begin{figure}[H]
  \begin{definition}[Punto interno]
    Un punto \( S_0 \) appartenente a un intorno \( D(f) \subseteq \mathbb{C} \) è
    interno a \( D(f) \) se e solo se esiste un disco \( B_\rho(S_0) \)  di raggio \( \rho \in \mathbb{R}_+ \) 
    centrato in \( S_0 \) tale che:
    \[
      B_\rho(S_0) \subset D(f)
    \] 
    Quindi \( D(f) \) è un dominio e \( B_\rho(S_0) \) è un sottoinsieme:
    \[
      B_\rho(S_0) = \left\{ S \in \mathbb{C} \;|\; \|S_0 - S\| < \rho \right\}
    \] 
    \begin{figure}[H]
      \centering
      \begin{tikzpicture}
        \def\r{1}
        \def\s{(2,2)}
        \def\d{\r * 2.5}

        \draw[->] (-2,0) -- (5,0) node[right] {$\mathbb{C}$};
        \draw[->] (0,-2) -- (0,5) node[above] {};
        \draw[green] \s circle (\r);
        \draw[blue] \s ++(-0.5,-0.7) circle (\d);
        \draw[fill=black] \s circle (0.05) node[above right] {$S_0$};
        \draw[fill=black] \s ++(\r,0) circle (0.05) node[above right] {$S$};
        \draw[<->] \s -- ++(\r,0) node[midway, below] {$\rho$};
        \node[green] at (2,1) [below] {$B_\rho(S_0)$};
        \node[blue] at (1.5,3.8) [above] {$D(f)$};
      \end{tikzpicture}
    \end{figure}
  \end{definition}
\end{figure}

\begin{definition}[Insieme aperto]
  È l'insieme di tutti i punti che sono definiti interni.

  \vspace{1em}
  \noindent
  Ad esempio:
  \begin{itemize}
    \item insieme \( \mathbb{C} \) 
    \item insieme \( \emptyset \) 
    \item i dischi in un punto \( S_0 \), \( B_{\rho}(S_0) = \left\{ S \in \mathbb{C} \;|\; \|S_0 - S\| < \rho \right\} \) 
    \item corone circolari centrate in un punto \( x \) ,
      \[
        \; \left\{ S \in \mathbb{C},\;
          \rho_1, \rho_2 \in \mathbb{R} \;|\;
        \rho_1 <  |S-x| < \rho_2\right\}
      \]
      \begin{figure}[H]
        \centering
        \begin{tikzpicture}
          \def\r{1}
          \def\s{(1,3)}
          \def\d{1.5}
          \def\x{(3.5,1.5)}

          \draw[->] (-2,0) -- (5,0) node[right] {$\mathbb{C}$};
          \draw[->] (0,-2) -- (0,5) node[above] {};
          \draw[green] \s circle (\r);
          \draw[blue] \x circle (\d);
          \draw[blue] \x circle (\d/2);
          \draw[fill=black] \s circle (0.05) node[above right] {$S_0$};
          \draw[fill=black] \x circle (0.05) node[above right] {$x$};
          \draw[<->] \x -- ++(\d/2,0) node[midway, below] {$\rho_1$};
          \draw[<->] \x ++(\d/2,0) -- ++(\d/2,0) node[midway, below] {$\rho_2$};
          \node[green] at (1,3.6 + \r) [below] {$B_\rho(S_0)$};
        \end{tikzpicture}
      \end{figure}
    \item semipiani destri o sinistri, superiori o inferiori
      \begin{figure}[H]
        \centering
        \begin{tikzpicture}
          \def\pdx{1.5}
          \def\psx{-1.5}

          \def\psp{1.5}
          \def\pif{-1.5}


          \def\x{3}
          \def\y{3}
          \draw[->] (-\x,0) -- (\x,0) node[right] {$\Re$};
          \draw[->] (0,-\y) -- (0,\y) node[above] {$\Im$};

          \draw[red,dashed] (\psx,\y) -- (\psx,-\y);
          \draw[red,fill,opacity=0.2] (-\x,\y) rectangle (\psx,-\y);

          \draw[blue,dashed] (\pdx,\y) -- (\pdx,-\y);
          \draw[blue,fill,opacity=0.2] (\pdx,\y) rectangle (\x,-\y);


          \draw[green,dashed] (-\x,\psp) -- (\x,\psp);
          \draw[green,fill,opacity=0.2] (-\x,\psp) rectangle (\x,\y);

          \draw[cyan,dashed] (-\x,\pif) -- (\x,\pif);
          \draw[cyan,fill,opacity=0.2] (-\x,-\y) rectangle (\x,\pif);

          \node[red,left,align=right] at (-\x,0) {$\Pi_1$\\Semipiano\\Sinistro};
          \node[blue,right,xshift=0.5cm,align=left] at (\x,0) {$\Pi_2$\\Semipiano\\Destro};
          \node[green,above,yshift=0.5cm,align=center] at (0,\y) {Semipiano\\Superiore\\$\Pi_3$};
          \node[cyan,below,align=center] at (0,-\y) {$\Pi_4$\\Semipiano\\Inferiore};
        \end{tikzpicture}
      \end{figure}
  \end{itemize}
\end{definition}

\subsection{Funzioni complesse}
\[
f: D(f) \to \mathbb{C} \quad D(f) \subseteq \mathbb{C} \text{ e aperto}
\] 
Alcuni esempi sono:
\begin{itemize}
  \item \( S \to S \quad D(f) = \mathbb{C} \) 
  \item \( S \to S^2 \quad D(f) = \mathbb{C} \) 
  \item \( S \to \Re(S) + j \Im(S)^2 \quad D(f) = \mathbb{C} \) 
  \item \( S \to \sum_{k=0}^{n} a_k^S \quad a_k \in \mathbb{C};\;\; k,\;n \in \mathbb{Z} \)
    (Funzioni polinomiali)
  \item \( \sum_{k=0}^{\infty} a_k (S - S_0)^k \quad a_k \in \mathbb{C};\;\; S,\;S_0 \in \mathbb{C};\;\; k \in \mathbb{Z}\) 
    (Serie di potenze)
\end{itemize}

\subsection{Funzioni polinomiali}
\[
  P(s) = \sum_{k=0}^{n} a_k \cdot \underbrace{S^k}_{\text{Variabile complessa}}
\] 
Con \( n=2 \):
\[
a_0 S^0 + a_1 S^1 + a_2 S^2
\] 
Con \( n=3 \):
\[
a_0 S^0 + a_1 S^1 + a_2 S^2 + a_3 S^3
\] 
\subsubsection{Risoluzione}
Per risolvere una funzione polinomiale si usano le solite tecniche, ad esempio:
\[
S^2-2S+1 = (S-1)^2
\] 
Che ha una sola soluzione, ma con molteplicità 2.

\begin{figure}[H]
  \begin{theorem}[Teorema fondamentale delle radici]
    Ogni polinomio \( P(S) \) a coefficienti complessi di grado \( n > 0 \) ha
    \( n \) \textbf{radici complesse} ed è decomponibile in un solo modo
    \[
      P(s) = a_n \prod_{r=1}^{r} (s - s_r)^{\mu_r}
    \] 
    Dove:

    \noindent
    \( s_r \) sono delle radici

    \noindent \( \mu_r \) sono le molteplicità delle radici

    \noindent \( a_n \) è il coefficiente del polinomio

    \vspace{1em}
    \noindent
    \( \sum_{r=1}^{\infty} \mu_r = n \) 
  \end{theorem}
\end{figure}

\section{Segnali}
Il segnale più presente è quello \textbf{sinusoidale}.
\subsection{Segnali continui}
\[
t \in \mathbb{R}\;\; \mathbb{R} \to \mathbb{C};\;\; \mathbb{R} \to \mathbb{R}
\] 

\subsubsection{Sinusoidale}
\[
x(t) = A \cdot cos(\omega_0 t + \phi) 
\] 
Dove:
\begin{itemize}
  \item \( A \) è l'ampiezza
  \item \( \omega \) è la frequenza \( f = \frac{1}{T} \) (\( T = \text{ periodo} \) )
  \item \( \phi \) è la fase
\end{itemize}

\begin{figure}[H]
  \centering
  \begin{tikzpicture}
    \draw[->] (-3,0) -- (3,0) node[right] {$k$};
    \draw[->] (0,-1.5) -- (0,1.5) node[above] {$x(k)$};
    \draw[blue,domain=-3:3,samples=100] plot ({\x},{cos(\x*5 r)});

    \draw (1.25,1) -- ++(0,0.2) -- ++(1.25,0) node[midway,above] {$T_0 = \frac{2\pi}{\omega_0}$} -- ++(0,-0.2);

    \draw[fill] (-2.3,0.5) circle (1pt) node[above right, xshift=-2, scale=0.8] {$A cos\phi$};
  \end{tikzpicture}
  \caption{Funzione sinusoidale}
\end{figure}

\begin{itemize}
  \item \textbf{Periodico}
    \[
      x(t) = x(t + T_0)
    \] 
    \[
      A cos(\omega_0 t + \phi) = A cos(\omega_0 t + \omega_0 T_0 + \phi)
    \] 
    \[
      T_0 = \frac{2 \pi m}{\omega_0} \to \text{ periodo } \frac{2\pi}{\omega_0} \quad m \in \mathbb{R} \text{ (indica il multiplo)}
    \] 
    Il sistema si comporta nello stesso modo per ogni periodo.

  \item \textbf{Shift temporale} equivale ad un cambio di fase
    \[
      A cos[\omega_0 (t - t_0)] = A cos(\omega_0t + \underbrace{\omega_0t_0}_{\Delta \phi})
    \] 
    \[
      A cos[\omega_0 (t+t_0) + \phi] = A cos(\omega_o t + \omega_0 t_0 + \phi)
    \] 

  \item Se \( \phi = 0 \) Il segnale assume la seguente forma:
    \[
    A = cos(\omega_0 t)
    \] 
    \begin{figure}[H]
      \centering
      \begin{tikzpicture}
        \draw[->] (-3,0) -- (3,0) node[right] {$k$};
        \draw[->] (0,-1.5) -- (0,1.5) node[above] {$x(k)$};
        \draw[blue,domain=-3:3,samples=100] plot ({\x},{cos(\x*3 r)});

        \draw[blue,fill] (0,1) circle (1pt) node[above right, blue] {$A$};
        \draw[blue,fill] (0,-1) circle (1pt) node[right, blue] {$-A$};
      \end{tikzpicture}
      \caption{Funzione coseno}
    \end{figure}

    e ha le seguenti proprietà:
    \begin{itemize}
      \item Periodico
        \[
        T_0 = \frac{2\pi}{\omega_o}
        \] 
      \item Pari
        \[
          x(t) = x(-t)
        \] 
    \end{itemize}
    Se \( \phi = -\frac{\pi}{2} \) Il segnale assume la seguente forma:
    \[
    x(t) = \begin{cases}
      A cos(\omega_0 t - \frac{\pi}{2})\\
      A sin(\omega_0 t)\\
      A cos(\omega_0 (t + \frac{T_0}{4}))
    \end{cases}
    \] 
    \begin{figure}[H]
      \centering
      \begin{tikzpicture}
        \draw[->] (-3,0) -- (3,0) node[right] {$k$};
        \draw[->] (0,-1.5) -- (0,1.5) node[above] {$x(k)$};
        \draw[blue,domain=-3:3,samples=100] plot ({\x},{sin(\x*3 r)});
      \end{tikzpicture}
      \caption{Funzione seno}
    \end{figure}
    e ha le seguenti proprietà:
    \begin{itemize}
      \item Periodico
        \[
        T_0 = \frac{2\pi}{\omega_0}
        \] 
      \item Dispari
        \[
        x(t) = -x(-t)
        \] 
    \end{itemize}
\end{itemize}

\subsubsection{Esponenziali reali}
\[
  x(t) = C e^{\alpha t}
\] 
\begin{figure}[H]
  \centering
  \begin{tikzpicture}
    \draw[->] (-3,0) -- (3,0) node[right] {$t$};
    \draw[->] (0,-0.2) -- (0,3) node[above] {$x(t)$};
    \draw[domain=-3:1.1,smooth,variable=\x,blue] plot ({\x},{exp(\x)}) node[above right, scale=0.9] {$e^{\alpha t} \;\; (\alpha > 0)$};
    \draw[fill] (0,1) circle (0.03) node[right] {C};
  \end{tikzpicture}
  \caption{Esempo di segnale}
\end{figure}

\begin{figure}[H]
  \centering
  \begin{tikzpicture}
    \draw[->] (-3,0) -- (3,0) node[right] {$t$};
    \draw[->] (0,-0.2) -- (0,3) node[above] {$x(t)$};
    \draw[domain=-1.1:3,smooth,variable=\x,cyan] plot ({\x},{exp(-\x)}) node[above right, scale=0.9] {$e^{\alpha t} \;\; (\alpha < 0)$};
    \draw[fill] (0,1) circle (0.03) node[right] {C};
  \end{tikzpicture}
  \caption{Esempo di segnale}
\end{figure}

\begin{itemize}
  \item \textbf{Shift Temporale}
    \[
      C e^{\alpha(t+t_0)} = C e^{\alpha t} \cdot \underbrace{C e^{\alpha t_0}}_{\mathbb{R}}
    \] 
\end{itemize}

\subsubsection{Esponenziali complessi}
\[
  x(t) = C e^{\alpha t}
\] 
\[
  C = |c| e^{j \Theta} \text{ (polare)}
\] 
\[
  \alpha = r + j \omega_o \text{ (cartesiana)}
\] 
\vspace{1em}
\[
  x(t) = |c| e^{j \Theta} \cdot e^{(r + j \omega_0)t} =
\] 
\[
 = |c| e^{rt} \cdot e^{j(\omega_0 t + \Theta)} \text{ (Eulero)}
\] 
\begin{definition}[Eulero]
  La formula di Eulero è la seguente:
  \[
    e^{j(\omega_0 t + \Theta)} = cos(\omega_0 t + \Theta) + j sin(\omega_0 t + \Theta)
  \] 
\end{definition}

\[
  x(t) = |c| e^{rt} \cdot (cos(\omega_0 t + \Theta) + j sin(\omega_0 t + \Theta))
\] 

\begin{figure}[H]
  \centering
  \begin{tikzpicture}
    \def\sigm{0.35}
    \def\omeg{3.3}
    \draw[->] (-4,0) -- (4,0) node[right] {$n$};
    \draw[->] (0,-3.5) -- (0,4) node[above] {$A$};
    \node at (-3,4) {$Ae^{\alpha n} \cdot cos(\omega_0 n + \phi)$};

    \draw[domain=-4:4,smooth,variable=\x,red] plot ({\x},{exp(\sigm * \x)}) node[above right, scale=0.9,align=center] {Inviluppo};
    \draw[domain=-4:4,smooth,variable=\x,red] plot ({\x},{-exp(\sigm * \x)}) node[above right, scale=0.9,align=center] {};
    \draw[domain=-4:4,smooth,variable=\x,blue,samples=70] plot ({\x},{exp(\sigm*\x)*cos(\omeg*\x r)}) node[above right, scale=0.9,align=center] {$r>0$};
  \end{tikzpicture}
  \caption{Esponenziale sinusoidale continuo con \( r > 0 \)}
\end{figure}
\begin{figure}[H]
  \centering
  \begin{tikzpicture}
    \def\sigm{-0.35}
    \def\omeg{3.3}
    \draw[->] (-4,0) -- (4,0) node[right] {$n$};
    \draw[->] (0,-3.5) -- (0,4) node[above] {$A$};
    \node at (-3,4) {$Ae^{\alpha n} \cdot cos(\omega_0 n + \phi)$};

    \draw[domain=-4:4,smooth,variable=\x,red] plot ({\x},{exp(\sigm * \x)}) node[above right, scale=0.9,align=center] {Inviluppo};
    \draw[domain=-4:4,smooth,variable=\x,red] plot ({\x},{-exp(\sigm * \x)}) node[above right, scale=0.9,align=center] {};
    \draw[domain=-4:4,smooth,variable=\x,blue,samples=70] plot ({\x},{exp(\sigm*\x)*cos(\omeg*\x r)}) node[above right, scale=0.9,align=center] {$r>0$};
  \end{tikzpicture}
  \caption{Esponenziale sinusoidale continuo con \( r < 0 \)}
\end{figure}



\subsection{Segnali discreti}
\[
k \in \mathbb{Z}\;\; \mathbb{Z} \to \mathbb{Z};\;\; \mathbb{Z} \to \mathbb{R};\;\; \mathbb{Z} \to \mathbb{C}
\] 
\subsubsection{Sinusoidale}
\[
  x[n] = A cos(\Omega_0 n + \phi)
\] 
Dove:
\begin{itemize}
  \item \( A \) è l'ampiezza
  \item \( \Omega_0 \) è la frequenza \( f = \frac{\Omega_0}{2\pi} \) 
  \item \( \phi \) è la fase
\end{itemize}
\begin{figure}[H]
  \centering
  \begin{tikzpicture}
    \draw[->] (-3,0) -- (3,0) node[right] {$k$};
    \draw[->] (0,-1.5) -- (0,1.5) node[above] {$x(k)$};
    \draw[domain=-3:3,smooth,variable=\x,red,thin,samples=100] plot ({\x},{sin(\x*5 r)}) node[above right, red] {Inviluppo};
    \foreach \x in {-3,-2.8,...,3}
    \draw[blue,thick] (\x,0) -- (\x,{sin(\x*5 r)});
  \end{tikzpicture}
  \caption{Esempio di segnale sinusoidale discreto}
\end{figure}

\begin{itemize}
  \item \textbf{Shift temporale} equivale ad un cambio di fase
    \[
      A cos[\Omega_0 (n + n_0)] = A cos(\Omega_0 n + \underbrace{\Omega_0 n_0}_{\Delta \phi})
    \] 

  \item Se \( \phi = 0 \) Il segnale rimane coseno ed è pari
    \[
      x[n] = x[-n]
    \] 
    Se \( \phi = -\frac{\pi}{2} \) Il segnale diventa un seno ed è dispari:
    \[
      x[n] = -x[-n]
    \] 
\end{itemize}

\begin{itemize}
  \item Il cambio di fase equivale allo shift temporale?
    \[
      A cos[\Omega_0 (n + n_0)] \stackrel{?}{=} A cos[\Omega_0 n + \Omega_0 \omega_0 \phi]
    \] 
    dove \( \phi \) è il rapporto tra \( \phi \) e \( \Omega_0 \) è un numero intero:
    \[
    \phi = \Omega_0 \omega_0
    \] 
    \[
    \frac{\phi}{\Omega_0} = n_0
    \] 
    \begin{figure}[H]
      \begin{example}
        \begin{figure}[H]
          \centering
          \begin{tikzpicture}
            \draw[->] (-3,0) -- (3,0) node[right] {$k$};
            \draw[->] (0,-1.5) -- (0,1.5) node[above] {$x(k)$};
            \draw[blue,domain=-3:3,samples=100] plot ({\x},{sin(\x*3 r)});
            \draw[blue,domain=-3:3] plot[ycomb] ({\x},{sin(\x*3 r)});
          \end{tikzpicture}
          \caption{Calcolo del cambio di fase}
        \end{figure}
        \[
          \phi = \frac{2\pi}{12}
        \] 
        \[
        \Omega_0 = ?
        \] 
        \[
          \frac{2\pi}{\Omega_0} = \frac{2\pi}{12} = 12 = \Omega_0 \quad (m = 1)
        \] 
      \end{example}
    \end{figure}

    \begin{figure}[H]
      \begin{example}
        \[
          \phi = \frac{1}{6} \to \frac{2\pi}{\Omega_0} = \frac{1}{6} = \Omega_0 = \frac{1}{3}\pi
        \] 
      \end{example}
    \end{figure}
\end{itemize}

\subsubsection{Esponenziali Reali}
\[
  x[n] = C e^{\beta n} = C \alpha^n \quad \alpha = e^\beta;\;\; C,\;\alpha \in \mathbb{R}
\] 
\begin{figure}[H]
  \centering
  \begin{tikzpicture}
    \def\sigm{0.35}
    \def\omeg{3.3}
    \draw[->] (-4,0) -- (4,0) node[right] {$n$};
    \draw[->] (0,-0.2) -- (0,4) node[above] {$A$};

    \draw[domain=-4:4,smooth,variable=\x,blue,samples=50] plot[ycomb] ({\x},{exp(\sigm * \x)}) node[below right,scale=0.9,align=center] {$\alpha > 0$\\$|\alpha| > 1$};
    \draw[domain=-4:4,smooth,variable=\x,red] plot ({\x},{exp(\sigm * \x)}) node[above right, scale=0.9,align=center] {Inviluppo};
  \end{tikzpicture}
  \caption{Esponenziale reale con \( \alpha > 0 \; |\alpha| > 1 \)}
\end{figure}
\begin{figure}[H]
  \centering
  \begin{tikzpicture}
    \def\sigm{0.35}
    \def\omeg{3.3}
    \draw[->] (-4,0) -- (4,0) node[right] {$n$};
    \draw[->] (0,-3.5) -- (0,4) node[above] {$A$};
    \node at (-3,4) {$Ae^{\alpha n} \cdot cos(\omega_0 n + \phi)$};

    \draw[domain=-4:4,smooth,variable=\x,red] plot ({\x},{exp(\sigm * \x)}) node[above right, scale=0.9,align=center] {Inviluppo};
    \draw[domain=-4:4,smooth,variable=\x,red] plot ({\x},{-exp(\sigm * \x)}) node[above right, scale=0.9,align=center] {};
    \draw[domain=-4:4,smooth,variable=\x,blue,samples=70] plot[ycomb] ({\x},{exp(\sigm*\x)*cos(\omeg*\x r)}) node[above right, scale=0.9,align=center] {$\alpha > 0$\\$|\alpha| < 1$};
  \end{tikzpicture}
  \caption{Esponenziale sinusoidale con \( \alpha > 0 \; |\alpha| < 1 \)}
\end{figure}

\begin{figure}[H]
  \centering
  \begin{tikzpicture}
    \def\sigm{-0.35}
    \def\omeg{3.3}
    \draw[->] (-4,0) -- (4,0) node[right] {$n$};
    \draw[->] (0,-0.2) -- (0,4) node[above] {$A$};

    \draw[domain=-4:4,smooth,variable=\x,blue,samples=50] plot[ycomb] ({\x},{exp(\sigm * \x)}) node[below right,scale=0.9,align=center,xshift=0.3cm] {$\alpha > 0$\\$|\alpha| > 1$};
    \draw[domain=-4:4,smooth,variable=\x,red] plot ({\x},{exp(\sigm * \x)}) node[above right, scale=0.9,align=center] {Inviluppo};
  \end{tikzpicture}
  \caption{Esponenziale reale con \( \alpha > 0 \; |\alpha| < 1 \)}
\end{figure}
\begin{figure}[H]
  \centering
  \begin{tikzpicture}
    \def\sigm{-0.35}
    \def\omeg{3.3}
    \draw[->] (-4,0) -- (4,0) node[right] {$n$};
    \draw[->] (0,-3.5) -- (0,4) node[above] {$A$};
    \node at (3,4) {$Ae^{\alpha n} \cdot cos(\omega_0 n + \phi)$};

    \draw[domain=-4:4,smooth,variable=\x,red] plot ({\x},{exp(\sigm * \x)}) node[above right, scale=0.9,align=center] {};
    \draw[domain=-4:4,smooth,variable=\x,red] plot ({\x},{-exp(\sigm * \x)}) node[below right, scale=0.9,align=center] {Inviluppo};
    \draw[domain=-4:4,smooth,variable=\x,blue,samples=70] plot[ycomb] ({\x},{exp(\sigm*\x)*cos(\omeg*\x r)}) node[above right, scale=0.9,align=center] {$\alpha < 0$\\$|\alpha| < 1$};
  \end{tikzpicture}
  \caption{Esponenziale sinusoidale con \( \alpha < 0 \; |\alpha| < 1 \)}
\end{figure}

\subsection{Rappresentazione}
I segnali che appartengono all'insieme dei complessi sono tridimensionali e sono
rappresentati in 2 assi cartesiani diversi, che mostrano lo stesso segnale, ma
da 2 punti di vista differenti. Il segnale viene generato da un punto che  
che continua a girare periodicamente su una circonferenza proiettando su entrambi
gli assi il suo valore.
\begin{figure}[H]
  \centering
  \begin{tikzpicture}
    \draw[->] (-3,0) -- (3,0) node[right] {$\Re$};
    \draw[->] (0,-3) -- (0,3) node[above] {$\Im$};

    \draw[cyan] (0,0) circle (2.5);
    
    \draw[->,red,rotate=30] (0,0) -- (2.5,0) node[midway, above] {};
    \draw[red] (1.5,0) arc (0:30:1.5) node[midway, right] {$\Theta$};
    \draw[->,red] (2,0) arc (0:360:2) node[midway, right] {};
  \end{tikzpicture}
  \caption{Circonferenza di proiezione}
\end{figure}

La proiezione sull'asse dei reali è la parte reale del segnale:
\begin{figure}[H]
  \centering
  \begin{tikzpicture}
    \draw[->] (-0.2,0) -- (5,0) node[right] {$t$};
    \draw[->] (0,-1.5) -- (0,1.5) node[above] {$\Re$};

    \draw[blue,domain=0:5,smooth,samples=100] plot ({\x},{cos(\x*3 r)});
  \end{tikzpicture}
  \caption{Proiezione su asse reale}
\end{figure}

La proiezione sull'asse degli immaginari è la parte immaginaria del segnale ed è
traslata di \( \frac{\pi}{2} \) rispetto alla parte reale:
\begin{figure}[H]
  \centering
  \begin{tikzpicture}
    \draw[->] (-0.2,0) -- (5,0) node[right] {$t$};
    \draw[->] (0,-1.5) -- (0,1.5) node[above] {$\Im$};

    \draw[red,domain=0:5,smooth,samples=100] plot ({\x},{sin(\x*3 r)});
  \end{tikzpicture}
  \caption{Proiezione su asse immaginario}
\end{figure}

In 3 dimensioni diventa:
\begin{figure}[H]
  \centering
  \begin{tikzpicture}
    \begin{axis}[
      axis lines = center,
      xlabel = \( \Re \),
      ylabel = \( t \),
      zlabel = \( \Im \),
      ymin = 0,
      ymax = 15,
      view={45}{30}
      ]


      \addplot3[
        domain=0:25,
        cyan,
        samples=500,
        samples y=1,
        ] (
        {cos(deg(x))},
        {4*x/(2*pi)},
        {sin(deg(x))}
        );
    \end{axis}
  \end{tikzpicture}
  \caption{Circonferenza di proiezione in 3D}
\end{figure}

\end{document}
