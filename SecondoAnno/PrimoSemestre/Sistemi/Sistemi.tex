\documentclass[a4paper]{article}

\usepackage[utf8]{inputenc}
\usepackage[T1]{fontenc}
\usepackage{textcomp}
\usepackage[italian]{babel}
\usepackage{amsmath, amssymb}
\usepackage{booktabs,xltabular}
\usepackage{amsfonts}
\usepackage{cancel}
\usepackage{mdframed}
\usepackage{makecell}
\usepackage{float}
\usepackage{xcolor}
\usepackage{listings}
\usepackage{graphicx}
\usepackage{tikz}
\usetikzlibrary{shapes, arrows, automata, petri, decorations.pathreplacing, positioning, calc}
\usepackage{circuitikz}
\usepackage[label=corner]{karnaugh-map}
\graphicspath{{./figures/}}

\usepackage{ntheorem}
\newtheorem{theorem}{Teorema}

\usepackage{import}
\usepackage{pdfpages}
\usepackage{transparent}
\usepackage{xcolor}

\usepackage{hyperref}
\hypersetup{
    colorlinks=false,
}

% Code blocks
\definecolor{codegreen}{rgb}{0,0.6,0}
\definecolor{codegray}{rgb}{0.5,0.5,0.5}
\definecolor{codepurple}{rgb}{0.58,0,0.82}
\definecolor{backcolour}{rgb}{0.95,0.95,0.95}

\lstdefinestyle{mystyle}{
	backgroundcolor=\color{backcolour},
	commentstyle=\color{codegreen},
	keywordstyle=\color{magenta},
	numberstyle=\tiny\color{codegray},
	stringstyle=\color{codepurple},
	basicstyle=\ttfamily\footnotesize,
	breakatwhitespace=false,
	breaklines=true,
	captionpos=b,
	keepspaces=true,
	numbers=left,
	numbersep=5pt,
	showspaces=false,
	showstringspaces=false,
	showtabs=false,
	tabsize=2
}

\lstset{style=mystyle}

\usepackage{color}

\definecolor{dkgreen}{rgb}{0,0.6,0}
\definecolor{gray}{rgb}{0.5,0.5,0.5}
\definecolor{mauve}{rgb}{0.58,0,0.82}

\lstset{frame=tb,
	aboveskip=3mm,
	belowskip=3mm,
	showstringspaces=false,
	columns=flexible,
	basicstyle={\small\ttfamily},
	numbers=none,
	numberstyle=\tiny\color{gray},
	keywordstyle=\color{blue},
	commentstyle=\color{dkgreen},
	stringstyle=\color{mauve},
	breaklines=true,
	breakatwhitespace=true,
	tabsize=3
}

\usepackage{import}
\usepackage{pdfpages}
\usepackage{transparent}
\usepackage{xcolor}



% Useful definitions frame
\theoremstyle{break}
\theoremheaderfont{\bfseries}
\newmdtheoremenv[%
	linecolor=gray,leftmargin=0,%
	rightmargin=0,
	innertopmargin=8pt,%
	innerbottommargin=8pt,
	ntheorem]{define}{Definizioni utili}[section]

% Example frame
\theoremstyle{break}
\theoremheaderfont{\bfseries}
\newmdtheoremenv[%
	linecolor=gray,leftmargin=0,%
	rightmargin=0,
	innertopmargin=8pt,%
	innerbottommargin=8pt,
	ntheorem]{example}{Esempio}[section]

% Important definition frame
\theoremstyle{break}
\theoremheaderfont{\bfseries}
\newmdtheoremenv[%
	linecolor=gray,leftmargin=0,%
	rightmargin=0,
	backgroundcolor=gray!40,%
	innertopmargin=8pt,%
	innerbottommargin=8pt,
	ntheorem]{definition}{Definizione}[section]

% Exercise frame
\theoremstyle{break}
\theoremheaderfont{\bfseries}
\newmdtheoremenv[%
	linecolor=gray,leftmargin=0,%
	rightmargin=0,
	innertopmargin=8pt,%
	innerbottommargin=8pt,
	ntheorem]{exercise}{Esercizio}[section]

% figure support
\usepackage{import}
\usepackage{xifthen}
\pdfminorversion=7
\usepackage{pdfpages}
\usepackage{transparent}
\newcommand{\incfig}[1]{%
	\def\svgwidth{\columnwidth}
	\import{./figures/}{#1.pdf_tex}
}

% FSM tikz
\tikzset{
    place/.style={
        circle,
        thick,
        draw=black,
        minimum size=6mm,
    },
        state/.style={
        circle,
        thick,
        draw=blue!75,
        fill=blue!20,
        minimum size=6mm,
    },
}

\pdfsuppresswarningpagegroup=1

\begin{document}

\begin{titlepage}
	\begin{center}
		\vspace*{1cm}

		\Huge
		\textbf{Analisi 1}

		\vspace{0.5cm}
		\LARGE
		UniVR - Dipartimento di Informatica

		\vspace{1.5cm}

		\textbf{Fabio Irimie}

		\vfill


		\vspace{0.8cm}

    Corso di Giacomo Canevari

		1° Semestre 2023/2024

	\end{center}
\end{titlepage}


\tableofcontents
\pagebreak

% Info:
% Esame: 4 esercizi. Si possono portare appunti, solo scritti a mano e non troppi.
% Libro: Segnali e sistemi (Giuseppe Ricci & Maria Elena Valeher)
% Libro2: Signals & systems second edition (Alan V. Oppenheim & Alan S. Willsky)
\section{Concetti base}
Un sistema è formato da \textbf{segnali trasmessi}, un'esempio di segnale
è la voce che usiamo per comunicare tra di noi. Il sistema prende le informazioni
ricevute dal segnale e le rielabora. 

\noindent 
Degli esempi di sistema sono:
\begin{itemize}
  \item Microfono-Casse
  \item Freno della macchina
\end{itemize}

\subsection{Tipi di segnali}
I segnali possono essere di due tipi:

\begin{itemize}
  \item \textbf{Segnali a tempo continuo}: Segnali che hanno infiniti punti per ogni
    infinitesimo di tempo.
    \begin{figure}[H]
      \centering
      \begin{tikzpicture}
        \draw[->] (-1,0) -- (5,0) node[right] {$t$};
        \draw[->] (0,-1.5) -- (0,1.5) node[above] {$x(t)$};
        \draw[domain=0:4.5,smooth,variable=\x,blue] plot ({\x},{sin(\x*5 r)});
      \end{tikzpicture}
      \caption{Esempio di segnale a tempo continuo}
    \end{figure}
  \item \textbf{Segnali a tempo discreto}: Segnali che hanno un numero finito di punti
    per ogni intervallo di tempo.
    \begin{figure}[H]
      \centering
      \begin{tikzpicture}
        \draw[->] (-1,0) -- (5,0) node[right] {$n$};
        \draw[->] (0,-1.5) -- (0,1.5) node[above] {$x[n]$};
        \draw[domain=0:4.5,smooth,variable=\x,white] plot ({\x},{sin(\x*5 r)});
        \foreach \x in {0,0.2,...,4.5}
        \draw[blue,thick] (\x,0) -- (\x,{sin(\x*5 r)});
      \end{tikzpicture}
      \caption{Esempio di segnale a tempo discreto}
    \end{figure}

\end{itemize}

\noindent
Per elaborare i dati attraverso un computer bisogna convertire un segnale continuo
in uno discreto, questo processo è chiamato \textbf{campionamento}.
Una volta campionato il segnale si deve \textbf{quantizzare}, ovvero approssimare
il valore del segnale a un valore discreto. Infine si fa \textbf{encoding}, ovvero
si codifica il segnale per poterlo adattare ad un altro tipo di segnale.

\noindent
I segnali possono essere di dimensioni diverse, ad esempio:
\begin{itemize}
  \item L'andamento di una borsa è un segnale a 1 dimensione.
  \item Una foto in bianco e nero è un segnale a 2 dimensioni \( (x,y) \).
  \item Una foto colorata è un segnale multidimensionale \( (x,y)^3 \) per
    rappresentare ogni colore (R,G,B).
\end{itemize}

\subsection{Rappresentazione dei sistemi}
Un sistema lo rappresentiamo con un blocco, dove all'ingresso mettiamo il segnale
in ingresso e all'uscita il segnale in uscita.
\begin{figure}[H]
  \centering
  \begin{tikzpicture}
    \node[draw,minimum width=1cm,minimum height=1cm] (A) at (0,0) {\( \Sigma \) };
    \draw[->] (-1,0) -- (A) node[midway,above left] {Ingresso};
    \draw[->] (A) -- (1,0) node[midway,above right] {Uscita};
  \end{tikzpicture}
  \caption{Rappresentazione di un sistema}
\end{figure}

\noindent Per gestire un segnale si possono usare anche più sistemi che gestiscono
input e output per ottenere un risultato finale.

\end{document}
