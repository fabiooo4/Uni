\documentclass[a4paper]{article}
\usepackage{import}
\usepackage[utf8]{inputenc}
\usepackage[T1]{fontenc}
\usepackage{textcomp}
\usepackage[italian]{babel}
\usepackage{amsmath, amssymb}
\usepackage{booktabs,xltabular}
\usepackage{amsfonts}
\usepackage{subcaption}
\usepackage{amsthm}
\usepackage{cancel}
\usepackage{mdframed}
\usepackage{makecell}
\usepackage{float}
\usepackage{xcolor}
\usepackage{listings}
\usepackage{gensymb}
\usepackage{graphicx}
\usepackage{bodeplot}
\usepackage{physics}
\usepackage{tikz}
\usetikzlibrary{shapes, arrows, automata, petri, decorations.markings, decorations.pathreplacing, positioning, calc, quotes}
\usepackage{circuitikz}
\usepackage[label=corner]{karnaugh-map}
\graphicspath{{./figures/}}

% Set default font to sans-serif
\renewcommand{\familydefault}{\sfdefault} 
\usepackage{eulervm}

\usepackage{forest}

\usepackage{mathtools}
\DeclarePairedDelimiter\ceil{\lceil}{\rceil}
\DeclarePairedDelimiter\floor{\lfloor}{\rfloor}

% \usepackage{ntheorem}

\usepackage{import}
\usepackage{pdfpages}
\usepackage{transparent}
\usepackage{xcolor}

\usepackage{hyperref}
\hypersetup{
    colorlinks=false,
}

% Code blocks
\definecolor{codegreen}{rgb}{0,0.6,0}
\definecolor{codegray}{rgb}{0.5,0.5,0.5}
\definecolor{codepurple}{rgb}{0.58,0,0.82}
\definecolor{backcolour}{rgb}{0.95,0.95,0.95}

\lstdefinestyle{mystyle}{
	backgroundcolor=\color{backcolour},
	commentstyle=\color{codegreen},
	keywordstyle=\color{magenta},
	numberstyle=\tiny\color{codegray},
	stringstyle=\color{codepurple},
	basicstyle=\ttfamily\footnotesize,
	breakatwhitespace=false,
	breaklines=true,
	captionpos=b,
	keepspaces=true,
	numbers=left,
	numbersep=5pt,
	showspaces=false,
	showstringspaces=false,
	showtabs=false,
	tabsize=2
}

\lstset{style=mystyle}

\usepackage{color}
\usepackage{import}
\usepackage{pdfpages}
\usepackage{transparent}
\usepackage{xcolor}

% Example frame
\theoremstyle{definition}
\newmdtheoremenv[%
	linecolor=gray,leftmargin=0,%
	rightmargin=0,
	innertopmargin=8pt,%
	innerbottommargin=8pt,
	ntheorem]{example}{Esempio}[section]

% Important definition frame
\theoremstyle{definition}
\newmdtheoremenv[%
	linecolor=gray,leftmargin=0,%
	rightmargin=0,
	backgroundcolor=gray!40,%
	innertopmargin=8pt,%
	innerbottommargin=8pt,
	ntheorem]{definition}{Definizione}[section]

% Exercise frame
\theoremstyle{definition}
\newmdtheoremenv[%
	linecolor=gray,leftmargin=0,%
	rightmargin=0,
	innertopmargin=8pt,%
	innerbottommargin=8pt,
	ntheorem]{exercise}{Esercizio}[section]

% Theorem frame
\theoremstyle{definition}
\newmdtheoremenv[%
  linecolor=gray,leftmargin=0,%
  rightmargin=0,
  innertopmargin=8pt,%
  innerbottommargin=8pt,
  ntheorem]{theorem}{Teorema}[section]

\theoremstyle{definition}
\newmdtheoremenv[%
  linecolor=white,leftmargin=0,%
  rightmargin=0,
  innertopmargin=8pt,%
  innerbottommargin=8pt,
  ntheorem]{define}{Definizione utile}[section]

% figure support
\usepackage{import}
\usepackage{xifthen}
\pdfminorversion=7
\usepackage{pdfpages}
\usepackage{transparent}
\newcommand{\incfig}[1]{%
	\def\svgwidth{\columnwidth}
	\import{./figures/}{#1.pdf_tex}
}

% FSM tikz
\tikzset{
    place/.style={
        circle,
        thick,
        draw=black,
        minimum size=6mm,
    },
        state/.style={
        circle,
        thick,
        draw=black,
        fill=white,
        minimum size=6mm,
    },
}

\pdfsuppresswarningpagegroup=1

\usepackage{pgfplots}
\pgfplotsset{compat=1.18,width=10cm}

% Save plots as pdf and reuse them without compiling every time
\usetikzlibrary{external}
\tikzexternalize[prefix=figures/tikz/, optimize=false]


\begin{document}

\begin{titlepage}
	\begin{center}
		\vspace*{1cm}

		\Huge
		\textbf{Probabilità e Statistica\\Esercizi}

		\vspace{0.5cm}
		\LARGE
		UniVR - Dipartimento di Informatica

		\vspace{1.5cm}

		\textbf{Fabio Irimie}

		\vfill


		\vspace{0.8cm}


		2° Semestre 2023/2024

	\end{center}
\end{titlepage}


\tableofcontents
\pagebreak

\section{Introduzione}
Un'algoritmo è una sequenza \textbf{finita} di \textbf{istruzioni} volta a risolvere un problema.
Per implementarlo nel pratico si scrive un \textbf{programma}, cioè l'applicazione di
un linguaggio di programmazione, oppure si può descrivere in modo informale
attraverso del \textbf{pseudocodice} che non lo implementa in modo preciso,
ma spiega i passi per farlo.
\\
Ogni algoritmo può essere implementato in modi diversi, sta al programmatore
capire qual'è l'opzione migliore e scegliere in base alle proprie necessità.

\subsection{Confronto tra algoritmi}
Ogni algoritmo si può confrontare con gli altri in base a tanti fattori, come:
\begin{itemize}
  \item \textbf{Complessità}: quanto ci vuole ad eseguire l'algoritmo
  \item \textbf{Memoria}: quanto spazio in memoria occupa l'algoritmo
\end{itemize}

\subsection{Rappresentazione dei dati}
Per implementare un algoritmo bisogna riuscire a strutturare i dati in maniera tale
da riuscire a manipolarli in modo efficiente.

\section{Calcolo della complessità}
La complessità di un algoritmo mette in relazione il numero di istruzioni da eseguire
con la dimensione del problema, e quindi è una funzione che dipende dalla dimensione
del problema.

\vspace{1em}
\noindent
La \textbf{dimensione del problema} è un insieme di oggetti adeguato a dare un idea
chiara di quanto è grande il problema da risolvere, ma sta a noi decidere come
misurare il problema.

\noindent
Ad esempio una matrice è più comoda da misurare come il numero di righe e il numero
di colonne, al posto di misurarla come il numero di elementi totali.

\vspace{1em}
\noindent
La complessità di solito si calcola come il \textbf{caso peggiore}, cioè il
limite superiore di esecuzione dell'algoritmo.

\subsection{Linguaggi di programmazione}
Ogni linguaggio di programmazione è formato da diversi blocchi:
\begin{enumerate}
  \item \textbf{Blocco iterativo}: un tipico blocco di codice eseguito sequenzialmente
    e tipicamente finisce con un punto e virgola.
  \item \textbf{Blocco condizionale}: un blocco di codice che viene eseguito solo
    se una condizione è vera.
  \item \textbf{Blocco iterativo}: un blocco di codice che viene eseguito
    ripetutamente finché una condizione è vera.
\end{enumerate}

\noindent
Questi sono i blocchi base della programmazione e se riusciamo a calcolare
la complessità di ognuno di questi blocchi possiamo calcolare più facilmente
la complessità di un intero algoritmo.

\subsubsection{Blocchi iterativi}
\[
  I_1 \;\; c_1(n)
\] 
\[
  I_2 \;\; c_2(n)
\] 
\[
  \vdots \;\;\;\;\;\; \vdots
\] 
\[
  I_l \;\; c_l(n)
\]
Se ogni blocco ha complessità \( c_1(n) \), allora la complessità totale è data
da:
\[
\sum_{i=1}^{l} c_i(n)
\] 

\subsubsection{Blocchi condizionali}
\[
  \text{IF cond} \;\; c_{cond}(n)
\] 
\[
  I_1 \quad \quad c_1(n)
\] 
\[
  \hspace{-1.75cm} \text{ELSE}
\] 
\[
  I_2 \quad \quad c_2(n)
\] 
La complessità totale è data da:
\[
  c(n) = c_{cond}(n) + \max(c_1(n), c_2(n))
\] 
A volte la condizione è un test sulla dimensione del problema e in quel caso si
può scrivere una complessità più precisa.

\subsubsection{Blocchi iterativi}
\[
  \text{WHILE cond} \;\; c_{cond}(n)
\] 
\[
  I \hspace{1.6cm} c_0(n)
\] 
Si cerca di trovare un limite superiore \( m \) al limite di iterazioni.

\vspace{1em}
\noindent
Di conseguenza la complessità totale è data da:
\[
  c_{cond}(n) + m(c_{cond}(n) + c_0(n))
\]

\subsection{Esempio}
\begin{figure}[H]
  \begin{example}
    Calcoliamo la complessità della moltiplicazione tra 2 matrici:
    \[
      A_{n \times m} \cdot B_{m \times l} = C_{n \times l}
    \] 
    L'algoritmo è il seguente:
    \begin{lstlisting}[language=Scala]

  for i <- 1 to n // n ( 5 ml + 4l + 2) + n + 1
    for j <- 1 to l // l (5m + 2 + 1) + 1 + l 
      c[i][j] <- 0
      for k <- 1 to m // (m + 1 + m(4))
        // 3 (moltiplicazione, somma e assegnamento)
        // 1 (incremento for) 
        c[i][j] += a[i][k] * b[k][j]
    \end{lstlisting}

    \noindent
    Partiamo calcolando la complessità del ciclo for più interno. Non ha
    senso tenere in considerazione tutti i dati, ma solo quelli rilevanti. In
    questo caso avremo:
    \[
      (m + 1 + m(4)) = 5m + 1
    \] 
    Questa complessità contiene informazioni poco rilevanti perchè possono far
    riferimento alla velocità della cpu e un millisecondo in più o in meno non cambia
    nulla se teniamo in considerazione solo l'incognita abbiamo:
    \[
      m
    \]
    Questo semplifica molto i calcoli, rendendo meno probabili gli errori. Siccome
    la complessità si calcola su numeri molto grandi, le costanti piccole prima o poi
    verranno tolte perchè poco influenti.

    \vspace{1em}
    \noindent
    La complessità totale alla fine sarebbe stata:
    \[
      5nml+4ml+2n+n+1
    \] 
    Ma ciò che ci interessa veramente è:
    \[
      5\color{red}nml\color{black}+4ml+2n+n+1
    \] 
    Se non consideriamo le costanti inutili, la complessità finale è:
    \[
      nml
    \]
    Nella maggior perte dei casi ci si concentra soltanto sull'ordine di grandezza
    della complessità, e non sulle costanti.
  \end{example}
\end{figure}

\subsection{Ordine di grandezza}
L'ordine di grandezza è una funzione che approssima la complessità di un algoritmo:
\[
f \in O(g)
\] 
\[
  \exists c > 0\; \exists \bar{n}\;\; \forall n \ge \bar{n}\;\; f(n) \le c g(n)
\] 
\begin{figure}[H]
  \centering
  \begin{tikzpicture}
    % axis
    \draw[->] (-0.2,0) -- (6,0) node[right] {$t$};
    \draw[->] (0,-0.2) -- (0,5) node[above] {$y$};

    \draw[red, domain=0:5, samples=100, smooth] plot ({\x},{exp(\x/3)}) node[right] {$cg$};
    \draw[blue, domain=0:5.3, samples=100, smooth] plot ({\x},{exp(\x/5) + sin(deg(5*\x))/2}) node[right] {f};

    \draw[dashed] (1.72,1.77) -- (1.75,0) node[below] {$\bar{n}$};
  \end{tikzpicture}
  \caption{Esempio di funzione \(f \in O(g)\)}
\end{figure}

\[
f \in \Omega(g)
\] 
\[
  \exists c > 0\; \exists \bar{n}\;\; \forall n \ge \bar{n}\;\; f(n) \ge cg(n)
\] 

\[
f \in \Theta(g)
\] 
\[
  f \in O(g) \land f \in \Omega(g)
\]

\vspace{1em}
\noindent
Per gli algoritmi:
\begin{figure}[H]
  \begin{definition}
    \[
      A \in O(f)
    \] 
    So che l'algoritmo \( A \) termina entro il tempo definito dalla funzione \( f \).
    Di conseguenza se un algoritmo termina entro un tempo \( f \) allora sicuramente
    termina entro un tempo \( g \) più grande. Ad esempio:
    \[
      A \in O(n) \Rightarrow A \in O(n^2)
    \] 
    Questa affermazione è \textbf{corretta}, ma \textbf{non accurata}.

    \vspace{1em}
    \[
      A \in \Omega(f)
    \] 
    Significa che esiste uno schema di input tale che se \( g(n) \) è il numero di
    passi necessari per risolvere l'istanza \( n \) allora:
    \[
      g \in \Omega(f)
    \] 
    Quindi l'algoritmo non termina in un tempo minore di \( f \).

    \vspace{1em}
    \noindent
    Calcolando la complessità si troverà lo schema di input tale che:
    \[
      g \in O(f)
    \]
    cioè il limite superiore di esecuzione dell'algoritmo.

    \noindent
    Successivamente ci si chiede se esistono algoritmi migliori e si 
    troverà lo schema di input tale che:
    \[
      g \in \Omega(f)
    \]
    cioè il limite inferiore di esecuzione dell'algoritmo.

    \noindent
    Se i due limiti coincidono allora:
    \[
      g \in \Theta(f)
    \]
    abbiamo trovato il tempo di esecuzione dell'algoritmo.
  \end{definition}
\end{figure}

\begin{figure}[H]
  \begin{theorem}[Teorema di Skolem]
    Se c'è una formula che vale coi quantificatori esistenziali, allora nel linguaggio
    si possono aggiungere delle costanti al posto delle costanti quantificate e assumere
    che la formula sia valida con quelle costanti.
  \end{theorem}
\end{figure}

\subsubsection{Esempi di dimostrazioni}
\begin{figure}[H]
  \begin{example}
    È vero che \( n \in O(2n) \)?

    \noindent
    Se prendiamo \( c = 1 \) e \( \bar{n} = 1 \) allora:
    \[
    n \le c2n
    \] 
    Quindi è vero
  \end{example}

\end{figure}
\begin{figure}[H]
\begin{example}
  È vero che \( 2n \in O(n) \)?

  \noindent
  Se prendiamo \( c = 2 \) e \( \bar{n} = 1 \) allora:
  \[
    2n \le 2n
  \] 
  Quindi è vero
\end{example}
\end{figure}
\begin{figure}[H]
  \begin{example}
    È vero che \( f \in O(g) \iff g \in \Omega(f) \)?

    \noindent
    Dimostro l'implicazione da entrambe le parti:
    \begin{itemize}
      \item \( \to \): Usando il teorema di Skolem:
        \[
          \forall n \ge \bar{n}\;\; f(n) \le cg(n)
        \] 
        Trasformo la disequazione:
        \[
          \forall n \ge \bar{n}\;\; \frac{f(n)}{c} \le g(n)
        \] 
        \[
          \forall n \ge \bar{n}\;\; g(n) \ge \frac{f(n)}{c}
        \] 
        \[
          \forall n \ge \bar{n}\;\; g(n) \ge \frac{1}{c} f(n) \quad \square
        \] 
        Se la definizione di \( \Omega(g) \) è:
        \[
          \exists c' > 0\; \exists \bar{n}'\;\; \forall n \ge \bar{n}'\;\; f(n) \ge c'g(n)
        \]
        sappiamo che:
        \[
          c' = \frac{1}{c}
        \] 
      \item \( \leftarrow \): Usando il teorema di Skolem:
        \[
          \forall n \ge \bar{n}'\;\; g(n) \ge c'f(n)
        \] 
        Trasformo la disequazione:
        \[
          \forall n \ge \bar{n}'\;\; \frac{g(n)}{c'}\ge f(n)
        \] 
        \[
          \forall n \ge \bar{n}'\;\; f(n) \le \frac{1}{c'} g(n) \quad \square
        \] 
    \end{itemize}
  \end{example}
\end{figure}

\begin{figure}[H]
  \begin{example}
    \[
    f_1 \in O(g) \; f_2 \in O(g) \Rightarrow f_1 + f_2 \in O(g)
    \] 
    Dimostrazione:

    Ipotesi
    \[
      \bar{n}_1 c_1\; \forall n > n_1 \quad f_1(n) \le c_1 g(n)
    \] 
    \[
      \bar{n}_1 c_2\; \forall n > n_2 \quad f_2(n) \le c_2 g(n)
    \] 

    \[
    f_1(n) + f_2(n) \le (c_1 + c_2)g(n) \quad \square
    \] 
    Quindi:
    \[
    c = (c_1 + c_2)
    \] 
    \[
      \bar{n} = \max(\bar{n}_1, \bar{n}_2)
    \] 
  \end{example}
\end{figure}

\begin{figure}[H]
  \begin{example}
    Se
    \[
    f_1 \in O(g_1) \; f_2 \in O(g_2)
    \] 
    è vero che:
    \[
    f_1 \cdot f_2 \in O(g_1 \cdot g_2)
    \] 
    Dimostrazione:

    Ipotesi
    \[
      \bar{n}_1 c_1\; \forall n > \bar{n}_1 \quad f_1(n) \le c_1 g_1(n)
    \] 
    \[
      \bar{n}_2 c_2\; \forall n > \bar{n}_2 \quad f_2(n) \le c_2 g_2(n)
    \]

    \[
      f_1(n) \cdot f_2(n) \le (c_1 \cdot c_2) (g_1(n) \cdot g_2(n)) \quad \square
    \] 
    Quindi:
    \[
      c = c_1 \cdot c_2
    \]
    \[
      \bar{n} = \max(\bar{n}_1, \bar{n}_2)
    \]
  \end{example}
\end{figure}

\section{Studio degli algoritmi}
Il problema dell'ordinamento si definisce stabilendo la relazione che deve esistere tra
\textbf{input} e \textbf{output} del sistema.
\begin{itemize}
  \item \textbf{Input}: Sequenza \( (a_1,\ldots,a_n) \) di oggetti su cui è definita una
    relazione di ordinamento, cioè l'unico modo per capire la differenza tra due oggetti
    è confrontarli.

  \item \textbf{Output}: Permutazione \( (a'_1,\ldots,a'_n) \) di \( (a_1,\ldots,a_n) \) 
    tale che:
    \[
    \forall i < j \;\; a'_i \le a'_j
    \] 
\end{itemize}
L'obiettivo è trovare un algoritmo che segua la relazione di ordinamento definita e risolva
il problema nel minor tempo possibile.

\subsection{Insertion sort}
Divide la sequenza in due parti:
\begin{itemize}
  \item \textbf{Parte ordinata}: Sequenza di elementi ordinati
  \item \textbf{Parte non ordinata}: Sequenza di elementi non ordinati
\end{itemize}
\begin{figure}[H]
  \centering
  \begin{tikzpicture}
    \draw[fill, fill opacity=0.2] (0,0) rectangle (3,1) node[midway, black, opacity=1]
      {Ordinato};
    \draw (3,1) rectangle (8,0) node[midway] {Non ordinato};

    \draw[<-] (3.5,0) -- ++(0,-0.5) node[below] {j};
    \draw[<-] (3,0) -- ++(0,-0.5) node[below] {i};
  \end{tikzpicture}
  \caption{Parte ordinata e non ordinata}
\end{figure}

\noindent
Pseudocodice:
\begin{lstlisting}[language=Scala]
insertion_sort(A)
  for j <- 2 to length[A] // A sinistra di j e tutto ordinato-
    key <- A[j]                //                            |
    i <- j - 1                 //                            | O(n)
    while i > 0 and A[i] > key // --                         |
      A[i + 1] <- A[i]         //  | O(n)                    |
      i--                      // --                     -----
    A[i + 1] <- key            
\end{lstlisting}

\noindent
La complessità di questo algoritmo è:
\[
  O(n^2)
\] 
Per capirlo è sufficiente guardare il numero di cicli nidificati e quante volte eseguono
il codice all'interno.

\vspace{1em}
\noindent
Se l'array è già ordinato la complessità è:
\[
\Omega(n)
\] 
Con l'input peggiore possibile la complessità è:
\[
\Omega(n^2)
\]
di conseguenza, visto che vale \( O(n^2) \) e \( \Omega(n^2) \) vale:
\[
\Theta(n^2)
\] 

Quanto spazio in memoria utilizza questo algoritmo?
\begin{itemize}
  \item Variabile j
  \item Variabile i
  \item Variabile key
\end{itemize}
A prescindere da quanto è grande l'array utilizzato, di conseguenza la memoria utilizzata
è costante.

\begin{itemize}
  \item \textbf{Ordinamento in loco}: se la quantità di memoria extra che deve usare 
    non dipende dalla dimensione del problema allora si dice che l'algoritmo è in loco.

  \item \textbf{Ordinamento non in loco}: se la quantità di memoria extra che deve usare
    dipende dalla dimensione del problema allora si dice che l'algoritmo è non in loco.

  \item \textbf{Stabilità}: La posizione relativa di elementi uguali non viene modificata
\end{itemize}
L'insertion sort ordina in loco ed è stabile.

\subsection{Fattoriale}
\begin{lstlisting}[language=Scala]
Fatt(n)
  if n = 0
    ret 1
  else 
    ret n * Fatt(n - 1)
\end{lstlisting}
L'argomento della funzione ci fa capire la complessità dell'algoritmo:
\[
  T(n) = \begin{cases}
    1 & \text{se } n = 0 \\
    T(n - 1) + 1 & \text{se } n > 0
  \end{cases}
\] 
Con problemi ricorsivi si avrà una complessità con funzioni definite ricorsivamente.
Questo si risolve induttivamente:
\[
  \begin{aligned}
    T(n) & = 1 + T(n-1)\\
         & = 1 + 1 + T(n-2)\\
         & = 1 + 1 + 1 + T(n-3)\\
         & = \underbrace{1 + 1 + \ldots + 1}_{i} + T(n-i)\\
  \end{aligned}
\] 
La condizione di uscita è: \( n-i = 0 \quad n = i \) 
\[
\begin{aligned}
         & = \underbrace{1 + 1 + \ldots + 1}_{n} + T(n-n)\\
         & = n + 1 = \Theta(n)
\end{aligned}
\] 
Questo si chiama passaggio iterativo.

\begin{example}
  \[
    T(n) = 2T\left(\floor*{\frac{n}{2}}\right) + n
  \] 
  Questa funzione si può riscrivere come:
  \[
  T(n) = \begin{cases}
    \text{Costante} & \text{se } n < a \\
    2T\left(\floor*{\frac{n}{2}}\right) + n & \text{se } n \ge a
  \end{cases}
  \] 

  \vspace{1em}
  \noindent
  Se la complessità fosse già data bisognerebbe soltanto verificare se è corretta.
  Usando il metodo di sostituzione:
  \[
    T(n) = cn \log n
  \] 
  sostituiamo nella funzione di partenza:
  \[
    \begin{aligned}
      T(n)  & = 2T\left(\floor*{\frac{n}{2}}\right) + n\\
            & \le 2c\left(\floor*{\frac{n}{2}}\right) \log \floor*{\frac{n}{2}} + n\\
            & \le \cancel{2} c \frac{n}{\cancel{2}} \log \frac{n}{2} + n\\
            & = cn \log n - cn \log 2 + n\\
            & \stackrel{?}{\le} cn \log n \quad \text{se } n- cn \log 2 \le 0\\
    \end{aligned}
  \] 
  \[
    c \ge \frac{n}{n \log 2} = \frac{1}{\log 2}
  \] 
  Il metodo di sostituzione dice che quando si arriva ad avere una disequazione
  corrispondente all'ipotesi, allora la soluzione è corretta se soddisfa una certa ipotesi.
\end{example}

\begin{example}
  \[
    T(n) = T\left(\floor*{\frac{n}{2}}\right) + T\left(\ceil*{\frac{n}{2}}\right) + 1 \quad \in O(n)
  \] 
  \[
  T(n) \le cn
  \] 
  \[
  \begin{aligned}
    T(n) & = T\left(\floor*{\frac{n}{2}}\right) + T\left(\ceil*{\frac{n}{2}}\right) + 1\\
         & \le c\left(\floor*{\frac{n}{2}}\right) + c\left(\ceil*{\frac{n}{2}}\right) + 1\\
         & = c \left( \left\lfloor \frac{n}{2} \right\rfloor + \left\lceil \frac{n}{2} \right\rceil  \right) + 1\\
         & = cn + 1 \stackrel{?}{\le} cn
  \end{aligned}
  \] 
  Il metodo utilizzato non funziona perchè rimane l'1 e non si può togliere in alcun modo.
  Per risolvere questo problema bisogna risolverne uno più forte:
  \[
  T(n) \le cn - b
  \] 
  \[
  \begin{aligned}
    T(n) & = T\left(\floor*{\frac{n}{2}}\right) + T\left(\ceil*{\frac{n}{2}}\right) + 1\\
         & \le c\left(\floor*{\frac{n}{2}}\right) -b + c\left(\ceil*{\frac{n}{2}}\right) -b + 1\\
         & = c \left( \left\lfloor \frac{n}{2} \right\rfloor + \left\lceil \frac{n}{2} \right\rceil  \right) - 2b + 1\\
         & = cn - 2b + 1 \stackrel{?}{\le} cn - b\\
         & = \underbrace{cn - b} + \underbrace{1 - b}_{\le 0} \le cn - b \quad \text{se } b \ge 1\\
  \end{aligned}
  \] 
  Se la proprietà vale per questo problema allora vale anche per il problema iniziale
  perchè è meno forte.
\end{example}

\begin{example}
  \[
    \begin{aligned}
      T(n) & = 3T \left( \left\lfloor \frac{n}{4} \right\rfloor \right) + n\\
           & = n + 3T \left( \left\lfloor \frac{n}{4} \right\rfloor \right)\\
           & = n + 3 \left( \left\lfloor \frac{n}{4} \right\rfloor + 3T 
           \left( \left\lfloor \frac{\left\lfloor \frac{n}{4} \right\rfloor}{4} \right\rfloor
           \right)  \right)\\
           & = n + 3 \left\lfloor \frac{n}{4} \right\rfloor + 3^2 T 
           \left( \left\lfloor \frac{n}{4^2} \right\rfloor \right)\\
           & \le n + 3 \left\lfloor \frac{n}{4} \right\rfloor + 3^2 
           \left( \left\lfloor \frac{n}{4^2} \right\rfloor + 3T \left( 
           \left\lfloor \frac{\left\lfloor \frac{n}{4^2} \right\rfloor}{4} \right\rfloor
           \right)  \right) \\
           & = n + 3 \left\lfloor \frac{n}{4} \right\rfloor + 3^2
           \left\lfloor \frac{n}{4^2} \right\rfloor + 3^3 T
           \left( \left\lfloor \frac{n}{4^3} \right\rfloor \right) \\
           & = n + 3 \left\lfloor \frac{n}{4} \right\rfloor + \ldots + 3^{i-1}
           \left\lfloor \frac{n}{4^{i-1}} \right\rfloor + 3^i T
           \left( \left\lfloor \frac{n}{4^i} \right\rfloor \right) 
    \end{aligned}
  \] 
  Per trovare il caso base poniamo l'argomento di T molto piccolo:
  \[
    \begin{aligned}
      \frac{n}{4^i} & < 1\\
      4^i & > n\\
      i & > \log_4 n
    \end{aligned}
  \] 
  L'equazione diventa:
  \[
    \begin{aligned}
      & \le n + 3 \left\lfloor \frac{n}{4} \right\rfloor + \ldots + 3^{\log_4 n - 1}
      \left\lfloor \frac{n}{4^{\log_4 n - 1}} \right\rfloor + 3^{\log_4 n} c\\
    \end{aligned}
  \] 
  Si può togliere l'approssimazione per difetto per ottenere un maggiorante:
  \[
  \begin{aligned}
    & \le n \left( 1 + \frac{3}{4} + \left( \frac{3}{4} \right)^2 + \ldots +
    \left( \frac{3}{4} \right)^{\log_4 n-1} \right) + 3^{\log_4 n} c\\
    & \le n \left( \sum_{i=0}^{\infty} \left( \frac{3}{4} \right)^i \right) + c 3^{\log_4 n}\\
  \end{aligned}
  \] 
  Per capire l'ordine di grandezza di \( 3^{\log_4 n} \) si può scrivere come:
  \[
    3^{\log_4 n} = n^{\left( \log_n 3^{\log_4 n} \right) } = n^{\log_4 n \cdot \log_n 3}
    = n^{\log_4 3}
  \] 
  Quindi la complessità è:
  \[
  \begin{aligned}
    & = O(n) + O(n^{\log_4 3})\\
  \end{aligned}
  \] 
  Si ha che una funzione è uguale al termine noto della funzione originale e l'altra
  che è uguale al logaritmo dei termini noti. Se usassimo delle variabili uscirebbe:
  \[
    \begin{aligned}
      T(n) & = a T \left(  \frac{n}{b}  \right) + f(n)\\
           & = O(f(n)) + O(n^{\log_b a})
    \end{aligned}
  \] 
\end{example}

\subsection{Teorema dell'esperto}
\begin{theorem}[Teorema dell'esperto o Master theorem]
  Per un'equazione di ricorrenza del tipo:
  \[
    T(n) = a T \left(  \frac{n}{b}  \right) + f(n)\\
  \] 
  Si distinguono 3 casi:
  \begin{itemize}
    \item \( f(n) \in O(n^{\log_b a - \varepsilon}) \) allora \( T(n); \in \Theta(n^{\log_b a}) \)  
    \item \( f(n) \in \Theta(n^{\log_b a}) \) allora \( T(n) \in \Theta(f(n) \log n) \) 
    \item \( f(n) \in \Omega(n^{\log_b a + \varepsilon}) \) allora \( T(n) \in \Theta(f(n)) \) 
  \end{itemize}
\end{theorem}

\begin{example}
  \[
  T(n) = 9 T\left(\frac{n}{3}\right) + n
  \] 
  Applico il teorema dell'esperto:
  \[
  \begin{aligned}
    a & = 9\\
    b & = 3\\
    f(n) & = n\\
  \end{aligned}
  \] 
  \[
    n^{\log_b a} = n^{\log_3 9} = n^2
  \] 
  Verifico se esiste un \( \varepsilon \) tale che:
  \[
    n \in O(n^{2 - \varepsilon})
  \]
  prendo \( \varepsilon = -\frac{1}{2} \) e verifico:
  \[
    n \in O(n^2 \cdot n^{-\frac{1}{2}})
  \] 
  Quindi ho trovato il caso 1 del teorema dell'esperto.
  \[
    T(n) \in \Theta(n^2)
  \] 
\end{example}

\begin{example}
  \[
  T(n) = T \left( \frac{2n}{3} \right) + 1
  \] 
  Applico il teorema dell'esperto:
  \[
    \begin{aligned}
      a & = 1\\
      b & = \frac{3}{2}\\
      f(n) & = n^0\\
    \end{aligned}
  \]

  \[
    n^{\log_b a} = n^{\log_{\frac{3}{2}} 1} = n^0
  \] 
  Si nota che le due funzioni hanno lo stesso ordine di grandezza, quindi siamo nel secondo
  caso del teorema dell'esperto.
  \[
    T(n) \in \Theta(\log n)
  \] 
\end{example}

\begin{example}
  \[
    T(n) = 3T \left( \frac{n}{4} \right) + n \log n
  \] 
  Applico il teorema dell'esperto:
  \[
    \begin{aligned}
      a & = 3\\
      b & = 4\\
      f(n) & = n \log n\\
    \end{aligned}
  \]
  \[
    n^{\log_b a} = n^{\log_4 3}
  \]
  \[
    n \log n \in \Omega(n^{\log_4 3})
  \]
  Esiste un \( \varepsilon \) tale che:
  \[
    n \log n \in \Omega(n^{\log_4 3 + \varepsilon})
  \]
  perchè basta che sia compreso tra \( \log_4 3 \) e \( 1 \).
  
  \vspace{1em}
  \noindent
  Quindi siamo nel terzo caso del teorema dell'esperto.
  \[
    T(n) \in \Theta(n \log n)
  \]
\end{example}

\begin{example}
  \[
  T(n) = 2T \left( \frac{n}{2} \right) + n \log n
  \] 
  Applico il teorema dell'esperto:
  \[
    \begin{aligned}
      a & = 2\\
      b & = 2\\
      f(n) & = n \log n\\
    \end{aligned}
  \]
  \[
    n^{\log_b a} = n^{\log_2 2} = n
  \]
  \[
    n \log n \in \Omega(n)
  \]
  Verifico se esiiste un \( \varepsilon \), quindi divido per \( n \):
  \[
    \log n \in \Omega(n^{\varepsilon})
  \] 
  Quindi si nota che questa proprietà non è verificata, quindi non si può applicare il
  teorema dell'esperto.
\end{example}

\subsection{Merge sort}
Questo algoritmo di ordinamento è basato sulla tecnica divide et impera:
\begin{itemize}
  \item \textbf{Divide}: Dividi il problema in sottoproblemi più piccoli
  \item \textbf{Impera}: Risolvi i sottoproblemi in modo ricorsivo
  \item \textbf{Combina}: Unisci le soluzioni dei sottoproblemi per risolvere il problema
    originale
\end{itemize}
Questo algoritmo divide la sequenza in due parti uguali e le ordina separatamente, successivamente
le unisce in modo ordinato. La complessità, comsiderando il merge con complessità lineare,
risulta:
\[
  T(n) = 2T\left(\frac{n}{2}\right) + n
\] 
Applicando il teorema dell'esperto si ottiene:
\[
\begin{aligned}
  a & = 2\\
  b & = 2\\
  f(n) & = n\\
\end{aligned}
\] 
\[
  n^{\log_b a} = n
\] 
\[
  n \in \Theta(n)
\] 
Quindi siamo nel secondo caso del teorema dell'esperto:
\[
  T(n) \in \Theta(n \log n)
\]

\vspace{1em}
\noindent
Definizione del merge sort:
\begin{lstlisting}[language=Scala]
// A: Array da ordinare
// P: Indice di partenza
// r: Indice di arrivo
merge_sort(A, p, r)         // --
  if p < r                  //  |
    q <- floor((p + r) / 2) //  | 
    merge_sort(A, p, q)     //  | O(n log n)
    merge_sort(A, q + 1, r) //  |
    merge(A, p, q, r)       // --
\end{lstlisting}
\begin{lstlisting}[language=Scala]
// A: Array da ordinare
// P: Indice di partenza
// q: Indice di mezzo
// r: Indice di arrivo
merge(A, p, q, r)
  i <- 1
  j <- p
  k <- q + 1
  // Ordina gli elementi di A in B
  while(j <= q and k <= r)                // --
    if j <= q and (k > r or A[j] <= A[k]) //  |
      B[i] <- A[j]                        //  |
      j++                                 //  |
    else                                  //  | O(n)
      B[i] <- A[k]                        //  |
      k++                                 //  |
    i++                                   // --

  // Copia gli elementi di B in A
  for i <- 1 to r - p + 1                 // -|
    A[p + i - 1] <- B[i]                  // -| O(n)
\end{lstlisting}

\noindent
L'algoritmo è stablie perchè non vengono scambiati elementi uguali e non è in loco perchè
utilizza un array di appoggio.

\subsection{Heap}
È un albero semicompleto (ogni nodo ha 2 figli ad ogni livello tranne l'ultimo che è
completo solo fino ad un certo punto) in cui i nodi contengono oggetti con relazioni di
ordinamento.
\begin{figure}[H]
  \centering
  \begin{forest}
    for tree={
    draw,
    circle,
    minimum size=2em,
    inner sep=1pt,
    s sep=1cm,
  }
    [1
      [2
        [4
        [8]
        [9]
        ]
        [5
        [10]
        ]
      ]
      [3
        [6]
        [7]
      ]
    ]
  \end{forest}
  \caption{Heap con l'indice di un array associato ai nodi}
\end{figure}

\subsubsection{Proprietà}
\( \forall \) nodo il contenuto del nodo è \( \ge \) del contenuto dei figli.
Per calcolare il numero di nodi di un albero binario si usa la formula:
\[
  N = 2^0 + 2^1 + 2^2 + \ldots + 2^{h-1} = \frac{1-2^h}{1-2} = 2^h - 1
\] 
dove \( h \) è l'altezza dell'albero.
Il numero di foglie di un albero sono la metà dei nodi.

\vspace{1em}
\noindent
Definiamo una funzione che "aggiusta" i figli di un nodo per mantenere la proprietà di heap:
\begin{lstlisting}[language=Scala]
  heapify(A, i) // O(n)
    l <- left[i] // Indice del figlio sinistro (2i)
    r <- right[i] // Indice del figlio destro (2i+1)
    if l < H.heap_size and H[l] > H[i]
      largest <- l
    else
      largest <- i

    if r < H.heap_size and H[r] > H[largest]
      largest <- r
    if largest != i
      swap(H[i], H[largest])
      heapify(H, largest)
\end{lstlisting}
Ora si vuole definire una funzione che costruisce un heap da un array:
\begin{lstlisting}[language=Scala]
  build_heap(A) // O(n)
    heapsize(a) <- length[A]
    for i <- floor(length[A]/2) downto 1
      heapify(A, i)
\end{lstlisting}
Una volta definito un heap si possono fare diverse operazioni, come ad esempio estrarre
il nodo massimo:
\begin{lstlisting}[language=Scala]
  extract_max(A)
    H[1] <- H[H.heap_size]
    H.heap_size <- H.heap_size - 1
    heapify(H,1)
\end{lstlisting}
\subsubsection{Heap sort}
Heap sort è un algoritmo di ordinamento basato su heap.
\begin{lstlisting}[language=Scala]
  heap_sort(A) // O(n log n)
    build_heap(A) // n
    for i <- length[A] downto 2
      scambia(A[1], A[i])
      heapsize(A)--
      heapify(A, 1) // log i
\end{lstlisting}


La complessità dell'algoritmo è precisamente:
\[
\sum_{i=1}^{n} \log i = \log \prod_{i=1}^{n} i = \log n! = \Theta(\log n^n) = \Theta(n \log n)
\] 
Per la formula di Stirling \( n! \) ha ordine di grandezza \( n^n \). Questo algoritmo
è in loco e instabile.

\vspace{1em}
\noindent
Il caso pessimo è un array ordinato al contrario (\( O(n \log n)\)) e il caso migliore
è un array già ordinato (\( \Omega(n \log n)\)), quindi la complessità è:
\[
\Theta(n \log n)
\]

\subsection{Quick sort}
Il concetto di questo algoritmo è quello di mettere prima in disordine l'algoritmo e poi
ordinarlo. L'algoritmo divide l'array in 2 parti e ordina ricorsivamente le due parti; a
quel punto l'array è ordinato.

\begin{lstlisting}[language=Scala]
  // A: Array da ordinare 
  // p: Indice di partenza
  // r: Indice di arrivo
  quick_sort(A, p, r)
    if p < r // Ordina solo se l'array ha piu' di un elemento
      q <- partition(A, p, r) // Dividi l'array in due parti
      quick_sort(A, p, q) // Ordina sinistra
      quick_sort(A, q + 1, r) // Ordina destra
\end{lstlisting}
\begin{lstlisting}[language=Scala]
  partition(A, p, r)
    x <- A[p] // Elemento perno (o pivot)
    i <- p - 1
    j <- r + 1
    while true
      repeat // Ripete finche' la condizione non e' soddisfatta
        j--  // n/2
      until A[j] <= x // Trova un elemento che non puo' stare a destra
      repeat
        i++  // n/2
      until A[i] >= x // Trova un elemento che non puo' stare a sinistra
      if i < j
        swap(A[i], A[j]) // n/2
      else
        return j // alla fine j puntera' all'ultimo elemento di sinistra
\end{lstlisting}

\noindent
Questo algoritmo è in loco e non è stabile. La sua complessità nel caso peggiore è:
\[
  \begin{aligned}
    T(n) &= T(\text{partition}) + T(q) + T(n-q)\\
         &= n + T(q) + T(n-q)\\
         &= n + \cancel{T(1)} + T(n-1)\\
         &= n + T(n-1)\\
         &= \Theta(n^2)
  \end{aligned}
\] 
Il caso peggiore è un array già ordinato.

\vspace{1em}
\noindent
Mediamente ci si aspetta che l'array venga diviso in 2 parti molto simili, quindi
la complessità è \( O(n \log n) \) perchè:
\[
0 < c < 1
\] 
\[
  T(n) = n + T(cn) + T((1-c)n)
\] 

\begin{lstlisting}[language=Scala]
  rand_partition(A, p, r)
    i <- random(p, r)
    swap(A[i], A[p])
    return partition(A, p, r)
\end{lstlisting}
La complessità è la media di tutte le complessità con probabilità \( \frac{1}{n} \) 
\[
  \begin{aligned}
    T(n) = n + \frac{1}{n}\left( T(1) + T(n-1) \right) + \frac{n-1}{n}\left( T(2) + T(n-2) \right)\\
    + \ldots + \frac{1}{n}\left( T(n-1) + T(1) \right)
  \end{aligned}
\] 
\[
  \begin{aligned}
    T(n) &= n + \frac{1}{n} \sum_{i} \left( T(i) + T(n-i) \right) \\
         &= n + \frac{2}{n} \sum_{i} T(i)\\
  \end{aligned}
\] 

\subsection{Algoritmi di ordinamento non per confronti} 
Si possono avere algoritmi di ordinamento con complessità \( < n \log n \)?

\vspace{1em}
\noindent
Qualsiasi algoritmo che ordina per confronti deve fare almeno \( n \log n \) confronti
nel caso pessimo
\begin{figure}[H]
  \centering
  \begin{forest}
    for tree={
    minimum size=2em,
    inner sep=1pt,
    s sep=1cm,
  }
    [\( x \to n! \) 
      [\( x_1 \) 
        [\( x_{1,1} \)]
        [\( x_{1,2} \) ]
      ]
      [\( x_2 \) 
        [\( x_{2,1} \) ]
        [\( x_{2,2} \) ]
      ]
    ]
  \end{forest}
  \caption{Heap con l'indice di un array associato ai nodi}
\end{figure}
\noindent
Le foglie rappresentano ogni singola combinazione possibile. Il numero di foglie è
\( n! \) e l'altezza sarà sempre
\[
h \ge \log_2 n! = n \log n
\] 

\subsubsection{Counting sort}
Si vogliono ordinare \( n \) numeri con valori da \( 1 \) a \( k \). L'idea di questo
algoritmo è quella di creare un'array che contiene il numero di occorrenze di un
certo valore (rappresentato dall'indice).

\begin{lstlisting}[language=Scala]
counting_sort(A, k) 
  for i <- 1 to k // k
    C[i] <- 0 // Inizializzazione di un array a 0

  for j <- 1 to length[A] // n
    C[A[j]]++ // Conteggio delle occorrenze

  // k
  for i <- 2 to k         // In ogni indice metto il numero di
    C[i] <- C[i] + C[i-1] // elementi minori o uguali
                          // al numero dell'indice
  // Alla fine l'array C conterra' l'ultima posizione di occorrenza per ogni elemento

  for j <- length[A] downto 1 // n
    B[C[A[j]]] <- A[j] // Inserimento dell'elemento in posizione
    C[A[j]]--          // Decremento della posizione di occorrenza
\end{lstlisting}
La complessità di questo algoritmo è \( O(n + k) \) e siccome sappiamo che \( k \) è
una costante fissata a priori la complessità è \( O(n) \). Non è in loco, ma è stabile

\subsubsection{Radix sort}
Il radix sort è un ordinamento lessico grafico, cioè si ordinano le cifre partendo da
quella meno significativa e se sono uguali si passa a quella più significativa.

La complessità dell'algoritmo è:
\[
 \Theta(l(n + k))
\] 
dove:
\[
\begin{aligned}
  l & = \text{numero di cifre} = \log_k n\\
  n & = \text{numero di elementi}\\
  k & = \text{numero di valori possibili}
\end{aligned}
\] 
Se rappresentiamo i numeri in base \( n \), allora si avrà la seguente rappresentazione:
\[
   \ldots \;\; n^2 \;\; n^1 \;\; n^0
\] 
e ad esempio per rappresentare \( n^2 - 1 \) valori possibili serviranno \( 2 \) cifre.
cifre.

\subsubsection{Bucket sort}
Dato un array di numeri con \textbf{supporto infinito} e \textbf{distribuzione nota}, si può dividere
l'array in \( k \) parti (bucket) uguali (equiprobabili) e ordinare ricorsivamente. Ogni coppia
di gruppi deve essere totalmente ordinata, cioè ogni elemento del primo gruppo deve essere
minore di ogni elemento del secondo gruppo. Una volta ordinati i gruppi (con un algoritmo
di ordinamento a scelta) si concatenano in modo ordinato.

\vspace{1em}
\noindent
Il caso peggiore è quello in cui tutti gli elementi finiscono in un singolo bucket,
la probabilità che questo accada è molto bassa:
\[
  \underbrace{\frac{1}{n} \cdot \frac{1}{n} \cdot \dots \cdot \frac{1}{n}}_{n-1}
  = \frac{1}{n^{n-1}}
\] 
e la sua complessità diventa:
\[
  O(n^2)
\] 

\vspace{1em}
\noindent
Nel caso medio si ha che per creare i bucket si ha una complessità \( O(n) \) e per
assegnare gli elementi ai bucket si ha una complessità \( O(n) \). Per ogni bucket
ci si aspetta che il numero di elementi al suo interno sia una \textbf{costante},
quindi \textbf{indipendente dal valore di \( n \)}. Per ordinare un bucket si ha una
complessità \( O(1) \) siccome il numero di elementi è costante. La complessità totale
è quindi:
\[
  \Theta(n)
\]

\vspace{1em}
\noindent
Formalizzando si ha:

\noindent
Sia \( X_{ij} \) la variabile casuale che vale: 
$\begin{cases}
  1 & \text{se l'elemento } i \text{ va nel bucket } j\\
  0 & \text{altrimenti}
\end{cases}$

\noindent
Per esprimere il numero di elementi nel bucket \( j \) si può scrivere:
\[
  N_j = \sum_{i} X_{ij}
\] 
La complessità di questo algoritmo sarà quindi:
\[
  C = \sum_{j} (N_j)^2
\] 
Per ottenere il valore medio della complessità:
\[
  \begin{aligned}
    E[C] &= E\left[\sum_{j} (N_j)^2\right] \\
         &= \sum_{j} E[(N_j)^2] \\
         &= \sum_{j} \left( Var[N_j] + E[N_j]^2 \right) \\
  \end{aligned}
\] 
sappiamo che \( N_j = \sum_{i} X_{ij} \), quindi la media è:
\[
  E[N_j] = \sum_{j}^{n} E[X_{ij}] = \sum_{j}^{n} \frac{1}{n} = 1
\] 
e la varianza è:
\[
  Var[N_j] = \sum_{j}^{n} Var[X_{ij}] = \sum_{j}^{n} \frac{1}{n} \left( 1 - \frac{1}{n} \right) = 1 - \frac{1}{n}
\] 
La complessità diventa:
\[
  \begin{aligned}
    E[C] &= \sum_{j} \left( \left( 1 - \frac{1}{n} \right) - 1 \right) \\
         &= \sum_{j} 2 - \frac{1}{n} \\
         &= 2n - 1
  \end{aligned}
\] 

\section{Algoritmi di selezione}
Dato in input un array \( A \) di oggetti su cui è definita una relazione di ordinamento
e un indice \( i \) compreso tra \( 1 \) e \( n \) (\( n \) è il numero di oggetti
nell'array), l'output dell'algoritmo è l'oggetto che si trova in posizione \( i \)
nell'array ordinato.
\begin{lstlisting}[language=Scala]
selezione(A, i)
  ordina(A) // O(n log n)
  return A[i]
\end{lstlisting}
Quindi la complessità di questo algoritmo nel caso peggiore è \( O(n \log n) \)
(limite superiore). È possibile selezionare un elemento in tempo lineare? Analizziamo
un caso particolare dell'algoritmo di selezione, ovvero la ricerca del minimo (o del massimo).


\subsection{Ricerca del minimo o del massimo}
\vspace{1em}
\noindent
In tempo lineare si può trovare il minimo e il massimo
di un array:
\begin{lstlisting}[language=Scala]
minimo(A)
  min <- A[1]
  for i <- 2 to length[A]
    if A[i] < min
      min <- A[i]
  return min
\end{lstlisting}
trovare il minimo equivale a trovare \texttt{selezione(A, 1)} e trovare il massimo
equivale a trovare \texttt{selezione(A, n)}. Si può però andare sotto la complessità
lineare?

\vspace{1em}
\noindent
Per trovare il massimo (o il minimo) elemento \( n \) di un array bisogna fare
\textbf{almeno} \( n-1 \) confronti perchè bisogna confrontare ogni elemento con
l'elemento massimo (o minimo) trovato per poter dire se è il massimo (o minimo).
Di conseguenza, non è possibile avere un algoritmo per la ricerca del massimo (o minimo)
in cui c'è un elemento che non "perde" mai ai confronti (cioè risulta sempre il più 
grande) e non viene dichiarato essere il più grande (o più piccolo).

\vspace{1em}
\noindent
\textbf{Dimostrazione}:
Per dimostrarlo si può prendere un array in cui l'elemento \( a \) non perde mai ai
confronti, ma l'algoritmo dichiara che il massimo è l'elemento \( b \). Allora si rilancia
l'algoritmo sostituendo l'elemento \( a \) con \( a = \text{\texttt{max(b+1,a)}} \) e si
ripete l'algoritmo con questo secondo array in cui \( a \) è l'elemento più grande. Si ha
quindi che i confronti in cui \( a \) non è coinvolto rimangono gli stessi e i confronti
in cui \( a \) è coinvolto non cambiano perchè anche prima \( a \) non perdeva mai ai
confronti, di conseguenza l'algoritmo dichiarerà che il massimo è \( b \) e quindi
l'algoritmo non è corretto, dimostrando che non esiste un algoritmo che trova il massimo
in meno di \( n-1 \) confronti.

\vspace{1em}
\noindent
Abbiamo quindi trovato che la complessità del massimo (o minimo) nel caso migliore è 
\( \Omega(n) \) (limite inferiore) e nel caso peggiore è \( O(n) \) (limite superiore).
Di conseguenza la complessità è \( \Theta(n) \).

\subsubsection{Ricerca del minimo e del massimo contemporaneamente}
Si potrebbe implementare unendo i 2 algoritmi precedenti:
\begin{lstlisting}[language=Scala]
min_max(A)
  min <- A[1]
  max <- A[1]
  for i <- 2 to length[A]
    if A[i] < min
      min <- A[i]
    if A[i] > max
      max <- A[i]
  return (min, max)
\end{lstlisting}
Questo algoritmo esegue \( n-1 + n-1 = 2n-2 \) confronti.

\begin{itemize}
  \item \textbf{Limite inferiore}: Potenzialmente ogni oggetto potrebbe essere il minimo
    o il massimo. Sia \( m \) il numero di oggetti potenzialmente minimi e \( M \) il
    numero di oggetti potenzialmente massimi. Sia \( n \) il numero di oggetti nell'array.
    \begin{itemize}
      \item All'inizio \( m+M = 2n \) perchè ogni oggetto può essere sia minimo che 
        massimo.
      \item Alla fine \( m+M = 2 \) perchè alla fine ci sarà un solo minimo e un solo 
        massimo.
    \end{itemize}
    Quando viene fatto un confronto \( m+M \) può diminuire.
    \begin{itemize}
      \item Se si confrontano due oggetti che sono potenzialmente sia minimi che massimi,
        allora \( m+M \) diminuisce di \( 2 \) perchè:
        \[
          a < b
        \] 
        \( b \) non può essere il minimo e \( a \) non può essere il massimo e si perdono
        2 potenzialità.

      \item Se si confrontano due potenziali minimi (o massimi), allora \( m+M \) 
        diminuisce di \( 1 \) perchè:
        \[
          a < b
        \]
        \( b \) non può essere il minimo e si perde 1 potenzialità.
    \end{itemize}
    Un buon algoritmo dovrebbe scegliere di confrontare sempre 2 oggetti che sono
    entrambi potenziali minimi o potenziali massimi.

    \vspace{1em}
    \noindent
    Due oggetti che sono potenzialmente sia minimi che massimi esistono
    se \( m+M > n+1 \) perchè se bisogna distribuire n potenzialità ne avanzano
    due che devono essere assegnate a due oggetti che hanno già una potenzialità.
    Quindi fino a quando \( m+M \) continua ad essere almeno \( n+2 \) si riesce a
    far diminuire \( m+M \) di 2 ad ogni confronto.

    Questa diminuzione si può fare \( \left\lfloor \frac{n}{2} \right\rfloor \) volte,
    successivamente \( m+M \) potrà calare solo di 1 ad ogni confronto.
    
    \vspace{1em}
    \noindent
    Successivamente il numero di oggetti rimane:
    \[
      \begin{cases}
        n+1 & \text{se } n \text{ è dispari}\\
        n & \text{se } n \text{ è pari}
      \end{cases}
    \] 
    \begin{itemize}
      \item \( n \) dispari:
        \[
          \begin{aligned}
            &n+1 - 2 + \left\lfloor \frac{n}{2} \right\rfloor\\
            &= n-1 + \left\lfloor \frac{n}{2} \right\rfloor\\
            &= \left\lfloor \frac{3}{2}n \right\rfloor - 1\\
            &= \left\lceil \frac{3}{2}n \right\rceil - 2\\
          \end{aligned}
        \] 

      \item \( n \) pari:
        \[
          \begin{aligned}
            &n - 2 + \left\lfloor \frac{n}{2} \right\rfloor \\
            &= n-2 + \frac{n}{2}\\
            &= \frac{3}{2}n - 2\\
            &= \left\lceil \frac{3}{2}n \right\rceil -2
          \end{aligned}
        \]
    \end{itemize}
    Quindi la complessità è \( \Omega(\left\lceil \frac{3}{2}n \right\rceil -2) = \Omega(n)
    \) (limite inferiore). Meglio di così non si può fare, ma non è detto che esista
    un algoritmo che raggiunga questo limite inferiore.
\end{itemize}
Un algoritmo che raggiunge il limite inferiore è il seguente:
\begin{enumerate}
  \item Dividi gli oggetti in 2 gruppi:
    \[
      \underbrace{
        \underbrace{
          \begin{aligned}
          &a_1\\
          &a_2\\
          &\vdots\\
          &a_{\left\lfloor \frac{n}{2} \right\rfloor}
          \end{aligned}
        }_{\text{Potenziali minimi}}
        \quad
        \underbrace{
          \begin{aligned}
        &b_1\\
        &b_2\\
        &\vdots\\
        &b_{\left\lceil \frac{n}{2} \right\rceil}
          \end{aligned}
        }_{\text{Potenziali massimi}}
      }_{\text{Potenziali sia minimi che massimi}}
    \] 

  \item Confronta \( a_i \) con \( b_i \), supponendo \( a_i < b_i \) (mette a sinistra
    i più piccoli e a destra i più grandi)

  \item Cerca il minimo degli \( a_i \) e cerca il massimo dei \( b_i \):

  \item Sistema l'eventuale elemento in più se l'array è dispari
\end{enumerate}

\subsection{Randomized select}
Si può implementare un algoritmo che divide l'array in 2 parti allo stesso modo
in cui viene effettuata la \texttt{partition} di quick sort:
\begin{lstlisting}[language=Scala]
// A: Array
// p: Indice di partenza
// r: Indice di arrivo
// i: Indice che stiamo cercando (compreso tra 1 e r-p+1)
randomized_select(A, p, r, i)
  if p = r
    return A[p]
  q <- randomized_partition(A, p, r)
  k <- q - p + 1 // Numero di elementi a sinistra 
  // Controlla se l'elemento cercato e' a sinistra o a destra
  if i <= k
    return randomized_select(A, p, q, i) // Cerca a sinistra
  else
    return randomized_select(A, q+1, r, i-k) // Cerca a destra
\end{lstlisting}
\begin{itemize}
  \item 
    Se dividessimo sempre a metà si avrebbe:
    \[
      T(n) = n + T\left(\frac{n}{2}\right) = \Theta(n) \text{ (terzo caso del teorema dell'esperto)}
    \] 

  \item Mediamente:
    \[
      \begin{aligned}
        T(n) &= n + \frac{1}{n} T \left( max(1,n-1) \right) + \frac{1}{n} T \left( max(2,n-2) \right)
        + \dots\\
             &= n + \frac{2}{n} \sum_{i=\frac{n}{2}}^{n-1} T \left( i \right)\\
      \end{aligned}
    \] 
    La complessita media è lineare.

    Si esegue un solo ramo, che nel caso pessimo è quello con più elementi. La risoluzione
    è la stessa del quick sort.
\end{itemize}
Esiste un algoritmo che esegue la ricerca in tempo lineare anche nel caso peggiore?

Si potrebbe cercare un elemento perno più ottimale, cioè che divida l'array in
\textbf{parti proporzionali}:
\begin{enumerate}
  \item Dividi gli oggetti in \( \left\lfloor \frac{n}{5} \right\rfloor \) gruppi di
    5 elementi più un eventuale gruppo con meno di 5 elementi.

  \item Calcola il mediano di ogni gruppo di 5 elementi (si ordina e si prende l'elemento
    centrale). \( \Theta(n) \)

  \item Calcola ricorsivamente il mediano \( x \) dei mediani
    \[
      T\left(\left\lceil \frac{n}{5} \right\rceil\right)
    \] 

  \item Partiziona con perno \( x \) e calcola \( k \) (numero di elementi a sinistra).
    \( \Theta (n) \) 

  \item Se \( i<k \) cerca a sinistra l'elemento \( i \), altrimenti cerca a destra
    l'elemento \( i-k \). La chiamata ricorsiva va fatta su un numero di elementi
    sufficientemente piccolo, e deve risultare un proporzione di \( n \), quindi
    ad esempio dividere in gruppi da 3 elementi non funzionerebbe.
    \[
    T(?)
    \] 
    \[
      \begin{aligned}
        m_1 \;\;&\to\;\; m_2 \;\;&\to\;\; m_3 \;\;&\to\;\; \color{red}\underset{x}{m_4} \;\;&\to\;\; \color{green!50!black}m_5
        \;\;&\to\;\; \color{green!50!black}m_6 \;\;&\to\;\; \color{blue}m_7\\
             &&&& \downarrow \quad&\qquad \downarrow &\downarrow \;\;\\
             &&&& \color{green!50!black}m_{5,4} &\qquad \color{green!50!black}m_{6,4} & \color{blue}m_{7,4} \\
             &&&& \downarrow \quad&\qquad \downarrow &\\
             &&&& \color{green!50!black}m_{5,5} &\qquad \color{green!50!black}m_{6,5} & \\
      \end{aligned}
    \] 
    Gli elementi verdi sono maggiori dell'elemento \( x \) e ogni elemento verde avrà
    2 elementi maggiori di esso (tranne nel caso del gruppo con meno di 5 elementi
    rappresentato in blu).
    \[
      \text{\#left} \le 3 \cdot  \left(\underbrace{\left\lceil \frac{1}{2} \left\lceil \frac{n}{5} \right\rceil  \right\rceil}_{\text{
          verdi + blu + rosso
      }} - \underbrace{2}_{\text{rosso + blu}} \right) 
      = \frac{7}{10} n + 6
    \] 
    \[
      \text{\#right} \ge 3 \cdot  \left(\underbrace{\left\lceil \frac{1}{2} \left\lceil \frac{n}{5} \right\rceil  \right\rceil}_{\text{
          verdi + blu + rosso
      }} - \underbrace{2}_{\text{rosso + blu}} \right) 
      = \frac{7}{10} n + 6
    \] 
    Da ogni parte si hanno almeno \( \frac{7}{10} n + 6 \) elementi.

    \vspace{1em}
    \noindent
    Quindi abbiamo trovato \( T(?) \):
    \[
      T(n) = \Theta (n) + T\left(\left\lceil \frac{n}{5} \right\rceil\right) + T\left(\frac{7}{10} n + 6\right)
    \] 
    Dimostriamo con il metodo di sostituzione, supponendo \( T(n) \le cn \), che la
    disequazione sia vera:
    \[
    \begin{aligned}
      T(n) &\le n + c \left\lceil \frac{n}{5} \right\rceil + c(\frac{7}{10}n + 6) \
    \end{aligned}
    \] 
    Non sappiamo se \( \frac{7}{10}n + 6 \) è minore di \( n \), quindi lo calcoliamo:
    \[
    \begin{aligned}
      \frac{7}{10}n + 6 &\le n\\
      7n + 60 &\le 10n\\
      3n &\ge 60\\
      n &\ge 20
    \end{aligned}
    \] 
    Quindi per valori di \( n \le 20 \) la disequazione non è vera. Consideriamo quindi
    \( \bar{n} > 20 \) e \( n > \bar{n} \). Togliendo l'approssimazione per eccesso si ha:
    \[
    \begin{aligned}
      T(n) &\le n + c + \frac{c}{5}n + \frac{7}{10}n + 6c \\
           &\le \frac{9}{10}cn + 7c + n \\
           &\stackrel{?}{\le} cn \\
           &= cn + \left(-\frac{1}{10}cn + 7c + n\right) \le cn \text{ quando } \\
           & \left( n + 7c - \frac{1}{10}cn \right) \le cn
    \end{aligned}
    \] 
    Quindi \( T(n) \le cn \) e quindi \( T(n) = O(n) \). Abbiamo trovato un limite superiore
    e un limite inferiore, quindi la complessità è \( T(n) = \Theta(n) \) . Il problema è
    che le costanti sono così alte che nella pratica è meglio il
    \texttt{randomized\_select}.
\end{enumerate}

\vspace{1em}
\noindent
Esiste un modo per strutturare meglio le informazioni nel calcolatore per trovare
l'elemento cercato in tempo \( \log n \)?
Si possono implementare delle \textbf{Strutture dati} che permettono di fare
ricerche in tempo logaritmico.

\section{Strutture dati}
Una struttura dati è un modo per organizzare i dati in modo da poterli manipolare
in modo efficiente. Bisogna avere un modo per comunicare con le strutture dati, senza
dover sapere come sono implementate.
\subsection{Stack}
Ad esempio se consideriamo uno stack, si possono individuare le seguenti operazioni:
\begin{itemize}
  \item \texttt{new()}: Crea uno stack vuoto
  \item \texttt{push(S,x)}: Inserisce un elemento \( x \) nello stack \( S \) 
  \item \texttt{top(S)}: Restituisce l'elemento in cima allo stack \( S \) 
  \item \texttt{pop(S)}: Rimuove l'elemento in cima allo stack \( S \) 
  \item \texttt{is\_empty()}: Restituisce vero se lo stack è vuoto
\end{itemize}
Da queste operazioni si possono definire certe proprietà dello stack:
\begin{itemize}
  \item \texttt{top(push(S,x)) = x}
  \item \texttt{pop(push(S,x)) = S}
\end{itemize}
Questo ci dice che lo stack è LIFO (Last In First Out). 

\vspace{1em}
\noindent
Abbiamo quindi definito
un'algebra dei termini da cui possono definire tutte le operazioni possibili,
ad esempio uno stack è definito come una sequenza di push:
\[
  \text{\texttt{push(push(push(empty(),1),2),3)}}
\] 

\subsection{Queue}
Una coda è una struttura dati in cui si possono inserire elementi in fondo e
rimuoverli dall'inizio. Le operazioni possibili sono:
\begin{itemize}
  \item \texttt{new()}: Crea una coda vuota
  \item \texttt{enqueue(Q,x)}: Inserisce un elemento \( x \) in fondo alla coda \( Q \) 
  \item \texttt{dequeue(Q)}: Rimuove l'elemento in cima alla coda \( Q \) 
  \item \texttt{front(Q)}: Restituisce l'elemento in cima alla coda \( Q \) 
  \item \texttt{is\_empty()}: Restituisce vero se la coda è vuota
\end{itemize}

\subsection{Albero binario}
Un albero binario è una struttura dati in cui ogni nodo ha al massimo 2 figli.
\begin{figure}[H]
  \centering
  \begin{forest}
    for tree={
    circle,
    draw,
    minimum size=2em,
    inner sep=1pt,
    s sep=1cm,
  }
    [\( x \)
      [\( y \)
        [\( z \)]
      ]
      [\( v \)
        [\( u \)]
        [\( t \)]
      ]
    ]
  \end{forest}
  \caption{Albero binario}
\end{figure}
\noindent
Le operazioni possibili su un albero binario sono:
\begin{itemize}
  \item \texttt{new()}: Crea un albero vuoto
  \item \texttt{insert(T, x)}: Crea un nuovo nodo con valore \( x \) e lo aggiunge
    all'albero \( T \)
  \item \texttt{extract(T, x)}: Rimuove il nodo con valore \( x \) dall'albero \( T \)
  \item \texttt{is\_empty()}: Restituisce vero se l'albero è vuoto
  \item \texttt{left(T)}: Restituisce il figlio sinistro dell'albero \( T \) 
  \item \texttt{right(T)}: Restituisce il figlio destro dell'albero \( T \) 
  \item \texttt{value(T)}: Restituisce il valore del nodo dell'albero \( T \)
\end{itemize}
Un albero è \textbf{bilanciato} quando la differenza tra l'altezza del sottoalbero sinistro
e di quello destro è al massimo 1. Un albero è \textbf{completo} quando tutti i livelli
sono completi, cioè tutti i nodi sono presenti, tranne l'ultimo, che può essere incompleto.

\vspace{1em}
\noindent
La profondità di un albero di \( n \) nodi è:
\[
P(n) = 1 + P\left( \left\lceil \frac{n-1}{2} \right\rceil \right) = \Theta(\log n)
\] 

\subsubsection{Albero binario di ricerca}
Un albero binario di ricerca è un albero binario in cui per ogni nodo \( x \) valgono
le seguenti proprietà:
\begin{itemize}
  \item Tutti i nodi nel sottoalbero sinistro di \( x \) hanno valore minore di \( x \) 
  \item Tutti i nodi nel sottoalbero destro di \( x \) hanno valore maggiore di \( x \)
\end{itemize}

\begin{figure}[H]
  \centering
  \begin{forest}
    for tree={
    circle,
    draw,
    minimum size=2em,
    inner sep=1pt,
    s sep=1cm,
  }
    [\( 25 \)
      [\( 17 \)
        [\( 2 \)]
        [\( 20 \)]
      ]
      [\( 30 \)
        [\( 27 \)]
        [\( 40 \)]
      ]
    ]
  \end{forest}
  \caption{Albero binario di ricerca}
\end{figure}

\noindent
Le operazioni possibili su un albero binario di ricerca sono:
\begin{itemize}
  \item \texttt{insert(T, x)}: Inserisce un nodo con valore \( x \) nell'albero \( T \)
  \item \texttt{extract(T, x)}: Rimuove il nodo con valore \( x \) dall'albero \( T \)
  \item \texttt{search(T, x)}: Restituisce vero se il valore \( x \) è 
    presente nell'albero \( T \)
  \item \texttt{min(T)}: Restituisce il valore minimo dell'albero \( T \)
  \item \texttt{max(T)}: Restituisce il valore massimo dell'albero \( T \)
\end{itemize}

\vspace{1em}
\noindent
Non è possibile inserire un nodo in un albero binario di ricerca in tempo logaritmico
perchè potrebbe dover essere inserito in modo da sbilanciare l'albero, quindi per
riottenere un albero bilanciato non rimane che spostare tutti gli elementi in una
posizione diversa.

Per ottenere la complessità logaritmica, bisogna utilizzare un albero che si può
sbilanciare, ma non troppo.

\begin{itemize}
  \item \textbf{Inserimento}
    Per inserire un nodo in un albero binario di ricerca si può procedere nel seguente modo:
    \begin{enumerate}
      \item Si parte dalla radice e si scende fino a trovare il nodo in cui inserire il
        nuovo nodo rispettando le proprietà dell'albero binario di ricerca
      \item Si inserisce il nodo in quel punto
    \end{enumerate}

  \item \textbf{Rimozione}
    Per rimuovere un elemento da un albero binario di ricerca si possono avere 3 casi:
    \begin{enumerate}
      \item Il nodo da rimuovere è una foglia: si può rimuovere direttamente
      \item Il nodo da rimuovere ha un solo figlio: si può sostituire il nodo con il figlio
      \item Il nodo da rimuovere ha due figli: si può sostituire il nodo con il minimo
        del sottoalbero destro o con il massimo del sottoalbero sinistro
    \end{enumerate}
    
\end{itemize}
Una volta che un nodo viene aggiunto o rimosso non si può più essere certi che 
l'albero sia ancora bilanciato, quindi bisogna fare in modo che l'albero venga
bilanciato ogni volta che viene modificato e la complessità non sarà più logaritmica.

\subsubsection{RB-Tree (Red-Black Tree)}
Gli alberi rosso-neri sono alberi binari di ricerca in cui ogni nodo può essere
rosso o nero.
\begin{itemize}
  \item \textbf{Nero}: Il nodo nero indica che il nodo è a posto
  \item \textbf{Rosso}: Il nodo rosso è un nodo ausiliario
\end{itemize}
Questo tipo di albero ha le seguenti proprietà:
\begin{enumerate}
  \item Ogni nodo è rosso o nero
  \item Ogni foglia è nera
  \item I figli di un nodo rosso sono neri
  \item Ogni cammino dalla radice ad una foglia contiene lo stesso numero di nodi neri
\end{enumerate}
\begin{example}
  Abbiamo un albero con \( l \) nodi neri nel sotto albero destro, quindi per la proprietà
  4, il sotto albero sinistro deve avere \( l \) nodi neri:
  \begin{figure}[H]
    \centering
    \begin{forest}
      for tree={
      circle,
      draw,
      minimum size=2em,
      inner sep=1pt,
      s sep=1cm,
    }
    [\color{black}b
      [\color{red}r
        [\color{black}b
        [\color{red}r
        [\color{black}b]
        ]
        ]
      ]
      [\color{black}b
        [\color{black}b ]
      ]
    ]
    \end{forest}
    \caption{Albero rosso-nero}
  \end{figure}

  \noindent Il sotto albero sinistro avrà \( 2l \) nodi totali.
\end{example}

\begin{example}
  \label{ex:rb-tree}
  Prendiamo ad esempio il seguente albero RB:
  \begin{figure}[H]
    \centering
    \begin{forest}
      for tree={
      circle,
      draw,
      minimum size=2em,
      inner sep=1pt,
      s sep=0.2cm,
      scale=0.8
    }
    [26
      [\color{red}17
        [14
          [\color{red}10
            [7
              [\color{red}3
                [nil]
                [nil]
              ]
            ]
            [12
              [nil]
              [nil]
            ]
          ]
          [16
            [\color{red}15
              [nil]
              [nil]
            ]
          ]
        ]
        [21
          [19
            [\color{red}20
              [nil]
              [nil]
            ]
          ]
          [23
            [nil]
            [nil]
          ]
        ]
      ]
      [41
        [\color{red}30
          [28
            [nil]
            [nil]
          ]
          [38
            [\color{red}33
              [nil]
              [nil]
            ]
            [\color{red}39
              [nil]
              [nil]
            ]
          ]
        ]
        [47
          [nil]
          [nil]
        ]
      ]
    ]
    \end{forest}
    \caption{Albero rosso-nero}
  \end{figure}

  \noindent
  Tutte le proprietà sono verificate, però tutte le foglie sono dei \texttt{nil} (nero)
  e questo è uno spreco di memoria, perchè metà dei nodi di un albero sono foglie e quindi
  metà dei nodi sono utilizzati per la sentinella.
  
  Questo si può risolvere facendo puntare tutte le sentinelle allo stesso valore \texttt{nil},
  però facendo così non si riesce più a risalire all'elemento padre di una foglia.
\end{example}

\noindent
Il concetto principale di questo tipo di alberi è quello della \textbf{black height},
cioè il numero di nodi neri che si incontrano lungo un cammino dalla radice ad una foglia.
Prendendo in considerazione l'esempio \ref{ex:rb-tree} i nodi:
\begin{itemize}
  \item \textbf{nil}: Hanno black height 0
  \item \textbf{3}: Ha black height 1
  \item \textbf{19}: Ha black height 1
  \item \textbf{14}: Ha black height 2
  \item \textbf{26}: Ha black height 3
  \item ecc...
\end{itemize}

\vspace{1em}
\noindent
\textbf{Lemma:}

\noindent
Per ogni nodo \( x \) dell'albero, il sottoalbero radicato in \( x \) contiene almeno
\( 2^{bh(x)} - 1 \) nodi interni (nodi che non sono una foglia).

\vspace{1em}
\noindent
\textbf{Dimostrazione:}

\noindent
Dimostriamo per induzione su \( bh(x) \):
\begin{itemize}
  \item \textbf{Caso base}: Se \( x \) è una foglia, allora \( bh(x) = 0 \) e il sottoalbero
    radicato in \( x \) contiene 0 nodi interni.
    \[
      2^{bh(x)} - 1 = 2^0 - 1 = 0
    \] 

  \item Se \( x \) è un nodo con un figlio destro \( b \) e un figlio sinistro
    \( a \) allora:
    \begin{figure}[H]
      \centering
      \begin{forest}
        for tree={
        circle,
        draw,
        minimum size=2em,
        inner sep=1pt,
        s sep=1cm,
      }
        [x
          [a]
          [b]
        ]
      \end{forest}
    \end{figure}
    \[
      bh(a) \ge bh(x) -1 \quad \text{e} \quad bh(b) \ge bh(x) -1
    \]
    \[
      \#nodi(x) \ge \#nodi(a) + \#nodi(b) + 1
    \] 
    \[
    \Downarrow
    \] 
    \[
      \begin{aligned}
        \#nodi(x) &\ge 2^{bh(a)} - 1 + 2^{bh(b)} - 1 + 1\\
                  &\ge 2^{bh(x) - 1} - 1 + 2^{bh(x) - 1} \cancel{- 1} \cancel{+ 1}\\
                  &= 2 \cdot 2^{bh(x) - 1} - 1\\
                  &= 2^{bh(x)} - 1
      \end{aligned}
    \] 
    Quindi
    \[
      \#nodi(x) \ge 2^{bh(x)} - 1
    \] 
\end{itemize}

\vspace{1em}
\noindent
\textbf{Complessità}:
Consideriamo un albero RB di altezza \( h \) (distanza tra la radice e la foglia più
lontana, escludendo la radice) e radice \( x \), si ha che \( bh(x) \) vale:
\[
  bh(x) \ge \frac{h}{2}
\] 
Il numero di nodi \( n \) (per il lemma) vale \( 2^{\frac{h}{2}} \), si ha quindi che
l'altezza di un RB albero è \textbf{almeno} il doppio dell'altezza di un albero binario:
\[
  \begin{aligned}
    n &\ge 2^{\frac{h}{2}-1}\\
    2^{\frac{h}{2}} &\le n+1\\
    \frac{h}{2} &\le \log_2(n+1)\\
    h &\le 2 \log_2(n+1)
  \end{aligned}
\] 

\begin{itemize}
  \item \textbf{Inserimento}: L'inserimento di un nodo in un albero RB si fa allo stesso
    modo di un albero binario di ricerca. Per non violare la proprietà 4, si fa
    di colore rosso e si salva un puntatore al nuovo oggetto, questo perchè il nuovo albero
    è un RB albero, \textbf{tranne} per l'oggetto puntato dal puntatore perchè potrebbe
    avere un padre rosso. Si propaga l'anomalia verso la radice dell'albero per poter
    cambiare il colore della radice mantenendo tutte le proprietà. Per risolvere il problema
    si può fare una \textbf{rotazione} a destra o a sinistra.

    \vspace{1em}
    \noindent
    Prendiamo in considerazione il seguente albero:
    \begin{figure}[H]
      \centering
      \begin{forest}
        for tree={
        circle,
        draw,
        minimum size=2em,
        inner sep=1pt,
        s sep=1cm,
      }
      [y
        [x
          [\( \alpha \)]
          [\( \beta \)]
        ]
        [\( \gamma  \)]
      ]
      \end{forest}
    \end{figure}

    \noindent
    La rotazione a destra di \( y \) è:
    \begin{figure}[H]
      \centering
      \begin{forest}
        for tree={
        circle,
        draw,
        minimum size=2em,
        inner sep=1pt,
        s sep=1cm,
      }
      [x
        [\( \alpha \)]
        [y
          [\( \beta \)]
          [\( \gamma \)]
        ]
      ]
      \end{forest}
    \end{figure}
    Questa operazione mantiene tutte le proprietà dell'albero binario di ricerca.
    L'operazione opposta è la rotazione a sinistra. Il tempo di esecuzione di una
    rotazione è costante.

    \vspace{1em}
    \noindent
    Lo pseudocodice dell'algoritmo per risistemare l'albero è il seguente:
\begin{lstlisting}[language=Scala]
// Albero con radice sicuramente nera
// x: Nodo da cui si propaga l'anomalia

while x != root and color(parent(x)) == red
  // Se il padre e' figlio sinistro del nonno
  if parent(x) == left(parent(parent(x)))
    y <- right(parent(parent(x))
    if color(y) == red
      color(parent(parent(x))) <- red
      color(parent(x)) <- black
      color(y) <- black
      x <- parent(parent(x))
    else
      if x == right(parent(x))
        x <- parent(x)
        left_rotate(x)

      color(parent(x)) <- black
      color(parent(parent(x))) <- red
      right_rotate(parent(parent(x)))
      x <- root

  // Se il padre e' figlio sinistro del nonno
  // Si fanno le stesso operazioni con le direzioni invertite
  if parent(x) == right(parent(parent(x)))
    y <- left(parent(parent(x))
    if color(y) == red
      color(parent(parent(x))) <- red
      color(parent(x)) <- black
      color(y) <- black
      x <- parent(parent(x))
    else
      if x == left(parent(x))
        x <- parent(x)
        right_rotate(x)

      color(parent(x)) <- black
      color(parent(parent(x))) <- red
      left_rotate(parent(parent(x)))
      x <- root
\end{lstlisting}

\begin{example}
  Un esempio del predcedente algoritmo è il seguente:
  \begin{figure}[H]
    \centering
    \begin{forest}
      for tree={
        circle,
        draw,
        minimum size=2em,
        inner sep=1pt,
        s sep=1cm,
      }
      [11
        [\color{red}2
          [1]
          [7
            [\color{red}5
            [\color{red}4 \color{black}\( (x) \)]
            ]
            [\color{red}8 \color{black}\( (y) \)]
          ]
        ]
        [14]
      ]
    \end{forest}
  \end{figure}
  \noindent
  Si cambiano i colori dei nodi \( parent(x) \) e \( y \) e si sposta l'anomalia verso
  l'alto
  \begin{figure}[H]
    \centering
    \begin{forest}
      for tree={
        circle,
        draw,
        minimum size=2em,
        inner sep=1pt,
        s sep=1cm,
      }
      [11
        [\color{red}2
          [1]
          [\color{red}7 \color{black}\( (x) \)
            [5
            [\color{red}4]
            ]
            [8 ]
          ]
        ]
        [14 \( (y) \) ]
      ]
    \end{forest}
  \end{figure}

  \noindent
  Prendiamo la coppia 2 e 7 e ruotiamo a sinistra
  \begin{figure}[H]
    \centering
    \begin{forest}
      for tree={
        circle,
        draw,
        minimum size=2em,
        inner sep=1pt,
        s sep=1cm,
      }
      [11
        [\color{red}7
        [\color{red}2 \color{black}\( (x) \)
          [1]
          [5
            [\color{red}4]
          ]
        ]
          [8
          ]
        ]
        [14 \( (y) \) ]
      ]
    \end{forest}
  \end{figure}
  \noindent
  Si cambiano i colori e si effettua una rotazione a destra
  \begin{figure}[H]
    \centering
    \begin{forest}
      for tree={
        circle,
        draw,
        minimum size=2em,
        inner sep=1pt,
        s sep=1cm,
      }
      [7
        [\color{red}2
          [1]
          [5
            [\color{red}4]
          ]
        ]
        [\color{red}11
          [8]
          [14]
        ]
      ]
    \end{forest}
  \end{figure}
  \noindent
  L'albero ora non ha più l'anomalia.
\end{example}
\end{itemize}


\subsection{Lista doppiamente puntata}
Una lista doppiamente puntata è una struttura dati in cui ogni nodo ha un puntatore
al nodo precedente e al nodo successivo.

\vspace{1em}
\noindent
Per rimuovere un elemento \( x \) dalla lista bisogna:
\begin{enumerate}
  \item Trovare il nodo \( x \)
  \item Collegare il nodo precedente a \( x \) con il nodo successivo a \( x \)
\end{enumerate}
Il problema di questa soluzione è che bisogna anche gestire i casi degli estremi
separatamente. Per evitare di gestire i casi specifici si possono aggiungere dei
nodi sentinella all'inizio e alla fine della lista.


\end{document}
