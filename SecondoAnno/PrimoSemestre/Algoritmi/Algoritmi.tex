\documentclass[a4paper]{article}

\usepackage[utf8]{inputenc}
\usepackage[T1]{fontenc}
\usepackage{textcomp}
\usepackage[italian]{babel}
\usepackage{amsmath, amssymb}
\usepackage{booktabs,xltabular}
\usepackage{amsfonts}
\usepackage{cancel}
\usepackage{mdframed}
\usepackage{makecell}
\usepackage{float}
\usepackage{xcolor}
\usepackage{listings}
\usepackage{graphicx}
\usepackage{tikz}
\usetikzlibrary{shapes, arrows, automata, petri, decorations.pathreplacing, positioning, calc}
\usepackage{circuitikz}
\usepackage[label=corner]{karnaugh-map}
\graphicspath{{./figures/}}

\usepackage{ntheorem}
\newtheorem{theorem}{Teorema}

\usepackage{import}
\usepackage{pdfpages}
\usepackage{transparent}
\usepackage{xcolor}

\usepackage{hyperref}
\hypersetup{
  colorlinks=false,
}

% Code blocks
\definecolor{codegreen}{rgb}{0,0.6,0}
\definecolor{codegray}{rgb}{0.5,0.5,0.5}
\definecolor{codepurple}{rgb}{0.58,0,0.82}
\definecolor{backcolour}{rgb}{0.95,0.95,0.95}

\lstdefinestyle{mystyle}{
  backgroundcolor=\color{backcolour},
  commentstyle=\color{codegreen},
  keywordstyle=\color{magenta},
  numberstyle=\tiny\color{codegray},
  stringstyle=\color{codepurple},
  basicstyle=\ttfamily\footnotesize,
  breakatwhitespace=false,
  breaklines=true,
  captionpos=b,
  keepspaces=true,
  numbers=left,
  numbersep=5pt,
  showspaces=false,
  showstringspaces=false,
  showtabs=false,
  tabsize=2
}

\lstset{style=mystyle}

\usepackage{color}

\definecolor{dkgreen}{rgb}{0,0.6,0}
\definecolor{gray}{rgb}{0.5,0.5,0.5}
\definecolor{mauve}{rgb}{0.58,0,0.82}


\usepackage{import}
\usepackage{pdfpages}
\usepackage{transparent}
\usepackage{xcolor}



% Useful definitions frame
\theoremstyle{break}
\theoremheaderfont{\bfseries}
\newmdtheoremenv[%
linecolor=gray,leftmargin=0,%
rightmargin=0,
innertopmargin=8pt,%
innerbottommargin=8pt,
ntheorem]{define}{Definizioni utili}[section]

% Example frame
\theoremstyle{break}
\theoremheaderfont{\bfseries}
\newmdtheoremenv[%
linecolor=gray,leftmargin=0,%
rightmargin=0,
innertopmargin=8pt,%
innerbottommargin=8pt,
ntheorem]{example}{Esempio}[section]

% Important definition frame
\theoremstyle{break}
\theoremheaderfont{\bfseries}
\newmdtheoremenv[%
linecolor=gray,leftmargin=0,%
rightmargin=0,
backgroundcolor=gray!40,%
innertopmargin=8pt,%
innerbottommargin=8pt,
ntheorem]{definition}{Definizione}[section]

% Exercise frame
\theoremstyle{break}
\theoremheaderfont{\bfseries}
\newmdtheoremenv[%
linecolor=gray,leftmargin=0,%
rightmargin=0,
innertopmargin=8pt,%
innerbottommargin=8pt,
ntheorem]{exercise}{Esercizio}[section]

% figure support
\usepackage{import}
\usepackage{xifthen}
\pdfminorversion=7
\usepackage{pdfpages}
\usepackage{transparent}
\newcommand{\incfig}[1]{%
  \def\svgwidth{\columnwidth}
  \import{./figures/}{#1.pdf_tex}
}

% FSM tikz
\tikzset{
  place/.style={
    circle,
    thick,
    draw=black,
    minimum size=6mm,
  },
  state/.style={
    circle,
    thick,
    draw=blue!75,
    fill=blue!20,
    minimum size=6mm,
  },
}

\pdfsuppresswarningpagegroup=1

\begin{document}

\begin{titlepage}
	\begin{center}
		\vspace*{1cm}

		\Huge
		\textbf{Probabilità e Statistica\\Esercizi}

		\vspace{0.5cm}
		\LARGE
		UniVR - Dipartimento di Informatica

		\vspace{1.5cm}

		\textbf{Fabio Irimie}

		\vfill


		\vspace{0.8cm}


		2° Semestre 2023/2024

	\end{center}
\end{titlepage}


\tableofcontents
\pagebreak

\section{Introduzione}
Un'algoritmo è una sequenza \textbf{finita} di \textbf{istruzioni} volta a risolvere un problema.
Per implementarlo nel pratico si scrive un \textbf{programma}, cioè l'applicazione di
un linguaggio di programmazione, oppure si può descrivere in modo informale
attraverso del \textbf{pseudocodice} che non lo implementa in modo preciso,
ma spiega i passi per farlo.
\\
Ogni algoritmo può essere implementato in modi diversi, sta al programmatore
capire qual'è l'opzione migliore e scegliere in base alle proprie necessità.

\subsection{Confronto tra algoritmi}
Ogni algoritmo si può confrontare con gli altri in base a tanti fattori, come:
\begin{itemize}
  \item \textbf{Complessità}: quanto ci vuole ad eseguire l'algoritmo
  \item \textbf{Memoria}: quanto spazio in memoria occupa l'algoritmo
\end{itemize}

\subsection{Rappresentazione dei dati}
Per implementare un algoritmo bisogna riuscire a strutturare i dati in maniera tale
da riuscire a manipolarli in modo efficiente.

\section{Calcolo della complessità}
La complessità di un algoritmo mette in relazione il numero di istruzioni da eseguire
con la dimensione del problema, e quindi è una funzione che dipende dalla dimensione
del problema.

\vspace{1em}
\noindent
La \textbf{dimensione del problema} è un insieme di oggetti adeguato a dare un idea
chiara di quanto è grande il problema da risolvere, ma sta a noi decidere come
misurare il problema.

\noindent
Ad esempio una matrice è più comoda da misurare come il numero di righe e il numero
di colonne, al posto di misurarla come il numero di elementi totali.

\vspace{1em}
\noindent
La complessità di solito si calcola come il \textbf{caso peggiore}, cioè il
limite superiore di esecuzione dell'algoritmo.

\subsection{Linguaggi di programmazione}
Ogni linguaggio di programmazione è formato da diversi blocchi:
\begin{enumerate}
  \item \textbf{Blocco iterativo}: un tipico blocco di codice eseguito sequenzialmente
    e tipicamente finisce con un punto e virgola.
  \item \textbf{Blocco condizionale}: un blocco di codice che viene eseguito solo
    se una condizione è vera.
  \item \textbf{Blocco iterativo}: un blocco di codice che viene eseguito
    ripetutamente finché una condizione è vera.
\end{enumerate}

\noindent
Questi sono i blocchi base della programmazione e se riusciamo a calcolare
la complessità di ognuno di questi blocchi possiamo calcolare più facilmente
la complessità di un intero algoritmo.

\subsubsection{Blocchi iterativi}
\[
  I_1 \;\; c_1(n)
\] 
\[
  I_2 \;\; c_2(n)
\] 
\[
  \vdots \;\;\;\;\;\; \vdots
\] 
\[
  I_l \;\; c_l(n)
\]
Se ogni blocco ha complessità \( c_1(n) \), allora la complessità totale è data
da:
\[
\sum_{i=1}^{l} c_i(n)
\] 

\subsubsection{Blocchi condizionali}
\[
  \text{IF cond} \;\; c_{cond}(n)
\] 
\[
  I_1 \quad \quad c_1(n)
\] 
\[
  \hspace{-1.75cm} \text{ELSE}
\] 
\[
  I_2 \quad \quad c_2(n)
\] 
La complessità totale è data da:
\[
  c(n) = c_{cond}(n) + \max(c_1(n), c_2(n))
\] 
A volte la condizione è un test sulla dimensione del problema e in quel caso si
può scrivere una complessità più precisa.

\subsubsection{Blocchi iterativi}
\[
  \text{WHILE cond} \;\; c_{cond}(n)
\] 
\[
  I \hspace{1.6cm} c_0(n)
\] 
Si cerca di trovare un limite superiore \( m \) al limite di iterazioni.

\vspace{1em}
\noindent
Di conseguenza la complessità totale è data da:
\[
  c_{cond}(n) + m(c_{cond}(n) + c_0(n))
\]

\subsection{Esempio}
\begin{figure}[H]
  \begin{example}
    Calcoliamo la complessità della moltiplicazione tra 2 matrici:
    \[
      A_{n \times m} \cdot B_{m \times l} = C_{n \times l}
    \] 
    L'algoritmo è il seguente:
    \begin{lstlisting}[language=Scala]

  for i <- 1 to n // n ( 5 ml + 4l + 2) + n + 1
    for j <- 1 to l // l (5m + 2 + 1) + 1 + l 
      c[i][j] <- 0
      for k <- 1 to m // (m + 1 + m(4))
        // 3 (moltiplicazione, somma e assegnamento)
        // 1 (incremento for) 
        c[i][j] += a[i][k] * b[k][j]
    \end{lstlisting}

    \noindent
    Partiamo calcolando la complessità del ciclo for più interno. Non ha
    senso tenere in considerazione tutti i dati, ma solo quelli rilevanti. In
    questo caso avremo:
    \[
      (m + 1 + m(4)) = 5m + 1
    \] 
    Questa complessità contiene informazioni poco rilevanti perchè possono far
    riferimento alla velocità della cpu e un millisecondo in più o in meno non cambia
    nulla se teniamo in considerazione solo l'incognita abbiamo:
    \[
      m
    \]
    Questo semplifica molto i calcoli, rendendo meno probabili gli errori. Siccome
    la complessità si calcola su numeri molto grandi, le costanti piccole prima o poi
    verranno tolte perchè poco influenti.

    \vspace{1em}
    \noindent
    La complessità totale alla fine sarebbe stata:
    \[
      5nml+4ml+2n+n+1
    \] 
    Ma ciò che ci interessa veramente è:
    \[
      5\color{red}nml\color{black}+4ml+2n+n+1
    \] 
    Se non consideriamo le costanti inutili, la complessità finale è:
    \[
      nml
    \]
    Nella maggior perte dei casi ci si concentra soltanto sull'ordine di grandezza
    della complessità, e non sulle costanti.
  \end{example}
\end{figure}

\subsection{Ordine di grandezza}
L'ordine di grandezza è una funzione che approssima la complessità di un algoritmo:
\[
f \in O(g)
\] 
\[
  \exists c > 0\; \exists \bar{n}\;\; \forall n \ge \bar{n}\;\; f(n) \le c g(n)
\] 
\label{07-10-D1}

\[
f \in \Omega(g)
\] 
Da fare a casa
\end{document}
