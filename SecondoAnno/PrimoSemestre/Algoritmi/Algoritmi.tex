\documentclass[a4paper]{article}
\usepackage{import}
\usepackage[utf8]{inputenc}
\usepackage[T1]{fontenc}
\usepackage{textcomp}
\usepackage[italian]{babel}
\usepackage{amsmath, amssymb}
\usepackage{booktabs,xltabular}
\usepackage{amsfonts}
\usepackage{subcaption}
\usepackage{amsthm}
\usepackage{cancel}
\usepackage{mdframed}
\usepackage{makecell}
\usepackage{float}
\usepackage{xcolor}
\usepackage{listings}
\usepackage{gensymb}
\usepackage{graphicx}
\usepackage{bodeplot}
\usepackage{physics}
\usepackage{tikz}
\usetikzlibrary{shapes, arrows, automata, petri, decorations.markings, decorations.pathreplacing, positioning, calc, quotes}
\usepackage{circuitikz}
\usepackage[label=corner]{karnaugh-map}
\graphicspath{{./figures/}}

% Set default font to sans-serif
\renewcommand{\familydefault}{\sfdefault} 
\usepackage{eulervm}

\usepackage{forest}

\usepackage{mathtools}
\DeclarePairedDelimiter\ceil{\lceil}{\rceil}
\DeclarePairedDelimiter\floor{\lfloor}{\rfloor}

% \usepackage{ntheorem}

\usepackage{import}
\usepackage{pdfpages}
\usepackage{transparent}
\usepackage{xcolor}

\usepackage{hyperref}
\hypersetup{
    colorlinks=false,
}

% Code blocks
\definecolor{codegreen}{rgb}{0,0.6,0}
\definecolor{codegray}{rgb}{0.5,0.5,0.5}
\definecolor{codepurple}{rgb}{0.58,0,0.82}
\definecolor{backcolour}{rgb}{0.95,0.95,0.95}

\lstdefinestyle{mystyle}{
	backgroundcolor=\color{backcolour},
	commentstyle=\color{codegreen},
	keywordstyle=\color{magenta},
	numberstyle=\tiny\color{codegray},
	stringstyle=\color{codepurple},
	basicstyle=\ttfamily\footnotesize,
	breakatwhitespace=false,
	breaklines=true,
	captionpos=b,
	keepspaces=true,
	numbers=left,
	numbersep=5pt,
	showspaces=false,
	showstringspaces=false,
	showtabs=false,
	tabsize=2
}

\lstset{style=mystyle}

\usepackage{color}
\usepackage{import}
\usepackage{pdfpages}
\usepackage{transparent}
\usepackage{xcolor}

% Example frame
\theoremstyle{definition}
\newmdtheoremenv[%
	linecolor=gray,leftmargin=0,%
	rightmargin=0,
	innertopmargin=8pt,%
	innerbottommargin=8pt,
	ntheorem]{example}{Esempio}[section]

% Important definition frame
\theoremstyle{definition}
\newmdtheoremenv[%
	linecolor=gray,leftmargin=0,%
	rightmargin=0,
	backgroundcolor=gray!40,%
	innertopmargin=8pt,%
	innerbottommargin=8pt,
	ntheorem]{definition}{Definizione}[section]

% Exercise frame
\theoremstyle{definition}
\newmdtheoremenv[%
	linecolor=gray,leftmargin=0,%
	rightmargin=0,
	innertopmargin=8pt,%
	innerbottommargin=8pt,
	ntheorem]{exercise}{Esercizio}[section]

% Theorem frame
\theoremstyle{definition}
\newmdtheoremenv[%
  linecolor=gray,leftmargin=0,%
  rightmargin=0,
  innertopmargin=8pt,%
  innerbottommargin=8pt,
  ntheorem]{theorem}{Teorema}[section]

\theoremstyle{definition}
\newmdtheoremenv[%
  linecolor=white,leftmargin=0,%
  rightmargin=0,
  innertopmargin=8pt,%
  innerbottommargin=8pt,
  ntheorem]{define}{Definizione utile}[section]

% figure support
\usepackage{import}
\usepackage{xifthen}
\pdfminorversion=7
\usepackage{pdfpages}
\usepackage{transparent}
\newcommand{\incfig}[1]{%
	\def\svgwidth{\columnwidth}
	\import{./figures/}{#1.pdf_tex}
}

% FSM tikz
\tikzset{
    place/.style={
        circle,
        thick,
        draw=black,
        minimum size=6mm,
    },
        state/.style={
        circle,
        thick,
        draw=black,
        fill=white,
        minimum size=6mm,
    },
}

\pdfsuppresswarningpagegroup=1

\usepackage{pgfplots}
\pgfplotsset{compat=1.18,width=10cm}

% Save plots as pdf and reuse them without compiling every time
\usetikzlibrary{external}
\tikzexternalize[prefix=figures/tikz/, optimize=false]


\begin{document}

\begin{titlepage}
	\begin{center}
		\vspace*{1cm}

		\Huge
		\textbf{Probabilità e Statistica\\Esercizi}

		\vspace{0.5cm}
		\LARGE
		UniVR - Dipartimento di Informatica

		\vspace{1.5cm}

		\textbf{Fabio Irimie}

		\vfill


		\vspace{0.8cm}


		2° Semestre 2023/2024

	\end{center}
\end{titlepage}


\tableofcontents
\pagebreak

\section{Introduzione}
Un'algoritmo è una sequenza \textbf{finita} di \textbf{istruzioni} volta a risolvere un problema.
Per implementarlo nel pratico si scrive un \textbf{programma}, cioè l'applicazione di
un linguaggio di programmazione, oppure si può descrivere in modo informale
attraverso del \textbf{pseudocodice} che non lo implementa in modo preciso,
ma spiega i passi per farlo.
\\
Ogni algoritmo può essere implementato in modi diversi, sta al programmatore
capire qual'è l'opzione migliore e scegliere in base alle proprie necessità.

\subsection{Confronto tra algoritmi}
Ogni algoritmo si può confrontare con gli altri in base a tanti fattori, come:
\begin{itemize}
  \item \textbf{Complessità}: quanto ci vuole ad eseguire l'algoritmo
  \item \textbf{Memoria}: quanto spazio in memoria occupa l'algoritmo
\end{itemize}

\subsection{Rappresentazione dei dati}
Per implementare un algoritmo bisogna riuscire a strutturare i dati in maniera tale
da riuscire a manipolarli in modo efficiente.

\section{Calcolo della complessità}
La complessità di un algoritmo mette in relazione il numero di istruzioni da eseguire
con la dimensione del problema, e quindi è una funzione che dipende dalla dimensione
del problema.

\vspace{1em}
\noindent
La \textbf{dimensione del problema} è un insieme di oggetti adeguato a dare un idea
chiara di quanto è grande il problema da risolvere, ma sta a noi decidere come
misurare il problema.

\noindent
Ad esempio una matrice è più comoda da misurare come il numero di righe e il numero
di colonne, al posto di misurarla come il numero di elementi totali.

\vspace{1em}
\noindent
La complessità di solito si calcola come il \textbf{caso peggiore}, cioè il
limite superiore di esecuzione dell'algoritmo.

\subsection{Linguaggi di programmazione}
Ogni linguaggio di programmazione è formato da diversi blocchi:
\begin{enumerate}
  \item \textbf{Blocco iterativo}: un tipico blocco di codice eseguito sequenzialmente
    e tipicamente finisce con un punto e virgola.
  \item \textbf{Blocco condizionale}: un blocco di codice che viene eseguito solo
    se una condizione è vera.
  \item \textbf{Blocco iterativo}: un blocco di codice che viene eseguito
    ripetutamente finché una condizione è vera.
\end{enumerate}

\noindent
Questi sono i blocchi base della programmazione e se riusciamo a calcolare
la complessità di ognuno di questi blocchi possiamo calcolare più facilmente
la complessità di un intero algoritmo.

\subsubsection{Blocchi iterativi}
\[
  I_1 \;\; c_1(n)
\] 
\[
  I_2 \;\; c_2(n)
\] 
\[
  \vdots \;\;\;\;\;\; \vdots
\] 
\[
  I_l \;\; c_l(n)
\]
Se ogni blocco ha complessità \( c_1(n) \), allora la complessità totale è data
da:
\[
\sum_{i=1}^{l} c_i(n)
\] 

\subsubsection{Blocchi condizionali}
\[
  \text{IF cond} \;\; c_{cond}(n)
\] 
\[
  I_1 \quad \quad c_1(n)
\] 
\[
  \hspace{-1.75cm} \text{ELSE}
\] 
\[
  I_2 \quad \quad c_2(n)
\] 
La complessità totale è data da:
\[
  c(n) = c_{cond}(n) + \max(c_1(n), c_2(n))
\] 
A volte la condizione è un test sulla dimensione del problema e in quel caso si
può scrivere una complessità più precisa.

\subsubsection{Blocchi iterativi}
\[
  \text{WHILE cond} \;\; c_{cond}(n)
\] 
\[
  I \hspace{1.6cm} c_0(n)
\] 
Si cerca di trovare un limite superiore \( m \) al limite di iterazioni.

\vspace{1em}
\noindent
Di conseguenza la complessità totale è data da:
\[
  c_{cond}(n) + m(c_{cond}(n) + c_0(n))
\]

\subsection{Esempio}
\begin{figure}[H]
  \begin{example}
    Calcoliamo la complessità della moltiplicazione tra 2 matrici:
    \[
      A_{n \times m} \cdot B_{m \times l} = C_{n \times l}
    \] 
    L'algoritmo è il seguente:
    \begin{lstlisting}[language=Scala]

  for i <- 1 to n // n ( 5 ml + 4l + 2) + n + 1
    for j <- 1 to l // l (5m + 2 + 1) + 1 + l 
      c[i][j] <- 0
      for k <- 1 to m // (m + 1 + m(4))
        // 3 (moltiplicazione, somma e assegnamento)
        // 1 (incremento for) 
        c[i][j] += a[i][k] * b[k][j]
    \end{lstlisting}

    \noindent
    Partiamo calcolando la complessità del ciclo for più interno. Non ha
    senso tenere in considerazione tutti i dati, ma solo quelli rilevanti. In
    questo caso avremo:
    \[
      (m + 1 + m(4)) = 5m + 1
    \] 
    Questa complessità contiene informazioni poco rilevanti perchè possono far
    riferimento alla velocità della cpu e un millisecondo in più o in meno non cambia
    nulla se teniamo in considerazione solo l'incognita abbiamo:
    \[
      m
    \]
    Questo semplifica molto i calcoli, rendendo meno probabili gli errori. Siccome
    la complessità si calcola su numeri molto grandi, le costanti piccole prima o poi
    verranno tolte perchè poco influenti.

    \vspace{1em}
    \noindent
    La complessità totale alla fine sarebbe stata:
    \[
      5nml+4ml+2n+n+1
    \] 
    Ma ciò che ci interessa veramente è:
    \[
      5\color{red}nml\color{black}+4ml+2n+n+1
    \] 
    Se non consideriamo le costanti inutili, la complessità finale è:
    \[
      nml
    \]
    Nella maggior perte dei casi ci si concentra soltanto sull'ordine di grandezza
    della complessità, e non sulle costanti.
  \end{example}
\end{figure}

\subsection{Ordine di grandezza}
L'ordine di grandezza è una funzione che approssima la complessità di un algoritmo:
\[
f \in O(g)
\] 
\[
  \exists c > 0\; \exists \bar{n}\;\; \forall n \ge \bar{n}\;\; f(n) \le c g(n)
\] 
\begin{figure}[H]
  \centering
  \begin{tikzpicture}
    % axis
    \draw[->] (-0.2,0) -- (6,0) node[right] {$t$};
    \draw[->] (0,-0.2) -- (0,5) node[above] {$y$};

    \draw[red, domain=0:5, samples=100, smooth] plot ({\x},{exp(\x/3)}) node[right] {$cg$};
    \draw[blue, domain=0:5.3, samples=100, smooth] plot ({\x},{exp(\x/5) + sin(deg(5*\x))/2}) node[right] {f};

    \draw[dashed] (1.72,1.77) -- (1.75,0) node[below] {$\bar{n}$};
  \end{tikzpicture}
  \caption{Esempio di funzione \(f \in O(g)\)}
\end{figure}

\[
f \in \Omega(g)
\] 
\[
  \exists c > 0\; \exists \bar{n}\;\; \forall n \ge \bar{n}\;\; f(n) \ge cg(n)
\] 

\[
f \in \Theta(g)
\] 
\[
  f \in O(g) \land f \in \Omega(g)
\]

\vspace{1em}
\noindent
Per gli algoritmi:
\begin{figure}[H]
  \begin{definition}
    \[
      A \in O(f)
    \] 
    So che l'algoritmo \( A \) termina entro il tempo definito dalla funzione \( f \).
    Di conseguenza se un algoritmo termina entro un tempo \( f \) allora sicuramente
    termina entro un tempo \( g \) più grande. Ad esempio:
    \[
      A \in O(n) \Rightarrow A \in O(n^2)
    \] 
    Questa affermazione è \textbf{corretta}, ma \textbf{non accurata}.

    \vspace{1em}
    \[
      A \in \Omega(f)
    \] 
    Significa che esiste uno schema di input tale che se \( g(n) \) è il numero di
    passi necessari per risolvere l'istanza \( n \) allora:
    \[
      g \in \Omega(f)
    \] 
    Quindi l'algoritmo non termina in un tempo minore di \( f \).

    \vspace{1em}
    \noindent
    Calcolando la complessità si troverà lo schema di input tale che:
    \[
      g \in O(f)
    \]
    cioè il limite superiore di esecuzione dell'algoritmo.

    \noindent
    Successivamente ci si chiede se esistono algoritmi migliori e si 
    troverà lo schema di input tale che:
    \[
      g \in \Omega(f)
    \]
    cioè il limite inferiore di esecuzione dell'algoritmo.

    \noindent
    Se i due limiti coincidono allora:
    \[
      g \in \Theta(f)
    \]
    abbiamo trovato il tempo di esecuzione dell'algoritmo.
  \end{definition}
\end{figure}

\begin{figure}[H]
  \begin{theorem}[Teorema di Skolem]
    Se c'è una formula che vale coi quantificatori esistenziali, allora nel linguaggio
    si possono aggiungere delle costanti al posto delle costanti quantificate e assumere
    che la formula sia valida con quelle costanti.
  \end{theorem}
\end{figure}

\subsubsection{Esempi di dimostrazioni}
\begin{figure}[H]
  \begin{example}
    È vero che \( n \in O(2n) \)?

    \noindent
    Se prendiamo \( c = 1 \) e \( \bar{n} = 1 \) allora:
    \[
    n \le c2n
    \] 
    Quindi è vero
  \end{example}

\end{figure}
\begin{figure}[H]
\begin{example}
  È vero che \( 2n \in O(n) \)?

  \noindent
  Se prendiamo \( c = 2 \) e \( \bar{n} = 1 \) allora:
  \[
    2n \le 2n
  \] 
  Quindi è vero
\end{example}
\end{figure}
\begin{figure}[H]
  \begin{example}
    È vero che \( f \in O(g) \iff g \in \Omega(f) \)?

    \noindent
    Dimostro l'implicazione da entrambe le parti:
    \begin{itemize}
      \item \( \to \): Usando il teorema di Skolem:
        \[
          \forall n \ge \bar{n}\;\; f(n) \le cg(n)
        \] 
        Trasformo la disequazione:
        \[
          \forall n \ge \bar{n}\;\; \frac{f(n)}{c} \le g(n)
        \] 
        \[
          \forall n \ge \bar{n}\;\; g(n) \ge \frac{f(n)}{c}
        \] 
        \[
          \forall n \ge \bar{n}\;\; g(n) \ge \frac{1}{c} f(n) \quad \square
        \] 
        Se la definizione di \( \Omega(g) \) è:
        \[
          \exists c' > 0\; \exists \bar{n}'\;\; \forall n \ge \bar{n}'\;\; f(n) \ge c'g(n)
        \]
        sappiamo che:
        \[
          c' = \frac{1}{c}
        \] 
      \item \( \leftarrow \): Usando il teorema di Skolem:
        \[
          \forall n \ge \bar{n}'\;\; g(n) \ge c'f(n)
        \] 
        Trasformo la disequazione:
        \[
          \forall n \ge \bar{n}'\;\; \frac{g(n)}{c'}\ge f(n)
        \] 
        \[
          \forall n \ge \bar{n}'\;\; f(n) \le \frac{1}{c'} g(n) \quad \square
        \] 
    \end{itemize}
  \end{example}
\end{figure}

\begin{figure}[H]
  \begin{example}
    \[
    f_1 \in O(g) \; f_2 \in O(g) \Rightarrow f_1 + f_2 \in O(g)
    \] 
    Dimostrazione:

    Ipotesi
    \[
      \bar{n}_1 c_1\; \forall n > n_1 \quad f_1(n) \le c_1 g(n)
    \] 
    \[
      \bar{n}_1 c_2\; \forall n > n_2 \quad f_2(n) \le c_2 g(n)
    \] 

    \[
    f_1(n) + f_2(n) \le (c_1 + c_2)g(n) \quad \square
    \] 
    Quindi:
    \[
    c = (c_1 + c_2)
    \] 
    \[
      \bar{n} = \max(\bar{n}_1, \bar{n}_2)
    \] 
  \end{example}
\end{figure}

\begin{figure}[H]
  \begin{example}
    Se
    \[
    f_1 \in O(g_1) \; f_2 \in O(g_2)
    \] 
    è vero che:
    \[
    f_1 \cdot f_2 \in O(g_1 \cdot g_2)
    \] 
    Dimostrazione:

    Ipotesi
    \[
      \bar{n}_1 c_1\; \forall n > \bar{n}_1 \quad f_1(n) \le c_1 g_1(n)
    \] 
    \[
      \bar{n}_2 c_2\; \forall n > \bar{n}_2 \quad f_2(n) \le c_2 g_2(n)
    \]

    \[
      f_1(n) \cdot f_2(n) \le (c_1 \cdot c_2) (g_1(n) \cdot g_2(n)) \quad \square
    \] 
    Quindi:
    \[
      c = c_1 \cdot c_2
    \]
    \[
      \bar{n} = \max(\bar{n}_1, \bar{n}_2)
    \]
  \end{example}
\end{figure}

\section{Studio degli algoritmi}
Il problema dell'ordinamento si definisce stabilendo la relazione che deve esistere tra
\textbf{input} e \textbf{output} del sistema.
\begin{itemize}
  \item \textbf{Input}: Sequenza \( (a_1,\ldots,a_n) \) di oggetti su cui è definita una
    relazione di ordinamento, cioè l'unico modo per capire la differenza tra due oggetti
    è confrontarli.

  \item \textbf{Output}: Permutazione \( (a'_1,\ldots,a'_n) \) di \( (a_1,\ldots,a_n) \) 
    tale che:
    \[
    \forall i < j \;\; a'_i \le a'_j
    \] 
\end{itemize}
L'obiettivo è trovare un algoritmo che segua la relazione di ordinamento definita e risolva
il problema nel minor tempo possibile.

\subsection{Insertion sort}
Divide la sequenza in due parti:
\begin{itemize}
  \item \textbf{Parte ordinata}: Sequenza di elementi ordinati
  \item \textbf{Parte non ordinata}: Sequenza di elementi non ordinati
\end{itemize}
\begin{figure}[H]
  \centering
  \begin{tikzpicture}
    \draw[fill, fill opacity=0.2] (0,0) rectangle (3,1) node[midway, black, opacity=1]
      {Ordinato};
    \draw (3,1) rectangle (8,0) node[midway] {Non ordinato};

    \draw[<-] (3.5,0) -- ++(0,-0.5) node[below] {j};
    \draw[<-] (3,0) -- ++(0,-0.5) node[below] {i};
  \end{tikzpicture}
  \caption{Parte ordinata e non ordinata}
\end{figure}

\noindent
Pseudocodice:
\begin{lstlisting}[language=Scala]
insertion_sort(A)
  for j <- 2 to length[A] // A sinistra di j e tutto ordinato-
    key <- A[j]                //                            |
    i <- j - 1                 //                            | O(n)
    while i > 0 and A[i] > key // --                         |
      A[i + 1] <- A[i]         //  | O(n)                    |
      i--                      // --                     -----
    A[i + 1] <- key            
\end{lstlisting}

\noindent
La complessità di questo algoritmo è:
\[
  O(n^2)
\] 
Per capirlo è sufficiente guardare il numero di cicli nidificati e quante volte eseguono
il codice all'interno.

\vspace{1em}
\noindent
Se l'array è già ordinato la complessità è:
\[
\Omega(n)
\] 
Con l'input peggiore possibile la complessità è:
\[
\Omega(n^2)
\]
di conseguenza, visto che vale \( O(n^2) \) e \( \Omega(n^2) \) vale:
\[
\Theta(n^2)
\] 

Quanto spazio in memoria utilizza questo algoritmo?
\begin{itemize}
  \item Variabile j
  \item Variabile i
  \item Variabile key
\end{itemize}
A prescindere da quanto è grande l'array utilizzato, di conseguenza la memoria utilizzata
è costante.

\begin{itemize}
  \item \textbf{Ordinamento in loco}: se la quantità di memoria extra che deve usare 
    non dipende dalla dimensione del problema allora si dice che l'algoritmo è in loco.

  \item \textbf{Ordinamento non in loco}: se la quantità di memoria extra che deve usare
    dipende dalla dimensione del problema allora si dice che l'algoritmo è non in loco.

  \item \textbf{Stabilità}: La posizione relativa di elementi uguali non viene modificata
\end{itemize}
L'insertion sort ordina in loco ed è stabile.

\subsection{Fattoriale}
\begin{lstlisting}[language=Scala]
Fatt(n)
  if n = 0
    ret 1
  else 
    ret n * Fatt(n - 1)
\end{lstlisting}
L'argomento della funzione ci fa capire la complessità dell'algoritmo:
\[
  T(n) = \begin{cases}
    1 & \text{se } n = 0 \\
    T(n - 1) + 1 & \text{se } n > 0
  \end{cases}
\] 
Con problemi ricorsivi si avrà una complessità con funzioni definite ricorsivamente.
Questo si risolve induttivamente:
\[
  \begin{aligned}
    T(n) & = 1 + T(n-1)\\
         & = 1 + 1 + T(n-2)\\
         & = 1 + 1 + 1 + T(n-3)\\
         & = \underbrace{1 + 1 + \ldots + 1}_{i} + T(n-i)\\
  \end{aligned}
\] 
La condizione di uscita è: \( n-i = 0 \quad n = i \) 
\[
\begin{aligned}
         & = \underbrace{1 + 1 + \ldots + 1}_{n} + T(n-n)\\
         & = n + 1 = \Theta(n)
\end{aligned}
\] 
Questo si chiama passaggio iterativo.

\begin{example}
  \[
    T(n) = 2T\left(\floor*{\frac{n}{2}}\right) + n
  \] 
  Questa funzione si può riscrivere come:
  \[
  T(n) = \begin{cases}
    \text{Costante} & \text{se } n < a \\
    2T\left(\floor*{\frac{n}{2}}\right) + n & \text{se } n \ge a
  \end{cases}
  \] 

  \vspace{1em}
  \noindent
  Se la complessità fosse già data bisognerebbe soltanto verificare se è corretta.
  Usando il metodo di sostituzione:
  \[
    T(n) = cn \log n
  \] 
  sostituiamo nella funzione di partenza:
  \[
    \begin{aligned}
      T(n)  & = 2T\left(\floor*{\frac{n}{2}}\right) + n\\
            & \le 2c\left(\floor*{\frac{n}{2}}\right) \log \floor*{\frac{n}{2}} + n\\
            & \le \cancel{2} c \frac{n}{\cancel{2}} \log \frac{n}{2} + n\\
            & = cn \log n - cn \log 2 + n\\
            & \stackrel{?}{\le} cn \log n \quad \text{se } n- cn \log 2 \le 0\\
    \end{aligned}
  \] 
  \[
    c \ge \frac{n}{n \log 2} = \frac{1}{\log 2}
  \] 
  Il metodo di sostituzione dice che quando si arriva ad avere una disequazione
  corrispondente all'ipotesi, allora la soluzione è corretta se soddisfa una certa ipotesi.
\end{example}

\begin{example}
  \[
    T(n) = T\left(\floor*{\frac{n}{2}}\right) + T\left(\ceil*{\frac{n}{2}}\right) + 1 \quad \in O(n)
  \] 
  \[
  T(n) \le cn
  \] 
  \[
  \begin{aligned}
    T(n) & = T\left(\floor*{\frac{n}{2}}\right) + T\left(\ceil*{\frac{n}{2}}\right) + 1\\
         & \le c\left(\floor*{\frac{n}{2}}\right) + c\left(\ceil*{\frac{n}{2}}\right) + 1\\
         & = c \left( \left\lfloor \frac{n}{2} \right\rfloor + \left\lceil \frac{n}{2} \right\rceil  \right) + 1\\
         & = cn + 1 \stackrel{?}{\le} cn
  \end{aligned}
  \] 
  Il metodo utilizzato non funziona perchè rimane l'1 e non si può togliere in alcun modo.
  Per risolvere questo problema bisogna risolverne uno più forte:
  \[
  T(n) \le cn - b
  \] 
  \[
  \begin{aligned}
    T(n) & = T\left(\floor*{\frac{n}{2}}\right) + T\left(\ceil*{\frac{n}{2}}\right) + 1\\
         & \le c\left(\floor*{\frac{n}{2}}\right) -b + c\left(\ceil*{\frac{n}{2}}\right) -b + 1\\
         & = c \left( \left\lfloor \frac{n}{2} \right\rfloor + \left\lceil \frac{n}{2} \right\rceil  \right) - 2b + 1\\
         & = cn - 2b + 1 \stackrel{?}{\le} cn - b\\
         & = \underbrace{cn - b} + \underbrace{1 - b}_{\le 0} \le cn - b \quad \text{se } b \ge 1\\
  \end{aligned}
  \] 
  Se la proprietà vale per questo problema allora vale anche per il problema iniziale
  perchè è meno forte.
\end{example}

\begin{example}
  \[
    \begin{aligned}
      T(n) & = 3T \left( \left\lfloor \frac{n}{4} \right\rfloor \right) + n\\
           & = n + 3T \left( \left\lfloor \frac{n}{4} \right\rfloor \right)\\
           & = n + 3 \left( \left\lfloor \frac{n}{4} \right\rfloor + 3T 
           \left( \left\lfloor \frac{\left\lfloor \frac{n}{4} \right\rfloor}{4} \right\rfloor
           \right)  \right)\\
           & = n + 3 \left\lfloor \frac{n}{4} \right\rfloor + 3^2 T 
           \left( \left\lfloor \frac{n}{4^2} \right\rfloor \right)\\
           & \le n + 3 \left\lfloor \frac{n}{4} \right\rfloor + 3^2 
           \left( \left\lfloor \frac{n}{4^2} \right\rfloor + 3T \left( 
           \left\lfloor \frac{\left\lfloor \frac{n}{4^2} \right\rfloor}{4} \right\rfloor
           \right)  \right) \\
           & = n + 3 \left\lfloor \frac{n}{4} \right\rfloor + 3^2
           \left\lfloor \frac{n}{4^2} \right\rfloor + 3^3 T
           \left( \left\lfloor \frac{n}{4^3} \right\rfloor \right) \\
           & = n + 3 \left\lfloor \frac{n}{4} \right\rfloor + \ldots + 3^{i-1}
           \left\lfloor \frac{n}{4^{i-1}} \right\rfloor + 3^i T
           \left( \left\lfloor \frac{n}{4^i} \right\rfloor \right) 
    \end{aligned}
  \] 
  Per trovare il caso base poniamo l'argomento di T molto piccolo:
  \[
    \begin{aligned}
      \frac{n}{4^i} & < 1\\
      4^i & > n\\
      i & > \log_4 n
    \end{aligned}
  \] 
  L'equazione diventa:
  \[
    \begin{aligned}
      & \le n + 3 \left\lfloor \frac{n}{4} \right\rfloor + \ldots + 3^{\log_4 n - 1}
      \left\lfloor \frac{n}{4^{\log_4 n - 1}} \right\rfloor + 3^{\log_4 n} c\\
    \end{aligned}
  \] 
  Si può togliere l'approssimazione per difetto per ottenere un maggiorante:
  \[
  \begin{aligned}
    & \le n \left( 1 + \frac{3}{4} + \left( \frac{3}{4} \right)^2 + \ldots +
    \left( \frac{3}{4} \right)^{\log_4 n-1} \right) + 3^{\log_4 n} c\\
    & \le n \left( \sum_{i=0}^{\infty} \left( \frac{3}{4} \right)^i \right) + c 3^{\log_4 n}\\
  \end{aligned}
  \] 
  Per capire l'ordine di grandezza di \( 3^{\log_4 n} \) si può scrivere come:
  \[
    3^{\log_4 n} = n^{\left( \log_n 3^{\log_4 n} \right) } = n^{\log_4 n \cdot \log_n 3}
    = n^{\log_4 3}
  \] 
  Quindi la complessità è:
  \[
  \begin{aligned}
    & = O(n) + O(n^{\log_4 3})\\
  \end{aligned}
  \] 
  Si ha che una funzione è uguale al termine noto della funzione originale e l'altra
  che è uguale al logaritmo dei termini noti. Se usassimo delle variabili uscirebbe:
  \[
    \begin{aligned}
      T(n) & = a T \left(  \frac{n}{b}  \right) + f(n)\\
           & = O(f(n)) + O(n^{\log_b a})
    \end{aligned}
  \] 
\end{example}

\subsection{Teorema dell'esperto}
\begin{theorem}[Teorema dell'esperto o Master theorem]
  Per un'equazione di ricorrenza del tipo:
  \[
    T(n) = a T \left(  \frac{n}{b}  \right) + f(n)\\
  \] 
  Si distinguono 3 casi:
  \begin{itemize}
    \item \( f(n) \in O(n^{\log_b a - \varepsilon}) \) allora \( T(n); \in \Theta(n^{\log_b a}) \)  
    \item \( f(n) \in \Theta(n^{\log_b a}) \) allora \( T(n) \in \Theta(f(n) \log n) \) 
    \item \( f(n) \in \Omega(n^{\log_b a + \varepsilon}) \) allora \( T(n) \in \Theta(f(n)) \) 
  \end{itemize}
\end{theorem}

\begin{example}
  \[
  T(n) = 9 T\left(\frac{n}{3}\right) + n
  \] 
  Applico il teorema dell'esperto:
  \[
  \begin{aligned}
    a & = 9\\
    b & = 3\\
    f(n) & = n\\
  \end{aligned}
  \] 
  \[
    n^{\log_b a} = n^{\log_3 9} = n^2
  \] 
  Verifico se esiste un \( \varepsilon \) tale che:
  \[
    n \in O(n^{2 - \varepsilon})
  \]
  prendo \( \varepsilon = -\frac{1}{2} \) e verifico:
  \[
    n \in O(n^2 \cdot n^{-\frac{1}{2}})
  \] 
  Quindi ho trovato il caso 1 del teorema dell'esperto.
  \[
    T(n) \in \Theta(n^2)
  \] 
\end{example}

\begin{example}
  \[
  T(n) = T \left( \frac{2n}{3} \right) + 1
  \] 
  Applico il teorema dell'esperto:
  \[
    \begin{aligned}
      a & = 1\\
      b & = \frac{3}{2}\\
      f(n) & = n^0\\
    \end{aligned}
  \]

  \[
    n^{\log_b a} = n^{\log_{\frac{3}{2}} 1} = n^0
  \] 
  Si nota che le due funzioni hanno lo stesso ordine di grandezza, quindi siamo nel secondo
  caso del teorema dell'esperto.
  \[
    T(n) \in \Theta(\log n)
  \] 
\end{example}

\begin{example}
  \[
    T(n) = 3T \left( \frac{n}{4} \right) + n \log n
  \] 
  Applico il teorema dell'esperto:
  \[
    \begin{aligned}
      a & = 3\\
      b & = 4\\
      f(n) & = n \log n\\
    \end{aligned}
  \]
  \[
    n^{\log_b a} = n^{\log_4 3}
  \]
  \[
    n \log n \in \Omega(n^{\log_4 3})
  \]
  Esiste un \( \varepsilon \) tale che:
  \[
    n \log n \in \Omega(n^{\log_4 3 + \varepsilon})
  \]
  perchè basta che sia compreso tra \( \log_4 3 \) e \( 1 \).
  
  \vspace{1em}
  \noindent
  Quindi siamo nel terzo caso del teorema dell'esperto.
  \[
    T(n) \in \Theta(n \log n)
  \]
\end{example}

\begin{example}
  \[
  T(n) = 2T \left( \frac{n}{2} \right) + n \log n
  \] 
  Applico il teorema dell'esperto:
  \[
    \begin{aligned}
      a & = 2\\
      b & = 2\\
      f(n) & = n \log n\\
    \end{aligned}
  \]
  \[
    n^{\log_b a} = n^{\log_2 2} = n
  \]
  \[
    n \log n \in \Omega(n)
  \]
  Verifico se esiiste un \( \varepsilon \), quindi divido per \( n \):
  \[
    \log n \in \Omega(n^{\varepsilon})
  \] 
  Quindi si nota che questa proprietà non è verificata, quindi non si può applicare il
  teorema dell'esperto.
\end{example}

\subsection{Merge sort}
Questo algoritmo di ordinamento è basato sulla tecnica divide et impera:
\begin{itemize}
  \item \textbf{Divide}: Dividi il problema in sottoproblemi più piccoli
  \item \textbf{Impera}: Risolvi i sottoproblemi in modo ricorsivo
  \item \textbf{Combina}: Unisci le soluzioni dei sottoproblemi per risolvere il problema
    originale
\end{itemize}
Questo algoritmo divide la sequenza in due parti uguali e le ordina separatamente, successivamente
le unisce in modo ordinato. La complessità, comsiderando il merge con complessità lineare,
risulta:
\[
  T(n) = 2T\left(\frac{n}{2}\right) + n
\] 
Applicando il teorema dell'esperto si ottiene:
\[
\begin{aligned}
  a & = 2\\
  b & = 2\\
  f(n) & = n\\
\end{aligned}
\] 
\[
  n^{\log_b a} = n
\] 
\[
  n \in \Theta(n)
\] 
Quindi siamo nel secondo caso del teorema dell'esperto:
\[
  T(n) \in \Theta(n \log n)
\]

\vspace{1em}
\noindent
Definizione del merge sort:
\begin{lstlisting}[language=Scala]
// A: Array da ordinare
// P: Indice di partenza
// r: Indice di arrivo
merge_sort(A, p, r)         // --
  if p < r                  //  |
    q <- floor((p + r) / 2) //  | 
    merge_sort(A, p, q)     //  | O(n log n)
    merge_sort(A, q + 1, r) //  |
    merge(A, p, q, r)       // --
\end{lstlisting}
\begin{lstlisting}[language=Scala]
// A: Array da ordinare
// P: Indice di partenza
// q: Indice di mezzo
// r: Indice di arrivo
merge(A, p, q, r)
  i <- 1
  j <- p
  k <- q + 1
  // Ordina gli elementi di A in B
  while(j <= q and k <= r)                // --
    if j <= q and (k > r or A[j] <= A[k]) //  |
      B[i] <- A[j]                        //  |
      j++                                 //  |
    else                                  //  | O(n)
      B[i] <- A[k]                        //  |
      k++                                 //  |
    i++                                   // --

  // Copia gli elementi di B in A
  for i <- 1 to r - p + 1                 // -|
    A[p + i - 1] <- B[i]                  // -| O(n)
\end{lstlisting}

\noindent
L'algoritmo è stablie perchè non vengono scambiati elementi uguali e non è in loco perchè
utilizza un array di appoggio.

\subsection{Heap}
È un albero semicompleto (ogni nodo ha 2 figli ad ogni livello tranne l'ultimo che è
completo solo fino ad un certo punto) in cui i nodi contengono oggetti con relazioni di
ordinamento.
\begin{figure}[H]
  \centering
  \begin{forest}
    for tree={
    draw,
    circle,
    minimum size=2em,
    inner sep=1pt,
    s sep=1cm,
  }
    [1
      [2
        [4
        [8]
        [9]
        ]
        [5
        [10]
        ]
      ]
      [3
        [6]
        [7]
      ]
    ]
  \end{forest}
  \caption{Heap con l'indice di un array associato ai nodi}
\end{figure}

\subsubsection{Proprietà}
\( \forall \) nodo il contenuto del nodo è \( \ge \) del contenuto dei figli.
Per calcolare il numero di nodi di un albero binario si usa la formula:
\[
  N = 2^0 + 2^1 + 2^2 + \ldots + 2^{h-1} = \frac{1-2^h}{1-2} = 2^h - 1
\] 
dove \( h \) è l'altezza dell'albero.
Il numero di foglie di un albero sono la metà dei nodi.

\vspace{1em}
\noindent
Definiamo una funzione che "aggiusta" i figli di un nodo per mantenere la proprietà di heap:
\begin{lstlisting}[language=Scala]
  heapify(A, i) // O(n)
    l <- left[i] // Indice del figlio sinistro (2i)
    r <- right[i] // Indice del figlio destro (2i+1)
    if l < H.heap_size and H[l] > H[i]
      largest <- l
    else
      largest <- i

    if r < H.heap_size and H[r] > H[largest]
      largest <- r
    if largest != i
      swap(H[i], H[largest])
      heapify(H, largest)
\end{lstlisting}
Ora si vuole definire una funzione che costruisce un heap da un array:
\begin{lstlisting}[language=Scala]
  build_heap(A) // O(n)
    heapsize(a) <- length[A]
    for i <- floor(length[A]/2) downto 1
      heapify(A, i)
\end{lstlisting}
Una volta definito un heap si possono fare diverse operazioni, come ad esempio estrarre
il nodo massimo:
\begin{lstlisting}[language=Scala]
  extract_max(A)
    H[1] <- H[H.heap_size]
    H.heap_size <- H.heap_size - 1
    heapify(H,1)
\end{lstlisting}
\subsubsection{Heap sort}
Heap sort è un algoritmo di ordinamento basato su heap.
\begin{lstlisting}[language=Scala]
  heap_sort(A) // O(n log n)
    build_heap(A) // n
    for i <- length[A] downto 2
      scambia(A[1], A[i])
      heapsize(A)--
      heapify(A, 1) // log i
\end{lstlisting}


La complessità dell'algoritmo è \( O(n \log n) \), ma più precisamente:
\[
\sum_{i=1}^{n} \log i = \log \prod_{i=1}^{n} i = \log n! = \Theta(\log n^n) = \Theta(n \log n)
\] 
Per la formula di Stirling \( n! \) ha ordine di grandezza \( n^n \). Questo algoritmo
è in loco e stabile.

\vspace{1em}
\noindent
Il caso pessimo è un array ordinato al contrario.

\subsection{Quick sort}
Il concetto di questo algoritmo è quello di mettere prima in disordine l'algoritmo e poi
ordinarlo. L'algoritmo divide l'array in 2 parti e ordina ricorsivamente le due parti; a
quel punto l'array è ordinato.

\begin{lstlisting}[language=Scala]
  // A: Array da ordinare 
  // p: Indice di partenza
  // r: Indice di arrivo
  quick_sort(A, p, r)
    if p < r // Ordina solo se l'array ha piu' di un elemento
      q <- partition(A, p, r) // Dividi l'array in due parti
      quick_sort(A, p, q) // Ordina sinistra
      quick_sort(A, q + 1, r) // Ordina destra
\end{lstlisting}
\begin{lstlisting}[language=Scala]
  partition(A, p, r)
    x <- A[p] // Elemento perno (o pivot)
    i <- p - 1
    j <- r + 1
    while true
      repeat // Ripete finche' la condizione non e' soddisfatta
        j--  // n/2
      until A[j] <= x // Trova un elemento che non puo' stare a destra
      repeat
        i++  // n/2
      until A[i] >= x // Trova un elemento che non puo' stare a sinistra
      if i < j
        swap(A[i], A[j]) // n/2
      else
        return j // alla fine j puntera' all'ultimo elemento di sinistra
\end{lstlisting}

\noindent
Questo algoritmo è in loco e non è stabile. La sua complessità nel caso peggiore è:
\[
  \begin{aligned}
    T(n) &= T(\text{partition}) + T(q) + T(n-q)\\
         &= n + T(q) + T(n-q)\\
         &= n + \cancel{T(1)} + T(n-1)\\
         &= n + T(n-1)\\
         &= \Theta(n^2)
  \end{aligned}
\] 
Il caso peggiore è un array già ordinato.

\vspace{1em}
\noindent
Mediamente ci si aspetta che l'array venga diviso in 2 parti molto simili, quindi
la complessità è \( O(n \log n) \) perchè:
\[
0 < c < 1
\] 
\[
  T(n) = n + T(cn) + T((1-c)n)
\] 

\begin{lstlisting}[language=Scala]
  rand_partition(A, p, r)
    i <- random(p, r)
    swap(A[i], A[p])
    return partition(A, p, r)
\end{lstlisting}
La complessità è la media di tutte le complessità con probabilità \( \frac{1}{n} \) 
\[
  \begin{aligned}
    T(n) = n + \frac{1}{n}\left( T(1) + T(n-1) \right) + \frac{n-1}{n}\left( T(2) + T(n-2) \right)\\
    + \ldots + \frac{1}{n}\left( T(n-1) + T(1) \right)
  \end{aligned}
\] 
\[
  \begin{aligned}
    T(n) &= n + \frac{1}{n} \sum_{i} \left( T(i) + T(n-i) \right) \\
         &= n + \frac{2}{n} \sum_{i} T(i)\\
  \end{aligned}
\] 

\subsection{Algoritmi di ordinamento non per confronti} 
Si possono avere algoritmi di ordinamento con complessità \( < n \log n \)?

\vspace{1em}
\noindent
Qualsiasi algoritmo che ordina per confronti deve fare almeno \( n \log n \) confronti
nel caso pessimo
\begin{figure}[H]
  \centering
  \begin{forest}
    for tree={
    minimum size=2em,
    inner sep=1pt,
    s sep=1cm,
  }
    [\( x \to n! \) 
      [\( x_1 \) 
        [\( x_{1,1} \)]
        [\( x_{1,2} \) ]
      ]
      [\( x_2 \) 
        [\( x_{2,1} \) ]
        [\( x_{2,2} \) ]
      ]
    ]
  \end{forest}
  \caption{Heap con l'indice di un array associato ai nodi}
\end{figure}
\noindent
Le foglie rappresentano ogni singola combinazione possibile. Il numero di foglie è
\( n! \) e l'altezza sarà sempre
\[
h \ge \log_2 n! = n \log n
\] 

\subsubsection{Counting sort}
Si vogliono ordinare \( n \) numeri con valori da \( 1 \) a \( k \). L'idea di questo
algoritmo è quella di creare un'array che contiene il numero di occorrenze di un
certo valore (rappresentato dall'indice).

\begin{lstlisting}[language=Scala]
counting_sort(A, k) 
  for i <- 1 to k // k
    C[i] <- 0 // Inizializzazione di un array a 0

  for j <- 1 to length[A] // n
    C[A[j]]++ // Conteggio delle occorrenze

  // k
  for i <- 2 to k         // In ogni indice metto il numero di
    C[i] <- C[i] + C[i-1] // elementi minori o uguali
                          // al numero dell'indice
  // Alla fine l'array C conterra' l'ultima posizione di occorrenza per ogni elemento

  for j <- length[A] downto 1 // n
    B[C[A[j]]] <- A[j] // Inserimento dell'elemento in posizione
    C[A[j]]--          // Decremento della posizione di occorrenza
\end{lstlisting}
La complessità di questo algoritmo è \( O(n + k) \) e siccome sappiamo che \( k \) è
una costante fissata a priori la complessità è \( O(n) \). Non è in loco, ma è stabile

\subsubsection{Radix sort}
Il radix sort è un ordinamento lessico grafico, cioè si ordinano le cifre partendo da
quella meno significativa e se sono uguali si passa a quella più significativa.

La complessità dell'algoritmo è:
\[
 \Theta(l(n + k))
\] 
dove:
\[
\begin{aligned}
  l & = \text{numero di cifre} = \log_k n\\
  n & = \text{numero di elementi}\\
  k & = \text{numero di valori possibili}
\end{aligned}
\] 
Se rappresentiamo i numeri in base \( n \), allora si avrà la seguente rappresentazione:
\[
   \ldots \;\; n^2 \;\; n^1 \;\; n^0
\] 
e ad esempio per rappresentare \( n^2 - 1 \) valori possibili serviranno \( 2 \) cifre.
cifre.

\subsubsection{Bucket sort}
Dato un array di numeri con \textbf{supporto infinito} e \textbf{distribuzione nota}, si può dividere
l'array in \( k \) parti (bucket) uguali (equiprobabili) e ordinare ricorsivamente. Ogni coppia
di gruppi deve essere totalmente ordinata, cioè ogni elemento del primo gruppo deve essere
minore di ogni elemento del secondo gruppo. Una volta ordinati i gruppi (con un algoritmo
di ordinamento a scelta) si concatenano in modo ordinato.

\vspace{1em}
\noindent
Il caso peggiore è quello in cui tutti gli elementi finiscono in un singolo bucket,
la probabilità che questo accada è molto bassa:
\[
  \underbrace{\frac{1}{n} \cdot \frac{1}{n} \cdot \dots \cdot \frac{1}{n}}_{n-1}
  = \frac{1}{n^{n-1}}
\] 
e la sua complessità diventa:
\[
  O(n^2)
\] 

\vspace{1em}
\noindent
Nel caso medio si ha che per creare i bucket si ha una complessità \( O(n) \) e per
assegnare gli elementi ai bucket si ha una complessità \( O(n) \). Per ogni bucket
ci si aspetta che il numero di elementi al suo interno sia una \textbf{costante},
quindi \textbf{indipendente dal valore di \( n \)}. Per ordinare un bucket si ha una
complessità \( O(1) \) siccome il numero di elementi è costante. La complessità totale
è quindi:
\[
  \Theta(n)
\]

\vspace{1em}
\noindent
Formalizzando si ha:

\noindent
Sia \( X_{ij} \) la variabile casuale che vale: 
$\begin{cases}
  1 & \text{se l'elemento } i \text{ va nel bucket } j\\
  0 & \text{altrimenti}
\end{cases}$

\noindent
Per esprimere il numero di elementi nel bucket \( j \) si può scrivere:
\[
  N_j = \sum_{i} X_{ij}
\] 
La complessità di questo algoritmo sarà quindi:
\[
  C = \sum_{j} (N_j)^2
\] 
Per ottenere il valore medio della complessità:
\[
  \begin{aligned}
    E[C] &= E\left[\sum_{j} (N_j)^2\right] \\
         &= \sum_{j} E[(N_j)^2] \\
         &= \sum_{j} \left( Var[N_j] + E[N_j]^2 \right) \\
  \end{aligned}
\] 
sappiamo che \( N_j = \sum_{i} X_{ij} \), quindi la media è:
\[
  E[N_j] = \sum_{j}^{n} E[X_{ij}] = \sum_{j}^{n} \frac{1}{n} = 1
\] 
e la varianza è:
\[
  Var[N_j] = \sum_{j}^{n} Var[X_{ij}] = \sum_{j}^{n} \frac{1}{n} \left( 1 - \frac{1}{n} \right) = 1 - \frac{1}{n}
\] 
La complessità diventa:
\[
  \begin{aligned}
    E[C] &= \sum_{j} \left( \left( 1 - \frac{1}{n} \right) - 1 \right) \\
         &= \sum_{j} 2 - \frac{1}{n} \\
         &= 2n - 1
  \end{aligned}
\] 

\end{document}
