\documentclass[a4paper]{article}
\usepackage{import}
\usepackage[utf8]{inputenc}
\usepackage[T1]{fontenc}
\usepackage{textcomp}
\usepackage[italian]{babel}
\usepackage{amsmath, amssymb}
\usepackage{booktabs,xltabular}
\usepackage{amsfonts}
\usepackage{subcaption}
\usepackage{amsthm}
\usepackage{cancel}
\usepackage{mdframed}
\usepackage{makecell}
\usepackage{float}
\usepackage{xcolor}
\usepackage{listings}
\usepackage{gensymb}
\usepackage{graphicx}
\usepackage{bodeplot}
\usepackage{physics}
\usepackage{tikz}
\usetikzlibrary{shapes, arrows, automata, petri, decorations.markings, decorations.pathreplacing, positioning, calc, quotes}
\usepackage{circuitikz}
\usepackage[label=corner]{karnaugh-map}
\graphicspath{{./figures/}}

% Set default font to sans-serif
\renewcommand{\familydefault}{\sfdefault} 
\usepackage{eulervm}

\usepackage{forest}

\usepackage{mathtools}
\DeclarePairedDelimiter\ceil{\lceil}{\rceil}
\DeclarePairedDelimiter\floor{\lfloor}{\rfloor}

% \usepackage{ntheorem}

\usepackage{import}
\usepackage{pdfpages}
\usepackage{transparent}
\usepackage{xcolor}

\usepackage{hyperref}
\hypersetup{
    colorlinks=false,
}

% Code blocks
\definecolor{codegreen}{rgb}{0,0.6,0}
\definecolor{codegray}{rgb}{0.5,0.5,0.5}
\definecolor{codepurple}{rgb}{0.58,0,0.82}
\definecolor{backcolour}{rgb}{0.95,0.95,0.95}

\lstdefinestyle{mystyle}{
	backgroundcolor=\color{backcolour},
	commentstyle=\color{codegreen},
	keywordstyle=\color{magenta},
	numberstyle=\tiny\color{codegray},
	stringstyle=\color{codepurple},
	basicstyle=\ttfamily\footnotesize,
	breakatwhitespace=false,
	breaklines=true,
	captionpos=b,
	keepspaces=true,
	numbers=left,
	numbersep=5pt,
	showspaces=false,
	showstringspaces=false,
	showtabs=false,
	tabsize=2
}

\lstset{style=mystyle}

\usepackage{color}
\usepackage{import}
\usepackage{pdfpages}
\usepackage{transparent}
\usepackage{xcolor}

% Example frame
\theoremstyle{definition}
\newmdtheoremenv[%
	linecolor=gray,leftmargin=0,%
	rightmargin=0,
	innertopmargin=8pt,%
	innerbottommargin=8pt,
	ntheorem]{example}{Esempio}[section]

% Important definition frame
\theoremstyle{definition}
\newmdtheoremenv[%
	linecolor=gray,leftmargin=0,%
	rightmargin=0,
	backgroundcolor=gray!40,%
	innertopmargin=8pt,%
	innerbottommargin=8pt,
	ntheorem]{definition}{Definizione}[section]

% Exercise frame
\theoremstyle{definition}
\newmdtheoremenv[%
	linecolor=gray,leftmargin=0,%
	rightmargin=0,
	innertopmargin=8pt,%
	innerbottommargin=8pt,
	ntheorem]{exercise}{Esercizio}[section]

% Theorem frame
\theoremstyle{definition}
\newmdtheoremenv[%
  linecolor=gray,leftmargin=0,%
  rightmargin=0,
  innertopmargin=8pt,%
  innerbottommargin=8pt,
  ntheorem]{theorem}{Teorema}[section]

\theoremstyle{definition}
\newmdtheoremenv[%
  linecolor=white,leftmargin=0,%
  rightmargin=0,
  innertopmargin=8pt,%
  innerbottommargin=8pt,
  ntheorem]{define}{Definizione utile}[section]

% figure support
\usepackage{import}
\usepackage{xifthen}
\pdfminorversion=7
\usepackage{pdfpages}
\usepackage{transparent}
\newcommand{\incfig}[1]{%
	\def\svgwidth{\columnwidth}
	\import{./figures/}{#1.pdf_tex}
}

% FSM tikz
\tikzset{
    place/.style={
        circle,
        thick,
        draw=black,
        minimum size=6mm,
    },
        state/.style={
        circle,
        thick,
        draw=black,
        fill=white,
        minimum size=6mm,
    },
}

\pdfsuppresswarningpagegroup=1

\usepackage{pgfplots}
\pgfplotsset{compat=1.18,width=10cm}

% Save plots as pdf and reuse them without compiling every time
\usetikzlibrary{external}
\tikzexternalize[prefix=figures/tikz/, optimize=false]


\begin{document}

\begin{titlepage}
	\begin{center}
		\vspace*{1cm}

		\Huge
		\textbf{Probabilità e Statistica\\Esercizi}

		\vspace{0.5cm}
		\LARGE
		UniVR - Dipartimento di Informatica

		\vspace{1.5cm}

		\textbf{Fabio Irimie}

		\vfill


		\vspace{0.8cm}


		2° Semestre 2023/2024

	\end{center}
\end{titlepage}


\tableofcontents
\pagebreak

\section{Ricorrenza}
\subsection{Esercizio 1}
Usando il metodo di sostituzione, dimostrare che la ricorrenza:
\[
  L(n) = \begin{cases}
    1 & \text{se } n < n_0\\
    L\left(\frac{n}{3}\right) + L\left(\frac{2n}{3}\right) & \text{se } n \ge n_0
  \end{cases}
\] 
ha un limite asintotico inferiore \( L(n) \in \Omega(n) \) e deducetene che 
\( L(n) \in \Theta(n) \) 

\subsubsection{Risoluzione}
Sostituiamo \( L(n) = cn \) per verificare se \( \Omega(n) \) è il limite inferiore.
\[
  \begin{aligned}
    L(n) &= L\left(\frac{n}{3}\right) + L\left(\frac{2n}{3}\right)\\
         &\ge  c \frac{n}{3} + c \frac{2n}{3}\\
         &= c \left( \frac{n}{3} + \frac{2n}{3} \right) \\
         &= c \left( \frac{\cancel{3}n}{\cancel{3}} \right) \\
         &= cn
  \end{aligned}
\] 
Siccome \( cn \ge cn \) abbiamo verificato quindi che il limite asintotico inferiore è
\( \Omega(n) \). Siccome le due parti sono uguali, abbiamo anche verificato che
\( L(n) \in \Theta(n) \).

\subsection{Esercizio 2}
Usando il metodo di sostituzione, dimostrare che la ricorrenza:
\[
  T(n) = T\left(\frac{n}{3}\right) + T\left(\frac{2n}{3}\right) + \Theta(n)
\] 
ha soluzione \( T(n) = \Omega(n \lg n) \) e deducetene che \( T(n) = \Theta(n \lg n) \) 

\subsubsection{Risoluzione}
Sostituiamo \( T(n) = cn \lg n \) per verificare \( T(n) \) ha soluzione in 
\( \Omega(n \lg n) \) 
\[
  \begin{aligned}
    T(n) &= T\left(\frac{n}{3}\right) + T\left(\frac{2n}{3}\right) + \Theta(n)\\
         &\ge c \frac{n}{3} \lg \frac{n}{3} + c \frac{2n}{3} \lg \frac{2n}{3} + cn \\
         &= c \left( \frac{n}{3} \lg \frac{n}{3} +\frac{2n}{3} \lg \frac{2n}{3} + n \right) \\
         &= c \left( \frac{n}{3} \left( \lg \frac{n}{3} + 2 \lg \frac{2n}{3} + 3 \right)  \right) \\
         &= c \frac{n}{3} \left( \lg \frac{n}{3} + 2 \lg \frac{2n}{3} + 3 \right) \\
         &= c \frac{n}{3} \left( \lg \frac{n}{3} + 2 \lg \frac{n}{3} + 2 \lg 2 + 3 \right) \\
         &= c \frac{n}{3} \left( 3 \lg \frac{n}{3} + 3 + 2 \lg 2 \right) \\
         &= c \frac{n}{3} \left( 3 + 2 \lg 2 \right) + c n \lg \frac{n}{3} \\
         &= c \frac{n}{3} + c n \lg \frac{n}{3} \\
         &= c \frac{n}{3} + c n \left( \lg n - \lg 3 \right) \\
         &= c \frac{n}{3} + c n \lg n \\
         &= cn + cn \lg n \stackrel{?}{\ge} cn \lg n \\
         &= \frac{\cancel{cn} + \cancel{cn} \lg n}{\cancel{cn}} \stackrel{?}{\ge}
         \frac{\cancel{cn} \lg n}{\cancel{cn}} \\
         &= 1 + \lg n \stackrel{?}{\ge} n \lg n \quad \surd
  \end{aligned}
\] 
Abbiamo quindi verificato che \( T(n) \in \Omega(n \lg n) \).

\end{document}
