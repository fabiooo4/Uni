\documentclass[a4paper]{article}
\usepackage{import}
\usepackage[utf8]{inputenc}
\usepackage[T1]{fontenc}
\usepackage{textcomp}
\usepackage[italian]{babel}
\usepackage{amsmath, amssymb}
\usepackage{booktabs,xltabular}
\usepackage{amsfonts}
\usepackage{amsthm}
\usepackage{cancel}
\usepackage{mdframed}
\usepackage{makecell}
\usepackage{float}
\usepackage{xcolor}
\usepackage{listings}
\usepackage{gensymb}
\usepackage{graphicx}
\usepackage{bodeplot}
\usepackage{tikz}
\usetikzlibrary{shapes, arrows, automata, petri, decorations.markings, decorations.pathreplacing, positioning, calc}
\usepackage{circuitikz}
\usepackage[label=corner]{karnaugh-map}
\graphicspath{{./figures/}}

% Set default font to sans-serif
\renewcommand{\familydefault}{\sfdefault} 
\usepackage{eulervm}

\usepackage{forest}

\usepackage{mathtools}
\DeclarePairedDelimiter\ceil{\lceil}{\rceil}
\DeclarePairedDelimiter\floor{\lfloor}{\rfloor}

% \usepackage{ntheorem}

\usepackage{import}
\usepackage{pdfpages}
\usepackage{transparent}
\usepackage{xcolor}

\usepackage{hyperref}
\hypersetup{
    colorlinks=false,
}

% Code blocks
\definecolor{codegreen}{rgb}{0,0.6,0}
\definecolor{codegray}{rgb}{0.5,0.5,0.5}
\definecolor{codepurple}{rgb}{0.58,0,0.82}
\definecolor{backcolour}{rgb}{0.95,0.95,0.95}

\lstdefinestyle{mystyle}{
	backgroundcolor=\color{backcolour},
	commentstyle=\color{codegreen},
	keywordstyle=\color{magenta},
	numberstyle=\tiny\color{codegray},
	stringstyle=\color{codepurple},
	basicstyle=\ttfamily\footnotesize,
	breakatwhitespace=false,
	breaklines=true,
	captionpos=b,
	keepspaces=true,
	numbers=left,
	numbersep=5pt,
	showspaces=false,
	showstringspaces=false,
	showtabs=false,
	tabsize=2
}

\lstset{style=mystyle}

\usepackage{color}
\usepackage{import}
\usepackage{pdfpages}
\usepackage{transparent}
\usepackage{xcolor}

% Example frame
\theoremstyle{definition}
\newmdtheoremenv[%
	linecolor=gray,leftmargin=0,%
	rightmargin=0,
	innertopmargin=8pt,%
	innerbottommargin=8pt,
	ntheorem]{example}{Esempio}[section]

% Important definition frame
\theoremstyle{definition}
\newmdtheoremenv[%
	linecolor=gray,leftmargin=0,%
	rightmargin=0,
	backgroundcolor=gray!40,%
	innertopmargin=8pt,%
	innerbottommargin=8pt,
	ntheorem]{definition}{Definizione}[section]

% Exercise frame
\theoremstyle{definition}
\newmdtheoremenv[%
	linecolor=gray,leftmargin=0,%
	rightmargin=0,
	innertopmargin=8pt,%
	innerbottommargin=8pt,
	ntheorem]{exercise}{Esercizio}[section]

% Theorem frame
\theoremstyle{definition}
\newmdtheoremenv[%
  linecolor=gray,leftmargin=0,%
  rightmargin=0,
  innertopmargin=8pt,%
  innerbottommargin=8pt,
  ntheorem]{theorem}{Teorema}[section]

\theoremstyle{definition}
\newmdtheoremenv[%
  linecolor=gray,leftmargin=0,%
  rightmargin=0,
  innertopmargin=8pt,%
  innerbottommargin=8pt,
  ntheorem]{define}{Definizione utile}[section]

% figure support
\usepackage{import}
\usepackage{xifthen}
\pdfminorversion=7
\usepackage{pdfpages}
\usepackage{transparent}
\newcommand{\incfig}[1]{%
	\def\svgwidth{\columnwidth}
	\import{./figures/}{#1.pdf_tex}
}

% FSM tikz
\tikzset{
    place/.style={
        circle,
        thick,
        draw=black,
        minimum size=6mm,
    },
        state/.style={
        circle,
        thick,
        draw=blue!75,
        fill=blue!20,
        minimum size=6mm,
    },
}

\usepackage{pgfplots}
\pgfplotsset{compat=1.18}

\pdfsuppresswarningpagegroup=1


\begin{document}

\begin{titlepage}
	\begin{center}
		\vspace*{1cm}

		\Huge
		\textbf{Analisi 1}

		\vspace{0.5cm}
		\LARGE
		UniVR - Dipartimento di Informatica

		\vspace{1.5cm}

		\textbf{Fabio Irimie}

		\vfill


		\vspace{0.8cm}

    Corso di Giacomo Canevari

		1° Semestre 2023/2024

	\end{center}
\end{titlepage}


\tableofcontents
\pagebreak

\section{Introduzione}

\section{Sistema Operativo}
Il \textbf{sistema operativo} è il livello del software che si pone tra l'hardware
e gli utenti. E quindi il sistema operativo incapsula la macchina fisica.
Per mettere in comunicazione l'utente e l'hardware solitamente si usano le 
\textbf{applicazioni}, ma quando si vuole accedere direttamente all'hardware
si usano le interfacce utente, ad esempio:
\begin{enumerate}
  \item \textbf{Interfaccia grafica} (GUI)
  \item \textbf{Command line} (Terminale o Shell)
  \item \textbf{Touch screen}
\end{enumerate}
\noindent
Gli obiettivi principali sono:
\begin{itemize}
  \item Facilitare l'uso del computer
  \item Rendere efficiente l'utilizzo dell'hardware
  \item Evitare conflitti nell'allocazione delle risorse hardware e software
\end{itemize}
Questo rimuove la necessità di conoscere la struttura dell'hardware attraverso l'
\textbf{astrazione} facilitando la programmazione.

\subsection{Compiti del sistema operativo}
\subsubsection{Gestione delle risorse}
Il sistema operativo deve gestire le risorse hardware, come ad esempio i dischi, la memoria,
gli input/output e la CPU. Deve anche gestire le risorse software, come ad esempio i file,
i programmi e la memoria virtuale.

\subsubsection{Programma di controllo}
Un altro compito del sistema operativo è quello di controllare l'esecuzione dei programmi
e del corretto utilizzo del sistema.

\subsection{Storia dei sistemi operativi}


% Il sistema operativo fornisce dei servizi per comunicare con l'hardware e questi
% servizi possono essere usati tramite delle \textbf{system calls}, ad esempio:
%
% \begin{itemize}
%   \item Esecuzione dei programmi
%   \item Gestione dei file
%   \item Operazioni I/O
%   \item Gestione degli errori
%   \item Comunicazione
% \end{itemize}
%
% \noindent
% L'unico programma che è sempre in esecuzione su un computer è il \textbf{kernel}.
%
% \subsection{Operazioni}
% \begin{itemize}
%   \item \textbf{Bootstrap program}: è una piccola porzione di codice che inizializza
%     il sistema e carica il kernel
%
%   \item Viene caricato il kernel
%
%   \item Vengono caricati i \textbf{system daemons}, cioè dei servizi forniti al
%     di fuori del kernel
%
%   \item Gestione delle chiamate di sistema:
%     \begin{itemize}
%       \item Hardware interrupt
%       \item Software interrupt
%     \end{itemize}
% \end{itemize}
%
% \noindent
% Su un computer vengono eseguiti più programmi alla volta salvando la coda dei
% processi da eseguire in memoria e lo \textbf{scheduler} si occupa di gestire
% l'ordine di esecuzione e di interruzione. Per permettere di eseguire più processi
% alla volta si utilizza il \textbf{time sharing}, cioè la CPU cambia processo così
% frequentemente che si crea l'illusione che i processi vengano eseguiti in parallelo
% anche se in realtà non è così.
%
% \section{Modalità di esecuzione}
% Il sistema operativo può eseguire il codice in 2 modalità:
% \begin{enumerate}
%   \item \textbf{Modalità utente}: il codice viene eseguito in modo normale
%   \item \textbf{Modalità kernel}: il codice viene eseguito con privilegi speciali
%     che permettono di accedere all'hardware
% \end{enumerate}
%
% \noindent
% Per capire in che modalità si sta eseguendo il codice si utilizza un bit nel
% \textbf{program status word} (PSW) che indica la modalità di esecuzione chiamato
% \textbf{mode bit}.
%
% \noindent Si può entrere in modalità kernel tramite:
% \begin{enumerate}
%   \item \textbf{Nuovo processo}: Per creare un nuovo processo, il kernel copia il
%     programma nella memoria, setta il program counter alla prima istruzione del
%     processo e setta lo stack pointer alla base dello stack del processo e infine
%     si torna in modalità utente. 
%
%     \item \textbf{Ritorno da un interrupt o system call}: Quando il kernel finisce
%       di gestire la richiesta, riprende l'esecuzione del processo che ha chiamato
%       l'interrupt o la system call e torna in modalità utente.
%       
%     \item \textbf{Cambio di contesto}: Quando il kernel decide di cambiare il
%       processo in esecuzione, salva lo stato del processo corrente e carica lo
%       stato del nuovo processo e torna in modalità utente.
% \end{enumerate}
%
% \noindent
% Per prevenire che un processo faccia un ciclo infinito esiste un timer che interrompe
% il processo se viene eseguito per troppo tempo.
%
% \subsection{Protezione e sicurezza}
% \begin{enumerate}
%   \item \textbf{Protezione}: qualsiasi meccanismo per controllare l'accesso dei
%     processi o degli utenti alle risorse del sistema.
%
%   \item \textbf{Sicurezza}: protezione da accessi esterni non autorizzati.
% \end{enumerate}
%
% \noindent
% Ogni utente è identificato da un \textbf{user id} e ogni utente può far parte di
% un gruppo, identificato da un \textbf{group id}. Ogni file ha un \textbf{owner} e
% un \textbf{group owner} e per ogni file ci sono 3 tipi di permessi:
% \begin{enumerate}
%   \item \textbf{Read}
%   \item \textbf{Write}
%   \item \textbf{Execute}
% \end{enumerate}
%
% \section{Linux}
% \subsection{Filesystem}
% In linux qualsiasi cosa è un file, cioè un contenitore di dati. I principali
% tipi di file sono:
% \begin{enumerate}
%   \item \textbf{Regolare}: File classici utente
%   \item \textbf{Directory}: Informazioni relative ad altri file
%   \item \textbf{Pipe}: Comunicazioni tra processi
%   \item \textbf{Link}: Alias
%   \item \textbf{Socket}: Comunicazioni tra processi
%   \item \textbf{Speciale}: Dispositivi
% \end{enumerate}
%
% \noindent Tutti i file sono organizzati in un albero, chiamato \textbf{file system}.
% \begin{enumerate}
%   \item \textbf{root}: La radice del file system: /
%   \item \textbf{nodo}: File (se ha già figli è una directory)
% \end{enumerate}
%
%
\end{document}
