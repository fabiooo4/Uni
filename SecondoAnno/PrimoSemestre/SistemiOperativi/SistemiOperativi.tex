\documentclass[a4paper]{article}
\usepackage{import}
\usepackage[utf8]{inputenc}
\usepackage[T1]{fontenc}
\usepackage{textcomp}
\usepackage[italian]{babel}
\usepackage{amsmath, amssymb}
\usepackage{booktabs,xltabular}
\usepackage{amsfonts}
\usepackage{subcaption}
\usepackage{amsthm}
\usepackage{cancel}
\usepackage{mdframed}
\usepackage{makecell}
\usepackage{float}
\usepackage{xcolor}
\usepackage{listings}
\usepackage{gensymb}
\usepackage{graphicx}
\usepackage{bodeplot}
\usepackage{physics}
\usepackage{tikz}
\usetikzlibrary{shapes, arrows, automata, petri, decorations.markings, decorations.pathreplacing, positioning, calc, quotes}
\usepackage{circuitikz}
\usepackage[label=corner]{karnaugh-map}
\graphicspath{{./figures/}}

% Set default font to sans-serif
\renewcommand{\familydefault}{\sfdefault} 
\usepackage{eulervm}

\usepackage{forest}

\usepackage{mathtools}
\DeclarePairedDelimiter\ceil{\lceil}{\rceil}
\DeclarePairedDelimiter\floor{\lfloor}{\rfloor}

% \usepackage{ntheorem}

\usepackage{import}
\usepackage{pdfpages}
\usepackage{transparent}
\usepackage{xcolor}

\usepackage{hyperref}
\hypersetup{
    colorlinks=false,
}

% Code blocks
\definecolor{codegreen}{rgb}{0,0.6,0}
\definecolor{codegray}{rgb}{0.5,0.5,0.5}
\definecolor{codepurple}{rgb}{0.58,0,0.82}
\definecolor{backcolour}{rgb}{0.95,0.95,0.95}

\lstdefinestyle{mystyle}{
	backgroundcolor=\color{backcolour},
	commentstyle=\color{codegreen},
	keywordstyle=\color{magenta},
	numberstyle=\tiny\color{codegray},
	stringstyle=\color{codepurple},
	basicstyle=\ttfamily\footnotesize,
	breakatwhitespace=false,
	breaklines=true,
	captionpos=b,
	keepspaces=true,
	numbers=left,
	numbersep=5pt,
	showspaces=false,
	showstringspaces=false,
	showtabs=false,
	tabsize=2
}

\lstset{style=mystyle}

\usepackage{color}
\usepackage{import}
\usepackage{pdfpages}
\usepackage{transparent}
\usepackage{xcolor}

% Example frame
\theoremstyle{definition}
\newmdtheoremenv[%
	linecolor=gray,leftmargin=0,%
	rightmargin=0,
	innertopmargin=8pt,%
	innerbottommargin=8pt,
	ntheorem]{example}{Esempio}[section]

% Important definition frame
\theoremstyle{definition}
\newmdtheoremenv[%
	linecolor=gray,leftmargin=0,%
	rightmargin=0,
	backgroundcolor=gray!40,%
	innertopmargin=8pt,%
	innerbottommargin=8pt,
	ntheorem]{definition}{Definizione}[section]

% Exercise frame
\theoremstyle{definition}
\newmdtheoremenv[%
	linecolor=gray,leftmargin=0,%
	rightmargin=0,
	innertopmargin=8pt,%
	innerbottommargin=8pt,
	ntheorem]{exercise}{Esercizio}[section]

% Theorem frame
\theoremstyle{definition}
\newmdtheoremenv[%
  linecolor=gray,leftmargin=0,%
  rightmargin=0,
  innertopmargin=8pt,%
  innerbottommargin=8pt,
  ntheorem]{theorem}{Teorema}[section]

\theoremstyle{definition}
\newmdtheoremenv[%
  linecolor=white,leftmargin=0,%
  rightmargin=0,
  innertopmargin=8pt,%
  innerbottommargin=8pt,
  ntheorem]{define}{Definizione utile}[section]

% figure support
\usepackage{import}
\usepackage{xifthen}
\pdfminorversion=7
\usepackage{pdfpages}
\usepackage{transparent}
\newcommand{\incfig}[1]{%
	\def\svgwidth{\columnwidth}
	\import{./figures/}{#1.pdf_tex}
}

% FSM tikz
\tikzset{
    place/.style={
        circle,
        thick,
        draw=black,
        minimum size=6mm,
    },
        state/.style={
        circle,
        thick,
        draw=black,
        fill=white,
        minimum size=6mm,
    },
}

\pdfsuppresswarningpagegroup=1

\usepackage{pgfplots}
\pgfplotsset{compat=1.18,width=10cm}

% Save plots as pdf and reuse them without compiling every time
\usetikzlibrary{external}
\tikzexternalize[prefix=figures/tikz/, optimize=false]


\begin{document}

\begin{titlepage}
	\begin{center}
		\vspace*{1cm}

		\Huge
		\textbf{Probabilità e Statistica\\Esercizi}

		\vspace{0.5cm}
		\LARGE
		UniVR - Dipartimento di Informatica

		\vspace{1.5cm}

		\textbf{Fabio Irimie}

		\vfill


		\vspace{0.8cm}


		2° Semestre 2023/2024

	\end{center}
\end{titlepage}


\tableofcontents
\pagebreak

\section{Introduzione}

\section{Sistema Operativo}
Il \textbf{sistema operativo} è il livello del software che si pone tra l'hardware
e gli utenti. E quindi il sistema operativo incapsula la macchina fisica.
Per mettere in comunicazione l'utente e l'hardware solitamente si usano le 
\textbf{applicazioni}, ma quando si vuole accedere direttamente all'hardware
si usano le interfacce utente, ad esempio:
\begin{enumerate}
  \item \textbf{Interfaccia grafica} (GUI)
  \item \textbf{Command line} (Terminale o Shell)
  \item \textbf{Touch screen}
\end{enumerate}
\noindent
Gli obiettivi principali sono:
\begin{itemize}
  \item Facilitare l'uso del computer
  \item Rendere efficiente l'utilizzo dell'hardware
  \item Evitare conflitti nell'allocazione delle risorse hardware e software
\end{itemize}
Questo rimuove la necessità di conoscere la struttura dell'hardware attraverso l'
\textbf{astrazione} facilitando la programmazione.

\subsection{Compiti del sistema operativo}
\subsubsection{Gestione delle risorse}
Il sistema operativo deve gestire le risorse hardware, come ad esempio i dischi, la memoria,
gli input/output e la CPU. Deve anche gestire le risorse software, come ad esempio i file,
i programmi e la memoria virtuale.

\subsubsection{Programma di controllo}
Un altro compito del sistema operativo è quello di controllare l'esecuzione dei programmi
e del corretto utilizzo del sistema.

\subsection{Interrupt}
Un \textbf{interrupt} è un segnale hardware che interrompe il normale flusso di esecuzione
di un programma. Gli interrupt possono essere generati da:
\begin{itemize}
  \item \textbf{Hardware}: Per esempio quando un dispositivo ha finito un'operazione
  \item \textbf{Software}: Per esempio quando un programma chiama una system call
\end{itemize}
Questo serve per permettere alla CPU di lavorare per più tempo.

\subsubsection{Operazioni input/output}
Per gestire gli input/output
ad esempio si usano i \textbf{device driver} che sono programmi che permettono di
programmare la periferica per comunicare con il sistema operativo. Finchè il dispositivo
non ha finito l'operazione, la CPU esegue altri processi e quando riceve l'interrupt
dal dispositivo che segnala la fine dell'operazione, la CPU interrompe il processo
corrente e inizia a gestire l'interrupt.

\begin{itemize}
  \item \textbf{Buffering}: Sovrapposizione di CPU e I/O dello \textbf{stesso} processo
  \item \textbf{Spooling}: Sovrapposizione di CPU e I/O di \textbf{diversi} processi
\end{itemize}
Lo spooling serve quando l'elaborazione dei dati letti da un processo è più veloce
della lettura stessa, quindi si avrebbero dei tempi morti rimossi con lo spooling che
permette di elaborare i dati, già pronti, letti da un altro processo.

\subsection{Multiprogrammazione}
È la possibilità di tenere caricati in memoria più programmi e di eseguirli in modo
alternato. Questo permette di sfruttare al meglio la CPU e di ridurre i tempi morti
dovuti all'I/O.
\begin{figure}[H]
  \centering
  \begin{tikzpicture}
    \node[above left] at (0,0) {CPU};
    \node[below left] at (0,0) {I/O};
    \node[right] at (6.7,0) {Sequenziale};

    \draw (0,0) rectangle ++(1,0.5) node[midway] {$P_1$};
    \draw (1,0) rectangle ++(1.5,-0.5) node[midway] {$P_1$};
    \draw (2.5,0) rectangle ++(0.5,0.5) node[midway] {$P_1$};

    \draw (3,0) rectangle ++(1.3,0.5) node[midway] {$P_2$};
    \draw (4.3,0) rectangle ++(1.5,-0.5) node[midway] {$P_2$};
    \draw (5.7,0) rectangle ++(0.5,0.5) node[midway] {$P_2$};
  \end{tikzpicture}
  \caption{Senza multiprogrammazione}
\end{figure}
\begin{figure}[H]
  \centering
  \begin{tikzpicture}
    \node[above left] at (0,0) {CPU};
    \node[below left] at (0,0) {I/O};
    \node[right] at (5.5,0) {Multiprogrammato};

    \draw (0,0) rectangle ++(1,0.5) node[midway] {$P_1$};
    \draw (1,0) rectangle ++(1.5,-0.5) node[midway] {$P_1$};
    \draw (2.5,0) rectangle ++(0.5,0.5) node[midway] {$P_1$};

    \draw (1,0) rectangle ++(1.3,0.5) node[midway] {$P_2$};
    \draw (2.5,0) rectangle ++(1.5,-0.5) node[midway] {$P_2$};
    \draw (4,0) rectangle ++(0.5,0.5) node[midway] {$P_2$};
  \end{tikzpicture}
  \caption{Con multiprogrammazione}
\end{figure}
\noindent
Per eseguire più programmi alla volta, non soltanto quando un programma è in attesa
di I/O, si utilizza il concetto di \textbf{time sharing} che permette di eseguire
più programmi in modo alternato ad intervalli di tempo molto brevi chiamati \textbf{
quanto di tempo} creando l'impressione che i programmi vengano eseguiti in parallelo.

\vspace{1em}
\noindent
Quando si hanno tanti processi che concorrono per l'utilizzo della CPU bisogna implementare
delle regole per decidere quale processo eseguire. Queste regole sono definite dal
sistema operativo, nello specifico dallo \textbf{scheduler}.

\subsection{Protezione}
La possibilità di eseguire più processi contemporaneamente crea dei problemi, come ad
esempio la possibilità che un processo possa accedere ad un'altra area di memoria di
un altro processo. Per evitare questo si usano delle tecniche di \textbf{protezione}:
\begin{itemize}
  \item \textbf{Protezione I/O}: Programmi diversi non devono usare I/O contemporaneamente
  \item \textbf{Protezione della memoria}: Un processo non può leggere o scrivere in un'area di
    memoria che non gli appartiene
  \item \textbf{Protezione della CPU}: Un processo non può usare la CPU per più tempo di quello
    che gli è stato assegnato
\end{itemize}
In generale la protezione è realizzata tramite il meccanismo della \textbf{modalità duale}
di esecuzione:
\begin{itemize}
  \item \textbf{Modalità utente}: Il programma viene eseguito in modo normale, senza accesso
    alle risorse hardware
  \item \textbf{Modalità kernel}: Il programma viene eseguito con privilegi speciali
    che permettono di accedere all'hardware
\end{itemize}

\subsubsection{Protezione I/O}
Quando un processo vuole usare un dispositivo I/O, deve passare per il \textbf{device
driver} che esegue le operazioni di I/O in modalità \textbf{supervisor} (o kernel mode).
Questo viene fatto attraverso una \textbf{system call}, cioè un interrupt software
che permette di passare dalla modalità utente alla modalità kernel per
esequire operazioni privilegiate. Al termine della system call si ritorna in modalità
utente.

\subsubsection{Protezione della memoria}
Per proteggere la memoria si devono imporre dei limiti di accesso alla memoria ai processi.
Il sistema operativo tiene traccia di questi limiti e li controlla ad ogni accesso alla 
memoria. 

\subsubsection{Protezione della CPU}
Per proteggere la CPU bisogna garantire che il sistema operativo abbia sempre il controllo
su di essa per poter interrompere qualsiasi processo che non restituisca il controllo
della CPU. Per fare ciò si usano i \textbf{timer} che interrompono il processo se
viene eseguito per troppo tempo.

\subsection{Tipi di sistema operativo}
Esistono diversi tipi di sistemi operativi che si differenziano e si classificano sulla
base di determinate architetture di calcolo e determinati scopi. I principali tipi sono:
\begin{itemize}
  \item \textbf{S.O. per PC e workstation}: Uso personale dell'elaboratore
  \item \textbf{S.O. di rete}: Separazione logica delle risorse remote e locali
  \item \textbf{S.O. distribuiti}: Non c'è la separazione logica tra risorse remote e
    locali, quindi l'accesso alle risorse remote viene effettuato nello stesso modo
    di quello alle risorse locali
  \item \textbf{S.O. real-time}: Vincoli sui tempi di risposta del sistema
  \item \textbf{S.O. embedded}: Per dispositivi con risorse limitate
\end{itemize}

\section{Componenti di un sistema operativo}
Le componenti principali di un sistema operativo sono:
\begin{itemize}
  \item \textbf{Gestione dei processi}
  \item \textbf{Gestione della memoria primaria} (RAM): Spazio dei processi
  \item \textbf{Gestione della memoria secondaria} (Memoria di massa): Spazio dei
    programmi e dei dati
  \item \textbf{Gestione dell'I/O}
  \item \textbf{Gestione dei file}
  \item \textbf{Protezione}
  \item \textbf{Rete}
  \item \textbf{Interprete dei comandi}
\end{itemize}

\subsection{Gestione dei processi}
Un processo è l'istanza di un programma in esecuzione. Ogni processo ha bisogno di
risorse per poter essere eseguito, come ad esempio la CPU, la memoria, i file e i
dispositivi I/O. Il sistema operativo deve gestire i processi e assegnare le risorse per
garantire che i processi vengano eseguiti in modo corretto e senza conflitti e deve
gestire anche se stesso perchè anche il sistema operativo è un'insieme di processi in
modalità kernel.

\vspace{1em}
\noindent
Il sistema operativo è responsabile della:
\begin{itemize}
  \item \textbf{Creazione e distruzione di processi}: Creare un processo significa
    allocare le risorse necessarie e inizializzare le strutture dati del processo;
    distruggere un processo significa rilasciare le risorse e liberare la memoria

  \item \textbf{Sospensione e riesumazione di processi}: Un processo viene sospeso
    quando deve aspettare un evento, come ad esempio un I/O

  \item \textbf{Sincronizzazione e comunicazione tra processi}: I processi devono
    poter comunicare tra di loro e sincronizzarsi per evitare conflitti
\end{itemize}

\subsection{Gestione della memoria primaria}
Un programma deve essere caricato in memoria per essere eseguito e nella RAM sono anche
presenti dati condivisi tra CPU e dispositivi I/O.

\vspace{1em}
\noindent
Il sistema operativo è responsabile di:
\begin{itemize}
  \item \textbf{Gestire lo spazio di memoria}: Decide quali parti della memoria allocare
    e a quali processi assegnarla

  \item \textbf{Decidere quale processo caricare in memoria quando esiste spazio disponibile}

  \item \textbf{Gestire l'allocazione e il rilascio dello spazio di memoria}
\end{itemize}

\subsection{Gestione della memoria secondaria}
La memoria secondaria è usata per memorizzare programmi e dati che non possono stare
nella memoria primaria e per mantenere grandi quantità di dati in modo permanente. Il
sistema operativo deve gestire l'ottimizzazione dell'uso della memoria secondaria.

\vspace{1em}
\noindent
Il sistema operativo è responsabile di:
\begin{itemize}
  \item \textbf{Gestire lo spazio sulla memoria di massa}

  \item \textbf{Allocare spazio su memoria di massa}

  \item \textbf{Ordinare gli accessi ai dispositivi}
\end{itemize}

\subsection{Gestione dell'I/O}
Il sistema operativo deve creare un livello software che permetta all'utente di usare
le periferiche senza doverne capire le caratteristiche fisiche.

\vspace{1em}
\noindent
Il sistema di I/O consiste di:
\begin{itemize}
  \item \textbf{Sistema per accumulare gli accessi ai dispositivi}: Buffering, cioè
    l'accumulo di dati in memoria prima di essere scritti o letti dal dispositivo

  \item \textbf{Generica interfaccia verso i device driver}

  \item \textbf{Device driver specifici per alcuni dispositivi}
\end{itemize}

\subsection{Gestione dei file}
I file servono per memorizzare informazioni su supporti fisici diversi controllati da
driver con caratteristiche diverse. Un file è l'\textbf{astrazione logica} per rendere
conveniente l'uso della memoria non volatile e permette di raccogliere informazioni
correlate, come ad esempio dati o programmi.

\vspace{1em}
\noindent
Il sistema operativo è responsabile di:
\begin{itemize}
  \item \textbf{Creare e cancellare file e directory}

  \item \textbf{Fornire primitive per la gestione di file e directory}: Ad esempio
    aprire, leggere, scrivere, chiudere, rinominare, cancellare ecc...

  \item \textbf{Gestire la corrispondnza tra file e spazio fisico su memoria di massa}:
    Bisogna tenere traccia di dove sono memorizzati i file

  \item \textbf{Salvare informazioni a scopo di backup}
\end{itemize}

\subsection{Protezione}
È il meccanismo per controllare l'accesso alle risorse da parte di utenti e processi.

\vspace{1em}
\noindent
Il sistema operativo è responsabile di:
\begin{itemize}
  \item \textbf{Definire accessi autorizzati e non}

  \item \textbf{Definire controlli per gli accessi}

  \item \textbf{Fornire strumenti per verificare le politiche di accesso}
\end{itemize}

\subsection{Rete}
Il sistema operativo deve gestire tutte le risorse di calcolo connesse tramite una rete.

\vspace{1em}
\noindent
Il sistema operativo è responsabile della gestione di:
\begin{itemize}
  \item \textbf{Processi distribuiti}
  \item \textbf{Memoria distribuita}
  \item \textbf{File system distribuito}
\end{itemize}

\subsection{Interprete dei comandi}
L'interprete dei comandi è un programma che permette all'utente di comunicare con il
sistema operativo. L'interprete dei comandi permette di:
\begin{itemize}
  \item \textbf{Creare e gestire processi}
  \item \textbf{Gestire l'I/O}
  \item \textbf{Gestire il disco, la memoria, il file system}
  \item \textbf{Gestire le protezioni}
  \item \textbf{Gestire la rete}
\end{itemize}
Un interprete dei comandi è la \textbf{shell}, che è un programma che legge e interpreta i comandi.

\subsection{System call}
Sono un'interfaccia tra processi e sistema operativo e permettono ai processi di
comunicare con il sistema operativo. Le system call sono chiamate tramite un'istruzione
software e permettono di eseguire operazioni privilegiate. Le possibili opzioni per
comunicare tra sistema operativo e processo sono:
\begin{enumerate}
  \item Passare i parametri della system call tramite registri
  \item Passare i parametri tramite lo stack del programma
    \begin{lstlisting}[language=C]
    // Programma utente
    void main() {
      ...
      A(x);
      ...
    }

    A(int x) {
      ...
      push x;
      _A(); // Vera e propria system call
      ...
    }

    _A() {
      scrivi 13
      TRAP // Interruzione software
      ...
    }
    \end{lstlisting}
    \begin{lstlisting}[language=C]
    // Sistema operativo
    Leggi 13
    Salta al gestore 13

    handler_13() {
      ...
    }
    \end{lstlisting}
  \item Memorizzare i parametri in una tabella in memoria
    \begin{itemize}
      \item L'indirizzo della tabella è passato in un registro o nello stack
    \end{itemize}
    \begin{figure}[H]
      \centering
      \begin{tikzpicture}
        \draw (0,0) rectangle ++(2,3);
        \node[below] at (1,0) {Prog. utente};
        \node at (1,2) {\texttt{syscall(x)}};
        \node[align=center,scale=0.7] (sys) at (1,1) {Carica indirizzo \texttt{x}\\System call \#13};

        \node[draw,minimum width=1.5cm] (x) at (4,3) {$x$};
        \node[above,align=center,scale=0.8] at (x.north) {Indirizzo della tabella\\in memoria};
        \node[below,align=center,scale=0.8] at (x.south) {Registro};

        \draw (6,0) rectangle ++(2,3);
        \draw (6,0.8) rectangle ++(2,1.4);

        \draw (8.2,2.2) -- ++(0.2,0) -- ++(0,-1.4) node[midway,right,align=center,scale=0.8]
          {Codice per\\system call\\\#13} -- ++(-0.2,0);

        \node[align=center,scale=0.8] (call) at (7,1.5) {Usa parametri\\contenuti nella\\tabella con\\indirizzo $x$};
        \node[below] at (7,0) {S.O.};

        \draw[->] (sys.east) -- (x.west);
        \draw[->] (x.east) -- (call.west);
      \end{tikzpicture}
      \caption{Passaggio dei parametri tramite tabella in memoria}
    \end{figure}
\end{enumerate}

\subsection{Programmi di sistema}
I programmi di sistema sono un'interfaccia comoda che ha l'utente delle operazioni che il
sistema operativo può fare tramite le sue system call, ad esempio:
\begin{itemize}
  \item \textbf{Gestione/manipolazione dei file}: Crea, copia, cancella...
  \item \textbf{Informazioni sullo stato del sistema}: Uso della CPU, data, spazio su 
    disco...
  \item \textbf{Strumenti di supporto alla programmazione}: Compilatori, assemblatori,
    editor...
  \item \textbf{Programmi di gestione della rete}: Login remoto, trasferimento file...
  \item \textbf{Formattazione documenti}
  \item \textbf{Mail}
  \item \textbf{Interprete dei comandi}
  \item \textbf{Utility varie}
\end{itemize}

% Il sistema operativo fornisce dei servizi per comunicare con l'hardware e questi
% servizi possono essere usati tramite delle \textbf{system calls}, ad esempio:
%
% \begin{itemize}
%   \item Esecuzione dei programmi
%   \item Gestione dei file
%   \item Operazioni I/O
%   \item Gestione degli errori
%   \item Comunicazione
% \end{itemize}
%
% \noindent
% L'unico programma che è sempre in esecuzione su un computer è il \textbf{kernel}.
%
% \subsection{Operazioni}
% \begin{itemize}
%   \item \textbf{Bootstrap program}: è una piccola porzione di codice che inizializza
%     il sistema e carica il kernel
%
%   \item Viene caricato il kernel
%
%   \item Vengono caricati i \textbf{system daemons}, cioè dei servizi forniti al
%     di fuori del kernel
%
%   \item Gestione delle chiamate di sistema:
%     \begin{itemize}
%       \item Hardware interrupt
%       \item Software interrupt
%     \end{itemize}
% \end{itemize}
%
% \noindent
% Su un computer vengono eseguiti più programmi alla volta salvando la coda dei
% processi da eseguire in memoria e lo \textbf{scheduler} si occupa di gestire
% l'ordine di esecuzione e di interruzione. Per permettere di eseguire più processi
% alla volta si utilizza il \textbf{time sharing}, cioè la CPU cambia processo così
% frequentemente che si crea l'illusione che i processi vengano eseguiti in parallelo
% anche se in realtà non è così.
%
% \section{Modalità di esecuzione}
% Il sistema operativo può eseguire il codice in 2 modalità:
% \begin{enumerate}
%   \item \textbf{Modalità utente}: il codice viene eseguito in modo normale
%   \item \textbf{Modalità kernel}: il codice viene eseguito con privilegi speciali
%     che permettono di accedere all'hardware
% \end{enumerate}
%
% \noindent
% Per capire in che modalità si sta eseguendo il codice si utilizza un bit nel
% \textbf{program status word} (PSW) che indica la modalità di esecuzione chiamato
% \textbf{mode bit}.
%
% \noindent Si può entrere in modalità kernel tramite:
% \begin{enumerate}
%   \item \textbf{Nuovo processo}: Per creare un nuovo processo, il kernel copia il
%     programma nella memoria, setta il program counter alla prima istruzione del
%     processo e setta lo stack pointer alla base dello stack del processo e infine
%     si torna in modalità utente. 
%
%     \item \textbf{Ritorno da un interrupt o system call}: Quando il kernel finisce
%       di gestire la richiesta, riprende l'esecuzione del processo che ha chiamato
%       l'interrupt o la system call e torna in modalità utente.
%       
%     \item \textbf{Cambio di contesto}: Quando il kernel decide di cambiare il
%       processo in esecuzione, salva lo stato del processo corrente e carica lo
%       stato del nuovo processo e torna in modalità utente.
% \end{enumerate}
%
% \noindent
% Per prevenire che un processo faccia un ciclo infinito esiste un timer che interrompe
% il processo se viene eseguito per troppo tempo.
%
% \subsection{Protezione e sicurezza}
% \begin{enumerate}
%   \item \textbf{Protezione}: qualsiasi meccanismo per controllare l'accesso dei
%     processi o degli utenti alle risorse del sistema.
%
%   \item \textbf{Sicurezza}: protezione da accessi esterni non autorizzati.
% \end{enumerate}
%
% \noindent
% Ogni utente è identificato da un \textbf{user id} e ogni utente può far parte di
% un gruppo, identificato da un \textbf{group id}. Ogni file ha un \textbf{owner} e
% un \textbf{group owner} e per ogni file ci sono 3 tipi di permessi:
% \begin{enumerate}
%   \item \textbf{Read}
%   \item \textbf{Write}
%   \item \textbf{Execute}
% \end{enumerate}
%
% \section{Linux}
% \subsection{Filesystem}
% In linux qualsiasi cosa è un file, cioè un contenitore di dati. I principali
% tipi di file sono:
% \begin{enumerate}
%   \item \textbf{Regolare}: File classici utente
%   \item \textbf{Directory}: Informazioni relative ad altri file
%   \item \textbf{Pipe}: Comunicazioni tra processi
%   \item \textbf{Link}: Alias
%   \item \textbf{Socket}: Comunicazioni tra processi
%   \item \textbf{Speciale}: Dispositivi
% \end{enumerate}
%
% \noindent Tutti i file sono organizzati in un albero, chiamato \textbf{file system}.
% \begin{enumerate}
%   \item \textbf{root}: La radice del file system: /
%   \item \textbf{nodo}: File (se ha già figli è una directory)
% \end{enumerate}
%
%
\end{document}
