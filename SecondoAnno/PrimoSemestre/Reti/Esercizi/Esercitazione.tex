\documentclass[a4paper]{article}
\usepackage{import}
\usepackage[utf8]{inputenc}
\usepackage[T1]{fontenc}
\usepackage{textcomp}
\usepackage[italian]{babel}
\usepackage{amsmath, amssymb}
\usepackage{booktabs,xltabular}
\usepackage{amsfonts}
\usepackage{subcaption}
\usepackage{amsthm}
\usepackage{cancel}
\usepackage{mdframed}
\usepackage{makecell}
\usepackage{float}
\usepackage{xcolor}
\usepackage{listings}
\usepackage{gensymb}
\usepackage{graphicx}
\usepackage{bodeplot}
\usepackage{physics}
\usepackage{tikz}
\usetikzlibrary{shapes, arrows, automata, petri, decorations.markings, decorations.pathreplacing, positioning, calc, quotes}
\usepackage{circuitikz}
\usepackage[label=corner]{karnaugh-map}
\graphicspath{{./figures/}}

% Set default font to sans-serif
\renewcommand{\familydefault}{\sfdefault} 
\usepackage{eulervm}

\usepackage{forest}

\usepackage{mathtools}
\DeclarePairedDelimiter\ceil{\lceil}{\rceil}
\DeclarePairedDelimiter\floor{\lfloor}{\rfloor}

% \usepackage{ntheorem}

\usepackage{import}
\usepackage{pdfpages}
\usepackage{transparent}
\usepackage{xcolor}

\usepackage{hyperref}
\hypersetup{
    colorlinks=false,
}

% Code blocks
\definecolor{codegreen}{rgb}{0,0.6,0}
\definecolor{codegray}{rgb}{0.5,0.5,0.5}
\definecolor{codepurple}{rgb}{0.58,0,0.82}
\definecolor{backcolour}{rgb}{0.95,0.95,0.95}

\lstdefinestyle{mystyle}{
	backgroundcolor=\color{backcolour},
	commentstyle=\color{codegreen},
	keywordstyle=\color{magenta},
	numberstyle=\tiny\color{codegray},
	stringstyle=\color{codepurple},
	basicstyle=\ttfamily\footnotesize,
	breakatwhitespace=false,
	breaklines=true,
	captionpos=b,
	keepspaces=true,
	numbers=left,
	numbersep=5pt,
	showspaces=false,
	showstringspaces=false,
	showtabs=false,
	tabsize=2
}

\lstset{style=mystyle}

\usepackage{color}
\usepackage{import}
\usepackage{pdfpages}
\usepackage{transparent}
\usepackage{xcolor}

% Example frame
\theoremstyle{definition}
\newmdtheoremenv[%
	linecolor=gray,leftmargin=0,%
	rightmargin=0,
	innertopmargin=8pt,%
	innerbottommargin=8pt,
	ntheorem]{example}{Esempio}[section]

% Important definition frame
\theoremstyle{definition}
\newmdtheoremenv[%
	linecolor=gray,leftmargin=0,%
	rightmargin=0,
	backgroundcolor=gray!40,%
	innertopmargin=8pt,%
	innerbottommargin=8pt,
	ntheorem]{definition}{Definizione}[section]

% Exercise frame
\theoremstyle{definition}
\newmdtheoremenv[%
	linecolor=gray,leftmargin=0,%
	rightmargin=0,
	innertopmargin=8pt,%
	innerbottommargin=8pt,
	ntheorem]{exercise}{Esercizio}[section]

% Theorem frame
\theoremstyle{definition}
\newmdtheoremenv[%
  linecolor=gray,leftmargin=0,%
  rightmargin=0,
  innertopmargin=8pt,%
  innerbottommargin=8pt,
  ntheorem]{theorem}{Teorema}[section]

\theoremstyle{definition}
\newmdtheoremenv[%
  linecolor=white,leftmargin=0,%
  rightmargin=0,
  innertopmargin=8pt,%
  innerbottommargin=8pt,
  ntheorem]{define}{Definizione utile}[section]

% figure support
\usepackage{import}
\usepackage{xifthen}
\pdfminorversion=7
\usepackage{pdfpages}
\usepackage{transparent}
\newcommand{\incfig}[1]{%
	\def\svgwidth{\columnwidth}
	\import{./figures/}{#1.pdf_tex}
}

% FSM tikz
\tikzset{
    place/.style={
        circle,
        thick,
        draw=black,
        minimum size=6mm,
    },
        state/.style={
        circle,
        thick,
        draw=black,
        fill=white,
        minimum size=6mm,
    },
}

\pdfsuppresswarningpagegroup=1

\usepackage{pgfplots}
\pgfplotsset{compat=1.18,width=10cm}

% Save plots as pdf and reuse them without compiling every time
\usetikzlibrary{external}
\tikzexternalize[prefix=figures/tikz/, optimize=false]


\begin{document}

\begin{titlepage}
	\begin{center}
		\vspace*{1cm}

		\Huge
		\textbf{Probabilità e Statistica\\Esercizi}

		\vspace{0.5cm}
		\LARGE
		UniVR - Dipartimento di Informatica

		\vspace{1.5cm}

		\textbf{Fabio Irimie}

		\vfill


		\vspace{0.8cm}


		2° Semestre 2023/2024

	\end{center}
\end{titlepage}


\tableofcontents
\pagebreak

\section{Indirizzamento}
\subsection{Esercizio 1}
Qual'è l'indirizzo di rete se ho il seguente indirizzo IP:
\[
  140.120.84.20/20
\] 

\subsubsection{Risoluzione}
L'indirizzo di rete corrisponde ai primi 20 bit dell'indirizzo IP, quindi bisogna
passare alla notazione binaria:
\[
  140.120.84.20 \to 10001100 \;\; 01111000 \;\; 01010100 \;\; 00010100
\] 
I primi 20 bit sono assegnati al prefisso:
\[
  \underbrace{10001100 \;\; 01111000 \;\; 0101}_{\text{Prefisso}} \;\;
  \underbrace{0100 \;\; 00010100}_{\text{Suffisso}}
\] 
Per ottenere l'indirizzo di rete bisogna azzerare i bit del suffisso:
\[
  \underbrace{10001100 \;\; 01111000 \;\; 0101}_{\text{Prefisso}} \;\;
  \underbrace{0000 \;\; 00000000}_{\text{Suffisso}}
\] 
che in notazione decimale puntata diventa:
\[
  140.120.80.0
\] 
La maschera di questo IP è:
\[
  \underbrace{11111111 \;\; 11111111 \;\; 1111}_{\text{Prefisso}} \;\;
  \underbrace{0000 \;\; 00000000}_{\text{Suffisso}}
\] 
che in notazione decimale puntata diventa:
\[
  255.255.240.0
\] 

\subsection{Esercizio 2}
Si hanno 3 LAN. All'insieme delle 3 LAN è stato assegnato il blocco:
\[
  165.5.1.0/24
\] 
Creare 3 sottoreti per le 3 LAN in modo che abbiano tutte lo stesso numero di host.

\subsubsection{Risoluzione}
Per prima cosa si trasforma l'indirizzo IP in notazione binaria:
\[
  \underbrace{1010 0101 \;\; 0000 0101 \;\; 0000 0001}_{\text{Prefisso}} \;\;
  \underbrace{0000 0000}_{\text{Suffisso}}
\] 
Per poter ottenere 3 sottoreti di dimensione servono 2 bit che vengoo presi dal suffisso
per identificare ciascuna delle 3 reti:
\[
  \underbrace{1010 0101 \;\; 0000 0101 \;\; 0000 0001}_{\text{Prefisso}} \;
  \underbrace{00}_{\text{Sottorete}} \;
  \underbrace{00 0000}_{\text{Suffisso}}
\] 
Le combinazioni possibili sono:
\begin{itemize}
  \item 
    \(
    \underbrace{1010 0101 \;\; 0000 0101 \;\; 0000 0001}_{\text{Prefisso}} \;
    \underbrace{00}_{\text{Sottorete}} \;
    \underbrace{00 0000}_{\text{Suffisso}}
    \) 

  \item 
    \(
    \underbrace{1010 0101 \;\; 0000 0101 \;\; 0000 0001}_{\text{Prefisso}} \;
    \underbrace{01}_{\text{Sottorete}} \;
    \underbrace{00 0000}_{\text{Suffisso}}
    \) 

  \item 
    \(
    \underbrace{1010 0101 \;\; 0000 0101 \;\; 0000 0001}_{\text{Prefisso}} \;
    \underbrace{10}_{\text{Sottorete}} \;
    \underbrace{00 0000}_{\text{Suffisso}}
    \) 

  \item 
    \(
    \underbrace{1010 0101 \;\; 0000 0101 \;\; 0000 0001}_{\text{Prefisso}} \;
    \underbrace{11}_{\text{Sottorete}} \;
    \underbrace{00 0000}_{\text{Suffisso}}
    \) 
\end{itemize}
Ci troviamo con 4 sottoreti con lo stesso numero di indirizzi \( \left( 2^6 = 64 \right) \).
Di queste 4 sottoreti ne utilizziamo 3 e l'ultima rimane libera per utilizzi futuri.

\vspace{1em}
\noindent
Traducendo i blocchi in notazione decimale puntata si ha:
\[
  \begin{aligned}
    165.5.1.0/26 \to \text{LAN 1}\\
    165.5.1.64/26 \to \text{LAN 2}\\
    165.5.1.128/26 \to \text{LAN 3}\\
    165.5.1.192/26 \to \text{Libero}
  \end{aligned}
\] 

\end{document}
