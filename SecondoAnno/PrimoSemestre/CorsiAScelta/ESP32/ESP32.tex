\documentclass[a4paper]{article}
\usepackage{import}
\usepackage[utf8]{inputenc}
\usepackage[T1]{fontenc}
\usepackage{textcomp}
\usepackage[italian]{babel}
\usepackage{amsmath, amssymb}
\usepackage{booktabs,xltabular}
\usepackage{amsfonts}
\usepackage{subcaption}
\usepackage{amsthm}
\usepackage{cancel}
\usepackage{mdframed}
\usepackage{makecell}
\usepackage{float}
\usepackage{xcolor}
\usepackage{listings}
\usepackage{gensymb}
\usepackage{graphicx}
\usepackage{bodeplot}
\usepackage{physics}
\usepackage{tikz}
\usetikzlibrary{shapes, arrows, automata, petri, decorations.markings, decorations.pathreplacing, positioning, calc, quotes}
\usepackage{circuitikz}
\usepackage[label=corner]{karnaugh-map}
\graphicspath{{./figures/}}

% Set default font to sans-serif
\renewcommand{\familydefault}{\sfdefault} 
\usepackage{eulervm}

\usepackage{forest}

\usepackage{mathtools}
\DeclarePairedDelimiter\ceil{\lceil}{\rceil}
\DeclarePairedDelimiter\floor{\lfloor}{\rfloor}

% \usepackage{ntheorem}

\usepackage{import}
\usepackage{pdfpages}
\usepackage{transparent}
\usepackage{xcolor}

\usepackage{hyperref}
\hypersetup{
    colorlinks=false,
}

% Code blocks
\definecolor{codegreen}{rgb}{0,0.6,0}
\definecolor{codegray}{rgb}{0.5,0.5,0.5}
\definecolor{codepurple}{rgb}{0.58,0,0.82}
\definecolor{backcolour}{rgb}{0.95,0.95,0.95}

\lstdefinestyle{mystyle}{
	backgroundcolor=\color{backcolour},
	commentstyle=\color{codegreen},
	keywordstyle=\color{magenta},
	numberstyle=\tiny\color{codegray},
	stringstyle=\color{codepurple},
	basicstyle=\ttfamily\footnotesize,
	breakatwhitespace=false,
	breaklines=true,
	captionpos=b,
	keepspaces=true,
	numbers=left,
	numbersep=5pt,
	showspaces=false,
	showstringspaces=false,
	showtabs=false,
	tabsize=2
}

\lstset{style=mystyle}

\usepackage{color}
\usepackage{import}
\usepackage{pdfpages}
\usepackage{transparent}
\usepackage{xcolor}

% Example frame
\theoremstyle{definition}
\newmdtheoremenv[%
	linecolor=gray,leftmargin=0,%
	rightmargin=0,
	innertopmargin=8pt,%
	innerbottommargin=8pt,
	ntheorem]{example}{Esempio}[section]

% Important definition frame
\theoremstyle{definition}
\newmdtheoremenv[%
	linecolor=gray,leftmargin=0,%
	rightmargin=0,
	backgroundcolor=gray!40,%
	innertopmargin=8pt,%
	innerbottommargin=8pt,
	ntheorem]{definition}{Definizione}[section]

% Exercise frame
\theoremstyle{definition}
\newmdtheoremenv[%
	linecolor=gray,leftmargin=0,%
	rightmargin=0,
	innertopmargin=8pt,%
	innerbottommargin=8pt,
	ntheorem]{exercise}{Esercizio}[section]

% Theorem frame
\theoremstyle{definition}
\newmdtheoremenv[%
  linecolor=gray,leftmargin=0,%
  rightmargin=0,
  innertopmargin=8pt,%
  innerbottommargin=8pt,
  ntheorem]{theorem}{Teorema}[section]

\theoremstyle{definition}
\newmdtheoremenv[%
  linecolor=white,leftmargin=0,%
  rightmargin=0,
  innertopmargin=8pt,%
  innerbottommargin=8pt,
  ntheorem]{define}{Definizione utile}[section]

% figure support
\usepackage{import}
\usepackage{xifthen}
\pdfminorversion=7
\usepackage{pdfpages}
\usepackage{transparent}
\newcommand{\incfig}[1]{%
	\def\svgwidth{\columnwidth}
	\import{./figures/}{#1.pdf_tex}
}

% FSM tikz
\tikzset{
    place/.style={
        circle,
        thick,
        draw=black,
        minimum size=6mm,
    },
        state/.style={
        circle,
        thick,
        draw=black,
        fill=white,
        minimum size=6mm,
    },
}

\pdfsuppresswarningpagegroup=1

\usepackage{pgfplots}
\pgfplotsset{compat=1.18,width=10cm}

% Save plots as pdf and reuse them without compiling every time
\usetikzlibrary{external}
\tikzexternalize[prefix=figures/tikz/, optimize=false]


\begin{document}

\begin{titlepage}
	\begin{center}
		\vspace*{1cm}

		\Huge
		\textbf{Probabilità e Statistica\\Esercizi}

		\vspace{0.5cm}
		\LARGE
		UniVR - Dipartimento di Informatica

		\vspace{1.5cm}

		\textbf{Fabio Irimie}

		\vfill


		\vspace{0.8cm}


		2° Semestre 2023/2024

	\end{center}
\end{titlepage}


\tableofcontents
\pagebreak

\section{Introduzione}
\subsection{Tipi di microcontrollori}
\subsubsection{STM}
Hanno più risorse hardware e ampio supporto dalla community. Fornisce come strumento
\textbf{STM32CubeMX} che permette di configurare il microcontrollore in modo grafico,
come ad esempio i GPIO (General Purpose Input Output) ecc... per dire alla CPU come 
interagire con ogni pin.

\subsubsection{Espressif}
L'ESP32 è il microcontrollore utilizzato all'interno di questo corso. Usato per il basso
costo e facilità di utilizzo e l'ambiente di sviluppo (\textbf{ESP-IDF SDK}) è fatto
molto bene. La documentazione è completa con tanto di esempi. Questo microcontrollore
è basato su \textbf{FreeRTOS}, un sistema operativo real time.

\subsection{Sistema operativo FreeRTOS}
È un sistema operativo open source altamente configurabile che fornisce primitive per
la creazione di thread, mutex, semafori, timer e memoria dinamica. Questo più che un 
sistema operativo come Linux, è un kernel che fornisce le primitive per la gestione
dei thread e delle risorse, come ad esempio lo scheduler ecc...

Un sistema operativo real time ha la caratteristica di avere un tempo di risposta
garantito, cioè se un evento deve essere gestito in un tempo massimo, il sistema
operativo deve garantire che l'evento venga gestito entro quel tempo.

\subsubsection{Schemmi di allocazione}
FreeRTOS mette a disposizione varie strategie di allocazione:
\begin{itemize}
  \item Solo allocazione
  \item Allocazione e deallocazione semplice
  \item Allocazione e deallocazione in coalescenza
  \item Coalescenza e heap frammentato
  \item Allocazione e deallocazione completa con mutua esclusione
\end{itemize}

\subsubsection{Estensioni}
FreeRTOS dà la possibilità di estendere il sistema operativo mediante librerie con
varie funzionalità come:
\begin{itemize}
  \item Stack TCP/IP
  \item File system
  \item Command line interface
  \item Interfaccia per i driver
  \item Ecc...
\end{itemize}

\section{Configurazione dell'ambiente di sviluppo}
La seguente configurazione è stata fatta su un sistema operativo Linux con il 
microcontrollore ESP32.

\subsection{Installazione del toolchain}
Per scaricare la toolchain su Visual Studio Code installare l'estensione \textbf{ESP-IDF}
e seguire le istruzioni presenti nell'estensione.

\subsubsection{Linux + Qualsiasi editor}
Il manuale di installazione si trova a: \href{https://docs.espressif.com/projects/esp-idf/en/v5.4/esp32/get-started/linux-macos-setup.html}{\textit{\underline{getting-started}}}.

\begin{enumerate}
  \item Installare i prerequisiti
    \begin{itemize}
      \item Fedora

\begin{lstlisting}[language=Bash]
sudo dnf install git wget flex bison gperf python3 cmake ninja-build ccache dfu-util libusbx
\end{lstlisting}

      \item Ubuntu e Debian

\begin{lstlisting}[language=Bash]
sudo apt-get install git wget flex bison gperf python3 python3-pip python3-venv cmake ninja-build ccache libffi-dev libssl-dev dfu-util libusb-1.0-0
\end{lstlisting}

      \item CentOS 7 \& 8

\begin{lstlisting}[language=Bash]
sudo yum -y update && sudo yum install git wget flex bison gperf python3 cmake ninja-build ccache dfu-util libusbx
\end{lstlisting}

      \item Arch

\begin{lstlisting}[language=Bash]
sudo pacman -S --needed gcc git make flex bison gperf python cmake ninja ccache dfu-util libusb
\end{lstlisting}

    \end{itemize}

  \item Scaricare le librerie fornite da Espressif in 
    \href{https://github.com/espressif/esp-idf}{\textit{\underline{ESP-IDF repository}}}.

\begin{lstlisting}[language=Bash]
mkdir -p ~/esp
cd ~/esp
git clone -b v5.4 --recursive https://github.com/espressif/esp-idf.git
\end{lstlisting}

    ESP-IDF verrà scaricato in \lstinline{~/esp/esp-idf}.

  \item Installazione dei tool

    \vspace{1em}
    \noindent
    Oltre alle librerie, è necessario installare gli strumenti necessari usati da ESP-IDF,
    come ad esempio il compilatore, il debugger, i pacchetti python ecc...

\begin{lstlisting}[language=Bash]
cd ~/esp/esp-idf
./install.sh esp32
\end{lstlisting}
    
    Questo script installerà tutti i pacchetti necessari per lo sviluppo con ESP32. Se
    si vuole installare per più microcontrollori, basta passare il nome del microcontrollore
    come argomento:

\begin{lstlisting}[language=Bash]
cd ~/esp/esp-idf
./install.sh esp32,esp32s2
\end{lstlisting}

    Oppure per scaricare gli strumenti per tutti i target supportati:

\begin{lstlisting}[language=Bash]
cd ~/esp/esp-idf
./install.sh all
\end{lstlisting}

  \item Configurare le variabili d'ambiente

    \vspace{1em}
    \noindent
    ESP-IDF fornisce uno script che imposta le variabili d'ambiente necessarie per
    compilare il codice. Questo script è da eseguire ogni volta che si vuole sviluppare
    con ESP-IDF:

\begin{lstlisting}[language=Bash]
. $HOME/esp/esp-idf/export.sh
\end{lstlisting}

    Per facilitare l'uso di questo script, si può aggiungere al file \lstinline{.bashrc}
    o \lstinline{.zshrc} il seguente alias (riavviare il terminale, o fare il source
    del file per rendere effettive le modifiche):

\begin{lstlisting}[language=Bash]
alias get_idf='. $HOME/esp/esp-idf/export.sh'
\end{lstlisting}

    In questo modo, ogni volta che si apre un terminale, basta eseguire il comando
    \lstinline{get_idf} per impostare le variabili d'ambiente.
\end{enumerate}

\subsection{Creazione ed esecuzione di un progetto}
Prima di creare un progetto bisogna entrare nell'ambiente di sviluppo con il comando
\lstinline{get_idf}. Questo comando rende disponibile il comando \lstinline{idf.py}.
\begin{itemize}
  \item Creare un progetto

    \vspace{1em}
    \noindent
    Per creare un progetto eseguire il seguente comando:

\begin{lstlisting}[language=Bash]
idf.py create-project <nome-progetto>
\end{lstlisting}
    
  \item Connetti il microcontrollore

    \vspace{1em}
    \noindent
    Connettere il microcontrollore e verificare in quale porta seriale è disponibile.
    Su linux le porte seriali si trovano in \lstinline{/dev/tty*}. Per verificare la
    lista delle porte seriali disponibili, eseguire il seguente comando:

\begin{lstlisting}[language=Bash]
ls /dev/tty*
\end{lstlisting}
      
    Se collegato tramite usb, il microcontrollore dovrebbe essere disponibile in
    \lstinline{/dev/ttyUSB0}. Questa porta seriale potrebbe essere in possesso di
    \textbf{root}, quindi per poterla usare bisogna dare i permessi all'utente:

\begin{lstlisting}[language=Bash]
sudo usermod -a -G dialout $USER
sudo chmod a+rw /dev/ttyUSB0
\end{lstlisting}

  \item Configurare il progetto

    \vspace{1em}
    \noindent
    Per configurare il progetto, entrare nella cartella del progetto e lanciare il
    seguente comando per impostare il microcontrollore di target:

\begin{lstlisting}[language=Bash]
idf.py set-target esp32
\end{lstlisting}

    Successivamente, per configurare il progetto, eseguire il seguente comando:

    \begin{lstlisting}[language=Bash]
idf.py menuconfig
\end{lstlisting}

    Questo comando aprirà un menu con tutte le configurazioni del progetto.

  \item Compilare il progetto

    \vspace{1em}
    \noindent
    Per compilare il progetto, eseguire il seguente comando:
\begin{lstlisting}[language=Bash]
idf.py build
\end{lstlisting}

  \item Flash del microcontrollore

    \vspace{1em}
    \noindent
    Per caricare il programma sul microcontrollore (flash), eseguire il seguente comando:

\begin{lstlisting}[language=Bash]
idf.py -p PORT flash
\end{lstlisting}

    Dove \lstinline{PORT} è la porta seriale in cui è collegato il microcontrollore.

  \item Monitorare il microcontrollore

    \vspace{1em}
    \noindent
    Per monitorare il microcontrollore, eseguire il seguente comando:

\begin{lstlisting}[language=Bash]
idf.py -p PORT monitor
\end{lstlisting}

  \item Consigli
    \begin{itemize}
      \item 
        Il flash compila in automatico, quindi per compilare, flashare e fare il monitoraggio
        si può eseguire il seguente comando:

\begin{lstlisting}[language=Bash]
idf.py -p PORT flash monitor
\end{lstlisting}

      \item Per eliminare il flash, eseguire il seguente comando:

\begin{lstlisting}[language=Bash]
idf.py -p PORT erase-flash
\end{lstlisting}

    \end{itemize}

\end{itemize}

\end{document}
