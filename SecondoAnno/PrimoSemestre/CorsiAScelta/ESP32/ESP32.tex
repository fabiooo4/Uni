\documentclass[a4paper]{article}
\usepackage{import}
\usepackage[utf8]{inputenc}
\usepackage[T1]{fontenc}
\usepackage{textcomp}
\usepackage[italian]{babel}
\usepackage{amsmath, amssymb}
\usepackage{booktabs,xltabular}
\usepackage{amsfonts}
\usepackage{subcaption}
\usepackage{amsthm}
\usepackage{cancel}
\usepackage{mdframed}
\usepackage{makecell}
\usepackage{float}
\usepackage{xcolor}
\usepackage{listings}
\usepackage{gensymb}
\usepackage{graphicx}
\usepackage{bodeplot}
\usepackage{physics}
\usepackage{tikz}
\usetikzlibrary{shapes, arrows, automata, petri, decorations.markings, decorations.pathreplacing, positioning, calc, quotes}
\usepackage{circuitikz}
\usepackage[label=corner]{karnaugh-map}
\graphicspath{{./figures/}}

% Set default font to sans-serif
\renewcommand{\familydefault}{\sfdefault} 
\usepackage{eulervm}

\usepackage{forest}

\usepackage{mathtools}
\DeclarePairedDelimiter\ceil{\lceil}{\rceil}
\DeclarePairedDelimiter\floor{\lfloor}{\rfloor}

% \usepackage{ntheorem}

\usepackage{import}
\usepackage{pdfpages}
\usepackage{transparent}
\usepackage{xcolor}

\usepackage{hyperref}
\hypersetup{
    colorlinks=false,
}

% Code blocks
\definecolor{codegreen}{rgb}{0,0.6,0}
\definecolor{codegray}{rgb}{0.5,0.5,0.5}
\definecolor{codepurple}{rgb}{0.58,0,0.82}
\definecolor{backcolour}{rgb}{0.95,0.95,0.95}

\lstdefinestyle{mystyle}{
	backgroundcolor=\color{backcolour},
	commentstyle=\color{codegreen},
	keywordstyle=\color{magenta},
	numberstyle=\tiny\color{codegray},
	stringstyle=\color{codepurple},
	basicstyle=\ttfamily\footnotesize,
	breakatwhitespace=false,
	breaklines=true,
	captionpos=b,
	keepspaces=true,
	numbers=left,
	numbersep=5pt,
	showspaces=false,
	showstringspaces=false,
	showtabs=false,
	tabsize=2
}

\lstset{style=mystyle}

\usepackage{color}
\usepackage{import}
\usepackage{pdfpages}
\usepackage{transparent}
\usepackage{xcolor}

% Example frame
\theoremstyle{definition}
\newmdtheoremenv[%
	linecolor=gray,leftmargin=0,%
	rightmargin=0,
	innertopmargin=8pt,%
	innerbottommargin=8pt,
	ntheorem]{example}{Esempio}[section]

% Important definition frame
\theoremstyle{definition}
\newmdtheoremenv[%
	linecolor=gray,leftmargin=0,%
	rightmargin=0,
	backgroundcolor=gray!40,%
	innertopmargin=8pt,%
	innerbottommargin=8pt,
	ntheorem]{definition}{Definizione}[section]

% Exercise frame
\theoremstyle{definition}
\newmdtheoremenv[%
	linecolor=gray,leftmargin=0,%
	rightmargin=0,
	innertopmargin=8pt,%
	innerbottommargin=8pt,
	ntheorem]{exercise}{Esercizio}[section]

% Theorem frame
\theoremstyle{definition}
\newmdtheoremenv[%
  linecolor=gray,leftmargin=0,%
  rightmargin=0,
  innertopmargin=8pt,%
  innerbottommargin=8pt,
  ntheorem]{theorem}{Teorema}[section]

\theoremstyle{definition}
\newmdtheoremenv[%
  linecolor=white,leftmargin=0,%
  rightmargin=0,
  innertopmargin=8pt,%
  innerbottommargin=8pt,
  ntheorem]{define}{Definizione utile}[section]

% figure support
\usepackage{import}
\usepackage{xifthen}
\pdfminorversion=7
\usepackage{pdfpages}
\usepackage{transparent}
\newcommand{\incfig}[1]{%
	\def\svgwidth{\columnwidth}
	\import{./figures/}{#1.pdf_tex}
}

% FSM tikz
\tikzset{
    place/.style={
        circle,
        thick,
        draw=black,
        minimum size=6mm,
    },
        state/.style={
        circle,
        thick,
        draw=black,
        fill=white,
        minimum size=6mm,
    },
}

\pdfsuppresswarningpagegroup=1

\usepackage{pgfplots}
\pgfplotsset{compat=1.18,width=10cm}

% Save plots as pdf and reuse them without compiling every time
\usetikzlibrary{external}
\tikzexternalize[prefix=figures/tikz/, optimize=false]


\begin{document}

\begin{titlepage}
	\begin{center}
		\vspace*{1cm}

		\Huge
		\textbf{Probabilità e Statistica\\Esercizi}

		\vspace{0.5cm}
		\LARGE
		UniVR - Dipartimento di Informatica

		\vspace{1.5cm}

		\textbf{Fabio Irimie}

		\vfill


		\vspace{0.8cm}


		2° Semestre 2023/2024

	\end{center}
\end{titlepage}


\tableofcontents
\pagebreak

\section{Introduzione}
\subsection{Tipi di microcontrollori}
\subsubsection{STM}
Hanno più risorse hardware e ampio supporto dalla community. Fornisce come strumento
\textbf{STM32CubeMX} che permette di configurare il microcontrollore in modo grafico,
come ad esempio i GPIO (General Purpose Input Output) ecc... per dire alla CPU come 
interagire con ogni pin.

\subsubsection{Espressif}
L'ESP32 è il microcontrollore utilizzato all'interno di questo corso. Usato per il basso
costo e facilità di utilizzo e l'ambiente di sviluppo (\textbf{ESP-IDF SDK}) è fatto
molto bene. La documentazione è completa con tanto di esempi. Questo microcontrollore
è basato su \textbf{FreeRTOS}, un sistema operativo real time.

\subsection{Sistema operativo FreeRTOS}
È un sistema operativo open source altamente configurabile che fornisce primitive per
la creazione di thread, mutex, semafori, timer e memoria dinamica. Questo più che un 
sistema operativo come Linux, è un kernel che fornisce le primitive per la gestione
dei thread e delle risorse, come ad esempio lo scheduler ecc...

Un sistema operativo real time ha la caratteristica di avere un tempo di risposta
garantito, cioè se un evento deve essere gestito in un tempo massimo, il sistema
operativo deve garantire che l'evento venga gestito entro quel tempo.

\subsubsection{Schemmi di allocazione}
FreeRTOS mette a disposizione varie strategie di allocazione:
\begin{itemize}
  \item Solo allocazione
  \item Allocazione e deallocazione semplice
  \item Allocazione e deallocazione in coalescenza
  \item Coalescenza e heap frammentato
  \item Allocazione e deallocazione completa con mutua esclusione
\end{itemize}

\subsubsection{Estensioni}
FreeRTOS dà la possibilità di estendere il sistema operativo mediante librerie con
varie funzionalità come:
\begin{itemize}
  \item Stack TCP/IP
  \item File system
  \item Command line interface
  \item Interfaccia per i driver
  \item Ecc...
\end{itemize}

\section{Configurazione dell'ambiente di sviluppo}
La seguente configurazione è stata fatta su un sistema operativo Linux con il 
microcontrollore ESP32.

\subsection{Installazione del toolchain}
Scaricare Visual Studio Code e installare l'estensione \textbf{ESP-IDF} seguendo
le istruzioni presenti nell'estensione. O scaricare l'SDK manualmente seguendo
questa guida \url{https://docs.espressif.com/projects/esp-idf/en/v5.4/esp32/get-started/}.

\end{document}
