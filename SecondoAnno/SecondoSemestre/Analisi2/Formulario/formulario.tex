\documentclass{article}
\usepackage{import}
\usepackage[utf8]{inputenc}
\usepackage[T1]{fontenc}
\usepackage{textcomp}
\usepackage[italian]{babel}
\usepackage{amsmath, amssymb}
\usepackage{booktabs,xltabular}
\usepackage{amsfonts}
\usepackage{subcaption}
\usepackage{amsthm}
\usepackage{cancel}
\usepackage{mdframed}
\usepackage{makecell}
\usepackage{float}
\usepackage{xcolor}
\usepackage{listings}
\usepackage{gensymb}
\usepackage{graphicx}
\usepackage{bodeplot}
\usepackage{physics}
\usepackage{tikz}
\usetikzlibrary{shapes, arrows, automata, petri, decorations.markings, decorations.pathreplacing, positioning, calc, quotes}
\usepackage{circuitikz}
\usepackage[label=corner]{karnaugh-map}
\graphicspath{{./figures/}}

% Set default font to sans-serif
\renewcommand{\familydefault}{\sfdefault} 
\usepackage{eulervm}

\usepackage{forest}

\usepackage{mathtools}
\DeclarePairedDelimiter\ceil{\lceil}{\rceil}
\DeclarePairedDelimiter\floor{\lfloor}{\rfloor}

% \usepackage{ntheorem}

\usepackage{import}
\usepackage{pdfpages}
\usepackage{transparent}
\usepackage{xcolor}

\usepackage{hyperref}
\hypersetup{
    colorlinks=false,
}

% Code blocks
\definecolor{codegreen}{rgb}{0,0.6,0}
\definecolor{codegray}{rgb}{0.5,0.5,0.5}
\definecolor{codepurple}{rgb}{0.58,0,0.82}
\definecolor{backcolour}{rgb}{0.95,0.95,0.95}

\lstdefinestyle{mystyle}{
	backgroundcolor=\color{backcolour},
	commentstyle=\color{codegreen},
	keywordstyle=\color{magenta},
	numberstyle=\tiny\color{codegray},
	stringstyle=\color{codepurple},
	basicstyle=\ttfamily\footnotesize,
	breakatwhitespace=false,
	breaklines=true,
	captionpos=b,
	keepspaces=true,
	numbers=left,
	numbersep=5pt,
	showspaces=false,
	showstringspaces=false,
	showtabs=false,
	tabsize=2
}

\lstset{style=mystyle}

\usepackage{color}
\usepackage{import}
\usepackage{pdfpages}
\usepackage{transparent}
\usepackage{xcolor}

% Example frame
\theoremstyle{definition}
\newmdtheoremenv[%
	linecolor=gray,leftmargin=0,%
	rightmargin=0,
	innertopmargin=8pt,%
	innerbottommargin=8pt,
	ntheorem]{example}{Esempio}[section]

% Important definition frame
\theoremstyle{definition}
\newmdtheoremenv[%
	linecolor=gray,leftmargin=0,%
	rightmargin=0,
	backgroundcolor=gray!40,%
	innertopmargin=8pt,%
	innerbottommargin=8pt,
	ntheorem]{definition}{Definizione}[section]

% Exercise frame
\theoremstyle{definition}
\newmdtheoremenv[%
	linecolor=gray,leftmargin=0,%
	rightmargin=0,
	innertopmargin=8pt,%
	innerbottommargin=8pt,
	ntheorem]{exercise}{Esercizio}[section]

% Theorem frame
\theoremstyle{definition}
\newmdtheoremenv[%
  linecolor=gray,leftmargin=0,%
  rightmargin=0,
  innertopmargin=8pt,%
  innerbottommargin=8pt,
  ntheorem]{theorem}{Teorema}[section]

\theoremstyle{definition}
\newmdtheoremenv[%
  linecolor=white,leftmargin=0,%
  rightmargin=0,
  innertopmargin=8pt,%
  innerbottommargin=8pt,
  ntheorem]{define}{Definizione utile}[section]

% figure support
\usepackage{import}
\usepackage{xifthen}
\pdfminorversion=7
\usepackage{pdfpages}
\usepackage{transparent}
\newcommand{\incfig}[1]{%
	\def\svgwidth{\columnwidth}
	\import{./figures/}{#1.pdf_tex}
}

% FSM tikz
\tikzset{
    place/.style={
        circle,
        thick,
        draw=black,
        minimum size=6mm,
    },
        state/.style={
        circle,
        thick,
        draw=black,
        fill=white,
        minimum size=6mm,
    },
}

\pdfsuppresswarningpagegroup=1

\usepackage{pgfplots}
\pgfplotsset{compat=1.18,width=10cm}

% Save plots as pdf and reuse them without compiling every time
\usetikzlibrary{external}
\tikzexternalize[prefix=figures/tikz/, optimize=false]


\begin{document}

\tableofcontents
\pagebreak

\section{Equazioni differenziali}

\subsection{Risoluzione delle equazioni differenziali  del primo ordine (Problema di Cauchy)}
\begin{enumerate}
  \item Spostare tutte le \( y \) da una parte dell'equazione:
    \begin{example}
      Ad esempio:
      \[
        \begin{cases}
          y' = \frac{y ^2}{y ^2 + 4}t\\
          y(0) = 2
        \end{cases}
      \]
      si può riscrivere come:
      \[
        \frac{y ^2 + 4}{y ^2}y' = t\\
      \]
    \end{example}

  \item Integrare da entrambe le parti per rimuovere la derivata:
    \begin{example}
      Ad esempio:
      \[
        \int \frac{y ^2 + 4}{y ^2}y' \, dy = \int t \, dt
      \]
      Sostituiamo:
      \[
        \begin{aligned}
          u &= y\\
          du &= y' \, dy
        \end{aligned}
      \] 
      quindi sostituendo \( y' \, dy \) con \( du \) si ha:
      \[
        \int \frac{u ^2 + 4}{u ^2} \, du = \int t \, dt
      \]
      risostituisco \( y \) al posto di \( u \) per non avere variabili diverse:
      \[
        \int \frac{y ^2 + 4}{y ^2} \, dy = \int t \, dt
      \]
    \end{example}

  \item Risolvere l'integrale
    \begin{example}
      Riprendendo l'esempio precedente:
      \[
        \begin{aligned}
          \int \frac{y ^2 + 4}{y ^2} \, dy &= \int t \, dt \\
          \int 1 \, dy + 4 \int y^{-2} \, dy &= \frac{t^2}{2} + c\\
          y - \frac{4}{y} &= \frac{t^2}{2} + c\\
          \frac{y^2 - 4}{y} &= \frac{t^2}{2} + c\\
        \end{aligned}
      \] 
    \end{example}

  \item Imporre le condizioni iniziali per trovare la costante \( c \):
    \begin{example}
      Riprendendo l'esempio precedente:
      \[
        y(0) = 2
      \] 
      \[
        \downarrow
      \] 
      \[
        \begin{aligned}
          \frac{2^2 - 4}{2} &= \frac{0}{2} + c\\
          0 &= 0 + c\\
          c &= 0
        \end{aligned}
      \] 
    \end{example}

  \item Sostituire la costante trovata nella soluzione dell'integrale:
    \begin{example}
      Riprendendo l'esempio precedente:
      \[
        \begin{aligned}
          \frac{y^2 - 4}{y} &= \frac{t^2}{2} + 0\\
          y^2 - 4 &= \frac{t^2}{2}y\\
          y^2 - \frac{t^2}{2}y - 4 &= 0
        \end{aligned}
      \] 
    \end{example}

  \item Risolvere l'equazione trovata nei punti definiti nelle condizioni iniziali
    per trovare la soluzione dell'equazione differenziale:
    \begin{example}
      Riprendendo l'esempio precedente:
      \[
        y^2 - \frac{t^2}{2}y - 4 = 0
      \] 
      \[
        \begin{aligned}
          y(t) &= \frac{\frac{t^2}{2} \pm \sqrt{\left(\frac{t^2}{2}\right)^2 + 16}}{2}\\
               &= \frac{t^2 \pm \sqrt{t^4+64}}{4}\\
        \end{aligned}
      \] 

      \[
        y(0) = 2
      \] 
      \[
        \downarrow
      \] 
      \[
        \begin{aligned}
          y(0) &= \frac{0 \pm \sqrt{0+64}}{4}\\
          2 &= \frac{\pm 8}{4}\\
          2 &= \pm 2
        \end{aligned}
      \] 
      Solo la soluzione \( y = 2 \) è accettabile, e quindi la soluzione dell'equazione differenziale
      è:
      \[
        y(t) = \frac{t^2 + \sqrt{t^4+64}}{4}
      \] 
    \end{example}
\end{enumerate}

\subsection{Risoluzione delle equazioni differenziali lineari del secondo ordine a
coefficienti costanti (Problema di Cauchy)}
\begin{enumerate}
  \item Risolvere l'equazione omogenea associata all'equazione differenziale.

    Se le soluzioni sono complesse coniugate si possono riscrivere come:
    \[
      \begin{aligned}
        r_1 &= \alpha + i \beta = e^{\alpha x} \cos(\beta x) \\
        r_2 &= \alpha - i \beta = e^{\alpha x} \sin(\beta x)
      \end{aligned}
    \] 
    con \( \alpha, \beta \in \mathbb{R} \) e \( z(x) \) è:
    \[
      z(x) = e^{\alpha x} \left( c_1 \cos(\beta x) + c_2 \sin(\beta x) \right)
    \] 

    \begin{example}
      Ad esempio:
      \[
        y''-6y'+9y = 3t+2 \quad \to \quad r^2 - 6r + 9 = 0
      \] 
      Per la regola del trinomio speciale si ha:
      \[
        \left( r-3 \right)^2
      \] 
      e le soluzioni sono:
      \[
        \begin{aligned}
          r_1 &= 3 \\
          r_2 &= 3
        \end{aligned}
      \] 
    \end{example}

  \item Trovare la soluzione generale (equivale alla risposta libera in Sistemi)
    con la seguente formula:
      \[
        z(x) = \sum_{i=1}^{r} \sum_{l=0}^{\mu_i -1} c_{i,l} \cdot 
        e^{\lambda_i x} \cdot \frac{x^l}{l!}
      \] 
      dove:
      \[
      \begin{aligned}
        r &\text{ è il numero di radici distinte} \\
        \mu_i &\text{ è la molteplicità della radice } \lambda_i \\
        c_{i,l} &\text{ sono costanti da determinare} \\
      \end{aligned}
      \] 
    \begin{example}
      Calcoliamo la soluzione generale dell'equazione differenziale:
      \[
        z(t) = c_1 e^{3t} + c_2 t e^{3t}
      \] 
    \end{example}

  \item Trovare una soluzione particolare (equivale alla risposta forzata in Sistemi)
    dell'equazione differenziale. La soluzione dipende dalla funzione $f(x)$ dove:
    \[
      y'' + a y' + b y = f(x)
    \] 
    \begin{enumerate}
      \item Se \( f(x) = p_r(x) \) è un polinomio di grado \( r \) allora la soluzione
        particolare è un polinomio di grado \( r \) con coefficienti indeterminati:
        \begin{table}[H]
          \centering
          \begin{tabular}{|c|c|}
            \hline
            \( \bar{y}(x) = q_r(x) \) & se \( b \neq 0 \) \\
            \hline
            \( \bar{y}(x) = xq_{r}(x) \) & se \( b = 0 \) e \( a \neq 0 \) \\
            \hline
            \( \bar{y}(x) = x^2q_{r}(x) \) & se \( b = 0 \) e \( a = 0 \) \\
            \hline
          \end{tabular}
        \end{table}
      \begin{example}
        Ad esempio:
        \begin{itemize}
          \item Se consideriamo \( y'' - 6y' + 9y = 3t + 2 \) 
            Abbiamo che
            \[
              f(x) = p_1(t) = 3t + 2
            \]
            con \( a = -6 \) e \( b = 9 \) 
            quindi siamo nel primo caso:
            \[
              \bar{y}(t) = q_0 + q_1 t
            \] 

          \item Se consideriamo \( y'' - 6y' = t^2+2 \) abbiamo che
            \[
              f(x) = p_2(t) = t^2 + 2
            \] 
            con \( a = -6 \) e \( b = 0 \) quindi siamo nel secondo caso:
            \[
              \bar{y}(t) = q_0 t + q_1 t^2 + q_2 t^3
            \] 

          \item Se consideriamo \( y'' = t-1 \) abbiamo che
            \[
              f(x) = p_1(t) = t-1
            \] 
            con \( a = 0 \) e \( b = 0 \) quindi siamo nel terzo caso:
            \[
              \bar{y}(t) = q_0 t^2 + q_1 t^3
            \]
        \end{itemize}
      \end{example}

    \item Se \( f(x) = Ae^{\lambda x} \) con \( \lambda \in \mathbb{C} \) allora
      la soluzione particolare è del tipo:
      \[
        \bar{y}(x) = e^{\lambda x} \gamma(x)
      \] 
      Per trovare \( \gamma (x) \) bisogna risolvere la seguente equazione:
      \[
        \gamma'' + \gamma'(2 \lambda + a) + \gamma(\lambda^2 + a \lambda + b) = A
      \] 
      quindi si distinguono i seguenti casi:
      \begin{table}[H]
        \centering
        \begin{tabular}{|c|c|c|}
          \hline
          se \( \lambda^2 + a \lambda + b \neq 0 \) & 
          \(\gamma (x) = \frac{A}{\lambda^2 + a \lambda + b} \) &
          \( \bar{y}(x) = \frac{A e^{\lambda x}}{\lambda^2 + a \lambda + b} \) \\
          \hline
          se \( \lambda^2 + a \lambda + b = 0 \) e \( 2 \lambda + a \neq 0 \) &
          \( \gamma (x) = \frac{A x}{2 \lambda + a} \) &
          \( \bar{y}(x) = \frac{A x e^{\lambda x}}{2 \lambda + a} \) \\
          \hline
          se \( \lambda^2 + a \lambda + b = 0 \) e \( 2 \lambda + a = 0 \) &
          \( \gamma (x) = \frac{A}{2}x^2 \) &
          \( \bar{y}(x) = \frac{A}{2} x^2 e^{\lambda x} \) \\
          \hline
        \end{tabular}
      \end{table}

      \noindent
      In questa classe particolare di termini noti del tipo \( A e^{\lambda x} \) con
      \( \lambda \in \mathbb{C} \) rientrano anche i casi:
      \[
        cos \omega x,\; sin \omega x,\; e^{\mu x} cos \omega x,\; e^{\mu x} sin \omega x
      \] 
      con \( \mu \in \mathbb{R} \) 

      \begin{example}
        Prendiamo ad esempio la seguente equazione differenziale:
        \[
          y'' -4y' +8y = e^{-2t}
        \] 
        Una soluzione particolare sarà del tipo:
        \[
          \bar{y}(t) = e^{-2t} \gamma(t)
        \] 
        con \( A = 1 \) e \( \lambda = -2 \) quindi:
        \[
          \begin{aligned}
            \gamma'' + \gamma'(2 \cdot -2 - 4) + \gamma (-2^2 - 4 \cdot -2 + 8) &= 1\\
            \gamma'' - 8 \gamma' + 20 \gamma = 1
          \end{aligned}
        \]
        Siamo nel primo caso quindi:
        \[
          \gamma(t) = \frac{A}{\lambda^2 + a \lambda + b} = \frac{1}{4 + 8 + 8} = \frac{1}{20}
        \]
        quindi la soluzione particolare è:
        \[
          \bar{y}(t) = \frac{e^{-2t}}{20}
        \] 
      \end{example}

    \item Se \( f(x) = A e^{\mu x} cos(\omega x) \),
      ricordando che:
      \[
        A e^{(\mu + i \omega)x} = A e^{\mu x} \left(\cos(\omega x) + i \sin(\omega x)\right)
      \] 
      si può risolvere l'equazione con termine noto \( f(x) = A e^{(\mu + i \omega)x} \)
      e poi prendere solo la parte reale della soluzione complessa. Analogamente se
      \( f(x) = A e^{\mu x} sin(\omega x) \) si procede andando a prendere la parte
      immaginaria.

      Alternativamente si può calcolare una soluzione particolare senza numeri complessi
      e tale soluzione è del tipo:
      \[
        \bar{y}(x) = e^{x} \left( c_1 \cos(3x) + c_2 \sin(3x) \right) 
      \] 
      In ogni caso se il termine noto contiene solo seno oppure coseno, quando si cerca 
      una soluzione particolare si deve sempre cercare come combinazione lineare di 
      entrambi seno e coseno.

    \item Se \( f(x) \) è una combinazione lineare dei termini precedenti (ad esempio
      un polinomio più un esponenziale) allora per linearità si trova la soluzione
      dell'equazione che ha come termine noto il primo termine, poi si trova la soluzione
      dell'equazione che ha come termine noto il secondo termine e si sommano le due
      soluzioni per trovare la soluzione particolare dell'equazione differenziale di partenza.
    \end{enumerate}

  \item Trovare la soluzione finale dell'equazione differenziale sommando la soluzione
    generale con la soluzione particolare (equivale alla risposta totale in Sistemi):
    \[
      y(x) = z(x) + \bar{y}(x)
    \] 

  \item Trovare le costanti \( c_1, c_2, \ldots \) imponendo le condizioni iniziali:
    \[
      \begin{cases}
        y(0) = y_0 \\
        y'(0) = y'_0 \\
        \vdots
      \end{cases}
    \] 

  \item Sostituire le costanti trovate nella soluzione:
    \[
      y(x) = z(x) + \bar{y}(x)
    \] 
    per trovare la soluzione finale dell'equazione differenziale.
\end{enumerate}
    
\end{document}
