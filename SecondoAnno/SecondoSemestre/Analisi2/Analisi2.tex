\documentclass[a4paper]{article}
\usepackage{import}
\usepackage[utf8]{inputenc}
\usepackage[T1]{fontenc}
\usepackage{textcomp}
\usepackage[italian]{babel}
\usepackage{amsmath, amssymb}
\usepackage{booktabs,xltabular}
\usepackage{amsfonts}
\usepackage{subcaption}
\usepackage{amsthm}
\usepackage{cancel}
\usepackage{mdframed}
\usepackage{makecell}
\usepackage{float}
\usepackage{xcolor}
\usepackage{listings}
\usepackage{gensymb}
\usepackage{graphicx}
\usepackage{bodeplot}
\usepackage{physics}
\usepackage{tikz}
\usetikzlibrary{shapes, arrows, automata, petri, decorations.markings, decorations.pathreplacing, positioning, calc, quotes}
\usepackage{circuitikz}
\usepackage[label=corner]{karnaugh-map}
\graphicspath{{./figures/}}

% Set default font to sans-serif
\renewcommand{\familydefault}{\sfdefault} 
\usepackage{eulervm}

\usepackage{forest}

\usepackage{mathtools}
\DeclarePairedDelimiter\ceil{\lceil}{\rceil}
\DeclarePairedDelimiter\floor{\lfloor}{\rfloor}

% \usepackage{ntheorem}

\usepackage{import}
\usepackage{pdfpages}
\usepackage{transparent}
\usepackage{xcolor}

\usepackage{hyperref}
\hypersetup{
    colorlinks=false,
}

% Code blocks
\definecolor{codegreen}{rgb}{0,0.6,0}
\definecolor{codegray}{rgb}{0.5,0.5,0.5}
\definecolor{codepurple}{rgb}{0.58,0,0.82}
\definecolor{backcolour}{rgb}{0.95,0.95,0.95}

\lstdefinestyle{mystyle}{
	backgroundcolor=\color{backcolour},
	commentstyle=\color{codegreen},
	keywordstyle=\color{magenta},
	numberstyle=\tiny\color{codegray},
	stringstyle=\color{codepurple},
	basicstyle=\ttfamily\footnotesize,
	breakatwhitespace=false,
	breaklines=true,
	captionpos=b,
	keepspaces=true,
	numbers=left,
	numbersep=5pt,
	showspaces=false,
	showstringspaces=false,
	showtabs=false,
	tabsize=2
}

\lstset{style=mystyle}

\usepackage{color}
\usepackage{import}
\usepackage{pdfpages}
\usepackage{transparent}
\usepackage{xcolor}

% Example frame
\theoremstyle{definition}
\newmdtheoremenv[%
	linecolor=gray,leftmargin=0,%
	rightmargin=0,
	innertopmargin=8pt,%
	innerbottommargin=8pt,
	ntheorem]{example}{Esempio}[section]

% Important definition frame
\theoremstyle{definition}
\newmdtheoremenv[%
	linecolor=gray,leftmargin=0,%
	rightmargin=0,
	backgroundcolor=gray!40,%
	innertopmargin=8pt,%
	innerbottommargin=8pt,
	ntheorem]{definition}{Definizione}[section]

% Exercise frame
\theoremstyle{definition}
\newmdtheoremenv[%
	linecolor=gray,leftmargin=0,%
	rightmargin=0,
	innertopmargin=8pt,%
	innerbottommargin=8pt,
	ntheorem]{exercise}{Esercizio}[section]

% Theorem frame
\theoremstyle{definition}
\newmdtheoremenv[%
  linecolor=gray,leftmargin=0,%
  rightmargin=0,
  innertopmargin=8pt,%
  innerbottommargin=8pt,
  ntheorem]{theorem}{Teorema}[section]

\theoremstyle{definition}
\newmdtheoremenv[%
  linecolor=white,leftmargin=0,%
  rightmargin=0,
  innertopmargin=8pt,%
  innerbottommargin=8pt,
  ntheorem]{define}{Definizione utile}[section]

% figure support
\usepackage{import}
\usepackage{xifthen}
\pdfminorversion=7
\usepackage{pdfpages}
\usepackage{transparent}
\newcommand{\incfig}[1]{%
	\def\svgwidth{\columnwidth}
	\import{./figures/}{#1.pdf_tex}
}

% FSM tikz
\tikzset{
    place/.style={
        circle,
        thick,
        draw=black,
        minimum size=6mm,
    },
        state/.style={
        circle,
        thick,
        draw=black,
        fill=white,
        minimum size=6mm,
    },
}

\pdfsuppresswarningpagegroup=1

\usepackage{pgfplots}
\pgfplotsset{compat=1.18,width=10cm}

% Save plots as pdf and reuse them without compiling every time
\usetikzlibrary{external}
\tikzexternalize[prefix=figures/tikz/, optimize=false]


\begin{document}
\begin{titlepage}
	\begin{center}
		\vspace*{1cm}

		\Huge
		\textbf{Probabilità e Statistica\\Esercizi}

		\vspace{0.5cm}
		\LARGE
		UniVR - Dipartimento di Informatica

		\vspace{1.5cm}

		\textbf{Fabio Irimie}

		\vfill


		\vspace{0.8cm}


		2° Semestre 2023/2024

	\end{center}
\end{titlepage}


\tableofcontents
\pagebreak

% Info:
% Esame: Ci sarà un quiz di sbarramento che bisogna passare per poter dare l'esame.
%        All'esame ci sono 8 esercizi da fare in 3 ore. Ci sarà un parziale.
\section{Equazioni differenziali}
L'equazione differenziale ordinaria (EDO) è un'equazione del tipo:
\[
  F(t, y, y', y^n, \ldots, y^{(n)}) = 0
\] 
dove \( y(t) \) è la funzione incognita e \( F \) è un operatore differenziale che lega le
variabili \( t, y, y', y^n, \ldots, y^{(n)} \) a valori reali.

Si dice \textbf{soluzione} di un'EDO nell'intervallo \( I \subset \mathbb{R} \) una
\textbf{funzione} \( \varphi  \), definita almeno in \( I \) 
e a valori reali per cui risulti:
\[
  F(t, \varphi(t), \varphi'(t), \ldots, \varphi^{(n)}(t)) = 0 \quad \forall t \in I
\] 

\begin{example}
  Se prendiamo in considerazione una popolazione di conigli con tasso di natalità
  \( \lambda \), abbiamo che l'equazione che descrive la crescita della popolazione
  è:
  \[
    \dot{N}(t) = \lambda N(t)
  \] 

  \vspace{1em}
  \noindent
  Se \( N(t) = 0 \) si ha una \textbf{soluzione stazionaria} dell'equazione differenziale
  perchè quando si raggiunge 0 la popolazione non cresce più. Risolviamo:
  \[
    \frac{\dot{N(t)}}{N(t)} = \lambda
  \] 
  Facciamo un cambio di variabile \( N(t) = n \) \( \dot{N(t)}dt = dn \) 
  \[
  \int \frac{\dot{N(t)}}{N(t)} \, dt = \int \frac{1}{n} \, dn = \log(N(t)) = \lambda t + c
  \] 
  \[
    N(t) = c e^{\lambda t} \quad c > 0
  \] 
  Consideriamo che al tempo 0 la popolazione sia di 100 conigli: \( N(0) = 100 \),
  abbiamo che:
  \[
  N(0) = c = 100
  \] 
  e la soluzione del problema è:
  \[
    N(t) = 100 e^{\lambda t}
  \] 
\end{example}

\subsection{Equazioni differenziali di primo ordine}
Un'equazione differenziale di primo ordine è un'equazione del tipo:
\[
  F(t,y,y') = 0
\] 


\end{document}
