\documentclass[a4paper]{article}
\usepackage{import}
\usepackage[utf8]{inputenc}
\usepackage[T1]{fontenc}
\usepackage{textcomp}
\usepackage[italian]{babel}
\usepackage{amsmath, amssymb}
\usepackage{booktabs,xltabular}
\usepackage{amsfonts}
\usepackage{subcaption}
\usepackage{amsthm}
\usepackage{cancel}
\usepackage{mdframed}
\usepackage{makecell}
\usepackage{float}
\usepackage{xcolor}
\usepackage{listings}
\usepackage{gensymb}
\usepackage{graphicx}
\usepackage{bodeplot}
\usepackage{physics}
\usepackage{tikz}
\usetikzlibrary{shapes, arrows, automata, petri, decorations.markings, decorations.pathreplacing, positioning, calc, quotes}
\usepackage{circuitikz}
\usepackage[label=corner]{karnaugh-map}
\graphicspath{{./figures/}}

% Set default font to sans-serif
\renewcommand{\familydefault}{\sfdefault} 
\usepackage{eulervm}

\usepackage{forest}

\usepackage{mathtools}
\DeclarePairedDelimiter\ceil{\lceil}{\rceil}
\DeclarePairedDelimiter\floor{\lfloor}{\rfloor}

% \usepackage{ntheorem}

\usepackage{import}
\usepackage{pdfpages}
\usepackage{transparent}
\usepackage{xcolor}

\usepackage{hyperref}
\hypersetup{
    colorlinks=false,
}

% Code blocks
\definecolor{codegreen}{rgb}{0,0.6,0}
\definecolor{codegray}{rgb}{0.5,0.5,0.5}
\definecolor{codepurple}{rgb}{0.58,0,0.82}
\definecolor{backcolour}{rgb}{0.95,0.95,0.95}

\lstdefinestyle{mystyle}{
	backgroundcolor=\color{backcolour},
	commentstyle=\color{codegreen},
	keywordstyle=\color{magenta},
	numberstyle=\tiny\color{codegray},
	stringstyle=\color{codepurple},
	basicstyle=\ttfamily\footnotesize,
	breakatwhitespace=false,
	breaklines=true,
	captionpos=b,
	keepspaces=true,
	numbers=left,
	numbersep=5pt,
	showspaces=false,
	showstringspaces=false,
	showtabs=false,
	tabsize=2
}

\lstset{style=mystyle}

\usepackage{color}
\usepackage{import}
\usepackage{pdfpages}
\usepackage{transparent}
\usepackage{xcolor}

% Example frame
\theoremstyle{definition}
\newmdtheoremenv[%
	linecolor=gray,leftmargin=0,%
	rightmargin=0,
	innertopmargin=8pt,%
	innerbottommargin=8pt,
	ntheorem]{example}{Esempio}[section]

% Important definition frame
\theoremstyle{definition}
\newmdtheoremenv[%
	linecolor=gray,leftmargin=0,%
	rightmargin=0,
	backgroundcolor=gray!40,%
	innertopmargin=8pt,%
	innerbottommargin=8pt,
	ntheorem]{definition}{Definizione}[section]

% Exercise frame
\theoremstyle{definition}
\newmdtheoremenv[%
	linecolor=gray,leftmargin=0,%
	rightmargin=0,
	innertopmargin=8pt,%
	innerbottommargin=8pt,
	ntheorem]{exercise}{Esercizio}[section]

% Theorem frame
\theoremstyle{definition}
\newmdtheoremenv[%
  linecolor=gray,leftmargin=0,%
  rightmargin=0,
  innertopmargin=8pt,%
  innerbottommargin=8pt,
  ntheorem]{theorem}{Teorema}[section]

\theoremstyle{definition}
\newmdtheoremenv[%
  linecolor=white,leftmargin=0,%
  rightmargin=0,
  innertopmargin=8pt,%
  innerbottommargin=8pt,
  ntheorem]{define}{Definizione utile}[section]

% figure support
\usepackage{import}
\usepackage{xifthen}
\pdfminorversion=7
\usepackage{pdfpages}
\usepackage{transparent}
\newcommand{\incfig}[1]{%
	\def\svgwidth{\columnwidth}
	\import{./figures/}{#1.pdf_tex}
}

% FSM tikz
\tikzset{
    place/.style={
        circle,
        thick,
        draw=black,
        minimum size=6mm,
    },
        state/.style={
        circle,
        thick,
        draw=black,
        fill=white,
        minimum size=6mm,
    },
}

\pdfsuppresswarningpagegroup=1

\usepackage{pgfplots}
\pgfplotsset{compat=1.18,width=10cm}

% Save plots as pdf and reuse them without compiling every time
\usetikzlibrary{external}
\tikzexternalize[prefix=figures/tikz/, optimize=false]


\begin{document}

\begin{titlepage}
	\begin{center}
		\vspace*{1cm}

		\Huge
		\textbf{Probabilità e Statistica\\Esercizi}

		\vspace{0.5cm}
		\LARGE
		UniVR - Dipartimento di Informatica

		\vspace{1.5cm}

		\textbf{Fabio Irimie}

		\vfill


		\vspace{0.8cm}


		2° Semestre 2023/2024

	\end{center}
\end{titlepage}


\tableofcontents
\pagebreak

\section{Grafi}
I grafi permettono di risolvere problemi particolarmente complessi, ma la parte difficile
è la conversione di un problema in un grafo. I grafi sono costituiti da nodi e archi:
\begin{figure}[H]
  \centering
  \begin{tikzpicture}
    \node[draw,circle] (1) at (0,0) {1};
    \node[draw,circle] (3) [below left=of 1] {3};
    \node[draw,circle] (4) [below right=of 1] {4};
    \node[draw,circle] (2) [right=of 1] {2};

    \draw (1) -- (3);
    \draw (1) -- (4);
    \draw (1) -- (2);
    \draw (4) -- (2);
  \end{tikzpicture}
  \caption{Esempio di grafo}
  \label{fig:grafo1}
\end{figure}
\begin{itemize}
  \item \textbf{Nodi}: rappresentano gli elementi del problema.
  \item \textbf{Archi}: rappresentano le relazioni tra i nodi.
\end{itemize}
I grafi in cui gli archi hanno un valore (o peso) vengono chiamati \textbf{grafi pesati}.
Si possono anche aggiungere delle direzioni agli archi, ottenendo così un 
\textbf{grafo orientato}, in cui un arco si può attraversare in un solo verso.
\begin{definition}[Cammino]
  Un \textbf{cammino} è una sequenza di nodi per cui esiste un arco tra ogni coppia di nodi adiacenti.
\end{definition}
In un cammino, la ripetizione di un nodo rappresenta un \textbf{loop} e questo cammino
viene detto \textbf{cammino ciclico}. (un cammino senza cicli si dice \textbf{cammino
semplice})

Il \textbf{grado} di un nodo è il numero di archi che incidono sul nodo. Ha senso parlare
di grado di un nodo solo quando il grafo non è orientato perchè così ogni arco viene
contato una sola volta.
\begin{itemize}
  \item \textbf{Grado entrante}: numero di archi entranti in un nodo.
  \item \textbf{Grado uscente}: numero di archi uscenti da un nodo.
\end{itemize}

\vspace{1em}
\noindent
La definizione formale di un grafo è la seguente:
\begin{definition}
  Un grafo è definito come una coppia $G = (V, E)$ dove:
  \begin{itemize}
    \item $V$ è un insieme di nodi.
    \item $E$ è un insieme di archi:
      \[
        E \subseteq V \times V
      \] 
  \end{itemize}
\end{definition}
Dallla figura \ref{fig:grafo1} si ha che:
\begin{itemize}
  \item $V = \{1, 2, 3, 4\}$.
  \item $E = \left\{(1, 3), (3,1), (1,1), (1,4), (4,1), (1,2), (2,4), (4,2)\right\}$.
\end{itemize}
La definizione formale dei concetti precedenti è:
\begin{definition}
  Il \textbf{grado uscente} di un nodo $v$ in un grafo orientato $G = (V, E)$ è il numero
  di archi uscenti da $v$ (\( |\ldots| \) è la cardinalità di un insieme):
  \[
    \text{grado\_uscente}(v) = \left|\{u \;|\; (v, u) \in E\}\right|
  \]
\end{definition}
\begin{definition}
  Il \textbf{grado entrante} di un nodo $v$ in un grafo orientato $G = (V, E)$ è il numero
  di archi entranti in $v$:
  \[
    \text{grado\_entrante}(v) = \left|\{u \;|\; (u, v) \in E\}\right|
  \]
\end{definition}
\begin{definition}
  Un cammino è una sequenza di nodi in cui per ogni coppia di nodi consecutivi esiste un arco:
  \[
    \forall i \in \{0 \ldots n-1\} \quad (v_i, v_{i+1}) \in E
  \] 
\end{definition}

\subsection{Rappresentazione di un grafo}
Per rappresentare un grafo ci sono due modi:
\begin{itemize}
  \item \textbf{Rappresentazione per liste di adiacenza}:
    Si crea una lista in cui si rappresentano i nodi e
    ad ogni nodo si associa la lista di tutti i nodi raggiungibili tramite un arco.
    Prendiamo in considerazione la figura \ref{fig:grafo1}:
    \begin{figure}[H]
      \centering
      \begin{tikzpicture}[
        node distance=0cm and 0.5cm
        ]
        \node[draw,rectangle,minimum width=1cm,minimum height=0.5cm] (1) at (0,0) {1};
        \node[draw,rectangle,minimum width=1cm,minimum height=0.5cm] (2) [below=of 1] {2};
        \node[draw,rectangle,minimum width=1cm,minimum height=0.5cm] (3) [below=of 2] {3};
        \node[draw,rectangle,minimum width=1cm,minimum height=0.5cm] (4) [below=of 3] {4};

        \node[draw,rectangle,minimum size=0.5cm] (1-1-1) [right=of 1] {3};
        \node[draw,rectangle,minimum size=0.5cm] (1-1-2) [right=0cm of 1-1-1] {p};
        \node[draw,rectangle,minimum size=0.5cm] (1-1-3) [right=0cm of 1-1-2] {};
        \draw[->] (1) -- (1-1-1);
        \node[draw,rectangle,minimum size=0.5cm] (1-2-1) [right=of 1-1-3] {1};
        \node[draw,rectangle,minimum size=0.5cm] (1-2-2) [right=0cm of 1-2-1] {p};
        \node[draw,rectangle,minimum size=0.5cm] (1-2-3) [right=0cm of 1-2-2] {};
        \draw[->] (1-1-3.center) -- (1-2-1);
        \node[draw,rectangle,minimum size=0.5cm] (1-3-1) [right=of 1-2-3] {4};
        \node[draw,rectangle,minimum size=0.5cm] (1-3-2) [right=0cm of 1-3-1] {p};
        \node[draw,rectangle,minimum size=0.5cm] (1-3-3) [right=0cm of 1-3-2] {};
        \draw[->] (1-2-3.center) -- (1-3-1);
        \node[draw,rectangle,minimum size=0.5cm] (1-4-1) [right=of 1-3-3] {2};
        \node[draw,rectangle,minimum size=0.5cm] (1-4-2) [right=0cm of 1-4-1] {p};
        \node[draw,rectangle,minimum size=0.5cm] (1-4-3) [right=0cm of 1-4-2] {\footnotesize/};
        \draw[->] (1-3-3.center) -- (1-4-1);


        \node[draw,rectangle,minimum size=0.5cm] (2-1-1) [right=of 2] {4};
        \node[draw,rectangle,minimum size=0.5cm] (2-1-2) [right=0cm of 2-1-1] {p};
        \node[draw,rectangle,minimum size=0.5cm] (2-1-3) [right=0cm of 2-1-2] {\footnotesize/};
        \draw[->] (2) -- (2-1-1);


        \node[draw,rectangle,minimum size=0.5cm] (3-1-1) [right=of 3] {1};
        \node[draw,rectangle,minimum size=0.5cm] (3-1-2) [right=0cm of 3-1-1] {p};
        \node[draw,rectangle,minimum size=0.5cm] (3-1-3) [right=0cm of 3-1-2] {\footnotesize/};
        \draw[->] (3) -- (3-1-1);


        \node[draw,rectangle,minimum size=0.5cm] (4-1-1) [right=of 4] {1};
        \node[draw,rectangle,minimum size=0.5cm] (4-1-2) [right=0cm of 4-1-1] {p};
        \node[draw,rectangle,minimum size=0.5cm] (4-1-3) [right=0cm of 4-1-2] {};
        \draw[->] (4) -- (4-1-1);
        \node[draw,rectangle,minimum size=0.5cm] (4-2-1) [right=of 4-1-3] {2};
        \node[draw,rectangle,minimum size=0.5cm] (4-2-2) [right=0cm of 4-2-1] {p};
        \node[draw,rectangle,minimum size=0.5cm] (4-2-3) [right=0cm of 4-2-2] {\footnotesize/};
        \draw[->] (4-1-3.center) -- (4-2-1);
      \end{tikzpicture}
      \caption{Rappresentazione per liste di adiacenza}
    \end{figure}
    \noindent
    Lo spazio in memoria occupato è \(\Theta(|V| + |E|)\).

  \item \textbf{Rappresentazione per matrice di adiacenza}:
    Si crea una matrice \(A\) di dimensione \(|V| \times |V|\) in cui \(A_{ij} = 1\) se esiste
    un arco tra i nodi \(i\) e \(j\), altrimenti \(A_{ij} = 0\).
    Prendiamo in considerazione la figura \ref{fig:grafo1}, dove \( p \) è il peso dell'arco:
    \begin{table}[H]
      \centering
      \begin{tabular}{c|cccc}
        / & 1 & 2 & 3 & 4 \\
        \hline
        1 & 1 & 1 & 1 & 1 \\
        2 & 0 & 0 & 0 & 1 \\
        3 & 1 & 0 & 0 & 0 \\
        4 & 1 & 1 & 0 & 0 \\
      \end{tabular}
      \caption{Rappresentazione per matrice di adiacenza}
    \end{table}
    \noindent
    Lo spazio in memoria occupato è \(\Theta(|V|^2)\).
\end{itemize}

\noindent
\begin{itemize}
  \item Un \textbf{grafo trasposto} è un grafo in cui tutti gli archi sono invertiti.

  \item
    La \textbf{chiusura transitiva di un grafo} è un grafo in cui se esiste un cammino tra due nodi
    allora esiste un arco diretto tra i due nodi:
    \begin{figure}[H]
      \centering
      \begin{tikzpicture}
        \node[draw,circle] (a) at (0,0) {a};
        \node[draw,circle] (b) [right=of a] {b};
        \node[draw,circle] (c) [right=of b] {c};

        \draw[->] (a) -- (b);
        \draw[->] (b) -- (c);
        \draw[->] (a) to [bend left=45] (c);
      \end{tikzpicture}
      \caption{Grafo con chiusura transitiva}
    \end{figure}

  \item Il \textbf{diametro} è il percorso più lungo fra i percorsi minimi
\end{itemize}

\subsection{Esplorazione di un grafo}
\subsubsection{Visita in ampiezza (BFS: Breath First Search)}
La visita in ampiezza (o a ventaglio) è un algoritmo che permette di visitare tutti i nodi
di un grafo partendo da un nodo iniziale. L'algoritmo è il seguente:
\begin{lstlisting}[language=Scala]
// G e' un grafo composto da un insieme di nodi V e un insieme di archi E
// s e' un nodo dell'arco
bfs(G, s)
  for u in G.V
    u.color <- white // non esplorato
    u.distance <- +inf // distanza dal nodo s
    u.parent <- NIL // nodo da cui si arriva a u

  s.color <- gray // scoperto, ma non esplorato
  s.distance <- 0
  s.parent <- NIL
  Q <- {s} // coda FIFO che contiene i nodi scoperti non esplorati

  while Q != empty
    u <- q.head

    for v in G.adj(u) // lista di nodi adiacenti a u
      if v.color == white
        v.color <- gray
        v.distance <- u.distance + 1
        v.parent <- u
        Q.enqueue(v)

  Q.dequeue()
  u.color <- black // esplorato
\end{lstlisting}
La complessità di questo algoritmo è \(O(|V| + |E|)\).
\begin{example}
  L'algoritmo passo per passo è il seguente, dove i colori rappresentano:
  \begin{itemize}
    \item Nero: non esplorato,
    \item \textcolor{blue}{Blu}: scoperto, ma non esplorato,
    \item \textcolor{red}{Rosso}: esplorato,
  \end{itemize}
  \begin{enumerate}
    \item Primo passo:
      \begin{table}[H]
        \centering
        \begin{tabular}{c|c}
          Distanza & 0 \\
          Coda & s 
        \end{tabular}
      \end{table}
      \begin{figure}[H]
        \centering
        \begin{tikzpicture}[
          every node/.style={minimum size=0.7cm},
          ]
          \node[draw,circle] (r) {} node[above] at (r.north) {r};
          \node[draw,circle,blue] (s) [right=of r] {0} node[above] at (s.north) {s};
          \node[draw,circle] (t) [right=of s] {} node[above] at (t.north) {t};
          \node[draw,circle] (u) [right=of t] {} node[above] at (u.north) {u};

          \node[draw,circle] (v) [below=of r] {} node[below] at (v.south) {v};
          \node[draw,circle] (w) [below=of s] {} node[below] at (w.south) {w};
          \node[draw,circle] (x) [below=of t] {} node[below] at (x.south) {x};
          \node[draw,circle] (y) [below=of u] {} node[below] at (y.south) {y};

          \draw (r) -- (s);
          \draw (r) -- (v);
          \draw (s) -- (w);
          \draw (t) -- (w);
          \draw (t) -- (x);
          \draw (t) -- (u);
          \draw (w) -- (x);
          \draw (u) -- (y);
          \draw (x) -- (y);
        \end{tikzpicture}
      \end{figure}

    \item Secondo passo:
      \begin{table}[H]
        \centering
        \begin{tabular}{c|ccc}
          Distanza & 0 & 1 & 1\\
          Coda & s & w & r
        \end{tabular}
      \end{table}
      \begin{figure}[H]
        \centering
        \begin{tikzpicture}[
          every node/.style={minimum size=0.7cm},
          ]
          \node[draw,circle,blue] (r) {1} node[above] at (r.north) {r};
          \node[draw,circle,red] (s) [right=of r] {0} node[above] at (s.north) {s};
          \node[draw,circle] (t) [right=of s] {} node[above] at (t.north) {t};
          \node[draw,circle] (u) [right=of t] {} node[above] at (u.north) {u};

          \node[draw,circle] (v) [below=of r] {} node[below] at (v.south) {v};
          \node[draw,circle,blue] (w) [below=of s] {1} node[below] at (w.south) {w};
          \node[draw,circle] (x) [below=of t] {} node[below] at (x.south) {x};
          \node[draw,circle] (y) [below=of u] {} node[below] at (y.south) {y};

          \draw[->,thick] (r) -- (s);
          \draw (r) -- (v);
          \draw[<-,thick] (s) -- (w);
          \draw (t) -- (w);
          \draw (t) -- (x);
          \draw (t) -- (u);
          \draw (w) -- (x);
          \draw (u) -- (y);
          \draw (x) -- (y);
        \end{tikzpicture}
      \end{figure}

    \item Terzo passo:
      \begin{table}[H]
        \centering
        \begin{tabular}{c|cccccc}
          Distanza & 0 & 1 & 1 & 2 & 2 & 2\\
          Coda & $\cancel{\text{s}}$ & $\cancel{\text{w}}$ & $\cancel{\text{r}}$
               & t & x & v
        \end{tabular}
      \end{table}
      \begin{figure}[H]
        \centering
        \begin{tikzpicture}[
          every node/.style={minimum size=0.7cm},
          ]
          \node[draw,circle,red] (r) {1} node[above] at (r.north) {r};
          \node[draw,circle,red] (s) [right=of r] {0} node[above] at (s.north) {s};
          \node[draw,circle,blue] (t) [right=of s] {2} node[above] at (t.north) {t};
          \node[draw,circle] (u) [right=of t] {} node[above] at (u.north) {u};

          \node[draw,circle,blue] (v) [below=of r] {2} node[below] at (v.south) {v};
          \node[draw,circle,red] (w) [below=of s] {1} node[below] at (w.south) {w};
          \node[draw,circle,blue] (x) [below=of t] {2} node[below] at (x.south) {x};
          \node[draw,circle] (y) [below=of u] {} node[below] at (y.south) {y};

          \draw[->,thick] (r) -- (s);
          \draw[<-,thick] (r) -- (v);
          \draw[<-,thick] (s) -- (w);
          \draw[->,thick] (t) -- (w);
          \draw (t) -- (x);
          \draw (t) -- (u);
          \draw[<-,thick] (w) -- (x);
          \draw (u) -- (y);
          \draw (x) -- (y);
        \end{tikzpicture}
      \end{figure}

    \item Quarto passo:
      \begin{table}[H]
        \centering
        \begin{tabular}{c|cccccccc}
          Distanza & 0 & 1 & 1 & 2 & 2 & 2 & 3 & 3\\
          Coda & $\cancel{\text{s}}$ & $\cancel{\text{w}}$ & $\cancel{\text{r}}$
               & $\cancel{\text{t}}$ & $\cancel{\text{x}}$ & $\cancel{\text{v}}$
               & u & y
        \end{tabular}
      \end{table}
      \begin{figure}[H]
        \centering
        \begin{tikzpicture}[
          every node/.style={minimum size=0.7cm},
          ]
          \node[draw,circle,red] (r) {1} node[above] at (r.north) {r};
          \node[draw,circle,red] (s) [right=of r] {0} node[above] at (s.north) {s};
          \node[draw,circle,red] (t) [right=of s] {2} node[above] at (t.north) {t};
          \node[draw,circle,blue] (u) [right=of t] {3} node[above] at (u.north) {u};

          \node[draw,circle,red] (v) [below=of r] {2} node[below] at (v.south) {v};
          \node[draw,circle,red] (w) [below=of s] {1} node[below] at (w.south) {w};
          \node[draw,circle,red] (x) [below=of t] {2} node[below] at (x.south) {x};
          \node[draw,circle,blue] (y) [below=of u] {3} node[below] at (y.south) {y};

          \draw[->,thick] (r) -- (s);
          \draw[<-,thick] (r) -- (v);
          \draw[<-,thick] (s) -- (w);
          \draw[->,thick] (t) -- (w);
          \draw (t) -- (x);
          \draw[<-,thick] (t) -- (u);
          \draw[<-,thick] (w) -- (x);
          \draw (u) -- (y);
          \draw[<-,thick] (x) -- (y);
        \end{tikzpicture}
      \end{figure}

    \item Quinto passo:
      \begin{table}[H]
        \centering
        \begin{tabular}{c|cccccccc}
          Distanza & 0 & 1 & 1 & 2 & 2 & 2 & 3 & 3\\
          Coda & $\cancel{\text{s}}$ & $\cancel{\text{w}}$ & $\cancel{\text{r}}$
               & $\cancel{\text{t}}$ & $\cancel{\text{x}}$ & $\cancel{\text{v}}$
               & $\cancel{\text{u}}$ & $\cancel{\text{y}}$
        \end{tabular}
      \end{table}
      \begin{figure}[H]
        \centering
        \begin{tikzpicture}[
          every node/.style={minimum size=0.7cm},
          ]
          \node[draw,circle,red] (r) {1} node[above] at (r.north) {r};
          \node[draw,circle,red] (s) [right=of r] {0} node[above] at (s.north) {s};
          \node[draw,circle,red] (t) [right=of s] {2} node[above] at (t.north) {t};
          \node[draw,circle,red] (u) [right=of t] {3} node[above] at (u.north) {u};

          \node[draw,circle,red] (v) [below=of r] {2} node[below] at (v.south) {v};
          \node[draw,circle,red] (w) [below=of s] {1} node[below] at (w.south) {w};
          \node[draw,circle,red] (x) [below=of t] {2} node[below] at (x.south) {x};
          \node[draw,circle,red] (y) [below=of u] {3} node[below] at (y.south) {y};

          \draw[->,thick] (r) -- (s);
          \draw[<-,thick] (r) -- (v);
          \draw[<-,thick] (s) -- (w);
          \draw[->,thick] (t) -- (w);
          \draw (t) -- (x);
          \draw[<-,thick] (t) -- (u);
          \draw[<-,thick] (w) -- (x);
          \draw (u) -- (y);
          \draw[<-,thick] (x) -- (y);
        \end{tikzpicture}
      \end{figure}
  \end{enumerate}
  Se si vuole trovare il cammino minimo tra due nodi, si parte dal nodo di destinazione
  e si risale al nodo di partenza seguendo il campo \texttt{parent} di ogni nodo.

  \vspace{1em}
  \noindent
  Questo algoritmo produce un \textbf{albero dei cammini di lunghezza minima} radicato in
  \( s \) che ha un cammino minimo per ogni nodo, se tale cammino esiste.
\end{example}

\noindent
\textbf{Dimostrazione}: Dimostriamo che l'algoritmo BFS produce sempre un albero dei 
cammini di lunghezza minima:

Sia \( \delta(v) \) la lunghezza del cammino minimo da \( s \) a \( v \). Dimostrare che
\[
  \forall v \quad v.distance = \delta(v)
\]
Per dimostrare l'uguaglianza dimostriamo che sia comtemporaneamente maggiore e uguale
e minore e uguale:

\vspace{1em}
\noindent
\textbf{Lemma 1.} \( \forall (u,v) \in E \quad \delta(v) \le \delta(u) + 1 \) 
\begin{figure}[H]
  \centering
  \begin{tikzpicture}
    \node[draw,circle] (s) at (0,0) {s};
    \node[draw,circle] (u) [below right=of s] {u};
    \node[draw,circle] (v) [above right=of s] {v};

    \draw[->,dashed] (s) -- (u) node[midway, below left] {$\delta(u)$};
    \draw[->,dashed] (s) -- (v) node[midway, above left] {$\delta(v)$};
    \draw[->] (u) -- (v) node[midway, right] {$\delta(u) + 1$};
  \end{tikzpicture}
  \caption{Lemma 1}
\end{figure}

\noindent
\textbf{Lemma 2.} \( \forall v \quad v.distance \ge \delta(v) \) perchè:
\[
  \begin{aligned}
    s.distance &= 0 \ge 0\\
    v.distance &= u.distance + 1 \ge \delta(u) + 1 \ge \delta(v)
  \end{aligned}
\] 

\noindent
\textbf{Lemma 3.} Nella coda \( Q \) ci sono smpre al più 2 valori e la coda è ordinata
per distanza crescente.
Sia \( \left<v_1, \ldots, v_r \right> \) il contenuto di \( Q \) in un qualche istante,
allora:
\[
  v_1.distance \le v_2.distance \le \ldots \le v_r.distance \le v_1.distance + 1
\] 
Questo è vero per ogni istruzione del programma, è un \textbf{invariante}.
Ogni istruzione che non modifica \( Q \) e non modifica le distanze non modifica l'invariante.
L'inizializzazione della coda e la modifica della distanza di un nodo da aggiungere alla
coda non modificano l'invariante. L'aggiunta di un nodo alla coda mantiene l'invariante.
Quindi tutte le istruzioni mantengono l'invariante.

\begin{theorem}
  Sia \( V_k \) l'insieme di nodi \( v \;\big|\; \delta(v) = k \), allora
  \( \forall v \in V_k \) esiste un punto dell'algoritmo in cui:
  \begin{itemize}
    \item \( v \) è grigio (scoperto, ma non esplorato).
    \item \( k \) è assegnato a \( v.distance \).
    \item se \( v \neq s \) allora \( v.parent = u \) per qualche \( u \in V_{k-1} \).
    \item \( v \) è inserito in coda
  \end{itemize}
\end{theorem}

\subsubsection{Visita in profondità (DFS: Depth First Search)}
L'algoritmo è il seguente:
\begin{lstlisting}[language=Scala]
// G e' un grafo composto da un insieme di nodi V e un insieme di archi E
dfs(G)
  for u in G.V
    u.color <- white // non esplorato
    u.parent <- NIL

  time <- 0

  for u in G.V
    if u.color == white
      dfs-visit(u)
\end{lstlisting}
\begin{lstlisting}[language=Scala]
// Le variabili della funzione dfs sono accessibili anche da dfs-visit
dfs_visit(u)
  u.color <- gray // scoperto, ma non esplorato
  u.start <- time <- time + 1

  for v in G.adj(u)
    if v.color == white // non esplorato
      v.parent <- u
      dfs-visit(v)

  u.color <- black // esplorato
  u.finish <- time <- time + 1
\end{lstlisting}
La complessità di questo algoritmo è \(O(|V| + |E|)\).
\begin{example}
  I numeri a sinistra indicano il tempo di inizio della visita e i numeri a destra
  la fine della visita
  \begin{figure}[H]
    \centering
    \begin{tikzpicture}[
      every node/.style={minimum size=1.2cm},
      ]
      \node[draw,circle] (y) {3/6} node[above] at (y.north) {y};
      \node[draw,circle] (z) [right=of y] {2/9} node[above] at (z.north) {z};
      \node[draw,circle] (s) [right=of z] {1/10} node[above] at (s.north) {s};
      \node[draw,circle] (t) [right=of s] {11/16} node[above] at (t.north) {t};

      \node[draw,circle] (x) [below=of y] {4/5} node[below] at (x.south) {x};
      \node[draw,circle] (w) [below=of z] {7/8} node[below] at (w.south) {w};
      \node[draw,circle] (v) [below=of s] {12/13} node[below] at (v.south) {v};
      \node[draw,circle] (u) [below=of t] {14/15} node[below] at (u.south) {u};

      \draw[->] (y) -- (x);
      \draw[<-] (y) -- (z);
      \draw[->] (z) -- (w);
      \draw[<-] (z) -- (s);
      \draw[<-] (s) -- (v);
      \draw[->] (t) -- (v);
      \draw[->] (t) -- (u);
      \draw[->] (v) -- (w);
      \draw[->] (w) -- (x);
      \draw[->] (x) -- (z);
      \draw[->] (s) -- (w);

      \draw[->,thick] (x) to [bend left] (y);
      \draw[->,thick] (y) to [bend left] (z);
      \draw[->,thick] (z) to [bend left] (s);
      \draw[->,thick] (w) to [bend left] (z);

      \draw[->,thick] (v) to [bend left] (t);
      \draw[->,thick] (u) to [bend left] (t);

      \node[below left=0cm and 0.2cm of y] {Parent};
    \end{tikzpicture}
    \caption{Visita in profondità}
  \end{figure}
\end{example}

Riprendendo l'esempio precedente scriviamo i passaggi nel seguente modo:
Il tipo di nodo è denotato dal tipo di parentesi:
\begin{itemize}
  \item Parentesi aperta: inizio a visitare il nodo
  \item Parentesi chiusa: fine della visita del nodo
\end{itemize}

\begin{table}[H]
  \centering
  \setlength{\tabcolsep}{2pt}
  \begin{tabular}{ccccccccccccccccc}
    Tempo & 1 & 2 & 3 & 4 & 5 & 6 & 7 & 8 & 9 & 10 & 11 & 12 & 13 & 14 & 15 & 16\\
  \hline
          & (s & (z & (y & (x & x) & y) & (w & w) & z) & s) & (t & (v & v) & (u & u) & t)\\
  \end{tabular}
\end{table}
Questa espressione è ben parentesizzata:
\begin{figure}[H]
  \centering
  \begin{tikzpicture}[scale=0.7]
    \node (1) at (1,0) {1} node[below=0.2cm] at (1) {(s};
    \node (2) at (2,0) {2} node[below=0.2cm] at (2) {(z};
    \node (3) at (3,0) {3} node[below=0.2cm] at (3) {(y};
    \node (4) at (4,0) {4} node[below=0.2cm] at (4) {(x};
    \node (5) at (5,0) {5} node[below=0.2cm] at (5) {x)};
    \node (6) at (6,0) {6} node[below=0.2cm] at (6) {y)};
    \node (7) at (7,0) {7} node[below=0.2cm] at (7) {(w};
    \node (8) at (8,0) {8} node[below=0.2cm] at (8) {w)};
    \node (9) at (9,0) {9} node[below=0.2cm] at (9) {z)};
    \node (10) at (10,0) {10} node[below=0.2cm] at (10) {s)};
    \node (11) at (11,0) {11} node[below=0.2cm] at (11) {(t};
    \node (12) at (12,0) {12} node[below=0.2cm] at (12) {(v};
    \node (13) at (13,0) {13} node[below=0.2cm] at (13) {v)};
    \node (14) at (14,0) {14} node[below=0.2cm] at (14) {(u};
    \node (15) at (15,0) {15} node[below=0.2cm] at (15) {u)};
    \node (16) at (16,0) {16} node[below=0.2cm] at (16) {t)};

    \draw (4,1) rectangle (5,1.6) node[below left] {x} node[midway] (x) {};

    \draw (3,2) rectangle (6,2.6) node[below left] {y} node[midway] (y) {};
    \draw (7,2) rectangle (8,2.6) node[below left] {w} node[midway] (w) {};

    \draw (2,3) rectangle (9,3.6) node[below left] {z} node[midway] (z) {};
    \draw (12,3) rectangle (13,3.6) node[below left] {v} node[midway] (v) {};
    \draw (14,3) rectangle (15,3.6) node[below left] {u} node[midway] (u) {};

    \draw (1,4) rectangle (10,4.6) node[below left] {s} node[midway] (s) {};
    \draw (11,4) rectangle (16,4.6) node[below left] {t} node[midway] (t) {};

    \draw[->] (y) -- (x);
    \draw[->] (z) -- (y);
    \draw[->] (z) -- (w);
    \draw[->] (s) -- (z);
    \draw[->] (t) -- (v);
    \draw[->] (t) -- (u);
  \end{tikzpicture}
  \caption{Visualizzazione dell'albero}
\end{figure}

\noindent
Gli archi si dividono in:
\begin{itemize}
  \item \textbf{Arco dell'albero} (T): è un arco che collega un nodo a un suo discendente
  \item \textbf{Arco all'indietro} (B): è un arco che collega un nodo a un suo antenato
  \item \textbf{Arco in avanti} (F): è un arco che collega un nodo a un discendente non diretto
  \item \textbf{Arco trasversale} (C): è un arco che collega due nodi non correlati
\end{itemize}
Quindi nell'esempio precedente abbiamo:
\begin{figure}[H]
  \centering
  \begin{tikzpicture}[
    every node/.style={minimum size=1.2cm},
    ]
    \node[draw,circle] (y) {3/6} node[above] at (y.north) {y};
    \node[draw,circle] (z) [right=of y] {2/9} node[above] at (z.north) {z};
    \node[draw,circle] (s) [right=of z] {1/10} node[above] at (s.north) {s};
    \node[draw,circle] (t) [right=of s] {11/16} node[above] at (t.north) {t};

    \node[draw,circle] (x) [below=of y] {4/5} node[below] at (x.south) {x};
    \node[draw,circle] (w) [below=of z] {7/8} node[below] at (w.south) {w};
    \node[draw,circle] (v) [below=of s] {12/13} node[below] at (v.south) {v};
    \node[draw,circle] (u) [below=of t] {14/15} node[below] at (u.south) {u};

    \draw[->] (y) -- (x) node[midway, right=-0.3cm] {T};
    \draw[<-] (y) -- (z) node[midway, above=-0.3cm] {T};
    \draw[->] (z) -- (w) node[midway, right=-0.3cm] {T};
    \draw[<-] (z) -- (s) node[midway, above=-0.3cm] {T};
    \draw[<-] (s) -- (v) node[midway, right=-0.3cm] {C};
    \draw[->] (t) -- (v) node[midway, right=-0.2cm] {T};
    \draw[->] (t) -- (u) node[midway, right=-0.3cm] {T};
    \draw[->] (v) -- (w) node[midway, above=-0.3cm] {C};
    \draw[->] (w) -- (x) node[midway, above=-0.3cm] {C};
    \draw[->] (x) -- (z) node[midway, right=-0.3cm] {B};
    \draw[->] (s) -- (w) node[midway, right=-0.3cm] {F};
  \end{tikzpicture}
  \caption{Tipi di archi}
\end{figure}

nel nostro algoritmo abbiamo che il clolore degli archi distingue i vari tipi:
\begin{itemize}
  \item \textbf{Bianco} (non esplorato): arco trasversale (T)
  \item \textbf{Grigio} (scoperto, ma non esplorato): arco all'indietro (B)
  \item \textbf{Nero} (esplorato): arco dell'albero o arco in avanti (F,C)
\end{itemize}
Da questo consegue che se ci sono archi all'indietro è presente un ciclo, quindi esiste
un algoritmo di complessità \(O(|V| + |E|)\) per trovare se un grafo è ciclico e questo
algoritmo è il DFS.

\begin{theorem}
  Dopo una DFS \( \forall u, v \) gli intervalli \( [\text{\texttt{u.start}},
  \text{\texttt{u.finish}}] \) sono disgiunti, oppure uno sottointervallo dell'altro

  \vspace{1em}
  \noindent
  \textbf{Dimostrazione}:
  \begin{enumerate}
    \item Supponiamo che \( \text{\texttt{u.start}} < \text{\texttt{v.start}} \)
      \begin{enumerate}
        \item Se \( \text{\texttt{u.finish}} < \text{\texttt{v.start}} \) allora
          i due intervalli sono disgiunti
        \item Se \( \text{\texttt{u.start}} < \text{\texttt{v.finish}} \) allora
          \( v \) è un sottointervallo di \( u \) 
      \end{enumerate}
      \textbf{Corollario}: In DFS \( v \) discende da \( u \) se e solo se:
      \[
        \text{\texttt{u.start}} < \text{\texttt{v.start}} < \text{\texttt{v.finish}} < \text{\texttt{u.finish}}
      \] 
  \end{enumerate}
\end{theorem}
\begin{theorem}
  Nella foresta di alberi generata da una DFS, un nodo \( v \) è un discendente di un nodo
  \( u \) se e solo se al tempo \( \text{\texttt{u.start}} \) esiste un cammino da \( u \) 
  a \( v \) fatto di soli nodi bianchi (non esplorati).

  \vspace{1em}
  \noindent
  \textbf{Dimostrazione}: Supponiamo che \( v \) discende da \( u \), sia \( w \) un nodo
  del cammino da \( u \to v \) della foresta:
  \[
    \text{\texttt{u.start}} < \text{\texttt{w.start}}
  \] 
  Quindi nel momento in cui \( u \) viene scoperto, \( w \) è ancora bianco.
  \begin{figure}[H]
    \begin{itemize}
      \item Nero: non esplorato,
      \item \textcolor{blue}{Blu}: scoperto, ma non esplorato,
      \item \textcolor{red}{Rosso}: esplorato,
    \end{itemize}
    \centering
    \begin{tikzpicture}
      \node[draw,circle,red] (u) {u};
      \node[draw,circle] (w) [right=of u] {w};
      \node[draw,circle] (v) [right=of w] {v};

      \draw[->] (u) -- (w);
      \draw[->] (w) -- (v);
    \end{tikzpicture}
  \end{figure}

  \vspace{1em}
  \noindent
  \( v \) raggiungibile da \( u \) al tempo \( \text{\texttt{u.start}} \) con cammino
  di nodi bianchi (non esplorati). Supponiamo per assurdo che \( v \) non discende da \( u \) 
  \begin{figure}[H]
    \centering
    \begin{tikzpicture}[
      every node/.style={minimum size=0.8cm},
      ]
      \node[draw,circle] (u) {\( u \) };
      \node[draw,circle] (u1) [right=of u] {\( u_1 \)};
      \node[draw,circle] (u2) [right=of u1] {\( u_2 \)};
      \node[draw,circle] (w) [right=of u2] {\( w \)};
      \node[draw,circle] (v) [right=of w] {\( v \)};

      \draw[->] (u) -- (u1);
      \draw[->] (u1) -- (u2);
      \draw[->] (u2) -- (w);
      \draw[->] (w) -- (v);
    \end{tikzpicture}
  \end{figure}
  \noindent
  Supponiamo, senza perdita di generalità, che il predecessore di \( v \) discende da \( u \).
  Sia \( w \) il predecessore di \( v \), allora \( w \) discende da \( u \), quindi:
  \[
    \text{\texttt{w.finish}} < \text{\texttt{u.finish}}
  \] 
  di conseguenza:
  \[
    \text{\texttt{u.start}} < \text{\texttt{w.start}} < 
    \underbrace{\text{\texttt{v.start}} < \text{\texttt{v.finish}}}_{\text{Sottointervallo
    di \( u \)}}
    < \text{\texttt{w.finish}} < \text{\texttt{u.finish}}
  \] 
  Quindi l'intervallo \( v \) è un sottointervallo di \( u \) contraddicendo l'ipotesi
  e dimostrando che \( v \) discende da \( u \).
\end{theorem}

\begin{theorem}
  Un grafo è aciclico se e solo se DFS \textbf{non} trova archi indietro, ed è ciclico
  se e solo se trova almeno un arco indietro.

  \vspace{1em}
  \noindent
  \textbf{Dimostrazione}: consideriamo un qualsiasi grafo ciclico. DFS scoprirà un nodo
  per primo e quel nodo lo chiamiamo \( u \) e ci sarà un nodo scoperto per ultimo
  chiamato \( v \). Se \( v \) discende da \( u \) allora esiste un cammino da \( u \) a
  \( v \) fatto di nodi bianchi (non esplorati) e quindi esiste un ciclo.
\end{theorem}

\begin{example}
  Un robot si vuole vestire e deve indossare degli indumenti in un certo ordine. Bisogna
  trovare un algoritmo che trovi una soluzione tenendo in considerazione tutti i vincoli.

  Una rappresentazione più astratta del problema potrebbe essere un grafo orientato
  in cui i nodi sono gli indumenti e tra due nodi c'è un arco se un indumento deve essere
  indossato prima dell'altro. Ad esempio:
  \begin{figure}[H]
    \centering
    \begin{tikzpicture}
      \node[draw,ellipse] (slip) {Slip};
      \node[draw,ellipse] (pantaloni) [below=of slip] {Pantaloni};
      \node[draw,ellipse] (cintura) [below=of pantaloni] {Cintura};
      \node[draw,ellipse] (camicia) [above right=of pantaloni] {Camicia};
      \node[draw,ellipse] (cravatta) [below=of camicia] {Cravatta};
      \node[draw,ellipse] (giacca) [below=of cravatta] {Giacca};
      \node[draw,ellipse] (calzini) [above left=of pantaloni] {Calzini};
      \node[draw,ellipse] (scarpe) [below=of calzini] {Scarpe};
      \node[draw,ellipse] (orologio) [left=of calzini] {Orologio};

      \draw[->] (slip) -- (pantaloni);
      \draw[->] (pantaloni) -- (cintura);
      \draw[->] (camicia) -- (pantaloni);
      \draw[->] (camicia) -- (cravatta);
      \draw[->] (cravatta) -- (giacca);
      \draw[->] (cintura) -- (giacca);
      \draw[->] (calzini) -- (scarpe);
      \draw[->] (pantaloni) -- (scarpe);
    \end{tikzpicture}
    \caption{Esempio di rappresentazione del problema}
  \end{figure}
  \noindent
  Questo grafo \textbf{rappresenta una relazione} di ordinamento \textbf{parziale} tra gli indumenti
  (se non sono presenti cicli). L'obiettivo è di prendere la relazione parziale e renderla
  totale, ma mantenendola compatibile coi vincoli già imposti. In questo caso ad esempio
  bisogna imporre altri vincoli e dire che l'orologio va messo prima di qualcos'altro
  mantenendo l'ordinamento parziale dato all'inizio.

  Questo problema si chiama \textbf{Ordinamento topologico} e la risoluzione è la seguente:
\begin{lstlisting}[language=Scala]
// G e' un grafo
topological_sorting(G)
  stack = dfs(G) // DFS ritorna una pila con i nodi in ordine decrescente di finish
\end{lstlisting}
  Questo algoritmo ha complessità \(O(|V| + |E|)\), quindi si può risolvere un ordinamento
  topologico in tempo lineare.

  \vspace{1em}
  \noindent
  L'applicazione dell'algoritmo sul grafo preso in esempio è la seguente consideramdo che
  si inizi visitando la camicia (i numeri a sinistra indicano il tempo di inizio della visita
  e i numeri a destra la fine della visita):
  \begin{figure}[H]
    \centering
    \begin{tikzpicture}
      \node[draw,ellipse] (slip) {Slip}
        node[above,scale=0.7] at (slip.north) {17/18};
      \node[draw,ellipse] (pantaloni) [below=of slip] {Pantaloni}
        node[above=0.2cm,scale=0.7] at (pantaloni.west) {6/11};
      \node[draw,ellipse] (cintura) [below=of pantaloni] {Cintura}
        node[left,scale=0.7] at (cintura.west) {7/8};
      \node[draw,ellipse] (camicia) [above right=of pantaloni] {Camicia}
        node[right,scale=0.7] at (camicia.east) {1/12};
      \node[draw,ellipse] (cravatta) [below=of camicia] {Cravatta}
        node[right,scale=0.7] at (cravatta.east) {2/5};
      \node[draw,ellipse] (giacca) [below=of cravatta] {Giacca}
        node[right,scale=0.7] at (giacca.east) {3/4};
      \node[draw,ellipse] (calzini) [above left=of pantaloni] {Calzini}
        node[above,scale=0.7] at (calzini.north) {15/16};
      \node[draw,ellipse] (scarpe) [below=of calzini] {Scarpe}
        node[left,scale=0.7] at (scarpe.west) {9/10};
      \node[draw,ellipse] (orologio) [left=of calzini] {Orologio}
        node[above,scale=0.7] at (orologio.north) {13/14};

      \draw[->] (slip) -- (pantaloni);
      \draw[->] (pantaloni) -- (cintura);
      \draw[->] (camicia) -- (pantaloni);
      \draw[->] (camicia) -- (cravatta);
      \draw[->] (cravatta) -- (giacca);
      \draw[->] (cintura) -- (giacca);
      \draw[->] (calzini) -- (scarpe);
      \draw[->] (pantaloni) -- (scarpe);
    \end{tikzpicture}
    \caption{Esempio di rappresentazione del problema}
  \end{figure}
  \begin{table}[H]
    \centering
    \begin{tabular}{|c|}
      \hline
      Slip\\
      \hline
      Calzini\\
      \hline
      Orologio\\
      \hline
      Camicia\\
      \hline
      Pantaloni\\
      \hline
      Scarpe\\
      \hline
      Cintura\\
      \hline
      Cravatta\\
      \hline
      Giacca\\
      \hline
    \end{tabular}
    \caption{Stack}
  \end{table}
  \noindent
  Per dire che l'algoritmo funziona bisogna dimostrare che:
  \[
    \forall (u,v) \in E \quad \text{\texttt{v.finish}} < \text{\texttt{u.finish}}
  \] 
  \textbf{Dimostrazione}: Quando \( (u,v) \) viene esplorato, allora se:
  \begin{itemize}
    \item \( v \) è bianco (non esplorato) allora \( v \) è un discendente di \( u \),
      quindi:
      \[
        \text{\texttt{v.finish}} < \text{\texttt{u.finish}}
      \] 
    \item \( v \) è grigio (scoperto, ma non esplorato) non è possibile, perchè se \( v \) 
      vosse grigio ci sarebbe un grafo ciclico e la soluzione non esiste.
    \item \( v \) è nero (esplorato) allora \( u \) non è ancora stato esplorato, quindi: 
      \[
        \text{\texttt{v.finish}} < \text{\texttt{u.finish}}
      \] 
  \end{itemize}
  Quindi l'algoritmo funziona.
\end{example}

\begin{theorem}
  In un grafo aciclico c'è per forza almeno un nodo che non ha archi entranti.

  \vspace{1em}
  \noindent
  \textbf{Dimostrazione}: Supponiamo per assurdo che tutti i nodi abbiano almeno un arco
  entrante. Si può creare una catena di nodi di grandezza \( |V|+1 \) all'infinito, che
  prima o poi si ripeterà, creando un ciclo.
\end{theorem}

\begin{example}
  Consideriamo le strade d'Italia come grafo, dove i nodi sono le città e gli archi
  sono le strade a doppio senso. Ci sono parti non raggiungibili (come la Sardegna)
  e quindi ci sono più grafi separati chiamati \textbf{componenti connesse}.
\begin{lstlisting}[language=Scala]
cc(G)
  for v in G.V
    make_set(v)
  for (u,v) in G.E
    union(u,v)
\end{lstlisting}
  Questo algoritmo costruisce le componenti connesse di un grafo in tempo \(O(|V| + |E|)\).
  Quindi dati 2 elementi si può trovare in tempo costante se appartengono alla stessa
  componente connessa e di conseguenza se sono raggiungibili.

  Un'alternativa è modificare il DFS in modo che quando si visita un nodo si metta
  un'etichetta con il numero della componente connessa.
\begin{lstlisting}[language=Scala]
dfs_cc(G)
  for v in G.V
    v.color <- white
    v.parent <- NIL

  cc <- 0
  time <- 0

  for u in G.V
    if u.color == white
      cc <- cc + 1
      dfs_visit_cc(u, cc)
\end{lstlisting}
\begin{lstlisting}[language=Scala]
dfs_visit_cc(u, cc)
  u.color <- gray
  u.start <- time <- time + 1

  u.cc <- cc

  for v in G.adj(u)
    if v.color == white
      v.parent <- u
      dfs_visit_cc(v, cc)

  u.color <- black
  u.finish <- time <- time + 1
\end{lstlisting}
\end{example}

\subsection{Componenti fortemente connesse}
\begin{definition}
  Dato un grafo orientato \( G = (V,E) \) una componente fortemente connessa (SCC) è un
  sottoinsieme \( V' \subseteq V \) tale che:
  \[
    \forall u,v \in V' \quad \exists \text{ cammino da } u \to v \quad \text{e} \quad
    \exists \text{ cammino da } v \to u
  \]
\end{definition}
\begin{example}
  Nel seguente esempio ci sono 2 componenti fortemente connesse, una composta da \( a,b,c,d \)
  e l'altra da \( e,f,g \).
  \begin{figure}[H]
    \centering
    \begin{tikzpicture}[
      every node/.style={minimum size=0.8cm},
      ]
      \node[draw,circle,blue] (a) {a};
      \node[draw,circle,blue] (b) [below left=of a] {b};
      \node[draw,circle,blue] (d) [below right=of b] {d};
      \node[draw,circle,blue] (c) [below right=of a] {c};

      \draw[->] (a) -- (b);
      \draw[->] (b) -- (d);
      \draw[->] (d) -- (c);
      \draw[->] (c) -- (a);


      \node[draw,circle,red] (e) [right=of c] {e};
      \node[draw,circle,red] (f) [below right=of e] {f};
      \node[draw,circle,red] (g) [below left=of f] {g};
      \draw[->] (c) -- (e);

      \draw[->] (e) -- (f);
      \draw[->] (f) -- (g);
      \draw[->] (g) -- (e);
    \end{tikzpicture}
    \caption{Esempio di grafo con 2 componenti fortemente connesse}
  \end{figure}
\end{example}
L'algoritmo per individuare le componenti fortemente connesse è il seguente:
\begin{lstlisting}[language=Scala]
scc(G)
  finish[] = dfs_finish(G) // Usa DFS per trovare i tempi di fine
  Gt = G.transpose() // Grafo trasposto
  dfs_from_highest_finish(Gt) // Visita i nodi in ordine decrescente di finish
\end{lstlisting}
Questo algoritmo ha complessità \(O(|V| + |E|)\).
\begin{example}
  L'algoritmo eseguito passo passo su un grafo è il seguente:
  \begin{figure}[H]
    \centering
    \begin{tikzpicture}[
      every node/.style={minimum size=1.2cm},
      ]
      \node[draw,circle,blue] (a) {13/14} node[above] at (a.north) {a};
      \node[draw,circle,blue] (b) [right=of a] {11/16} node[above] at (b.north) {b};
      \node[draw,circle,red] (c) [right=of b] {1/10} node[above] at (c.north) {c};
      \node[draw,circle,red] (d) [right=of c] {8/9} node[above] at (d.north) {d};

      \node[draw,circle,blue] (e) [below=of a] {12/15} node[below] at (e.south) {e};
      \node[draw,circle,red] (f) [below=of b] {3/4} node[below] at (f.south) {f};
      \node[draw,circle,red] (g) [below=of c] {2/7} node[below] at (g.south) {g};
      \node[draw,circle,red] (h) [below=of d] {5/6} node[below] at (h.south) {h};

      \draw[->] (a) -- (b);
      \draw[->] (b) -- (c);
      \draw[->] (c) edge [bend left] (d);
      \draw[->] (d) edge [bend left] (c);
      \draw[->] (d) -- (h);
      \draw[->] (c) -- (g);
      \draw[->] (g) edge [bend left] (f);
      \draw[->] (f) edge [bend left] (g);
      \draw[->] (e) -- (f);
      \draw[->] (e) -- (a);
      \draw[->] (b) -- (e);
      \draw[->] (b) -- (f);
      \draw[->] (g) -- (h);
      \draw[->] (h) edge [loop right] (h);
    \end{tikzpicture}
    \caption{Tipi di archi}
  \end{figure}
  Mentre sul grafo trasposto:
  \begin{figure}[H]
    \centering
    \begin{tikzpicture}[
      every node/.style={minimum size=1.2cm},
      ]
      \node[draw,circle,blue] (a) {2/5} node[above] at (a.north) {a};
      \node[draw,circle,blue] (b) [right=of a] {1/6} node[above] at (b.north) {b};
      \node[draw,circle,red] (c) [right=of b] {7/10} node[above] at (c.north) {c};
      \node[draw,circle,red] (d) [right=of c] {8/9} node[above] at (d.north) {d};

      \node[draw,circle,blue] (e) [below=of a] {3/4} node[below] at (e.south) {e};
      \node[draw,circle,green!50!black] (f) [below=of b] {12/13} node[below] at (f.south) {f};
      \node[draw,circle,green!50!black] (g) [below=of c] {11/14} node[below] at (g.south) {g};
      \node[draw,circle,magenta] (h) [below=of d] {15/16} node[below] at (h.south) {h};

      \draw[<-] (a) -- (b);
      \draw[<-] (b) -- (c);
      \draw[<-] (c) edge [bend left] (d);
      \draw[<-] (d) edge [bend left] (c);
      \draw[<-] (d) -- (h);
      \draw[<-] (c) -- (g);
      \draw[<-] (g) edge [bend left] (f);
      \draw[<-] (f) edge [bend left] (g);
      \draw[<-] (e) -- (f);
      \draw[<-] (e) -- (a);
      \draw[<-] (b) -- (e);
      \draw[<-] (b) -- (f);
      \draw[<-] (g) -- (h);
      \draw[<-] (h) edge [loop right] (h);
    \end{tikzpicture}
    \caption{Tipi di archi}
  \end{figure}
\end{example}

\vspace{1em}
\noindent
\textbf{Lemma}: Dopo la prima DFS, ogni componente fortemente connessa è interamente
inclusa in un albero della foresta DFS.

\vspace{1em}
\noindent
Il \textbf{grafo delle componenti fortemente connesse} è un grafo in cui ogni nodo corrisponde
a una componente fortemente connessa e gli archi tra due nodi indicano che esiste un
arco tra due nodi delle due componenti connesse.

Se c'è un ciclo tra due componenti fortemente connesse, tutti i nodi che fanno parte 
del ciclo sono in una stessa componente fortemente connessa.

\vspace{1em}
\noindent
Un grafo delle componenti fortemente connesse contiene un cammino infinito che raggiunge
infinite volte uno stato finale quando il cammino arriva in una componente fortemente
connessa che contiene almeno 2 nodi o un solo nodo con un autoanello.

\subsubsection{Cammini minimi}
I ponti di Verona sono tutti rotti e non si possono più riparare, ma bisogna garantire
un servizio minimo, cioè permettere ad ogni persona di spostarsi in qualunque zona a
piacimento. Quindi tra 2 zone c'è bisogno di un collegamento. Ogni ponte ha un costo
di riparazione.

L'obiettivo è quello di trovare il costo minimo per riparare i ponti in modo che
ogni zona sia raggiungibile da ogni altra zona.

\vspace{1em}
\noindent
I nodi saranno le zone e gli archi saranno i ponti. Sappiamo che se il numero di
nodi è \( n \), allora il numero di ponti per garantire la connettività è \( n-1 \).
E per induzione se si aggiunge un nodo bisogna aggiungere un arco.
\begin{enumerate}
  \item Se si vuole minimizzare il costo, bisogna minimizzare il numero di ponti riparati.
  \item Presi due nodi esiste un unico modo per andare da un nodo all'altro. Se non fosse
    così vuol dire che ci sono più ponti che collegano due nodi e quindi il numero di
    ponti non è minimo. (Quindi il grafo è un albero)
  \item Bisogna trovare un albero, tra tutti gli alberi che garantisce la connettività
    del grafo (chiamati \textbf{albero di copertura} o \textbf{spanning tree}), con
    costo minimo. Quindi bisogna trovare il \textbf{Minimum Spanning Tree} (MST).
  \item Se i pesi negativi, allora la soluzione potrebbe non essere più un albero,
    però da questa soluzione si può trovare un albero che collega tutti i nodi con
    costo minimo. Quindi si può risolvere il problema in modo più generale senza
    imporre i pesi positivi.
  \item Si può assegnare ad ogni ponte un indice di gradimento e trovare il MST
    che massimizza il gradimento. Si può usare un algoritmo che calcola
    il costo minimo e invertire i segni per trovare il massimo.
\end{enumerate}

\vspace{1em}
\noindent
L'idea dell'algoritmo è quella di \textbf{tagliare} in due un grafo, cioè una bipartizione
dell'insieme dei nodi. Se gli estremi di un arco fanno parte dello stesso insieme, allora
l'arco è un \textbf{arco interno}, altrimenti è un \textbf{arco di taglio}. Tra tutti
gli archi del taglio, quello con costo minimo è chiamato \textbf{arco sicuro}.

\textbf{Lemma}: Dato un albero qualsiasi con un taglio, fra tutti gli MST c'è almeno uno
che contiene l'arco sicuro.

Con questo si può ridurre il problema ad uno dello stesso tipo, ma con un nodo in meno,
e questo si può fare rappresentando due nodi collegati da un arco sicuro come un nodo
singolo e rifare un altro taglio (senza dividere ciò che è stato unito prima).

L'algoritmo è il seguente (algoritmo di Kruskal):
\begin{lstlisting}[language=Scala]
// G e' un grafo
// w e' una matrice che associa ad ogni arco il suo peso
kruskal(G, w)
  A <- {}

  for v in G.V
    make_set(v)

  sort(G.E, w.not_descending) // Ordina per peso NON decrescente
  for (u,v) in G.E
    if find_set(u) != find_set(v)
      A <- union(A, {(u,v)})
      union(u,v)

  return A
\end{lstlisting}

\begin{example}
L'applicazione di questo algoritmo è la seguente:
\begin{figure}[H]
  \centering
  \begin{tikzpicture}
    \node[draw,circle] (a) {a};
    \node[draw,circle] (b) [above right=of a] {b};
    \node[draw,circle] (f) [below right=of b] {f};
    \node[draw,circle] (d) [above right=of f] {d};
    \node[draw,circle] (c) [below left=of f] {c};
    \node[draw,circle] (e) [below right=of f] {e};

    \draw (a) -- (b) node[midway, above] {4};
    \draw (a) -- (c) node[midway, left] {8};
    \draw (b) -- (c) node[midway, left] {11};
    \draw (b) -- (f) node[midway, above right] {9};
    \draw (f) -- (d) node[midway, below right] {2};
    \draw (d) -- (e) node[midway, right] {6};
    \draw (e) -- (f) node[midway, above right] {5};
    \draw (c) -- (f) node[midway, above left] {7};
    \draw (c) -- (e) node[midway, below] {1};
    \draw (b) -- (d) node[midway, above] {8};
  \end{tikzpicture}
  \caption{Esempio di grafo}
\end{figure}
\begin{enumerate}
  \item 
    Eseguiamo il primo taglio e prendiamo il nodo sicuro:
    \begin{figure}[H]
      \centering
      \begin{tikzpicture}
        \node[draw,circle] (a) {a};
        \node[draw,circle] (b) [above right=of a] {b};
        \node[draw,circle] (f) [below right=of b] {f};
        \node[draw,circle] (d) [above right=of f] {d};
        \node[draw,circle] (c) [below left=of f] {c};
        \node[draw,circle] (e) [below right=of f] {e};

        \draw (a) -- (b) node[midway, above] {4};
        \draw (a) -- (c) node[midway, left] {8};
        \draw (b) -- (c) node[midway, left] {11};
        \draw (b) -- (f) node[midway, above right] {9};
        \draw (f) -- (d) node[midway, below right] {2};
        \draw (d) -- (e) node[midway, right] {6};
        \draw (e) -- (f) node[midway, above right] {5};
        \draw (c) -- (f) node[midway, above left] {7};
        \draw[thick,blue] (c) -- (e) node[midway, below] {1};
        \draw (b) -- (d) node[midway, above] {8};

        \draw[dashed] (2.2,2) -- (2.2,-2);
      \end{tikzpicture}
    \end{figure}

  \item Secondo taglio
    \begin{figure}[H]
      \centering
      \begin{tikzpicture}
        \node[draw,circle] (a) {a};
        \node[draw,circle] (b) [above right=of a] {b};
        \node[draw,circle] (f) [below right=of b] {f};
        \node[draw,circle] (d) [above right=of f] {d};
        \node[draw,circle] (c) [below left=of f] {c};
        \node[draw,circle] (e) [below right=of f] {e};

        \draw (a) -- (b) node[midway, above] {4};
        \draw (a) -- (c) node[midway, left] {8};
        \draw (b) -- (c) node[midway, left] {11};
        \draw (b) -- (f) node[midway, above right] {9};
        \draw[thick,blue] (f) -- (d) node[midway, below right] {2};
        \draw (d) -- (e) node[midway, right] {6};
        \draw (e) -- (f) node[midway, above right] {5};
        \draw (c) -- (f) node[midway, above left] {7};
        \draw[thick] (c) -- (e) node[midway, below] {1};
        \draw (b) -- (d) node[midway, above] {8};

        \draw[dashed] (3,2) -- (5,0);
      \end{tikzpicture}
    \end{figure}

  \item Terzo taglio
    \begin{figure}[H]
      \centering
      \begin{tikzpicture}
        \node[draw,circle] (a) {a};
        \node[draw,circle] (b) [above right=of a] {b};
        \node[draw,circle] (f) [below right=of b] {f};
        \node[draw,circle] (d) [above right=of f] {d};
        \node[draw,circle] (c) [below left=of f] {c};
        \node[draw,circle] (e) [below right=of f] {e};

        \draw[thick,blue] (a) -- (b) node[midway, above] {4};
        \draw (a) -- (c) node[midway, left] {8};
        \draw (b) -- (c) node[midway, left] {11};
        \draw (b) -- (f) node[midway, above right] {9};
        \draw[thick] (f) -- (d) node[midway, below right] {2};
        \draw (d) -- (e) node[midway, right] {6};
        \draw (e) -- (f) node[midway, above right] {5};
        \draw (c) -- (f) node[midway, above left] {7};
        \draw[thick] (c) -- (e) node[midway, below] {1};
        \draw (b) -- (d) node[midway, above] {8};

        \draw[dashed] (0.8,2) -- (0.8,-2);
      \end{tikzpicture}
    \end{figure}

  \item Quarto taglio
    \begin{figure}[H]
      \centering
      \begin{tikzpicture}
        \node[draw,circle] (a) {a};
        \node[draw,circle] (b) [above right=of a] {b};
        \node[draw,circle] (f) [below right=of b] {f};
        \node[draw,circle] (d) [above right=of f] {d};
        \node[draw,circle] (c) [below left=of f] {c};
        \node[draw,circle] (e) [below right=of f] {e};

        \draw[thick] (a) -- (b) node[midway, above] {4};
        \draw (a) -- (c) node[midway, left] {8};
        \draw (b) -- (c) node[midway, left] {11};
        \draw (b) -- (f) node[midway, above right] {9};
        \draw[thick] (f) -- (d) node[midway, below right] {2};
        \draw (d) -- (e) node[midway, right] {6};
        \draw[thick,blue] (e) -- (f) node[midway, above right] {5};
        \draw (c) -- (f) node[midway, above left] {7};
        \draw[thick] (c) -- (e) node[midway, below] {1};
        \draw (b) -- (d) node[midway, above] {8};

        \draw[dashed] (-1,-0.8) -- (5,-0.8);
      \end{tikzpicture}
    \end{figure}

  \item Quinto taglio
    \begin{figure}[H]
      \centering
      \begin{tikzpicture}
        \node[draw,circle] (a) {a};
        \node[draw,circle] (b) [above right=of a] {b};
        \node[draw,circle] (f) [below right=of b] {f};
        \node[draw,circle] (d) [above right=of f] {d};
        \node[draw,circle] (c) [below left=of f] {c};
        \node[draw,circle] (e) [below right=of f] {e};

        \draw[thick] (a) -- (b) node[midway, above] {4};
        \draw[thick,blue] (a) -- (c) node[midway, left] {8};
        \draw (b) -- (c) node[midway, left] {11};
        \draw (b) -- (f) node[midway, above right] {9};
        \draw[thick] (f) -- (d) node[midway, below right] {2};
        \draw (d) -- (e) node[midway, right] {6};
        \draw[thick] (e) -- (f) node[midway, above right] {5};
        \draw (c) -- (f) node[midway, above left] {7};
        \draw[thick] (c) -- (e) node[midway, below] {1};
        \draw (b) -- (d) node[midway, above] {8};

        \draw[dashed] (0,-1.5) -- (3,2.5);
      \end{tikzpicture}
    \end{figure}
  \item Alla fine la soluzione è:
    \begin{figure}[H]
      \centering
      \begin{tikzpicture}
        \node[draw,circle] (a) {a};
        \node[draw,circle] (b) [above right=of a] {b};
        \node[draw,circle] (f) [below right=of b] {f};
        \node[draw,circle] (d) [above right=of f] {d};
        \node[draw,circle] (c) [below left=of f] {c};
        \node[draw,circle] (e) [below right=of f] {e};

        \draw[thick,blue] (a) -- (b) node[midway, above] {4};
        \draw[thick,blue] (a) -- (c) node[midway, left] {8};
        \draw (b) -- (c) node[midway, left] {11};
        \draw (b) -- (f) node[midway, above right] {9};
        \draw[thick,blue] (f) -- (d) node[midway, below right] {2};
        \draw (d) -- (e) node[midway, right] {6};
        \draw[thick,blue] (e) -- (f) node[midway, above right] {5};
        \draw (c) -- (f) node[midway, above left] {7};
        \draw[thick,blue] (c) -- (e) node[midway, below] {1};
        \draw (b) -- (d) node[midway, above] {8};
      \end{tikzpicture}
    \end{figure}
\end{enumerate}
\end{example}

\vspace{1em}
\noindent
Un altro algoritmo per risolvere questo problema è l'algoritmo di Prim:
\begin{lstlisting}[language=Scala]
// G e' un grafo
// w e' una matrice che associa ad ogni arco il suo peso
// s e' il nodo di partenza
prim(G,w,s)
  Q <- G.V
  for u in G.V
    u.key <- +inf
    s.key <- 0
    s.parent <- NIL

    while Q != {}
      u <- extract_min(Q)
      for v in G.adj(u)
        if Q.contains(v) and w(u,v) < v.key
          v.parent <- u
          v.key <- w(u,v)
\end{lstlisting}
La complessità di questo algoritmo è:
\begin{itemize}
  \item \( V \) per la \texttt{make\_set}
  \item \( E \log E \) per l'ordinamento
  \item \( E \) per la \texttt{find\_set}
  \item \( V \) union
\end{itemize}
Il risultato finale è:
\[
  (V+E) \alpha(\ldots) + E \log E
\] 
Quindi la complessità è:
\[
  O(E \log E)
\]
\begin{example}
  L'applicazione dell'algoritmo è la seguente
  \begin{enumerate}
    \item Il grafo è il seguente
      \begin{figure}[H]
        \centering
        \begin{tikzpicture}
          \node[draw,circle,thick,blue,double] (a) {a} node[above] at (a.north) {0};
          \node[draw,circle] (b) [above right=of a] {b} node[above] at (b.north) {$\infty$};
          \node[draw,circle] (f) [below right=of b] {f} node[above] at (f.north) {$\infty$};
          \node[draw,circle] (d) [above right=of f] {d} node[above] at (d.north) {$\infty$};
          \node[draw,circle] (c) [below left=of f] {c} node[below] at (c.south) {$\infty$};
          \node[draw,circle] (e) [below right=of f] {e} node[below] at (e.south) {$\infty$};

          \draw (a) -- (b) node[midway, above] {4};
          \draw (a) -- (c) node[midway, left] {8};
          \draw (b) -- (c) node[midway, left] {11};
          \draw (b) -- (f) node[midway, above right] {9};
          \draw (f) -- (d) node[midway, below right] {2};
          \draw (d) -- (e) node[midway, right] {6};
          \draw (e) -- (f) node[midway, above right] {5};
          \draw (c) -- (f) node[midway, above left] {7};
          \draw (c) -- (e) node[midway, below] {1};
          \draw (b) -- (d) node[midway, above] {8};
        \end{tikzpicture}
        \caption{Esempio di grafo}
      \end{figure}

    \item Secondo passo
      \begin{figure}[H]
        \centering
        \begin{tikzpicture}
          \node[draw,circle,thick,blue,double] (a) {a} node[above] at (a.north) {0};
          \node[draw,circle,thick,blue] (b) [above right=of a] {b} node[above] at (b.north) {$4$};
          \node[draw,circle] (f) [below right=of b] {f} node[above] at (f.north) {$\infty$};
          \node[draw,circle] (d) [above right=of f] {d} node[above] at (d.north) {$\infty$};
          \node[draw,circle] (c) [below left=of f] {c} node[below] at (c.south) {$8$};
          \node[draw,circle] (e) [below right=of f] {e} node[below] at (e.south) {$\infty$};

          \draw[<-,thick,blue] (a) -- (b) node[midway, above] {4};
          \draw[<-] (a) -- (c) node[midway, left] {8};
          \draw (b) -- (c) node[midway, left] {11};
          \draw (b) -- (f) node[midway, above right] {9};
          \draw (f) -- (d) node[midway, below right] {2};
          \draw (d) -- (e) node[midway, right] {6};
          \draw (e) -- (f) node[midway, above right] {5};
          \draw (c) -- (f) node[midway, above left] {7};
          \draw (c) -- (e) node[midway, below] {1};
          \draw (b) -- (d) node[midway, above] {8};
        \end{tikzpicture}
        \caption{Esempio di grafo}
      \end{figure}

    \item Terzo passo
      \begin{figure}[H]
        \centering
        \begin{tikzpicture}
          \node[draw,circle,thick,blue,double] (a) {a} node[above] at (a.north) {0};
          \node[draw,circle,thick,blue] (b) [above right=of a] {b} node[above] at (b.north) {$4$};
          \node[draw,circle] (f) [below right=of b] {f} node[above] at (f.north) {$9$};
          \node[draw,circle,thick,blue] (d) [above right=of f] {d} node[above] at (d.north) {$8$};
          \node[draw,circle] (c) [below left=of f] {c} node[below] at (c.south) {$8$};
          \node[draw,circle] (e) [below right=of f] {e} node[below] at (e.south) {$\infty$};

          \draw[<-,thick,blue] (a) -- (b) node[midway, above] {4};
          \draw (a) -- (c) node[midway, left] {8};
          \draw (b) -- (c) node[midway, left] {11};
          \draw (b) -- (f) node[midway, above right] {9};
          \draw (f) -- (d) node[midway, below right] {2};
          \draw (d) -- (e) node[midway, right] {6};
          \draw (e) -- (f) node[midway, above right] {5};
          \draw (c) -- (f) node[midway, above left] {7};
          \draw (c) -- (e) node[midway, below] {1};
          \draw[<-,thick,blue] (b) -- (d) node[midway, above] {8};
        \end{tikzpicture}
        \caption{Esempio di grafo}
      \end{figure}

    \item Quarto passo
      \begin{figure}[H]
        \centering
        \begin{tikzpicture}
          \node[draw,circle,thick,blue,double] (a) {a} node[above] at (a.north) {0};
          \node[draw,circle,thick,blue] (b) [above right=of a] {b} node[above] at (b.north) {$4$};
          \node[draw,circle,thick,blue] (f) [below right=of b] {f} node[above] at (f.north) {$2$};
          \node[draw,circle,thick,blue] (d) [above right=of f] {d} node[above] at (d.north) {$8$};
          \node[draw,circle] (c) [below left=of f] {c} node[below] at (c.south) {$8$};
          \node[draw,circle] (e) [below right=of f] {e} node[below] at (e.south) {$6$};

          \draw[<-,thick,blue] (a) -- (b) node[midway, above] {4};
          \draw (a) -- (c) node[midway, left] {8};
          \draw (b) -- (c) node[midway, left] {11};
          \draw (b) -- (f) node[midway, above right] {9};
          \draw[->,thick,blue] (f) -- (d) node[midway, below right] {2};
          \draw (d) -- (e) node[midway, right] {6};
          \draw (e) -- (f) node[midway, above right] {5};
          \draw (c) -- (f) node[midway, above left] {7};
          \draw (c) -- (e) node[midway, below] {1};
          \draw[<-,thick,blue] (b) -- (d) node[midway, above] {8};
        \end{tikzpicture}
        \caption{Esempio di grafo}
      \end{figure}

    \item Quinto passo
      \begin{figure}[H]
        \centering
        \begin{tikzpicture}
          \node[draw,circle,thick,blue,double] (a) {a} node[above] at (a.north) {0};
          \node[draw,circle,thick,blue] (b) [above right=of a] {b} node[above] at (b.north) {$4$};
          \node[draw,circle,thick,blue] (f) [below right=of b] {f} node[above] at (f.north) {$2$};
          \node[draw,circle,thick,blue] (d) [above right=of f] {d} node[above] at (d.north) {$8$};
          \node[draw,circle] (c) [below left=of f] {c} node[below] at (c.south) {$7$};
          \node[draw,circle,thick,blue] (e) [below right=of f] {e} node[below] at (e.south) {$5$};

          \draw[<-,thick,blue] (a) -- (b) node[midway, above] {4};
          \draw (a) -- (c) node[midway, left] {8};
          \draw (b) -- (c) node[midway, left] {11};
          \draw (b) -- (f) node[midway, above right] {9};
          \draw[->,thick,blue] (f) -- (d) node[midway, below right] {2};
          \draw (d) -- (e) node[midway, right] {6};
          \draw[->,thick,blue] (e) -- (f) node[midway, above right] {5};
          \draw (c) -- (f) node[midway, above left] {7};
          \draw (c) -- (e) node[midway, below] {1};
          \draw[<-,thick,blue] (b) -- (d) node[midway, above] {8};
        \end{tikzpicture}
        \caption{Esempio di grafo}
      \end{figure}

    \item Ultimo passo
      \begin{figure}[H]
        \centering
        \begin{tikzpicture}
          \node[draw,circle,thick,blue,double] (a) {a} node[above] at (a.north) {0};
          \node[draw,circle,thick,blue] (b) [above right=of a] {b} node[above] at (b.north) {$4$};
          \node[draw,circle,thick,blue] (f) [below right=of b] {f} node[above] at (f.north) {$2$};
          \node[draw,circle,thick,blue] (d) [above right=of f] {d} node[above] at (d.north) {$8$};
          \node[draw,circle,thick,blue] (c) [below left=of f] {c} node[below] at (c.south) {$1$};
          \node[draw,circle,thick,blue] (e) [below right=of f] {e} node[below] at (e.south) {$5$};

          \draw[<-,thick,blue] (a) -- (b) node[midway, above] {4};
          \draw (a) -- (c) node[midway, left] {8};
          \draw (b) -- (c) node[midway, left] {11};
          \draw (b) -- (f) node[midway, above right] {9};
          \draw[->,thick,blue] (f) -- (d) node[midway, below right] {2};
          \draw (d) -- (e) node[midway, right] {6};
          \draw[->,thick,blue] (e) -- (f) node[midway, above right] {5};
          \draw (c) -- (f) node[midway, above left] {7};
          \draw[<-,thick,blue] (c) -- (e) node[midway, below] {1};
          \draw[<-,thick,blue] (b) -- (d) node[midway, above] {8};
        \end{tikzpicture}
        \caption{Esempio di grafo}
      \end{figure}
  \end{enumerate}
  Anche con questo algoritmo abbiamo trovato un MST.
\end{example}

\vspace{1em}
\noindent
Esiste un algoritmo per risolvere il problema dei cammini minimi che parte dal presupposto
che tutti gli archi non devono essere negativi:
\begin{lstlisting}[language=Scala]
// G e' un grafo
// s e' il nodo di partenza
init(G, s)
  for v in G.V
    v.distance <- +inf
    v.parent <- NIL
  s.distance <- 0

// u e v sono due nodi
// w rappresenta il tempo di percorrenza tra i 2 nodi
// Prende l'arco u-v che costa w e controlla se
// esiste un cammino minimo migliore di quello da s a v
// passando per u
relax(u, v, w)
  if v.distance > u.distance + w(u,v)
    v.distance <- u.distance + w(u,v)
    v.parent <- u

// G e' un grafo
// s e' il nodo di partenza
// w indica la distanza tra i nodi
dijkstra(G,s,w)
  init(G,s)
  Q <- G.V
  while Q != {}
    u <- extract_min(Q)
    for v in G.adj(u)
      relax(u,v,w)
\end{lstlisting}
In questo algoritmo la relax viene effettuata una sola volta per arco perchè ogni
nodo viene estratto dalla coda una sola volta e la lista di adiacenza viene esaminata
soltanto una volta per nodo.

\vspace{1em}
\noindent
La complessità di questo algoritmo è
\begin{itemize}
  \item \( V \) volte la init
  \item \( V \) volte la extract min
  \item \( E \) volte la reduce key, quindi a seconda di come viene
    implementata la coda cambia la complessità dell'algoritmo
    \begin{itemize}
      \item Liste non ordinate: Complessità \( V \) per trovare il nodo minimo
        \begin{itemize}
          \item \( V^2 \) per la extract min, quindi la complessità diventa:
            \[
              V + V^2 + E = O(V^2)
            \] 
        \end{itemize}

      \item Liste ordinate: La extract min è costante, la reduce key è lineare
        \begin{itemize}
          \item \( V \) volte un lavoro costante ed \( E \) volte un lavoro linere:
            \[
              EV
            \] 
        \end{itemize}

      \item Heap: La extract min è logaritmica, la reduce key è costante
        \begin{itemize}
          \item 
            \[
              (V + E) \log V
            \] 
        \end{itemize}
    \end{itemize}
\end{itemize}

\vspace{1em}
\noindent
Un altro algoritmo per trovare i cammini minimi è quello di Bellman-Ford:
\begin{lstlisting}[language=Scala]
bellman_ford(G, w, s)
  init(G,s)
  for i <- 1 to |G.V| - 1 // --
    for (u,v) in G.E      //  | VE
      relax(u,v,w)        // --

  for (u,v) in G.E        // E
    if v.distance > u.distance + w(u,v)
      return false // ciclo negativo
  return true
\end{lstlisting}
Se esistono archi rilassabili la soluzione non è stata trovata, se invece non esistono
archi rilassabili è stata trovata la soluzione. La complessità di questo algoritmo è:
\[
  O(VE)
\] 
\vspace{1em}
\noindent
\textbf{Lemma}: Dopo \( i \) iterazioni di Bellman-Ford, tutti i cammini minimi di
lunghezza al più \( i \) sono stati trovati.

Quindi l'algoritmo di Bellman-Ford trova sempre la soluzione se essa esiste.

\vspace{1em}
\noindent
Consideriamo un grafo che contiene un ciclo negativo.
\[
  u_1 \to u_2 \to u_3 \to \ldots \to u_n \to u_1
\] 
Supponiamo per assurdo che
Bellman-Ford non dia false, sappiamo che per ognuno dei seguenti archi vale la
seguente disequazione:
\[
\begin{aligned}
  u_2.\text{distance} & \le u_1.\text{distance} + w(u_1,u_2) \\
  u_3.\text{distance} & \le u_2.\text{distance} + w(u_2,u_3) \\
  \vdots\\
  u_n.\text{distance} & \le u_{n-1}.\text{distance} + w(u_{n-1},u_n) \\
  u_1.\text{distance} & \le u_{n}.\text{distance} + w(u_{n},u_1) \\
\end{aligned}
\] 
Se sommiamo queste cose si ottiene:
\[
  \sum_{i=1}^n u_i.\text{distance} \le \sum_{i=1}^m u_i + \text{costo\_ciclico}
\] 
\[
  \cancel{\sum_{i=1}^n u_i.\text{distance}} \le \cancel{\sum_{i=1}^m u_i} + \text{costo\_ciclico}
\] 
Quindi otteniamo una contraddizione.

\vspace{1em}
\noindent
Per risolvere il problema dei cammini minimi sui grafi aciclici esiste l'algoritmo
Directed Acyclic Graph Shortest Path:
\begin{lstlisting}[language=Scala]
dag_sp(G, s, w)
  init(G,s) // V
  topological_sort(G) // V + E
  for u in G.V        // -------+ 
    for v in G.adj(u) // -| < E | V = V + E
      relax(u,v,w)    // -------+
\end{lstlisting}
Ogni arco viene rilassato soltanto una volta. La complessità di questo algoritmo è:
\[
  V + (V + E) + (V + E) = 3V + 2E = O(V + E)
\] 

\begin{exercise}
  In una gara vengono consegnati dei dischi con un diametro e una morbidezz.
  L'obiettivo è quello di formare la pila più alta, ma un piatto grande non può essere
  messo sopra un piatto piccolo e un piatto duro non deve essere messo sopra un piatto
  morbido.

  \vspace{1em}
  \noindent
  Il problema si può rappresentare come un grafo in cui i nodi sono i dischi e due
  nodi \( a \) e \( b \) sono connessi se \( a \) può stare sotto a \( b \). Gli
  archi sono orientati e il peso è \( -1 \).

  \vspace{1em}
  \noindent
  Non è definito però il nodo di partenza, quindi bisogna ridurre il problema ad uno
  in cui il nodo di partenza è definito, e di conseguenza bisogna trasformare il grafo.
  Si possono aggiungere due nodi che sono collegati a tutti gli altri nodi con archi
  a peso 0 che funzionano rispettivamente da punto di partenza e da punto di arrivo.
  Questi nodi sono chiamati \textbf{supernodi}.

  \vspace{1em}
  \noindent
  Se esistessero più piatti uguali, allora essi dovrebbero essere messi tutti insieme,
  quidi si aggiunge al peso il numero di piatti uguali messi nella pila. In questo
  modo il cammino di costo minimo risolve il problema.
\end{exercise}

\subsubsection{Cammini minimi con tutte le sorgenti}
Il problema dei cammini minimi con tutte le sorgenti è la risoluzione del problema
dei cammini minimi per ogni nodo del grafo. Si avrà un vettore di parent e un vettore
di distanze che rappresentano la soluzione per ogni sorgente:
\[
  \begin{aligned}
    \pi = \begin{pmatrix} 
      \pi_1\\
      \pi_2\\
      \vdots\\
      \pi_n
    \end{pmatrix} 
    \quad
    D = \begin{pmatrix} 
      D_1\\
      D_2\\
      \vdots\\
      D_n
    \end{pmatrix}
  \end{aligned}
\] 
dove \( \pi \) è la matrice dei parent e \( D \) è la matrice delle distanze.

La matrice \( D \) ha 0 sulla diagonale e infinito altrove:
\[
  \begin{pmatrix} 
    0 & \infty & \ldots & \infty\\
    \infty & 0 & \ldots & \infty\\
    \vdots & \vdots & \ddots & \vdots\\
    \infty & \infty & \ldots & 0
  \end{pmatrix} 
\] 

Per calcolare \( D \) si usa il seguente algoritmo:
\begin{lstlisting}[language=Scala]
// G e' un grafo
// w e' la matrice dei pesi
distance_matrix(G,w)
  D <- init(G)
  for l <- 1 to n-1
    D <- extend_SP(D, w)
\end{lstlisting}
\begin{lstlisting}[language=Scala]
// D e' la matrice delle distanze
// w e' la matrice dei pesi
extend_sp(D, w)
  n <- rows(D)
  for i <- 1 to n // per ogni istanza
    for j <- 1 to n // per ogni sorgente
      D_1[i][j] <- inf
      for k <- 1 to n // per ogni destinazione
        D_1[i][j] <- min(D_1[i][j], D[i][k] + w[k][j])
\end{lstlisting}
Per ogni arco si tenta di "rilassare" l'arco \( i \to j \) passando per \( k \):
\begin{figure}[H]
  \centering
  \begin{tikzpicture}
    \node (i) {i};
    \node (k) [right=of i] {k};
    \node (j) [above right=of i] {j};

    \draw[->] (i) -- (k) node[midway, below] {D[i][k]};
    \draw[->] (k) -- (j) node[midway, right] {w[k][j]};
    \draw[->] (i) -- (j) node[midway, above left] {D\_1[i][j]};
  \end{tikzpicture}
\end{figure}
L'algoritmo che moltiplica due matrici è:
\begin{lstlisting}[language=Scala]
mult(D,w)
  n <- rows(D)
  for i <- 1 to n
    for j <- 1 to n
      D_1[i][j] <- 0
      for k <- 1 to n
        D_1[i][j] <- D_1[i][j] + D[i][k] * w[k][j]
\end{lstlisting}
si nota che la moltiplicazione è molto simile alla extend\_sp, solo che al posto
dell'operazione di sommma si fa un minimo e al posto del prodotto si fa una somma.
Quindi Bellman Ford è una moltiplicazione tra matrici, ma in un algebra diversa:
\[
  D \cdot w
\] 
Nello specifico si sta facendo un prodotto tante volte tenendo in considerazione che
dopo aver trovato la soluzione il risultato non varia più:
\[
  I \cdot w \cdot w \cdots w = w^{n-1} = w^{n} = w^{n+l} \forall l > 0
\] 
Questo algoritmo ha una complessità di: \( \Theta (n^4) \), però si possono sfruttare
tutti quegli algoritmi fatti per migliorare la complessità della moltiplicazione tra
matrici.

Per eseguire moltiplicazioni di seguito si può sfruttare il \textbf{iterative squaring}:
\[
  \begin{aligned}
    w^2 & \leftarrow w \cdot w \\
    w^4 & \leftarrow w^2 \cdot w^2 \\
    w^8 & \leftarrow w^4 \cdot w^4 \\
    \vdots & \\
  \end{aligned}
\] 
e questo permette di calcolare potenze pari in \( log(n) \) operazioni.

\vspace{1em}
\noindent
L'algoritmo di Floyd Warshal calcola gli elementi \( D_{ij}^k \) dove \( k \) è il numero
di nodi intermedi che si possono usare per calcolare il cammino minimo tra \( i \) e \( j \):
\[
  D^0 = w
\] 
La soluzione sarà quindi \( D^n \) 
\begin{lstlisting}[language=Scala]
floyd_warshal(G,w)
  n <- rows(G)
  D^0 <- w
  for k <- 1 to n
    for i <- 1 to n
      for j <- 1 to n
        D^k[i][j] <- min(D^{k-1}[i][j], D^{k-1}[i][k] + D^{k-1}[k][j])
\end{lstlisting}
\begin{figure}[H]
  \centering
  \begin{tikzpicture}
    \node (i) {i};
    \node (k) [right=of i] {k};
    \node (j) [above right=of i] {j};

    \draw[->] (i) -- (k) node[midway, below, scale=0.8] {\( [1 .. k-1] \) };
    \draw[->] (k) -- (j) node[midway, right, scale=0.8] {\( [1 .. k-1] \) };
    \draw[->] (i) -- (j) node[midway, above left, scale=0.8] {\( [1 .. k-1] \) };
  \end{tikzpicture}
\end{figure}

\vspace{1em}
\noindent
Se i pesi degli archi fossero tutti positivi si potrebbe applicare l'algoritmo di
Dijkstra ad ogni singola sorgente, questo algoritmo si chiama algoritmo di Johnson:
\begin{lstlisting}[language=Scala]
johnson(G,w)
  for i <- 1 to n
    D[i] = dijkstra(G, i, w)
\end{lstlisting}
e la complessità è:
\[
  V(V+E) \log(V)
\] 
se il grafo è connesso si ottiene:
\[
  VE \log(V)
\] 

Il problema è che questo algoritmo non si può usare su grafi con archi negativi. Per farlo
si deve trasformare il grafo in modo da non avere archi negativi e mantenere il fatto che
i cammini minimi non cambiano tra il grafo originale e quello trasformato. Se la trasformazione
ha costo maggiore dell'algoritmo da eseguire, non ha senso farla.

I cammini minimi dipendono solo dal punto di partenza e punto di arrivo, quindi Johnson
ha definito la seguente funzione:
\[
  h: V \to R \quad \hat{w}(u,v) = w(u,v) + h(v) - h(u)
\] 
quindi:
\[
  \hat{w}(v_0,v_1, \ldots, v_k) = \sum_{i=0}^{k-1} \hat{w}(v_i,v_{i+1}) =
\] 
\[
\begin{aligned}
  w(v_0,v_1)& + \cancel{h(v_1)} - h(v_0)\\
  w(v_1,v_2)& + \cancel{h(v_2)} - \cancel{h(v_1)}\\
  \vdots\\
  w(v_{k-1},v_k)& + \cancel{h(v_k)} - \cancel{h(v_{k-1})}\\
\end{aligned}
\] 
\[
  = \sum_{i=0}^{k-1} w(v_i,v_{i+1}) + h(v_k) - h(v_0)
\] 
Questa trasformazione produce nuovi pesi in cui i cammini minimi non cambiano. Bisogna ora
trovare la funzione \( h \) che produca pesi che non sono negativi. La disequazione di
Bellman Ford era:
\[
  v.distance \le u.distance + w(u,v)
\] 
e questo diventa:
\[
  w(u,v) + u.distance - v.distance \ge 0
\] 

\subsection{Grafo bipartito}
Un grafo bipartito è un grafo in cui ogni ogni arco unisce due nodi appartenenti a
due insiemi distinti.
\begin{figure}[H]
  \centering
  \begin{tikzpicture}
    \node[draw,circle,blue] (a) {a};
    \node[draw,circle,blue] (b) [below=of a] {b};
    \node[draw,circle,blue] (c) [below=of b] {c};

    \node[draw,circle,red] (e) [right=of a] {e};
    \node[draw,circle,red] (f) [below=of e] {f};
    \node[draw,circle,red] (g) [below=of f] {g};

    \draw (a) -- (e);
    \draw (e) -- (b);
    \draw (e) -- (c);
    \draw (c) -- (f);
    \draw (c) -- (g);
  \end{tikzpicture}
  \caption{Esempio di grafo bipartito}
\end{figure}
Un altro esempio sarebbe un grafo numerato in cui si dividono i numeri pari e quelli dispari.

\vspace{1em}
\noindent
Se c'è una bipartizione si possono colorare diversamente i nodi appartenenti ai due
insiemi. Quindi un grafo è bipartito se e solo se è \textbf{bicolorabile}.

Una volta scelto il colore del primo nodo, il colore di tutti gli altri è vincolato
dalla scelta del primo nodo. Se si incontra un nodo già colorato e il colore è
diverso da quello che si aspetta, il grafo non è bipartito.

\vspace{1em}
\noindent
Un algoritmo per trovare se un grafo è bipartito è il seguente:
\begin{lstlisting}[language=Scala]
is_bipartito(u, color, G)
  // DFS visit
  u.discovered <- gray // scoperto, ma non esplorato
  u.color <- color // colore della bipartizione

  for v in G.adj(u)
    if v.discovered == white // non esplorato
      v.parent <- u
      v.color <- !color
      is_bipartito(v, !color, G)
    else v.color == c
      return false

  u.color <- black // esplorato
  return true
\end{lstlisting}

\subsection{Problema di flusso massimo}
\begin{example}
  Piove e dai monti viene giù molta acqua che finirà a valle. Vogliamo che quest'acqua
  defluisca nel mare passando per una rete di canali. L'obiettivo è sapere quanta acqua
  può passare per i canali senza che questi straripino.

  Si vuole trasferire un flusso materiale da un punto sorgente a un punto di destinazione rendendo
  massimo il flusso.
\end{example}
Questo tipo di problema è rappresentabile con i grafi tramite una \textbf{rete di flusso}.
I nodi rappresentano i punti di partenza e arrivo del flusso, gli archi rappresentano i canali.
Per ogni canale è indicato quante unità di flusso \textbf{netto} passano rispetto al totale.
\begin{figure}[H]
  \centering
  \begin{tikzpicture}[
    node distance=1.5cm,
    every edge quotes/.style = {auto, font=\footnotesize, sloped}
    ]
    \node[draw,circle,minimum size=0.8cm] (s) {s};
    \node[draw,circle,minimum size=0.8cm] (a1) [above right=of s] {a};
    \node[draw,circle,minimum size=0.8cm] (a2) [right=of a1] {b};
    \node[draw,circle,minimum size=0.8cm] (d) [below right=of a2] {d};
    \node[draw,circle,minimum size=0.8cm] (b1) [below right=of s] {c};
    \node[draw,circle,minimum size=0.8cm] (b2) [right=of b1] {e};

    \draw[->] (s) edge["11/16"] (a1);
    \draw[->] (s) edge["8/13",swap] (b1);
    \draw[->] (a1) edge[bend right,"0/10"] (b1);
    \draw[<-] (a1) edge[bend left,"1/4"] (b1);
    \draw[->] (a1) edge["12/12"] (a2);
    \draw[->] (b1) edge["11/14"] (b2);
    \draw[->] (b2) edge["7/7"] (a2);
    \draw[->] (a2) edge["15/20"] (d);
    \draw[->] (b2) edge["4/4",swap] (d);
    \draw[->] (a2) edge["4/9",swap] (b1);
  \end{tikzpicture}
  \caption{Esempio di rete di flusso}
\end{figure}

\subsubsection{Funzione di flusso e le sue proprietà}
La funzione di flusso è:
\[
  f: V \times V \to \mathbb{R}
\] 
e deve soddisfare:
\begin{itemize}
  \item Il flusso di un arco non deve superare la propria capacità:
    \[
      \forall (u,v) \; f(u,v) < c(u,v)
    \]

  \item Anti simmetria, cioè il flusso da \( u \) a \( v \) è uguale al
    negativo del flusso da \( v \) a \( u \):
    \[
      f(u,v) = -f(v,u)
    \] 

  \item Conservazione: ciò che entra è uguale a ciò che esce, non si consuma niente
    \[
      \forall u \in  V - \{s,d\} \; \sum_v f(u,v) = 0
    \] 
    oppure
    \[
      |f| = \sum_v f(s,v) = \sum_u f(u,d)
    \] 
\end{itemize}
\vspace{1em}
\noindent
\begin{define}
  Il \textbf{taglio di un grafo} è una bipartizione (S,T) dell'insieme dei nodi del grafo 
  \[
    S \cup T = V
  \] 
  Ogni arco che va da un nodo in \( S \) a un nodo in \( T \) è detto \textbf{arco di taglio}.
\end{define}

Se assegnamo il nodo sorgente all'insieme \( s \in S \) e il nodo destinazione
all'insieme \( d \in T \) possiamo definire la quantità di liquido che attraversa il
taglio come:
\[
  \sum_{\underset{v \in T}{u \in S}} f(u,v)
\] 
Per ogni taglio che separa \( s \) e \( d \) il flusso che attraversa il taglio è
sempre uguale al modulo del flusso che esce da \( s \): \( |f| \) 

\textbf{Dimostrazione}: 
\begin{itemize}
  \item Caso base: c'è solo il nodo sorgente, quindi il flusso è la somma del flusso
    uscente:
    \[
      |f| = \sum_{v} f(s,v)
    \]
  \item Caso induttivo: si aggiunge un nuovo nodo all'interno del taglio,
    la variazione di flusso per un nuovo arco all'interno del taglio
    è 0 perchè la somma del flusso che entra e che esce dal nuovo arco è 0.
    \[
      |f| = \sum_{v} f(s,v) + 0
    \]
\end{itemize}
Il flusso massimo non può superare la capacità minima del taglio.

\subsubsection{Rete residua}
La rete residua è una rete in cui gli archi hanno una capacità residua, cioè
quanto flusso può ancora passare. La rete residua è definita come:
\[
  \forall u,v \; c_f(u,v) = c(u,v) - f(u,v)
\] 
\begin{figure}[H]
  \centering
  \begin{tikzpicture}[
    node distance=1.5cm,
    every edge quotes/.style = {auto, font=\footnotesize, sloped}
    ]
    \node[draw,circle,minimum size=0.8cm] (s) {s};
    \node[draw,circle,minimum size=0.8cm] (a1) [above right=of s] {a};
    \node[draw,circle,minimum size=0.8cm] (a2) [right=of a1] {b};
    \node[draw,circle,minimum size=0.8cm] (d) [below right=of a2] {d};
    \node[draw,circle,minimum size=0.8cm] (b1) [below right=of s] {c};
    \node[draw,circle,minimum size=0.8cm] (b2) [right=of b1] {e};

    \draw[->] (s) edge[bend left=10,"5"] (a1);
    \draw[<-] (s) edge[bend right=10,"11",swap] (a1);

    \draw[->] (s) edge[bend left=10,"5"] (b1);
    \draw[<-] (s) edge[bend right=10,"8",swap] (b1);

    \draw[->] (a1) edge[bend right=10,"11",swap] (b1);
    \draw[<-] (a1) edge[bend left=10,"3"] (b1);

    \draw[<-] (a1) edge["12"] (a2);

    \draw[->] (b1) edge[bend left=10,"3"] (b2);
    \draw[<-] (b1) edge[bend right=10,"11",swap] (b2);

    \draw[<-] (b2) edge["7"] (a2);
    
    \draw[->] (a2) edge[bend left=10,"5"] (d);
    \draw[<-] (a2) edge[bend right=10,"15",swap] (d);

    \draw[<-] (b2) edge["4",swap] (d);

    \draw[->] (a2) edge[bend left=10,"5",swap] (b1);
    \draw[<-] (a2) edge[bend right=10,"4"] (b1);
  \end{tikzpicture}
  \caption{Esempio di rete residua}
\end{figure}
\noindent
Si dice \textbf{cammino aumentante} un cammino che va da \( s \) a \( d \) in cui
il flusso può essere aumentato, e quindi il residuo diminuito. In questo cammino
si dice \textbf{arco critico} l'arco con capacità residua minima.
\begin{figure}[H]
  \centering
  \begin{tikzpicture}[
    node distance=1.5cm,
    every edge quotes/.style = {auto, font=\footnotesize, sloped}
    ]
    \node[draw,circle,minimum size=0.8cm] (s) {s};
    \node[draw,circle,minimum size=0.8cm] (a1) [above right=of s] {a};
    \node[draw,circle,minimum size=0.8cm] (a2) [right=of a1] {b};
    \node[draw,circle,minimum size=0.8cm] (d) [below right=of a2] {d};
    \node[draw,circle,minimum size=0.8cm] (b1) [below right=of s] {c};
    \node[draw,circle,minimum size=0.8cm] (b2) [right=of b1] {e};

    \draw[->] (s) edge[bend left=10,"5"] (a1);
    \draw[<-] (s) edge[bend right=10,"11",swap] (a1);

    \draw[->] (s) edge[bend left=10,"5"] (b1);
    \draw[<-] (s) edge[bend right=10,"8",swap] (b1);
    \draw[->,blue,thick] (s) edge[] (b1);

    \draw[->] (a1) edge[bend right=10,"11",swap] (b1);
    \draw[<-] (a1) edge[bend left=10,"3"] (b1);

    \draw[<-] (a1) edge["12"] (a2);

    \draw[->] (b1) edge[bend left=10,"3"] (b2);
    \draw[<-] (b1) edge[bend right=10,"11",swap] (b2);

    \draw[<-] (b2) edge["7"] (a2);
    
    \draw[->] (a2) edge[bend left=10,"5"] (d);
    \draw[<-] (a2) edge[bend right=10,"15",swap] (d);
    \draw[->,blue,thick] (a2) edge[] (d);

    \draw[<-] (b2) edge["4",swap] (d);

    \draw[->] (a2) edge[bend left=10,"5",swap] (b1);
    \draw[<-] (a2) edge[bend right=10,"4"] (b1);
    \draw[->,blue,thick] (b1) edge[] (a2);
  \end{tikzpicture}
  \caption{Esempio di cammino aumentante}
\end{figure}
\noindent
Quando non si trovano più cammini aumentanti si dice che il flusso è massimo.
Esiste l'algoritmo di Ford-Fulkerson che trova il flusso massimo in un grafo iterando
al massimo \( |f^*| \), cioè il flusso ottimo. La complessità di questo algoritmo
è \( O(|f^*| \cdot |V + E|) \) e si basa sull'idea di trovare un cammino
aumentante e aumentare di una unità il flusso lungo quel cammino fino a quando non
si trovano più cammini aumentanti.







\end{document}
