\documentclass[a4paper]{article}
\usepackage{import}
\usepackage[utf8]{inputenc}
\usepackage[T1]{fontenc}
\usepackage{textcomp}
\usepackage[italian]{babel}
\usepackage{amsmath, amssymb}
\usepackage{booktabs,xltabular}
\usepackage{amsfonts}
\usepackage{amsthm}
\usepackage{cancel}
\usepackage{mdframed}
\usepackage{makecell}
\usepackage{float}
\usepackage{xcolor}
\usepackage{listings}
\usepackage{gensymb}
\usepackage{graphicx}
\usepackage{bodeplot}
\usepackage{tikz}
\usetikzlibrary{shapes, arrows, automata, petri, decorations.markings, decorations.pathreplacing, positioning, calc}
\usepackage{circuitikz}
\usepackage[label=corner]{karnaugh-map}
\graphicspath{{./figures/}}

% Set default font to sans-serif
\renewcommand{\familydefault}{\sfdefault} 
\usepackage{eulervm}

\usepackage{forest}

\usepackage{mathtools}
\DeclarePairedDelimiter\ceil{\lceil}{\rceil}
\DeclarePairedDelimiter\floor{\lfloor}{\rfloor}

% \usepackage{ntheorem}

\usepackage{import}
\usepackage{pdfpages}
\usepackage{transparent}
\usepackage{xcolor}

\usepackage{hyperref}
\hypersetup{
    colorlinks=false,
}

% Code blocks
\definecolor{codegreen}{rgb}{0,0.6,0}
\definecolor{codegray}{rgb}{0.5,0.5,0.5}
\definecolor{codepurple}{rgb}{0.58,0,0.82}
\definecolor{backcolour}{rgb}{0.95,0.95,0.95}

\lstdefinestyle{mystyle}{
	backgroundcolor=\color{backcolour},
	commentstyle=\color{codegreen},
	keywordstyle=\color{magenta},
	numberstyle=\tiny\color{codegray},
	stringstyle=\color{codepurple},
	basicstyle=\ttfamily\footnotesize,
	breakatwhitespace=false,
	breaklines=true,
	captionpos=b,
	keepspaces=true,
	numbers=left,
	numbersep=5pt,
	showspaces=false,
	showstringspaces=false,
	showtabs=false,
	tabsize=2
}

\lstset{style=mystyle}

\usepackage{color}
\usepackage{import}
\usepackage{pdfpages}
\usepackage{transparent}
\usepackage{xcolor}

% Example frame
\theoremstyle{definition}
\newmdtheoremenv[%
	linecolor=gray,leftmargin=0,%
	rightmargin=0,
	innertopmargin=8pt,%
	innerbottommargin=8pt,
	ntheorem]{example}{Esempio}[section]

% Important definition frame
\theoremstyle{definition}
\newmdtheoremenv[%
	linecolor=gray,leftmargin=0,%
	rightmargin=0,
	backgroundcolor=gray!40,%
	innertopmargin=8pt,%
	innerbottommargin=8pt,
	ntheorem]{definition}{Definizione}[section]

% Exercise frame
\theoremstyle{definition}
\newmdtheoremenv[%
	linecolor=gray,leftmargin=0,%
	rightmargin=0,
	innertopmargin=8pt,%
	innerbottommargin=8pt,
	ntheorem]{exercise}{Esercizio}[section]

% Theorem frame
\theoremstyle{definition}
\newmdtheoremenv[%
  linecolor=gray,leftmargin=0,%
  rightmargin=0,
  innertopmargin=8pt,%
  innerbottommargin=8pt,
  ntheorem]{theorem}{Teorema}[section]

\theoremstyle{definition}
\newmdtheoremenv[%
  linecolor=gray,leftmargin=0,%
  rightmargin=0,
  innertopmargin=8pt,%
  innerbottommargin=8pt,
  ntheorem]{define}{Definizione utile}[section]

% figure support
\usepackage{import}
\usepackage{xifthen}
\pdfminorversion=7
\usepackage{pdfpages}
\usepackage{transparent}
\newcommand{\incfig}[1]{%
	\def\svgwidth{\columnwidth}
	\import{./figures/}{#1.pdf_tex}
}

% FSM tikz
\tikzset{
    place/.style={
        circle,
        thick,
        draw=black,
        minimum size=6mm,
    },
        state/.style={
        circle,
        thick,
        draw=blue!75,
        fill=blue!20,
        minimum size=6mm,
    },
}

\usepackage{pgfplots}
\pgfplotsset{compat=1.18}

\pdfsuppresswarningpagegroup=1


% Info: 
% Libro: Fisica: Elettromagnetismo e Onde
% Esame: Scritto di 2 ore e orale facoltativo per aumentare il voto
\begin{document}
\begin{titlepage}
	\begin{center}
		\vspace*{1cm}

		\Huge
		\textbf{Analisi 1}

		\vspace{0.5cm}
		\LARGE
		UniVR - Dipartimento di Informatica

		\vspace{1.5cm}

		\textbf{Fabio Irimie}

		\vfill


		\vspace{0.8cm}

    Corso di Giacomo Canevari

		1° Semestre 2023/2024

	\end{center}
\end{titlepage}


\tableofcontents
\pagebreak

\section{Introduzione}
L'oggetto dello studio prinicpale di questo costo è la \textbf{forza elettromagnetica}
\( \vec{F}_{em} \), più precisamente la \textbf{teoria di campo}. 

\begin{define}
  La forza è l'interazione tra due oggetti.
\end{define}

\noindent
In natura esistono solo 4 forze che governano tutto ciò che è
misurabile in natura:
\begin{itemize}
  \item Forza di gravità: osservata quando negli oggetti interagenti c'è massa
  \item Forza elettromagnetica
  \item Forza elettronucleare forte
  \item Forza elettronucleare debole
\end{itemize}
Le ultime due riguardano la materia microscopica. Le prime due invece riguardano la
materia macroscopica e sono forze \textbf{a lungo raggio}, cioè ha effetto anche
a distanza.

Lo studio della forza elettromagnetica si può studiare con attraverso degli strumenti
che approssimano il comportamento delle entità a livello macroscopico senza preoccuparci
della natura microscopica.

\subsection{Campo e forza}
In fisica 1 si sono studiati i concetti delle forze, cioè ciò che agisce su un corpo con
una massa, ad esempio la caduta di un grave che è attratto dalla Terra per la forza di
gravità. La visione dei campi è una visione più generale e rappresenta la proprietà
di un ambiente di interagire con un corpo, ad esempio un \textbf{campo} di gravità.

\section{Forza elettrostatica}
Facendo esperimenti che non sono analizzabili con i concetti della fisica 1 si arriva
a capire che c'è una nuova interazione, la \textbf{forza elettrostatica} che ha 2 forme:
\begin{itemize}
  \item Forza attrattiva
  \item Forza repulsiva
\end{itemize}
Gli oggetti sono divisi in due classi:
\begin{itemize}
  \item Carica positiva
  \item Carica negativa
\end{itemize}
Gli oggetti della stessa classe si respingono, mentre quelli di classe diversa si respingono.
\label{05-03-2025-D1}
\begin{definition}[Carica elettrica]
  È chiamata \textbf{carica elettrica} \( q \) la proprietà che ha il corpo di esprimere
  questa forza. Le proprietà di questa carica elettrica sono \textbf{indipendenti} dal
  meccanismo che l'ha generata, cioè può essere generata in modo diverso, ma ha sempre le
  stesse proprietà. Questo implica che la carica è \textbf{preesistente} in natura.
\end{definition}

\subsection{Materia}
L'atomo è formato da un nucleo centrale formato da protoni, carichi positivamente, e da
neutroni, senza carica. Intorno al nucleo si ha una regione in cui si ha la probabilità
di trovare un'altra particella, carica negativa, chiamata elettrone.
\label{05-03-2025-D2}
\noindent
Il nucleo ha dimensione \( \approx 10^{-15}m \) e l'atomo ha dimensione
\( \approx 10^{-10}m \). La carica totale dell'atomo è nulla, quindi è \textbf{neutro} e
quindi la carica del nucleo è uguale alla carica degli elettroni, per la precisione
il numero di protoni è uguale al numero di elettroni. \( Z \) è il numero atomico, cioè
il numero di protoni.

Elettrone e protone hanno, in modulo, la stessa carica:
\[
  |q_{e^{-}}| = q_{p^{+}}
\] 
L'elettrone è una \textbf{particella elementare}, indivisibile e la sua carica è detta
\textbf{carica elementare}, cioè la più piccola unità di carica osservabile e vale:
\[
  e^- = 1.6 \times 10^{-19}C
\] 
La \textbf{carica elettrica} in natura è quindi \textbf{quantizzata}, ovvero deve
essere un multiplo della carica dell'elettrone. Inoltre la carica non si può generare,
si può \textbf{solo trasferire}.

\begin{definition}[Legge di conservazione della carica]
  In un sistema isolato, cioè non interagisce con altri sistemi, la carica totale 
  \( Q \) si conserva.
\end{definition}

\noindent
I componenti della materia hanno due comportamenti:
\begin{itemize}
  \item \textbf{Conduttore}: ad esempio il metallo, in cui gli elettroni sono liberi di
    muoversi
  \item \textbf{Dielettrico} (isolante): ad esempio il vetro, in cui le cariche non sono
    libere di muoversi, quindi vincolate, cioè non si riesce a strappare gli elettroni
    dall'atomo. Se si avvicina una carica positiva al dielettrico si avrà una deformazione
    delle cariche, ma non si ha una separazione di carica:
    \label{05-03-2025-D9}
\end{itemize}

\subsection{Elettrificazione}
L'elettrificazione è il trasferimento di carica da un corpo all'altro. Ci sono 3 
meccanismi di elettrificazione:
\begin{itemize}
  \item \textbf{Strofinio}
    Si prende una bacchetta di vetro e un panno di lana e si strofina la bacchetta.
    La bacchetta non è carica e meccanicamente con lo strofinia si strappa meccanicamente
    dagli atomi gli elettroni. La bacchetta diventa carica positivamente e il panno
    negativamente. Si avranno quindi le cariche \( q^+ \) della bacchetta e \( q^- \)
    del panno. Per la legge di conservazione della carica si ha:
    \[
      |q^-| = q^+
    \] 
    \label{05-03-2025-D3}
  \item \textbf{Induzione elettrostatica}
    Con la precedente bacchetta caricata positivamente si avvicina un oggetto metallico e
    si nota che le cariche negative \( -Q \)  del metallo si avvicinano il più possibile 
    alla bacchetta respingendo le cariche positive \( +Q \)  creando una 
    \textbf{separazione di carica per induzione}. La carica totale rimane nulla perchè 
    non sono migrati elettroni.
    \[
      |-Q| = +Q
    \] 
    \label{05-03-2025-D4}
    Se si allontana l'oggetto metallica si avrà una separazione meno potente.

    \vspace{1em}
    \noindent
    L'\textbf{elettroscopio} si usa per misurare la carica elettrica. È un oggetto metallico
    collegato a delle lamelle metalliche chiamate foglie:
    \label{05-03-2025-D5}

    Si misura la carica avvicinando la bacchetta e si osserva la forza repulsiva tra le
    foglie:
    \label{05-03-2025-D6}
    Se si allontana la bacchetta la separazione delle foglie diminuisce.
  \item \textbf{Contatto}
    Se prendiamo un oggetto metallico caricato positivamente e si mette a
    contatto con un filo conduttore si elettrifica il filo:
    \label{05-03-2025-D7}
    Se si attacca il filo a terra si scarica l'oggetto perchè le cariche migrano verso
    la terra, cioè un conduttore immensamente più grande e quindi la carica si distribuisce
    su tutta la superficie della terra e sull'oggetto metallico rimane una carica
    \textbf{approssimativamente nulla}.
    \label{05-03-2025-D8}
\end{itemize}

\subsection{Elettrostatica nel vuoto}
\textbf{Fatti sperimentali}:

\vspace{1em}
\noindent
Si crea un esperimento che permette di osservare il fenomeno che si vuole modellare.
Si prende una bilancia di torsione formata da un filo torcente a cui è appesa un'asta 
con una carica \( q^+_1 \) su un'estremità. Se si avvicina una carica dello stesso
segno \( q^-_2 \) si osserva che viene applicata una forza repulsiva \( F \) che fa
torcere il filo con un momento torcente
\[
  \tau_{\text{filo}} = (k \theta) = \tau_{\text{el}} = \vec{d} \times \vec{F}
\]
\label{05-03-2025-D10}
\subsubsection{Interazione di Coulomb}
Dai fatti sperimentali si nota che il modulo della forza è proporzionale al prodotto
delle cariche e inversamente proporzionale al quadrato della distanza tra le cariche:
\[
  | F_{\text{el}} | = k \frac{q_1 q_2}{r^2}
\] 
Si osserva anche che la forza elettrica \( F_{\text{el}} \) è una forza \textbf{centrale},
cioè la forza è diretta lungo la retta che congiunge le due cariche.

\( k \) è la costante di Coulomb e vale:
\[
  k = \frac{1}{4 \pi \varepsilon_0}
\]
dove \( \varepsilon_0 \) è la costante dielettrica del vuoto.
L'unità di misura della carica è il Coulomb:
\[
  [q] = C
\] 

\vspace{1em}
\noindent
Consideriamo la terna cartesiana con due cariche positive \( q^+_1 \) e \( q^+_2 \)
descritte dai raggi vettori \( \vec{r}_1 \) e \( \vec{r}_2 \). Sulla carica \( q^+_2 \) 
viene applicata una forza \( \vec{F}_{12} \) 
\label{05-03-2025-D11}

\noindent
Notazione: 
\begin{itemize}
  \item 
    Chiamo il vettore che va da \( \vec{r}_1 \) a \( \vec{r}_2 \) \( \vec{r}_{12} \).

  \item 
    Il versore è indicato con \( \hat{r} \) e rappresenta il vettore unitario:
    \[
      \hat{r} = \frac{\vec{r}}{|\vec{r}|}
    \] 
\end{itemize}


\vspace{1em}
\noindent
Calcoliamo la forza \( \vec{F}_{12} \) che agisce su \( q^+_2 \) da \( q^+_1 \):
\[
  \vec{F}_{12} = \frac{1}{4 \pi \varepsilon_0} \frac{q_1 q_2}{\left( \vec{r}_{2}
    - \vec{r}_1 \right)^2} \frac{\left( \vec{r}_{2} - \vec{r}_1 \right)}{|\vec{r}_{2}
  - \vec{r}_1|} = 
  \frac{1}{4 \pi \varepsilon_0} \cdot \frac{q_1 q_2}{r_{12}^2} \cdot \hat{r}_{12} \quad [N]
\] 

\subsubsection{Sistema di più cariche}
Con più cariche si osserva che vale il principio di sovrapposizione, cioè due fenomeni
si sommano in modo lineare; e vale la terza legge di Newton, cioè l'azione-reazione
\( \left( \vec{F}_{12} = - \vec{F}_{21} \right)  \).

Consideriamo un sistema discreto con \( n \) cariche \( q_1, q_2, \ldots, q_n \) e osserviamo
la carica \( q_0 \). Ognuna di queste cariche sarà descritta dal suo raggio vettore.
\label{05-03-2025-D12}
La forza che la carica \( q_i \) agisce su \( q_0 \) è:
\[
  \vec{F}_{i0} = \frac{q_i q_0}{4 \pi \varepsilon_0} \cdot \frac{\hat{r}_{i0}}{r_{i0}^2}
\] 
dove \( \vec{r}_{i0} = \vec{r}_0 - \vec{r}_i \).

Applichiamo questa formula osservando una ad una tutte le cariche come fatto per \( q_0 \)
per calcolare la forza totale applicata sulla carica \( q_0 \):
\[
  \vec{F}_{\text{tot}} = \sum_{i=1}^{n} \frac{q_i q_0}{4 \pi \varepsilon_0} \cdot 
  \frac{\hat{r}_{i0}}{r_{i0}^2} \quad \left[ N \right]
\] 
Questa forza ha direzione uguale alla somma delle forze.


\end{document}
