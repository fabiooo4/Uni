\documentclass[a4paper]{article}
\usepackage{import}
\usepackage[utf8]{inputenc}
\usepackage[T1]{fontenc}
\usepackage{textcomp}
\usepackage[italian]{babel}
\usepackage{amsmath, amssymb}
\usepackage{booktabs,xltabular}
\usepackage{amsfonts}
\usepackage{subcaption}
\usepackage{amsthm}
\usepackage{cancel}
\usepackage{mdframed}
\usepackage{makecell}
\usepackage{float}
\usepackage{xcolor}
\usepackage{listings}
\usepackage{gensymb}
\usepackage{graphicx}
\usepackage{bodeplot}
\usepackage{physics}
\usepackage{tikz}
\usetikzlibrary{shapes, arrows, automata, petri, decorations.markings, decorations.pathreplacing, positioning, calc, quotes}
\usepackage{circuitikz}
\usepackage[label=corner]{karnaugh-map}
\graphicspath{{./figures/}}

% Set default font to sans-serif
\renewcommand{\familydefault}{\sfdefault} 
\usepackage{eulervm}

\usepackage{forest}

\usepackage{mathtools}
\DeclarePairedDelimiter\ceil{\lceil}{\rceil}
\DeclarePairedDelimiter\floor{\lfloor}{\rfloor}

% \usepackage{ntheorem}

\usepackage{import}
\usepackage{pdfpages}
\usepackage{transparent}
\usepackage{xcolor}

\usepackage{hyperref}
\hypersetup{
    colorlinks=false,
}

% Code blocks
\definecolor{codegreen}{rgb}{0,0.6,0}
\definecolor{codegray}{rgb}{0.5,0.5,0.5}
\definecolor{codepurple}{rgb}{0.58,0,0.82}
\definecolor{backcolour}{rgb}{0.95,0.95,0.95}

\lstdefinestyle{mystyle}{
	backgroundcolor=\color{backcolour},
	commentstyle=\color{codegreen},
	keywordstyle=\color{magenta},
	numberstyle=\tiny\color{codegray},
	stringstyle=\color{codepurple},
	basicstyle=\ttfamily\footnotesize,
	breakatwhitespace=false,
	breaklines=true,
	captionpos=b,
	keepspaces=true,
	numbers=left,
	numbersep=5pt,
	showspaces=false,
	showstringspaces=false,
	showtabs=false,
	tabsize=2
}

\lstset{style=mystyle}

\usepackage{color}
\usepackage{import}
\usepackage{pdfpages}
\usepackage{transparent}
\usepackage{xcolor}

% Example frame
\theoremstyle{definition}
\newmdtheoremenv[%
	linecolor=gray,leftmargin=0,%
	rightmargin=0,
	innertopmargin=8pt,%
	innerbottommargin=8pt,
	ntheorem]{example}{Esempio}[section]

% Important definition frame
\theoremstyle{definition}
\newmdtheoremenv[%
	linecolor=gray,leftmargin=0,%
	rightmargin=0,
	backgroundcolor=gray!40,%
	innertopmargin=8pt,%
	innerbottommargin=8pt,
	ntheorem]{definition}{Definizione}[section]

% Exercise frame
\theoremstyle{definition}
\newmdtheoremenv[%
	linecolor=gray,leftmargin=0,%
	rightmargin=0,
	innertopmargin=8pt,%
	innerbottommargin=8pt,
	ntheorem]{exercise}{Esercizio}[section]

% Theorem frame
\theoremstyle{definition}
\newmdtheoremenv[%
  linecolor=gray,leftmargin=0,%
  rightmargin=0,
  innertopmargin=8pt,%
  innerbottommargin=8pt,
  ntheorem]{theorem}{Teorema}[section]

\theoremstyle{definition}
\newmdtheoremenv[%
  linecolor=white,leftmargin=0,%
  rightmargin=0,
  innertopmargin=8pt,%
  innerbottommargin=8pt,
  ntheorem]{define}{Definizione utile}[section]

% figure support
\usepackage{import}
\usepackage{xifthen}
\pdfminorversion=7
\usepackage{pdfpages}
\usepackage{transparent}
\newcommand{\incfig}[1]{%
	\def\svgwidth{\columnwidth}
	\import{./figures/}{#1.pdf_tex}
}

% FSM tikz
\tikzset{
    place/.style={
        circle,
        thick,
        draw=black,
        minimum size=6mm,
    },
        state/.style={
        circle,
        thick,
        draw=black,
        fill=white,
        minimum size=6mm,
    },
}

\pdfsuppresswarningpagegroup=1

\usepackage{pgfplots}
\pgfplotsset{compat=1.18,width=10cm}

% Save plots as pdf and reuse them without compiling every time
\usetikzlibrary{external}
\tikzexternalize[prefix=figures/tikz/, optimize=false]


\begin{document}

\begin{titlepage}
	\begin{center}
		\vspace*{1cm}

		\Huge
		\textbf{Probabilità e Statistica\\Esercizi}

		\vspace{0.5cm}
		\LARGE
		UniVR - Dipartimento di Informatica

		\vspace{1.5cm}

		\textbf{Fabio Irimie}

		\vfill


		\vspace{0.8cm}


		2° Semestre 2023/2024

	\end{center}
\end{titlepage}


\tableofcontents
\pagebreak

% Info:
% Libri: Sommerdille - Ingegneria del Software
%        Larman - Applicare UML e i Pattern
% Esame: Il risultato finale è la media dei 2 voti tra Ingegneria
%        e programmazione 2.
%        L'esame consiste in un esame orale concentrato su un
%        progetto da discutere, più alcune domande sulla teoria.
%        Il progetto è da scegliere fra 3 proposti dal docente.
%        Il progetto riguarda il progetto e lo sviluppo di un
%        software, bisogna anche implementare un sistema di
%        memorizzazione sulla memoria di massa (fare anche i test). 
%        Il progetto deve essere svolto in gruppi da 2 o 3 persone.
\section{Introduzione}
\subsection{Cos'è l'ingegneria del software?}
L'ingegneria del software è un insieme di metodologie, teorie, metodi e strumenti, che
guidano nello sviluppo di software \textbf{professionale} in modo che esso fornisca
le funzionalità richieste, sia performante e mantenibile, affidabile e usabile. Il
\textbf{software} non è solo il programma e l'eseguibile, ma anche \textbf{la 
documentazione associata}. Le attività principale dell'ingegneria del software sono:
\begin{itemize}
  \item \textbf{Specifica del software}: Il cliente e l'ingegnere del software definiscono
  le funzionalità e i vincoli del software da produrre.
  \item \textbf{Sviluppo}: Il software viene progettato e implementato.
  \item \textbf{Validazione}: Il software viene verificato per
    assicurarsi che soddisfi i requisiti forniti dal cliente.
  \item \textbf{Evoluzione}: Il software viene modificato per adattarlo a nuovi
    requisiti del cliente o del mercato. 
\end{itemize}

\subsubsection{Economia}
Il software spesso costa più dell'hardware, e il costo è più legato alla manutenzione che
allo sviluppo. Quindi l'ingegneria del software è importante per ridurre i costi di
manutenzione.

\subsubsection{Fallimento dei progetti}
I progetti spesso falliscono per la \textbf{crescente complessità del sistema}: nuove 
tecniche di sviluppo e nuove tecnologie rendono i sistemi sempre più complessi, e quindi 
più difficili da mantenere.

\subsection{Prodotti software}
\subsubsection{Prodotti generici}
Sono prodotti che vengono pubblicizzati e venduti a qualsiasi cliente che ne faccia 
richiesta. La specifica è di proprietà del produttore e le decisioni sulle modifiche
sono prese dai produttori.

\subsubsection{Prodotti personalizzati}
Sono prodotti che vengono commissionati da clienti specifici per soddisfare le loro
necessità. La specifica è di proprietà del cliente dopo una contrattazione con il
produttore e le decisioni sulle modifiche sono prese dal cliente.

\subsection{Caratteristiche di un buon software}
\begin{itemize}
  \item \textbf{Mantenibilità}: la facilità con cui il software può essere modificato
  per correggere difetti, migliorare le prestazioni o adattarlo a cambiamenti.

  \item \textbf{Affidabilità e sicurezza}: la capacità del software di svolgere le sue
  funzioni in modo corretto. I malfunzionamenti non devono causare danni fisici o economici.
  Utenti malintenzionati non devono poter violare la sicurezza del sistema.
  
  \item \textbf{Efficienza}: il software non deve sprecare risorse (CPU, memoria, ecc).

  \item \textbf{Accettabilità}: il software deve essere accettato dagli utenti per i quali
  è stato progettato.
\end{itemize}

\subsection{Problemi che influenzano il software}
\begin{itemize}
  \item \textbf{Eterogeneità}:
    I sistemi devono sempre di più operare in modo distribuito, e quindi devono essere in grado
    di comunicare con sistemi diversi. Oppure bisogna garantire che il software funzioni su
    piattaforme diverse.

  \item 
    \textbf{Cambiamento sociale o del business}:
    Il software deve essere in grado di adattarsi a cambiamenti sociali o organizzativi
    nelle aziende.

  \item 
    \textbf{Sicurezza e fiducia}:
    Siccome il software fa parte della vita di tutti i giorni è essenziale che ci sia
    fiducia nel software.

  \item 
    \textbf{Scalabilità}:
    Il software deve essere sviluppato a più scale, cioè soluzione che funziona in piccolo
    deve adattarsi anche a grandi scale senza rischiare di fallire.
\end{itemize}

\subsection{Tipi di applicazione}
Ci sono diversi tipi di sistemi software e non c'è un insieme universale di tecniche
applicabili a tutti i sistemi. Quindi i metodi e gli strumenti utilizzati dipendono
dal tipo di applicazione che si deve sviluppare, dalle richieste dei clienti
e dalle competenze degli sviluppatori. I principali tipi di applicazione sono:
\begin{itemize}
  \item 
    \textbf{Applicazioni stand-alone}:
    Sono applicazioni che si eseguono su un singolo computer locale e includono le
    funzionalità necessarie per l'utente.

  \item 
    \textbf{Applicazioni interattive transaction-based}:
    Sono applicazioni che sono eseguite su un computer in remoto e sono accessibili
    dagli utenti dai propri computer. Queste includono le applicazioni web come
    gli e-commerce.

  \item 
    \textbf{Sistemi embedded}:
    Sono applicazioni che controllano e gestiscono dispositivi hardware. Questi sistemi
    sono i più numerosi.

  \item 
    \textbf{Sistemi batch}:
    Questi sistemi elaborano grandi quantità di dati per produrre un output.

  \item 
    \textbf{Sistemi di intrattenimento}:
    Questi sistemi sono per uso personale e servono ad intrattenere l'utente.

  \item 
    \textbf{Sistemi di modellazione e simulazione}:
    Questi sistemi sono sviluppati da scienziati per modallare processi fisici o situazioni
    che includono tanti oggetti separati che interagiscono tra di loro.

  \item 
    \textbf{Sistemi di collezione di dati (o IOT)}:
    Questi sitemi raccolgono dati da sensori e li inviano ad un sistema centrale per
    l'elaborazione.

  \item 
    \textbf{Sistemi di sistemi}:
    Questi sistemi sono composti da più sistemi software che collaborano tra di loro.
\end{itemize}



\subsection{Principi fondamentali}
\begin{itemize}
  \item I sistemi devono essere sviluppati utilizzando un processo strutturato e ben
    pensato.
  \item L'affidabilità e la performance sono importanti per ogni tipo di software.
  \item Capire e gestire i requisiti del software è essenziale.
  \item Dove appropriato si dovrebbe riutilizzare software che è già stato sviluppato
    al posto di svilupparne uno da zero.
\end{itemize}

\section{Processi del software}
I \textbf{processi} del software sono un insieme di attività strutturate che sono necessarie
per sviluppare un sistema software. Ci sono diversi tipi di processi, ma tutti coinvolgono
le seguenti attività:
\begin{itemize}
  \item \textbf{Specificazione}
  \item \textbf{Sviluppo}
  \item \textbf{Design e implementazione}
  \item \textbf{Validazione}
  \item \textbf{Evoluzione}
\end{itemize}
Un modello di processo software è una rappresentazione astratta di un processo da un punto
di vista particolare.

Per descrivere e discutere i processi di solito si parla delle attività all'interno di
questi processi e dell'ordine in cui queste attività vengono svolte. Alcune descrizioni
possono anche includere:
\begin{itemize}
  \item \textbf{Prodotti}: cioè i risultati di un'attività.
  \item \textbf{Ruoli}: cioè le responsabilità di un individuo o di un gruppo.
  \item \textbf{Pre e post condizioni}: cioè le condizioni che devono essere soddisfatte
    prima e dopo un'attività.
\end{itemize}
Alcuni esempi di processi sono:
\begin{itemize}
  \item \textbf{Processi plan-driven}: sono processi dove tutte le attività sono pianificate
    in anticipo e tutti i progressi sono misurati rispetto al piano.
  \item \textbf{Processi agili}: sono processi in cui la pianificazione è incrementale ed
    è più facile adattarsi ai cambiamenti.
\end{itemize}
Nel pratico si utilizza un approccio ibrido tra i due.

\subsection{Modelli di processo}
\begin{itemize}
  \item \textbf{Modello a cascata}: è un modello plan-driven con fasi di specificazione e
    di sviluppo distinte e separate
  \item \textbf{Modello incrementale}: la specifica, lo sviluppo e la validazione sono
    intercalate. Può essere plan-driven o agile.
  \item \textbf{Integrazione e configurazione}: i sistemi sono sviluppati da componenti
    già esistenti e configurabili.
\end{itemize}
Nel pratico, sistemi molto grandi sono sviluppati utilizzando un processo che incorpora
elementi di tutti e tre i modelli.

\subsubsection{Modello a cascata}
\begin{figure}[H]
  \centering
  \begin{tikzpicture}[
    node distance=0.4cm and 0cm,
    every node/.style={draw,rectangle,align=center,scale=0.8,rounded corners=1mm},
    ]
    \node (1) {Definizione dei\\requisiti};
    \node[below right=of 1]
      (2) {Design del sistema\\e del software};
    \node[below right=of 2]
      (3) {Implementazione e\\unit testing};
    \node[below right=of 3]
      (4) {Integrazione e\\testing del sistema};
    \node[below right=of 4]
      (5) {Operazione e\\manutenzione};

    \draw[->] (1) -| (2);
    \draw[->] (2) -| (3);
    \draw[->] (3) -| (4);
    \draw[->] (4) -| (5);

    \draw (5) -- (1.south |- 52,52 |- 5.west);
    \draw[<-] (4.south) -- (4.south |- 52,52 |- 5.west);
    \draw[<-] (3.south) -- (3.south |- 52,52 |- 5.west);
    \draw[<-] (2.south) -- (2.south |- 52,52 |- 5.west);
    \draw[<-] (1.south) -- (1.south |- 52,52 |- 5.west);
  \end{tikzpicture}
  \caption{Modello a cascata}
\end{figure}

\end{document}
