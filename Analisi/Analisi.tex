\documentclass[a4paper]{article}

\usepackage[utf8]{inputenc}
\usepackage[T1]{fontenc}
\usepackage{textcomp}
\usepackage[italian]{babel}
\usepackage{amsmath, amssymb}
\usepackage[makeroom]{cancel}
\usepackage{amsfonts}
\usepackage{mdframed}
\usepackage{xcolor}
\usepackage{float}
\usepackage{tikz}
\usepackage{pgfplots}
\usetikzlibrary{pgfplots.fillbetween}
\pgfplotsset{compat=newest, ticks=none}
\usepackage{graphicx}
\graphicspath{{./figures/}}

\pgfdeclarelayer{ft}
\pgfdeclarelayer{bg}
\pgfsetlayers{bg,main,ft}

\usepackage{import}
\usepackage{pdfpages}
\usepackage{transparent}
\usepackage{xcolor}

\usepackage{ntheorem}
\newtheorem{theorem}{Teorema}

% Useful definitions frame
\theoremstyle{break}
\theoremheaderfont{\bfseries}
\newmdtheoremenv[%
	linecolor=gray,leftmargin=0,%
	rightmargin=0,
	innertopmargin=8pt,%
	ntheorem]{define}{Definizioni utili}[section]

% Example frame
\theoremstyle{break}
\theoremheaderfont{\bfseries}
\newmdtheoremenv[%
	linecolor=gray,leftmargin=0,%
	rightmargin=0,
	innertopmargin=8pt,%
	ntheorem]{example}{Esempio}[section]

% Important definition frame
\theoremstyle{break}
\theoremheaderfont{\bfseries}
\newmdtheoremenv[%
	linecolor=gray,leftmargin=0,%
	rightmargin=0,
	backgroundcolor=gray!40,%
	innertopmargin=8pt,%
	ntheorem]{definition}{Definizione}[section]

% Exercise frame
\theoremstyle{break}
\theoremheaderfont{\bfseries}
\newmdtheoremenv[%
	linecolor=gray,leftmargin=0,%
	rightmargin=0,
	innertopmargin=8pt,%
	ntheorem]{exercise}{Esercizio}[section]


% figure support
\usepackage{import}
\usepackage{xifthen}
\pdfminorversion=7
\usepackage{pdfpages}
\usepackage{transparent}
\newcommand{\incfig}[1]{%
	\def\svgwidth{\columnwidth}
	\import{./figures/}{#1.pdf_tex}
}

\pdfsuppresswarningpagegroup=1

\begin{document}
\begin{titlepage}
	\begin{center}
		\vspace*{1cm}

		\Huge
		\textbf{Probabilità e Statistica\\Esercizi}

		\vspace{0.5cm}
		\LARGE
		UniVR - Dipartimento di Informatica

		\vspace{1.5cm}

		\textbf{Fabio Irimie}

		\vfill


		\vspace{0.8cm}


		2° Semestre 2023/2024

	\end{center}
\end{titlepage}


\tableofcontents
\pagebreak

\section{Introduzione}


\subsection{Numeri reali}
I numeri reali sono descritti tramite rappresentazioni decimali limitate o illimitate, periodiche
o non periodiche, e sono tutti i numeri razionali e irrazioneli; questo insieme viene indicato con
il simbolo \( \mathbb{R} \)

Proprietà necessarie dei numeri reali:
\begin{itemize}
	\item \textbf{\( 1^a \) proprietà (Eudosso-Archimede)}: due grandezze sono confrontabili quando esiste
	      un multiplo della minore che supera la maggiore. Ciò significa che non possiamo confrontare
	      linee con superfici, o superfici con volumi, ecc.

	      Questa proprietà veniva assunta come definizione di grandezze omogenee.

	      \textbf{Assioma}: dati due numeri reali positivi \( a, b \) con \( 0 < a < b \) esiste un
	      intero \( n \) tale che \( na > b \).
	\item \textbf{\( 2^a \) proprietà (Intervalli inscatolati)}: date due serie di grandezze:
	      \( a_1, a_2, \ldots, a_n \)
	      e \( b_1, b_2, \ldots, b_n \): la prima crescente (numeri della famiglia \( a \)) e la seconda
	      decrescente (numeri della famiglia \( b \)), in cui ogni \( a_k \) è minore di \( b_k \) e tali
	      che per ogni altra grandezza \( d \) si ha \( b_k - a_k < c \) per qualche \( k \), allora
	      esiste una grandezza \( c \) tale che per ogni \( k\;\; a_k \le c \le b_k \).
\end{itemize}


\subsection{Maggiorante}
\begin{definition}
	Sia \( S \subseteq \mathbb{R} \) un sottoinsieme di numeri reali. Un numero \( y \in \mathbb{R} \) `
	è un maggiorante dell'insieme \( S \) se per ogni \( x \in S \) si ha che \( y \ge x \).
\end{definition}
Se sommassimo un qualsiasi numero positivo a questo maggiorante si otterrebbe un altro maggiorante.

Se l'interballo tendesse verso \( +\infty \) non si sarebbe alcun maggiorante poichè \( +\infty \) non
è un numero reale. Esempi:
\begin{itemize}
	\item \( I = (1,10] \): tutti i maggioranti sono quelli per \( y \ge 10 \)
	\item \( I = [0,3) \): tutti i maggioranti sono quelli per \( y \ge 3 \)
	\item \( \mathbb{R} = (-\infty,+\infty) \): non ha maggiorante
\end{itemize}


\subsection{Minorante}
\begin{definition}
	Sia \( S \subseteq \mathbb{R}\) un sottoinsieme di numeri reali. Un numero \( y \in \mathbb{R} \)
	è un minorante dell'insieme \( S \) se per ogni \( x \in S \) si ha che \( y \le x \).
\end{definition}
Se sottraessimo un qualsiasi numero negativo a questo minorante si otterrebbe un altro minorante.

Se l'intervallo tendesse verso \( -\infty \) non ci sarebbe alcun minorante poichè \( -\infty \) non
è un numero reale. Esempi:
\begin{itemize}
	\item \( I = (1,10] \): tutti i minoranti sono quelli per \( y \le 1 \)
	\item \( I = [9,3) \): tutti i minoranti sono quelli per \( y \le 9 \)
	\item \( \mathbb{R} = (-\infty,+\infty) \): non ha minorante
\end{itemize}


\subsection{Estremo superiore}
Dato un insieme \( S \subseteq \mathbb{R} \), \( S \) è un insieme limitato superiormente con \( y \in \mathbb{R} \)
estremo superiore di \( S \) se:
\begin{itemize}
	\item \( y \) è un maggiorante di \( S \)
	\item \( y \) è il più piccolo maggiorante di \( S \)
\end{itemize}
Se \( S \) è un insieme illimitato superiormente allora l'estremo superiore di \( S \) è \( sup(S)=+\infty \).
Esempi:
\begin{itemize}
	\item \( I = (1,10] \): \( sup(I) = 10 \)
	\item \( I = (-\infty,0) \): \( sup(I) = 0 \)
	\item \( \mathbb{R} = (-\infty, +\infty) \): \( sup(\mathbb{R}) = +\infty \)
\end{itemize}


\subsection{Estremo inferiore}
Dato un insieme \( S \subseteq \mathbb{R} \), \( S \) è un insieme limitato inferiormente con \( y \in \mathbb{R} \)
estremo inferiore di \( S \) se:
\begin{itemize}
	\item \( y \) è un minorante di \( S \)
	\item \( y \) è il più grande minorante di \( S \)
\end{itemize}
Se \( S \) è un insieme illimitato inferiormente allora l'estremo inferiore di \( S \) è \( inf(S)=-\infty \).
Esempi:
\begin{itemize}
	\item \( I=[1,8) \): \( inf(I) = 1 \)
	\item \( I=(-13,0) \): \( inf(I) = -13 \)
	\item \( \mathbb{R} = (-\infty, +\infty) \): \( inf(\mathbb{R}) = -\infty \)
\end{itemize}


\subsection{Massimo}
\begin{definition}
	Sia \( S \subseteq \mathbb{R} \) un sottoinsieme reale, dove \( y \in \mathbb{R} \) è il massimo
	di \( S \) se \( y \) è l'estremo superiore di \( S \) e se \( y \in S \).
\end{definition}
Quindi se l'estremo superiore di un insieme appartiene all'insieme stesso, esso si chiamerà
massimo indicato con \( Max(S)=y \).


\subsection{Minimo}
\begin{figure}[H]
	\begin{definition}
		Sia \( S \subseteq \mathbb{R} \) un sottoinsieme reale, dove \( y \in \mathbb{R} \) è il minimo
		di \( S \) se \( y \) è l'estremo inferiore di \( S \) e se \( y \in S \).
	\end{definition}
\end{figure}
Quindi se l'estremo inferiore di un insieme appartiene all'insieme stesso, esso si chiamerà
minimo indicato con \( Min(S)=y \).

\begin{theorem}
	Ogni insieme di numeri reali che sia limitato superiormente ha estremo superiore.
\end{theorem}


\subsection{Funzioni}
\begin{definition}
	Una \textbf{funzione} è una corrispondenza che collega gli elementi di due insiemi dove tutti
	gli elementi del primo insieme hanno associati un solo elemento del secondo insieme:
	\[
		f: A \to B
	\]
\end{definition}
Questa è una funzione se e solo se a ogni elemento di \( A \) è associato uno e uno solo elemento di \( B \).

Tradotto in simboli diventa:
\[
	\forall a \in A \;\exists !\; b \in B \;tale\;che\; f: A \to B
\]
Esempio di funzione corretta:
\begin{figure}[H]
	\begin{center}
		\begin{tikzpicture}
			\draw[draw] (-1.5,0) circle (1cm) node[above, yshift=1.1cm] {$A$};
			\draw[draw] (1.5,0) circle (1cm) node[above, yshift=1.1cm] {$B$};

			\draw[draw, fill, blue] (-1.2,0.7) circle (0.5mm) node (1) {};
			\draw[draw, fill, blue] (-1.6,0.4) circle (0.5mm) node (2) {};
			\draw[draw, fill, blue] (-1.3,-0.3) circle (0.5mm) node (3) {};
			\draw[draw, fill, blue] (-1.8,-0.6) circle (0.5mm) node (4) {};
			\draw[draw, fill, blue] (-1.7,0) circle (0.5mm) node (9) {};

			\draw[draw, fill, red] (1.2,0.7) circle (0.5mm) node (5) {};
			\draw[draw, fill, red] (1.6,0.4) circle (0.5mm) node (6) {};
			\draw[draw, fill, red] (1.3,-0.3) circle (0.5mm) node (7) {};
			\draw[draw, fill, red] (1.8,-0.6) circle (0.5mm) node (8) {};

			\draw[->] (1) -- (5);
			\draw[->] (2) -- (6);
			\draw[->] (3) -- (7);
			\draw[->] (4) -- (8);
			\draw[->] (9) -- (8);

		\end{tikzpicture}
	\end{center}
	\caption{Esempio di funzione corretta}
\end{figure}

\subsubsection{Dominio di una funzione}
\begin{definition}
	Dato un insieme di partenza \( A \) gli elementi ai quali è applicata la funzione \( f \) sono
	il dominio stesso della funzione
\end{definition}
Esempio:
\[
	x \to x^2\; con\; D=\mathbb{R}
\] \[
	x \to \sqrt{x}\; con\; D=[0,+\infty)
\]
Si può dare un nome simbolico alla funzione scrivendo in questo modo:
\[
	f(x)=x^2\; con\; D=\mathbb{R}
\] \[
	f(x)=\sqrt{x}\; con\; D=[0,+\infty)
\]



\section{Limiti}
I limiti sono il calcolo infinitesimale, ovvero il
calcolo che si occupa di studiare il comportamento di una funzione in un intorno
di un punto.

Nelle definizioni che seguono, è data una funzione \( f:A \to \mathbb{R} \) il cui
dominio \( A \subseteq \mathbb{R} \) è un insieme \textbf{non} limitato superiormente.
(Questa ipotesi serve per definire i limiti per \( x \to +\infty \) )

\begin{definition}
	Sia \( L \in \mathbb{R} \). Si dice che
	\[ \lim_{x \to +\infty} f(x) = L  \]
	Se e solo se
	\[
		\forall \epsilon > 0\;\;\; \exists k>0 \;t.c.\; \forall x \subset A\footnote{Il dominio della funzione},
	\]
	\[
		x \ge k \to L-\epsilon \le f(x) \le L+\epsilon
	\]
	(Notazione alternativa: \( f(x) \to L \) per \( x \to +\infty \) )\\
	\textbf{La condizione deve essere soddisfatta per ogni \( \epsilon \) }.
	\begin{figure}[H]
		\begin{center}
			\begin{tikzpicture}[scale=0.5, domain=0:13]
				\draw[->] (-0.5,0) -- (13,0) node[right] {$x$};
				\draw[->] (0,-0.5) -- (0,7) node[above] {$y$};

				\coordinate (A) at (0,0);
				\coordinate (B) at (1,1);
				\coordinate (C) at (2,2);
				\coordinate (D) at (3,6);
				\coordinate (E) at (4,5);
				\coordinate (F) at (5,3);
				\coordinate (G) at (6,2.12);
				\coordinate (H) at (7,3.9);
				\coordinate (I) at (8,3);
				\coordinate (J) at (9,3);
				\coordinate (K) at (10,3);
				\coordinate (L) at (11,3);
				\coordinate (M) at (12,3);
				\coordinate (N) at (13,3);

				\draw [red, thick] plot [smooth, tension=0.7] coordinates { (A) (B) (C) (D) (E) (F) (G) (H) (I) (J) (K) (L) (M) (N) };

				\draw [] (0,3) -- (13,3) node[right] {$L$};
				\draw [dashed] (0,4) -- (13,4) node[right, scale=0.5] {$L+\epsilon$};
				\draw [dashed] (0,2) -- (13,2) node[right, scale=0.5] {$L-\epsilon$};

				\draw[fill, fill opacity=0.2, cyan] (0,4) rectangle (13,2);

				\draw [dashed] (0,3.2) -- (13,3.2) node[above right, scale=0.5] {$L+\epsilon^1$};
				\draw [dashed] (0,2.8) -- (13,2.8) node[below right, scale=0.5] {$L-\epsilon^1$};

				\draw[fill, fill opacity=0.2, green] (0,3.2) rectangle (13,2.8);

				\draw [dashed] (4.5,4) -- (4.5,0) node[below] {$k$};
				\draw [dashed] (7.8,3.2) -- (7.8,0) node[below] {$k^1$};
			\end{tikzpicture}
		\end{center}
		\caption{Definizione di limite}
	\end{figure}
	Per la definizione di limite, la funzione deve entrare in un intorno di \( L \) e non uscirne più.
	Questo vale per ogni \( \epsilon \), quindi anche per \( \epsilon^1 \).
\end{definition}
\begin{figure}[H]
	\begin{definition}
		Si dice che
		\[
			\lim_{x \to +\infty} f(x) = +\infty
		\]
		Se e solo se
		\[
			\forall M > 0\;\; \exists k>0\; \;t.c.\;\; \forall x \in A,
		\]
		\[
			x \ge k \to f(x) \ge M
		\]
		(Notazione alternativa: \( f(x) \to +\infty \) per \( x \to +\infty \))
		\begin{figure}[H]
			\begin{center}
				\begin{tikzpicture}[scale=0.5, domain=0:13]
					\draw[->] (-0.5,0) -- (13,0) node[right] {$x$};
					\draw[->] (0,-0.5) -- (0,7) node[above] {$y$};

					\coordinate (A) at (0,0);
					\coordinate (B) at (1,1);
					\coordinate (C) at (2,2);
					\coordinate (D) at (3,6);
					\coordinate (E) at (4,5);
					\coordinate (F) at (5,3);
					\coordinate (G) at (6,2.12);
					\coordinate (H) at (7,3.9);
					\coordinate (I) at (8,3);
					\coordinate (J) at (9,5);
					\coordinate (K) at (10,5);
					\coordinate (L) at (11,5);
					\coordinate (M) at (12,6);
					\coordinate (N) at (13,7);

					\draw [red, thick] plot [smooth, tension=0.7] coordinates { (A) (B) (C) (D) (E) (F) (G) (H) (I) (J) (K) (L) (M) (N) };

					\draw [] (0,4.8) -- (13,4.8) node[right] {$M$};
					\draw [] (0,6) -- (13,6) node[right] {$M^1$};

					\draw[fill, fill opacity=0.2, cyan] (0,4.8) rectangle (13,7);

					\draw [dashed] (8.8,4.8) -- (8.8,0) node[below] {$k$};
					\draw [dashed] (12,6) -- (12,0) node[below] {$k^1$};
				\end{tikzpicture}
			\end{center}
			\caption{Definizione di limite a \( +\infty \)}
		\end{figure}
	\end{definition}
\end{figure}
\begin{figure}[H]
	\begin{definition}
		Si dice che
		\[
			\lim_{x \to +\infty} f(x) = -\infty
		\]
		Se e solo se
		\[
			\forall M > 0\;\; \exists k>0\; \;t.c.\;\; \forall x \in A,
		\]
		\[
			x \ge k \to f(x) \le -M
		\]
		(Notazione alternativa: \( f(x) \to -\infty \) per \( x \to +\infty \))
		\begin{figure}[H]
			\begin{center}
				\begin{tikzpicture}[scale=0.5, domain=0:13]
					\draw[->] (-0.5,0) -- (13,0) node[right] {$x$};
					\draw[->] (0,-7) -- (0,2) node[above] {$y$};

					\coordinate (A) at (0,0);
					\coordinate (B) at (1,1);
					\coordinate (C) at (2,-2);
					\coordinate (D) at (3,-6);
					\coordinate (E) at (4,-5);
					\coordinate (F) at (5,-3);
					\coordinate (G) at (6,-2.12);
					\coordinate (H) at (7,-3.9);
					\coordinate (I) at (8,-3);
					\coordinate (J) at (9,-5);
					\coordinate (K) at (10,-5);
					\coordinate (L) at (11,-5);
					\coordinate (M) at (12,-6);
					\coordinate (N) at (13,-7);

					\draw [red, thick] plot [smooth, tension=0.7] coordinates { (A) (B) (C) (D) (E) (F) (G) (H) (I) (J) (K) (L) (M) (N) };

					\draw [] (0,-4.8) -- (13,-4.8) node[right] {$M$};
					\draw [] (0,-6) -- (13,-6) node[right] {$M^1$};

					\draw[fill, fill opacity=0.2, cyan] (0,-4.8) rectangle (13,-7);

					\draw [dashed] (8.8,-4.8) -- (8.8,0) node[above] {$k$};
					\draw [dashed] (12,-6) -- (12,0) node[above] {$k^1$};
				\end{tikzpicture}
			\end{center}
			\caption{Definizione di limite a \( -\infty \)}
		\end{figure}
	\end{definition}
\end{figure}


\subsection{Esempi}
\begin{example}
	\[
		\lim_{x \to +\infty} \frac{1}{x}=0\;\;\;Dominio=\mathbb{R}/\{0\}
	\]

	\begin{figure}[H]
		\begin{center}
			\begin{tikzpicture}
				\draw[->] (-0.5, 0) -- (7, 0) node[right] {$x$};
				\draw[->] (0, -2) -- (0, 5) node[above] {$y$};
				\draw[domain=0.2:7, smooth, variable=\x, red, thick] plot ({\x}, {1/\x});

				\draw [dashed] (0,1) -- (7,1) node[right, scale=0.8] {$0+\epsilon$};
				\draw [dashed] (0,-1) -- (7,-1) node[right, scale=0.8] {$0-\epsilon$};

				\draw[fill, fill opacity=0.2, cyan] (0,1) rectangle (7,-1);
			\end{tikzpicture}
		\end{center}
		\caption{Esempio di limite}
	\end{figure}
	Sia dato \( \epsilon > 0 \) arbitrario. Definisco \( k := \frac{1}{\epsilon} \).\\
	Sia dato \( x > 0 \) arbitrario, supponiamo \( x \ge k \). Allora
	\[
		0-\epsilon \le 0 \le \frac{1}{x} \le \frac{1}{k} = \frac{1}{\frac{1}{\epsilon}} = \epsilon
	\]
	Quindi, ho dimostrato che la definizione di limite è soddisfatta (con \( L=0 \)).
\end{example}
\begin{example}
	\[
		\lim_{x \to +\infty} x = +\infty
	\]

	\begin{figure}[H]
		\begin{center}
			\begin{tikzpicture}
				\draw[->] (-0.5, 0) -- (5, 0) node[right] {$x$};
				\draw[->] (0, -0.5) -- (0, 5) node[above] {$y$};
				\draw[domain=0:5, smooth, variable=\x, red, thick] plot ({\x}, {\x});

				\draw [dashed] (0,3) -- (5,3) node[right, scale=0.8] {$M$};

				\draw[fill, fill opacity=0.2, cyan] (0,3) rectangle (5,5);

				\draw [dashed] (3,3) -- (3,0) node[below] {$k$};
			\end{tikzpicture}
		\end{center}
		\caption{Esempio di limite a \( +\infty \)}
	\end{figure}
	Sia dato \( M>0 \) arbitrario. Definisco \( k := M \).\\
	Sia dato \( x \ge k \). Allora \( x \ge M \).\\
	Quindi è verificata la definizione di limite.

\end{example}


\subsection{Osservazioni}
\textbf{Non} è detto che un limite esista.
\[
	\lim_{x \to +\infty} sin(x)
\]
\[
	\lim_{x \to +\infty} cos(x)
\]
\begin{figure}[H]
	\begin{center}
		\begin{tikzpicture}
			\draw[->] (-5, 0) -- (5, 0) node[right] {$x$};
			\draw[->] (0, -2) -- (0, 2) node[above] {$y$};
			\draw[domain=-5:5, smooth, variable=\x, red, thick] plot ({\x}, {sin(\x r)});

			\draw [dashed] (-5,0.2) -- (5,0.2) node[right, scale=0.5] {$0+\epsilon$};
			\draw [dashed] (-5,-0.2) -- (5,-0.2) node[right, scale=0.5] {$0+\epsilon$};

			\draw[fill, fill opacity=0.2, cyan] (-5,0.2) rectangle (5,-0.2);
		\end{tikzpicture}
	\end{center}
	\caption{Esempio di limite non esistente}
\end{figure}
La funzione non entra in un intevallo limitato senza poi uscirne, quindi non esiste il limite.

\begin{figure}[H]
	\begin{center}
		\begin{tikzpicture}[scale=0.5, domain=0:13]
			\draw[->] (-0.5,0) -- (13,0) node[right] {$x$};
			\draw[->] (0,-0.5) -- (0,7) node[above] {$y$};

			\coordinate (A) at (0,0);
			\coordinate (B) at (1,1);
			\coordinate (C) at (2,2);
			\coordinate (D) at (3,6);
			\coordinate (E) at (4,5);
			\coordinate (F) at (5,3);
			\coordinate (G) at (6,2.12);
			\coordinate (H) at (7,3.9);
			\coordinate (I) at (8,3);
			\coordinate (J) at (9,3);
			\coordinate (K) at (10,3);
			\coordinate (L) at (11,3);
			\coordinate (M) at (12,3);
			\coordinate (N) at (13,3);

			\draw [red, thick] plot [smooth, tension=0.7] coordinates { (A) (B) (C) (D) (E) (F) (G) (H) (I) (J) (K) (L) (M) (N) };

			\draw [] (0,3) -- (13,3) node[right] {$L$};


			\draw [dashed] (0,3.2) -- (13,3.2) node[above right, scale=0.5] {$L+\epsilon$};
			\draw [dashed] (0,2.8) -- (13,2.8) node[below right, scale=0.5] {$L-\epsilon$};

			\draw[fill, fill opacity=0.2, cyan] (0,3.2) rectangle (13,2.8);

			\draw [dashed] (0,1) -- (13,1) node[right, scale=0.5] {$L+\epsilon^1$};
			\draw [dashed] (0,0.6) -- (13,0.6) node[right, scale=0.5] {$L-\epsilon^1$};

			\draw[fill, fill opacity=0.2, cyan] (0,1) rectangle (13,0.6);
		\end{tikzpicture}
	\end{center}
	\caption{Esempio di limite non esistente}
\end{figure}
Tuttavia, se una funzione ammette limite, allora esso è unico. Questa funzione dovrebbe
entrare in entrambe le strisce e non uscirne più, ma questo non è possibile.


\subsection{Risultati utili per il calcolo dei limiti}
\begin{theorem}[Algebra dei limiti]
	Sia \( A \subseteq \mathbb{R} \) un insieme non limitato superiormente, \( f \) e \( g \)
	due funzioni. \( A \to \mathbb{R} \). Supponiamo che i limiti
	\[
		F:= \lim_{x \to +\infty} f(x)
	\]
	\[
		G:= \lim_{x \to +\infty} g(x)
	\]
	esistano e siano \textbf{finiti}. Allora
	\[
		\lim_{x \to +\infty} (f(x) + g(x)) = F+G
	\]
	\[
		\lim_{x \to +\infty} (f(x) - g(x)) = F-G
	\]
	\[
		\lim_{x \to +\infty} (f(x) \cdot g(x)) = F \cdot G
	\]
	\[
		\lim_{x \to +\infty} \frac{f(x)}{g(x)} = \frac{F}{G}\;\;\;se\;G \neq 0
	\]
	Il teorema si estende \textbf{parzialmente} nel caso \( F \) o \( G \) siano infiniti, secondo
	le regole seguenti:
	\begin{itemize}
		\item \(
		      F + \infty = +\infty,\;\; F - \infty = -\infty\;\; \forall F \in \mathbb{R}
		      \)
		\item \(
		      +\infty + \infty = +\infty,\;\; +\infty - \infty = -\infty
		      \)
		\item \(
		      F \cdot \infty = \infty, \;\; \forall F \in \mathbb{R},\; F \neq 0
		      \)
		\item \(
		      \infty \cdot \infty = \infty
		      \)
		\item \(
		      \frac{F}{\infty} = 0 \;\; \forall F \in \mathbb{R}
		      \)
		\item \(
		      \frac{F}{0} = \infty \;\; \forall F \in \mathbb{R},\; F \neq 0
		      \)
		\item \(
		      \frac{0}{\infty} = 0
		      \)
		\item \(
		      \frac{\infty}{0} = \infty
		      \)
	\end{itemize}
	Il segno di \( \infty \) è da determinare secondo la regola usuale.
\end{theorem}


\subsection{Forme indeterminate}
Sono dei casi in cui il teorema \textbf{non} si applica e tutto può succdere:
\begin{itemize}
	\item \( +\infty - \infty \)
	\item \( 0 \cdot \infty \)
	\item \( \frac{0}{0} \)
	\item \( \frac{\infty}{\infty} \)
	\item \( 1^{\infty} \)
	\item \( 0^{0} \)
	\item \( \infty^{0} \)
\end{itemize}
\paragraph{\textbf{N.B.:}} in questo contesto, \( 0 \) , \( \infty \) e \( 1 \) sono da intendersi
come abbreviazioni.

\subsection{Esempi di calcolo di limiti}
\begin{example}
	\[ \lim_{x \to +\infty} (x^2+\frac{1}{x})  \]
	\[
		\underbrace{x^2}_\text{\( +\infty \)} + \underbrace{\frac{1}{x}}_\text{\( 0 \)} \to +\infty
	\]
	Per \( x \to +\infty \) (per il teorema dell'algebra dei limiti)
\end{example}
\begin{example}
	\[
		\lim_{x \to +\infty} x^2-x^3 = +\infty - \infty
	\]
	\[
		\underbrace{x^3}_\text{\( +\infty \)}(\underbrace{\frac{1}{x}}_\text{\( 0 \)} - 1) \to -\infty
	\]
	Per \( x \to +\infty \)
\end{example}
\begin{example}
	\[
		\lim_{x \to +\infty} (5x^6-4x) = +\infty - \infty
	\]
	\[
		\underbrace{x}_\text{\( +\infty \)}(\underbrace{5x^5}_\text{\( +\infty \)} - 4) \to +\infty
	\]
\end{example}


\subsection{Limiti razionali}
Se \( P \) è un polinomio di grado \( p \) e \( Q \) è un polinomio di grado \( q \), allora
\[
	\lim_{x \to +\infty} \frac{P(x)}{Q(x)}=
	\begin{cases}
		\pm \infty\;\;\; se\; p > q                          \\
		0\;\;\; se\; p < q                                   \\
		coefficiente\; denominante\; di\; P\;\;\; se\; p = q \\
		coefficiente\; denominante\; di\; Q\;\;\; se\; p = q
	\end{cases}
\]

\subsection{Limiti delle funzioni monotone}
\begin{theorem}[di monotonia]
	Sia \( A \subseteq \mathbb{R}\) un insieme non limitato superiormente e sia \( f:\;A \to \mathbb{R} \)
	una funzione monotona\footnote{Le funzioni \textbf{monotone} sono funzioni che
		sono sempre crescenti o sempre decrescenti}. Allora
	\[
		\lim_{x \to +\infty} f(x)\;\;esiste\;e
	\]
	\[
		\lim_{x \to +\infty} f(x) =
		\begin{cases}
			sup\{ f(x):\; x \in A \}\;\;\; se\;f\;cresce\;(non decrescecnte) \\
			inf\{ f(x):\; x \in A \}\;\;\; se\;f\;decresce\;(non crescente)
		\end{cases}
	\]
\end{theorem}
\( f: (0, +\infty) \to \mathbb{R} \)\\
\( f \) è strettamente crescente e limitata (l'immagine di \( f \) è un insieme limitato).

\begin{figure}[H]
	\begin{center}
		\begin{tikzpicture}[scale=0.5, domain=0:7]
			\draw[->] (-0.5,0) -- (7,0) node[right] {$x$};
			\draw[->] (0,-0.5) -- (0,7) node[above] {$y$};

			\coordinate (A) at (0,0);
			\coordinate (B) at (1,2);
			\coordinate (C) at (2,3.5);
			\coordinate (D) at (3,4.5);
			\coordinate (E) at (4,5);
			\coordinate (F) at (5,5);
			\coordinate (G) at (6,5);
			\coordinate (H) at (7,5);

			\draw [red, thick] plot [smooth, tension=0.8] coordinates { (A) (B) (C) (D) (E) (F) (H) };

			\draw [dashed] (0,5) -- (6,5) node[left, xshift=-3cm] {$5$};
		\end{tikzpicture}
	\end{center}
	\caption{Esempio di funzione monotona}
\end{figure}
\[
	\lim_{x \to +\infty} f(x) = 5
\]
\( g: (0,+\infty) \to \mathbb{R}\) è strettamente crescente e non limitata
\begin{figure}[H]
	\begin{center}
		\begin{tikzpicture}[scale=1.5]
			\draw[->] (-0.2, 0) -- (2, 0) node[right] {$x$};
			\draw[->] (0, -0.2) -- (0, 2) node[above] {$y$};
			\draw[domain=0.1:1.5, smooth, variable=\x, red, thick] plot ({\x}, {(\x)^2});

			\draw [dashed] (0,1) -- (2,1) node[right, scale=0.5] {};
		\end{tikzpicture}
	\end{center}
	\caption{Esempio di funzione monotona non limitata}
\end{figure}
\[
	\lim_{x \to +\infty} g(x) = +\infty
\]
\vspace{1cm}

\begin{figure}[H]
	\begin{center}
		\( f: \mathbb{R} \to \mathbb{R} \)
		\begin{tikzpicture}[scale=0.5, domain=0:13]
			\draw[->] (-0.5,0) -- (13,0) node[right] {$x$};
			\draw[->] (0,-0.5) -- (0,7) node[above] {$y$};

			\coordinate (A) at (0,0);
			\coordinate (B) at (1,1);
			\coordinate (C) at (2,2);
			\coordinate (D) at (3,6);
			\coordinate (E) at (4,5);
			\coordinate (F) at (5,3);
			\coordinate (G) at (6,2.12);
			\coordinate (H) at (7,3.9);
			\coordinate (I) at (8,3);
			\coordinate (J) at (9,3);
			\coordinate (K) at (10,3);
			\coordinate (L) at (11,3);
			\coordinate (M) at (12,3);
			\coordinate (N) at (13,3);

			\draw [red, thick] plot [smooth, tension=0.7] coordinates { (A) (B) (C) (D) (E) (F) (G) (H) (I) (J) (K) (L) (M) (N) };

			\draw [dashed] (7.2,0) -- (7.2,7) node[below, yshift=-3.5cm] {$5$};
		\end{tikzpicture}
	\end{center}
	\caption{Esempio di funzione ristrettamente monotona}
\end{figure}
Questa funzione non è monotona, ma se guardiamo ciò che succede epr \( x>5 \) si ottiene
una funzione monotona. Quindi la funzione globalmente non è monotona, ma è decrescente
ristrettamente a partire da \( x=5 \).

Per il teorema di monotonia, \[
	\lim_{x \to +\infty} f(x) = L
\]
\begin{example}
	\[
		\lim_{x \to +\infty} log(x) = +\infty
	\]
	\begin{figure}[H]
		\begin{center}
			\begin{tikzpicture}
				\draw[->] (0, 0) -- (5, 0) node[right] {$x$};
				\draw[->] (0, -2) -- (0, 2) node[above] {$y$};
				\draw[domain=0.1:5, smooth, variable=\x, red, thick] plot ({\x}, {ln((\x))});

				\draw [dashed] (0,1) -- (5,1) node[right, scale=0.5] {};
			\end{tikzpicture}
		\end{center}
		\caption{Esempio di funzione monotona non limitata}
	\end{figure}
	Per il teorema di monotonia:
	\[
		\lim_{x \to +\infty} log(x) = sup\{ log(x): x>0 \}
	\]
	\[
		\ge sup\{ log(e^n): n \in \mathbb{Z}, n>0 \}\;\;scelto\;arbitrariamente
	\]
	\[
		= sup\{ n \cdot log(e): n \in \mathbb{Z}, n>0 \} = +\infty
	\]
	Abbiamo dimostrato (per il postulato di Eudosso - Archimede) che il limite di questa
	funzione è uguale a \( +\infty \).
\end{example}
\begin{figure}[H]
	\begin{exercise}
		Dimostrare che:
		\[
			\lim_{x \to +\infty} e^x = +\infty
		\]
		\begin{figure}[H]
			\begin{center}
				\begin{tikzpicture}
					\draw[->] (-3, 0) -- (3, 0) node[right] {$x$};
					\draw[->] (0, -2) -- (0, 2) node[above] {$y$};
					\draw[domain=-3:0.8, smooth, variable=\x, red, thick] plot ({\x}, {e^(\x)});
				\end{tikzpicture}
			\end{center}
			\caption{Esempio di funzione monotona non limitata}
		\end{figure}
		E similmente che:
		\[
			\lim_{x \to +\infty} a^x = +\infty\;\;\;\forall a \in (0,+\infty)
		\]
	\end{exercise}
\end{figure}

\subsubsection{Variante}
Sia \( A \subseteq \mathbb{R} \) non limitato superiormente e siano \( f,g: A \to \mathbb{R} \)
\( t.c.\;f(x) \le g(x)\;\forall x \in A \).\\
Se \( \lim_{x \to +\infty} f(x)= +\infty \) allora \( \lim_{x \to +\infty} g(x) = +\infty \).
\begin{figure}[H]
	\begin{center}
		\begin{tikzpicture}
			\draw[->] (-0.2, 0) -- (3, 0) node[right] {$x$};
			\draw[->] (0, -0.2) -- (0, 4) node[above] {$y$};
			\draw[domain=0.1:2, smooth, variable=\x, blue, thick] plot ({\x}, {\x^2}) node [below right] {g};
			\draw[domain=0.1:2.7, smooth, variable=\x, red, thick] plot ({\x}, {\x^2/2}) node [above right] {f};
		\end{tikzpicture}
	\end{center}
	\caption{Teorema del confronto tra i limiti con 2 funzioni positive}
\end{figure}
Se \( \lim_{x \to +\infty} f(x)= -\infty \) allora \( \lim_{x \to +\infty} g(x) = -\infty \).
\begin{figure}[H]
	\begin{center}
		\begin{tikzpicture}
			\draw[->] (-0.2, 0) -- (3, 0) node[right] {$x$};
			\draw[->] (0, -4) -- (0, 0.5) node[above] {$y$};
			\draw[domain=0.1:2, smooth, variable=\x, blue, thick] plot ({\x}, {-(\x^2)}) node [below right] {f};
			\draw[domain=0.1:2.7, smooth, variable=\x, red, thick] plot ({\x}, {-(\x^2/2)}) node [above right] {g};
		\end{tikzpicture}
	\end{center}
	\caption{Teorema del confronto tra i limiti con 2 funzioni negative}
\end{figure}

\subsection{Limiti per \( x \to -\infty \) }
Sia \( A \subseteq \mathbb{R} \) un insieme non limitato inferiormente, \( f: A \to \mathbb{R} \) ,
\( L \in \mathbb{R} \cup \{+\infty, -\infty\}  \).
Diremo che:
\[
	\lim_{x \to -\infty} f(x) = L
\]
se e solo se
\[
	\lim_{x \to +\infty} f(-t) = L
\]
\begin{center}
	\(
	x=-t
	\)\\
	se \( x \to -\infty \) \\
	allora \( t \to +\infty \)
\end{center}

\subsection{Limiti per \( x \to x_0 \) }
Sia \( f: A \subseteq \mathbb{R} \), \( x_0 \in \mathbb{R} \). Per definire il limite di \( f \)
quando \( x \to 0 \), serve che \( f \) sia definita "vicino a \( x_0 \)", in un senso opportuno.
Noi supporremo, ad esempio, che il dominio \( A \) contenga almeno un intervallo del tipo
\( (x_0-\delta, x_0) \) oppure \( (x_0, x_0-\delta) \), con \( \delta>0 \). \textbf{Non} è richiesto, invece,
che \( f \) sia definita in \( x_0 \).
\begin{example}
	\[
		A = (-\infty,1) \cup (1,2)\;\;\; f: A \to \mathbb{R}
	\]
	\begin{figure}[H]
		\begin{center}
			\begin{tikzpicture}[scale=0.5, domain=0:13]
				\draw[->] (-0.5,3.5) -- (13,3.5) node[below] {$x$};
				\draw[->] (0,-0.5) -- (0,7) node[left] {$y$};

				\coordinate (A) at (-0.5,0);
				\coordinate (B) at (1,1);
				\coordinate (C) at (2,2);
				\coordinate (D) at (3,4);
				\coordinate (E) at (4,5);
				\coordinate (F) at (5,7);
				\coordinate (G) at (5,2);
				\coordinate (H) at (6,2);
				\coordinate (I) at (7,4);
				\coordinate (J) at (8,4);

				\draw [red, thick] plot [smooth, tension=0.7] coordinates { (A) (B) (C) (D) (E) (F)};
				\draw [red, thick] plot [smooth, tension=0.7] coordinates { (G) (H) (I) (J) };

				\draw[red] (5,2) circle (4pt);
				\draw[red] (8,4) circle (4pt);

				\draw [dashed] (5,0) -- (5,7) node[below, yshift=-3.5cm] {};

				\draw[fill, yellow, opacity=0.4] (-0.5,3.3) rectangle (0.5,3.7);
				\draw[fill, yellow, opacity=0.4] (4.5,3.3) rectangle (5.5,3.7);
				\draw[fill, yellow, opacity=0.4] (7.5,3.3) rectangle (8,3.7);

				\node [below left, scale=0.8] at (5,3.3) {1};
				\node [below, scale=0.8] at (8,3.3) {2};

			\end{tikzpicture}
		\end{center}
		\caption{Limiti su una funzione non continua}
	\end{figure}
	Posso definire \[
		\lim_{x \to -\infty} f(x),\; \lim_{x \to 2} f(x),\; \lim_{x \to 0} f(x),\; \lim_{x \to 0} f(x),\; \lim_{x \to 1} f(x)
	\]
	Non è detto però che tali limiti esistano
\end{example}
Sotto le ipotesi precedenti su \( f: A \subseteq \mathbb{R} \to \mathbb{R} \) e su \( x_0 \in \mathbb{R} \),
dato \( L \in \mathbb{R} \) diremo che \[
	\lim_{x \to x_0} f(x) = L
\]
se e solo se
\[
	\forall \epsilon > 0\;\;\; \exists \delta > 0\;t.c.\; \forall x \in A,
\]
\[
	x_0-\delta \le x \le x_0 + \delta\;\; e\;\; x \neq x_0
\]
\[
	\to L-\epsilon \le f(x) \le L+\epsilon
\]
\begin{figure}[H]
	\begin{center}
		\begin{tikzpicture}[scale=0.5, domain=0:13]
			\draw[->] (-0.5,0) -- (13,0) node[right] {$x$};
			\draw[->] (0,-0.5) -- (0,7) node[above] {$y$};

			\coordinate (A) at (0,0);
			\coordinate (B) at (1,1);
			\coordinate (C) at (2,2);
			\coordinate (D) at (3,1);
			\coordinate (E) at (4,3);
			\coordinate (F) at (5,4);
			\coordinate (G) at (6,4.2);
			\coordinate (H) at (7,4.5);
			\coordinate (I) at (8,4.5);
			\coordinate (J) at (9,5);
			\coordinate (K) at (10,6);
			\coordinate (L) at (11,7);
			\coordinate (M) at (12,6);
			\coordinate (N) at (13,7);

			\draw [red, thick] plot [smooth, tension=0.7] coordinates { (A) (B) (C) (D) (E) (F) (G) (H) (I) (J) (K) (L) (M) (N) };

			\draw [dashed] (0,5) -- (13,5) node[right] {$L$};
			\draw [dashed] (0,6) -- (13,6) node[right, scale=0.5] {$L+\epsilon$};
			\draw [dashed] (0,4) -- (13,4) node[right, scale=0.5] {$L-\epsilon$};

			\draw[fill, fill opacity=0.1, cyan] (0,4) rectangle (13,6);

			\draw [dashed] (9,7) -- (9,0) node[below] {$x_0$};

			\draw [dashed] (8,7) -- (8,0) node[below left] {$x_0-\delta$};
			\draw [dashed] (10,7) -- (10,0) node[below right] {$x_0+\delta$};

			\draw[fill, fill opacity=0.1, green] (8,0) rectangle (10,7);
		\end{tikzpicture}
	\end{center}
	\caption{Limite a \( x_0 \) }
\end{figure}

Sotto le ipotesi precedenti su \( f: A \subseteq \mathbb{R} \to \mathbb{R} \) e su \( x_0 \in \mathbb{R} \),
dato \( L \in \mathbb{R} \) diremo che \[
	\lim_{x \to x_0} f(x) = +\infty
\]
se e solo se
\[
	\forall M > 0\;\;\; \exists \delta > 0\;t.c.\; \forall x \in A,
\]
\[
	x_0-\delta \le x \le x_0 + \delta\;\; e\;\; x \neq x_0
\]
\[
	f(x) \ge M
\]
\begin{figure}[H]
	\begin{center}
		\begin{tikzpicture}[scale=0.5, domain=0:13]
			\draw[->] (-0.5,0) -- (13,0) node[right] {$x$};
			\draw[->] (0,-0.5) -- (0,7) node[above] {$y$};

			\coordinate (A) at (0,1);
			\coordinate (B) at (1,2);
			\coordinate (C) at (2,2);
			\coordinate (D) at (3,4);
			\coordinate (E) at (4,4.5);
			\coordinate (F) at (5.3,7);
			\coordinate (G) at (5.7,7);
			\coordinate (H) at (7,4.5);
			\coordinate (I) at (8,4.5);
			\coordinate (J) at (9,4);
			\coordinate (K) at (10,4.5);
			\coordinate (L) at (11,4);
			\coordinate (M) at (12,3);
			\coordinate (N) at (13,2);

			\draw [red, thick] plot [smooth, tension=0.7] coordinates { (A) (B) (C) (D) (E) (F) };
			\draw [red, thick] plot [smooth, tension=0.7] coordinates { (G) (H) (I) (J) (K) (L) (M) (N) };

			\draw [] (0,5) -- (13,5) node[right] {$M$};

			\draw[fill, fill opacity=0.1, cyan] (0,5) rectangle (13,7);

			\draw [dashed] (5.5,7) -- (5.5,0) node[below] {$x_0$};

			\draw [dashed] (4.5,7) -- (4.5,0) node[below left] {$x_0-\delta$};
			\draw [dashed] (6.5,7) -- (6.5,0) node[below right] {$x_0+\delta$};

			\draw[fill, fill opacity=0.1, green] (4.5,0) rectangle (6.5,7);
		\end{tikzpicture}
	\end{center}
	\caption{Limite a \( x_0 \) }
\end{figure}


Sotto le ipotesi precedenti su \( f: A \subseteq \mathbb{R} \to \mathbb{R} \) e su \( x_0 \in \mathbb{R} \),
dato \( L \in \mathbb{R} \) diremo che \[
	\lim_{x \to x_0} f(x) = -\infty
\]
se e solo se
\[
	\forall M > 0\;\;\; \exists \delta > 0\;t.c.\; \forall x \in A,
\]
\[
	x_0-\delta \le x \le x_0 + \delta\;\; e\;\; x \neq x_0
\]
\[
	f(x) \le M
\]
\subsection{Limiti unilateri}
Si possono anche dare le definizioni di limiti \textbf{unilateri}, da destra o da sinistra:
\[
	\lim_{x \to x_0^+} f(x) = \lim_{\underset{x > x_0}{x \to x_0}} f(x)
\]
\[
	\lim_{x \to x_0^-} f(x) = \lim_{\underset{x < x_0}{x \to x_0}} f(x)
\]
\begin{example}
	\[
		f: \mathbb{R} / \{0\} \to \mathbb{R}
	\]
	\[
		f(x) = \frac{1}{x}\;\; \forall x \in \mathbb{R} / \{0\}
	\]
	\[
		\lim_{x \to 0^+} (\frac{1}{x}) = +\infty
	\]
	\[
		\lim_{x \to 0^-} (\frac{1}{x}) = -\infty
	\]
	\[
		\lim_{x \to 0} (\frac{1}{x})\;\; \text{non esiste}
	\]

	\begin{figure}[H]
		\begin{center}
			\begin{tikzpicture}
				\draw[->] (-3, 0) -- (3, 0) node[right] {$x$};
				\draw[->] (0, -3) -- (0, 3) node[above] {$y$};
				\draw[domain=0.25:3, smooth, variable=\x, red, thick, yscale=0.7] plot ({\x}, {1/\x});
				\draw[domain=0.25:3, smooth, variable=\x, red, thick, yscale=0.7] plot ({-\x}, {-1/\x});
			\end{tikzpicture}
		\end{center}
		\caption{Limiti unilateri}
	\end{figure}

\end{example}

\subsection{Limiti di funzioni continue}
Sia \( A \subseteq \mathbb{R} \) un intervallo oppure un'unione finita di intervalli.
\begin{figure}[H]
	\begin{definition}
		Sia \( f: A \to  \mathbb{R} \), \( x_0 \in A \). Diremo che \( f \) è continua in \( x_0 \)
		se e solo se
		\[
			\lim_{x \to x_0} f(x) = f(x_0)
		\]
		Diremo che \( f \) è continua se e solo se \( f \) è continua in ogni punto del suo dominio
		\( x_0 \in A \).
	\end{definition}
\end{figure}

\begin{example}
	\[
		g: \mathbb{R} \to \mathbb{R}, \;\;\; g(x):=x\;\; \forall x \in \mathbb{R}
	\]
	è continua, perchè
	\[
		\lim_{x \to x_0} x = x_0\;\; \forall x_0 \in \mathbb{R}
	\]
	\begin{figure}[H]
		\begin{center}
			\begin{tikzpicture}
				\draw[->] (-2, 0) -- (2, 0) node[right] {$x$};
				\draw[->] (0, -2) -- (0, 2) node[above] {$y$};
				\draw[domain=-2:2, smooth, variable=\x, red, thick] plot ({\x}, {\x});
			\end{tikzpicture}
		\end{center}
		\caption{Eempio di funzione continua}
	\end{figure}

\end{example}
\begin{example}
	\[
		f: \mathbb{R} \to \mathbb{R}
	\]
	\[
		f(x):= \begin{cases}
			x\;\;\; se\; x \neq 2 \\
			31\;\;\; se\; x = 2
		\end{cases}
	\]
	Non è continua perchè
	\[
		\lim_{x \to 2} f(x) = 2 \neq f(2)
	\]
	Però \( f \) è continua in tutti gli \( x_0 \in  \mathbb{R} \), \( x_0 \neq 2 \):
	\[
		\lim_{x \to x_0} f(x) = f(x_0) = x_0
	\]
	\begin{figure}[H]
		\begin{center}
			\begin{tikzpicture}
				\draw[->] (-2, 0) -- (2, 0) node[right] {$x$};
				\draw[->] (0, -2) -- (0, 2) node[above] {$y$};
				\draw[domain=-2:0.45, smooth, variable=\x, red, thick] plot ({\x}, {\x});
				\draw[domain=0.55:2, smooth, variable=\x, red, thick] plot ({\x}, {\x});

				\draw[red] (0.5,0.5) circle (2pt);
				\draw[fill, red] (0.5,1.8) circle (2pt);
			\end{tikzpicture}
		\end{center}
		\caption{Eempio di funzione non continua}
	\end{figure}
\end{example}
\begin{example}
	\[
		h: \mathbb{R} / \{0\} \to \mathbb{R}
	\]
	\[
		h(x):= \frac{1}{x}\;\;\; \forall x \in \mathbb{R} / \{0\}
	\]
	Il dominio è un unione di 2 intervalli:
	\[
		(\mathbb{R}/0 = (-\infty,0) \cup (0,+\infty))
	\]
	È una funzione continua
	\begin{figure}[H]
		\begin{center}
			\begin{tikzpicture}
				\draw[->] (-3, 0) -- (3, 0) node[right] {$x$};
				\draw[->] (0, -3) -- (0, 3) node[above] {$y$};
				\draw[domain=0.25:3, smooth, variable=\x, red, thick, yscale=0.7] plot ({\x}, {1/\x});
				\draw[domain=0.25:3, smooth, variable=\x, red, thick, yscale=0.7] plot ({-\x}, {-1/\x});
			\end{tikzpicture}
		\end{center}
		\caption{Esmpio di funzione continua}
	\end{figure}

\end{example}
\begin{example}
	\[
		l: \mathbb{R} \to \mathbb{R}
	\]
	\[
		l(x):=\begin{cases}
			\frac{1}{x}\;\;\; se\; x \neq 0 \\
			5\;\;\; se\; x = 0
		\end{cases}
	\]
	Questa funzione non è continua perchè il limite a 0 non esiste:
	\[
		\lim_{x \to 0} l(x) = \nexists
	\]
	ma:
	\[
		\lim_{x \to 0} |l(x)| = +\infty
	\]
	\begin{figure}[H]
		\begin{center}
			\begin{tikzpicture}
				\draw[->] (-3, 0) -- (3, 0) node[right] {$x$};
				\draw[->] (0, -3) -- (0, 3) node[above] {$y$};
				\draw[domain=0.25:3, smooth, variable=\x, red, thick, yscale=0.7] plot ({\x}, {1/\x});
				\draw[domain=0.25:3, smooth, variable=\x, red, thick, yscale=0.7] plot ({-\x}, {-1/\x});
				\draw[fill, red] (0,1.5) circle (2pt) node[left] {5};
			\end{tikzpicture}
		\end{center}
		\caption{Esmpio di funzione non continua}
	\end{figure}
\end{example}

\begin{example}
	\[
		m: \mathbb{R} \to \mathbb{R}
	\]
	\[
		m(x):=\begin{cases}
			x^2\;\;\; se\; x \neq 0 \\
			-2\;\;\; se\; x = 0
		\end{cases}
	\]
	Non è continua perchè:
	\[
		\lim_{x \to 0} m(x) = \lim_{x \to 0} x^2 = 0 \neq m(0)
	\]
	\begin{figure}[H]
		\begin{center}
			\begin{tikzpicture}
				\draw[->] (-3, 0) -- (3, 0) node[right] {$x$};
				\draw[->] (0, -2) -- (0, 3) node[above] {$y$};
				\draw[domain=0.1:2, smooth, variable=\x, red, thick, yscale=0.7] plot ({\x}, {\x^2});
				\draw[domain=0.1:2, smooth, variable=\x, red, thick, yscale=0.7] plot ({-\x}, {\x^2});

				\draw[red, thick] (0,0) circle (3pt);
				\draw[fill, red] (0,-1) circle (2.5pt) node[left] {-2};
			\end{tikzpicture}
		\end{center}
		\caption{Esmpio di funzione non continua}
	\end{figure}
\end{example}

\section{Notazione o piccolo di Landau}
Si dimostra che:
\[
	\lim_{x \to 0} \frac{sin(x)}{x}=1\;\;\;\;\;\;\;(F.I.\; \frac{0}{0})
\]
Considero \( x > 0 \)
\begin{figure}[H]
	\begin{center}
		\begin{tikzpicture}
			\begin{axis}[
					axis lines = middle,
					xmin=-2, xmax=5, ymin=-2, ymax=5,
					axis equal,
					xlabel = $x$,
					ylabel = $y$,
					yticklabels={,,}
				]
				\draw (axis cs: 4,0) arc[radius =4, start angle= 0, end angle= 90];
				\draw[red] (axis cs: 1,0) arc[radius =1, start angle= 0, end angle= 42] node [below right, xshift=4, yshift=-2, scale=0.6] {\( x \)};

				\addplot [mark=none, dashed, black] coordinates {(4, -1) (4, 4)};
				\node [below right, red, scale=0.6] at (axis cs: 4,0) {\( U = (1,0) \) };
				\draw [fill, red] (axis cs: 4,0) circle (1.5pt);

				\addplot [mark=none, dashed, black] coordinates {(3, -1) (3, 3)};
				\node [below left, red, scale=0.6] at (axis cs: 3,0) {\( H = (cos(x),0) \) };
				\draw [fill, red] (axis cs: 3,0) circle (1.5pt);

				\addplot [mark=none, thick, red] coordinates {(3, 0) (3, 2.65)};

				\addplot [mark=none, thick, red] coordinates {(0, 0) (4, 3.55)};
				\node [below left, red, scale=0.6] at (axis cs: 3,3) {\( P = (cos(x),sin(x)) \) };
				\draw [fill, red] (axis cs: 3,2.65) circle (1.5pt);

				\addplot [mark=none, thick, red] coordinates {(4, 0) (4, 3.55)};
				\node [right, red, scale=0.6] at (axis cs: 4,3.55) {\( Q = (1,tan(x)) \) };
				\draw [fill, red] (axis cs: 4,3.55) circle (1.5pt);

				\addplot [mark=none, thick, red] coordinates {(0, 0) (4, 0)};
				\node [below left, red, solid, scale=0.6] at (axis cs: 0,0) {\( O = (0,0) \) };
				\draw [fill, red] (axis cs: 0,0) circle (1.5pt);
			\end{axis}
		\end{tikzpicture}
	\end{center}
	\caption{Grafico}
\end{figure}
Area del triangolo \( OHP \):
\begin{itemize}
	\item \( \le  \) area del settore \( OUP \)
	\item \( \le  \) area del triangolo \( OUQ \)
\end{itemize}
Area di \( OHP = \frac{1}{2} sin(x)cos(x) \) \\
Area di \( OUQ = \frac{1}{2} tan(x) = \frac{1}{2} \frac{sin(x)}{cos(x)} \) \\
Area di \( OUP \) : area del disco unitario = ampiezza dell'angolo \( P\hat{O}U \): ampiezza dell'angolo giro

da cui:
\[
	Area\; di\; OUP = \frac{\pi x}{2 \pi} = \frac{1}{2}x
\]
Pertanto:
\[
	\frac{1}{2}sin(x)cos(x) \le \frac{1}{2}x \le \frac{1}{2} \frac{sin(x)}{cos(x)}
\]
Moltiplico per \( \frac{2}{sin(x)} \) (assumendo che \( 0 < x < \frac{\pi}{2} \), così che \( sin(x) > 0 \)):
\[
	cos(x) \le \frac{x}{sin(x)} \le \frac{1}{cos(x)}
\]
da cui:
\[
	\underbrace{cos(x)}_{1} \le \frac{sin(x)}{x} \le \underbrace{\frac{1}{cos(x)}}_{1}
\]
\[
	per\; x \to  0^+
\]
Per il teorema del confronto, segue che
\[
	\lim_{x \to 0^+} \frac{sin(x)}{x} = 1.
\]
Il caso \( x \to 0^- \) è analogo. \( \square \)
\vspace{1cm}\\
Se definiamo:
\[
	q(x):=\frac{sin(x)}{x}-1
\]
posso concludere che:
\[
	\frac{sin(x)}{x}=1 + q(x) \Leftrightarrow sin(x)=x+xq(x)
\]
\[
	\lim_{x \to 0} q(x) = 0
\]
\begin{figure}[H]
	\begin{definition}
		\textbf{Notazione} \( o \) piccolo di Landau.

		Diremo che:
		\[
			f(x) = o(g(x))\;\;\;\;\;\; per\; x \to x_0
		\]
		se e solo se esiste una funzione \( q \) tale che:
		\[
			f(x) = g(x)q(x)\;\;\;\;\;\; (\forall x)
		\]
		\[
			\lim_{x \to x_0} q(x) = 0
		\]
	\end{definition}
\end{figure}
Ad esempio, possiamo dire che:
\[
	sin(x) = x + \underbrace{o(x)}_{g(x) q(x)}\;\;\;\;\;\; per\; x \to 0
\]
\subsection{Proprietà}
\begin{enumerate}
	\item \( f(x) = o(1)\) per \( x \to x_0 \Leftrightarrow \lim_{x \to x_0} f(x) = 0  \)
	\item \( o(g(x)) = g(x) o(1) \) per \( x \to x_0 \)
	\item \( o(g(x)) + o(g(x)) = o(g(x)) \) per \( x \to x_0 \) \\
	      Infatti,\[
		      o(g(x)) + o(g(x)) = g(x) q_1(x) + g(x) q_2(x)
	      \] dove
	      \[
		      \lim_{x \to x_0} q_1(x) = \lim_{x \to x_0} q_2(x) = 0
	      \] e quindi
	      \[
		      o(g(x)) + o(g(x)) = g(x) \underbrace{(q_1(x) + q_2(x))}_{0\;\; per\; x \to x_0} = o(g(x))
	      \]
	\item Se \( k \in \mathbb{R} \) è una costante,
	      \[
		      k o(g(x)) = o(g(x))\;\;\;\;\;per\;x \to x_0
	      \]
	\item \( f(x)o(g(x)) = o(f(x)g(x))\;\;\;\;\;per\; x \to x_0 \)
	\item In generale, \textbf{non} vale
	      \[
		      o(g(x)) - o(g(x)) = 0\;\;\;\;\;per\; x \to x_0
	      \]
	      Infatti,
	      \[
		      o(g(x)) - o(g(x)) = g(x)q_1(x) - g(x)q_2(x)
	      \]
	      dove
	      \[
		      \lim_{x \to x_0} q_1(x) = \lim_{x \to x_0} q_2(x) = 0
	      \]
	      ma \textbf{non} è detto che \( q_1(x)=q_2(x) \).

	      (Però è vero che \( o(g(x))-o(g(x)) = o(g(x))\;\;\;\;\;per\; x \to x_0 \) )
	\item Allo stesso modo, \textbf{non} è detto che
	      \[
		      \frac{o(g(x))}{o(g(x))} = 1 \;\;\;\;\;per\; x \to x_0
	      \] (forma indeterminata)
\end{enumerate}
\textbf{È molto importante specificare \( x \to x_0 \) }.

Ad esempio:
\begin{example}
	\[
		x^2 = o(x) \;\;\;\;\;per\; x \to 0
	\]
	\[
		x = o(x^2) \;\;\;\;\;per\; x \to +\infty
	\]
\end{example}

\subsection{Sviluppi di alcune funzioni elementari per \( x \to 0 \) }
\begin{itemize}
	\item \(
	      e^x = 1 + x + o(x)
	      \)
	\item \(
	      log(1+x) = x + o(x)
	      \)
	\item \(
	      (1+x)^\alpha = 1 + \alpha x + o(x)\;\;\;\;\; (con\; \alpha \in \mathbb{R}\; costante)
	      \)
	\item \(
	      sin(x) = x + o(x)
	      \)
	\item \(
	      cos(x) = 1 - \frac{x^2}{2} + o(x^2)
	      \)
\end{itemize}

\subsection{Funzioni continue}

Proprietà:
\begin{enumerate}
	\item Se \( f,g: A \subseteq \mathbb{R} \to  \mathbb{R} \) sono funzioni continue, allora sono continue anche
	      \[
		      f+g, f-g, fg, \frac{f}{g}
	      \]
	      (quest'ultima definita su \( \{x \in A: g(x) \neq 0\}  \) )
	\item Se \( f: A \to \mathbb{R},\;g: B \to \mathbb{R} \) con \( A \subseteq \mathbb{R},\; B \subseteq \mathbb{R} \)
	      sono funzioni continue tali che \( f(A) \subseteq B \), allora è continua anche la funzione composta
	      \[
		      g \circ f: A \to \mathbb{R}
	      \]
	      \[
		      (g \circ f)(x) := g(f(x))\;\;\;\;\;\;\; \forall x \in A
	      \]
\end{enumerate}
\begin{example}
	Sono funzioni continue:
	\begin{itemize}
		\item tutti i polinomi
		\item tutte le funzioni razionali (quozienti di polinomi)
		\item \( x \to x^\alpha \), con \( \alpha \in \mathbb{R} \) costante, laddove ben definito
		\item exp, log, sin, cos, tan, \( \ldots \)
		\item valore assoluto, \( x \in \mathbb{R} \to |x|:= \begin{cases}
			      x\;\;\; se\; x \ge 0 \\
			      -x\;\;\; se\; x < 0
		      \end{cases} \)
		\item funzioni composte, ad esempio:
		      \[
			      h_1: \mathbb{R} \to \mathbb{R},\;\;\; h_1(x):=sin(x^3 + 5x^4)\;\;\; \forall x \in \mathbb{R}
		      \]
		      \[
			      h_2:(2, +\infty) \to \mathbb{R}, \;\;\; h_2(x):=log(x^2-4)\;\;\; \forall x \in (2, +\infty)
		      \]
	\end{itemize}
\end{example}

\section{Derivate}
Sia \( A \) un intervallo aperto (del tipo \( A=(a,b) \) oppure \( A=(a,+\infty), A=(-\infty, a), A=\mathbb{R} \)),
oppure un'unione di intervalli aperti.

Sia \( f: A \to \mathbb{R},\;\;x_0 \in A \). \textbf{Retta tangente} al grafico di \( f \) nel
punto \( (x_0, f(x_0)) \)?
\begin{figure}[H]
	\begin{center}
		\begin{tikzpicture}[scale=0.5, domain=0:13]
			\draw[->] (-0.5,0) -- (8.5,0) node[right] {$x$};
			\draw[->] (0,-0.5) -- (0,7) node[above] {$y$};

			\coordinate (A) at (0,-0.5);
			\coordinate (B) at (1,1);
			\coordinate (C) at (2,2);
			\coordinate (D) at (3,2.2);
			\coordinate (E) at (4,2.5);
			\coordinate (F) at (5,4);
			\coordinate (G) at (6,6);
			\coordinate (H) at (7,6.5);
			\coordinate (I) at (8,6.5);

			\draw [red, thick] plot [smooth, tension=0.7] coordinates { (A) (B) (C) (D) (E) (F) (G) (H) (I) };

			\draw[fill, green] (2, 2) circle (4pt);
			\draw[fill, green] (5, 4) circle (4pt);

			\draw[green, thick] (0, 0.65) -- (8, 6);
			\draw[blue, thick] (0, 1.65) -- (8, 3);

			\draw[dashed] (2, 2) -- (2, 0) node[below] {$x_0$};
			\draw[dashed] (5, 4) -- (5, 0) node[below] {$x_0+h$};
		\end{tikzpicture}
	\end{center}
\end{figure}
Preso \( h \in \mathbb{R},\; h \neq 0 \), il coefficiente angolare (pendenza) della retta secante
il grafico nei punti \( (x_0, f(x_0)), \;\; (x_0+h, f(x_0+h)) \) è:
\[
	\frac{f(x_0+h)-f(x_0)}{h}
\]

\begin{definition}
	Una funzione \( f: A \to \mathbb{R} \) si dice \textbf{differenziabile} (o derivabile)
	in \( x_0 \in A \) se e solo se esiste ed è finito il limite:
	\[
		f'(x_0):= \lim_{h \to 0} \frac{f(x_0+h)-f(x_0)}{h}
	\]
	Tale limite è detto \textbf{derivata} di \( f \) in \( x_0 \). \( f \) si dice
	\textbf{differenziabile} (o derivabile) se e solo se è differenziabile in ogni punto
	del suo dominio.
\end{definition}

\subsubsection{Osservazioni}
\begin{enumerate}
	\item La retta tangente al grafico di \( f \) in \( (x_0, f(x_0)) \) è definita
	      come l'unica retta di pendenza \( f'(x_0) \) passante per \( (x_0, f(x_0) \).
	      Essa ha equazione:
	      \[
		      y = f'(x_0)(x-x_0) + f(x_0)
	      \]
	\item \( f \) è differenziabile in \( x_0 \) se e solo se vale:
	      \[
		      \frac{f(x_0+h)-f(x_0)}{h}=f'(x_0) + o(1)\;\;\; per\;h \to 0
	      \]
	      che equivale a dire:
	      \[
		      f(x_0+h)-f(x_0)=f'(x_0)h+h\,o(1)\;\;\; per\;h \to 0
	      \]
	      Quindi, \( f \) è differenziabile in \( x_0 \) se e solo se:
	      \[
		      f(x_0+h)=f(x_0)+f'(x_0)h + o(h) \;\;\; per\;h \to 0
	      \]
	      il che equivale (posto \( x=x_0+h \)) a:
	      \[
		      f(x)=f(x_0)+f'(x_0)(x-x_0)+o(x-x_0)\;\;\; per\;x \to x_0
	      \]
\end{enumerate}
\begin{figure}[H]
	\begin{example}
		\[
			f=e^x,\;\;\; x_0=0
		\]
		\begin{figure}[H]
			\begin{center}
        \begin{tikzpicture}[scale=1.5]
					\draw[->] (-2, 0) -- (1.6, 0) node[right] {$x$};
					\draw[->] (0, -0.5) -- (0, 2) node[above] {$y$};
					\draw[domain=-4:1.4, smooth, variable=\x, red, thick, yscale=0.5, xscale=0.5] plot ({\x}, {e^\x});
					\draw[domain=-2:3.2, smooth, variable=\x, lime, yscale=0.5, xscale=0.5] plot ({\x}, {\x+1});
					\draw[domain=-1:0.4, dashed, variable=\y] plot ({0.5}, {\y+1});

					\node[below left, scale=0.7] at (0,0) (0) {\( x_0=0 \) };
					\node[below, scale=1] at (0.5,0) (0) {\( x \) };

					\draw[fill, green, opacity=1] (0, -0.01) rectangle (0.5, 0.01);
					\draw[fill, blue, opacity=1] (0.49, 1) rectangle (0.51, 1.35);


					\draw[fill, green, opacity=1] (-0.6, -1.01) rectangle (-0.2, -0.99);
					\node[scale=0.5] at (0, -1) (0) {\( > \) };
					\draw[fill, blue, opacity=1] (0.2, -1.01) rectangle (0.4, -0.99);
				\end{tikzpicture}
			\end{center}
			\caption{Eempio di funzione continua}
		\end{figure}
		\[
			e^x = 1 + x + o(x)\;\;\; per\;x \to 0
		\]
		dunque
		\[
			e'^0=1
		\]
		Che sarebbe il coefficiente di x nell'equazione \( e^x=1+x+o(x) \)
	\end{example}
\end{figure}
Si può anche scrivere (Notazione di Leibnitz):
\[
	f'(x_0) = \frac{df}{dx}(x_0)
\]
\begin{example}
	Una funzione costante è differenziabile con derivata
	\[
		(5')(x_0)= \lim_{h \to 0} \frac{5-5}{h} = 0
	\]
\end{example}

\subsection{Proprietà delle funzioni differenziabili}
Dove non specificato, supporremo sempre che il dominio \( A \)  sia un intervallo
aperto o un'unione di intervalli aperti.

\textbf{Proprietà}: Se \( f:A \to \mathbb{R} \) è differenziabile in \( x_0 \), allora
\( f \) è continua in \( x_0 \).

Dimostrazione:
\[
	\lim_{x \to x_0} f(x) = \lim_{x \to x_0} (f(x_0)+f'(x_0)(x-x_0)+o(x-x_0)) = f(x_0) \square
\]
\textbf{Non} vale il viceversa: \( f \) può essere continua senza essere differenziabile.
\begin{figure}[H]
	\begin{example}
		\[
			f(x):= |x|,\;\;\; \forall x \in \mathbb{R}
		\]
		\[
			x_0 = 0
		\]
		\[
			f'+(0)=\lim_{h \to 0^+} \frac{f(h)-f(0)}{h} = \lim_{h \to 0^+} \frac{|h|}{h} = 1
		\]
		\[
			f'-(0)=\lim_{h \to 0^-} \frac{f(h)-f(0)}{h} = \lim_{h \to 0^-} \frac{|h|}{h} = -1
		\]
 		\begin{figure}[H]
			\begin{center}
				\begin{tikzpicture}
					\draw[->] (-2, 0) -- (2, 0) node[right] {$x$};
					\draw[->] (0, -0.5) -- (0, 2) node[above] {$y$};
					\draw[domain=0:3.5, smooth, variable=\x, red, thick, yscale=0.5, xscale=0.5] plot ({\x}, {\x});
					\draw[domain=0:-3.5, smooth, variable=\x, red, thick, yscale=0.5, xscale=0.5] plot ({\x}, {-\x});
					\draw[domain=-1:0.1, dashed, variable=\y] plot ({1.1}, {\y+1});

					\node[below, scale=1] at (1.1,0) (0) {\( x_0 \) };
				\end{tikzpicture}
			\end{center}
      \caption{Esempio di funzione continua e non differenziabile}
		\end{figure}

	\end{example}
\end{figure}
Le derivate destra e sinistra in \( x_0 = 0 \) esistono e sono entrambe finite, ma sono
\textbf{diverse} tra loro: \( f \) ha un \textbf{punto angoloso} in \( x_0=0 \).

\begin{figure}[H]
	\begin{example}
		\[
			g: [0,+\infty) \to \mathbb{R},\;\;\; g(x):= \sqrt{x}\;\;\; \forall x \ge 0
		\]
		\[
			x_0=0
		\]
		\[
			g'+(0)= \lim_{h \to 0^+} \frac{g(h)-g(0)}{h} = \lim_{h \to 0^+} \frac{\sqrt{h}}{h} = \lim_{h \to 0^+} \frac{1}{\sqrt{h} }  = +\infty
		\]
 		\begin{figure}[H]
			\begin{center}
				\begin{tikzpicture}
					\draw[->] (-0.5, 0) -- (4, 0) node[right] {$x$};
					\draw[->] (0, -0.5) -- (0, 2) node[above] {$y$};
					\draw[domain=0:7, smooth, tension=0.5, variable=\x, red, thick, yscale=0.7, xscale=0.5] plot ({\x}, {sqrt(\x)});
					\draw[domain=-1:0.1, dashed, variable=\y] plot ({1.2}, {\y+1});

					\node[below, scale=1] at (1.2,0) (0) {\( x_0 \) };
          
          \draw[fill, red] (0,0) circle (1.5pt);
				\end{tikzpicture}
			\end{center}
      \caption{Esempio di funzione continua e non differenziabile}
		\end{figure}

		Il limite (destro, in questo caso) del rapporto incrementale esiste, ma è infinito: \( g \)
		ha una \textbf{cuspide} o \textbf{punto a tangente verticale} in \( x_0=0 \).
	\end{example}
\end{figure}

\begin{definition}
	Un punto \( x_0 \in A \) si dice punto di \( \begin{cases}
	 \text{massimo}\\
   \text{minimo}
	\end{cases} \)  locale di una funzione \( f: A \to \mathbb{R} \)
	se esiste \( \delta > 0 \) tc \( \begin{cases}
		f(x_0) \ge f(x) \\
		f(x_0) \le f(x)
	\end{cases} \) per ogni:
	\[
		x \in (x_0 - \delta, x_0 + \delta) \cap A
	\]
  		\begin{figure}[H]
			\begin{center}
				\begin{tikzpicture}[scale=0.6, domain=0:13]
					\draw[->] (-0.5,0) -- (13,0) node[right] {$x$};

					\coordinate (A) at (0,6);
					\coordinate (B) at (1,3);
					\coordinate (C) at (2,4);
					\coordinate (D) at (3,1);
					\coordinate (E) at (4,2);
					\coordinate (F) at (5,0);
					\coordinate (G) at (6,-1);
					\coordinate (H) at (7,0);
          \coordinate (I) at (8,3);
          \coordinate (J) at (9,2);
          \coordinate (K) at (10,4);
          \coordinate (L) at (11,5.5);
          \coordinate (M) at (12,6);
          \coordinate (N) at (13,6);

					\draw [red, thick] plot [smooth, tension=0.7] coordinates { (A) (B) (C) (D) (E) (F) (G) (H) (I) (J) (K) (L) (M) (N) };

          \draw[fill, blue] (1, 0) circle (4pt);
          \draw[blue, dashed] (1, 0) -- (1, 3);

          \draw[fill, green] (1.9, 0) circle (4pt);
          \draw[green, dashed] (1.9, 0) -- (1.9, 4);

          \draw[fill, blue] (3.1, 0) circle (4pt);
          \draw[blue, dashed] (3.1, 0) -- (3.1, 1);

          \draw[fill, green] (3.9, 0) circle (4pt);
          \draw[green, dashed] (3.9, 0) -- (3.9, 2);

          \draw[fill, blue] (6, 0) circle (4pt);
          \draw[blue, dashed] (6, 0) -- (6, -1);
          \node[blue, above, scale=0.8] at (6, 0.5) {\( x_0 \) };
          \draw (5.2, 0.1) -- (5.2, -1) node[below, scale=0.6] {\( x_0 - \delta \)};
          \draw (6.8, 0.1) -- (6.8, -1) node[below, scale=0.6] {\( x_0 + \delta \)};

          \draw[fill, green] (8.1, 0) circle (4pt);
          \draw[green, dashed] (8.1, 0) -- (8.1, 3);

          \draw[fill, blue] (9, 0) circle (4pt);
          \draw[blue, dashed] (9, 0) -- (9, 2);

          \draw[fill, green] (12, 0) circle (4pt);
          \draw[green, dashed] (12, 0) -- (12, 6);
				\end{tikzpicture}
			\end{center}
			\caption{Teorema degli zeri}
		\end{figure}

  I punti di massimo o minimo locale si chiamano anche \textbf{estremi locali}.
\end{definition}


\subsection{Derivate delle funzioni inverse}
\begin{figure}[H]
  \begin{example}
    Consideriamo
    \[
    tan: (-\frac{\pi}{2}, \frac{\pi}{2}) \to \mathbb{R}
    \] 
    \[
    tan(x) = \frac{sin(x)}{cos(x)} \forall x \in (-\frac{\pi}{2}, \frac{\pi}{2})
    \] 
 		\begin{figure}[H]
			\begin{center}
        \begin{tikzpicture}[scale=1.3]
					\draw[->] (-2, 0) -- (2, 0) node[right] {$x$};
					\draw[->] (0, -2) -- (0, 2) node[above] {$y$};
					\draw[domain=-1.3:1.3, smooth, variable=\x, red, thick, yscale=0.5, xscale=0.5] plot ({\x}, {tan(\x r)});

					\draw[domain=-1:0.8, dashed, variable=\y] plot ({0.68}, {\y+1}) node[below, yshift=-69] {\( \frac{\pi}{2} \)};
					\draw[domain=-0.8:1, dashed, variable=\y] plot ({-0.68}, {\y-1}) node[below, yshift=18] {\( -\frac{\pi}{2} \)};

				\end{tikzpicture}
			\end{center}
      \caption{Esempio di funzione continua e non differenziabile}
		\end{figure}

    La tangente è differenziabile
    \[
      \frac{d}{dx}(tan(x)) = \frac{(sin)'(x)cos(x)-(cos)'(x)sin(x)}{cos^2(x)}
    \] 
    \[
    =\frac{cos^2(x)+sin^2(x)}{cos^2(x)} = \frac{1}{cos^2(x)} = 1+tan^2(x) > 0 \forall x \in (-\frac{\pi}{2}, \frac{\pi}{2})
    \] 
    \begin{itemize}
      \item 
        La tangente è strettamente crescente, quindi iniettiva
      \item La tangente è suriettiva: per ogni \( y \in \mathbb{R} \), esiste \[
      x \in (-\frac{\pi}{2}, \frac{\pi}{2}) \;tc\; tan(x)=y
      \] 
      Infatti la tangente è continua e 
      \[
      \lim_{x \to (-\frac{\pi}{2})^+} tan(x) = -\infty
      \] 
      \[
      \lim_{x \to (\frac{\pi}{2})^-} tan(x) = +\infty
      \] 
    \end{itemize}
    Quindi il teorema degli zeri implica che esiste \( x \in (\frac{-\pi}{2}, \frac{\pi}{2})\;tc\; tan(x)=y \)
  \end{example}
\end{figure}
\( tan: (\frac{-\pi}{2}, \frac{\pi}{2}) \to \mathbb{R} \) è biettiva, quindi per ogni \( y \in \mathbb{R} \)
esiste un unico numero reale, che indicheremo \( arctan(y) \), tale che
\[
  \begin{cases}
    -\frac{\pi}{2} < arctan(y) < \frac{\pi}{2}\\
    tan(arctan(y)) = y
  \end{cases}
\] 
 		\begin{figure}[H]
			\begin{center}
        \begin{tikzpicture}[scale=1.3]
					\draw[->] (-2, 0) -- (2, 0) node[right] {$y$};
					\draw[->] (0, -2) -- (0, 2) node[above] {$arctan(y)$};
					\draw[domain=-4:4, smooth, variable=\x, red, thick, yscale=0.5, xscale=0.5] plot ({\x}, {rad(atan(\x))});
				\end{tikzpicture}
			\end{center}
      \caption{Esempio di funzione continua e non differenziabile}
		\end{figure}

La funzione \( arctan \) è differenziabile? Se sì, chi è la sua derivata?

Supponiamo già di sapere che \( arctan \) è differenziabile (è vero, ma andrebbe dimostrato)
\[
tan(arctan(y)) = y \;\;\;\;\forall y \in \mathbb{R}
\] 
Deriviamo ambo i membri:
\[
\frac{d}{dy}(tan(arctan(y))) = 1
\] 
\[
\frac{d}{dy}(tan(arctan(y)))= tan'(arctan(y)) \cdot (arctan(y))'
\] 
\[
tan'(x) = 1 + tan^2(x) = (1 + (tan(arctan(y))))^2 \cdot  (arctan(y))'
\] 
\[
= (1+y^2) \cdot (arctan(y))'
\] 
Dunque:
\[
  (1+y^2)arctan'(y) = 1
\] 
e quindi:
\[
arctan'(y) = \frac{1}{1+y^2}
\] 
Quanto fatto ha validità più generale:

Sia \( I \subseteq \mathbb{R} \) un intervallo, \( f: I \to \mathbb{R} \) una funzione
fifferenziabile tale che \( f'(x) \neq 0\;\; \forall x \in \mathbb{R}  \).
Allora esiste la funzione inversa:
\[
g: f(I) \to I
\] 
tale che \( f(g(y))=y\; \forall y \in f(I),\; g(f(x)) = x\; \forall x \in I \)

Inoltre, \( g \) è differenziabile e vale:
\[
g'(y) = \frac{1}{f'(g(y))}
\] 
per ogni \( y \in f(I) \) 

\begin{figure}[H]
  \begin{exercise}
    Trovare le derivate delle funzioni:
    \[
    arccos: [-1,1] \to [0,\pi]
    \] 
    \[
    arcsin: [-1,1] \to [-\frac{\pi}{2}, \frac{\pi}{2}]
    \] 
  \end{exercise}
\end{figure}

\section{Derivate successive}
Sia \( A \subseteq \mathbb{R} \) un intervallo aperto, o un'unione di intervalli aperti.
Sia \( f: A \to \mathbb{R} \).

Se (e solo se) \( f \) è differenziabile e \( f' \) è differenziabile, si dice
che \( f \) è differenziabile due volte. Si scrive \( f'' \) per la derivata seconda di
\( f \) (cioè la derivata di \( f \)).

Similmente si definiscono le funzioni differenziabili \( 3,4,5, \ldots, \) infinite volte.
Notazione per le derivate successive:
\[
f' = f^{(1)} = \frac{df}{dx}
\] 
\[
f'' = f^{(2)} = \frac{d^2f}{dx^2}
\] 
\[
f''' = f^{(3)} = \frac{d^3f}{dx^3}
\] 
\[
f^{(n)} = \frac{d^nf}{dx^n}
\] 
\subsection{Funzioni convesse e concave}
Sia \( I \subseteq \mathbb{R} \) intervallo. Una funzione \( f: I \to \mathbb{R} \)
si dice \( \begin{cases}
  \text{convessa}\\
  \text{concava}
\end{cases} \) se e solo se la corda tra due punti qualsiasi del grafico di \( f \) sta
tutta \( \begin{cases}
  \text{sopra}\\
  \text{sotto}
\end{cases} \) il grafico di \( f \). 
\begin{figure}[H]
	\centering
	\begin{tikzpicture}[scale=0.6, domain=0:10]
		\coordinate (A) at (0,5);
		\coordinate (B) at (1,3);
		\coordinate (C) at (2,2);
		\coordinate (D) at (3,1);
		\coordinate (E) at (4,1);
		\coordinate (F) at (5,2);
		\coordinate (G) at (6,3);
		\coordinate (H) at (7,5);

		\draw [->] (0,0) -- (7,0) node[right] {$x$};
		\draw [->] (0,0) -- (0,5) node[above] {$y$};

		\draw [red, ultra thick] plot [smooth, tension=0.7] coordinates { (A) (B) (C) (D) (E) (F) (G) (H)};

    \draw [orange] (1,3) -- (5,2);
    \draw [orange] (2,2) -- (6,3);
    \draw [orange] (3,1) -- (7,5);
	\end{tikzpicture}
    \caption{Funzione convessa}
\end{figure}

\begin{figure}[H]
	\centering
	\begin{tikzpicture}[scale=0.6, domain=0:10]
		\coordinate (A) at (0,0);
		\coordinate (B) at (1,2);
		\coordinate (C) at (2,3);
		\coordinate (D) at (3,4);
		\coordinate (E) at (4,4);
		\coordinate (F) at (5,3);
		\coordinate (G) at (6,2);
		\coordinate (H) at (7,0);

		\draw [->] (0,0) -- (7,0) node[right] {$x$};
		\draw [->] (0,0) -- (0,5) node[above] {$y$};

		\draw [blue, ultra thick] plot [smooth, tension=0.7] coordinates { (A) (B) (C) (D) (E) (F) (G) (H)};

    \draw [cyan] (1,2) -- (5,3);
    \draw [cyan] (2,3) -- (6,2);
    \draw [cyan] (3,4) -- (7,0);
	\end{tikzpicture}
    \caption{Funzione concava}
\end{figure}

In maniera equivalente, \( f \) è convessa se e solo se per ogni \( x \in I \), ogni
\( \overline{x} \in I \) ed ogni \( t \in [0,1] \), vale
\[
f(tx+(1-t))\overline{x}) \le tf(x) + (1-t)f(\overline{x})
\] 

\subsection{Proprietà delle funzioni convesse (o concave)}
\begin{enumerate}
  \item Se \( I \subseteq \mathbb{R} \) è un intervallo, \( f: I \to \mathbb{R} \) 
    una funzione differenziabile due volte. Se \( \begin{cases}
    f''(x) \ge 0\\
    f''(x) \le 0
  \end{cases} \) in ogni punto di \( I \), allora \( f \) è \( \begin{cases}
    \text{convessa}\\
    \text{concava}
  \end{cases} \) 
  \item Se \( I \subseteq \mathbb{R} \) è un intervallo, \( f: I \to \mathbb{R} \) è
    differenziabile e \( \begin{cases}
      \text{convessa}\\
      \text{concava}
    \end{cases} \) e \( x_0 \in I \), allora la retta tangente a \( f \) nel
    punto \( (x_0, f(x_0)) \) sta tutta \( \begin{cases}
      \text{sopra}\\
      \text{sotto}
    \end{cases} \) il grafico di \( f \). 
 		\begin{figure}[H]
			\begin{center}
        \begin{tikzpicture}[scale=1.5]
					\draw[->] (-2, 0) -- (2, 0) node[right] {$x$};
					\draw[->] (0, -1) -- (0, 2) node[above] {$y$};
					\draw[domain=-2:1.6, smooth, variable=\x, red, thick, yscale=0.5, xscale=0.5] plot ({\x}, {(\x*\x)+\x});
					\draw[domain=0:1.8, smooth, variable=\x, green, yscale=0.5, xscale=0.5] plot ({\x}, {(3*\x)-1});

          \draw[fill, red, scale=0.5] (1, 2) circle (3pt);
				\end{tikzpicture}
			\end{center}
		\end{figure}

  \item Se \( I \subseteq \mathbb{R} \) è un intervallo, \( f: I \to \mathbb{R} \) una
    funzione differenziabile e \( \begin{cases}
      \text{convessa}\\
      \text{concava}
    \end{cases} \) e \( x_0 \in I \) un punto critico di \( f \) \( (f'(x_0)=0) \),
    allora \( x_0 \) è un punto di \( \begin{cases}
      \text{minimo}\\
      \text{massimo}
    \end{cases} \) di \( f \).
     		\begin{figure}[H]
			\begin{center}
        \begin{tikzpicture}[scale=1.5]
					\draw[->] (-0.5, 0) -- (2, 0) node[right] {$x$};
					\draw[->] (0, -1) -- (0, 1) node[above] {$y$};
					\draw[domain=0:3, smooth, variable=\x, red, thick, yscale=0.5, xscale=0.5] plot ({\x}, {(\x*\x)-3*\x+1});
					\draw[domain=-1:4, smooth, variable=\x, orange, yscale=0.5, xscale=0.5] plot ({\x}, {-1.25});

          \draw[fill, red, scale=0.5] (1.5, -1.25) circle (3pt);
				\end{tikzpicture}
			\end{center}
		\end{figure}

 		\begin{figure}[H]
			\begin{center}
        \begin{tikzpicture}[scale=1.5]
					\draw[->] (-0.5, 0) -- (2, 0) node[right] {$x$};
					\draw[->] (0, -1) -- (0, 1) node[above] {$y$};
					\draw[domain=0:3, smooth, variable=\x, blue, thick, yscale=0.5, xscale=0.5] plot ({\x}, {-(\x*\x)+3*\x-1});
					\draw[domain=-1:4, smooth, variable=\x, cyan, yscale=0.5, xscale=0.5] plot ({\x}, {1.25});

          \draw[fill, blue, scale=0.5] (1.5, 1.25) circle (3pt);
				\end{tikzpicture}
			\end{center}
		\end{figure}

  \item Se \( f:I \to \mathbb{R} \) è differenziabile due volte e \( x_0 \in I \) è
    tale che \( f'(x_0) = 0 \) e \( \begin{cases}
      f''(x_0) > 0\\
      f''(x_0) < 0
    \end{cases} \) allora \( x_0 \) è un punto di \( \begin{cases}
      \text{minimo}\\
      \text{massimo}
    \end{cases} \) locale per \( f \).
\end{enumerate}

\begin{figure}[H]
  \begin{example}
    \[
    f: \mathbb{R}\to \mathbb{R}
    \] 
    \[
    f(x) = x^3-x^2\;\;\; \forall x \in \mathbb{R}
    \] 
    \[
    f'(x) = 3x^2-2x
    \] 
    \[
    f''(x) = 6x-2
    \] 
		\begin{figure}[H]
			\begin{center}
        \begin{tikzpicture}[scale=1.3]
					\draw[->] (-2, 0) -- (2, 0) node[right] {$x$};
					\draw[->] (0, -2) -- (0, 2) node[above] {$y$};
					\draw[domain=-1:1.7, smooth, variable=\x, red, thick, yscale=1, xscale=1] plot ({\x}, {(\x*\x*\x)-(\x*\x)});

          \draw[fill, red] (0,0) circle (1.5pt);
          \draw[fill, red] (1,0) circle (1.5pt);
          \draw (0.66, -0.15) -- (0.66, 0) node[above, scale=0.8] {\( \frac{2}{3} \) }; 

          \draw[orange, thick, ->] (-1.5, 0.5) -- (-0.5, 1.5);
          \draw[orange, thick, ->] (0.5, 1.5) -- (1, 0.5);
          \draw[orange, thick, ->] (1.5, 0.5) -- (2, 1.5);
				\end{tikzpicture}
			\end{center}
		\end{figure}


    \[
    f(x) = 0 \Leftrightarrow x^2(x-1) = 0 \Leftrightarrow x = 0\;\; oppure
    \] 
    \[
    f(x) \ge 0 \Leftrightarrow x^2(x-1) \ge 0 \Leftrightarrow x \ge 1\;\; oppure\;\; x = 0
    \] 
    \[
    f'(x) = 0 \Leftrightarrow x=0 \;\; oppure\;\; x=\frac{2}{3}
    \] 
    \[
    f'(x) \ge 0 \Leftrightarrow x \le 0 \;\;oppure\;\; x \ge \frac{2}{3}
    \] 
    Quindi \( f \) è cresccente in \( (-\infty, 0) \) e in \( \frac{2}{3}, +\infty \); \( f \) 
    è decrescente in \( (0, \frac{2}{3}) \). In \( x=0 \) ho un punto di massimo locale,
    in \( x=\frac{2}{3} \) ho un punto di minimo locale.
    \[
    f''(x) = 6x-2
    \] 
    \[
    f''(x) = 0 \Leftrightarrow x = \frac{1}{3}
    \] 
    \[
    f''(x) \ge 0 \Leftrightarrow x \ge \frac{1}{3}
    \] 
    \( f \) è convessa in \( (\frac{1}{3}, +\infty) \) e concava in \( (-\infty, \frac{1}{3}) \);
    \( f \) ha \( x = \frac{1}{3} \) è un punto di flesso di \( f \).
  \end{example}
\end{figure}

\section{Teoremi}
\subsection{Teorema dei carabinieri}
\begin{theorem}[del confronto tra i limiti, o dei carabinieri]
	Sia \( A \subseteq \mathbb{R} \) un insieme non limitato superiormente e siano
	\( f,g,h: A \to \mathbb{R} \). Supponiamo che
	\[
		f(x) \le g(x) \le h(x)\;\;\;\forall x \in A
	\]
	Supponiamo inoltre che i limiti
	\[
		\lim_{x \to +\infty} f(x) = \lim_{x \to +\infty} h(x) = L
	\]
	esistano (e che siano uguali tra di loro). Allora
	\[
		\lim_{x \to +\infty} g(x) = L
	\]
	\begin{figure}[H]
		\begin{center}
			\begin{tikzpicture}
				\draw[->] (-0.2, 0) -- (5, 0) node[right] {$x$};
				\draw[->] (0, -0.2) -- (0, 4) node[above] {$y$};
				\draw[domain=0.48:5, smooth, variable=\x, blue, thick] plot ({\x}, {(-1/\x)+2}) node [below right] {h};
				\draw[domain=0.48:5, smooth, variable=\x, green, thick] plot ({\x}, {(1/\x)+2}) node [above right] {f};
				\draw[domain=0.48:10, smooth, variable=\x, red, thick, xscale=0.5] plot ({\x}, {(sin(\x r)/\x)+2}) node [right] {g};

				\draw [dashed] (0,2) node[left] {L} -- (5,2);
			\end{tikzpicture}
		\end{center}
		\caption{Teorema del confronto tra i limiti}
	\end{figure}
	Dobbiamo dimostrare che
	\[
		\forall \epsilon > 0\;\; \exists k>0\; \;t.c.\;\; \forall x \in A,
	\]
	\[
		x \ge k \to L-\epsilon \le g(x) \le L+\epsilon
	\]
	Prendiamo dunque \( \epsilon>0 \) arbitrario. Poichè \( \lim_{x \to +\infty} f(x)=L\),
	sappiamo che esiste \( k_f >0 \) t.c.
	\[
		\forall x \in A,\;\;\;\; x \ge k_f \to L-\epsilon \le f(x) \le L+\epsilon
	\]
	Allo stesso modo, poichè \( \lim_{x \to +\infty} h(x)=L\),
	sappiamo che esiste \( k_h >0 \) t.c.
	\[
		\forall x \in A,\;\;\;\; x \ge k_h \to L-\epsilon \le h(x) \le L+\epsilon
	\]
	Definiamo \( k:= max\{k_f,k_h\} \). Comunque preso \( x \in A \),
	se \( x \ge k \) allora vale che
	\[
		L-\epsilon \le f(x) \le g(x) \le h(x) \le L+\epsilon
	\]

\end{theorem}

\subsection{Teorema di Weiestrass}
\begin{definition}
	Teorema di Weierstrass

	Sia \( [a, b] \) un intervallo chiuso e limitato, \( f: [a,b]  \to \mathbb{R}\) continua.

	Allora esistono
	\[
		x_{max}, x_{min} \in [a,b]\;\; t.c.\;\; f(x_{min}) \le f(x) \le f(x_{max}) \forall x \in [a,b]
	\]
	\begin{figure}[H]
		\begin{center}
			\begin{tikzpicture}
				\draw[->] (-0.2, 0) -- (4, 0) node[right] {$x$};
				\draw[->] (0, -1.5) -- (0, 1.5) node[left] {$y$};
				\draw[domain=0.5:6, smooth, variable=\x, red, thick, xscale=0.5] plot ({\x}, {(sin(\x r)}) node [right] {f};

				\draw[dashed] (0.25, 0.5) -- (0.25, 0) node[below] {$a$};
				\draw[dashed] (3, -0.3) -- (3, 0) node[above] {$b$};

				\draw[dashed, blue] (0.78, 1) -- (0.78, 0) node[below, scale=0.6] {$x_{max}$};
				\draw[dashed, blue] (2.38, -1) -- (2.38, 0) node[above, scale=0.6] {$x_{min}$};
			\end{tikzpicture}
		\end{center}
		\caption{Teorema di Weiestrass}
	\end{figure}
	Ogni funzione continua, avrà quindi un punto di minimo e un punto di massimo
\end{definition}
\subsubsection{Osservazioni}
\begin{itemize}
	\item In particolare, \( f \) è limitata
	\item I punti \( x_{min}, x_{max} \) si dicono punti di minimo e di massimo \textbf{globali} di \( f \)
	\item I punti di minimo e massimo globali possono essere non unici e coincidere con gli estremi
	      \( a,b \) dell'intervallo
\end{itemize}

\begin{example}
	\[
		cos: [0, 4 \pi] \to \mathbb{R}
	\]
	\begin{figure}[H]
		\begin{center}
			\begin{tikzpicture}[xscale=0.5]
				\draw[->] (-0.2, 0) -- (13.5, 0) node[right] {$x$};
				\draw[->] (0, -1.5) -- (0, 1.5) node[left] {$y$};
				\draw[domain=0:12.5, smooth, variable=\x, red, thick,yscale=0.5] plot ({\x}, {(cos(\x r)}) node [right] {};

				\node [below left, scale=0.9, green] at (0,0) (0) {0};
				\node [above, scale=0.9, blue] at (3.14,0) (pi) {$\pi$};
				\node [below, scale=0.9, green] at (6.28,0) (2pi) {$2\pi$};
				\node [above, scale=0.9, blue] at (9.42,0) (3pi) {$3\pi$};
				\node [below, scale=0.9, green] at (12.56,0) (4pi) {$4\pi$};

				\draw (3.14,-0.08) -- (3.14, 0.08);
				\draw (6.28,-0.08) -- (6.28, 0.08);
				\draw (9.42,-0.08) -- (9.42, 0.08);
				\draw (12.56,-0.08) -- (12.56, 0.08);

				\node[align=center, scale=0.8, blue] at (6,1.5) (globalmax) {Punti di massimo\\
					globale};
				\node[align=center, scale=0.8, green] at (6,-1.5) (globalmin) {Punti di minimo\\
					globale};

				\draw[blue] (globalmax) -- (pi);
				\draw[blue] (globalmax) -- (3pi);

				\draw[green] (globalmin) -- (0);
				\draw[green] (globalmin) -- (2pi);
				\draw[green] (globalmin) -- (4pi);

			\end{tikzpicture}
		\end{center}
		\caption{Teorema di Weiestrass}
	\end{figure}
\end{example}
Se vengono meno le ipotesi del teorema, può venir meno la conclusione.
\subsubsection{Esempi}
\begin{figure}[H]
	\begin{example}
		\[
			f_1: (0,1) \to \mathbb{R},\;\;\;\; f_1(x):= x \;\;\; \forall x \in (0,1)
		\]
		\begin{figure}[H]
			\begin{center}
				\begin{tikzpicture}[scale=2]
					\draw[->] (-0.2, 0) -- (1.2, 0) node[right] {$x$};
					\draw[->] (0, -0.2) -- (0, 1.2) node[left] {$y$};
					\draw[domain=0.04:0.96, smooth, variable=\x, red, thick] plot ({\x}, {\x}) node [right] {};

					\draw[red] (0,0) circle (1.5pt);
					\draw[red] (1,1) circle (1.5pt);

					\draw[dashed, red] (1,1) -- (1,0) node [below, scale=0.7] {1};
					\node[below left, red, scale=0.7] at (0,0) {0};
				\end{tikzpicture}
			\end{center}
			\caption{Esempio di funzione continua}
		\end{figure}
		Questa funzione è continua, ma per come è definita \textbf{non} ammette nè massimo nè minimo perchè
		il \textbf{dominio non è chiuso}.
	\end{example}
\end{figure}
\begin{figure}[H]
	\begin{example}
		\[
			f_2: (0, +\infty) \to \mathbb{R},\;\;\;\; f_2(x):= x sin(x)\;\;\; \forall x \in (0, +\infty)
		\]

		\begin{figure}[H]
			\begin{center}
				\begin{tikzpicture}[xscale=0.5, yscale=0.2]
					\draw[->] (-0.2, 0) -- (13.5, 0) node[right] {$x$};
					\draw[->] (0, -13.5) -- (0, 13.5) node[left] {$y$};
					\draw[domain=0:12.56, smooth, variable=\x, red, thick] plot ({\x}, {(\x * sin(\x r)}) node [right] {};
					\draw[domain=0:12.5, dashed, variable=\x] plot ({\x}, {\x}) node [right] {$y = x$};
					\draw[domain=0:12.5, dashed, variable=\x] plot ({\x}, {-\x}) node [right] {$y = -x$};
				\end{tikzpicture}
			\end{center}
			\caption{Esempio di funzione continua}
		\end{figure}

		Questa funzione è continua, ma \textbf{non} possiede nè punti di massimo, nè punti di minimo perchè
		la funzione ha ampiezza sempre crescente.
	\end{example}
\end{figure}
\begin{example}
	\[
		f_3: [-1, 1] \to \mathbb{R}
	\]
	\[
		f_3(x):=\begin{cases}
			1-x \;\;\; se\; 0 < x \le 1 \\
			0 \;\;\; se\; x = 0         \\
			-x-1 \;\;\; se\; -1 \le x < 0
		\end{cases}
	\]
	\begin{figure}[H]
		\begin{center}
			\begin{tikzpicture}[scale=2]
				\draw[->] (-1.2, 0) -- (1.2, 0) node[right] {$x$};
				\draw[->] (0, -1.2) -- (0, 1.2) node[left] {$y$};
				\draw[domain=0.04:1, smooth, variable=\x, red, thick] plot ({\x}, {1-\x}) node [right] {};
				\draw[domain=-1:-0.04, smooth, variable=\x, red, thick] plot ({\x}, {-\x-1}) node [right] {};

				\draw[red, fill] (0,0) circle (1.5pt);
				\draw[red] (0,1) circle (1.5pt);
				\draw[red] (0,-1) circle (1.5pt);

				\node[below, red, scale=0.7] at (1,0) {1};
				\node[below, red, scale=0.7] at (-1,0) {-1};
				\node[below left, red, scale=0.7] at (0,0) {0};

				\node[left, red, scale=0.7, xshift=-5] at (0,1) {1};
				\node[right, red, scale=0.7, xshift=5] at (0,-1) {-1};
			\end{tikzpicture}
		\end{center}
		\caption{Esempio di funzione non continua}
	\end{figure}

	Questa funzione \textbf{non} ammette punti di massimo e di minimo perchè non è continua.
\end{example}
\subsection{Teorema degli zeri}
\begin{figure}[H]
	\begin{definition}
		Teorema degli zeri (o di Bolzano)

		Sia \( [a,b] \) un intervallo chiuso e limitato, \( f: [a,b] \to \mathbb{R} \) una funzione continua.
		Se
		\[
			f(a)f(b) < 0
		\]
		allora esiste \( c \in (a,b) \) tale che \( f(c) = 0 \)

		\begin{figure}[H]
			\begin{center}
				\begin{tikzpicture}[scale=0.5, domain=0:13]
					\draw[->] (-0.5,0) -- (8,0) node[right] {$x$};

					\coordinate (A) at (0,3);
					\coordinate (B) at (1,2);
					\coordinate (C) at (2,3);
					\coordinate (D) at (3,1);
					\coordinate (E) at (4,0);
					\coordinate (F) at (5,-2);
					\coordinate (G) at (6,-2);
					\coordinate (H) at (7,-2.5);

					\draw [red, thick] plot [smooth, tension=0.7] coordinates { (A) (B) (C) (D) (E) (F) (G) (H) };

					\draw (0,0.2) -- (0,-0.2) node[below] {$a$};
					\draw (7,0.2) -- (7,-0.2) node[below] {$b$};

					\draw[red, fill] (0,3) circle (3pt);
					\draw[red, fill] (7,-2.5) circle (3pt);
					\draw[red, fill] (4,0) circle (3pt);

				\end{tikzpicture}
			\end{center}
			\caption{Teorema degli zeri}
		\end{figure}

	\end{definition}
\end{figure}
Se vengono meno le ipotesi, può venir meno la conclusione.
\subsubsection{Esempi}
\begin{example}
	\[
		g_1: [-1,1] \to \mathbb{R}
	\]
	\[
		g_1(x)= \begin{cases}
			-1 \;\;\; se\; -1 \le x <0 \\
			1 \;\;\; se\; 0 \le x \le 1
		\end{cases}
	\]
	\begin{figure}[H]
		\begin{center}
			\begin{tikzpicture}[scale=2]
				\draw[->] (-1.2, 0) -- (1.2, 0) node[right] {$x$};
				\draw[->] (0, -1.2) -- (0, 1.2) node[left] {$y$};
				\draw[domain=0.04:1, smooth, variable=\x, red, thick] plot ({\x}, {0.5}) node [right] {};
				\draw[domain=-1:-0.04, smooth, variable=\x, red, thick] plot ({\x}, {-0.5}) node [right] {};

				\draw[red, fill] (0,0.5) circle (1.5pt);
				\draw[red] (0,-0.5) circle (1.5pt);
			\end{tikzpicture}
		\end{center}
		\caption{Esempio di funzione non continua}
	\end{figure}

	Questa funzione non è continua, quindi non si applica il teorema.
\end{example}

\begin{example}
	\[
		g_2: [-1,1] / \{0\} \to \mathbb{R}
	\]
	\[
		g_2(x):= \frac{1}{x}\;\;\; \forall x \in [-1,1] / \{0\}
	\]
	\begin{figure}[H]
		\begin{center}
			\begin{tikzpicture}
				\draw[->] (-1, 0) -- (1, 0) node[right] {$x$};
				\draw[->] (0, -3) -- (0, 3) node[left] {$y$};
				\draw[domain=0.17:1, smooth, variable=\x, red, thick, yscale=0.5, xscale=0.5] plot ({\x}, {1/\x}) node [right] {};
				\draw[domain=-1:-0.17, smooth, variable=\x, red, thick, yscale=0.5, xscale=0.5] plot ({\x}, {1/\x}) node [right] {};

				\draw[red, fill] (0.5,0.5) circle (1.5pt);
				\draw[red, fill] (-0.5,-0.5) circle (1.5pt);
			\end{tikzpicture}
		\end{center}
		\caption{Esempio di funzione continua}
	\end{figure}

	Questa funzione è continua, ma non si annulla mai perchè il dominio della funzione \textbf{non è un intervallo}
	,ma un intervallo privato di un valore, quindi non si applica il teorema.
\end{example}


\subsection{Teorema di Fermat}
\begin{theorem}[di Fermat]
  Sia \( x_0 \in A \) un estremo locale di una funzione \( f: A \to \mathbb{R} \).
  Se \( f \) è differenziabile in \( x_0 \) e se \( x_0 \) è \textbf{interno} ad \( A \)
  (cioè, \( f \) è definita in un intorno di \( x_0 \)), allora:
  \[
  f'(x_0)=0
  \] 
  (I punti dove \( f'(x_0)=0 \) si dicono \textbf{punti critici di \( f \) })
  \begin{figure}[H]
	\begin{center}
		\begin{tikzpicture}[scale=0.8, domain=0:13]
			\draw[->] (-0.5,0) -- (8.5,0) node[right] {$x$};
			\draw[->] (0,-0.5) -- (0,7) node[above] {$y$};

			\coordinate (C) at (2,7);
			\coordinate (D) at (3,6);
			\coordinate (E) at (4,2);
			\coordinate (F) at (5,3);
			\coordinate (G) at (6,4);
			\coordinate (H) at (7,6);
			\coordinate (I) at (8,5);

			\draw [red, thick] plot [smooth, tension=0.7] coordinates { (C) (D) (E) (F) (G) (H) (I) };

      \draw[blue] (0,1.85) -- (8.5,1.85);
      \draw[fill, blue] (4.2,1.85) circle (2pt);

      \draw[dashed] (4.2, 1.85) -- (4.2, 0) node[below, scale=0.8] {\( x_0 \)};
      \draw[dashed] (5, 3) -- (5, 0) node[below, scale=0.8, xshift=10] {\( x_0 + h \)};
		\end{tikzpicture}
	\end{center}
  \caption{Teorema di Fermat}
\end{figure}
\end{theorem}

\subsubsection{Dimostrazione}
Supponiamo ad esempio \( x_0 \) minimo locale di \( f \). Prendo \( h \in \mathbb{R} \),
\( h \neq 0 \). Se \( |h| \) è abbastanza piccolo,
\[
f(x_0+h)-f(x_0) \ge 0\;\;\;\;\text{perchè \( x_0 \) è minimo locale}
\] 
Se \( h > 0 \):
\[
  \frac{f(x_0+h) - f(x_0)}{h} \ge 0
\] 
Se \( h < 0 \):
\[
\frac{f(x_0)-f(x_0)}{h} \le 0
\] 
Poichè \( f \) è differenziabile in \( x_0 \), so che esistono:
\[
  \lim_{h \to 0^+} \frac{f(x_0 + h) - f(x_0)}{h} 
\] 
\[
  \lim_{h \to 0^-} \frac{f(x_0 + h) - f(x_0)}{h} 
\] 
e i due limiti sono uguali tra loro e uguali a \( f'(x_0) \). L'unica possibilità \( f'(x_0)=0 \) 

\subsection{Teorema di Lagrange}
\begin{figure}[H]
  \begin{definition}
    Teorema di Lagrange o del valor medio. Sia \( f: [a,b]  \to \mathbb{R}\) una funzione
    continua in \( [a,b] \) e differenziabile in \( (a,b) \). Allora esiste 
    \( c \in (a,b) \) tale che:
    \[
      f'(c) = \frac{f(b)-f(a)}{b-a}
    \]
      \begin{figure}[H]
	\begin{center}
		\begin{tikzpicture}[scale=0.8, domain=0:8.5]
			\draw[->] (-0.5,0) -- (8.5,0) node[right] {$x$};
			\draw[->] (0,-0.5) -- (0,6) node[above] {$y$};

			\coordinate (C) at (2,2);
			\coordinate (D) at (3,1);
			\coordinate (E) at (4,1.5);
			\coordinate (F) at (5,3);
			\coordinate (G) at (6,5);
			\coordinate (H) at (7,6);
			\coordinate (I) at (8,6);

			\draw [red, thick] plot [smooth, tension=0.7] coordinates { (C) (D) (E) (F) (G) (H) (I) };

      \draw[blue] (2,2) -- (8,6);
      \draw[dashed] (2, 2) -- (2, 0) node[below] {\( a \)};
      \draw[dashed] (8, 6) -- (8, 0) node[below] {\( b \)};
      
      \draw[blue] (2,2) -- (8,6);
      \draw[green, yshift=20] (5,4) -- (8,6);
      \draw[green, yshift=-56] (2,2) -- (5,4);

      \draw[fill, green] (3.6,1.1) circle (2pt);
      \draw[dashed, green] (3.6,1.1) -- (3.6, 0) node[below] {\( c \)}; 

      \draw[fill, green] (6.75,5.85) circle (2pt);
      \draw[dashed, green] (6.75,5.85) -- (6.75, 0) node[below] {\( c_1 \)};

		\end{tikzpicture}
	\end{center}
  \caption{Teorema di Lagrange}
\end{figure}

  \end{definition}
\end{figure}
\textbf{Corollario}: Sia \( I \) un intervallo, \( f: I \to \mathbb{R} \) una funzione
differenziabile se:
\[
\begin{cases}
  f'=0\\
  f'\ge 0\\
  f'>0\\
  f'\le 0\\
  f'<0
\end{cases}
\] 
in tutti i punti di \( I \), allora \( f \) è:
\[
\begin{cases}
  \text{costante}\\
  \text{non decrescente}\\
  \text{strettamente crescente}\\
  \text{non crescente}\\
  \text{strettamente decrescente}
\end{cases}
\] 
Qui è importante assumere che il dominio sia un intervallo
\begin{figure}[H]
  \begin{example}
    \[ f: (0,1) \cup (2,3) \to \mathbb{R}, \]
    \[
      f(x):=\begin{cases}
        -1 \;\;\; se\; 0<x<1\\
        \frac{1}{2} \;\;\; se\; 2<x<3
      \end{cases}
    \] 
    \[
    \lim_{h \to 0} \frac{f(x_0+h)-f(x_0)}{h} = 0
    \] 
          \begin{figure}[H]
	\begin{center}
		\begin{tikzpicture}[scale=1, yscale=2, domain=0:8.5]
			\draw[->] (-0.5,0) -- (4,0) node[right] {$x$};
			\draw[->] (0,-1.3) -- (0,1) node[above] {$y$};

      \draw (1,0.05) -- (1,-0.05) node[below] {1};
      \draw (2,0.05) -- (2,-0.05) node[below] {2};
      \draw (3,0.05) -- (3,-0.05) node[below] {3};

      \node[left] at (0, -1) {-1};

      \draw (-0.1,0.5) -- (0.1,0.5) node[left, xshift=-2] {\( \frac{1}{2} \) };

      \draw[red, yscale=0.5] (0, -2) circle (2.5pt);
      \draw[red, yscale=0.5] (1, -2) circle (2.5pt);
      \draw[red, thick] (0.09,-1) -- (0.91,-1);

      \draw[red, yscale=0.5] (2, 1) circle (2.5pt);
      \draw[red, yscale=0.5] (3, 1) circle (2.5pt);
      \draw[red, thick] (2.09,0.5) -- (2.91,0.5);

      \draw[green, dashed] (0.5, -1) -- (0.5, 0) node[above] {\( x_0 \)};
		\end{tikzpicture}
	\end{center}
  \caption{Teorema di Lagrange}
\end{figure}

    \( f \) è differenziabile e ha \( f'=0 \) ovunque, ma non è costante (il dominio non è un intervallo).
  \end{example}
\end{figure}
\subsubsection{Dimostrazione}
\begin{enumerate}
  \item 
    Sia \( f: [a,b] \to \mathbb{R} \) continua, differenziabile in \( (a,b) \) , tale
    che \( f(a)=f(b) \).

    Devo dimostrare che esiste \[ c \in (a,b)\;tc\; f'(c)=0 \]
    Per il teorema di Weierstrass, \( f \) possiede un punto di massimo \( x_{max} \) 
    e un punto di minimo \( x_{min} \) globali. 

    Se \( x_{min} \in (a,b) \), allora scelgo \( c:= x_{min} \) e per il teorema di Fermat,
    so che \( f'(c)=0 \).

    Se \( x_{max} \in (a,b) \), allora scelgo \( c:=x_{max} \) e per il teorema di Fermat,
    so che \( f'(c)=0 \).

    Altrimenti, ho \( \{x_{max},x_{min}\}= \{a,b\}   \). Grazie all'ipotesi \( f(a)=f(b) \),
    posso allora dedurre che \( f \) è costante, dunque \( f'=0 \) in tutto \( [a,b] \).
  \item \textbf{Caso generale:}
    Definisco \( g: [a,b] \to \mathbb{R} \),
    \[
      g(x):= f(x)-\underbrace{\left(\frac{f(b)-f(a)}{b-a}(x-a)+f(a)\right)}_{\text{Equazione della corda AB}}\;\;\;\forall x \in [a,b]
    \] 
    Ora \( g \) è continnua, \( g \) è differenziabile, in \( (a,b) \),
    \[
    g(a)=f(a) - f(a) = 0
    \] 
    \[
    g(b) = f(b) -f(b) = 0
    \] 
    Dunque, per quanto dimostrato nel passo precedente, esiste \( c \in (a,b) \) tale che
    \[
    g'(c)=0
    \] 
    perchè:
    \[
    g'(x) = f'(x) - \frac{f(b)-f(a)}{b-a}
    \] 
\end{enumerate}
\subsubsection{Dimostrazione del corollario}
Prendo \( a \in I, b \in I \) qualsiasi; devo dimostrare che \( f(a) = f(b) \).
Se \( a = b \), non c'è nulla da dimostrare. Suppongo ad esempio \( a < b \) (se no li scambio).
Allora \( f \) è definita su tutto \( [a,b] \) (perchè \( I \) è un intervallo, dunque \( [a,b] \supseteq I\) ).
\begin{figure}[H]
  \begin{center}
    \begin{tikzpicture}
      \draw[->] (0,0) -- (5,0);
      \draw[->] (6,0) -- (9,0);

      \draw[fill, blue, opacity=0.3] (0.5,-0.2) rectangle (2.5,0.2);
      \draw[fill, blue, opacity=0.3] (3.5,-0.2) rectangle (4.5,0.2);
      \draw (1.5,0.1) -- (1.5,-0.1) node[below, yshift=-1] {\( a \)};
      \draw (4,0.1) -- (4,-0.1) node[below, yshift=-1] {\( b \)};

      \node at (3,-0.5) {I};

      \draw[fill, red, opacity=0.3] (6.5,-0.2) rectangle (8.5,0.2);
      \draw (7,0.1) -- (7,-0.1) node[below, yshift=-1] {\( a \)};
      \draw (8,0.1) -- (8,-0.1) node[below, yshift=-1] {\( b \)};

      \node at (7.5,-0.5) {I};
    \end{tikzpicture}
  \end{center}
\end{figure}
Inoltre \( f:[a,b] \to \mathbb{R} \) è continua (differenziabile \( \Rightarrow \) continua),
differenziabile in \( (a,b) \) e quindi, per il teorema di Lagrange, esiste \( c \in (a,b) \) t.c.
\[
f'(c)=\frac{f(b)-f(a)}{b-a}
\] 
Ma, per ipotesi, \( f'(c)=0 \), da cioè \( f(b)-f(a)=0 \), cioè \( f(b)=f(a) \).

\subsection{Teorema de l'Hopital}
Si applica al calcolo dei limiti della forma \( \lim_{x \to x_0} \frac{f(x)}{g(x)} \), con \( f,g \) 
funzioni differenziabili, \textbf{purchè} il limite si presenti sotto la forma (indeterminata)
\( \frac{0}{0} \) oppure \( \frac{\infty}{\infty} \). Il teorema riduce il calcolo del limite
dato al calcolo di:
\[
\lim_{x \to x_0} \frac{f'(x)}{g'(x)} \;\;\;\; \text{(purchè esista)}
\] 

\begin{figure}[H]
  \begin{definition}
    Siano \( f,g: [a,+\infty) \to \mathbb{R} \) due funzioni differenziabili. Supponiamo che:
    \[
    \lim_{x \to +\infty} f(x) = \lim_{x \to +\infty} g(x) = 0
    \] 
    oppure
    \[
    \lim_{x \to +\infty} f(x) = \lim_{x \to +\infty} g(x) = \pm \infty
    \] 
    Supponiamo inoltre che \( g'(x) \neq 0\;\;\; \forall x\; \in  [a,+\infty)  \) e che il
    limite
    \[
    \lim_{x \to +\infty} \frac{f'(x)}{g'(x)} 
    \] 
    esista. Allora
    \[ \lim_{x \to +\infty} \frac{f(x)}{g(x)} = \lim_{x \to +\infty} \frac{f'(x)}{g'(x)}  \]
  \end{definition}
\end{figure}
Si possono scrivere varianti per il calcolo dei limiti quando \( x \to x_0 \) con \( x \in \mathbb{R} \) 
oppure \( x \to -\infty \).

\subsubsection{Esempi}
\begin{figure}[H]
  \begin{example}
    \[
    \lim_{x \to +\infty} \frac{e^x}{x}
    \] 
    Considero il rapporto tra le derivate:
    \[
    \lim_{x \to +\infty} \frac{e^x}{1} = +\infty
    \] 
    Per il teorema di de l'Hopital:
    \[
    \lim_{x \to +\infty} \frac{e^x}{x} = +\infty
    \] 
  \end{example}
\end{figure}

\begin{figure}[H]
  \begin{example}
    \[
    \lim_{x \to +\infty} \frac{e^x}{x^2}
    \] 
    Considero il rapporto tra le derivate:
    \[
    \lim_{x \to +\infty} \frac{e^x}{2x} = \frac{1}{2} \lim_{x \to +\infty} \frac{e^x}{x} = +\infty
    \] 
    Per il teorema di de l'Hopital:
    \[
    \lim_{x \to +\infty} \frac{e^x}{x^2} = +\infty
    \] 
  \end{example}
\end{figure}

Si può dimostrare che per ogni \( \alpha > 0 \),
\[
\lim_{x \to +\infty} \frac{e^x}{x^\alpha} = +\infty,
\] 
\[
\lim_{x \to +\infty} \frac{log(x)}{x^\alpha} = 0
\] 
	\begin{figure}[H]
		\begin{center}
			\begin{tikzpicture}
				\draw[->] (-0.5, 0) -- (3, 0) node[right] {$x$};
				\draw[->] (0, -1.5) -- (0, 3) node[left] {$y$};
				\draw[domain=-0.5:1.1, smooth, variable=\x, red, thick, yscale=1, xscale=1] plot ({\x}, {e^\x}) node [right] {\( e^x \) };
				\draw[domain=0:1.75, smooth, variable=\x, green, thick, yscale=1, xscale=1] plot ({\x}, {\x^2}) node [right] {\( x^2 \) };
				\draw[domain=0:3, smooth, variable=\x, green, thick, yscale=1, xscale=1] plot ({\x}, {\x}) node [right] {\( x \) };
        \draw[domain=0:3, smooth, variable=\x, green, thick, yscale=1, xscale=1] plot ({\x}, {\x^(1/2)}) node [right] {\( x^{\frac{1}{2}} \) };
				\draw[domain=0.2:3, smooth, variable=\x, blue, thick, yscale=1, xscale=1] plot ({\x}, {ln(\x)}) node [right] {\( \ln(x) \) };
			\end{tikzpicture}
		\end{center}
    \caption{Confronto tra funzioni}
	\end{figure}

Quando \( x \to +\infty \), l'esponenziale cresce più velocemente di tutte le potenze
(ad esponente positivo); il logaritmo più lentamente.

\section{Sviluppi di Taylor}
Sia \( f: I \to \mathbb{R} \) (con \( I \subseteq \mathbb{R} \) intervallo) una funzione
differenziabile \( x_0 \in I \). Per definizione di differenziabilità:
\[
  f(x) = \underbrace{f(x_0) + f'(x_0)(x-x_0)}_{\text{retta tangente valutata in \( x_0 \) }} + o(x-x_0) \;\;\; per\; x \to x_0
\] 
\begin{figure}[H]
	\begin{center}
		\begin{tikzpicture}[scale=0.8, domain=0:8.5]
			\draw[->] (-0.5,0) -- (5.5,0) node[right] {$x$};
			\draw[->] (0,-0.5) -- (0,6) node[above] {$y$};

			\coordinate (C) at (0,1.9);
			\coordinate (D) at (1,2);
			\coordinate (E) at (2,1.9);
			\coordinate (F) at (3,4);
			\coordinate (G) at (4,5.5);
			\coordinate (H) at (5,6.1);

			\draw [red, thick] plot [smooth, tension=0.7] coordinates { (C) (D) (E) (F) (G) (H) };

      \draw[dashed] (2, 1.9) -- (2, 0) node[below] {\( x_0 \)};
      \draw[dashed] (2.5, 2.8) -- (2.5, 0) node[below] {\( x \)};
      
      \draw[orange] (0,0.45) -- (5,4);

      \draw[fill, orange] (2,1.9) circle (2pt);
		\end{tikzpicture}
	\end{center}
  \caption{Sviluppo di Taylor}
\end{figure}

Se \( f \) è differenziabile due o più volte, si possono dare approssimazioni locali ancora
migliori.

\subsection{Notazione}
Dato \( n \in \mathbb{Z},\; n \ge 0 \), si definisce il fattoriale di \( n \) come:
\[
\begin{cases}
  0! := 1 \hspace{3.6cm} se \; n = 0\\
  n! := 1 \cdot 2 \cdot 3 \cdot \ldots \cdot (n-1) \cdot n \;\;\; se\; n \ge 0
\end{cases}
\] 
\subsection{Polinomi di Taylor}
Sia \( I \subseteq \mathbb{R} \) intervallo, \( f: I \to \mathbb{R} \) differenziabile \( n \) volte,
\( x_0 \in I \). Si definisce il \textbf{Polinomio di Taylor} di \( f \) di centro \( x_0 \) 
ed ordine \( n \) come:
\[
P(x) = \sum_{j=0}^{n} \frac{f^{(j)}(x_0)}{j!}(x-x_0)^j
\]  
Quando \( x_0 = 0 \), si parla anche di polinomio di McLaurin.

\begin{figure}[H]
  \begin{example}
    Calcolare il polinomio di Taylor di exp di centro 0 e ordine 7.
    \[
    exp: \mathbb{R} \to \mathbb{R}, \;\;\; x_0 = 0,\;\;\; n = 7
    \] 
    \[
    exp(x) = e^x
    \] 
    \[
    exp'(x) = exp
    \] 
    \[
    exp''(x) = exp' = exp
    \] 
    \[
      exp^{(j)} = exp \;\;\; \forall j \in \mathbb{N}
    \] 
    \[
    exp(0) = 1
    \] 
    Polinomio di Taylor:
    \[
      P(x) = \sum_{j=0}^{7} \frac{1}{j!}x^j = 1 + x + \frac{x^2}{2} + \ldots + \frac{x^7}{7!}
    \] 
  \end{example}
\end{figure}

\begin{figure}[H]
  \begin{theorem}
    Sia \( I \subseteq \mathbb{R} \) intervallo, \( x_0 \in I \), \( f: I \to \mathbb{R} \) una
    funzione differenziabile \( n \) volte, \( P \) il suo polinomio di Taylor di centro \( x_0 \) 
    ed ordine \( n \). Allora:
    \[
    f(x) = P(x) + o((x-x_0)^n)\;\;\;\;\;\; per\; x \to x_0
    \] 
  \end{theorem}
\end{figure}

\subsection{Polinomi notevoli}
\begin{itemize}
  \item \[
      e^x = 1+x+\frac{x^2}{2} + \frac{x^3}{3!} + \frac{x^4}{4!} + \ldots + \frac{x^n}{n!} + o(x^n)
  \] 
  \item \[
    \sin(x) = x - \frac{x^3}{3!} + \frac{x^5}{5!} - \frac{x^7}{7!} + \ldots + (-1)^n \frac{x^{2n+1}}{(2n+1)!} + o(x^{2n+1})
\] 
  \item
    \[
    \cos(x) = 1 - \frac{x^2}{2!} + \frac{x^4}{4!} - \frac{x^8}{8!} + \ldots + (-1)^n \frac{x^{2n}}{(2n)!} + o(x^{2n})
    \] 
  \item
    \[
    \log(1+x) = x - \frac{x^2}{2} + \frac{x^3}{3} - \frac{x^4}{4} + \ldots + (-1)^{n-1} \frac{x^n}{n} + o(x^n)
    \] 
  \item
    \[
    \arctan(x) = x - \frac{x^3}{3} + \frac{x^5}{5} - \frac{x^7}{7} + \ldots + (-1)^{n-1} \frac{x^{2n-1}}{2n-1} + o(x^{2n-1})
    \] 
\end{itemize}

\begin{figure}[H]
  \begin{exercise}
    \[
    \lim_{n \to +\infty} \left( \left( \frac{2n^2+1}{2n^2} \right)^n -1-\frac{1}{2n} \right) n^2
    \] 
    Chiamo \( a_n \) l'espressione da calcolare: 
    \[
    a_n = \left( \left( \frac{2n^2+1}{2n^2} \right)^n -1-\frac{1}{2n} \right) n^2
    \] 
    \[
    = n^2 \left( \left( 1 + \frac{1}{2n^2} \right) -1-\frac{1}{2n} \right)
    \] 
    \[
    = n^2 \left( exp\left(n\log\left(1+\frac{1}{2n^2}\right) \right) -1-\frac{1}{2n} \right)
    \] 
    Per la regola dell'o piccolo \( \log(1+x) = x +o(x) \;\; per\; x \to 0 \):
    \[
    = n^2 \left( exp \left( n \left( \frac{1}{2n^2} + o \left( \frac{1}{2n^2} \right)  \right) -1 -\frac{1}{2n} \right)  \right) 
    \] 
    \[
    = n^2 \left( exp \left( \frac{1}{2n} + o \left( \frac{1}{n} \right)  \right) -1 - \frac{1}{2n} \right) 
    \] 
    Per la regola dell'o piccolo \( exp(x) = 1 + x + o(x) \):
    \[
      = n^2 \left( \not1 + \not\frac{1}{2n} + o \left( \frac{1}{n} \right) -\not1 -\not\frac{1}{2n} \right) 
    \] 
    \[
    n^2 \cdot o \left( \frac{1}{n} \right) = o(n) = n \cdot o(1)\;\;\; per n \to +\infty
    \] 

    Questa è una forma indeterminata \( \infty \cdot 0 \).
    \[
    a_n = n^2 \left( exp \left( n log \left( 1+\frac{1}{2n^2} \right) -1 -\frac{1}{2n} \right)  \right) 
    \] 
    Applico \( log(1+x) = x - \frac{x^2}{2} + o(x^2) \) per \( x \to 0 \), con \( x = \frac{1}{2n} \to 0 \) 
    per \( n \to +\infty \).
    \[
    a_n = n^2 \left( exp \left( n \left( \frac{1}{2n^2} - \frac{1}{2} \cdot \frac{1}{4n^4} + o \left( \frac{1}{n^4} \right)  \right) -1 -\frac{1}{2n} \right)  \right) 
    \] 
    \[
    = n^2 \left( exp \left( \frac{1}{2n} - \frac{1}{8n^3} + o \left( \frac{1}{n^3} \right)  \right) -1 -\frac{1}{2n} \right) 
    \] 
    Applico \( e^x = 1 + x + \frac{x^2}{2} + o(x) \) con \( x = \frac{1}{2n} - \frac{1}{8n^3}+ o(\frac{1}{n^3}) \to 0 \) 
    \small\[
    a_n = n^2 \left( 1 + \frac{1}{2n} - \frac{1}{8n^3} + o \left( \frac{1}{n^3} \right) + \frac{1}{2} \left( \frac{1}{2n} - \frac{1}{8n^3} + o \left( \frac{1}{n^3} \right)  \right)^2 -\not1 -\not\frac{1}{2n}  \right) 
    \] 
    \[
    = n^2 \left( -\frac{1}{8n^3} + o \left( \frac{1}{n^3} \right) + \frac{1}{8n^2} + o \left( \frac{1}{2n^2} \right)  \right) 
    \] 
    \[
    = n^2 \left( \frac{1}{8n^2} + o \left( \frac{1}{n^2} \right)  \right) = \frac{1}{8} + o(1) \;\;\; per\; n \to +\infty
    \] 
  \end{exercise}
\end{figure}

\section{Integrali}
\( f: [a,b] \to \mathbb{R}\) continua (su un intervallo chiuso e limitato).
\begin{figure}[H]
	\begin{center}
		\begin{tikzpicture}[scale=0.8, domain=0:8.5]
			\draw[->] (-0.5,0) -- (5.5,0) node[right] {$x$};
			\draw[->] (0,-0.5) -- (0,6) node[above] {$y$};

			\coordinate (C) at (0,3.5);
			\coordinate (D) at (1,3);
			\coordinate (E) at (2,4);
			\coordinate (F) at (3,3.5);
			\coordinate (G) at (4,4);
			\coordinate (H) at (5,6);

			\draw [red, thick] plot [smooth, tension=0.7] coordinates { (C) (D) (E) (F) (G) (H) };

      \draw[fill, orange] (0.5,3.13) circle (1.5pt);

      \draw[orange] (0,0) rectangle (0.5,3.13);
      \draw[fill, orange, opacity=0.3] (0,0) rectangle (0.5,3.13);

      \draw[orange] (0.5,0) rectangle (1,3);
      \draw[fill, orange, opacity=0.3] (0.5,0) rectangle (1,3);

      \draw[orange] (1,0) rectangle (1.5,3.5);
      \draw[fill, orange, opacity=0.3] (1,0) rectangle (1.5,3.5);

      \draw[orange] (1.5,0) rectangle (2,4);
      \draw[fill, orange, opacity=0.3] (1.5,0) rectangle (2,4);

      \draw[orange] (2,0) rectangle (2.5,3.8);
      \draw[fill, orange, opacity=0.3] (2,0) rectangle (2.5,3.8);

      \draw[orange] (2.5,0) rectangle (3,3.5);
      \draw[fill, orange, opacity=0.3] (2.5,0) rectangle (3,3.5);

      \draw[orange] (3,0) rectangle (3.5,3.56);
      \draw[fill, orange, opacity=0.3] (3,0) rectangle (3.5,3.56);

      \draw[orange] (3.5,0) rectangle (4,4);
      \draw[fill, orange, opacity=0.3] (3.5,0) rectangle (4,4);

      \draw[orange] (4,0) rectangle (4.5,4.88);
      \draw[fill, orange, opacity=0.3] (4,0) rectangle (4.5,4.88);

      \draw[orange] (4.5,0) rectangle (5,6);
      \draw[fill, orange, opacity=0.3] (4.5,0) rectangle (5,6);

      \node at (0,0) [below, scale=0.5] {\( a = x_0 \)};
      \node at (5,0) [below, scale=0.5] {\( b = x_N \)};

      \node at (0.5,0) [below, scale=0.5] {\( x_1 \)};
      \node at (1,0) [below, scale=0.5] {\( x_2 \)};
      \node at (1.5,0) [below, scale=0.5] {\( x_3 \)};
		\end{tikzpicture}
	\end{center}
  \caption{Somma di Riemann}
\end{figure}

Sia \( N \in \mathbb{N} \), \( N \ge 1 \). Suddivido \( [a,b] \) in \( N \) intervalli,
delimitati da punti equidistanti:
\[
a=x_0 < x_1 < x_2 < \ldots < x_N = b
\] 
(dove \( x_j - x_{j-1} = \frac{1}{N}(b-a)\) per ogni \( j=1, \ldots, N \)).

Considero la \textbf{somma di Riemann} associata a tale suddivisione di \( [a,b] \)
\[
 \sum_{j=1}^{N} f(x_j)(x_j-x_{j-1})
\] 
Se \( f: [a,b] \to \mathbb{R} \) è continua, si dimostra che esiste ed è finito:
\[
 \int_{a}^{b}f(x)\;dx = \lim_{N \to +\infty} \sum_{j=1}^{N} f(x_j)(x_j-x_{j-1}) 
\] 
Tale limite si dice \textbf{integrale} (definito) di \( f \).

\subsection{Osservazioni}
\begin{enumerate}
  \item Si possono considerare varianti diverse, senza che il valore del limite cambi. Ad esempio:
    \[
      \int_{a}^{b} f(x)\;dx = \lim_{N \to +\infty} \sum_{j=1}^{N} f(x_j)(x_j-x_{j-1}) \;\;\; \text{(Figura \ref{fig:SommadiRiemann1})}
    \] 
    \[
      = \lim_{N \to +\infty} \sum_{j=1}^{N} (\underset{[x_{j-1}, x_j]}{max} f)(x_j-x_{j-1})\;\;\; \text{(Figura \ref{fig:SommadiRiemann2})}
    \] 
    \[
    = \lim_{N \to +\infty} \sum_{j=1}^{N} (\underset{[x_{j-1}, x_j]}{min} f)(x_j-x_{j-1}) \;\;\; \text{(Figura \ref{fig:SommadiRiemann3})}
    \] 
    (purchè \( f \) sia continua).
    \begin{figure}[H]
	\begin{center}
		\begin{tikzpicture}[scale=0.8, domain=0:8.5]
			\draw[->] (-0.5,0) -- (5.5,0) node[right] {$x$};
			\draw[->] (0,-0.5) -- (0,6) node[above] {$y$};

			\coordinate (C) at (0,3.5);
			\coordinate (D) at (1,3);
			\coordinate (E) at (2,4);
			\coordinate (F) at (3,3.5);
			\coordinate (G) at (4,4);
			\coordinate (H) at (5,6);

			\draw [red, thick] plot [smooth, tension=0.7] coordinates { (C) (D) (E) (F) (G) (H) };


      \draw[orange] (0,0) rectangle (0.5,3.5);
      \draw[fill, orange, opacity=0.3] (0,0) rectangle (0.5,3.5);

      \draw[orange] (0.5,0) rectangle (1,3.13);
      \draw[fill, orange, opacity=0.3] (0.5,0) rectangle (1,3.13);

      \draw[orange] (1,0) rectangle (1.5,3);
      \draw[fill, orange, opacity=0.3] (1,0) rectangle (1.5,3);

      \draw[orange] (1.5,0) rectangle (2,3.5);
      \draw[fill, orange, opacity=0.3] (1.5,0) rectangle (2,3.5);

      \draw[orange] (2,0) rectangle (2.5,4);
      \draw[fill, orange, opacity=0.3] (2,0) rectangle (2.5,4);

      \draw[orange] (2.5,0) rectangle (3,3.8);
      \draw[fill, orange, opacity=0.3] (2.5,0) rectangle (3,3.8);

      \draw[orange] (3,0) rectangle (3.5,3.5);
      \draw[fill, orange, opacity=0.3] (3,0) rectangle (3.5,3.5);

      \draw[orange] (3.5,0) rectangle (4,3.58);
      \draw[fill, orange, opacity=0.3] (3.5,0) rectangle (4,3.58);

      \draw[orange] (4,0) rectangle (4.5,4);
      \draw[fill, orange, opacity=0.3] (4,0) rectangle (4.5,4);

      \draw[orange] (4.5,0) rectangle (5,4.88);
      \draw[fill, orange, opacity=0.3] (4.5,0) rectangle (5,4.88);
		\end{tikzpicture}
	\end{center}
  \caption{Variante 1}
  \label{fig:SommadiRiemann1}
\end{figure}

    \begin{figure}[H]
	\begin{center}
		\begin{tikzpicture}[scale=0.8, domain=0:8.5]
			\draw[->] (-0.5,0) -- (5.5,0) node[right] {$x$};
			\draw[->] (0,-0.5) -- (0,6) node[above] {$y$};

			\coordinate (C) at (0,3.5);
			\coordinate (D) at (1,3);
			\coordinate (E) at (2,4);
			\coordinate (F) at (3,3.5);
			\coordinate (G) at (4,4);
			\coordinate (H) at (5,6);

			\draw [red, thick] plot [smooth, tension=0.7] coordinates { (C) (D) (E) (F) (G) (H) };


      \draw[orange] (0,0) rectangle (0.5,3.5);
      \draw[fill, orange, opacity=0.3] (0,0) rectangle (0.5,3.5);

      \draw[orange] (0.5,0) rectangle (1,3.13);
      \draw[fill, orange, opacity=0.3] (0.5,0) rectangle (1,3.13);

      \draw[orange] (1,0) rectangle (1.5,3.5);
      \draw[fill, orange, opacity=0.3] (1,0) rectangle (1.5,3.5);

      \draw[orange] (1.5,0) rectangle (2,4);
      \draw[fill, orange, opacity=0.3] (1.5,0) rectangle (2,4);

      \draw[orange] (2,0) rectangle (2.5,4);
      \draw[fill, orange, opacity=0.3] (2,0) rectangle (2.5,4);

      \draw[orange] (2.5,0) rectangle (3,3.8);
      \draw[fill, orange, opacity=0.3] (2.5,0) rectangle (3,3.8);

      \draw[orange] (3,0) rectangle (3.5,3.56);
      \draw[fill, orange, opacity=0.3] (3,0) rectangle (3.5,3.56);

      \draw[orange] (3.5,0) rectangle (4,4);
      \draw[fill, orange, opacity=0.3] (3.5,0) rectangle (4,4);

      \draw[orange] (4,0) rectangle (4.5,4.88);
      \draw[fill, orange, opacity=0.3] (4,0) rectangle (4.5,4.88);

      \draw[orange] (4.5,0) rectangle (5,6);
      \draw[fill, orange, opacity=0.3] (4.5,0) rectangle (5,6);
		\end{tikzpicture}
	\end{center}
  \caption{Variante 2}
  \label{fig:SommadiRiemann2}
\end{figure}

    \begin{figure}[H]
	\begin{center}
		\begin{tikzpicture}[scale=0.8, domain=0:8.5]
			\draw[->] (-0.5,0) -- (5.5,0) node[right] {$x$};
			\draw[->] (0,-0.5) -- (0,6) node[above] {$y$};

			\coordinate (C) at (0,3.5);
			\coordinate (D) at (1,3);
			\coordinate (E) at (2,4);
			\coordinate (F) at (3,3.5);
			\coordinate (G) at (4,4);
			\coordinate (H) at (5,6);

			\draw [red, thick] plot [smooth, tension=0.7] coordinates { (C) (D) (E) (F) (G) (H) };


      \draw[orange] (0,0) rectangle (0.5,3.13);
      \draw[fill, orange, opacity=0.3] (0,0) rectangle (0.5,3.13);

      \draw[orange] (0.5,0) rectangle (1,3);
      \draw[fill, orange, opacity=0.3] (0.5,0) rectangle (1,3);

      \draw[orange] (1,0) rectangle (1.5,3);
      \draw[fill, orange, opacity=0.3] (1,0) rectangle (1.5,3);

      \draw[orange] (1.5,0) rectangle (2,3.5);
      \draw[fill, orange, opacity=0.3] (1.5,0) rectangle (2,3.5);

      \draw[orange] (2,0) rectangle (2.5,3.8);
      \draw[fill, orange, opacity=0.3] (2,0) rectangle (2.5,3.8);

      \draw[orange] (2.5,0) rectangle (3,3.5);
      \draw[fill, orange, opacity=0.3] (2.5,0) rectangle (3,3.5);

      \draw[orange] (3,0) rectangle (3.5,3.5);
      \draw[fill, orange, opacity=0.3] (3,0) rectangle (3.5,3.5);

      \draw[orange] (3.5,0) rectangle (4,3.58);
      \draw[fill, orange, opacity=0.3] (3.5,0) rectangle (4,3.58);

      \draw[orange] (4,0) rectangle (4.5,4);
      \draw[fill, orange, opacity=0.3] (4,0) rectangle (4.5,4);

      \draw[orange] (4.5,0) rectangle (5,4.88);
      \draw[fill, orange, opacity=0.3] (4.5,0) rectangle (5,4.88);
		\end{tikzpicture}
	\end{center}
  \caption{Variante 3}
  \label{fig:SommadiRiemann3}
\end{figure}

  \item Se \( f \ge  0 \), \( \int_{a}^{b} f(x) \) rappresenta l'area racchiusa tra il grafico
    di \( f \) e l'asse \( x \).
  \item In generale, \( \int_{a}^{b} f(x) \) rappresenta l'area \textbf{con segno} racchiusa
    tra il grafico di \( f \) e l'asse \( x \).
    \[
    \int_{a}^{b} f(x)\;dx =\;\; \text{Area della regione gialla \( - \)  area della regione azzurra}
    \] 
 	\begin{figure}[H]
		\begin{center}
			\begin{tikzpicture}
				\draw[name path=x] (-0.5, 0) -- (9, 0) node[right] {};
				\draw[->] (9, 0) -- (10, 0) node[right] {$x$};
				\draw[->] (0, -1.5) -- (0, 3) node[left] {$y$};
				\draw[domain=0:3.15, smooth, variable=\x, thick, yscale=1, xscale=1, name path=sin1] plot ({\x}, {sin(\x r)}) node [right] {};
				\draw[domain=3.15:6.28, smooth, variable=\x, thick, yscale=1, xscale=1, name path=sin2] plot ({\x}, {sin(\x r)}) node [right] {};
				\draw[domain=6.28:9, smooth, variable=\x, thick, yscale=1, xscale=1, name path=sin3] plot ({\x}, {sin(\x r)}) node [right] {};

        \draw[dashed] (9, 0.4) -- (9, 0) node[below] {\( b \)};
        \node at (0, 0) [below right] {\( a \)};

        \tikzfillbetween[of=sin1 and x, on layer=ft]{yellow, opacity=0.3};
        \tikzfillbetween[of=sin2 and x, on layer=ft]{blue, opacity=0.3};
        \tikzfillbetween[of=sin3 and x, on layer=ft]{yellow, opacity=0.3};

      \node[scale=1.5] at (1.6, 0.5) {$+$};
      \node[scale=1.5] at (4.7, -0.5) {$-$};
      \node[scale=1.5] at (7.8, 0.5) {$+$};
			\end{tikzpicture}
		\end{center}
    \caption{Confronto tra funzioni}
	\end{figure}

\end{enumerate}

\subsection{Proprietà di base}
Tutte queste proprietà si applicano a funzioni continue di segno qualsiasi.
\begin{itemize}
  \item \[
      \int_{a}^{b} (f(x) + g(x)) \; dx = \int_{a}^{b} f(x)\;dx + \int_{a}^{b} g(x)\;dx
  \] 
  \item Se \( k \) è costante
  \[
    \int_{a}^{b} (k \cdot f(x))\;dx = k \cdot \int_{a}^{b} f(x)\;dx
  \] 
  ma in generale \textbf{non vale}
  \[
    \int_{a}^{b} f(x) \cdot g(x) \; dx \neq \int_{a}^{b} f(x)\;dx \cdot \int_{a}^{b} g(x)\;dx
  \] 
  \item Se \( f(x) \le g(x) \) per ogni \( x \in [a,b] \), allora:
    \[
      \int_{a}^{b} f(x)\;dx \le \int_{a}^{b} g(x)\;dx
    \] 
  \item Se \( a < b < c \), allora:
    \[
      \int_{a}^{c} f(x)\; dx = \int_{a}^{b} f(x)\;dx + \int_{b}^{c} f(x)\;dx
    \] 
\end{itemize}

\subsection{Teorema fondamentale del calcolo integrale}
\begin{figure}[H]
  \begin{theorem}
    Sia \( f: [a,b] \to \mathbb{R} \) continua. Allora la funzione \( F: [a,b] \to \mathbb{R} \) definita da:
    \[
    F(x):= \int_{a}^{x} f(t)\;dt \;\;\; \forall x \in [a,b]
    \] 
    è differenziabile e \( F'(x) = f(x) \;\;\; \forall x \in [a,b] \).
    
    Una funzione differenziabile \( P \) tale che \( P'=f \) si chiama una \textbf{primitiva} di \( f \).
    L'insieme di tutte le primitive di \( f \) si chiama \textbf{integrale indefinito} di \( f \) e
    si denota con:
    \[
      \int f(x)\;dx
    \] 
  \end{theorem}
\end{figure}

\subsubsection{Dimostrazione}
Prendiamo un qualsiasi \( x_0 \in [a,b] \). Sia:
\[
  R.I.\footnote{\textbf{R.I.}: Rapporto Incrementale}(x) := \frac{F(x) - F(x_0)}{x-x_0} \;\;\; per x \in [a,b]
\] 
Dobbiamo dimostrare che \( \lim_{x \to x_0} R.I.(x) = f(x_0) \), cioè che:
\[
  \forall \varepsilon > 0\;\; \exists \delta > 0 \;t.c.\; \forall x \in [a,b],
\] 
\[
x_0 - \delta \le x \le x_0 + \delta, \;\; x \neq x_0 \to f(x_0) -\varepsilon \le R.I.(x) \le f(x_0) + \varepsilon 
\] 
Prendiamo \( \varepsilon > 0 \) qualsiasi. Poichè \( f \) è continua, sappiamo che \( \lim_{x \to x_0} f(x) =f(x_0) \) 
e dunque esiste \( \delta > 0 \) tale che per ogni \( x \in [a,b] \),
\[
  x_0 - \delta \le x \le x_0 + \delta.\;\;\; x \neq 0 \to \underbrace{f(x_0) - \varepsilon \le f(x) \le f(x_0) +\varepsilon}_{\circ}
\] 
Prendiamo ora \( x \in [a,b] \) tale che \( x_0 < x \le x_0 + \delta \) 
\[
  R.I.(x) = \frac{1}{x-x_0} \left( \int_{a}^{x} f(t)\;dt - \int_{a}^{x_0} f(t)\;dt \right) =
\] 
\[
  = \frac{1}{x-x_0} \int_{x_0}^{x} f(t)\;dt
\] 
Grazie a \( \circ \):
\[
\frac{1}{x-x_0} \int_{x_0}^{x} (f(x_0) -\varepsilon )\; dt \le R.I.(x) \le \frac{1}{x-x_0} \int_{x_0}^{x} (f(x_0) + \varepsilon )\;dt
\] 
Se \( c \) è costante:
\[
  \int_{x_0}^{x} c\;dt = c(x-x_0) \;\;\; \text{(area di un rettangolo)}
\] 
e quindi:
\[
f(x_0) - \varepsilon \le R.I.(x) \le f(x_0) + \varepsilon
\] 
Stesso ragionamento se \( x_0 -\delta \le  x < x_0 \). \( \;\; \square \) 

\subsubsection{Corollario}
Sia \( f: [a,b] \to \mathbb{R} \) una funzione continua. Sia \( P:[a,b] \to \mathbb{R} \) una funzione
differenziabile tale che \( P' = f \). Allora:
\[
\int_{a}^{b} f(x)\;dx = P(b) - P(a)
\] 

\subsubsection{Dimostrazione del corollario}
Come prima, sia \( F(x):= \int_{a}^{x} f(t)\;dt \) per \( x \in [a,b] \). Allora:
\[
  (F-P)' = F' - P' = f - f = 0
\] 
Quindi \( F-P=C \), con \( C \in \mathbb{R} \) costante. Inoltre,
\[
C = F(a) - P(a) = \int_{a}^{a} f(t)\;dt - P(a) = 0 - P(a) = -P(a)
\] 
Quindi:
\[
  \int_{a}^{b} f(t)\;dt = F(b) = F(b) - P(b) + P(b)
\] 
\[
= C + P(b) = -P(a) + P(b)
\] 



\subsection{Esempi}
\begin{figure}[H]
  \begin{example}
    \[
      \int 0\;dx = C \;\;\; \text{dove \( C \in  \mathbb{R} \) è una generica costante }
    \] 
  \end{example}
\end{figure}

\begin{figure}[H]
  \begin{example}
    \[
    \int 1\;dx = x + C
    \] 
  \end{example}
\end{figure}

\begin{figure}[H]
  \begin{example}
    \[
    \int e^x\;dx = e^x + C
    \] 
  \end{example}
\end{figure}

\subsection{Alcune primitive elementari}
\begin{enumerate}
  \item \[
  \int e^x\;dx = e^x + C
  \] 
  \item \[
  \int sin(x)\;dx = -cos(x) + C
  \] 
  \item \[
  \int cos(x)\;dx = sin(x) + C
  \] 
  \item \[
  \int x^\alpha\; dx = \frac{x^{\alpha+1}}{\alpha+1} + C,\;\;\; \text{con \( \alpha \in \mathbb{R} \) costante e \( \alpha \neq -1 \) }
  \] 
  \item \[
  \int x^{-1}\;dx = \int \frac{1}{x}\;dx = \log|x| + C
  \] 
  \item \[
  \int \frac{1}{1+x^2}\;dx = \arctan(x) + C
  \] 
\end{enumerate}

\subsection{Osservazioni}
Data una funzione continua \( f: [a,b] \to \mathbb{R} \), si definisce:
\[
  \int_{a}^{a} f(x)\;dx := 0
\] 
\[
\int_{b}^{a} f(x)\;dx := - \int_{a}^{b} f(x)\;dx
\] 
Questa notazione è utile soprattutto quando si integra per sostituzione.

\subsubsection{Esempi}
\begin{figure}[H]
  \begin{example}
    \[
      \int_{0}^{1} (x^6 - 3x^2 + 3)\; dx
    \] 
    Calcolo prima l'integrale indefinito:
    \[
    \int (x^6 - 3x^2 + 3)\;dx = \int x^6\;dx - 3 \int x^2\;dx + \int 3\;dx =
    \] 
    \[
      = \frac{x^7}{7} - \cancel{3} \frac{x^3}{\cancel{3}} + 3x + C
    \] 
    Per il teorema fondamentale del calcolo integrale:
    \[
      \int_{0}^{1} (x^6 - 3x^2 + 3)\;dx = \left[ \frac{x^7}{7} - x^3 + 3x \right]_{0}^{1} = \underbrace{\frac{1}{7} - 1 + 3}_{P(1)} - \underbrace{(0 - 0 + 0)}_{P(0)} = \frac{15}{7}
    \] 
  \end{example}
\end{figure}

\begin{figure}[H]
  \begin{example}
    \[
      \int_{0}^{1} \left(x^5 - \frac{3}{x^2+1} + 5x \right)\; dx
    \] 
    Calcolo prima l'integrale indefinito:
    \[
    \int \left(x^5 - \frac{3}{x^2+1} + 5x \right)\;dx = \int x^5\;dx - 3 \int \frac{1}{x^2+1}\;dx + 5 \int x\;dx =
    \] 
    \[
    = \frac{x^6}{6} - 3 \arctan(x) + \frac{5x^2}{2} + C
    \] 
    Per il teorema fondamentale del calcolo integrale:
    \[
      \int_{0}^{1} \left(x^5 - \frac{3}{x^2+1} + 5x \right)\;dx = \left[ \frac{x^6}{6} - 3 \arctan(x) + \frac{5x^2}{2} \right]_{0}^{1} =
    \] 
    \[
     \frac{1}{6} - 3 \arctan(1) + \frac{5}{2} - (0 - 0 + 0) = \frac{1}{6} - 3 \arctan(1) + \frac{5}{2}
    \] 
  \end{example}
\end{figure}

\begin{figure}[H]
  \begin{example}
    \[
      \int_{1}^{2} \left( \frac{1}{x^3} + \sqrt{x}  \right)\; dx
    \] 
    Calcolo prima l'integrale indefinito:
    \[
      \int \left( \frac{1}{x^3} + \sqrt{x}  \right)\; dx = \int x^{-3} \;dx + \int x^{\frac{1}{2}} \;dx = -\frac{1}{2x^2} + \frac{2}{3} x^{\frac{3}{2}} + C
    \] 
    Per il teorema fondamentale del calcolo integrale:
    \[
    \int_{1}^{2} \left( \frac{1}{x^3} + \sqrt{x}  \right)\; dx = \left[ -\frac{1}{2x^2} + \frac{2}{3} x^{\frac{3}{2}} \right]_{1}^{2} = -\frac{1}{8} + \frac{16}{3} - \left( -\frac{1}{2} + \frac{2}{3} \right) = \frac{1}{24}
    \] 
  \end{example}
\end{figure}

\begin{figure}[H]
  \begin{example}
    \[
    \int \frac{1}{\sqrt{5x + 7} }\;dx = \int \frac{1}{\sqrt{y} }\;dx
    \] 
    Integrazione per sostituzione:
    \[
    y = 5x + 7
    \] 
    \[
    dy = \frac{dy}{dx} \cdot dx = \frac{d}{dx} (5x + 7) \cdot  dx = 5\;dx \Leftrightarrow dx = \frac{dy}{5}
    \] 
    \[
      \int \frac{1}{\sqrt{5x + 7} }\;dx = \int \frac{1}{\sqrt{y} }\;dx = \int \frac{1}{\sqrt{y} } \cdot \frac{1}{5}\; dy = \frac{1}{5} \int y^{-\frac{1}{2}} \;dy =
    \] 
    \[
    = \frac{1}{5} \frac{y^{\frac{1}{2}}}{\frac{1}{2}} + C = \frac{2}{5} \sqrt{y} + C = \frac{2}{5} \sqrt{5x + 7} + C
    \] 
  \end{example}
\end{figure}

\begin{figure}[H]
  \begin{example}
    \[
      \int_{0}^{\sqrt{\pi } } x \sin(x^2)\;dx
    \] 
    Sostituiamo \( \left[ y=x^2 \right]  \) quindi \( dy = 2x \cdot dx \) (anche gli estremi di integrazione)
    \[
      \frac{1}{2} \int_{0}^{\pi } \sin(y)\;dy = \frac{1}{2} \left[ -\cos(y) \right]_{0}^{\pi } = 
    \] 
    \[
    = \frac{1}{2} (-\cos(\pi ) - (-\cos(0))) = \frac{1}{2} (+1 + 1) = 1
    \] 
  \end{example}
\end{figure}

\begin{figure}[H]
  \begin{example}
    \[
      \int_{\frac{1}{\pi }}^{\frac{2}{\pi }} \frac{1}{x^2} \cos(\frac{1}{x})\;dx
    \] 
    Sostituiamo \( y = \frac{1}{x} \) quindi \( dy = -\frac{1}{x^2}\;dx \) 
    \[
      = - \int_{\pi }^{\frac{\pi }{2}} \cos(y)\;dy =  \int_{\frac{\pi }{2}}^{\pi } \cos(y)\;dy = \left[ \sin(y) \right]_{y=\frac{\pi }{2}}^{\pi } =
    \] 
    \[
    = \sin(\pi ) - \sin(\frac{\pi }{2}) = 0 - 1 = -1
    \] 
  \end{example}
\end{figure}

\subsection{Integrazione delle funzioni razionali}
Esiste, di fatto, un algoritmo che permette di calcolare gli integrali delle funzioni razionali
(quozienti di polinomi). Lo schema generale è:
\begin{enumerate}
  \item ricondursi al caso in cui il grado del \textbf{denominatore} sia maggiore del grado
    del \textbf{numeratore};
  \item scomporre il denominatore;
  \item scrivere l'integranda come somma di funzioni più semplici;
  \item integrare ogni frazione singolarmente.
\end{enumerate}
Si ricorda che un polinomio di grado due \( P(x) =ax^2 + bx + c \), si scompone come:
\[
P(x) = a(x-x_+)(x-x_-)
\] 
dove:
\[
x_{\pm} = \frac{-b \pm \sqrt{b^2 - 4ac}}{2a}
\] 
sono gli zeri di \( P \). (Purchè \( b^2 - 4ac \ge 0\) )

\subsubsection{Esempi}
\begin{figure}[H]
  \begin{example}
    \[
    \int \frac{1}{x^2-3x+2}\;dx
    \] 
    Ho \( x^2 -3x + 2 = (x-1)(x-2)\). Cerco di scrivere:
    \[
    \frac{1}{x^2-3x+2} = \frac{A}{x-1} + \frac{B}{x-2}
    \] 
    Con \( A,B \) costanti da determinare 
    \[
      \frac{A}{x-1} + \frac{B}{x-2} = \frac{Ax - 2A + Bx - B}{(x-1)(x-2)} = \frac{(A+B)x -2A-B}{X^2-3x+2}
    \] 
    Per far si che questa sia uguale a \( \frac{1}{x^2-3x+2} \) devo imporre delle condizioni su \( A \) e \( B \) 
    
    \[
      \begin{cases}
        A+B = 0 \\
        -2A-B = 1
      \end{cases}
      \Leftrightarrow\;\;
      \begin{cases}
        B = -A\\
        -A = 1
      \end{cases}
      \Leftrightarrow\;\;
      \begin{cases}
        A = -1\\
        B = 1
      \end{cases}
    \] 

    \[
    \int \frac{dx}{x^2-3x+2} = \int \left(\frac{-1}{x-1} + \frac{1}{x-2}\right)\;dx = 
    \] 
    \[
    = -\int \frac{1}{x-1}\;dx + \int \frac{1}{x-2}\;dx = -\log|x-1| + \log|x-2| + C
    \] 
  \end{example}
\end{figure}

\begin{figure}[H]
  \begin{example}
    \[
    \int \frac{x^2-2}{x^2-3x+2}\;dx
    \] 
    Verifico che il grado del numeratore sia minore del grado del denominatore:
    \[
      \frac{x^2-2}{x^2-3x+2} = \frac{x^2 \overbrace{-3x+2 + 3x-2}^{\text{sommo e sottraggo}}  -2}{x^2-3x+2} =
    \] 
    \[
    \frac{x^2-3x+2}{x^2-3x+2} + \frac{3x-4}{x^2-3x+2} = 1 + \frac{3x-4}{x^2-3x+2}
    \] 
    Scompongo il denominatore: \( x^2-3x+2 = (x-1)(x-2) \) e voglio trovare \( A \) e \( B \) 
    costanti tali che:
    \[
    \frac{3x-4}{x^2-3x+2} = \frac{A}{x-1} + \frac{B}{x-2} = \frac{(A+B)x -2A-B}{x^2-3x+2}
    \] 
    Devo imporre:
    \[
    \begin{cases}
      A+B = 3 \\
      -2A-B = -4
    \end{cases}
    \Leftrightarrow\;\;
    \begin{cases}
      A = 1 \\
      B = 3 - A = 2
    \end{cases}
    \] 
    \[
    \int \frac{x^2-2}{x^2-3x+2}\;dx = \int \left( 1 + \frac{1}{x-1} + \frac{2}{x-2} \right)\; dx=
    \] 
    \[
    = \int 1\;dx + \int \frac{1}{x-1}\;dx + \int \frac{2}{x-2}\;dx = x - \log|x-1| + 2 \log|x-2| + C
    \] 
  \end{example}
\end{figure}

\begin{figure}[H]
  \begin{example}
    \[
    \int \frac{1}{x^2+4x+5}\;dx
    \] 
    \( x^2 + 4x + 5 \) non ha radici reali! Uso:
    \[
    \int \frac{1}{1+y^2}\;dy = \arctan(y) + C
    \] 
    Devo ricondurmi a scrivere \( x^2+4x+5 \) come \textbf{somma di quadrati}:
    \[
    x^2+4x+5 = x^2 + 4x + 4 - 4 + 5 = (x+2)^2 + 1
    \] 
    \[
    \int \frac{1}{x^2+4x+5}\;dx = \int \frac{1}{(x+2)^2 + 1}\;dx=
    \] 
    Sostituisco \( y = x+2 \) quindi \( dy = dx \) e:
    \[
    = \int \frac{1}{y^2 + 1}\;dy = \arctan(y) + C = \arctan(x+2) + C
    \] 
  \end{example}
\end{figure}

\begin{figure}[H]
  \begin{example}
    \[
      \int \frac{x^2}{x^2+4x+5}\;dx
    \] 
    \[
      \frac{x^2}{x^2+4x+5} = \frac{x^2 + 4x + 5 - 4x - 5}{x^2+4x+5} = 1 - \frac{4x+5}{x^2+4x+5}
    \] 
    Denominatore privo di radici reali, quindi cerco di utilizzare:
    \[
      \int \frac{f'(x)}{f(x)}\;dx = \log|f(x)| + C
    \] 
    Osservo:
    \[
    \frac{d}{dx} (x^2+4x+5) = 2x + 4
    \] 
    \[
    \frac{x^2}{x^2+4x+5} = 1 - \frac{2 \cdot 2x + 2 \cdot 4 + -2 \cdot 4 + 5}{x^2+4x+5} =
    \] 
    \[
    = 1 - 2 \cdot \frac{2x+4}{x^2+4x+5} - \frac{3}{x^2+4x+5} =
    \] 
    \[
    1 - 2 \cdot \frac{2x+4}{x^2+4x+5} - \frac{3}{(x+2)^2 + 1}
    \] 
    \[
    \int \frac{x^2}{x^2+4x+5}\;dx = x - 2 \log(x^2+4x+5) - 3 \arctan(x+2) + C
    \] 
  \end{example}
\end{figure}

\subsection{Integrazione per parti}
\begin{definition}[Proposizione]
  Siano \( f,g : [a,b] \to \mathbb{R} \) due funzioni differenziabili. Allora, vale:
  \[
    \int_{a}^{b} f(x)\;g'(x)\;dx = \left[ f(x)\;g(x) \right]_{a}^{b} - \int_{a}^{b} f'(x)\;g(x)\;dx
  \] 
  \[
    \int f(x)\;g'(x)\;dx = \left[ f(x)\;g(x) \right] - \int f(x)\;g'(x)\;dx
  \] 
\end{definition}

\subsubsection{Esempi}
\begin{figure}[H]
  \begin{example}
    \[
      \int \underbrace{x}_{f(x)}\,\underbrace{e^x}_{g'(x)}\;dx
    \] 
    \[
    \int x\,e^x\;dx = x\,e^x - \int e^x\;dx = x\,e^x - e^x + C
    \] 
  \end{example}
\end{figure}

\begin{figure}[H]
  \begin{example}
    \[
      \int log(x)\;dx = \int \underbrace{1}_{g'(x)} \cdot \underbrace{log(x)}_{f(x)}\;dx =
    \] 
    \[
    = xlog(x) - \int \frac{1}{x} x \; dx = xlog(x) - x + C
    \] 
  \end{example}
\end{figure}

\begin{figure}[H]
  \begin{example}
    \[
      \int_{0}^{\pi } \sin^2(x)\;dx = \int_{0}^{\pi } \sin(x) \cdot \sin(x)\;dx =
    \] 
    \[
      = -\left[\cos(x) \cdot \sin(x) \right]_{0}^{\pi } + \int_{0}^{\pi } \cos(x) \cdot \cos(x)\;dx =
    \] 
    \[
      = -0 + 0 + \int_{0}^{\pi } \cos^2(x)\;dx = 
    \] 
    \[
      = \int_{0}^{\pi } (1 - \sin^2(x))\;dx = \int_{0}^{\pi } 1\;dx - \int_{0}^{\pi } \sin^2(x)\;dx
    \] 
    \[
      1 \cdot (\pi  - 0) - \int_{0}^{\pi } \sin^2(x)\;dx
    \] 
    Abbiamo dimostrato:
    \[
      \left[ A = \pi -A \Leftrightarrow 2A = \pi \Leftrightarrow A = \frac{\pi }{2} \right] 
    \] 
    Quindi:
    \[
      \int_{0}^{\pi } \sin^2(x)\;dx = \frac{\pi }{2}
    \] 
  \end{example}
\end{figure}

\begin{figure}[H]
  \begin{example}
    \[
      \int_{0}^{2} \frac{\log(2x+1)}{(2x+1)^2}\;dx
    \] 
    La funzione integranda è ben definita e continua su \( [0,2] \). Cambio di variabile:
    \[
    y = 2x+1
    \] 
    \[
    dy = \frac{d}{dx}(2x+1)\;dx = 2\;dx
    \] 
    \[
      \int_{0}^{2} \frac{\log(2x+1)}{(2x+1)^2}\;dx = \frac{1}{2} \int_{1}^{5} \frac{\log(y)}{y^2}\;dy = \frac{1}{2} \int_{1}^{5} y^{-2} \cdot \log(y) \;dy =
    \] 
    \[
      = \frac{1}{2} \left[ -y^{-1} \cdot \log(y) \right]_{y=1}^{5} + \int_{1}^{5} y^{-2}\;dx =
    \] 
    \[
      = \frac{1}{2} \left[ -\frac{\log(y)}{y} \right]_{y=1}^{5} + \left[ -y^{-1} \right]_{y=1}^{5} =
    \] 
    \[
      = \frac{1}{2} \left( -\frac{\log(5)}{5} + \frac{\log(1)}{1} \right) + \left( -\frac{1}{5} + \frac{1}{1} \right) = \frac{1}{2} \left( -\frac{\log(5)}{5} \right) - \frac{4}{5} =
    \] 
    \[
      = -\frac{\log(5)}{10} - \frac{4}{5} = -\frac{1}{10} \log(5) - \frac{4}{5} 
    \] 
  \end{example}
\end{figure}

\section{Serie}
Sia \( \{a_n\}_{n \in \mathbb{N}}  \) una successione di numeri reali (cioè, una funzione \( \mathbb{N} \to \mathbb{R} \)).
Vogliamo dare un senso preciso alla "somma infinita"
\[
a_0 + a_1 + a_2 + a_3 + a_4 + \ldots
\] 

\begin{figure}[H]
  \begin{definition}
    Data una successione di numeri reali \( \{a_n\}  \), si definisce la somma della \textbf{serie}
    di termine generale \( \{a_n\}  \) come:
    \[
    \sum_{n=0}^{+\infty} a_n = \lim_{N \to +\infty} \sum_{n=0}^{N} a_n
    \] 
    Si dice che la serie \( \begin{cases}
      \text{converge}\\
      \text{diverge}\\
      \text{oscilla}
    \end{cases} \) se il limite \( \begin{cases}
      \text{esiste finito}\\
      \text{esiste infinito}\\
      \text{non esiste}
    \end{cases} \) 
  \end{definition}
\end{figure}

\begin{figure}[H]
  \begin{example}
    \[
      \sum_{n=0}^{+\infty} 2^{-n} = 1 + \frac{1}{2} + \frac{1}{4} + \frac{1}{8} + \ldots = \lim_{N \to +\infty} \sum_{n=0}^{N} 2^{-n}
    \] 
    \begin{center}
      \begin{tikzpicture}
        \draw (0,0) -- (10,0);

        \foreach \x in {0,1,2} {
          \draw (\x*3,-0.2) -- (\x*3,0.2);
          \node[scale=0.8] at (\x*3, -0.5) {\x};
        }

        \node[scale=0.8] at (4.5, -0.5) {1.5};
        \draw (4.5,-0.2) -- (4.5,0.2);

        \draw (5,-0.2) -- (5,0.2);

        \draw (5.25,-0.2) -- (5.25,0.2);

        \draw[fill, yellow, opacity=0.5] (0,-0.2) rectangle (6,0.2);

        \draw (0,0.3) -- ++(0,0.2) -- ++(3,0) node[above, scale=0.8, xshift=-50] {\( 1 \) } -- ++(0,-0.2);
        \draw (3,0.3) -- ++(0,0.2) -- ++(1.5,0) node[above, scale=0.8, xshift=-25] {\( \frac{1}{2} \) } -- ++(0,-0.2);
        \draw (4.5,0.3) -- ++(0,0.2) -- ++(0.5,0) node[above, scale=0.8, xshift=-10] {\( \frac{1}{4} \) } -- ++(0,-0.2);
        \draw (5,0.3) -- ++(0,0.2) -- ++(0.25,0) node[above, scale=0.8, xshift=-5] {\( \frac{1}{8} \) } -- ++(0,-0.2);
      \end{tikzpicture}
    \end{center}
    Considero prima:
    \[
      \sum_{n=0}^{N} 2^{-n} \;\;\text{con \( N \in \mathbb{N} \) qualsiasi.}
    \] 
    \[
    \left(\sum_{n=0}^{N} (1/2)^n \right) \cdot \left( 1- \frac{1}{2} \right) =
    \] 
    \[
    = \left( 1 + \frac{1}{2} + \frac{1}{4} + \frac{1}{8} + \ldots + \frac{1}{2^N} \right) \cdot \left( 1 - \frac{1}{3} \right) =
    \] 
    Somma telescopica:
    \[
      = 1 \cancel{+ \frac{1}{2}} \cancel{+\frac{1}{4}} \cancel{+ \frac{1}{8}} \cancel{+ \ldots} \cancel{+ \frac{1}{2^N}} \cancel{- \frac{1}{2}} \cancel{- \frac{1}{4}} \cancel{- \frac{1}{8}} \cancel{- \ldots} - \frac{1}{2^{N+1}} = 1 - \frac{1}{2^{N+1}}
    \] 
    Dunque:
    \[
      \sum_{n=0}^{N} \left( \frac{1}{2} \right)^n = \frac{1 - \frac{1}{2^{N+1}}}{1 - \frac{1}{2}} \underset{N \to +\infty}{\to } \frac{1}{1-\frac{1}{2}} = 2
    \] 
  \end{example}
\end{figure}

\subsection{Osservazioni}
\begin{itemize}
  \item Possiamo anche definire
    \[
    \sum_{n=1}^{+\infty} a_n = \lim_{N \to +\infty} \sum_{n=1}^{N} a_n
    \] 
    \[
    \sum_{n=57}^{+\infty} a_n = \lim_{N \to +\infty} \sum_{n=57}^{N} a_n \ldots
    \] 
  \item Nessuno garantisce che la somma della serie esista. Ad esempio, sia:
    \[
    a_n = (-1)^n = \begin{cases}
      1 \;\;\; \text{se \( n \) pari}\\
      -1 \;\;\; \text{se \( n \) dispari}
    \end{cases}
    \] 
    \[
    \sum_{n=0}^{+\infty} a_n = 1 - 1 + 1 - 1 + 1 - 1 + \ldots
    \] 
    La serie \textbf{Oscilla}. La somma delle serie \textbf{non esiste}, infatti:
    \[
    \sum_{n=0}^{N} a_n = \begin{cases}
      1 \;\;\; \text{se \( N \) pari}\\
      0 \;\;\; \text{se \( N \) dispari}
    \end{cases}
    \] 
  \item Se la serie \( \sum_{n=0}^{+\infty} a_n \) converge, allora necessariamente \( \lim_{n \to +\infty} a_n = 0 \).
    Infatti, sia \( S:= \sum_{n=0}^{+\infty} a_n \in \mathbb{R} \). Allora
    \[
      a_N = \sum_{n=0}^{N} a_n - \sum_{n=0}^{N-1} a_n \underset{N \to +\infty}{\to } S - S = 0
    \] 
    (perchè se \( f(N) \underset{N \to +\infty}{\to } S \) allora \( f(N-1) \underset{N \to +\infty}{\to } S \)).
    \textbf{Non vale} il viceversa: potrebbe benissimo capitare che \( \lim_{n \to +\infty} a_n = 0 \),
    pero \( \sum_{n=0}^{+\infty} a_n \) non converga.
\end{itemize}

\subsubsection{Esempi importanti}
\begin{enumerate}
  \item \textbf{Serie geometrica} (di ragione \( x \in \mathbb{R} \)):
    \[
    \sum_{n=1}^{+\infty} x^n
    \] 
    Essa converge se e solo se \( -1 < x < 1 \) e in tal caso si ha:
    \[
    \sum_{n=0}^{+\infty} x^n = \frac{1}{1-x}
    \] 
    \textbf{Conseguenza}: \( 0,\overline{9} = 1 \) \\
    \textbf{Dimostrazione}: \( 0,\overline{9} = \frac{9}{10} + \frac{9}{100} + \frac{9}{1000} + \frac{9}{10000} + \ldots = \) 
    \[
    = 9 \cdot \sum_{n=1}^{+\infty} \frac{1}{10^n} = 9 \cdot \left( \sum_{n=1}^{+\infty} \frac{1}{10^n} -1 \right) =
    \] 
    \[
    = 9 \cdot \left( \frac{1}{1-\frac{1}{10}} -1 \right) = 9 \cdot \frac{10}{9} - 9 = 10 - 9 = 1
    \] 
  \item \textbf{Serie esponenziale}:
    \[
    \text{per ogni}\;\; x \in \mathbb{R},\;\; \sum_{n=0}^{+\infty} \frac{x^n}{n!} = e^x
    \] 
  \item \textbf{Serie armonica}:
    \[
    \sum_{n=1}^{+\infty} \frac{1}{n} = +\infty
    \] 
    \textbf{Dimostrazione}:
    \begin{center}
      \begin{tikzpicture}
				\draw[->] (-0.5, 0) -- (3, 0) node[right] {$x$};
				\draw[->] (0, -0.5) -- (0, 3) node[left] {$y$};
				\draw[domain=0.3:3, smooth, variable=\x, red, thick, yscale=1, xscale=1] plot ({\x}, {1/\x}) node [above] {\( \frac{1}{x} \) };

        \draw[fill, blue, opacity=0.3] (0.35,0) rectangle (0.7,3);
        \draw[blue, thin] (0.35,0) rectangle (0.7,3);
        \draw[fill, blue, opacity=0.3] (0.7,0) rectangle (1.05,2);
        \draw[blue, thin] (0.7,0) rectangle (1.05,2);
        \draw[fill, blue, opacity=0.3] (1.05,0) rectangle (1.4,1.5);
        \draw[blue, thin] (1.05,0) rectangle (1.4,1.5);
        \draw[fill, blue, opacity=0.3] (1.4,0) rectangle (1.75,1.2);
        \draw[blue, thin] (1.4,0) rectangle (1.75,1.2);
        \draw[fill, blue, opacity=0.3] (1.75,0) rectangle (2.1,1);
        \draw[blue, thin] (1.75,0) rectangle (2.1,1);
        \draw[fill, blue, opacity=0.3] (2.1,0) rectangle (2.45,0.8);
        \draw[blue, thin] (2.1,0) rectangle (2.45,0.8);
        \draw[fill, blue, opacity=0.3] (2.45,0) rectangle (2.8,0.7);
        \draw[blue, thin] (2.45,0) rectangle (2.8,0.7);

        \node[scale=0.8, below] at (0.35, 0) {\( 1 \)};
        \node[scale=0.8, below] at (0.7, 0) {\( 2 \)};
        \node[scale=0.8, below] at (1.05, 0) {\( 3 \)};
        \node[scale=0.8, below] at (2.1, 0) {\( N \)};
        \node[scale=0.8, below] at (2.45, 0) {\( N+1 \)};
      \end{tikzpicture}
    \end{center}
    Area dell'unione dei rettangoli = \( 1 + \frac{1}{2} + \frac{1}{3} + \ldots + \frac{1}{N} \) \\
    Area sottesa al grafico \( y = \frac{1}{x} \) per \( 1 \le x \le N+1 =\)
    \[= \int_{1}^{N+1} \frac{1}{x}\;dx  = \log(N+1) \]
    Dunque:
    \[
    \sum_{n=1}^{N} \frac{1}{n} \ge \log(N+1) \underset{N \to +\infty}{\to } +\infty
    \] 
    \textbf{Osservazione}:\\
    Facendo stime più precise, si potrebbe dimostrare che vale:
    \[
    \sum_{n=1}^{N} \frac{1}{n} = \log(N+1) + \gamma + o(1) \;\; per N \to +\infty
    \] 
    dove \( \gamma  \) è un opportuno numero reale, chiamato la costante di \textbf{Eulero-Mascheroni}
    (\( \gamma \approx 0.577 \ldots \)).
  \item \textbf{Serie armonica generalizzata}:
    Sia \( s \in \mathbb{R} \) un parametro. La serie
    \[
    \sum_{n=1}^{+\infty} \frac{1}{n^s}
    \] 
    Converge se \( x > 1 \) e diverge se \( s \le 1 \) (o oscilla)
\end{enumerate}

\subsection{Criteri per studiare il carattere di una serie}
Per carattere si intende:
\begin{itemize}
  \item Convergente
  \item Divergente
  \item Oscillante
\end{itemize}
\begin{enumerate}
  \item \textbf{Serie a termini positivi}: criteri del \( \begin{cases}
      \text{confronto}\\
      \text{confronto asintotico}\\
      \text{rapporto}
    \end{cases} \) 
  \item \textbf{Serie a segno alterno}: criterio di Leibniz
  \item \textbf{Serie generali}: convergenza assoluta
\end{enumerate}
\subsubsection{Serie a termini positivi}
\[
\sum_{n=0}^{+\infty} a_n, \;\;\; con\;\; a_n \ge 0 \;\;\; \forall n \in \mathbb{N}
\] 
\textbf{Osservazioni generali}:
\begin{enumerate}
  \item Le serie a termini positivi convergono o divergono, ma \textbf{non} oscillano mai,
    perchè 
    \[
    a_0 \le a_0 + a_1 \le a_0 + a_1 + a_2 \le a_0 + a_1 + a_2 + a_3 \le \ldots
    \] 
    Siccome \( a_n \ge 0  \forall n\), la successione delle somme parziali è non decrescente,
    quindi
    \[
    \lim_{N \to +\infty} \sum_{n=0}^{N} a_n \;\;\;\text{esiste per monotonia}
    \] 
  \item I criteri qui di seguito si applicano anche alle serie i cui termini sono positivi
    "da un certo punto in pou" (cioè, \( a_n \ge 0 \) per ogni \( n \) abbastanza grande, maggiore
    o uguale di un certo \( n_0 \) ). Infatti
    \[
      \sum_{n=0}^{+\infty} a_n = \underbrace{\sum_{n=0}^{n_0-1} a_n}_{\text{Somma \textbf{finita}}} + \sum_{n=n_0}^{+\infty} a_n
    \] 
    Il carattere della prima serie dipende soltanto dal carattere della serie con somma infinita.
    Quindi \( \sum_{n=0}^{+\infty} a_n \) converge se e solo se \( \sum_{n=n_0}^{+\infty} a_n \) converge.
\end{enumerate}
\subsubsection{Criterio del confronto}
Siano \( \{a_n\}_{n \in \mathbb{N}},\;\{b_n\}_{n \in \mathbb{N}}  \) successioni di numeri reali
tali che:
\[
0 \le a_n \le b_n \;\;\; \forall n \in \mathbb{N}
\] 
Se \( \sum_{n=0}^{+\infty} b_n \) converge, allora \( \sum_{n=0}^{+\infty} a_n \) converge e vale:
\[
\sum_{n=0}^{+\infty} a_n \le \sum_{n=0}^{+\infty} b_n
\] 
Se \( \sum_{n=0}^{+\infty} a_n \) diverge, allora \( \sum_{n=0}^{+\infty} b_n \) diverge.
\begin{figure}[H]
  \begin{example}
    \[
    \sum_{n=1}^{+\infty} \frac{\sin^2(n)}{n^2}
    \] 
    \[
    0 \le \frac{\sin^2(n)}{n^2} \le \frac{1}{n^2} \;\;\; \forall n \in \mathbb{N} \;\; n \ge 1
    \] 
    \[
    \sum_{n=1}^{+\infty} \frac{1}{n^2} \;\;\text{converge (serie armonica generalizzata)}
    \] 
    Per confronto \( \sum_{n=1}^{+\infty} \frac{\sin^2(n)}{n^2} \) converge.
  \end{example}
\end{figure}
\begin{figure}[H]
  \begin{example}
    \[
    \sum_{n=1}^{+\infty} \frac{\ln(n)}{n}
    \] 
    \[
    0 \le \frac{\ln(n)}{n},\;\; \frac{\ln(n)}{n} \ge \frac{\ln(2)}{n} \;\;\; \forall n \in \mathbb{N} \;\; n \ge 2
    \] 
    \[
    \sum_{n=2}^{+\infty} \frac{\ln(2)}{n} = (\ln(2)) \sum_{n=2}^{+\infty} \frac{1}{n} = +\infty
    \] 
    Per confronto \( \sum_{n=2}^{+\infty} \frac{\ln(n)}{n} \) diverge.
  \end{example}
\end{figure}
\subsubsection{Criterio del confronto asintotico}
Siano \( \{a_n\}_{n \in \mathbb{N}},\;\{b_n\}_{n \in \mathbb{N}}  \) successioni di numeri reali tali che
\( a_n \ge 0,\;\; b_n \ge 0 \;\;\; \forall n \in \mathbb{N} \).

Supponiamo che il limite
\[
\Lambda := \lim_{n \to +\infty} \frac{a_n}{b_n}
\] 
esista e sia \( 0 < \Lambda < +\infty \). Allora:
\[
\sum_{n=0}^{+\infty} a_n \;\; \text{converge se e solo se}\;\; \sum_{n=0}^{+\infty} b_n \;\; \text{converge.}
\] 
\begin{figure}[H]
  \begin{example}
    \[
    \sum_{n=1}^{+\infty} \frac{5n}{7n^3+n\sin(n)}
    \] 
    È una serie a termini positivi, perchè \( 5n \ge 0 \), \( |n\sin(n)| \le |n| \),
    \( 7n^3+n\sin(n) \ge 7n^3 - n \ge 0 \;\;\forall n \in \mathbb{N} \). Sia:
    \[
    a_n := \frac{5n}{7n^3+n\sin(n)},\;\;\; b_n := \frac{n}{n^3} = \frac{1}{n^2}
    \] 
    \[
    \lim_{n \to +\infty} \frac{a_n}{b_n} = \lim_{n \to +\infty} \frac{5n \cdot n^2}{7n^3+n\sin(n)} =
    \] 
    \[
    = \lim_{n \to +\infty} \frac{5}{7 + \frac{\sin(n)}{n^2}} = \frac{5}{7}
    \] 
    \[
    \sum_{n=1}^{+\infty} b_n = \sum_{n=1}^{+\infty} \frac{1}{n^2} \;\;\text{converge}
    \] 
    Per confronto asintotico
  \end{example}
\end{figure}
\subsubsection{Corollario/Caso particolare}
\( P \) è un polinomio di grado \( p \) e \( Q \) è un polinomio di grado \( q \) , allora
\[
\sum_{n=0}^{+\infty} \frac{P(n)}{Q(n)} \;\;\text{converge se e solo se}\;\; \sum_{n=1}^{+\infty} \frac{n^p}{n^q}\;\; \text{converge}
\] 
\begin{figure}[H]
  \begin{example}
    \[
      \sum_{n=1}^{+\infty} \frac{n^{23} + 42 n^2 + 3}{n^{25} - 31n^2 + 23n - 3}
    \] 
    \[
      \sum_{n=1}^{+\infty} \frac{n^{23}}{n^{25}} = \sum_{n=1}^{+\infty} \frac{1}{n^2} \;\;\text{converge}
    \] 
    La serie iniziale converge per confronto asintotico.
  \end{example}
\end{figure}
\subsubsection{Criterio del rapporto}
Sia \( \{a_n\}_{n \in \mathbb{N}}  \) una successione di numeri reali tale che \( a_n > 0 \;\; \forall n \in \mathbb{N} \).
Supponiamo che il limite \[
L := \lim_{n \to +\infty} \frac{a_n+1}{a_n}
\] 
esista, finito o infinito.
\begin{itemize}
  \item Se \( L < 1 \), allora \( \sum_{n=0}^{+\infty} a_n \) converge.
  \item Se \( L > 1 \), allora \( \sum_{n=0}^{+\infty} a_n \) diverge.
  \item Se \( L = 1 \), allora non possiamo concludere nulla.
\end{itemize}

\begin{figure}[H]
  \begin{example}
    \[
    \sum_{n=0}^{+\infty} \frac{n^3+1}{n!}
    \] 
    Serie a termini \( > 0 \), \( a_n := \frac{n^3+1}{n!} \) 
    \[
    \frac{a_n+1}{a_n} = \frac{(n+1)^3+1}{(n+1)!} \cdot \frac{n!}{n^3+1} = \frac{(n+1)^3 + 1}{n^3+1} \cdot \frac{n!}{(n+1)!} =
    \] 
    \[
      \left[ (n+1)! = (n!)\cdot (n+1) \right] 
    \] 
    \[
      = \frac{1}{n+1} \cdot \frac{\cancel{n^3} ((1+\frac{1}{n})^3 + \frac{1}{n^3})}{\cancel{n^3} (1+ \frac{1}{n^3})}
    \] 
    \[
    \frac{a_n+1}{a_n} \to 0 \;\;\text{per \( n \to +\infty \)}
    \] 
    Per il criterio del rapporto \( \sum_{n=0}^{+\infty} \frac{n^3+1}{n!} \) converge.
  \end{example}
\end{figure}
\begin{figure}[H]
  \begin{example}
    Per ogni \( x \in (0, +\infty) \) dato, \( \sum_{n=0}^{+\infty} \frac{x^n}{n!} \) converge.

    Serie a termini \( >0 \). Sia:
    \[
    a_n := \frac{x^n}{n!}
    \] 
    \[
      \frac{a_n +1}{a_n} = \frac{x^{n+1}}{(n+1)!} \cdot \frac{n!}{x^n} = \frac{x}{n+1} \underset{n \to +\infty}{\to } 0 < 1
    \] 
    Per il criterio del rapporto \( \sum_{n=0}^{+\infty} \frac{x^n}{n!} \) converge.
  \end{example}
\end{figure}
\subsubsection{Serie a segni alterni}
\[
\sum_{n=0}^{+\infty} (-1)^n b_n = b_0 - b_1 + b_2 - b_3 + b_4 - \ldots
\] 
con \( b_n > 0 \) per ogni \( n \in  \mathbb{N} \). Ad esempio:
\[
\sum_{n=1}^{+\infty} \frac{(-1)^n}{n} = -1 + \frac{1}{2} - \frac{1}{3} + \frac{1}{4} - \ldots
\] 
è una serie a segni alterni,
\[
\sum_{n=1}^{+\infty} (-1)^n \sin(n) = -\sin(1) + \sin(2) - \sin(3) + \sin(4) - \ldots
\] 
\textbf{Non} è una serie a segni alterni. Invece lo è:
\[
\sum_{n=1}^{+\infty} (-1)^n |\sin(n)| = -|\sin(1)| + |\sin(2)| - |\sin(3)| + |\sin(4)| - \ldots
\] 
\subsubsection{Criterio di Leibnitz}
Sia \( \{b_n\}_{n \in \mathbb{N}}  \) una successione di numeri reali, tali che:
\begin{itemize}
  \item \( b_n \ge 0\;\;\; \forall n \in \mathbb{N} \) 
  \item \( b_{n+1} \le b_n\;\;\;\forall n \in \mathbb{N} \) 
  \item \( \lim_{n \to +\infty} b_n = 0 \)
\end{itemize}
(successione decresccente)
\[
\text{Allora}\;\; \sum_{n=0}^{+\infty} (-1)^n b_n \;\;\;\text{converge}
\] 
\begin{figure}[H]
  \begin{example}
    \[
    \sum_{n=1}^{+\infty} \frac{(-1)^n}{n}
    \] 
    Sia \( b_n := \frac{1}{n} \). Valgono le ipotesi del criterio di Leibnitz:
    \begin{itemize}
      \item \( b_n \ge 0 \;\;\; \forall n \in \mathbb{N} \;\;\; \surd\)
      \item \( b_{n+1} \le b_n \Leftrightarrow \frac{1}{n+1} \le \frac{1}{n} \;\;\;\forall n \in \mathbb{N} \;\;\; \surd \)
      \item \( \lim_{n \to +\infty} b_n =  0 \;\;\; \surd \)
    \end{itemize}
    Quindi \( \sum_{n=1}^{+\infty} (-1)^n b_n = -1 + \frac{1}{2} - \frac{1}{3} + \frac{1}{4} - \ldots \) 
    \label{D1}
    \[
    \sum_{n=1}^{+\infty} \frac{1}{n} = +\infty
    \] 
  \end{example}
\end{figure}
\end{document}
