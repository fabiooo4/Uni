\documentclass[a4paper]{article}

\usepackage[utf8]{inputenc}
\usepackage[T1]{fontenc}
\usepackage{textcomp}
\usepackage[italian]{babel}
\usepackage{amsmath, amssymb}
\usepackage{amsfonts}
\usepackage{mdframed}
\usepackage{xcolor}
\usepackage{float}
\usepackage{tikz}
\usepackage{graphicx}
\graphicspath{{./figures/}}

\usepackage{import}
\usepackage{pdfpages}
\usepackage{transparent}
\usepackage{xcolor}

\usepackage{ntheorem}
\newtheorem{theorem}{Teorema}

% Useful definitions frame
\theoremstyle{break}
\theoremheaderfont{\bfseries}
\newmdtheoremenv[%
	linecolor=gray,leftmargin=0,%
	rightmargin=0,
	innertopmargin=8pt,%
	ntheorem]{define}{Definizioni utili}[section]

% Example frame
\theoremstyle{break}
\theoremheaderfont{\bfseries}
\newmdtheoremenv[%
	linecolor=gray,leftmargin=0,%
	rightmargin=0,
	innertopmargin=8pt,%
	ntheorem]{example}{Esempio}[section]

% Important definition frame
\theoremstyle{break}
\theoremheaderfont{\bfseries}
\newmdtheoremenv[%
	linecolor=gray,leftmargin=0,%
	rightmargin=0,
	backgroundcolor=gray!40,%
	innertopmargin=8pt,%
	ntheorem]{definition}{Definizione}[section]

% Exercise frame
\theoremstyle{break}
\theoremheaderfont{\bfseries}
\newmdtheoremenv[%
	linecolor=gray,leftmargin=0,%
	rightmargin=0,
	innertopmargin=8pt,%
	ntheorem]{exercise}{Esercizio}[section]


% figure support
\usepackage{import}
\usepackage{xifthen}
\pdfminorversion=7
\usepackage{pdfpages}
\usepackage{transparent}
\newcommand{\incfig}[1]{%
	\def\svgwidth{\columnwidth}
	\import{./figures/}{#1.pdf_tex}
}

\pdfsuppresswarningpagegroup=1

\begin{document}
\begin{titlepage}
	\begin{center}
		\vspace*{1cm}

		\Huge
		\textbf{Probabilità e Statistica\\Esercizi}

		\vspace{0.5cm}
		\LARGE
		UniVR - Dipartimento di Informatica

		\vspace{1.5cm}

		\textbf{Fabio Irimie}

		\vfill


		\vspace{0.8cm}


		2° Semestre 2023/2024

	\end{center}
\end{titlepage}


\tableofcontents
\pagebreak

\section{Introduzione}
\subsection{Maggiorante}

\subsection{Minorante}

\subsection{Estremo superiore}

\subsection{Estremo inferiore}

\section{Limiti}
I limiti sono il calcolo infinitesimale, ovvero il
calcolo che si occupa di studiare il comportamento di una funzione in un intorno
di un punto.

Nelle definizioni che seguono, è data una funzione \( f:A \to \mathbb{R} \) il cui
dominio \( A \subseteq \mathbb{R} \) è un insieme \textbf{non} limitato superiormente.
(Questa ipotesi serve per definire i limiti per \( x \to +\infty \) )

\begin{definition}
	Sia \( L \in \mathbb{R} \). Si dice che
	\[ \lim_{x \to +\infty} f(x) = L  \]
	Se e solo se
	\[
		\forall \epsilon > 0 \exists k>0 \;t.c.\; \forall x \subset A\footnote{Il dominio della funzione},
	\]
	\[
		x \ge k \to L-\epsilon \le f(x) \le L+\epsilon
	\]
	(Notazione alternativa: \( f(x) \to L \) per \( x \to +\infty \) )\\
	\textbf{La condizione deve essere soddisfatta per ogni \( \epsilon \) }.
	\label{D1}
\end{definition}
\begin{example}
	\[
		\lim_{x \to +\infty} \frac{1}{x} = 0
	\]
	\label{D2}
	Sia dato \( \epsilon > 0 \) arbitrario. Definisco \( k := \frac{1}{\epsilon} \).\\
	sia dato \( x > 0 \) arbitrario, supponiamo \( x \ge k \).  Allora
	\[
		0-\epsilon \le 0 \le \frac{1}{x} \le \frac{1}{k} = \frac{1}{\frac{1}{\epsilon}} = \epsilon
	\]
	Quindi, ho dimostrato che la definizione di limite è soddisfatta (con \( L=0 \)).
\end{example}
\begin{figure}[H]
	\begin{definition}
		Si dice che
		\[
			\lim_{x \to +\infty} f(x) = +\infty
		\]
		Se e solo se
		\[
			\forall M > 0\;\; \exists k>0\; \;t.c.\;\; \forall x \in A,
		\]
		\[
			x \ge k \to f(x) \ge M
		\]
		(Notazione alternativa: \( f(x) \to +\infty \) per \( x \to +\infty \))
		\label{D3}
	\end{definition}
\end{figure}
\begin{figure}[H]
	\begin{definition}
		Si dice che
		\[
			\lim_{x \to +\infty} f(x) = -\infty
		\]
		Se e solo se
		\[
			\forall M > 0\;\; \exists k>0\; \;t.c.\;\; \forall x \in A,
		\]
		\[
			x \ge k \to f(x) \le -M
		\]
		(Notazione alternativa: \( f(x) \to -\infty \) per \( x \to +\infty \))
		\label{D4}
	\end{definition}
\end{figure}
Quindi, ho dimostrato che la definizione di limmite è soddisfatta (con \( L = 0 \)).
\[
	\lim_{x \to +\infty} x = +\infty
\]
\label{D5}
Sia dato \( M>0 \) arbitrario. Definisco \( k := M \).\\
Sia dato \( x \ge k \). Allora \( x \ge M \).\\
Quindi è verificata la definizione di limite.
\subsection{Osservazioni}
\textbf{Non} è detto che un limite esista.
\begin{example}
	\[
		\lim_{x \to +\infty} sin(x)
	\]
	\[
		\lim_{x \to +\infty} cos(x)
	\]
	\label{D6}
	La funzione non entra in un intevallo limitato senza poi uscirne, quindi non esiste il limite.
\end{example}
Tuttavia, se una funzione ammette limite, allora esso è unico. La funzione dovrebbe
entrare in entrambe le strisce e non uscirne più, ma questo non è possibile.
\label{D7}
\subsection{Risultati utili per il calcolo dei limiti}
\begin{theorem}[Algebra dei limiti]
	Sia \( A \subseteq \mathbb{R} \) un insieme non limitato superiormente, \( f \) e \( g \)
	due funzioni. \( A \to \mathbb{R} \). Supponiamo che i limiti
	\[
		F:= \lim_{x \to +\infty} f(x)
	\]
	\[
		G:= \lim_{x \to +\infty} g(x)
	\]
	esistano e siano \textbf{finiti}. Allora
	\[
		\lim_{x \to +\infty} (f(x) + g(x)) = F+G
	\]
	\[
		\lim_{x \to +\infty} (f(x) \cdot g(x)) = F \cdot G
	\]
	TODO

	\textbf{parzialmente} al caso \( F \) o \( G \) siano infiniti, secondo
	le regole seguenti:
	\begin{itemize}
		\item \(
		      F + \infty = +\infty,\;\; F - \infty = -\infty\;\; \forall F \in \mathbb{R}
		      \)
		\item \(
		      +\infty + \infty = +\infty,\;\; +\infty - \infty = -\infty
		      \)
		\item \(
		      F \cdot \infty = \infty, \;\; \forall F \in \mathbb{R},\; F \neq 0
		      \)
		\item \(
		      \infty \cdot \infty = \infty
		      \)
		\item \(
		      \frac{F}{\infty} = 0 \;\; \forall F \in \mathbb{R}
		      \)
		\item \(
		      \frac{F}{0} = \infty \;\; \forall F \in \mathbb{R},\; F \neq 0
		      \)
		\item \(
		      \frac{0}{\infty} = 0
		      \)
		\item \(
		      \frac{\infty}{0} = \infty
		      \)
	\end{itemize}
	Il segno di \( \infty \) è da determinare secondo la regola usuale.
\end{theorem}

\subsection{Forme indeterminate}
Sono dei casi in cui il teorema \textbf{non} si applica e tutto può succdere:
\begin{itemize}
	\item \( +\infty - \infty \)
	\item \( 0 \cdot \infty \)
	\item \( \frac{0}{0} \)
	\item \( \frac{\infty}{\infty} \)
	\item \( 1^{\infty} \)
	\item \( 0^{0} \)
	\item \( \infty^{0} \)
\end{itemize}
\paragraph{\textbf{N.B.:}} in questo contesto, \( 0 \) , \( \infty \) TODO

\subsection{Esempi}
\begin{example}
	\[ \lim_{x \to +\infty} (x^2+\frac{1}{x})  \]
	\[
		\underbrace{x^2}_\text{\( +\infty \)} + \underbrace{\frac{1}{x}}_\text{\( 0 \)} \to +\infty
	\]
	Per \( x \to +\infty \) (per il teorema dell'algebra dei limiti)
\end{example}
\begin{example}
	\[
		\lim_{x \to +\infty} x^2-x^3 = +\infty - \infty
	\]
	\[
		\underbrace{x^3}_\text{\( +\infty \)}(\underbrace{\frac{1}{x}}_\text{\( 0 \)} - 1) \to -\infty
	\]
	Per \( x \to +\infty \)
\end{example}
\begin{example}
	TODO
\end{example}
\subsubsection{Esempi di limiti irrazionali}
\end{document}
