\documentclass[a4paper]{article}

\usepackage[utf8]{inputenc}
\usepackage[T1]{fontenc}
\usepackage{textcomp}
\usepackage[italian]{babel}
\usepackage{amsmath, amssymb}
\usepackage{amsfonts}
\usepackage{mdframed}
\usepackage{xcolor}
\usepackage{float}
\usepackage{tikz}
\usepackage{graphicx}
\graphicspath{{./figures/}}

\usepackage{import}
\usepackage{pdfpages}
\usepackage{transparent}
\usepackage{xcolor}

\usepackage{ntheorem}
\newtheorem{theorem}{Teorema}

% Useful definitions frame
\theoremstyle{break}
\theoremheaderfont{\bfseries}
\newmdtheoremenv[%
	linecolor=gray,leftmargin=0,%
	rightmargin=0,
	innertopmargin=8pt,%
	ntheorem]{define}{Definizioni utili}[section]

% Example frame
\theoremstyle{break}
\theoremheaderfont{\bfseries}
\newmdtheoremenv[%
	linecolor=gray,leftmargin=0,%
	rightmargin=0,
	innertopmargin=8pt,%
	ntheorem]{example}{Esempio}[section]

% Important definition frame
\theoremstyle{break}
\theoremheaderfont{\bfseries}
\newmdtheoremenv[%
	linecolor=gray,leftmargin=0,%
	rightmargin=0,
	backgroundcolor=gray!40,%
	innertopmargin=8pt,%
	ntheorem]{definition}{Definizione}[section]

% Exercise frame
\theoremstyle{break}
\theoremheaderfont{\bfseries}
\newmdtheoremenv[%
	linecolor=gray,leftmargin=0,%
	rightmargin=0,
	innertopmargin=8pt,%
	ntheorem]{exercise}{Esercizio}[section]


% figure support
\usepackage{import}
\usepackage{xifthen}
\pdfminorversion=7
\usepackage{pdfpages}
\usepackage{transparent}
\newcommand{\incfig}[1]{%
	\def\svgwidth{\columnwidth}
	\import{./figures/}{#1.pdf_tex}
}

\pdfsuppresswarningpagegroup=1

\begin{document}
\begin{titlepage}
	\begin{center}
		\vspace*{1cm}

		\Huge
		\textbf{Probabilità e Statistica\\Esercizi}

		\vspace{0.5cm}
		\LARGE
		UniVR - Dipartimento di Informatica

		\vspace{1.5cm}

		\textbf{Fabio Irimie}

		\vfill


		\vspace{0.8cm}


		2° Semestre 2023/2024

	\end{center}
\end{titlepage}


\tableofcontents
\pagebreak

\section{Introduzione}


\subsection{Numeri reali}
I numeri reali sono descritti tramite rappresentazioni decimali limitate o illimitate, periodiche
o non periodiche, e sono tutti i numeri razionali e irrazioneli; questo insieme viene indicato con
il simbolo \( \mathbb{R} \)

Proprietà necessarie dei numeri reali:
\begin{itemize}
	\item \textbf{\( 1^a \) proprietà (Eudosso-Archimede)}: due grandezze sono confrontabili quando esiste
	      un multiplo della minore che supera la maggiore. Ciò significa che non possiamo confrontare
	      linee con superfici, o superfici con volumi, ecc.

	      Questa proprietà veniva assunta come definizione di grandezze omogenee.

	      \textbf{Assioma}: dati due numeri reali positivi \( a, b \) con \( 0 < a < b \) esiste un
	      intero \( n \) tale che \( na > b \).
	\item \textbf{\( 2^a \) proprietà (Intervalli inscatolati)}: date due serie di grandezze:
	      \( a_1, a_2, \ldots, a_n \)
	      e \( b_1, b_2, \ldots, b_n \): la prima crescente (numeri della famiglia \( a \)) e la seconda
	      decrescente (numeri della famiglia \( b \)), in cui ogni \( a_k \) è minore di \( b_k \) e tali
	      che per ogni altra grandezza \( d \) si ha \( b_k - a_k < c \) per qualche \( k \), allora
	      esiste una grandezza \( c \) tale che per ogni \( k\;\; a_k \le c \le b_k \).
\end{itemize}


\subsection{Maggiorante}
\begin{definition}
	Sia \( S \subseteq \mathbb{R} \) un sottoinsieme di numeri reali. Un numero \( y \in \mathbb{R} \) `
	è un maggiorante dell'insieme \( S \) se per ogni \( x \in S \) si ha che \( y \ge x \).
\end{definition}
Se sommassimo un qualsiasi numero positivo a questo maggiorante si otterrebbe un altro maggiorante.

Se l'interballo tendesse verso \( +\infty \) non si sarebbe alcun maggiorante poichè \( +\infty \) non
è un numero reale. Esempi:
\begin{itemize}
	\item \( I = (1,10] \): tutti i maggioranti sono quelli per \( y \ge 10 \)
	\item \( I = [0,3) \): tutti i maggioranti sono quelli per \( y \ge 3 \)
	\item \( \mathbb{R} = (-\infty,+\infty) \): non ha maggiorante
\end{itemize}


\subsection{Minorante}
\begin{definition}
	Sia \( S \subseteq \mathbb{R}\) un sottoinsieme di numeri reali. Un numero \( y \in \mathbb{R} \)
	è un minorante dell'insieme \( S \) se per ogni \( x \in S \) si ha che \( y \le x \).
\end{definition}
Se sottraessimo un qualsiasi numero negativo a questo minorante si otterrebbe un altro minorante.

Se l'intervallo tendesse verso \( -\infty \) non ci sarebbe alcun minorante poichè \( -\infty \) non
è un numero reale. Esempi:
\begin{itemize}
	\item \( I = (1,10] \): tutti i minoranti sono quelli per \( y \le 1 \)
	\item \( I = [9,3) \): tutti i minoranti sono quelli per \( y \le 9 \)
	\item \( \mathbb{R} = (-\infty,+\infty) \): non ha minorante
\end{itemize}


\subsection{Estremo superiore}
Dato un insieme \( S \subseteq \mathbb{R} \), \( S \) è un insieme limitato superiormente con \( y \in \mathbb{R} \)
estremo superiore di \( S \) se:
\begin{itemize}
	\item \( y \) è un maggiorante di \( S \)
	\item \( y \) è il più piccolo maggiorante di \( S \)
\end{itemize}
Se \( S \) è un insieme illimitato superiormente allora l'estremo superiore di \( S \) è \( sup(S)=+\infty \).
Esempi:
\begin{itemize}
	\item \( I = (1,10] \): \( sup(I) = 10 \)
	\item \( I = (-\infty,0) \): \( sup(I) = 0 \)
	\item \( \mathbb{R} = (-\infty, +\infty) \): \( sup(\mathbb{R}) = +\infty \)
\end{itemize}


\subsection{Estremo inferiore}
Dato un insieme \( S \subseteq \mathbb{R} \), \( S \) è un insieme limitato inferiormente con \( y \in \mathbb{R} \)
estremo inferiore di \( S \) se:
\begin{itemize}
	\item \( y \) è un minorante di \( S \)
	\item \( y \) è il più grande minorante di \( S \)
\end{itemize}
Se \( S \) è un insieme illimitato inferiormente allora l'estremo inferiore di \( S \) è \( inf(S)=-\infty \).
Esempi:
\begin{itemize}
	\item \( I=[1,8) \): \( inf(I) = 1 \)
	\item \( I=(-13,0) \): \( inf(I) = -13 \)
	\item \( \mathbb{R} = (-\infty, +\infty) \): \( inf(\mathbb{R}) = -\infty \)
\end{itemize}


\subsection{Massimo}
\begin{definition}
	Sia \( S \subseteq \mathbb{R} \) un sottoinsieme reale, dove \( y \in \mathbb{R} \) è il massimo
	di \( S \) se \( y \) è l'estremo superiore di \( S \) e se \( y \in S \).
\end{definition}
Quindi se l'estremo superiore di un insieme appartiene all'insieme stesso, esso si chiamerà
massimo indicato con \( Max(S)=y \).


\subsection{Minimo}
\begin{figure}[H]
	\begin{definition}
		Sia \( S \subseteq \mathbb{R} \) un sottoinsieme reale, dove \( y \in \mathbb{R} \) è il minimo
		di \( S \) se \( y \) è l'estremo inferiore di \( S \) e se \( y \in S \).
	\end{definition}
\end{figure}
Quindi se l'estremo inferiore di un insieme appartiene all'insieme stesso, esso si chiamerà
minimo indicato con \( Min(S)=y \).

\begin{theorem}
	Ogni insieme di numeri reali che sia limitato superiormente ha estremo superiore.
\end{theorem}


\subsection{Funzioni}
\begin{definition}
	Una \textbf{funzione} è una corrispondenza che collega gli elementi di due insiemi dove tutti
	gli elementi del primo insieme hanno associati un solo elemento del secondo insieme:
	\[
		f: A \to B
	\]
\end{definition}
Questa è una funzione se e solo se a ogni elemento di \( A \) è associato uno e uno solo elemento di \( B \).

Tradotto in simboli diventa:
\[
	\forall a \in A \;\exists !\; b \in B \;tale\;che\; f: A \to B
\]
Esempio di funzione corretta:
\begin{figure}[H]
	\begin{center}
		\begin{tikzpicture}
			\draw[draw] (-1.5,0) circle (1cm) node[above, yshift=1.1cm] {$A$};
			\draw[draw] (1.5,0) circle (1cm) node[above, yshift=1.1cm] {$B$};

			\draw[draw, fill, blue] (-1.2,0.7) circle (0.5mm) node (1) {};
			\draw[draw, fill, blue] (-1.6,0.4) circle (0.5mm) node (2) {};
			\draw[draw, fill, blue] (-1.3,-0.3) circle (0.5mm) node (3) {};
			\draw[draw, fill, blue] (-1.8,-0.6) circle (0.5mm) node (4) {};
			\draw[draw, fill, blue] (-1.7,0) circle (0.5mm) node (9) {};

			\draw[draw, fill, red] (1.2,0.7) circle (0.5mm) node (5) {};
			\draw[draw, fill, red] (1.6,0.4) circle (0.5mm) node (6) {};
			\draw[draw, fill, red] (1.3,-0.3) circle (0.5mm) node (7) {};
			\draw[draw, fill, red] (1.8,-0.6) circle (0.5mm) node (8) {};

			\draw[->] (1) -- (5);
			\draw[->] (2) -- (6);
			\draw[->] (3) -- (7);
			\draw[->] (4) -- (8);
			\draw[->] (9) -- (8);

		\end{tikzpicture}
	\end{center}
	\caption{Esempio di funzione corretta}
\end{figure}

\subsubsection{Dominio di una funzione}
\begin{definition}
	Dato un insieme di partenza \( A \) gli elementi ai quali è applicata la funzione \( f \) sono
	il dominio stesso della funzione
\end{definition}
Esempio:
\[
	x \to x^2\; con\; D=\mathbb{R}
\] \[
	x \to \sqrt{x}\; con\; D=[0,+\infty)
\]
Si può dare un nome simbolico alla funzione scrivendo in questo modo:
\[
	f(x)=x^2\; con\; D=\mathbb{R}
\] \[
	f(x)=\sqrt{x}\; con\; D=[0,+\infty)
\]



\section{Limiti}
I limiti sono il calcolo infinitesimale, ovvero il
calcolo che si occupa di studiare il comportamento di una funzione in un intorno
di un punto.

Nelle definizioni che seguono, è data una funzione \( f:A \to \mathbb{R} \) il cui
dominio \( A \subseteq \mathbb{R} \) è un insieme \textbf{non} limitato superiormente.
(Questa ipotesi serve per definire i limiti per \( x \to +\infty \) )

\begin{definition}
	Sia \( L \in \mathbb{R} \). Si dice che
	\[ \lim_{x \to +\infty} f(x) = L  \]
	Se e solo se
	\[
		\forall \epsilon > 0\;\;\; \exists k>0 \;t.c.\; \forall x \subset A\footnote{Il dominio della funzione},
	\]
	\[
		x \ge k \to L-\epsilon \le f(x) \le L+\epsilon
	\]
	(Notazione alternativa: \( f(x) \to L \) per \( x \to +\infty \) )\\
	\textbf{La condizione deve essere soddisfatta per ogni \( \epsilon \) }.
	\begin{figure}[H]
		\begin{center}
			\begin{tikzpicture}[scale=0.5, domain=0:13]
				\draw[->] (-0.5,0) -- (13,0) node[right] {$x$};
				\draw[->] (0,-0.5) -- (0,7) node[above] {$y$};

				\coordinate (A) at (0,0);
				\coordinate (B) at (1,1);
				\coordinate (C) at (2,2);
				\coordinate (D) at (3,6);
				\coordinate (E) at (4,5);
				\coordinate (F) at (5,3);
				\coordinate (G) at (6,2.12);
				\coordinate (H) at (7,3.9);
				\coordinate (I) at (8,3);
				\coordinate (J) at (9,3);
				\coordinate (K) at (10,3);
				\coordinate (L) at (11,3);
				\coordinate (M) at (12,3);
				\coordinate (N) at (13,3);

				\draw [red, thick] plot [smooth, tension=0.7] coordinates { (A) (B) (C) (D) (E) (F) (G) (H) (I) (J) (K) (L) (M) (N) };

				\draw [] (0,3) -- (13,3) node[right] {$L$};
				\draw [dashed] (0,4) -- (13,4) node[right, scale=0.5] {$L+\epsilon$};
				\draw [dashed] (0,2) -- (13,2) node[right, scale=0.5] {$L-\epsilon$};

				\draw[fill, fill opacity=0.2, cyan] (0,4) rectangle (13,2);

				\draw [dashed] (0,3.2) -- (13,3.2) node[above right, scale=0.5] {$L+\epsilon^1$};
				\draw [dashed] (0,2.8) -- (13,2.8) node[below right, scale=0.5] {$L-\epsilon^1$};

				\draw[fill, fill opacity=0.2, green] (0,3.2) rectangle (13,2.8);

				\draw [dashed] (4.5,4) -- (4.5,0) node[below] {$k$};
				\draw [dashed] (7.8,3.2) -- (7.8,0) node[below] {$k^1$};
			\end{tikzpicture}
		\end{center}
		\caption{Definizione di limite}
	\end{figure}
	Per la definizione di limite, la funzione deve entrare in un intorno di \( L \) e non uscirne più.
	Questo vale per ogni \( \epsilon \), quindi anche per \( \epsilon^1 \).
\end{definition}
\begin{figure}[H]
	\begin{definition}
		Si dice che
		\[
			\lim_{x \to +\infty} f(x) = +\infty
		\]
		Se e solo se
		\[
			\forall M > 0\;\; \exists k>0\; \;t.c.\;\; \forall x \in A,
		\]
		\[
			x \ge k \to f(x) \ge M
		\]
		(Notazione alternativa: \( f(x) \to +\infty \) per \( x \to +\infty \))
		\begin{figure}[H]
			\begin{center}
				\begin{tikzpicture}[scale=0.5, domain=0:13]
					\draw[->] (-0.5,0) -- (13,0) node[right] {$x$};
					\draw[->] (0,-0.5) -- (0,7) node[above] {$y$};

					\coordinate (A) at (0,0);
					\coordinate (B) at (1,1);
					\coordinate (C) at (2,2);
					\coordinate (D) at (3,6);
					\coordinate (E) at (4,5);
					\coordinate (F) at (5,3);
					\coordinate (G) at (6,2.12);
					\coordinate (H) at (7,3.9);
					\coordinate (I) at (8,3);
					\coordinate (J) at (9,5);
					\coordinate (K) at (10,5);
					\coordinate (L) at (11,5);
					\coordinate (M) at (12,6);
					\coordinate (N) at (13,7);

					\draw [red, thick] plot [smooth, tension=0.7] coordinates { (A) (B) (C) (D) (E) (F) (G) (H) (I) (J) (K) (L) (M) (N) };

					\draw [] (0,4.8) -- (13,4.8) node[right] {$M$};
					\draw [] (0,6) -- (13,6) node[right] {$M^1$};

					\draw[fill, fill opacity=0.2, cyan] (0,4.8) rectangle (13,7);

					\draw [dashed] (8.8,4.8) -- (8.8,0) node[below] {$k$};
					\draw [dashed] (12,6) -- (12,0) node[below] {$k^1$};
				\end{tikzpicture}
			\end{center}
			\caption{Definizione di limite a \( +\infty \)}
		\end{figure}
	\end{definition}
\end{figure}
\begin{figure}[H]
	\begin{definition}
		Si dice che
		\[
			\lim_{x \to +\infty} f(x) = -\infty
		\]
		Se e solo se
		\[
			\forall M > 0\;\; \exists k>0\; \;t.c.\;\; \forall x \in A,
		\]
		\[
			x \ge k \to f(x) \le -M
		\]
		(Notazione alternativa: \( f(x) \to -\infty \) per \( x \to +\infty \))
		\begin{figure}[H]
			\begin{center}
				\begin{tikzpicture}[scale=0.5, domain=0:13]
					\draw[->] (-0.5,0) -- (13,0) node[right] {$x$};
					\draw[->] (0,-7) -- (0,2) node[above] {$y$};

					\coordinate (A) at (0,0);
					\coordinate (B) at (1,1);
					\coordinate (C) at (2,-2);
					\coordinate (D) at (3,-6);
					\coordinate (E) at (4,-5);
					\coordinate (F) at (5,-3);
					\coordinate (G) at (6,-2.12);
					\coordinate (H) at (7,-3.9);
					\coordinate (I) at (8,-3);
					\coordinate (J) at (9,-5);
					\coordinate (K) at (10,-5);
					\coordinate (L) at (11,-5);
					\coordinate (M) at (12,-6);
					\coordinate (N) at (13,-7);

					\draw [red, thick] plot [smooth, tension=0.7] coordinates { (A) (B) (C) (D) (E) (F) (G) (H) (I) (J) (K) (L) (M) (N) };

					\draw [] (0,-4.8) -- (13,-4.8) node[right] {$M$};
					\draw [] (0,-6) -- (13,-6) node[right] {$M^1$};

					\draw[fill, fill opacity=0.2, cyan] (0,-4.8) rectangle (13,-7);

					\draw [dashed] (8.8,-4.8) -- (8.8,0) node[above] {$k$};
					\draw [dashed] (12,-6) -- (12,0) node[above] {$k^1$};
				\end{tikzpicture}
			\end{center}
			\caption{Definizione di limite a \( -\infty \)}
		\end{figure}
	\end{definition}
\end{figure}


\subsection{Esempi}
\begin{example}
	\[
		\lim_{x \to +\infty} \frac{1}{x}=0\;\;\;Dominio=\mathbb{R}/\{0\}
	\]

	\begin{figure}[H]
		\begin{center}
			\begin{tikzpicture}
				\draw[->] (-0.5, 0) -- (7, 0) node[right] {$x$};
				\draw[->] (0, -2) -- (0, 5) node[above] {$y$};
				\draw[domain=0.2:7, smooth, variable=\x, red, thick] plot ({\x}, {1/\x});

				\draw [dashed] (0,1) -- (7,1) node[right, scale=0.8] {$0+\epsilon$};
				\draw [dashed] (0,-1) -- (7,-1) node[right, scale=0.8] {$0-\epsilon$};

				\draw[fill, fill opacity=0.2, cyan] (0,1) rectangle (7,-1);
			\end{tikzpicture}
		\end{center}
		\caption{Esempio di limite}
	\end{figure}
	Sia dato \( \epsilon > 0 \) arbitrario. Definisco \( k := \frac{1}{\epsilon} \).\\
	Sia dato \( x > 0 \) arbitrario, supponiamo \( x \ge k \). Allora
	\[
		0-\epsilon \le 0 \le \frac{1}{x} \le \frac{1}{k} = \frac{1}{\frac{1}{\epsilon}} = \epsilon
	\]
	Quindi, ho dimostrato che la definizione di limite è soddisfatta (con \( L=0 \)).
\end{example}
\begin{example}
	\[
		\lim_{x \to +\infty} x = +\infty
	\]

	\begin{figure}[H]
		\begin{center}
			\begin{tikzpicture}
				\draw[->] (-0.5, 0) -- (5, 0) node[right] {$x$};
				\draw[->] (0, -0.5) -- (0, 5) node[above] {$y$};
				\draw[domain=0:5, smooth, variable=\x, red, thick] plot ({\x}, {\x});

				\draw [dashed] (0,3) -- (5,3) node[right, scale=0.8] {$M$};

				\draw[fill, fill opacity=0.2, cyan] (0,3) rectangle (5,5);

				\draw [dashed] (3,3) -- (3,0) node[below] {$k$};
			\end{tikzpicture}
		\end{center}
		\caption{Esempio di limite a \( +\infty \)}
	\end{figure}
	Sia dato \( M>0 \) arbitrario. Definisco \( k := M \).\\
	Sia dato \( x \ge k \). Allora \( x \ge M \).\\
	Quindi è verificata la definizione di limite.

\end{example}


\subsection{Osservazioni}
\textbf{Non} è detto che un limite esista.
\[
	\lim_{x \to +\infty} sin(x)
\]
\[
	\lim_{x \to +\infty} cos(x)
\]
\begin{figure}[H]
	\begin{center}
		\begin{tikzpicture}
			\draw[->] (-5, 0) -- (5, 0) node[right] {$x$};
			\draw[->] (0, -2) -- (0, 2) node[above] {$y$};
			\draw[domain=-5:5, smooth, variable=\x, red, thick] plot ({\x}, {sin(\x r)});

			\draw [dashed] (-5,0.2) -- (5,0.2) node[right, scale=0.5] {$0+\epsilon$};
			\draw [dashed] (-5,-0.2) -- (5,-0.2) node[right, scale=0.5] {$0+\epsilon$};

			\draw[fill, fill opacity=0.2, cyan] (-5,0.2) rectangle (5,-0.2);
		\end{tikzpicture}
	\end{center}
	\caption{Esempio di limite non esistente}
\end{figure}
La funzione non entra in un intevallo limitato senza poi uscirne, quindi non esiste il limite.

\begin{figure}[H]
	\begin{center}
		\begin{tikzpicture}[scale=0.5, domain=0:13]
			\draw[->] (-0.5,0) -- (13,0) node[right] {$x$};
			\draw[->] (0,-0.5) -- (0,7) node[above] {$y$};

			\coordinate (A) at (0,0);
			\coordinate (B) at (1,1);
			\coordinate (C) at (2,2);
			\coordinate (D) at (3,6);
			\coordinate (E) at (4,5);
			\coordinate (F) at (5,3);
			\coordinate (G) at (6,2.12);
			\coordinate (H) at (7,3.9);
			\coordinate (I) at (8,3);
			\coordinate (J) at (9,3);
			\coordinate (K) at (10,3);
			\coordinate (L) at (11,3);
			\coordinate (M) at (12,3);
			\coordinate (N) at (13,3);

			\draw [red, thick] plot [smooth, tension=0.7] coordinates { (A) (B) (C) (D) (E) (F) (G) (H) (I) (J) (K) (L) (M) (N) };

			\draw [] (0,3) -- (13,3) node[right] {$L$};


			\draw [dashed] (0,3.2) -- (13,3.2) node[above right, scale=0.5] {$L+\epsilon$};
			\draw [dashed] (0,2.8) -- (13,2.8) node[below right, scale=0.5] {$L-\epsilon$};

			\draw[fill, fill opacity=0.2, cyan] (0,3.2) rectangle (13,2.8);

			\draw [dashed] (0,1) -- (13,1) node[right, scale=0.5] {$L+\epsilon^1$};
			\draw [dashed] (0,0.6) -- (13,0.6) node[right, scale=0.5] {$L-\epsilon^1$};

			\draw[fill, fill opacity=0.2, cyan] (0,1) rectangle (13,0.6);
		\end{tikzpicture}
	\end{center}
	\caption{Esempio di limite non esistente}
\end{figure}
Tuttavia, se una funzione ammette limite, allora esso è unico. Questa funzione dovrebbe
entrare in entrambe le strisce e non uscirne più, ma questo non è possibile.


\subsection{Risultati utili per il calcolo dei limiti}
\begin{theorem}[Algebra dei limiti]
	Sia \( A \subseteq \mathbb{R} \) un insieme non limitato superiormente, \( f \) e \( g \)
	due funzioni. \( A \to \mathbb{R} \). Supponiamo che i limiti
	\[
		F:= \lim_{x \to +\infty} f(x)
	\]
	\[
		G:= \lim_{x \to +\infty} g(x)
	\]
	esistano e siano \textbf{finiti}. Allora
	\[
		\lim_{x \to +\infty} (f(x) + g(x)) = F+G
	\]
	\[
		\lim_{x \to +\infty} (f(x) - g(x)) = F-G
	\]
	\[
		\lim_{x \to +\infty} (f(x) \cdot g(x)) = F \cdot G
	\]
	\[
		\lim_{x \to +\infty} \frac{f(x)}{g(x)} = \frac{F}{G}\;\;\;se\;G \neq 0
	\]
	Il teorema si estende \textbf{parzialmente} nel caso \( F \) o \( G \) siano infiniti, secondo
	le regole seguenti:
	\begin{itemize}
		\item \(
		      F + \infty = +\infty,\;\; F - \infty = -\infty\;\; \forall F \in \mathbb{R}
		      \)
		\item \(
		      +\infty + \infty = +\infty,\;\; +\infty - \infty = -\infty
		      \)
		\item \(
		      F \cdot \infty = \infty, \;\; \forall F \in \mathbb{R},\; F \neq 0
		      \)
		\item \(
		      \infty \cdot \infty = \infty
		      \)
		\item \(
		      \frac{F}{\infty} = 0 \;\; \forall F \in \mathbb{R}
		      \)
		\item \(
		      \frac{F}{0} = \infty \;\; \forall F \in \mathbb{R},\; F \neq 0
		      \)
		\item \(
		      \frac{0}{\infty} = 0
		      \)
		\item \(
		      \frac{\infty}{0} = \infty
		      \)
	\end{itemize}
	Il segno di \( \infty \) è da determinare secondo la regola usuale.
\end{theorem}


\subsection{Forme indeterminate}
Sono dei casi in cui il teorema \textbf{non} si applica e tutto può succdere:
\begin{itemize}
	\item \( +\infty - \infty \)
	\item \( 0 \cdot \infty \)
	\item \( \frac{0}{0} \)
	\item \( \frac{\infty}{\infty} \)
	\item \( 1^{\infty} \)
	\item \( 0^{0} \)
	\item \( \infty^{0} \)
\end{itemize}
\paragraph{\textbf{N.B.:}} in questo contesto, \( 0 \) , \( \infty \) e \( 1 \) sono da intendersi
come abbreviazioni.

\subsection{Esempi di calcolo di limiti}
\begin{example}
	\[ \lim_{x \to +\infty} (x^2+\frac{1}{x})  \]
	\[
		\underbrace{x^2}_\text{\( +\infty \)} + \underbrace{\frac{1}{x}}_\text{\( 0 \)} \to +\infty
	\]
	Per \( x \to +\infty \) (per il teorema dell'algebra dei limiti)
\end{example}
\begin{example}
	\[
		\lim_{x \to +\infty} x^2-x^3 = +\infty - \infty
	\]
	\[
		\underbrace{x^3}_\text{\( +\infty \)}(\underbrace{\frac{1}{x}}_\text{\( 0 \)} - 1) \to -\infty
	\]
	Per \( x \to +\infty \)
\end{example}
\begin{example}
	\[
		\lim_{x \to +\infty} (5x^6-4x) = +\infty - \infty
	\]
	\[
		\underbrace{x}_\text{\( +\infty \)}(\underbrace{5x^5}_\text{\( +\infty \)} - 4) \to +\infty
	\]
\end{example}


\subsection{Limiti razionali}
Se \( P \) è un polinomio di grado \( p \) e \( Q \) è un polinomio di grado \( q \), allora
\[
	\lim_{x \to +\infty} \frac{P(x)}{Q(x)}=
	\begin{cases}
		\pm \infty\;\;\; se\; p > q                          \\
		0\;\;\; se\; p < q                                   \\
		coefficiente\; denominante\; di\; P\;\;\; se\; p = q \\
		coefficiente\; denominante\; di\; Q\;\;\; se\; p = q
	\end{cases}
\]

\subsection{Limiti delle funzioni monotone}
\begin{theorem}[di monotonia]
	Sia \( A \subseteq \mathbb{R}\) un insieme non limitato superiormente e sia \( f:\;A \to \mathbb{R} \)
	una funzione monotona\footnote{Le funzioni \textbf{monotone} sono funzioni che
		sono sempre crescenti o sempre decrescenti}. Allora
	\[
		\lim_{x \to +\infty} f(x)\;\;esiste\;e
	\]
	\[
		\lim_{x \to +\infty} f(x) =
		\begin{cases}
			sup\{ f(x):\; x \in A \}\;\;\; se\;f\;cresce\;(non decrescecnte) \\
			inf\{ f(x):\; x \in A \}\;\;\; se\;f\;decresce\;(non crescente)
		\end{cases}
	\]
\end{theorem}
\( f: (0, +\infty) \to \mathbb{R} \)\\
\( f \) è strettamente crescente e limitata (l'immagine di \( f \) è un insieme limitato).

\begin{figure}[H]
	\begin{center}
		\begin{tikzpicture}[scale=0.5, domain=0:7]
			\draw[->] (-0.5,0) -- (7,0) node[right] {$x$};
			\draw[->] (0,-0.5) -- (0,7) node[above] {$y$};

			\coordinate (A) at (0,0);
			\coordinate (B) at (1,2);
			\coordinate (C) at (2,3.5);
			\coordinate (D) at (3,4.5);
			\coordinate (E) at (4,5);
			\coordinate (F) at (5,5);
			\coordinate (G) at (6,5);
			\coordinate (H) at (7,5);

			\draw [red, thick] plot [smooth, tension=0.8] coordinates { (A) (B) (C) (D) (E) (F) (H) };

			\draw [dashed] (0,5) -- (6,5) node[left, xshift=-3cm] {$5$};
		\end{tikzpicture}
	\end{center}
	\caption{Esempio di funzione monotona}
\end{figure}
\[
	\lim_{x \to +\infty} f(x) = 5
\]
\( g: (0,+\infty) \to \mathbb{R}\) è strettamente crescente e non limitata
\begin{figure}[H]
	\begin{center}
		\begin{tikzpicture}[scale=1.5]
			\draw[->] (-0.2, 0) -- (2, 0) node[right] {$x$};
			\draw[->] (0, -0.2) -- (0, 2) node[above] {$y$};
			\draw[domain=0.1:1.5, smooth, variable=\x, red, thick] plot ({\x}, {(\x)^2});

			\draw [dashed] (0,1) -- (2,1) node[right, scale=0.5] {};
		\end{tikzpicture}
	\end{center}
	\caption{Esempio di funzione monotona non limitata}
\end{figure}
\[
	\lim_{x \to +\infty} g(x) = +\infty
\]
\vspace{1cm}

\begin{figure}[H]
	\begin{center}
		\( f: \mathbb{R} \to \mathbb{R} \)
		\begin{tikzpicture}[scale=0.5, domain=0:13]
			\draw[->] (-0.5,0) -- (13,0) node[right] {$x$};
			\draw[->] (0,-0.5) -- (0,7) node[above] {$y$};

			\coordinate (A) at (0,0);
			\coordinate (B) at (1,1);
			\coordinate (C) at (2,2);
			\coordinate (D) at (3,6);
			\coordinate (E) at (4,5);
			\coordinate (F) at (5,3);
			\coordinate (G) at (6,2.12);
			\coordinate (H) at (7,3.9);
			\coordinate (I) at (8,3);
			\coordinate (J) at (9,3);
			\coordinate (K) at (10,3);
			\coordinate (L) at (11,3);
			\coordinate (M) at (12,3);
			\coordinate (N) at (13,3);

			\draw [red, thick] plot [smooth, tension=0.7] coordinates { (A) (B) (C) (D) (E) (F) (G) (H) (I) (J) (K) (L) (M) (N) };

			\draw [dashed] (7.2,0) -- (7.2,7) node[below, yshift=-3.5cm] {$5$};
		\end{tikzpicture}
	\end{center}
	\caption{Esempio di funzione ristrettamente monotona}
\end{figure}
Questa funzione non è monotona, ma se guardiamo ciò che succede epr \( x>5 \) si ottiene
una funzione monotona. Quindi la funzione globalmente non è monotona, ma è decrescente
ristrettamente a partire da \( x=5 \).

Per il teorema di monotonia, \[
	\lim_{x \to +\infty} f(x) = L
\]
\begin{example}
	\[
		\lim_{x \to +\infty} log(x) = +\infty
	\]
	\begin{figure}[H]
		\begin{center}
			\begin{tikzpicture}
				\draw[->] (0, 0) -- (5, 0) node[right] {$x$};
				\draw[->] (0, -2) -- (0, 2) node[above] {$y$};
				\draw[domain=0.1:5, smooth, variable=\x, red, thick] plot ({\x}, {ln((\x))});

				\draw [dashed] (0,1) -- (5,1) node[right, scale=0.5] {};
			\end{tikzpicture}
		\end{center}
		\caption{Esempio di funzione monotona non limitata}
	\end{figure}
	Per il teorema di monotonia:
	\[
		\lim_{x \to +\infty} log(x) = sup\{ log(x): x>0 \}
	\]
	\[
		\ge sup\{ log(e^n): n \in \mathbb{Z}, n>0 \}\;\;scelto\;arbitrariamente
	\]
	\[
		= sup\{ n \cdot log(e): n \in \mathbb{Z}, n>0 \} = +\infty
	\]
	Abbiamo dimostrato (per il postulato di Eudosso - Archimede) che il limite di questa
	funzione è uguale a \( +\infty \).
\end{example}
\begin{figure}[H]
	\begin{exercise}
		Dimostrare che:
		\[
			\lim_{x \to +\infty} e^x = +\infty
		\]
		\begin{figure}[H]
			\begin{center}
				\begin{tikzpicture}
					\draw[->] (-3, 0) -- (3, 0) node[right] {$x$};
					\draw[->] (0, -2) -- (0, 2) node[above] {$y$};
					\draw[domain=-3:0.8, smooth, variable=\x, red, thick] plot ({\x}, {e^(\x)});
				\end{tikzpicture}
			\end{center}
			\caption{Esempio di funzione monotona non limitata}
		\end{figure}
		E similmente che:
		\[
			\lim_{x \to +\infty} a^x = +\infty\;\;\;\forall a \in (0,+\infty)
		\]
	\end{exercise}
\end{figure}

\subsection{Teorema dei carabinieri}
\begin{theorem}[del confronto tra i limiti, o dei carabinieri]
	Sia \( A \subseteq \mathbb{R} \) un insieme non limitato superiormente e siano
	\( f,g,h: A \to \mathbb{R} \). Supponiamo che
	\[
		f(x) \le g(x) \le h(x)\;\;\;\forall x \in A
	\]
	Supponiamo inoltre che i limiti
	\[
		\lim_{x \to +\infty} f(x) = \lim_{x \to +\infty} h(x) = L
	\]
	esistano (e che siano uguali tra di loro). Allora
	\[
		\lim_{x \to +\infty} g(x) = L
	\]
	\begin{figure}[H]
		\begin{center}
			\begin{tikzpicture}
				\draw[->] (-0.2, 0) -- (5, 0) node[right] {$x$};
				\draw[->] (0, -0.2) -- (0, 4) node[above] {$y$};
				\draw[domain=0.48:5, smooth, variable=\x, blue, thick] plot ({\x}, {(-1/\x)+2}) node [below right] {h};
				\draw[domain=0.48:5, smooth, variable=\x, green, thick] plot ({\x}, {(1/\x)+2}) node [above right] {f};
				\draw[domain=0.48:10, smooth, variable=\x, red, thick, xscale=0.5] plot ({\x}, {(sin(\x r)/\x)+2}) node [right] {g};

				\draw [dashed] (0,2) node[left] {L} -- (5,2);
			\end{tikzpicture}
		\end{center}
		\caption{Teorema del confronto tra i limiti}
	\end{figure}
	Dobbiamo dimostrare che
	\[
		\forall \epsilon > 0\;\; \exists k>0\; \;t.c.\;\; \forall x \in A,
	\]
	\[
		x \ge k \to L-\epsilon \le g(x) \le L+\epsilon
	\]
	Prendiamo dunque \( \epsilon>0 \) arbitrario. Poichè \( \lim_{x \to +\infty} f(x)=L\),
	sappiamo che esiste \( k_f >0 \) t.c.
	\[
		\forall x \in A,\;\;\;\; x \ge k_f \to L-\epsilon \le f(x) \le L+\epsilon
	\]
	Allo stesso modo, poichè \( \lim_{x \to +\infty} h(x)=L\),
	sappiamo che esiste \( k_h >0 \) t.c.
	\[
		\forall x \in A,\;\;\;\; x \ge k_h \to L-\epsilon \le h(x) \le L+\epsilon
	\]
	Definiamo \( k:= max\{k_f,k_h\} \). Comunque preso \( x \in A \),
	se \( x \ge k \) allora vale che
	\[
		L-\epsilon \le f(x) \le g(x) \le h(x) \le L+\epsilon
	\]

\end{theorem}
\subsubsection{Variante}
Sia \( A \subseteq \mathbb{R} \) non limitato superiormente e siano \( f,g: A \to \mathbb{R} \)
\( t.c.\;f(x) \le g(x)\;\forall x \in A \).\\
Se \( \lim_{x \to +\infty} f(x)= +\infty \) allora \( \lim_{x \to +\infty} g(x) = +\infty \).
\begin{figure}[H]
	\begin{center}
		\begin{tikzpicture}
			\draw[->] (-0.2, 0) -- (3, 0) node[right] {$x$};
			\draw[->] (0, -0.2) -- (0, 4) node[above] {$y$};
			\draw[domain=0.1:2, smooth, variable=\x, blue, thick] plot ({\x}, {\x^2}) node [below right] {g};
			\draw[domain=0.1:2.7, smooth, variable=\x, red, thick] plot ({\x}, {\x^2/2}) node [above right] {f};
		\end{tikzpicture}
	\end{center}
	\caption{Teorema del confronto tra i limiti con 2 funzioni positive}
\end{figure}
Se \( \lim_{x \to +\infty} f(x)= -\infty \) allora \( \lim_{x \to +\infty} g(x) = -\infty \).
\begin{figure}[H]
	\begin{center}
		\begin{tikzpicture}
			\draw[->] (-0.2, 0) -- (3, 0) node[right] {$x$};
			\draw[->] (0, -4) -- (0, 0.5) node[above] {$y$};
			\draw[domain=0.1:2, smooth, variable=\x, blue, thick] plot ({\x}, {-(\x^2)}) node [below right] {f};
			\draw[domain=0.1:2.7, smooth, variable=\x, red, thick] plot ({\x}, {-(\x^2/2)}) node [above right] {g};
		\end{tikzpicture}
	\end{center}
	\caption{Teorema del confronto tra i limiti con 2 funzioni negative}
\end{figure}

\subsection{Limiti per \( x \to -\infty \) }
Sia \( A \subseteq \mathbb{R} \) un insieme non limitato inferiormente, \( f: A \to \mathbb{R} \) ,
\( L \in \mathbb{R} \cup \{+\infty, -\infty\}  \).
Diremo che:
\[
	\lim_{x \to -\infty} f(x) = L
\]
se e solo se
\[
	\lim_{x \to +\infty} f(-t) = L
\]
\begin{center}
	\(
	x=-t
	\)\\
	se \( x \to -\infty \) \\
	allora \( t \to +\infty \)
\end{center}

\subsection{Limiti per \( x \to x_0 \) }
Sia \( f: A \subseteq \mathbb{R} \), \( x_0 \in \mathbb{R} \). Per definire il limite di \( f \)
quando \( x \to 0 \), serve che \( f \) sia definita "vicino a \( x_0 \)", in un senso opportuno.
Noi supporremo, ad esempio, che il dominio \( A \) contenga almeno un intervallo del tipo
\( (x_0-\delta, x_0) \) oppure \( (x_0, x_0-\delta) \), con \( \delta>0 \). \textbf{Non} è richiesto, invece,
che \( f \) sia definita in \( x_0 \).
\begin{example}
	\[
		A = (-\infty,1) \cup (1,2)\;\;\; f: A \to \mathbb{R}
	\]
	\begin{figure}[H]
		\begin{center}
			\begin{tikzpicture}[scale=0.5, domain=0:13]
				\draw[->] (-0.5,3.5) -- (13,3.5) node[below] {$x$};
				\draw[->] (0,-0.5) -- (0,7) node[left] {$y$};

				\coordinate (A) at (-0.5,0);
				\coordinate (B) at (1,1);
				\coordinate (C) at (2,2);
				\coordinate (D) at (3,4);
				\coordinate (E) at (4,5);
				\coordinate (F) at (5,7);
				\coordinate (G) at (5,2);
				\coordinate (H) at (6,2);
				\coordinate (I) at (7,4);
				\coordinate (J) at (8,4);

				\draw [red, thick] plot [smooth, tension=0.7] coordinates { (A) (B) (C) (D) (E) (F)};
				\draw [red, thick] plot [smooth, tension=0.7] coordinates { (G) (H) (I) (J) };

				\draw[red] (5,2) circle (4pt);
				\draw[red] (8,4) circle (4pt);

				\draw [dashed] (5,0) -- (5,7) node[below, yshift=-3.5cm] {};

				\draw[fill, yellow, opacity=0.4] (-0.5,3.3) rectangle (0.5,3.7);
				\draw[fill, yellow, opacity=0.4] (4.5,3.3) rectangle (5.5,3.7);
				\draw[fill, yellow, opacity=0.4] (7.5,3.3) rectangle (8,3.7);

				\node [below left, scale=0.8] at (5,3.3) {1};
				\node [below, scale=0.8] at (8,3.3) {2};

			\end{tikzpicture}
		\end{center}
		\caption{Limiti su una funzione non continua}
	\end{figure}
	Posso definire \[
		\lim_{x \to -\infty} f(x),\; \lim_{x \to 2} f(x),\; \lim_{x \to 0} f(x),\; \lim_{x \to 0} f(x),\; \lim_{x \to 1} f(x)
	\]
	Non è detto però che tali limiti esistano
\end{example}
Sotto le ipotesi precedenti su \( f: A \subseteq \mathbb{R} \to \mathbb{R} \) e su \( x_0 \in \mathbb{R} \),
dato \( L \in \mathbb{R} \) diremo che \[
	\lim_{x \to x_0} f(x) = L
\]
se e solo se
\[
	\forall \epsilon > 0\;\;\; \exists \delta > 0\;t.c.\; \forall x \in A,
\]
\[
	x_0-\delta \le x \le x_0 + \delta\;\; e\;\; x \neq x_0
\]
\[
	\to L-\epsilon \le f(x) \le L+\epsilon
\]
\begin{figure}[H]
	\begin{center}
		\begin{tikzpicture}[scale=0.5, domain=0:13]
			\draw[->] (-0.5,0) -- (13,0) node[right] {$x$};
			\draw[->] (0,-0.5) -- (0,7) node[above] {$y$};

			\coordinate (A) at (0,0);
			\coordinate (B) at (1,1);
			\coordinate (C) at (2,2);
			\coordinate (D) at (3,1);
			\coordinate (E) at (4,3);
			\coordinate (F) at (5,4);
			\coordinate (G) at (6,4.2);
			\coordinate (H) at (7,4.5);
			\coordinate (I) at (8,4.5);
			\coordinate (J) at (9,5);
			\coordinate (K) at (10,6);
			\coordinate (L) at (11,7);
			\coordinate (M) at (12,6);
			\coordinate (N) at (13,7);

			\draw [red, thick] plot [smooth, tension=0.7] coordinates { (A) (B) (C) (D) (E) (F) (G) (H) (I) (J) (K) (L) (M) (N) };

			\draw [dashed] (0,5) -- (13,5) node[right] {$L$};
			\draw [dashed] (0,6) -- (13,6) node[right, scale=0.5] {$L+\epsilon$};
			\draw [dashed] (0,4) -- (13,4) node[right, scale=0.5] {$L-\epsilon$};

			\draw[fill, fill opacity=0.1, cyan] (0,4) rectangle (13,6);

			\draw [dashed] (9,7) -- (9,0) node[below] {$x_0$};

			\draw [dashed] (8,7) -- (8,0) node[below left] {$x_0-\delta$};
			\draw [dashed] (10,7) -- (10,0) node[below right] {$x_0+\delta$};

			\draw[fill, fill opacity=0.1, green] (8,0) rectangle (10,7);
		\end{tikzpicture}
	\end{center}
	\caption{Limite a \( x_0 \) }
\end{figure}

Sotto le ipotesi precedenti su \( f: A \subseteq \mathbb{R} \to \mathbb{R} \) e su \( x_0 \in \mathbb{R} \),
dato \( L \in \mathbb{R} \) diremo che \[
	\lim_{x \to x_0} f(x) = +\infty
\]
se e solo se
\[
	\forall M > 0\;\;\; \exists \delta > 0\;t.c.\; \forall x \in A,
\]
\[
	x_0-\delta \le x \le x_0 + \delta\;\; e\;\; x \neq x_0
\]
\[
	f(x) \ge M
\]
\begin{figure}[H]
	\begin{center}
		\begin{tikzpicture}[scale=0.5, domain=0:13]
			\draw[->] (-0.5,0) -- (13,0) node[right] {$x$};
			\draw[->] (0,-0.5) -- (0,7) node[above] {$y$};

			\coordinate (A) at (0,1);
			\coordinate (B) at (1,2);
			\coordinate (C) at (2,2);
			\coordinate (D) at (3,4);
			\coordinate (E) at (4,4.5);
			\coordinate (F) at (5.3,7);
			\coordinate (G) at (5.7,7);
			\coordinate (H) at (7,4.5);
			\coordinate (I) at (8,4.5);
			\coordinate (J) at (9,4);
			\coordinate (K) at (10,4.5);
			\coordinate (L) at (11,4);
			\coordinate (M) at (12,3);
			\coordinate (N) at (13,2);

			\draw [red, thick] plot [smooth, tension=0.7] coordinates { (A) (B) (C) (D) (E) (F) };
			\draw [red, thick] plot [smooth, tension=0.7] coordinates { (G) (H) (I) (J) (K) (L) (M) (N) };

			\draw [] (0,5) -- (13,5) node[right] {$M$};

			\draw[fill, fill opacity=0.1, cyan] (0,5) rectangle (13,7);

			\draw [dashed] (5.5,7) -- (5.5,0) node[below] {$x_0$};

			\draw [dashed] (4.5,7) -- (4.5,0) node[below left] {$x_0-\delta$};
			\draw [dashed] (6.5,7) -- (6.5,0) node[below right] {$x_0+\delta$};

			\draw[fill, fill opacity=0.1, green] (4.5,0) rectangle (6.5,7);
		\end{tikzpicture}
	\end{center}
	\caption{Limite a \( x_0 \) }
\end{figure}


Sotto le ipotesi precedenti su \( f: A \subseteq \mathbb{R} \to \mathbb{R} \) e su \( x_0 \in \mathbb{R} \),
dato \( L \in \mathbb{R} \) diremo che \[
	\lim_{x \to x_0} f(x) = -\infty
\]
se e solo se
\[
	\forall M > 0\;\;\; \exists \delta > 0\;t.c.\; \forall x \in A,
\]
\[
	x_0-\delta \le x \le x_0 + \delta\;\; e\;\; x \neq x_0
\]
\[
	f(x) \le M
\]
\subsection{Limiti unilateri}
Si possono anche dare le definizioni di limiti \textbf{unilateri}, da destra o da sinistra:
\[
	\lim_{x \to x_0^+} f(x) = \lim_{\underset{x > x_0}{x \to x_0}} f(x)
\]
\[
	\lim_{x \to x_0^-} f(x) = \lim_{\underset{x < x_0}{x \to x_0}} f(x)
\]
\begin{example}
	\[
		f: \mathbb{R} / \{0\} \to \mathbb{R}
	\]
	\[
		f(x) = \frac{1}{x}\;\; \forall x \in \mathbb{R} / \{0\}
	\]
	\[
		\lim_{x \to 0^+} (\frac{1}{x}) = +\infty
	\]
	\[
		\lim_{x \to 0^-} (\frac{1}{x}) = -\infty
	\]
	\[
		\lim_{x \to 0} (\frac{1}{x})\;\; \text{non esiste}
	\]

	\begin{figure}[H]
		\begin{center}
			\begin{tikzpicture}
				\draw[->] (-3, 0) -- (3, 0) node[right] {$x$};
				\draw[->] (0, -3) -- (0, 3) node[above] {$y$};
				\draw[domain=0.25:3, smooth, variable=\x, red, thick, yscale=0.7] plot ({\x}, {1/\x});
				\draw[domain=0.25:3, smooth, variable=\x, red, thick, yscale=0.7] plot ({-\x}, {-1/\x});
			\end{tikzpicture}
		\end{center}
		\caption{Limiti unilateri}
	\end{figure}

\end{example}

\subsection{Limiti di funzioni continue}
Sia \( A \subseteq \mathbb{R} \) un intervallo oppure un'unione finita di intervalli.
\begin{figure}[H]
	\begin{definition}
		Sia \( f: A \to  \mathbb{R} \), \( x_0 \in A \). Diremo che \( f \) è continua in \( x_0 \)
		se e solo se
		\[
			\lim_{x \to x_0} f(x) = f(x_0)
		\]
		Diremo che \( f \) è continua se e solo se \( f \) è continua in ogni punto del suo dominio
		\( x_0 \in A \).
	\end{definition}
\end{figure}

\begin{example}
	\[
		g: \mathbb{R} \to \mathbb{R}, \;\;\; g(x):=x\;\; \forall x \in \mathbb{R}
	\]
	è continua, perchè
	\[
		\lim_{x \to x_0} x = x_0\;\; \forall x_0 \in \mathbb{R}
	\]
	\begin{figure}[H]
		\begin{center}
			\begin{tikzpicture}
				\draw[->] (-2, 0) -- (2, 0) node[right] {$x$};
				\draw[->] (0, -2) -- (0, 2) node[above] {$y$};
				\draw[domain=-2:2, smooth, variable=\x, red, thick] plot ({\x}, {\x});
			\end{tikzpicture}
		\end{center}
		\caption{Eempio di funzione continua}
	\end{figure}

\end{example}
\begin{example}
	\[
		f: \mathbb{R} \to \mathbb{R}
	\]
	\[
		f(x):= \begin{cases}
			x\;\;\; se\; x \neq 2 \\
			31\;\;\; se\; x = 2
		\end{cases}
	\]
	Non è continua perchè
	\[
		\lim_{x \to 2} f(x) = 2 \neq f(2)
	\]
	Però \( f \) è continua in tutti gli \( x_0 \in  \mathbb{R} \), \( x_0 \neq 2 \):
	\[
		\lim_{x \to x_0} f(x) = f(x_0) = x_0
	\]
	\begin{figure}[H]
		\begin{center}
			\begin{tikzpicture}
				\draw[->] (-2, 0) -- (2, 0) node[right] {$x$};
				\draw[->] (0, -2) -- (0, 2) node[above] {$y$};
				\draw[domain=-2:0.45, smooth, variable=\x, red, thick] plot ({\x}, {\x});
				\draw[domain=0.55:2, smooth, variable=\x, red, thick] plot ({\x}, {\x});

				\draw[red] (0.5,0.5) circle (2pt);
				\draw[fill, red] (0.5,1.8) circle (2pt);
			\end{tikzpicture}
		\end{center}
		\caption{Eempio di funzione non continua}
	\end{figure}
\end{example}
\begin{example}
	\[
		h: \mathbb{R} / \{0\} \to \mathbb{R}
	\]
	\[
		h(x):= \frac{1}{x}\;\;\; \forall x \in \mathbb{R} / \{0\}
	\]
	Il dominio è un unione di 2 intervalli:
	\[
		(\mathbb{R}/0 = (-\infty,0) \cup (0,+\infty))
	\]
	È una funzione continua
	\begin{figure}[H]
		\begin{center}
			\begin{tikzpicture}
				\draw[->] (-3, 0) -- (3, 0) node[right] {$x$};
				\draw[->] (0, -3) -- (0, 3) node[above] {$y$};
				\draw[domain=0.25:3, smooth, variable=\x, red, thick, yscale=0.7] plot ({\x}, {1/\x});
				\draw[domain=0.25:3, smooth, variable=\x, red, thick, yscale=0.7] plot ({-\x}, {-1/\x});
			\end{tikzpicture}
		\end{center}
		\caption{Esmpio di funzione continua}
	\end{figure}

\end{example}
\begin{example}
	\[
		l: \mathbb{R} \to \mathbb{R}
	\]
	\[
		l(x):=\begin{cases}
			\frac{1}{x}\;\;\; se\; x \neq 0 \\
			5\;\;\; se\; x = 0
		\end{cases}
	\]
	Questa funzione non è continua perchè il limite a 0 non esiste:
    \[
        \lim_{x \to 0} l(x) = \nexists
    \]
    ma:
    \[
    \lim_{x \to 0} |l(x)| = +\infty
    \] 
	\begin{figure}[H]
		\begin{center}
			\begin{tikzpicture}
				\draw[->] (-3, 0) -- (3, 0) node[right] {$x$};
				\draw[->] (0, -3) -- (0, 3) node[above] {$y$};
				\draw[domain=0.25:3, smooth, variable=\x, red, thick, yscale=0.7] plot ({\x}, {1/\x});
				\draw[domain=0.25:3, smooth, variable=\x, red, thick, yscale=0.7] plot ({-\x}, {-1/\x});
				\draw[fill, red] (0,1.5) circle (2pt) node[left] {5};
			\end{tikzpicture}
		\end{center}
		\caption{Esmpio di funzione non continua}
	\end{figure}
\end{example}

\begin{example}
    \[
    m: \mathbb{R} \to \mathbb{R}
    \] 
    \[
    m(x):=\begin{cases}
        x^2\;\;\; se\; x \neq 0 \\
        -2\;\;\; se\; x = 0
    \end{cases}
    \]
    Non è continua perchè:
    \[
    \lim_{x \to 0} m(x) = \lim_{x \to 0} x^2 = 0 \neq m(0)
    \] 
    \begin{figure}[H]
		\begin{center}
			\begin{tikzpicture}
				\draw[->] (-3, 0) -- (3, 0) node[right] {$x$};
				\draw[->] (0, -2) -- (0, 3) node[above] {$y$};
				\draw[domain=0.1:2, smooth, variable=\x, red, thick, yscale=0.7] plot ({\x}, {\x^2});
				\draw[domain=0.1:2, smooth, variable=\x, red, thick, yscale=0.7] plot ({-\x}, {\x^2});

                \draw[red, thick] (0,0) circle (3pt);
                \draw[fill, red] (0,-1) circle (2.5pt) node[left] {-2};
			\end{tikzpicture}
		\end{center}
		\caption{Esmpio di funzione non continua}
	\end{figure}
\end{example}
\end{document}
