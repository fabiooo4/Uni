\documentclass[a4paper]{article}
\usepackage{import}
\usepackage[utf8]{inputenc}
\usepackage[T1]{fontenc}
\usepackage{textcomp}
\usepackage[italian]{babel}
\usepackage{amsmath, amssymb}
\usepackage{booktabs,xltabular}
\usepackage{amsfonts}
\usepackage{amsthm}
\usepackage{cancel}
\usepackage{mdframed}
\usepackage{makecell}
\usepackage{float}
\usepackage{xcolor}
\usepackage{listings}
\usepackage{gensymb}
\usepackage{graphicx}
\usepackage{bodeplot}
\usepackage{tikz}
\usetikzlibrary{shapes, arrows, automata, petri, decorations.markings, decorations.pathreplacing, positioning, calc}
\usepackage{circuitikz}
\usepackage[label=corner]{karnaugh-map}
\graphicspath{{./figures/}}

% Set default font to sans-serif
\renewcommand{\familydefault}{\sfdefault} 
\usepackage{eulervm}

\usepackage{forest}

\usepackage{mathtools}
\DeclarePairedDelimiter\ceil{\lceil}{\rceil}
\DeclarePairedDelimiter\floor{\lfloor}{\rfloor}

% \usepackage{ntheorem}

\usepackage{import}
\usepackage{pdfpages}
\usepackage{transparent}
\usepackage{xcolor}

\usepackage{hyperref}
\hypersetup{
    colorlinks=false,
}

% Code blocks
\definecolor{codegreen}{rgb}{0,0.6,0}
\definecolor{codegray}{rgb}{0.5,0.5,0.5}
\definecolor{codepurple}{rgb}{0.58,0,0.82}
\definecolor{backcolour}{rgb}{0.95,0.95,0.95}

\lstdefinestyle{mystyle}{
	backgroundcolor=\color{backcolour},
	commentstyle=\color{codegreen},
	keywordstyle=\color{magenta},
	numberstyle=\tiny\color{codegray},
	stringstyle=\color{codepurple},
	basicstyle=\ttfamily\footnotesize,
	breakatwhitespace=false,
	breaklines=true,
	captionpos=b,
	keepspaces=true,
	numbers=left,
	numbersep=5pt,
	showspaces=false,
	showstringspaces=false,
	showtabs=false,
	tabsize=2
}

\lstset{style=mystyle}

\usepackage{color}
\usepackage{import}
\usepackage{pdfpages}
\usepackage{transparent}
\usepackage{xcolor}

% Example frame
\theoremstyle{definition}
\newmdtheoremenv[%
	linecolor=gray,leftmargin=0,%
	rightmargin=0,
	innertopmargin=8pt,%
	innerbottommargin=8pt,
	ntheorem]{example}{Esempio}[section]

% Important definition frame
\theoremstyle{definition}
\newmdtheoremenv[%
	linecolor=gray,leftmargin=0,%
	rightmargin=0,
	backgroundcolor=gray!40,%
	innertopmargin=8pt,%
	innerbottommargin=8pt,
	ntheorem]{definition}{Definizione}[section]

% Exercise frame
\theoremstyle{definition}
\newmdtheoremenv[%
	linecolor=gray,leftmargin=0,%
	rightmargin=0,
	innertopmargin=8pt,%
	innerbottommargin=8pt,
	ntheorem]{exercise}{Esercizio}[section]

% Theorem frame
\theoremstyle{definition}
\newmdtheoremenv[%
  linecolor=gray,leftmargin=0,%
  rightmargin=0,
  innertopmargin=8pt,%
  innerbottommargin=8pt,
  ntheorem]{theorem}{Teorema}[section]

\theoremstyle{definition}
\newmdtheoremenv[%
  linecolor=gray,leftmargin=0,%
  rightmargin=0,
  innertopmargin=8pt,%
  innerbottommargin=8pt,
  ntheorem]{define}{Definizione utile}[section]

% figure support
\usepackage{import}
\usepackage{xifthen}
\pdfminorversion=7
\usepackage{pdfpages}
\usepackage{transparent}
\newcommand{\incfig}[1]{%
	\def\svgwidth{\columnwidth}
	\import{./figures/}{#1.pdf_tex}
}

% FSM tikz
\tikzset{
    place/.style={
        circle,
        thick,
        draw=black,
        minimum size=6mm,
    },
        state/.style={
        circle,
        thick,
        draw=blue!75,
        fill=blue!20,
        minimum size=6mm,
    },
}

\usepackage{pgfplots}
\pgfplotsset{compat=1.18}

\pdfsuppresswarningpagegroup=1


\pgfplotsset{width=7cm}

\begin{document}
\noindent
Lorem ipsum dolor sit amet, consectetur adipiscing elit. Sed ac purus sit amet
nisl tincidunt tincidunt. Nullam nec mi et neque pharetra sollicitudin. Praesent
imperdiet mi nec ante. Donec ullamcorper, felis non sodales commodo, lectus velit
ultrices augue, a dignissim nibh lectus placerat pede. Vivamus nunc nunc, molestie
nec, eleifend ut, fringilla vel, neque. Proin tellus mi, eleifend non venenatis
sit amet, ullamcorper at ligula. Nunc dapibus. Lorem ipsum dolor sit amet,
consectetur adipiscing elit. Sed ac purus sit amet nisl tincidunt tincidunt.

\begin{figure}[H]
  \centering
  \begin{tikzpicture}
    \begin{axis}[
      clip=false,
      width=0.5\textwidth,
      height=0.4\textwidth,
      xmin= -22, xmax= 22,
      ymin= 0, ymax = 5,
      ytick = \empty,
      xtick = {-10,10},
      xticklabel style = {scale=0.8},
      axis lines = middle,
    ]
      \addplot[-latex,blue,thick] coordinates {
          (-5,0) (-5,1)
        };
      \addplot[-latex,blue,thick] coordinates {
          (-10,0) (-10,2)
        };

      \addplot[-latex,blue,thick] coordinates {
          (5,0) (5,1)
        };
      \addplot[-latex,blue,thick] coordinates {
          (10,0) (10,2)
        };

      \node at (xticklabel* cs:0.5) [below,yshift=-0.5cm] {$f_c = 5$};
    \end{axis}
  \end{tikzpicture}
  \begin{tikzpicture}
    \node at (0,0) {};
    \draw (0,1.5) -- ++(0.5,0) -- ++(0.5,0.5) node[midway,below,yshift=-0.2cm] {$T_c$};
    \draw (0,1.5) ++(0.5,0) ++(0.5,0) -- ++(0.5,0);
  \end{tikzpicture}
  \hspace{0.1cm}
  \begin{tikzpicture}
    \begin{axis}[
      clip=false,
      width=0.5\textwidth,
      height=0.4\textwidth,
      xmin= -22, xmax= 22,
      ymin= 0, ymax = 5,
      ytick = \empty,
      xticklabel style = {scale=0.8},
      axis lines = middle,
    ]
      \addplot[-latex,blue,thick] coordinates {
          (-5,0) (-5,1)
        };
      \addplot[-latex,blue,thick] coordinates {
          (-10,0) (-10,2)
        };

      \addplot[-latex,blue,thick] coordinates {
          (5,0) (5,1)
        };
      \addplot[-latex,blue,thick] coordinates {
          (10,0) (10,2)
        };

      \addplot[-latex,green!50!black,thick] coordinates {
          (-15,0) (-15,1)
        };
      \addplot[-latex,green!50!black,thick] coordinates {
          (-10,2) (-10,3)
        };
      \addplot[-latex,green!50!black,thick] coordinates {
          (15,0) (15,1)
        };
      \addplot[-latex,green!50!black,thick] coordinates {
          (10,2) (10,3)
        };
      \addplot[-latex,green!50!black,thick] coordinates {
          (-5,1) (-5,3)
        };
      \addplot[-latex,green!50!black,thick] coordinates {
          (5,1) (5,3)
        };

      \addplot[-latex,red,thick] coordinates {
          (-20,0) (-20,1)
        };
      \addplot[-latex,red,thick] coordinates {
          (20,0) (20,1)
        };
      \addplot[-latex,red,thick] coordinates {
          (-15,1) (-15,2)
        };
      \addplot[-latex,red,thick] coordinates {
          (15,1) (15,2)
        };
      \addplot[-latex,red,thick] coordinates {
          (-10,3) (-10,4)
        };
      \addplot[-latex,red,thick] coordinates {
          (10,3) (10,4)
        };

      \draw[dashed] (axis cs:-10,0) rectangle (axis cs:10,5);
      \node at (axis cs:0,5) [above] {Rettangolo di supporto \( 2B \)};

      \node[opacity=0] at (xticklabel* cs:0.5) [below,yshift=-0.5cm] {$f_c = 5$};
    \end{axis}
  \end{tikzpicture}
  \[
  \hspace{6.7cm}
  \downarrow
  \] 
  \hspace{6.5cm}
  \begin{tikzpicture}
    \begin{axis}[
      clip=false,
      width=0.5\textwidth,
      height=0.4\textwidth,
      xmin= -22, xmax= 22,
      ymin= 0, ymax = 5,
      ytick = \empty,
      xticklabel style = {scale=0.8},
      axis lines = middle,
    ]
      \addplot[-latex,blue,thick] coordinates {
          (-5,0) (-5,3)
        };
      \addplot[-latex,blue,thick] coordinates {
          (5,0) (5,3)
        };
      \addplot[-latex,blue,thick] coordinates {
          (-10,0) (-10,4)
        };
      \addplot[-latex,blue,thick] coordinates {
          (10,0) (10,4)
        };
    \end{axis}
  \end{tikzpicture}
\end{figure}

\noindent
Lorem ipsum dolor sit amet, consectetur adipiscing elit. Sed ac purus sit amet
nisl tincidunt tincidunt. Nullam nec mi et neque pharetra sollicitudin. Praesent
imperdiet mi nec ante. Donec ullamcorper, felis non sodales commodo, lectus velit
ultrices augue, a dignissim nibh lectus placerat pede. Vivamus nunc nunc, molestie
nec, eleifend ut, fringilla vel, neque. Proin tellus mi, eleifend non venenatis
sit amet, ullamcorper at ligula. Nunc dapibus. Lorem ipsum dolor sit amet,
consectetur adipiscing elit. Sed ac purus sit amet nisl tincidunt tincidunt.
\end{document}
