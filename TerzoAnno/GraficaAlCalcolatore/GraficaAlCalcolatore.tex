\documentclass[a4paper]{article}
\usepackage{import}
\usepackage[utf8]{inputenc}
\usepackage[T1]{fontenc}
\usepackage{textcomp}
\usepackage[italian]{babel}
\usepackage{amsmath, amssymb}
\usepackage{booktabs,xltabular}
\usepackage{amsfonts}
\usepackage{subcaption}
\usepackage{amsthm}
\usepackage{cancel}
\usepackage{mdframed}
\usepackage{makecell}
\usepackage{float}
\usepackage{xcolor}
\usepackage{listings}
\usepackage{gensymb}
\usepackage{graphicx}
\usepackage{bodeplot}
\usepackage{physics}
\usepackage{tikz}
\usetikzlibrary{shapes, arrows, automata, petri, decorations.markings, decorations.pathreplacing, positioning, calc, quotes}
\usepackage{circuitikz}
\usepackage[label=corner]{karnaugh-map}
\graphicspath{{./figures/}}

% Set default font to sans-serif
\renewcommand{\familydefault}{\sfdefault} 
\usepackage{eulervm}

\usepackage{forest}

\usepackage{mathtools}
\DeclarePairedDelimiter\ceil{\lceil}{\rceil}
\DeclarePairedDelimiter\floor{\lfloor}{\rfloor}

% \usepackage{ntheorem}

\usepackage{import}
\usepackage{pdfpages}
\usepackage{transparent}
\usepackage{xcolor}

\usepackage{hyperref}
\hypersetup{
    colorlinks=false,
}

% Code blocks
\definecolor{codegreen}{rgb}{0,0.6,0}
\definecolor{codegray}{rgb}{0.5,0.5,0.5}
\definecolor{codepurple}{rgb}{0.58,0,0.82}
\definecolor{backcolour}{rgb}{0.95,0.95,0.95}

\lstdefinestyle{mystyle}{
	backgroundcolor=\color{backcolour},
	commentstyle=\color{codegreen},
	keywordstyle=\color{magenta},
	numberstyle=\tiny\color{codegray},
	stringstyle=\color{codepurple},
	basicstyle=\ttfamily\footnotesize,
	breakatwhitespace=false,
	breaklines=true,
	captionpos=b,
	keepspaces=true,
	numbers=left,
	numbersep=5pt,
	showspaces=false,
	showstringspaces=false,
	showtabs=false,
	tabsize=2
}

\lstset{style=mystyle}

\usepackage{color}
\usepackage{import}
\usepackage{pdfpages}
\usepackage{transparent}
\usepackage{xcolor}

% Example frame
\theoremstyle{definition}
\newmdtheoremenv[%
	linecolor=gray,leftmargin=0,%
	rightmargin=0,
	innertopmargin=8pt,%
	innerbottommargin=8pt,
	ntheorem]{example}{Esempio}[section]

% Important definition frame
\theoremstyle{definition}
\newmdtheoremenv[%
	linecolor=gray,leftmargin=0,%
	rightmargin=0,
	backgroundcolor=gray!40,%
	innertopmargin=8pt,%
	innerbottommargin=8pt,
	ntheorem]{definition}{Definizione}[section]

% Exercise frame
\theoremstyle{definition}
\newmdtheoremenv[%
	linecolor=gray,leftmargin=0,%
	rightmargin=0,
	innertopmargin=8pt,%
	innerbottommargin=8pt,
	ntheorem]{exercise}{Esercizio}[section]

% Theorem frame
\theoremstyle{definition}
\newmdtheoremenv[%
  linecolor=gray,leftmargin=0,%
  rightmargin=0,
  innertopmargin=8pt,%
  innerbottommargin=8pt,
  ntheorem]{theorem}{Teorema}[section]

\theoremstyle{definition}
\newmdtheoremenv[%
  linecolor=white,leftmargin=0,%
  rightmargin=0,
  innertopmargin=8pt,%
  innerbottommargin=8pt,
  ntheorem]{define}{Definizione utile}[section]

% figure support
\usepackage{import}
\usepackage{xifthen}
\pdfminorversion=7
\usepackage{pdfpages}
\usepackage{transparent}
\newcommand{\incfig}[1]{%
	\def\svgwidth{\columnwidth}
	\import{./figures/}{#1.pdf_tex}
}

% FSM tikz
\tikzset{
    place/.style={
        circle,
        thick,
        draw=black,
        minimum size=6mm,
    },
        state/.style={
        circle,
        thick,
        draw=black,
        fill=white,
        minimum size=6mm,
    },
}

\pdfsuppresswarningpagegroup=1

\usepackage{pgfplots}
\pgfplotsset{compat=1.18,width=10cm}

% Save plots as pdf and reuse them without compiling every time
\usetikzlibrary{external}
\tikzexternalize[prefix=figures/tikz/, optimize=false]


\begin{document}

\begin{titlepage}
	\begin{center}
		\vspace*{1cm}

		\Huge
		\textbf{Probabilità e Statistica\\Esercizi}

		\vspace{0.5cm}
		\LARGE
		UniVR - Dipartimento di Informatica

		\vspace{1.5cm}

		\textbf{Fabio Irimie}

		\vfill


		\vspace{0.8cm}


		2° Semestre 2023/2024

	\end{center}
\end{titlepage}


\tableofcontents
\pagebreak

\section{Introduzione}

\subsection{Interazione con l'utente}
Un modo per far interagire l'utente con il programma è l'interfaccia grafica. L'interazione
può essere ottenuta in vari modi, ad esempio:
\begin{itemize}
    \item Finestre di dialogo
    \item Realtà virtuale
    \item Realtà aumentata
    \item Giochi
\end{itemize}

\subsection{Sintesi e analisi}
La sintesi è il processo di creazione di un'immagine a partire da una descrizione matematica,
mentre l'analisi è il processo di estrazione di informazioni da un concetto già esistente.

\section{Modeling}
La modellazione 3D è un processo di \textbf{descrizione} di un oggetto o una scena al fine
di poterla disegnare. La descrizione avviene attraverso:
\begin{itemize}
  \item \textbf{Struttura}: Viene descritta dalla geometria degli oggetti e dalla loro
    posizione reciproca nello spazio tridimensionale
    \begin{itemize}
      \item Definizione geometrica
      \item Trasformazioni 3D
    \end{itemize}

  \item \textbf{Apparenza}: Descrive come la superficie del modello interagisce con
    la luce (colori, riflessi, ecc...)
    \begin{itemize}
      \item Definizione telecamere virtuali
      \item Definizione sorgenti di luce
      \item Definizione proprietà dei materiali
    \end{itemize}
\end{itemize}

\subsection{Definizione geometrica}
Ci sono varie tecniche di modellazione:
\begin{itemize}
  \item \textbf{Low poly diretta}, ad esempio con Wings3D. È la costruzione manuale di una mesh
    poligonale a bassa risoluzione, partendo anche da primitive già fatte.
  \item \textbf{Subdivision surfaces}, ad esempio con Blender. Si parte da una mesh
    poligonale a bassa risoluzione e si applicano algoritmi di suddivisione per
    ottenere superfici più lisce e dettagliate.
  \item \textbf{Digital sculpting}, ad esempio con ZBrush
  \item \textbf{Modellazione procedurale}, ad esempio con Houdini. Si utilizzano algoritmi
    e regole per generare automaticamente modelli 3D complessi, ad esempio generazione di
    terreni, vegetazione, edifici, ecc...
\end{itemize}

\subsubsection{Mesh}
Gli oggetti tridimensionali vengono codificati come una maglia (mesh) di triangoli.
I triangoli vengono utilizzati perchè sono il poligono più semplice che può essere utilizzato
per approssimare una qualsiasi superficie. Una mesh è composta da:
\begin{itemize}
  \item \textbf{Vertici}: Punti nello spazio 3D
  \item \textbf{Facce}: Insiemi di vertici che formano triangoli
\end{itemize}

\begin{definition}[Definizione matematica di mesh]
  Una mesh di triangoli è una discretizzazione lineare a tratti di una superficie
  continua (un "2-manifold") immersa in \( \mathbb{R}^3 \), cioè un oggetto bidimensionale
  che si trova in uno spazio tridimensionale. Le componenti sono:
  \begin{itemize}
    \item \textbf{Geometria}: i vertici, ciascuno con coordinate \( (x, y, z) \in \mathbb{R}^3 \)
    \item \textbf{Topologia}: come sono connessi tra loro i vertici, nel caso di una tri-mesh
      ogni faccia è definita da un insieme di tre vertici
  \end{itemize}
\end{definition}

\subsection{Trasformazioni 3D}
Per posizionare e orientare gli oggetti nello spazio 3D, si utilizzano trasformazioni
geometriche, che possono essere rappresentate tramite matrici. Le principali trasformazioni sono:
\begin{itemize}
  \item \textbf{Traslazione}: Spostamento di un oggetto da una posizione a un'altra
  \item \textbf{Rotazione}: Rotazione di un oggetto attorno a un asse specifico
  \item \textbf{Scalatura}: Modifica delle dimensioni di un oggetto lungo gli assi
    X, Y e Z
\end{itemize}

\subsection{Telecamere virtuali}
Per visualizzare una scena 3D su uno schermo 2D, è necessario utilizzare
una telecamera virtuale che definisce il punto di vista da cui viene osservata la scena.
Il problema è che nel passaggio dal 2D al 3D c'è perdita di informazione.

Per definire una telecamera virtuale, sono necessari:
\begin{itemize}
  \item \textbf{View point}: da dove si osserva
  \item \textbf{Look at point}: dove si guarda
  \item \textbf{View direction}: orientamento della telecamera
  \item \textbf{Regole di proiezione}:
    \begin{itemize}
      \item Ortografica: mantiene le proporzioni reali degli oggetti
      \item Prospettica: simula la visione umana, con oggetti
        più lontani che appaiono più piccoli
    \end{itemize}
\end{itemize}

\subsubsection{Proiezione}
Il mondo non è infinito, quindi bisogna definire il \textbf{cono di vista} (frustum),
che delimita la porzione di scena visibile dalla telecamera. Il frustum è definito
dal parallelepipedo delimitato da due piani:
\begin{itemize}
  \item \textbf{Near plane}: piano più vicino alla telecamera
  \item \textbf{Far plane}: piano più lontano dalla telecamera
\end{itemize}
Gli oggetti al di fuori del frustum non vengono proiettati (fase di clipping). La proiezione
avviene su un piano di vista (view plane), che rappresenta lo schermo 2D.

\subsection{Illuminazione}
Tramite l'illuminazione si riesce a distinguere la forma degli oggetti tridimensionali.
La modellazione delle luci della scena si occupa del loro posizionamento e del tipo di luce
utilizzata. I tipi di luce più comuni sono:
\begin{itemize}
  \item \textbf{Directional light}: luce proveniente da una direzione specifica, simula
    la luce solare
  \item \textbf{Point light}: luce che si propaga in tutte le direzioni da un punto
    specifico, simula una lampadina
  \item \textbf{Spotlight}: luce che si propaga in un cono da un punto specifico,
    simula un faro
  \item \textbf{Ambient light}: luce diffusa che illumina uniformemente tutta la scena,
    senza una direzione specifica
\end{itemize}

\subsection{Proprietà dei materiali}
Il materiale di cui è fatta la superficie di un oggetto condiziona il suo aspetto nel
momento in cui viene colpito dalla luce. Le proprietà principali dei materiali sono:
\begin{itemize}
  \item \textbf{Colore}
  \item \textbf{Riflettività}
  \item \textbf{Rugosità}
\end{itemize}

\section{Rendering}
Il rendering ha l'obiettivo di creare un'\textbf{immagine bidimensionale} a partire dalla
descrizione di una scena tridimensionale. Ogni pixel dell'immagine deve avere un colore
che dipende dalla geometria, dall'illuminazione e dalle proprietà dei materiali della
scena. Per renderizzare una scena c'è bisogno di tradurre il processo fisico della luce
in un algoritmo e questo ha bisogno di molte semplificazioni e approssimazioni.

\subsection{Modello fisico dell'illuminazione}
La luce è una radiazione elettromagnetica con lunghezza d'onda tra i 400 e i 700 nm che
parte da una \textbf{sorgente} verso un \textbf{ricevente}. La sorgente può essere
un \textbf{emittente} (sorgente primaria) oppure un \textbf{riflettore} (sorgente secondaria).
La lunghezza d'onda determina il colore della luce percepita dall'occhio umano:
\begin{itemize}
  \item Luce monocromatica: Quando è presente soltanto una lunghezza d'onda (es.
    laser)

  \item Luce policromatica: Quando sono presenti più lunghezze d'onda (es. luce solare)
\end{itemize}
L'energia trasportata dalla luce determina l'\textbf{intensità} luminosa.

\subsubsection{Interazione della luce con i materiali}
La luce può interagire con la materia in vari modi:
\begin{itemize}
  \item Una sorgente di luce (luce \textbf{incidente}) illumina la superficie di un oggetto
  \item Una parte della luce \textbf{riflessa} da un punto si distribuisce uniformemente in
    tutte le direzioni (luce \textbf{diffusa})
  \item Una parte della luce viene \textbf{riflessa} da un punto verso una direzione preferita
    (luce \textbf{speculare})
  \item Una parte della luce viene assorbita all'interno del materiale (luce \textbf{trasmessa})
\end{itemize}
Per generare l'immagine bisogna tenere conto della \textbf{quantità di luce} che viene
trasportata fino all'osservatore (camera virtuale) interagendo con i vari materiali.
Il pixel dell'immagine misura dunque la \textbf{radianza} o \textbf{luminanza} delle
superfici visibili dalla camera virtuale.
\begin{itemize}
  \item Fissata la camera virtuale e fissato un pixel si osserva una porzione di superficie
    di un oggetto della scena
  \item Il pixel prende un valore sulla base dello stato fotometrico di questa porzione di
    superficie.
\end{itemize}

\subsection{Equazione del rendering}
Per poter modellare al meglio i fenomeni fotometrici dei materiali viene definita la 
\textbf{Bidirectional Reflectance Distribution Function} (BRDF), che descrive come la luce
viene riflessa da una superficie in funzione della direzione di arrivo e della direzione
di uscita della luce:
\[
  f_r(x, \omega_i, \omega_o)
\] 
Dove:
\begin{itemize}
  \item \( x \) è il punto della superficie
  \item \( \omega_i \) è l'angolo della luce incidente
  \item \( \omega_o \) è l'angolo della luce riflessa
\end{itemize}

\subsubsection{Path tracing}
Il path tracing è un algoritmo di rendering che simula il comportamento della luce
nella scena tracciando il percorso dei raggi di luce. Al posto di calcolare quali raggi
partendo dalla sorgente di luce colpiscono la camera, si parte dalla camera e si
tracciano i raggi che da essa colpiscono la fonte di luce. Questo metodo è più efficiente
perchè si concentra solo sui raggi che effettivamente contribuiscono all'immagine finale.

\subsubsection{Algoritmi di lighting}
Ci sono diversi metodi per calcolare l'illuminazione in una scena 3D:
\begin{itemize}
  \item I metodi \textbf{locali} tengono conto solo dell'effetto delle sorgenti di luce
    dirette sulla superficie, senza considerare le interazioni tra le superfici.
    Esempi: \textbf{Flat shading}, \textbf{Gouraud shading}, \textbf{Phong shading}
  \item I metodi \textbf{globali} considerano tutte le interazioni della luce con le
    superfici della scena, inclusi riflessi, rifrazioni e ombre. Esempi: \textbf{Ray tracing},
    \textbf{Radiosity}, \textbf{Path tracing}
\end{itemize}

\subsubsection{Shading}
Lo shading ha lo scopo di determinare il colore effettivo dei pixel a partire da un modello
di illuminazione dato. Determina come e quando applicare il modello di illuminazione prescelto.
Ci sono tre tipi principali di shading:
\begin{itemize}
  \item \textbf{Flat shading}: A ogni primitiva geometrica (triangolo) è associato uno
    stesso colore uniforme. È il metodo più semplice e veloce, ma meno realistico.
  \item \textbf{Gouraud shading}: Il valore di illuminazione viene calcolato ai vertici
    e viene \textbf{interpolato} per i pixel (fragment) interni
  \item \textbf{Phong shading}: Vengono interpolate le normali ai vertici per ottenere le
    normali di ciascun pixel e il modello viene calcolato per ogni pixel. Produce risultati
    migliori e più realistici, ma è più costoso computazionalmente.
\end{itemize}

\section{Animazione}
L'animazione in 3D è il processo di creazione di modelli (o camera virtuale)
in movimento all'interno di una scena tridimensionale.

\subsection{Produzione del modello}
Il modello viene costruito con una descrizione geometrica del suo aspetto e in base
al tipo di \textbf{cinematica} adottata per l'animazione è possibile dotare il
modello di una struttura scheletrica (\textbf{rigging}) che ne consente il movimento.

\subsubsection{Rigging}
È la fase di preparazione del modello per fare in modo che possa essere animato. Vengono
inserite delle componenti aggiuntive al modello che permettono all'animazione di
diventare \textbf{parametrica} (cioè dipendente da parametri variabili nel tempo).
L'obiettivo è quello di definire i parametri che governano l'animazione in modo da
inserire una sorta di \textbf{semantica} nel \textbf{movimento}.

\vspace{1em}
\noindent
L'animazione è la fase in cui i parametri definiti dal rigging vengono istanziati per
generare i movimenti e le deformazioni desiderate.

\subsubsection{Animazione del volto}
Per animare un volto umano in modo realistico si può ricomporre il movimento del volto a
partire dal controllo dei movimenti elementari della muscolatura. Questo viene fatto
muovendo gruppi di punti di controllo o interpolando tra una forma di riferimento.
Le tecniche più diffuse sono:
\begin{itemize}
  \item \textbf{Blend based}: si costruiscono numerose versioni del modello
    (dette shape di espressioni di base). L'animazione deriva dalla combinazione
    di queste versioni.
  \item \textbf{Rig based}: si inseriscono delle "ossa" all'interno della faccia che
    controllano i movimenti e li propagano sulla superficie del volto.
  \item \textbf{Phisycally based}: vengono simulate le proprietà fisice ed elastiche dei tessuti
    molli della pelle e deu muscoli.
\end{itemize}

\end{document}
