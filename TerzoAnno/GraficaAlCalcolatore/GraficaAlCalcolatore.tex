\documentclass[a4paper]{article}
\usepackage{import}
\usepackage[utf8]{inputenc}
\usepackage[T1]{fontenc}
\usepackage{textcomp}
\usepackage[italian]{babel}
\usepackage{amsmath, amssymb}
\usepackage{booktabs,xltabular}
\usepackage{amsfonts}
\usepackage{amsthm}
\usepackage{cancel}
\usepackage{mdframed}
\usepackage{makecell}
\usepackage{float}
\usepackage{xcolor}
\usepackage{listings}
\usepackage{gensymb}
\usepackage{graphicx}
\usepackage{bodeplot}
\usepackage{tikz}
\usetikzlibrary{shapes, arrows, automata, petri, decorations.markings, decorations.pathreplacing, positioning, calc}
\usepackage{circuitikz}
\usepackage[label=corner]{karnaugh-map}
\graphicspath{{./figures/}}

% Set default font to sans-serif
\renewcommand{\familydefault}{\sfdefault} 
\usepackage{eulervm}

\usepackage{forest}

\usepackage{mathtools}
\DeclarePairedDelimiter\ceil{\lceil}{\rceil}
\DeclarePairedDelimiter\floor{\lfloor}{\rfloor}

% \usepackage{ntheorem}

\usepackage{import}
\usepackage{pdfpages}
\usepackage{transparent}
\usepackage{xcolor}

\usepackage{hyperref}
\hypersetup{
    colorlinks=false,
}

% Code blocks
\definecolor{codegreen}{rgb}{0,0.6,0}
\definecolor{codegray}{rgb}{0.5,0.5,0.5}
\definecolor{codepurple}{rgb}{0.58,0,0.82}
\definecolor{backcolour}{rgb}{0.95,0.95,0.95}

\lstdefinestyle{mystyle}{
	backgroundcolor=\color{backcolour},
	commentstyle=\color{codegreen},
	keywordstyle=\color{magenta},
	numberstyle=\tiny\color{codegray},
	stringstyle=\color{codepurple},
	basicstyle=\ttfamily\footnotesize,
	breakatwhitespace=false,
	breaklines=true,
	captionpos=b,
	keepspaces=true,
	numbers=left,
	numbersep=5pt,
	showspaces=false,
	showstringspaces=false,
	showtabs=false,
	tabsize=2
}

\lstset{style=mystyle}

\usepackage{color}
\usepackage{import}
\usepackage{pdfpages}
\usepackage{transparent}
\usepackage{xcolor}

% Example frame
\theoremstyle{definition}
\newmdtheoremenv[%
	linecolor=gray,leftmargin=0,%
	rightmargin=0,
	innertopmargin=8pt,%
	innerbottommargin=8pt,
	ntheorem]{example}{Esempio}[section]

% Important definition frame
\theoremstyle{definition}
\newmdtheoremenv[%
	linecolor=gray,leftmargin=0,%
	rightmargin=0,
	backgroundcolor=gray!40,%
	innertopmargin=8pt,%
	innerbottommargin=8pt,
	ntheorem]{definition}{Definizione}[section]

% Exercise frame
\theoremstyle{definition}
\newmdtheoremenv[%
	linecolor=gray,leftmargin=0,%
	rightmargin=0,
	innertopmargin=8pt,%
	innerbottommargin=8pt,
	ntheorem]{exercise}{Esercizio}[section]

% Theorem frame
\theoremstyle{definition}
\newmdtheoremenv[%
  linecolor=gray,leftmargin=0,%
  rightmargin=0,
  innertopmargin=8pt,%
  innerbottommargin=8pt,
  ntheorem]{theorem}{Teorema}[section]

\theoremstyle{definition}
\newmdtheoremenv[%
  linecolor=gray,leftmargin=0,%
  rightmargin=0,
  innertopmargin=8pt,%
  innerbottommargin=8pt,
  ntheorem]{define}{Definizione utile}[section]

% figure support
\usepackage{import}
\usepackage{xifthen}
\pdfminorversion=7
\usepackage{pdfpages}
\usepackage{transparent}
\newcommand{\incfig}[1]{%
	\def\svgwidth{\columnwidth}
	\import{./figures/}{#1.pdf_tex}
}

% FSM tikz
\tikzset{
    place/.style={
        circle,
        thick,
        draw=black,
        minimum size=6mm,
    },
        state/.style={
        circle,
        thick,
        draw=blue!75,
        fill=blue!20,
        minimum size=6mm,
    },
}

\usepackage{pgfplots}
\pgfplotsset{compat=1.18}

\pdfsuppresswarningpagegroup=1


\begin{document}

\begin{titlepage}
	\begin{center}
		\vspace*{1cm}

		\Huge
		\textbf{Analisi 1}

		\vspace{0.5cm}
		\LARGE
		UniVR - Dipartimento di Informatica

		\vspace{1.5cm}

		\textbf{Fabio Irimie}

		\vfill


		\vspace{0.8cm}

    Corso di Giacomo Canevari

		1° Semestre 2023/2024

	\end{center}
\end{titlepage}


\tableofcontents
\pagebreak

\section{Introduzione}

\subsection{Interazione con l'utente}
Un modo per far interagire l'utente con il programma è l'interfaccia grafica. L'interazione
può essere ottenuta in vari modi, ad esempio:
\begin{itemize}
    \item Finestre di dialogo
    \item Realtà virtuale
    \item Realtà aumentata
    \item Giochi
\end{itemize}

\subsection{Sintesi e analisi}
La sintesi è il processo di creazione di un'immagine a partire da una descrizione matematica,
mentre l'analisi è il processo di estrazione di informazioni da un concetto già esistente.

\section{Modelling}
La modellazione 3D è un processo di \textbf{descrizione} di un oggetto o una scena al fine
di poterla disegnare. La descrizione avviene attraverso:
\begin{itemize}
  \item \textbf{Struttura}: Viene descritta dalla geometria degli oggetti e dalla loro
    posizione reciproca nello spazio tridimensionale
    \begin{itemize}
      \item Definizione geometrica
      \item Trasformazioni 3D
    \end{itemize}

  \item \textbf{Apparenza}: Descrive come la superficie del modello interagisce con
    la luce (colori, riflessi, ecc...)
    \begin{itemize}
      \item Definizione telecamere virtuali
      \item Definizione sorgenti di luce
      \item Definizione proprietà dei materiali
    \end{itemize}
\end{itemize}

\subsection{Definizione geometrica}
Ci sono varie tecniche di modellazione:
\begin{itemize}
  \item \textbf{Low poly diretta}, ad esempio con Wings3D. È la costruzione manuale di una mesh
    poligonale a bassa risoluzione, partendo anche da primitive già fatte.
  \item \textbf{Subdivision surfaces}, ad esempio con Blender. Si parte da una mesh
    poligonale a bassa risoluzione e si applicano algoritmi di suddivisione per
    ottenere superfici più lisce e dettagliate.
  \item \textbf{Digital sculpting}, ad esempio con ZBrush
  \item \textbf{Modellazione procedurale}, ad esempio con Houdini. Si utilizzano algoritmi
    e regole per generare automaticamente modelli 3D complessi, ad esempio generazione di
    terreni, vegetazione, edifici, ecc...
\end{itemize}

\subsubsection{Mesh}
Gli oggetti tridimensionali vengono codificati come una maglia (mesh) di triangoli.
I triangoli vengono utilizzati perchè sono il poligono più semplice che può essere utilizzato
per approssimare una qualsiasi superficie. Una mesh è composta da:
\begin{itemize}
  \item \textbf{Vertici}: Punti nello spazio 3D
  \item \textbf{Facce}: Insiemi di vertici che formano triangoli
\end{itemize}

\begin{definition}[Definizione matematica di mesh]
  Una mesh di triangoli è una discretizzazione lineare a tratti di una superficie
  continua (un "2-manifold") immersa in \( \mathbb{R}^3 \), cioè un oggetto bidimensionale
  che si trova in uno spazio tridimensionale. Le componenti sono:
  \begin{itemize}
    \item \textbf{Geometria}: i vertici, ciascuno con coordinate \( (x, y, z) \in \mathbb{R}^3 \)
    \item \textbf{Topologia}: come sono connessi tra loro i vertici, nel caso di una tri-mesh
      ogni faccia è definita da un insieme di tre vertici
  \end{itemize}
\end{definition}

\subsection{Trasformazioni 3D}
Per posizionare e orientare gli oggetti nello spazio 3D, si utilizzano trasformazioni
geometriche, che possono essere rappresentate tramite matrici. Le principali trasformazioni sono:
\begin{itemize}
  \item \textbf{Traslazione}: Spostamento di un oggetto da una posizione a un'altra
  \item \textbf{Rotazione}: Rotazione di un oggetto attorno a un asse specifico
  \item \textbf{Scalatura}: Modifica delle dimensioni di un oggetto lungo gli assi
    X, Y e Z
\end{itemize}

\subsection{Telecamere virtuali}
Per visualizzare una scena 3D su uno schermo 2D, è necessario utilizzare
una telecamera virtuale che definisce il punto di vista da cui viene osservata la scena.
Il problema è che nel passaggio dal 2D al 3D c'è perdita di informazione.

Per definire una telecamera virtuale, sono necessari:
\begin{itemize}
  \item \textbf{View point}: da dove si osserva
  \item \textbf{Look at point}: dove si guarda
  \item \textbf{View direction}: orientamento della telecamera
  \item \textbf{Regole di proiezione}:
    \begin{itemize}
      \item Ortografica: mantiene le proporzioni reali degli oggetti
      \item Prospettica: simula la visione umana, con oggetti
        più lontani che appaiono più piccoli
    \end{itemize}
\end{itemize}

\subsubsection{Proiezione}
Il mondo non è infinito, quindi bisogna definire il \textbf{cono di vista} (frustum),
che delimita la porzione di scena visibile dalla telecamera. Il frustum è definito
dal parallelepipedo delimitato da due piani:
\begin{itemize}
  \item \textbf{Near plane}: piano più vicino alla telecamera
  \item \textbf{Far plane}: piano più lontano dalla telecamera
\end{itemize}
Gli oggetti al di fuori del frustum non vengono proiettati (fase di clipping). La proiezione
avviene su un piano di vista (view plane), che rappresenta lo schermo 2D.

\subsection{Illuminazione}
Tramite l'illuminazione si riesce a distinguere la forma degli oggetti tridimensionali.
La modellazione delle luci della scena si occupa del loro posizionamento e del tipo di luce
utilizzata. I tipi di luce più comuni sono:
\begin{itemize}
  \item \textbf{Directional light}: luce proveniente da una direzione specifica, simula
    la luce solare
  \item \textbf{Point light}: luce che si propaga in tutte le direzioni da un punto
    specifico, simula una lampadina
  \item \textbf{Spotlight}: luce che si propaga in un cono da un punto specifico,
    simula un faro
  \item \textbf{Ambient light}: luce diffusa che illumina uniformemente tutta la scena,
    senza una direzione specifica
\end{itemize}

\subsection{Proprietà dei materiali}
Il materiale di cui è fatta la superficie di un oggetto condiziona il suo aspetto nel
momento in cui viene colpito dalla luce. Le proprietà principali dei materiali sono:
\begin{itemize}
  \item \textbf{Colore}
  \item \textbf{Riflettività}
  \item \textbf{Rugosità}
\end{itemize}

\end{document}
