\documentclass[a4paper]{article}
\usepackage{import}
\usepackage[utf8]{inputenc}
\usepackage[T1]{fontenc}
\usepackage{textcomp}
\usepackage[italian]{babel}
\usepackage{amsmath, amssymb}
\usepackage{booktabs,xltabular}
\usepackage{amsfonts}
\usepackage{subcaption}
\usepackage{amsthm}
\usepackage{cancel}
\usepackage{mdframed}
\usepackage{makecell}
\usepackage{float}
\usepackage{xcolor}
\usepackage{listings}
\usepackage{gensymb}
\usepackage{graphicx}
\usepackage{bodeplot}
\usepackage{physics}
\usepackage{tikz}
\usetikzlibrary{shapes, arrows, automata, petri, decorations.markings, decorations.pathreplacing, positioning, calc, quotes}
\usepackage{circuitikz}
\usepackage[label=corner]{karnaugh-map}
\graphicspath{{./figures/}}

% Set default font to sans-serif
\renewcommand{\familydefault}{\sfdefault} 
\usepackage{eulervm}

\usepackage{forest}

\usepackage{mathtools}
\DeclarePairedDelimiter\ceil{\lceil}{\rceil}
\DeclarePairedDelimiter\floor{\lfloor}{\rfloor}

% \usepackage{ntheorem}

\usepackage{import}
\usepackage{pdfpages}
\usepackage{transparent}
\usepackage{xcolor}

\usepackage{hyperref}
\hypersetup{
    colorlinks=false,
}

% Code blocks
\definecolor{codegreen}{rgb}{0,0.6,0}
\definecolor{codegray}{rgb}{0.5,0.5,0.5}
\definecolor{codepurple}{rgb}{0.58,0,0.82}
\definecolor{backcolour}{rgb}{0.95,0.95,0.95}

\lstdefinestyle{mystyle}{
	backgroundcolor=\color{backcolour},
	commentstyle=\color{codegreen},
	keywordstyle=\color{magenta},
	numberstyle=\tiny\color{codegray},
	stringstyle=\color{codepurple},
	basicstyle=\ttfamily\footnotesize,
	breakatwhitespace=false,
	breaklines=true,
	captionpos=b,
	keepspaces=true,
	numbers=left,
	numbersep=5pt,
	showspaces=false,
	showstringspaces=false,
	showtabs=false,
	tabsize=2
}

\lstset{style=mystyle}

\usepackage{color}
\usepackage{import}
\usepackage{pdfpages}
\usepackage{transparent}
\usepackage{xcolor}

% Example frame
\theoremstyle{definition}
\newmdtheoremenv[%
	linecolor=gray,leftmargin=0,%
	rightmargin=0,
	innertopmargin=8pt,%
	innerbottommargin=8pt,
	ntheorem]{example}{Esempio}[section]

% Important definition frame
\theoremstyle{definition}
\newmdtheoremenv[%
	linecolor=gray,leftmargin=0,%
	rightmargin=0,
	backgroundcolor=gray!40,%
	innertopmargin=8pt,%
	innerbottommargin=8pt,
	ntheorem]{definition}{Definizione}[section]

% Exercise frame
\theoremstyle{definition}
\newmdtheoremenv[%
	linecolor=gray,leftmargin=0,%
	rightmargin=0,
	innertopmargin=8pt,%
	innerbottommargin=8pt,
	ntheorem]{exercise}{Esercizio}[section]

% Theorem frame
\theoremstyle{definition}
\newmdtheoremenv[%
  linecolor=gray,leftmargin=0,%
  rightmargin=0,
  innertopmargin=8pt,%
  innerbottommargin=8pt,
  ntheorem]{theorem}{Teorema}[section]

\theoremstyle{definition}
\newmdtheoremenv[%
  linecolor=white,leftmargin=0,%
  rightmargin=0,
  innertopmargin=8pt,%
  innerbottommargin=8pt,
  ntheorem]{define}{Definizione utile}[section]

% figure support
\usepackage{import}
\usepackage{xifthen}
\pdfminorversion=7
\usepackage{pdfpages}
\usepackage{transparent}
\newcommand{\incfig}[1]{%
	\def\svgwidth{\columnwidth}
	\import{./figures/}{#1.pdf_tex}
}

% FSM tikz
\tikzset{
    place/.style={
        circle,
        thick,
        draw=black,
        minimum size=6mm,
    },
        state/.style={
        circle,
        thick,
        draw=black,
        fill=white,
        minimum size=6mm,
    },
}

\pdfsuppresswarningpagegroup=1

\usepackage{pgfplots}
\pgfplotsset{compat=1.18,width=10cm}

% Save plots as pdf and reuse them without compiling every time
\usetikzlibrary{external}
\tikzexternalize[prefix=figures/tikz/, optimize=false]


\begin{document}

\begin{titlepage}
	\begin{center}
		\vspace*{1cm}

		\Huge
		\textbf{Probabilità e Statistica\\Esercizi}

		\vspace{0.5cm}
		\LARGE
		UniVR - Dipartimento di Informatica

		\vspace{1.5cm}

		\textbf{Fabio Irimie}

		\vfill


		\vspace{0.8cm}


		2° Semestre 2023/2024

	\end{center}
\end{titlepage}


\tableofcontents
\pagebreak

\section{Introduzione}
Le basi di dati sono raccolte di dati strutturati, organizzati in modo tale da permettere
un facile accesso. Questi dati sono persistenti, ovvero rimangono memorizzati anche dopo
la chiusura del programma che li ha creati.

\section{Sistema informativo}
Un sistema informativo è l'insieme delle attività umane e dei dispositivi di memorizzazione
ed elaborazione che organizza e gestisce l'informazione di interesse per un organizzazione
di dimensioni qualsiasi. Non contiene necessariamente dati memorizzati in un computer.

Un sistema informativo è composto da:
\begin{itemize}
  \item \textbf{Dato}: è l'elemento di conoscenza di base costituito da simboli che devono
    essere elaborati
  \item \textbf{Informazione}: è l'interpretazione dei dati che permette di ottenere una
    conoscenza più o meno esatta di fatti e situazioni
\end{itemize}


\subsection{Base di dati}
\begin{definition}
  Una \textbf{base di dati} è una \textbf{collezione di dati persistenei} utilizzati per
  rappresentare \textbf{con tecnologia informatica} le informazioni di interesse per un
  \textbf{sistema informativo}
\end{definition}
La soluzione convenzionale per la gestione dei dati è l'uso di file, ma questa presenta
alcuni problemi:
\begin{itemize}
  \item Scarsa efficienza nell'accesso ai dati (accesso sequenziale)
  \item Ridondanza nei dati
  \item Inconsistenza nei dati (aggiornamenti parziali)
  \item Progettazione dei dati replicata per ogni applicazione
\end{itemize}
Per risolvere questi problemi si è creato un livello di astrazione maggiore tra le
applicazioni e il filesystem, ovvero il \textbf{Data Base Management System (DBMS)}.

\begin{definition}
  Un \textbf{DBMS} è un sistema che gestisce su \textbf{memoria secondaria} collezioni
  di dati (chiamate "basi di dati"). Le caratteristiche principali sono:
  \begin{itemize}
    \item Grandi
    \item Condivise, cioè accessibili da più utenti
    \item Persistenti
  \end{itemize}
  Un DBMS assicura:
  \begin{itemize}
    \item Affidabilità, cioè nessuna perdita di dati
    \item Privatezza
    \item Accesso efficiente
  \end{itemize}
\end{definition}

\subsubsection{Modello dei dati}
Un \textbf{modello dei dati} è un insieme di strutture che permettono di descrivere
una base di dati. Per accedere a questi dati si usano delle \textbf{interrogazioni},
cioè delle richieste, in un linguaggio dichiarativo specifico, che permettono di ottenere
i dati desiderati.

Ci sono diversi linguaggi per interagire con un DBMS:
\begin{itemize}
  \item Linguaggio per la definizione dei dati (DDL), consente di definire la struttura
    della base di dati
  \item Linguaggio per l'interrogazione e aggiornamento dei dati (DML), consente
    di interrogare e aggiornare i dati
    \begin{itemize}
      \item Linguaggio di interrogazione: estrae informazioni da una base di dati,
        ad esempio SQL, algèbre relazionale, calcolo relazionale
      \item Linguaggio di manipolazione: popola la base di dati, modifica il suo contenuto
        con aggiunge, cancellazioni e variazioni sui dati, ad esempio SQL
    \end{itemize}
\end{itemize}

Il modello di dati è un insisme di \textbf{costrutti} forniti dal DBMS per descrivere la
struttura e le proprietà dell'informazione contenute in una base di dati.

Ci sono diversi tipi di modelli di dati:
\begin{itemize}
  \item \textbf{Modelli di dati del passato}:
    \begin{itemize}
      \item Modello reticolare
      \item modello gerarchico
    \end{itemize}

  \item \textbf{Modelli di dati attuali}:
    \begin{itemize}
      \item Modello relazionale
      \item Modello ad oggetti
      \item Modello a oggetti-relazionale
      \item Modello basato su documenti (JSON)
      \item Modelli NoSQL
    \end{itemize}
\end{itemize}

I modelli vengono utilizzati per creare:
\begin{itemize}
  \item \textbf{Schema di una base di dati}: è la descrizione della struttura e delle
    proprietà di una specifica base di dati fatta utilizzando i costrutti del modello dei
    dati (lo schema di una base di dati è invariante nel tempo)
  \item \textbf{Istanza di una base di dati}: è costituita dai \textbf{valori effettivi}
    che in un certo istante popolano le strutture dati (l'istanza di una base di dati
    varia nel tempo)
\end{itemize}

Lo schema di una base di dati è diviso in tre livelli:
\begin{itemize}
  \item \textbf{Schema esterno}: è la visione dell'utente della base di dati, cioè la
    parte di base di dati che interessa a un particolare utente o gruppo di utenti
  \item \textbf{Schema logico}: è la visione globale della base di dati, cioè la
    struttura logica della base di dati che descrive tutti i dati e le relazioni tra
    essi
  \item \textbf{Schema interno}: è la rappresentazione fisica della base di dati, cioè
    il modo in cui i dati sono effettivamente memorizzati nella memoria secondaria
\end{itemize}
Le proprietà dello schema sono:
\begin{itemize}
  \item \textbf{Indipendenza fisica}: lo schema logico della base di dati è completamente
    indipendente dallo schema interno
  \item \textbf{Indipendenza logica}: gli schemi esterni della base di dati sono
    indipendenti dallo schema logico
\end{itemize}


\end{document}
