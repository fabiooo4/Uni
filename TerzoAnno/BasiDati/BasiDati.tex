\documentclass[a4paper]{article}
\usepackage{import}
\usepackage[utf8]{inputenc}
\usepackage[T1]{fontenc}
\usepackage{textcomp}
\usepackage[italian]{babel}
\usepackage{amsmath, amssymb}
\usepackage{booktabs,xltabular}
\usepackage{amsfonts}
\usepackage{amsthm}
\usepackage{cancel}
\usepackage{mdframed}
\usepackage{makecell}
\usepackage{float}
\usepackage{xcolor}
\usepackage{listings}
\usepackage{gensymb}
\usepackage{graphicx}
\usepackage{bodeplot}
\usepackage{tikz}
\usetikzlibrary{shapes, arrows, automata, petri, decorations.markings, decorations.pathreplacing, positioning, calc}
\usepackage{circuitikz}
\usepackage[label=corner]{karnaugh-map}
\graphicspath{{./figures/}}

% Set default font to sans-serif
\renewcommand{\familydefault}{\sfdefault} 
\usepackage{eulervm}

\usepackage{forest}

\usepackage{mathtools}
\DeclarePairedDelimiter\ceil{\lceil}{\rceil}
\DeclarePairedDelimiter\floor{\lfloor}{\rfloor}

% \usepackage{ntheorem}

\usepackage{import}
\usepackage{pdfpages}
\usepackage{transparent}
\usepackage{xcolor}

\usepackage{hyperref}
\hypersetup{
    colorlinks=false,
}

% Code blocks
\definecolor{codegreen}{rgb}{0,0.6,0}
\definecolor{codegray}{rgb}{0.5,0.5,0.5}
\definecolor{codepurple}{rgb}{0.58,0,0.82}
\definecolor{backcolour}{rgb}{0.95,0.95,0.95}

\lstdefinestyle{mystyle}{
	backgroundcolor=\color{backcolour},
	commentstyle=\color{codegreen},
	keywordstyle=\color{magenta},
	numberstyle=\tiny\color{codegray},
	stringstyle=\color{codepurple},
	basicstyle=\ttfamily\footnotesize,
	breakatwhitespace=false,
	breaklines=true,
	captionpos=b,
	keepspaces=true,
	numbers=left,
	numbersep=5pt,
	showspaces=false,
	showstringspaces=false,
	showtabs=false,
	tabsize=2
}

\lstset{style=mystyle}

\usepackage{color}
\usepackage{import}
\usepackage{pdfpages}
\usepackage{transparent}
\usepackage{xcolor}

% Example frame
\theoremstyle{definition}
\newmdtheoremenv[%
	linecolor=gray,leftmargin=0,%
	rightmargin=0,
	innertopmargin=8pt,%
	innerbottommargin=8pt,
	ntheorem]{example}{Esempio}[section]

% Important definition frame
\theoremstyle{definition}
\newmdtheoremenv[%
	linecolor=gray,leftmargin=0,%
	rightmargin=0,
	backgroundcolor=gray!40,%
	innertopmargin=8pt,%
	innerbottommargin=8pt,
	ntheorem]{definition}{Definizione}[section]

% Exercise frame
\theoremstyle{definition}
\newmdtheoremenv[%
	linecolor=gray,leftmargin=0,%
	rightmargin=0,
	innertopmargin=8pt,%
	innerbottommargin=8pt,
	ntheorem]{exercise}{Esercizio}[section]

% Theorem frame
\theoremstyle{definition}
\newmdtheoremenv[%
  linecolor=gray,leftmargin=0,%
  rightmargin=0,
  innertopmargin=8pt,%
  innerbottommargin=8pt,
  ntheorem]{theorem}{Teorema}[section]

\theoremstyle{definition}
\newmdtheoremenv[%
  linecolor=gray,leftmargin=0,%
  rightmargin=0,
  innertopmargin=8pt,%
  innerbottommargin=8pt,
  ntheorem]{define}{Definizione utile}[section]

% figure support
\usepackage{import}
\usepackage{xifthen}
\pdfminorversion=7
\usepackage{pdfpages}
\usepackage{transparent}
\newcommand{\incfig}[1]{%
	\def\svgwidth{\columnwidth}
	\import{./figures/}{#1.pdf_tex}
}

% FSM tikz
\tikzset{
    place/.style={
        circle,
        thick,
        draw=black,
        minimum size=6mm,
    },
        state/.style={
        circle,
        thick,
        draw=blue!75,
        fill=blue!20,
        minimum size=6mm,
    },
}

\usepackage{pgfplots}
\pgfplotsset{compat=1.18}

\pdfsuppresswarningpagegroup=1


\begin{document}

\begin{titlepage}
	\begin{center}
		\vspace*{1cm}

		\Huge
		\textbf{Analisi 1}

		\vspace{0.5cm}
		\LARGE
		UniVR - Dipartimento di Informatica

		\vspace{1.5cm}

		\textbf{Fabio Irimie}

		\vfill


		\vspace{0.8cm}

    Corso di Giacomo Canevari

		1° Semestre 2023/2024

	\end{center}
\end{titlepage}


\tableofcontents
\pagebreak

\section{Introduzione}
Le basi di dati sono raccolte di dati strutturati, organizzati in modo tale da permettere
un facile accesso. Questi dati sono persistenti, ovvero rimangono memorizzati anche dopo
la chiusura del programma che li ha creati.

\section{Sistema informativo}
Un sistema informativo è l'insieme delle attività umane e dei dispositivi di memorizzazione
ed elaborazione che organizza e gestisce l'informazione di interesse per un organizzazione
di dimensioni qualsiasi. Non contiene necessariamente dati memorizzati in un computer.

Un sistema informativo è composto da:
\begin{itemize}
  \item \textbf{Dato}: è l'elemento di conoscenza di base costituito da simboli che devono
    essere elaborati
  \item \textbf{Informazione}: è l'interpretazione dei dati che permette di ottenere una
    conoscenza più o meno esatta di fatti e situazioni
\end{itemize}


\subsection{Base di dati}
\begin{definition}
  Una \textbf{base di dati} è una \textbf{collezione di dati persistenei} utilizzati per
  rappresentare \textbf{con tecnologia informatica} le informazioni di interesse per un
  \textbf{sistema informativo}
\end{definition}
La soluzione convenzionale per la gestione dei dati è l'uso di file, ma questa presenta
alcuni problemi:
\begin{itemize}
  \item Scarsa efficienza nell'accesso ai dati (accesso sequenziale)
  \item Ridondanza nei dati
  \item Inconsistenza nei dati (aggiornamenti parziali)
  \item Progettazione dei dati replicata per ogni applicazione
\end{itemize}
Per risolvere questi problemi si è creato un livello di astrazione maggiore tra le
applicazioni e il filesystem, ovvero il \textbf{Data Base Management System (DBMS)}.

\begin{definition}
  Un \textbf{DBMS} è un sistema che gestisce su \textbf{memoria secondaria} collezioni
  di dati (chiamate "basi di dati"). Le caratteristiche principali sono:
  \begin{itemize}
    \item Grandi
    \item Condivise, cioè accessibili da più utenti
    \item Persistenti
  \end{itemize}
  Un DBMS assicura:
  \begin{itemize}
    \item Affidabilità, cioè nessuna perdita di dati
    \item Privatezza
    \item Accesso efficiente
  \end{itemize}
\end{definition}

\subsubsection{Modello dei dati}
Un \textbf{modello dei dati} è un insieme di strutture che permettono di descrivere
una base di dati. Per accedere a questi dati si usano delle \textbf{interrogazioni},
cioè delle richieste, in un linguaggio dichiarativo specifico, che permettono di ottenere
i dati desiderati.

Ci sono diversi linguaggi per interagire con un DBMS:
\begin{itemize}
  \item Linguaggio per la definizione dei dati (DDL), consente di definire la struttura
    della base di dati
  \item Linguaggio per l'interrogazione e aggiornamento dei dati (DML), consente
    di interrogare e aggiornare i dati
    \begin{itemize}
      \item Linguaggio di interrogazione: estrae informazioni da una base di dati,
        ad esempio SQL, algèbre relazionale, calcolo relazionale
      \item Linguaggio di manipolazione: popola la base di dati, modifica il suo contenuto
        con aggiunge, cancellazioni e variazioni sui dati, ad esempio SQL
    \end{itemize}
\end{itemize}

Il modello di dati è un insisme di \textbf{costrutti} forniti dal DBMS per descrivere la
struttura e le proprietà dell'informazione contenute in una base di dati.

Ci sono diversi tipi di modelli di dati:
\begin{itemize}
  \item \textbf{Modelli di dati del passato}:
    \begin{itemize}
      \item Modello reticolare
      \item modello gerarchico
    \end{itemize}

  \item \textbf{Modelli di dati attuali}:
    \begin{itemize}
      \item Modello relazionale
      \item Modello ad oggetti
      \item Modello a oggetti-relazionale
      \item Modello basato su documenti (JSON)
      \item Modelli NoSQL
    \end{itemize}
\end{itemize}

I modelli vengono utilizzati per creare:
\begin{itemize}
  \item \textbf{Schema di una base di dati}: è la descrizione della struttura e delle
    proprietà di una specifica base di dati fatta utilizzando i costrutti del modello dei
    dati (lo schema di una base di dati è invariante nel tempo)
  \item \textbf{Istanza di una base di dati}: è costituita dai \textbf{valori effettivi}
    che in un certo istante popolano le strutture dati (l'istanza di una base di dati
    varia nel tempo)
\end{itemize}

Lo schema di una base di dati è diviso in tre livelli:
\begin{itemize}
  \item \textbf{Schema esterno}: è la visione dell'utente della base di dati, cioè la
    parte di base di dati che interessa a un particolare utente o gruppo di utenti
  \item \textbf{Schema logico}: è la visione globale della base di dati, cioè la
    struttura logica della base di dati che descrive tutti i dati e le relazioni tra
    essi
  \item \textbf{Schema interno}: è la rappresentazione fisica della base di dati, cioè
    il modo in cui i dati sono effettivamente memorizzati nella memoria secondaria
\end{itemize}
Le proprietà dello schema sono:
\begin{itemize}
  \item \textbf{Indipendenza fisica}: lo schema logico della base di dati è completamente
    indipendente dallo schema interno
  \item \textbf{Indipendenza logica}: gli schemi esterni della base di dati sono
    indipendenti dallo schema logico
\end{itemize}

\section{Progettazione di una base di dati}
Il ciclo di vita di un processo di automazione di un sistema informativo è diviso in diverse
fasi:
\begin{itemize}
  \item \textbf{Studio di fattibilità}: si valuta se l'automazione del sistema
    informativo è possibile e conveniente
  \item \textbf{Raccolta e analisi dei requisiti}: si individuano proprietà e funzionalità
    del sistema (dati e applicazioni) producendo una descrizione completa ma informale
  \item \textbf{Progettazione}: si produce una descrizione formale del sistema
    informativo
\end{itemize}
La progettazione si divide in due parti principali che vanno di pari passo:
\begin{itemize}
  \item \textbf{Progettazione dei dati}: si produce una descrizione formale dei dati
    (schema). Una volta progettati i dati vengono implementati in un DBMS
  \item \textbf{Progettazione delle applicazioni}: si produce una descrizione formale
    delle applicazioni (specifica)
\end{itemize}
Una volta implementati i dati e le applicazioni si passa alla fase di \textbf{validazione
e collaudo}

\subsection{Progettazione dei dati}
Una metodologia di progettazione dei dati è costituita da:
\begin{itemize}
  \item \textbf{Decomposizione}: dividere in passi le attività di progetto
  \item \textbf{Strategie}: individuare un insieme di strategie e criteri di scelta da
    seguire
  \item \textbf{Modelli di riferimento}: utilizzare modelli di dati e tecniche di
    progettazione consolidate
\end{itemize}
Una buona metodologia deve essere:
\begin{itemize}
  \item Generale
  \item Facile da usare
  \item Deve produrre un risultato di qualità
\end{itemize}

\subsection{Requisiti}
\subsubsection{Progettazione concettuale}
La progettazione concettuale è la prima fase della progettazione dei dati. Lo scopo è
quello di produrre una descrizione formale dei dati (schema concettuale). Lo schema
deve essere \textbf{indipendente dall'implementazione}.

\vspace{1em}
\noindent
Non è solo un progetto intermedio, ma costituisce anche una porzione del risultato finale
perchè rappresenta una descrizione di \textbf{alto livello} del contenuto della base di
dati, comprensibile anche per utenti poco esperti.

\subsubsection{Progettazione logica}
La progettazione logica è la seconda fase della progettazione dei dati. Lo scopo è
quello di tradurre lo schema concettuale in uno schema logico in modo da poterlo
utilizzare su un sistema specifico. Lo schema logico infatti è dipendente dalle
tecnologie utilizzate. Bisogna tenere anche in considerazione le operazioni più
frequenti che le applicazioni effettueranno sulla base di dati.

\subsubsection{Progettazione fisica}
La progettazione fisica è la terza fase della progettazione dei dati. L'obiettivo è quello
di ottimizzare l'accesso ai dati completando lo schema logico con i parametri relativi
alla memorizzazione fisica dei dati e con gli opportuni metodi d'accesso (\textbf{indici}).

\subsection{Strategie di progettazione}
Lo sviluppo do uno schema concettuale può essere visto come un processo di ingegnerizzazione
ed è quindi possibile applicare anche a tale processo le strategie classiche:
\begin{itemize}
  \item \textbf{Top-Down}: consiste nel considerare le specifiche \textbf{globalmente}
    e produrre uno schema iniziale completo ma con \textbf{pochi concetti} molto astratti.
    Si va poi a \textbf{raffinare} i concetti astratti fino ad arrivare allo schema
    concettuale completo in ogni dettaglio

  \item \textbf{Bottom-Up}: consiste nel \textbf{decomporre} le specifiche iniziali in
    parti elementari e produrre uno schema iniziale dettagliato

  \item \textbf{Inside-Out}: consiste nell'individuare nelle specifiche alcuni concetti
    importanti (detti concetti guida) e partendo da quelli generare gli schemi
    per i relativi concetti. Si procede fondendo gli schemi precedenti per generare
    lo schema finale.
\end{itemize}

\subsubsection{Analisi di qualità dello shema}
L'analisi della qualità dello schema concettuale prodotto può essere suddivisa in diverse
fasi:
\begin{itemize}
  \item \textbf{Verifica della correttezza}: Uno schema concettuale è corretto se utilizza
    correttamente i costrutti del modello concettuale adottato (nel nostro caso E-R).
    I possibili errori sono:
    \begin{itemize}
      \item Errori sintattici: si verificano quando i costrutti vengono usati senza seguire
        le regole sintattiche
      \item Errori semantici: si verificano quando i costrutti vengono usati senza seguire
        le regole semantiche, cioè la loro definizione
    \end{itemize}

  \item \textbf{Verifica della completezza}: Uno schema concettuale è completo se
    rappresenta tutti i requisiti espressi nelle specifiche.

  \item \textbf{Verifica di minimalità}: Uno schema è minimale quando tutti i concetti
    descritti nei requisiti sono rappresentati nello schema \textbf{una volta sola}.
    Uno schema non minimale contiene delle \textbf{ridondanze}. I casi di probabile
    ridondanza sono:
    \begin{itemize}
      \item Ereditarietà nelle relazioni: si manifesta quando due 
        entità padri sono in relazione e le loro entità figlie sono anch'esse in relazione.
        Per far si che non ci sia ridondanza in questo caso bisogna verificare che le 
        due relazioni non rappresentino lo stesso concetto

      \item Cicli di relazioni: si manifesta quando esistono dei cicli tra le relazioni
        di più entità. Non è detto che ci sia sicuramente ridondanza, ma per evitarla
        bisogna verificare se la relazione \( R_i \) si possa ottenere dalla composizione
        delle altre relazioni. Se è così allora \( R_i \) va eliminata in quanto ridondante.
    \end{itemize}

  \item \textbf{Verifica di leggibilità}: Uno schema concettuale è leggiible quando
    rappresenta tutti i requisiti in modo naturale e facilmente compresnsibile. Alcuni
    consigli per migliorare la leggibilità sono:
    \begin{itemize}
      \item Curare la scelta dei nomi delle componenti dello schema
      \item Disegnare bene lo schema
    \end{itemize}
    Si può migliorare la leggibilità anche semplificando lo schema. Le relazioni
    ternarie non sono facili da leggere e interpretare, quindi solitamente la loro
    eliminazione può comportare una migliore leggibilità dello schema.
    Le operazioni possibili sono:
    \begin{itemize}
      \item Trasformare una relazione ternaria in un'entità (sempre applicabile)
      \item Riduzione di una relazione ternariaa a due relazioni binarie\\
        (\textbf{precondizione}: una delle entità partecipa alla relazione ternaria
        con cardinalità (1,1) (spiegata nel capitolo \ref{Cardinalità})) 
      \item Trasformare una relazione sovrapposta in una generalizzazione esclusiva
        esplicitando l'entità che rappresenta la sovrapposizione
    \end{itemize}
\end{itemize}

\section{Progettazione Concettuale (Modello Entità - Relazione)}
È un modello, formale e non ambiguo, utilizzato per la progettazione concettuale di una
base di dati. Fornisce strumenti formali (costrutti), con sintassi grafica, per specificare la
struttura e le proprietà dei dati da rappresentare indipendentemente dalla tecnologia.

\vspace{1em}
\noindent
Ogni costrutto viene definito specificando:
\begin{itemize}
  \item Il suo significato (o semantica)
  \item La sua sintassi grafica
  \item La rappresentazione delle sue istanze (o occorrenze)
\end{itemize}

\vspace{1em}
\noindent
Progettare indipendentemente dalle tecnologie significa:
\begin{itemize}
  \item \textbf{Non considerare} eventuali ottimizzazioni
  \item \textbf{Considerare} tutti i requisiti senza semplificazioni o convenzioni
  \item \textbf{Considerare} sempre i processi di generazione e modifica dei dati
    per verificare che ogni situazione sia rappresentabile da un'istanza "pulita"
    della base di dati
\end{itemize}

\subsection{Entità}
Un'entità \( E \) rappresenta una \textbf{classe di oggetti} che hanno le seguenti
caratteristiche:
\begin{itemize}
  \item \textbf{Proprietà comuni}
  \item \textbf{Eistenza autonoma} rispetto ad altre classi di oggetti
  \item \textbf{Identificazione univoca}, cioè esiste una chiara corrispondenza tra
    gli oggetti istanze di entità e concetti istanziati nel sistema informativo
\end{itemize}
Un entità si rappresenta con un rettangolo che contiene il nome dell'entità:
\begin{figure}[H]
  \centering
  \begin{tikzpicture}
    \node[draw, rectangle, minimum width=3cm, minimum height=1cm ]
      (E) {Entità \( E \)};
  \end{tikzpicture}
  \caption{Rappresentazione grafica di un'entità}
\end{figure}

\subsubsection{Istanza}
Un'istanza dell'entità \( E \) è un \textbf{oggetto} appartenente alla classe rappresentata
da \( E \). Si indica con \( I(E) \) l'insieme delle istanze di \( E \) che esistono
nella base di dati in un certo istante e alla creazione della base di dati è vuota:
\( I(E) = \emptyset \).

\begin{example}
  Rappresentiamo con il costrutto entità il concetto di \textbf{persona}. Bisogna gestire
  nella base di dati le informazioni che descrivono un gruppo di persone.
  \begin{figure}[H]
    \centering
    \begin{tikzpicture}
      \node[draw, rectangle, minimum width=3cm, minimum height=1cm ]
        (E) {Persona};
    \end{tikzpicture}
    \caption{Rappresentazione grafica dell'entità Persona}
  \end{figure}
  L'insieme delle istanze dell'entità Persona è il seguente:
  \[
    I(\text{Persona}) = \{p_1, p_2, p_3, \ldots\}
  \]
\end{example}

\subsection{Relazione}
Una relazione \( R \) rappresenta un \textbf{legame logico} tra \textbf{due o più} entità.
Può esserci anche una relazione all'entità stessa (relazione ricorsiva).

\vspace{1em}
\noindent
Una relazione si rappresenta nello schema con un rombo a cui si collegano attraverso delle
linee le entità coinvolte nella relazione. Il nome della relazione viene scritto a fianco
al rombo:
\begin{figure}[H]
  \centering
  \begin{tikzpicture}
    \node[draw, rectangle, minimum width=2cm, minimum height=1cm ]
      (E1) {Entità \( E \)};
    \node[draw, diamond, minimum width=0.7cm, minimum height=0.7cm, right=1cm of E1]
      (R) {};
    \node[below=0.3cm] at (R) {Relazione \( R \)};
    \node[draw, rectangle, minimum width=2cm, minimum height=1cm, right=1cm of R]
      (E2) {Entità \( F \)};
    \draw (E1) -- (R);
    \draw (E2) -- (R);
  \end{tikzpicture}
  \caption{Rappresentazione grafica di una relazione tra due entità}
\end{figure}

\subsubsection{Istanza}
Data una relazione \( R \) tra \( n \) entità \( E_1, \ldots, E_n \) un'istanza della
relazione \( R \) è una \textbf{ennupla di istanze di entità}:
\[
  (e_1, \ldots, e_n) \; \text{dove } e_i \in I(E_i) \text{ per } 1 \le i \le n
\] 
La popolazione di \( R \) rappresenta l'insieme delle coppie di istanze delle entità
\( E \) e \( F \) che sono in relazione in un certo istante:
\[
  I(R) = \left\{ (e_i, f_j) \;\left|\; e_i \in I(E), f_j \in I(F) \right.\right\} 
\]
\begin{example}
  Supponiamo che nello schema ci siano le entità \textbf{Persona} e \textbf{Comune},
  bisogna gestire la \textbf{Residenza} delle persone nei comuni italiani.

  \vspace{2em}
  \noindent
  \begin{figure}[H]
    \centering
    \begin{tikzpicture}
      \node[draw, rectangle, minimum width=2cm, minimum height=1cm ]
        (E1) {Persona};
      \node[draw, diamond, minimum width=0.7cm, minimum height=0.7cm, right=1cm of E1]
        (R) {};
      \node[below=0.3cm] at (R) {Residenza};
      \node[draw, rectangle, minimum width=2cm, minimum height=1cm, right=1cm of R]
        (E2) {Comune};
      \draw (E1) -- (R);
      \draw (E2) -- (R);
    \end{tikzpicture}
    \caption{Rappresentazione grafica della relazione Residenza}
  \end{figure}
  \noindent
  Ciò implica che per esistere un'istanza di residenza devono esistere un'istanza
  di persona e un'istanza di comune.
\end{example}
Data una relazione \( R \) tra \( n \) entità \( \{E_1, E_2, \ldots, E_n\} \)
vale \textbf{sempre} la seguente proprietà sull'insieme delle istanze \( I(R) \):
\[
  I(R) \subseteq I(E_1) \times I(E_2) \times \ldots \times I(E_n)
\] 
La conseguenza di questa proprietà è che non è possibilie rappresentare la stessa ennupla
più volte.

\subsubsection{Relazione ricorsiva}
È una relazione binaria sulla stessa entità:
\begin{figure}[H]
  \centering
  \begin{tikzpicture}
    \node[draw, rectangle, minimum width=2cm, minimum height=1cm ]
      (E) {Entità \( E \)};
    \node[draw, diamond, minimum width=0.7cm, minimum height=0.7cm, above right=1cm of E]
      (R) {};
    \node[above=0.3cm] at (R) {Relazione \( R \)};

    \draw (E) |- (R);
    \draw (E) -| (R);
  \end{tikzpicture}
  \caption{Rappresentazione grafica di una relazione ricorsiva}
\end{figure}

\subsection{Attributo}
Rappresenta una proprietà elementare di un'entità o di una relazione.
Ogni attributo di un'entità o di una relazione associa ad ogni istanza \textbf{un solo}
valore appartenente ad un dominio di valori ammissibili. Può essere visto
come una funzione che ha come dominio le istanze dell'entità (o relazione) e come
codominio l'insieme dei valori ammissibili:
\[
  f_A: I(E) \mapsto D
\] 
dove \( a \) è un attributo dell'entità \( E \), mentre \( I(E) \) l'insieme delle
istanze di \( E \) e \( D \) è l'insieme dei valori ammissibili.

\vspace{1em}
\noindent
La sintassi grafica di un attributo è un cerchio \textbf{vuoto} collegato con una linea all'entità
con accanto il nome dell'attributo:
\begin{figure}[H]
  \centering
  \begin{tikzpicture}
    \node[draw, rectangle, minimum width=2cm, minimum height=1cm ]
      (E) {Entità \( E \)};
    \node[draw, circle, scale=0.7, right=0.5cm of E]
      (A) {};
    \node[right=0.1cm] at (A) {\( A \)};
    \draw (E) -- (A);
  \end{tikzpicture}
  \caption{Rappresentazione grafica di un attributo di un'entità}
\end{figure}

\begin{figure}[H]
  \centering
  \begin{tikzpicture}
    \node[draw, diamond, minimum width=0.7cm, minimum height=0.7cm ]
      (E) {};
    \node[below=0.3cm] at (E) {Relazione \( R \)};
    \node[draw, circle, scale=0.7, right=0.5cm of E]
      (A) {};
    \node[right=0.1cm] at (A) {\( A \)};
    \draw (E) -- (A);
  \end{tikzpicture}
  \caption{Rappresentazione grafica di un attributo di una relazione}
\end{figure}

\begin{example}
  Rappresentiamo il concetto di persona tramite un'entità, bisogna gestire nella base
  di dati il nome e il cognome di un gruppo di persone.
  \begin{figure}[H]
    \centering
    \begin{tikzpicture}
      \node[draw, rectangle, minimum width=2cm, minimum height=1cm ]
        (E) {Persona};
      \node[draw, circle, scale=0.7, yshift=0.4cm, right=0.5cm of E]
        (A1) {};
      \node[right=0.1cm] at (A1) {Nome};
      \node[draw, circle, scale=0.7, yshift=-0.4cm, right=0.5cm of E]
        (A2) {};
      \node[right=0.1cm] at (A2) {Cognome};
      \draw (A1) -- +(-0.65cm,0);
      \draw (A2) -- +(-0.65cm,0);
    \end{tikzpicture}
    \caption{Rappresentazione grafica dell'entità Persona con gli attributi Nome e Cognome}
  \end{figure}
\end{example}

\subsubsection{Attributo opzionale e multivalore}
L'attributo opzionale o multivalore si ottiene da un attributo normale specificando un vincolo
di cardinalità (spiegato più avanti per le relazioni \ref{Cardinalità}) sui valori che
l'attributo può assumere (il default è (1,1)). I valori possibili sono i seguenti:
\begin{itemize}
  \item (0,1): attributo opzionale
  \item (1,N): attributo multivalore obbligatorio
  \item (0,N): attributo multivalore opzionale
\end{itemize}

\vspace{1em}
\noindent
La sintassi grafica per rappresentare un attributo opzionale o multivalore è la seguente:
\begin{figure}[H]
  \centering
  \begin{tikzpicture}
    \node[draw, rectangle, minimum width=2cm, minimum height=1cm ]
      (E) {Entità \( E \)};
    \node[draw, circle, scale=0.7, right=1.3cm of E]
      (A) {};
    \node[right=0.1cm] at (A) {\( A \)};
    \draw (E) -- node[midway, above, scale=0.8] {(min,max)} (A);

  \end{tikzpicture}
  \caption{Rappresentazione grafica di un attributo opzionale o multivalore}
\end{figure}

\subsubsection{Attributo composto}
Permette di raggruppare gli attributi di un'entità o di una relazione che hanno un significato
comune.

\vspace{1em}
\noindent
La sintassi grafica per rappresentare un attributo composto è la seguente:
\begin{figure}[H]
  \centering
  \begin{tikzpicture}
    \node[draw, rectangle, minimum width=2cm, minimum height=1.5cm ]
      (E) {Entità \( E \)};
    \node[draw, circle, scale=0.7, right=1.3cm of E]
      (B) {};
    \node[right=0.1cm] at (B) {\( B \)};
    \draw (B) -- +(-0.58cm,0);

    \node[draw, circle, scale=0.7, above=0.2cm of B]
      (A) {};
    \node[right=0.1cm] at (A) {\( A \)};
    \draw (A) -- +(-0.62cm,0);

    \node[draw, circle, scale=0.7, below=0.2cm of B]
      (C) {};
    \node[right=0.1cm] at (C) {\( C \)};
    \draw (C) -- +(-0.62cm,0);

    \node[draw, ellipse, minimum height=1.3cm, right=0.5cm of E] (comp) {};

    \draw (E) -- (comp);
  \end{tikzpicture}
  \caption{Rappresentazione grafica di un attributo composto}
\end{figure}

\subsection{Identificatore}
Data un'entità \( E \), un identificatore è un insieme di proprietà (attributi e/o relazioni)
che \textbf{identificano univocamente} ogni istanza di \( E \). Un insieme di proprietà
identifica univocamente le istanze di un'entità \( E \) se \textbf{non esistono} due
istanze di \( E \) che presentano gli stessi valori o istanze nelle proprietà dell'insieme.

\vspace{1em}
\noindent
La sintassi grafica di un identificatore per un'entità con un solo attributo \( a \) è
la seguente:
\begin{figure}[H]
  \centering
  \begin{tikzpicture}
    \node[draw, rectangle, minimum width=2cm, minimum height=1cm ]
      (E) {Entità \( E \)};
    \node[draw, circle, scale=0.7, fill, right=0.5cm of E]
      (A) {};
    \node[right=0.1cm] at (A) {\( A \)};
    \draw (E) -- (A);
  \end{tikzpicture}
  \caption{Rappresentazione grafica di un identificatore con un solo attributo}
\end{figure}
\noindent
Per un identificatore costituito da più attributi \( a, b \) la sintassi grafica è la seguente:
\begin{figure}[H]
  \centering
  \begin{tikzpicture}
    \node[draw, rectangle, minimum width=2cm, minimum height=1cm ]
      (E) {Entità \( E \)};
    \node[draw, circle, scale=0.7, yshift=0.5cm, right=0.5cm of E]
      (A) {};
    \node[right=0.1cm] at (A) {\( A \)};
    \draw (A) -- +(-0.65cm,0);

    \node[draw, circle, scale=0.7, below=0.5cm of A]
      (B) {};
    \node[right=0.1cm] at (B) {\( B \)};
    \draw (B) -- +(-0.65cm,0);

    \node[draw, circle, scale=0.7, fill, above left=0.2cm of A]
      (I) {};
    \draw (I) -- +(0,-1.3cm);
  \end{tikzpicture}
  \caption{Rappresentazione grafica di un identificatore con più attributi}
\end{figure}
\noindent
Può anche esistere un identificatore costituito da una relazione \( R \) (deve essere
una funzione):
\begin{figure}[H]
  \centering
  \begin{tikzpicture}
    \node[draw, rectangle, minimum width=2cm, minimum height=1cm ]
      (E1) {Entità \( E \)};
    \node[draw, diamond, minimum width=0.7cm, minimum height=0.7cm, right=1cm of E1]
      (R) {};
    \node[below=0.3cm] at (R) {Relazione \( R \)};
    \node[draw, rectangle, minimum width=2cm, minimum height=1cm, right=1cm of R]
      (E2) {\( \ldots \)};
    \draw (E1) -- (R);
    \draw (E2) -- (R);

    \node[draw, circle, scale=0.7, yshift=0.5cm, right=0.5cm of E1]
      (A) {};
    \node[right=0.1cm] at (A) {\( A \)};
    \draw (A) -- +(-0.65cm,0);

    \node[draw, circle, scale=0.7, fill, above left=0.2cm of A]
      (I) {};
    \draw (I) -- +(0,-1.3cm);
  \end{tikzpicture}
  \caption{Rappresentazione grafica di un identificatore costituito da un attributo e una relazione}
\end{figure}
\noindent
Oppure un identificatore può essere costituito soltanto da una relazione \( R \):
\begin{figure}[H]
  \centering
  \begin{tikzpicture}
    \node[draw, rectangle, minimum width=2cm, minimum height=1cm ]
      (E1) {Entità \( E \)};
    \node[draw, diamond, minimum width=0.7cm, minimum height=0.7cm, right=1cm of E1]
      (R) {};
    \node[below=0.3cm] at (R) {Relazione \( R \)};
    \node[draw, rectangle, minimum width=2cm, minimum height=1cm, right=1cm of R]
      (E2) {\( \ldots \)};
    \draw (E1) -- (R);
    \draw (E2) -- (R);

    \node[draw, circle, scale=0.7, fill, above left=0.2cm of A]
      (I) {};
    \draw (I) -- +(0,-1.3cm);
  \end{tikzpicture}
  \caption{Rappresentazione grafica di un identificatore costituito da una relazione}
\end{figure}
\noindent
Ci sono due tipi di identificatori:
\begin{itemize}
  \item \textbf{Identificatori interni}: sono costituiti solo da attributi dell'entità
  \item \textbf{Identificatori esterni}: sono costituiti da almeno una relazione
    con un'altra entità
\end{itemize}

\vspace{1em}
\noindent
\textbf{Attenzione}: Non esistono identificatori sulle relazioni perchè ogni relazione
è già univocamente identificata dalle istanze delle entità coinvolte.

\begin{example}
  Rappresentiamo il concetto di persona tramite un'entità, bisogna gestire nella base
  di dati il codice fiscale, il nome, il cognome e la data di nascita di un gruppo
  di persone
  \begin{figure}[H]
    \centering
    \begin{tikzpicture}
      \node[draw, rectangle, minimum width=2cm, minimum height=1cm ]
        (E1) {Persona};
      \node[draw, circle, scale=0.7, yshift=0.6cm, right=0.5cm of E1]
        (nome) {};
      \node[right=0.1cm] at (nome) {Nome};
      \draw (nome) -- +(-0.65cm,0);

      \node[draw, circle, scale=0.7, yshift=0cm, right=0.5cm of E1]
        (cognome) {};
      \node[right=0.1cm] at (cognome) {Cognome};
      \draw (cognome) -- +(-0.65cm,0);

      \node[draw, circle, scale=0.7, yshift=-0.6cm, right=0.5cm of E1]
        (data_nascita) {};
      \node[right=0.1cm] at (data_nascita) {Data di nascita};
      \draw (data_nascita) -- +(-0.65cm,0);

      \node[draw, circle, scale=0.7, fill, above left=0.2cm of A]
        (I) {};
      \draw (I) -- +(0,-1.3cm);

      \node[draw, circle, fill, scale=0.7, below=0.5cm of E1]
        (fiscale) {};
      \node[right=0.1cm] at (fiscale) {Codice fiscale};
      \draw (fiscale) -- +(0,0.65cm);
    \end{tikzpicture}
    \caption{Esempio di identificatore interno}
  \end{figure}
\end{example}

\begin{definition}
  Un \textbf{vincolo di identificazione} limita la popolazione di un'entità impedendo
  l'esistenza di due istanze con gli stessi valori nelle proprietà che costituiscono
  l'identificatore.

  \vspace{1em}
  \noindent
  La scelta dell'identificatore va sempre fatta considerando le proprietà significative
  per il sistema informativo. A \textbf{livello concettuale} è quindi da \textbf{evitare}
  l'introduzione di nuovi attributi identificatori (ad esempio l'ID).
\end{definition}

\subsection{Cardinalità}\label{Cardinalità}
Tra due relazioni esistono dei vincoli di cardinalità che limitano il numero di istanze
di una entità che possono essere in relazione con una singola istanza dell'altra entità.
Data una relazione \( R \) i vincoli di cardinalità vengono specificati per ogni entità
\( E_i \) coinvolta nella relazione \( R \) e specificano il numero \textbf{minimo} e
\textbf{massimo} di istanze di \( R \) a cui un istanza di \( E_j \) deve o può 
partecipare.

\vspace{1em}
\noindent
La sintassi grafica per rappresentare i vincoli di cardinalità è la seguente:
\begin{figure}[H]
  \centering
  \begin{tikzpicture}
    \node[draw, rectangle, minimum width=2cm, minimum height=1cm ]
      (E1) {Entità \( E \)};
    \node[draw, diamond, minimum width=0.7cm, minimum height=0.7cm, right=2cm of E1]
      (R) {};
    \node[below=0.3cm] at (R) {Relazione \( R \)};
    \node[draw, rectangle, minimum width=2cm, minimum height=1cm, right=2cm of R]
      (E2) {Entità \( F \)};
    \draw (E1) -- node[midway, above, scale=0.8] {\( (min_1, max_1) \)} (R);
    \draw (E2) -- node[midway, above, scale=0.8] {\( (min_2, max_2) \)} (R);
  \end{tikzpicture}
  \caption{Rappresentazione grafica dei vincoli di cardinalità}
\end{figure}

\subsubsection{Valori possibili per \texorpdfstring{\( \text{MIN}_i \)}{MIN\_i}}
I possibili valori che il minimo \( \text{MIN}_i \) può assumere sono:
\begin{itemize}
  \item 0: Indica che la partecipazione alla relazione \( R \) delle istanze di \( E_i \)
    è \textbf{opzionale}

  \item 1: Indica che la partecipazione alla relazione \( R \) delle istanze di \( E_i \)
    è \textbf{obbligatoria}

  \item \( num > 1 \): Indica che per ogni istanza di \( E_i \) devono essere presenti
    almeno \( num \) occorrenze della relazione \( R \) che la coinvolgono
\end{itemize}

\subsubsection{Valori possibili per \texorpdfstring{\( \text{MAX}_i \)}{MAX\_i}}
\begin{itemize}
  \item 1: Indica che un'istanza di \( E_i \) può \textbf{al massimo} partecipare a una
    sola occorrenza della relazione \( R \) (se \( R \) è binaria questo indica che \( R \)
    \textbf{è una funzione})

  \item \( N \): Indica che un'istanza di \( E_i \) può partecipare a più occorrenze
    della relazione \( R \) senza limite massimo

  \item \( num > 1 \): Indica che per ogni istanza di \( E_i \) possono essere presenti
    \textbf{al massimo} \( num \) occorrenze della relazione \( R \) che la coinvolgono
\end{itemize}

\begin{example}
  Rappresentiamo il concetto di persona e comune tramite due entità, bisogna gestire
  la residenza delle persone nei comuni italiani. Questi requisiti non rappresentano
  la realtà perchè una persona deve avere obbligatoriamente una residenza e può avere
  al massimo una residenza, mentre un comune può avere zero o più persone residenti.
  \begin{figure}[H]
    \centering
    \begin{tikzpicture}
      \node[draw, rectangle, minimum width=2cm, minimum height=1cm ]
        (E1) {Persona};
      \node[draw, diamond, minimum width=0.7cm, minimum height=0.7cm, right=1.3cm of E1]
        (R) {};
      \node[below=0.3cm] at (R) {Residenza};
      \node[draw, rectangle, minimum width=2cm, minimum height=1cm, right=1.3cm of R]
        (E2) {Comune};
      \draw (E1) -- node[midway, above, scale=0.8] {\( (1, 1) \)} (R);
      \draw (E2) -- node[midway, above, scale=0.8] {\( (0, N) \)} (R);
    \end{tikzpicture}
    \caption{Esempio di Residenza con i vincoli di cardinalità}
  \end{figure}
\end{example}

\begin{definition}
  Una relazione \( R \) può partecipare ad un identificatore esterno di un'entità \( E \)
  \textbf{solo se} tale entità partecipa alla relazione \( R \) con vincolo di
  cardinalità \( (1, 1) \), quindi se \( R \) è una \textbf{funzione} che associa ad
  ogni istanza di \( E \) una e una sola istanza dell'altra entità coinvolta in \( R \).
  Questo vale anche quando \( R \) è la sola proprietà che partecipa all'identificatore.
\end{definition}

\vspace{1em}
\noindent
Si potrebbe utilizzare anche il diagramma UML per rappresentare i concetti del modello
Entità-Relazione, ma questa rappresentazione segue regole diverse, ad esempio la
posizione dove specificare nel diagramma UML i vincoli di cardinalità è invertita rispetto
all'ER.

\subsection{Generalizzazione di entità}
La generalizzazione è un legame logico (simile ad un'ereditarietà tra classi) tra un'entità
padre \( E \) e \( n \) \( (n > 0) \) entità figlie \( E_1, E_2, \ldots, E_n \) dove
\( E \) rappresenta una classe di oggetti più generale rispetto alle classi di oggetti
rappresentate dalle entità figlie.

\vspace{1em}
\noindent
La sintassi grafica per rappresentare una generalizzazione è la seguente:
\begin{figure}[H]
  \centering
  \begin{tikzpicture}
    \node[draw, rectangle, minimum width=2cm, minimum height=1cm]
      (E) {\( E \) };
    \node[draw, rectangle, minimum width=2cm, minimum height=1cm, below left=1.3cm of E]
      (E1) {\( E_1 \) };
    \node[draw, rectangle, minimum width=2cm, minimum height=1cm, below right=1.3cm of E]
      (E2) {\( E_2 \) };

    \draw[-latex] (E1.north) -- +(0,0.5) -| (E);
    \draw (E2.north) -- +(0,0.5) -| (E);

    \node[below=1.3cm of E] {\( \ldots \)};
  \end{tikzpicture}
  \caption{Esempio di Residenza con i vincoli di cardinalità}
\end{figure}

\subsubsection{Proprietà delle istanze generalizzate}
\begin{itemize}
  \item Ogni istanza di un'entità figlia \( E_i \) è anche istanza dell'entità padre \( E \)
  \item Ogni proprietà (attributi, identificatori e relazioni) dell'entità padre \( E \)
    è anche proprietà di ogni entità figlia \( E_i \) 
\end{itemize}
\begin{figure}[H]
  \centering
  \begin{tikzpicture}
    \draw (0,0) circle(2cm) node[yshift=-1.5cm] {\( I(E_{\text{padre}}) \)};
    \draw (-0.8,0) circle(1cm) node
      {\( I(E_1) \)};
    \draw (0.8,0) circle(1cm) node
      {\( I(E_2) \)};
  \end{tikzpicture}
  \caption{Istanze di entità generalizzate}
\end{figure}

\subsubsection{Classificazione delle generalizzazioni}
Una generalizzazione può essere classificata secondo due criteri:
\begin{itemize}
  \item \textbf{Totale} se ogni istanza dell'entità padre \( E \) è anche istanza
    di \textbf{almeno una} delle entità figlie \( E_i \), altrimenti si dice
    \textbf{parziale}

  \item \textbf{Esclusiva} se ogni istanza dell'entità padre \( E \) è anche istanza
    di \textbf{al massimo una} delle entità figlie \( E_i \), altrimenti si dice
    \textbf{sovrapposta}
\end{itemize}

\vspace{1em}
\noindent
Per indicare la classificazione di una generalizzazione si mettono le seguenti
notazioni accanto alla freccia della generalizzazione:
\begin{itemize}
  \item (p,s): parziale e sovrapposta: 
  \item (p,e): parziale ed esclusiva: non si possono avere istanze comuni tra le entità figlie
  \item (t,s): totale e sovrapposta: non si possono avere istanze dell'entità padre
    che non siano istanze di almeno una delle entità figlie
  \item (t,e): totale ed esclusiva: ogni istanza dell'entità padre è istanza
    di \textbf{esattamente una} delle entità figlie
\end{itemize}

\subsubsection{Relazioni ternarie}
Una relazione ternaria è una relazione che coinvolge tre entità e si usa quando:
\begin{itemize}
  \item Un'istanza del concetto per esistere nel sistema informativo richiede sempre
    la presenza di tre istanze di entità (una per ogni entità coinvolta nella relazione)

  \item Non esistono eccezioni al punto precedente
  \item Non esistono situazioni nelle quali la terna di entità debba essere rappresentata
    più volte
\end{itemize}
La sintassi grafica per rappresentare una relazione ternaria è la seguente (i vincoli
di cardinalità sono richiesti, ma qua sono omessi):

\begin{figure}[H]
  \centering
  \begin{tikzpicture}
    \node[draw, rectangle, minimum width=2cm, minimum height=1cm ]
      (E1) {Entità \( E \)};
    \node[draw, diamond, minimum width=0.7cm, minimum height=0.7cm, right=1cm of E1]
      (R) {};
    \node[above=0.3cm] at (R) {Relazione \( R \)};
    \node[draw, rectangle, minimum width=2cm, minimum height=1cm, right=1cm of R]
      (E2) {Entità \( F \)};
    \node[draw, rectangle, minimum width=2cm, minimum height=1cm, below=1cm of R]
      (E3) {Entità \( G \)};

    \draw (E1) -- (R);
    \draw (E2) -- (R);
    \draw (E3) -- (R);
  \end{tikzpicture}
  \caption{Rappresentazione grafica di una relazione tra due entità}
\end{figure}
\noindent
Un uso errato avviene quando per rappresentare lo stato del sistema informativo
sarebbe necessario che un'istanza della relazione \textbf{contenga solo una
coppia e non una terna} di istanze di entità (cioè quando manca un attore).
Quindi in questo caso si può sostituire la relazione ternaria con 3 relazioni binarie
indipendenti.

\vspace{1em}
\noindent
L'utilizzo delle relazioni ternarie è utile per "storicizzare" le relazioni binarie.
Nella rappresentazione di una relazione binaria storicizzata è necessario poter rappresentare
più volte \textbf{la stessa coppia in tempi diversi}. Pertanto la relazione binaria può
essere trasformata in una relazione ternaria includendo l'entità \textbf{Tempo} come
nuovo attore nella relazione in modo da poter rappresentare più volte la stessa coppia
in istanti di tempo diversi.

\section{Progettazione logica}
\subsection{Modello relazionale}
Il \textbf{modello relazionale} è un modello dei dati, quindi permette di definire le proprietà
che la popolazione di una base di dati deve rispettare. I costrutti principali del
modello relazionale sono:
\begin{itemize}
  \item \textbf{Domini di base}: insieme di valori atomici (non scomponibili) che
    rappresentano i tipi di dati elementari
  \item \textbf{Relazione (o tabella)}: rappresenta un insieme di entità o di relazioni
    tra entità. Sono definite come un insieme di ennuple o tuple.
  \item \textbf{Superchiavi, chiavi e chiavi primarie}: insieme di attributi che identificano
    univocamente ogni ennupla di una relazione
  \item \textbf{Vincoli di integrità referenziale}: vincoli che garantiscono la coerenza tra
    i dati memorizzati in relazioni diverse
  \item \textbf{Vincoli di integrità generici}: vincoli che garantiscono la coerenza dei dati
    memorizzati in una singola relazione
\end{itemize}

\subsection{Domini di base}
Sono i domini da cui si scelgono i valori delle proprietà delle istanze di informazione da
rappresentare. I domini tipici sono:
\begin{itemize}
  \item Caratteri
  \item Stringhe di caratteri
  \item Numeri interi
  \item Numeri decimali a virgola fissa
  \item Numeri decimali a virgola mobile
  \item Domini del tempo, per rappresentare istanti e intervalli di tempo
  \item ecc...
\end{itemize}
I domini presenti variano a seconda del sistema, ma in questo corso vedremo il linguagigo
SQL nel sistema postgreSQL.

\subsection{Relazione}
Una relazione può essere vista come una tabella.
\begin{example}
  \label{tab:esempio_relazione}
  Ad esempio:
  \begin{table}[H]
    \centering
    \begin{tabular}{|l|l|l|}
      \hline
      Milano & 20100 & 1.300.000 \\
      \hline
      Verona & 37100 & 350.000 \\
      \hline
      Brescia & 25100 & 250.000 \\
      \hline
    \end{tabular}
    \caption{Esempio di tabella che rappresenta alcune città italiane}
  \end{table}
\end{example}
Una tabella è un contenitore di dati la cui struttura è caratterizzata da una lista
di colonne:
\begin{itemize}
  \item I dati sono scritti nelle righe dove \textbf{ogni riga} descrive le caratteristiche di
    \textbf{un'istanza dell'informazione da rappresentare}
  \item I valori contenuti nell \textbf{colonne} descrivono sempre la \textbf{stessa
    proprietà delle istanze} di informazione da rappresentare
\end{itemize}

\begin{definition}[Definizione di relazione come insieme di ennulple (Lists)]
  Dati \( n \) insiemi di valori (domini) \( D_1, \ldots, D_n \), con
  \( n > 0 \) e indicato con \( D_1 \times \ldots \times D_n \) il loro prodotto
  cartesiano:
  \[
    D_1 \times \ldots \times D_n = \left\{ (v_1, \ldots, v_n) \;\left|\; v_1 \in D_1 \wedge \ldots
    \wedge v_n \in D_n \right. \right\}
  \] 
  una relazione \( \rho \) di grado \( n \) è un qualsiasi \textbf{sottoinsieme di}
  \( D_1 \times \ldots \times D_n \):
  \[
    \rho \subseteq D_1 \times \ldots \times D_n
  \] 
  dove \( \left( v_1, \ldots, v_n \right) \) è una ennupla della relazione e \( |\rho| \) 
  è la cardinalità della relazione (numero di ennuple).

  \vspace{1em}
  \noindent
  \textbf{Nota}:
  \begin{itemize}
    \item I domini \( D_1, \ldots, D_n \) possono essere a \textbf{cardinalità infinita},
      mentre le relazioni sono sempre a \textbf{cardinalità finita}
    \item Dalla definizione si deduce inoltre che:
      \begin{itemize}
        \item \textbf{Non è definito alcun ordinamento} sulle ennuple di una relazione
        \item \textbf{Non sono ammessi duplicati} di una ennupla
        \item Nella definizione di relazione come insieme di ennuple, i valori delle
          ennuple sono ordinati
      \end{itemize}
  \end{itemize}
\end{definition}

\begin{example}
  Riprendendo l'esempio \ref{tab:esempio_relazione}, la relazione può essere vista come
  insieme di ennuple (relazione città):
  \[
      \rho = \left.\begin{cases}
        \left( \text{Milano}, 20100, 1300000 \right),\\
        \left( \text{Verona}, 37100, 350000 \right),\\
        \left( \text{Brescia}, 25100, 250000 \right)\\
      \end{cases}
      \right\}
  \] 
  dove:
  \[
    \rho \subseteq D_1 \times D_2 \times D_3
  \] 
\end{example}

\subsubsection{Accesso ai valori di una ennupla}
Se \( t \) è una ennupla \( \left( v_1, \ldots, v_n \right)  \) il valore posto in i-esima
posizione si indica con la notazione:
\[
  t[i]
\] 
Questa modalità di accesso ai valori però non è efficace per l'uso pratico delle relazioni.
Si preferisce quindi assegnare un \textbf{nome alle colonne}; ciò conduce all'introduzione
della definizione di relazione come insieme di \textbf{tuple}.

\begin{definition}[Definizione di relazione come insieme di tuple (Mappings)]
  Sia \( X \) un insieme di nomi e sia \( \Delta \) l'insieme di tutti i domini di base
  ammessi dal modello. Si definisce la funzione:
  \[
    DOM: X \to \Delta
  \] 
  Tale funzione associa ad ogni nome \( A \in X \) un dominio \( DOM(A) \in \Delta \).
  I nomi di \( X \) si definiscono \textbf{attributi}.
  Una tupla \( t \) su \( X \) è una funzione:
  \[
    t: X \to \bigcup_{A \in X} DOM(A)
  \] 
  dove:
  \[
    t[A] = v \in DOM(A)
  \] 
  Quindi una relazione su \( X \) è un \textbf{insieme di tuple su } \( X \), dove
  \( X \) è l'insieme di attributi della relazione.

  \vspace{1em}
  \noindent
  \textbf{Nota}:
  \begin{itemize}
    \item Una relazione è un \textbf{insieme} di tuple e quindi \textbf{non può contenere
      tuple duplicate}
    \item I domini per gli attributi possono essere solo domini di base, \textbf{non sono
      ammessi altri domini}, nè \textbf{il prodotto cartesiano di domini}. Essenzialmente
      non sono ammessi attributi annidati
    \item In generale una base di dati relazionale è composta da \textbf{più relazioni}
  \end{itemize}
\end{definition}
\begin{example}
  Consideriamo l'esempio \ref{tab:esempio_relazione}. Definiamo l'insieme di attributi:
  \[
    X = \{ \text{Nome}, \text{CAP}, \text{Abitanti} \}
  \] 
  Definiamo il dominio di ciascun attributo:
  \[
    \begin{aligned}
      DOM(\text{Nome}) & = \text{Stringhe di caratteri} \\
      DOM(\text{CAP}) & = \text{Numeri interi} \\
      DOM(\text{Abitanti}) & = \text{Numeri interi}
    \end{aligned}
  \]
  La tabella della relazione è rappresentata come un insieme di tuple:
  \[
    \rho_X = \left\{
      t_1, t_2, t_3
    \right\}
  \] 
  dove:
  \[
    \begin{aligned}
      t_1[\text{Nome}] & = \text{Milano} & t_1[\text{CAP}] & = 20100 & t_1[\text{Abitanti}] & = 1300000 \\
      t_2[\text{Nome}] & = \text{Verona} & t_2[\text{CAP}] & = 37100 & t_2[\text{Abitanti}] & = 350000 \\
      t_3[\text{Nome}] & = \text{Brescia} & t_3[\text{CAP}] & = 25100 & t_3[\text{Abitanti}] & = 250000 \\
    \end{aligned}
  \] 
\end{example}

\subsection{Progettazione nel modello relazionale}
\subsubsection{Progettazione dei dati}
Nel modello relazionale lo schema è costituito da un insieme di relazioni (o tabelle).
Le dipendenze tra proprietà elementari (attributi) sono costituite dalle
\textbf{dipendenze funzionali} che in questo corso sono sostituite dalla progettazione
concettuale. 
Consideriamo i seguenti requisiti da implementare nel modello relazionale:

Vogliamo rappresentare in una base di dati relazionale le informazioni sulla
proprietà degli appartamenti siti nel comune di Verona.
\begin{itemize}
  \item Per ogni appartamento vogliamo memorizzare: il codice ecografico (univoco),
    la categoria catastale e l’indirizzo composto da: via, numero civico,
    subalterno.

  \item Per ogni proprietario si registra: il codice fiscale, il nome, il cognome e la data
    di nascita.

  \item Un appartamento può essere in comproprietà e un proprietario può
    possedere più appartamenti
\end{itemize}

\subsubsection{Relazione unica}
Una soluzione possibile è rappresentata da un'unica relazione che contiene tutte le
informazioni sugli appartamenti e sui proprietari:
\begin{figure}[H]
  \centering
  \includegraphics[width=1\textwidth]{esempio_relazionale}
  \caption{Schema relazionale con una sola relazione Proprietà}
\end{figure}
\noindent
Questa tabella associa ogni appartamento ad un proprietario perchè i dati dell'appartamento
e della persona fanno parte della stessa riga. Il problema è che non c'è distinzione
tra queste due categorie di informazioni. Questo porta a diversi problemi:
\begin{enumerate}
  \item \textbf{Mancata separazione delle informazioni}: le due categorie di
    informazione non possono esistere separatamente.

  \item \textbf{Ridondanza dei dati}: se un proprietario possiede più appartamenti
    i suoi dati vengono memorizzati più volte. Questo comporta un aumento dello spazio
    di memorizzazione e una possibile inconsistenza dello stesso dato.
\end{enumerate}

\subsubsection{Anomalie}
Le problematiche descritte nel paragrafo precedente portano sono descritte in modo
più dettagliato come \textbf{anomalie} che si dividono in:
\begin{itemize}
  \item \textbf{Anomalia di aggiornamento}: per aggiornare il valore di un attributo
    si è obbligati a modificare tale valore su più tuple

  \item \textbf{Anomalie di inserimento}: per inserire una nuova tupla è necessario
    inserire valori al momento sconosciuti (sostituibili da valori nulli) per gli
    attributi non disponibili
    
  \item \textbf{Anomalie di cancellazione}: per cancellare una tupla è necessario
    cancellare valori ancola validi oppure inserire valori nulli per gli attributi
    da cancellare
\end{itemize}

\subsubsection{Decomposizione della relazione unica}
Riprendiamo l'esempio precedente con una singola relazione e decomponiamola separando
le diverse categorie di informazioni in relazioni distinte. Ci sono diversi modi
per effettuare questa decomposizione:
\begin{enumerate}
  \item Separazione in due relazioni distinte per Appartamento e Proprietario
    \begin{figure}[H]
      \centering
      \includegraphics[width=1\textwidth]{esempio_relazionale_decomposto_1}
      \caption{Schema relazionale con relazioni distinte per Appartamento e Proprietario}
    \end{figure}
    \noindent
    Questa separazione risolve i problemi di ridondanza e anomalie di aggiornamento.
    Però manca l'informazione che mette in relazione la persona con l'appartamento.
    Quindi siccome si è persa un'informazione, questa decomposizione è detta
    \textbf{decomposizione con perdita}.

  \item Separazione in due relazioni distinte per Appartamento e Proprietario
    con l'aggiunta di un informazione identificante per la relazione Appartamento, in questo
    casoè 
    \begin{itemize}
      \item codice fiscale della Persona nella relazione Appartamento 
    \end{itemize}
    \begin{figure}[H]
      \centering
      \includegraphics[width=0.6\textwidth]{esempio_relazionale_decomposto_2_1}
      \caption{Relazione Appartamento con codice fiscale}
    \end{figure}
    \begin{figure}[H]
      \centering
      \includegraphics[width=0.6\textwidth]{esempio_relazionale_decomposto_2_2}
      \caption{Relazione Proprietario con codice ecografico}
    \end{figure}
    \noindent
    Siccome non esistono i puntatori bisogna duplicare i valori identificanti 
    nelle due relazioni. Anche questa decomposizione ha dati ridondanti.

  \item Separazione in due relazioni distinte per Appartamento e Proprietario
    con l'aggiunta di un informazione identificante per la relazione Proprietario, in questo
    caso è: 
    \begin{itemize}
      \item codice ecografico dell'Appartamento nella relazione Proprietario
    \end{itemize}
    \begin{figure}[H]
      \centering
      \includegraphics[width=0.6\textwidth]{esempio_relazionale_decomposto_2_1}
      \caption{Relazione Appartamento con codice fiscale}
    \end{figure}
    \begin{figure}[H]
      \centering
      \includegraphics[width=0.6\textwidth]{esempio_relazionale_decomposto_2_2}
      \caption{Relazione Proprietario}
    \end{figure}
    \noindent
    Siccome non esistono i puntatori bisogna duplicare i valori identificanti 
    nelle due relazioni. Anche questa decomposizione ha dati ridondanti.

  \item Separazione in due relazioni distinte per Appartamento e Proprietario
    con l'aggiunta di un informazione identificante per ciascuna relazione, in questo
    caso sono:
    \begin{itemize}
      \item codice fiscale nella relazione Appartamento 
      \item codice ecografico nella relazione Proprietario
    \end{itemize}
    \begin{figure}[H]
      \centering
      \includegraphics[width=0.6\textwidth]{esempio_relazionale_decomposto_3_1}
      \caption{Relazione Appartamento}
    \end{figure}
    \begin{figure}[H]
      \centering
      \includegraphics[width=0.6\textwidth]{esempio_relazionale_decomposto_3_2}
      \caption{Relazione Proprietario con codice ecografico}
    \end{figure}
    \noindent
    Anche in questo caso si presentano le stesse problematiche del punto precedente.

  \item Separazione in tre relazione distinte:
    \begin{itemize}
      \item Appartamento
      \item Persona
      \item Proprietà (relazione tra Appartamento e Persona)
    \end{itemize}
    Per rappresentare la relazione tra Appartamento e Persona si introduce
    una nuova relazione chiamata Proprietà che contiene i valori degli attributi
    identificanti di entrambe le relazioni.
    \begin{figure}[H]
      \centering
      \includegraphics[width=0.6\textwidth]{esempio_relazionale_decomposto_4_1}
      \caption{Relazione Appartamento}
    \end{figure}
    \begin{figure}[H]
      \centering
      \includegraphics[width=0.6\textwidth]{esempio_relazionale_decomposto_4_2}
      \caption{Relazione Persona}
    \end{figure}
    \begin{figure}[H]
      \centering
      \includegraphics[width=0.3\textwidth]{esempio_relazionale_decomposto_4_3}
      \caption{Relazione Proprietà}
    \end{figure}
    \noindent
    In questa decomposizione non esistono ridondanze e non si perdono informazioni.
    Quindi questa è la decomposizione corretta e \textbf{senza perdita}. L'unica
    ridondanza è quella degli attributi identificanti che però è necessaria per
    rappresentare la relazione tra le due entità.
\end{enumerate}

\subsubsection{Proprietà del modello relazionale}
\begin{itemize}
  \item I legami tra relazioni si realizzano attraverso la replicazione
    di un insieme di attributi, dove il legame tra due tuple si
    intende stabilito quando esse presentano gli stessi valori
    negli attributi replicati.

  \item Il modello relazionale è \textbf{value-based} (basato sui valori).
    Il modello non prevede l’uso di puntatori per rappresentare
    legami tra i dati. Questo implica:
    \begin{itemize}
      \item Completa indipendenza dall'implementazione e dallo schema fisico
      \item Facilità di trasferimento dei dati tra sistemi
      \item Viene rappresentato solo ciò che è rilevante per il sistema informativo
    \end{itemize}
\end{itemize}

\subsubsection{Terminologia}
\begin{itemize}
  \item \textbf{Schema di una relazione}: è costituito dal nome della relazione e
    da un insieme di nomi per i suoi attributi:
    \[
      R(X) \text{ oppure } R(A_1, \ldots, A_n) \text{ oppure } R(A_1: D_1, \ldots, A_n: D_n)
    \] 

  \item \textbf{Schema di una base di dati relazionale}: è costituito da un insieme di
    schemi di relazioni:
    \[
      S = \{ R_1(X_1), \ldots, R_n(X_n) \}
    \]
    con \( R_1 \neq \ldots \neq R_n \) 

  \item \textbf{Istanza di uan relazione di schema \( R(X) \)}: è un insieme \( r \) 
    di tuple su \( X \):
    \[
      r = \{ t_1, \ldots, t_k \}
    \]

  \item \textbf{Istanza di una base di dati relazionale di schema\\ \( S = \{ R_1(X_1), \ldots, R_n(X_n) \} \)}:
    è costituito da un insieme di istanze delle relazioni \( R_1(X_1), \ldots, R_n(X_n) \):
    \[
      db = \{ r_1, \ldots, r_n \}
    \] 
\end{itemize}

\subsection{Valori nulli}
Dalla definizione di relazione si ha che ogni tupla deve sempre contenere nei propri
attributi valori significativi appartenenti ai domini di base del modello. Non sempre
però in una base di dati reale esistono i valori per tutti gli attributi di una tupla.
Un valore può mancare nei seguenti casi:
\begin{itemize}
  \item Il valore di un attributo \( A \) è \textbf{inesistente} (attributo opzionale
    a livello concettuale), cioè non esiste per questa tupla un valore per l'attributo
    \( A \).

  \item Il valore di un attributo \( A \) è \textbf{sconosciuto}, cioè esiste un valore
    per l'attributo \( A \) di questa tupla ma non è noto alla base di dati.

  \item Il valore di un attributo \( A \) è inesistente o sconosciuto.
\end{itemize}
Per poter gestire tali situazioni viene introdotto nel modello relazionale un valore
speciale detto \textbf{valore nullo}. Quindi gli attributi di una tupla possono assumere
un valore del dominio oppure il valore nullo. Il valore nullo non dovrebbe essere abusato.

\vspace{1em}
\noindent
\begin{definition}[Tupla con valore nullo]
  Una tupla su \( X \) con valori nullo è definita come una funzione:
  \[
    t: X \to \{NULL\} \cup \left( \bigcup_{A \in X} DOM(A) \right)
  \] 
  dove:
  \[
    t[A] = v \in DOM(A) \vee t[A] = NULL
  \] 
\end{definition}

\subsubsection{Osservazioni}
\begin{itemize}
  \item La presenza di valori nulli è accettabile \textbf{solo in alcuni attributi}
    (non è possibile avere tuple di soli valori nulli).

  \item In particolare, negli \textbf{attributi replicati per rappresentare legami tra
    tuple} la presenza di valori nulli può rendere inutilizzabile l’informazione
\end{itemize}

\begin{example}
  \label{ex:orario_ferroviario}
  Rappresentare attraverso un insieme di relazioni le
  informazioni contenute in un orario ferroviario dove si
  riportino per ogni treno in partenza dalla stazione di
  Verona Porta Nuova:
  il numero, l’orario di partenza, la destinazione finale, la categoria,
  le fermate intermedie (con orario di fermata) e l’orario di arrivo
  alla destinazione finale.

  Lo schema relazionale risultante è il seguente:
  \[
    \begin{aligned}
      &TRENO(Numero, Categoria, OraPart, Destinazione, OraArr)\\
      &FERMATA(NumTreno, Stazione, Orario)
    \end{aligned}
  \] 
  In progettazione concettuale equivale a due entità (Treno e Stazione) collegate
  da una relazione (Fermata).
\end{example}

\subsection{Vincoli di integrità}
Consentono di precisare quali sono le condizioni (vincoli) che le istanze delle relazioni
devono rispettare per essere considerate valide.
\begin{definition}
  Un vincolo di integrità è una condizione, espressa da un \textbf{predicato}, che deve
  essere \textbf{sempre} soddisfatta da \textbf{ogni istanza} della base di dati.
\end{definition}

\vspace{1em}
\noindent
\begin{example}
  Consideriamo l'esempio \ref{ex:orario_ferroviario}, i predicati che esprimono possibili
  vincoli sono:
  \[
    \begin{aligned}
      \forall t \in TRENO:& t[OraPart] \in \left\{ 0, 1, \ldots, 23 \right\}\\
      \forall t \in TRENO:& t[MinutoPart] \in \left\{ 0, 1, \ldots, 59 \right\}\\
      \forall t \in TRENO:& t[Numero] > 0\\
      \forall t \in TRENO:& t[Numero] > 5000 \Rightarrow t[Categoria] = \text{'Regionale'}\\
      \forall t, t' \in TRENO:& t \neq t' \Rightarrow t[Numero] \neq t'[Numero]\\
      \forall f \in FERMATA:& \exists t \in TRENO: f[NumTreno] = t[Numero]\\
        \forall t \in TRENO:& t[Categoria] = \text{'Regionale'}\\
                            & \Rightarrow \exists f \in FERMATA: f[NumTreno] = t[Numero]
    \end{aligned}
  \] 
\end{example}

\subsubsection{Classificazione}
I vincoli di integrità si possono classificare in:
\begin{itemize}
  \item Vincoli di dominio
  \item Vincoli di tupla
  \item Vincoli intrarelazionali
    \begin{itemize}
      \item Chiavi (vincolo strutturale)
    \end{itemize}
  \item Vincoli interrelazionali
    \begin{itemize}
      \item Vincoli di integrità referenziale (vincolo strutturale)
    \end{itemize}
\end{itemize}
Ci sono anche vincoli che hanno più importanza degli altri in quanto garantiscono
\textbf{proprietà generali valide per tutti gli schemi relazionali}. Questi
vincoli sono detti \textbf{vincoli strutturali}.

\subsubsection{Vincoli di dominio}
Impongono una restrizione sui valori che un attributo può assumere. Ad esempio:
\[
    \begin{aligned}
      \forall t \in TRENO:& t[OraPart] \in \left\{ 0, 1, \ldots, 23 \right\}\\
      \forall t \in TRENO:& t[MinutoPart] \in \left\{ 0, 1, \ldots, 59 \right\}\\
      \forall t \in TRENO:& t[Numero] > 0\\
    \end{aligned}
\] 

\subsubsection{Vincoli di tupla}
Impongono uan restrizione alla \textbf{combinazione di valori che una tupla della relazione}
può assumere indipendentemente dalle altre tuple, ad esempio:
\[
  \begin{aligned}
    \forall t \in TRENO:& t[Numero] > 5000 \Rightarrow t[Categoria] = \text{'Regionale'}\\
  \end{aligned}
\] 

\subsubsection{Vincoli di intrarelazionali}
Impongono una restrizione al contenuto di uan relazione e specificano una condizione che 
\textbf{ogni tupla della relazione deve soddisfare rispetto alle altre tuple della medesima
relazione}. Una sottocateoria importante di questi vincoli include i \textbf{vincoli di chiave}:
\begin{itemize}
  \item \textbf{Superchiave}: Data una relazione di schema \( R(X) \), un insieme di
    attributi \( K \) sottoinsieme di \( X \) è \textbf{superchiave} per \( R(X) \) se
    per ogni istanza \( r \) di \( R(X) \) vale la seguente condizione:
    \[
      \forall t, t' \in r: t \neq t' \Rightarrow t[K] \neq t'[K]
    \] 
    dove:
    \[
      t[K] \neq t'[K] \equiv \exists A_i \in K: t[A_i] \neq t'[A_i]
    \] 

  \item \textbf{Chiave candidata}: Data una relazione di schema \( R(X) \), un insieme di
    attributi \( K \) sottoinsieme di \( X \) è \textbf{chiave candidata} (o chiave) per
    \( R(X) \) se \( K \) è superchiave per \( R(X) \) e vale la seguente condizione:
    \[
      \neg \exists K' \subseteq K: K' \text{ è superchiave per } R(X)
    \] 

    \begin{theorem}
      Esiste sempre una chiave candidata \( K \) per una relazione \( R(X) \)
    \end{theorem}

  \item \textbf{Chiave primaria}: Data una relazione di schema \( R(X) \) la sua
    \textbf{chiave primaria} è la chiave candidata scelta per \textbf{identificare le
    tuple della relazione}. La chiave primaria viene usata per rappresentare i legami
    tra le relazioni e ha le seguenti caratteristiche:
    \begin{itemize}
      \item Non contiene \textbf{mai valori nulli}
      \item Su \( K \) il sistema genera una struttura d'accesso ai dati (o indice)
        per supportare le interrogazioni
    \end{itemize}
\end{itemize}

\begin{example}
  Consideriamo l'esempio \ref{ex:orario_ferroviario}. I seguenti sottoinsiemi
  di \( X \) sono:
  \[
  \begin{aligned}
    K1 &= \{Numero\} \text{ è una superchiave e chiave candidata }\\
    K2 &= \{Numero, Categoria\} \text{ è una superchiave ma non chiave candidata}\\
    K3 &= \{OraPart, MinutoPart, Destinazione, Categoria\} \\
       &\text{ è una superchiave e chiave candidata}
  \end{aligned}
  \] 
\end{example}

\end{document}
