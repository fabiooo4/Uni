\documentclass[a4paper]{article}
\usepackage{import}
\usepackage[utf8]{inputenc}
\usepackage[T1]{fontenc}
\usepackage{textcomp}
\usepackage[italian]{babel}
\usepackage{amsmath, amssymb}
\usepackage{booktabs,xltabular}
\usepackage{amsfonts}
\usepackage{subcaption}
\usepackage{amsthm}
\usepackage{cancel}
\usepackage{mdframed}
\usepackage{makecell}
\usepackage{float}
\usepackage{xcolor}
\usepackage{listings}
\usepackage{gensymb}
\usepackage{graphicx}
\usepackage{bodeplot}
\usepackage{physics}
\usepackage{tikz}
\usetikzlibrary{shapes, arrows, automata, petri, decorations.markings, decorations.pathreplacing, positioning, calc, quotes}
\usepackage{circuitikz}
\usepackage[label=corner]{karnaugh-map}
\graphicspath{{./figures/}}

% Set default font to sans-serif
\renewcommand{\familydefault}{\sfdefault} 
\usepackage{eulervm}

\usepackage{forest}

\usepackage{mathtools}
\DeclarePairedDelimiter\ceil{\lceil}{\rceil}
\DeclarePairedDelimiter\floor{\lfloor}{\rfloor}

% \usepackage{ntheorem}

\usepackage{import}
\usepackage{pdfpages}
\usepackage{transparent}
\usepackage{xcolor}

\usepackage{hyperref}
\hypersetup{
    colorlinks=false,
}

% Code blocks
\definecolor{codegreen}{rgb}{0,0.6,0}
\definecolor{codegray}{rgb}{0.5,0.5,0.5}
\definecolor{codepurple}{rgb}{0.58,0,0.82}
\definecolor{backcolour}{rgb}{0.95,0.95,0.95}

\lstdefinestyle{mystyle}{
	backgroundcolor=\color{backcolour},
	commentstyle=\color{codegreen},
	keywordstyle=\color{magenta},
	numberstyle=\tiny\color{codegray},
	stringstyle=\color{codepurple},
	basicstyle=\ttfamily\footnotesize,
	breakatwhitespace=false,
	breaklines=true,
	captionpos=b,
	keepspaces=true,
	numbers=left,
	numbersep=5pt,
	showspaces=false,
	showstringspaces=false,
	showtabs=false,
	tabsize=2
}

\lstset{style=mystyle}

\usepackage{color}
\usepackage{import}
\usepackage{pdfpages}
\usepackage{transparent}
\usepackage{xcolor}

% Example frame
\theoremstyle{definition}
\newmdtheoremenv[%
	linecolor=gray,leftmargin=0,%
	rightmargin=0,
	innertopmargin=8pt,%
	innerbottommargin=8pt,
	ntheorem]{example}{Esempio}[section]

% Important definition frame
\theoremstyle{definition}
\newmdtheoremenv[%
	linecolor=gray,leftmargin=0,%
	rightmargin=0,
	backgroundcolor=gray!40,%
	innertopmargin=8pt,%
	innerbottommargin=8pt,
	ntheorem]{definition}{Definizione}[section]

% Exercise frame
\theoremstyle{definition}
\newmdtheoremenv[%
	linecolor=gray,leftmargin=0,%
	rightmargin=0,
	innertopmargin=8pt,%
	innerbottommargin=8pt,
	ntheorem]{exercise}{Esercizio}[section]

% Theorem frame
\theoremstyle{definition}
\newmdtheoremenv[%
  linecolor=gray,leftmargin=0,%
  rightmargin=0,
  innertopmargin=8pt,%
  innerbottommargin=8pt,
  ntheorem]{theorem}{Teorema}[section]

\theoremstyle{definition}
\newmdtheoremenv[%
  linecolor=white,leftmargin=0,%
  rightmargin=0,
  innertopmargin=8pt,%
  innerbottommargin=8pt,
  ntheorem]{define}{Definizione utile}[section]

% figure support
\usepackage{import}
\usepackage{xifthen}
\pdfminorversion=7
\usepackage{pdfpages}
\usepackage{transparent}
\newcommand{\incfig}[1]{%
	\def\svgwidth{\columnwidth}
	\import{./figures/}{#1.pdf_tex}
}

% FSM tikz
\tikzset{
    place/.style={
        circle,
        thick,
        draw=black,
        minimum size=6mm,
    },
        state/.style={
        circle,
        thick,
        draw=black,
        fill=white,
        minimum size=6mm,
    },
}

\pdfsuppresswarningpagegroup=1

\usepackage{pgfplots}
\pgfplotsset{compat=1.18,width=10cm}

% Save plots as pdf and reuse them without compiling every time
\usetikzlibrary{external}
\tikzexternalize[prefix=figures/tikz/, optimize=false]


\begin{document}

\begin{titlepage}
	\begin{center}
		\vspace*{1cm}

		\Huge
		\textbf{Probabilità e Statistica\\Esercizi}

		\vspace{0.5cm}
		\LARGE
		UniVR - Dipartimento di Informatica

		\vspace{1.5cm}

		\textbf{Fabio Irimie}

		\vfill


		\vspace{0.8cm}


		2° Semestre 2023/2024

	\end{center}
\end{titlepage}


\tableofcontents
\pagebreak

\section{Introduzione}
Nel 1950 Alan Turing pubblica un articolo intitolato "Computing Machinery and Intelligence" in cui propone
un esperimento per determinare se una macchina può essere considerata intelligente. L'esperimento, noto
come "test di Turing", coinvolge un interrogatore umano che comunica con due entità nascoste: una macchina
e un essere umano. L'interrogatore deve fare domande a entrambe le entità e, basandosi sulle risposte,
deve determinare quale delle due è la macchina. Se l'interrogatore non riesce a distinguere tra le risposte
della macchina e quelle dell'essere umano, la macchina è considerata intelligente.

\vspace{1em}
\noindent
In futuro l'attenzione si è spostata sulla ricerca di metodi per risolvere problemi che richiedono intelligenza
umana, utilizzando algoritmi e modelli matematici fino ad arrivare alle reti neurali e intelligenza artificiale.

\begin{definition}
  L'intelligenza artificiale è una disciplina che studia come \textbf{simulare} l'intelligenza umana in
  scenari complessi
\end{definition}

\subsection{Tipi di intelligenza artificiale}
\subsubsection{Autonomous agents}
Sono sistemi che percepiscono l'ambiente e agiscono in modo autonomo per raggiungere obiettivi specifici.

\subsubsection{Data analysis}
Utilizzo di algoritmi per analizzare grandi quantità di dati e estrarre informazioni utili e correlazioni
complesse.

\subsubsection{Machine Learning}
È lo sviluppo di algoritmi che permettono a dei modelli di apprendere dai dati
di esempio e migliorare le loro prestazioni nel tempo senza essere esplicitamente programmati.
Ad esempio riconoscimento di immagini.

L'apprendimento automatico è diviso in tre categorie principali:
\begin{itemize}
  \item \textbf{Unsupervised learning}: il modello viene addestrato su un insieme di dati non etichettati,
  dove l'obiettivo è scoprire strutture nascoste o pattern nei dati senza avere risposte corrette predefinite.
\item \textbf{Supervised learning}: il modello viene addestrato su un insieme di dati etichettati,
  dove ogni esempio di input è associato a una risposta corretta. L'obiettivo è che il modello impari a
  mappare gli input alle risposte corrette.
  \item \textbf{Reinforced learning}: il modello impara attraverso interazioni con l'ambiente, ricevendo
  ricompense o penalità in base alle azioni intraprese. L'obiettivo è massimizzare la ricompensa totale nel
  tempo.
\end{itemize}

\subsubsection{Time series analysis}
L'analisi delle serie temporali è un'area dell'apprendimento automatico che si concentra sull'analisi di dati
collezionati nel tempo. Le serie temporali sono sequenze di dati misurati a intervalli regolari, come
temperatura giornaliera, prezzi delle azioni o dati di vendita mensili. L'obiettivo dell'analisi delle
serie temporali è identificare pattern, tendenze e stagionalità nei dati per fare previsioni future.

\vspace{1em}
\noindent
Gli approcci comuni per l'analisi delle serie temporali includono:
\begin{itemize}
  \item \textbf{Riconoscimento di anomalie e cause}:
    è un processo di identificazione di dati o eventi che si discostano
    significativamente dal comportamento normale o atteso. Queste anomalie possono indicare problemi,
    errori o situazioni insolite che richiedono attenzione.

  \item \textbf{Generative transformers}:
    sono una classe di modelli che permettono di predirre il prossimo elemento in una
    sequenza di dati partendo dagli elementi precedenti, come ad esempio la parola successiva in una frase o il
    pixel successivo in un'immagine. Si sfrutta il concetto di \textbf{attenzione} per pesare l'importanza
    relativa delle diverse parti della sequenza di input durante la generazione dell'output.
\end{itemize}

\subsubsection{Intelligent Agents}
Un agente intelligente è un sistema che percepisce l'ambiente circostante attraverso sensori e agisce su
l'ambiente per raggiungere un obiettivo specifico. Gli elementi chiave di un agente intelligente includono:
\begin{itemize}
  \item \textbf{Performance measure}: misura il successo dell'agente nel raggiungere i suoi obiettivi
  \item \textbf{Rationality}: l'agente deve agire in modo da massimizzare la sua performance measure attesa
\end{itemize}

\subsection{Markov Decision Process (MDP)}
Un MDP è un modello matematico utilizzato per rappresentare problemi di decisione sequenziali. Gli elementi
principali sono:
\begin{itemize}
  \item \textbf{State}: rappresenta l'ambiente in un dato momento
  \item \textbf{Actions}: insieme delle azioni che l'agente può intraprendere
  \item \textbf{Transition model}: effetto che le azioni hanno sull'ambiente (potrebbero essere parzialmente incognite
    \[
      T: (state, action) \to next\_state
    \] 
  \item \textbf{Reward}: valore \textbf{immediato} dell'esecuzione di un'azione
    \[
      R: (state, action, next\_state) \to real\_number
    \] 
  \item \textbf{Policy}: strategia che l'agente utilizza per decidere quale azione intraprendere in ogni stato
    con l'obiettivo di massimizzare la ricompensa totale attesa nel tempo
    \[
      \pi: (state) \to action
    \] 
\end{itemize}

\subsection{Generative AI}
L'intelligenza artificiale generativa si riferisce a una classe di modelli di intelligenza artificiale
che sono in grado di generare nuovi contenuti, come testo, immagini, musica o video, a partire da dati di
addestramento. Questi modelli hanno miliardi di parametri e sono \textbf{preaddestrati} su grandi quantità
di dati. In sostanza questi modelli "predicono il futuro" basandosi sui dati su cui sono stati addestrati e
un \textbf{propmpt} (input dell'utente).

\end{document}
