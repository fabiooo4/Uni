\documentclass[a4paper]{article}
\usepackage{import}
\usepackage[utf8]{inputenc}
\usepackage[T1]{fontenc}
\usepackage{textcomp}
\usepackage[italian]{babel}
\usepackage{amsmath, amssymb}
\usepackage{booktabs,xltabular}
\usepackage{amsfonts}
\usepackage{amsthm}
\usepackage{cancel}
\usepackage{mdframed}
\usepackage{makecell}
\usepackage{float}
\usepackage{xcolor}
\usepackage{listings}
\usepackage{gensymb}
\usepackage{graphicx}
\usepackage{bodeplot}
\usepackage{tikz}
\usetikzlibrary{shapes, arrows, automata, petri, decorations.markings, decorations.pathreplacing, positioning, calc}
\usepackage{circuitikz}
\usepackage[label=corner]{karnaugh-map}
\graphicspath{{./figures/}}

% Set default font to sans-serif
\renewcommand{\familydefault}{\sfdefault} 
\usepackage{eulervm}

\usepackage{forest}

\usepackage{mathtools}
\DeclarePairedDelimiter\ceil{\lceil}{\rceil}
\DeclarePairedDelimiter\floor{\lfloor}{\rfloor}

% \usepackage{ntheorem}

\usepackage{import}
\usepackage{pdfpages}
\usepackage{transparent}
\usepackage{xcolor}

\usepackage{hyperref}
\hypersetup{
    colorlinks=false,
}

% Code blocks
\definecolor{codegreen}{rgb}{0,0.6,0}
\definecolor{codegray}{rgb}{0.5,0.5,0.5}
\definecolor{codepurple}{rgb}{0.58,0,0.82}
\definecolor{backcolour}{rgb}{0.95,0.95,0.95}

\lstdefinestyle{mystyle}{
	backgroundcolor=\color{backcolour},
	commentstyle=\color{codegreen},
	keywordstyle=\color{magenta},
	numberstyle=\tiny\color{codegray},
	stringstyle=\color{codepurple},
	basicstyle=\ttfamily\footnotesize,
	breakatwhitespace=false,
	breaklines=true,
	captionpos=b,
	keepspaces=true,
	numbers=left,
	numbersep=5pt,
	showspaces=false,
	showstringspaces=false,
	showtabs=false,
	tabsize=2
}

\lstset{style=mystyle}

\usepackage{color}
\usepackage{import}
\usepackage{pdfpages}
\usepackage{transparent}
\usepackage{xcolor}

% Example frame
\theoremstyle{definition}
\newmdtheoremenv[%
	linecolor=gray,leftmargin=0,%
	rightmargin=0,
	innertopmargin=8pt,%
	innerbottommargin=8pt,
	ntheorem]{example}{Esempio}[section]

% Important definition frame
\theoremstyle{definition}
\newmdtheoremenv[%
	linecolor=gray,leftmargin=0,%
	rightmargin=0,
	backgroundcolor=gray!40,%
	innertopmargin=8pt,%
	innerbottommargin=8pt,
	ntheorem]{definition}{Definizione}[section]

% Exercise frame
\theoremstyle{definition}
\newmdtheoremenv[%
	linecolor=gray,leftmargin=0,%
	rightmargin=0,
	innertopmargin=8pt,%
	innerbottommargin=8pt,
	ntheorem]{exercise}{Esercizio}[section]

% Theorem frame
\theoremstyle{definition}
\newmdtheoremenv[%
  linecolor=gray,leftmargin=0,%
  rightmargin=0,
  innertopmargin=8pt,%
  innerbottommargin=8pt,
  ntheorem]{theorem}{Teorema}[section]

\theoremstyle{definition}
\newmdtheoremenv[%
  linecolor=gray,leftmargin=0,%
  rightmargin=0,
  innertopmargin=8pt,%
  innerbottommargin=8pt,
  ntheorem]{define}{Definizione utile}[section]

% figure support
\usepackage{import}
\usepackage{xifthen}
\pdfminorversion=7
\usepackage{pdfpages}
\usepackage{transparent}
\newcommand{\incfig}[1]{%
	\def\svgwidth{\columnwidth}
	\import{./figures/}{#1.pdf_tex}
}

% FSM tikz
\tikzset{
    place/.style={
        circle,
        thick,
        draw=black,
        minimum size=6mm,
    },
        state/.style={
        circle,
        thick,
        draw=blue!75,
        fill=blue!20,
        minimum size=6mm,
    },
}

\usepackage{pgfplots}
\pgfplotsset{compat=1.18}

\pdfsuppresswarningpagegroup=1


\begin{document}

\begin{titlepage}
	\begin{center}
		\vspace*{1cm}

		\Huge
		\textbf{Analisi 1}

		\vspace{0.5cm}
		\LARGE
		UniVR - Dipartimento di Informatica

		\vspace{1.5cm}

		\textbf{Fabio Irimie}

		\vfill


		\vspace{0.8cm}

    Corso di Giacomo Canevari

		1° Semestre 2023/2024

	\end{center}
\end{titlepage}


\tableofcontents
\pagebreak

\section{Introduzione}
\subsection{Cos'è l'informatica?}
È una scienza che studia la calcolabilità, cioè cerca di capire che problemi si possono
risolvere con un programma. Nasce dall'unione di matematica, ingegneria e logica. Il
computer è solo uno strumento, mentre la matematica è il linguaggio con cui si
creano algoritmi che permettono di risolvere i problemi.

\subsection{Origini dell'informatica}
Hilbert, nel 1900, si pose l'obiettivo di formalizzare tutta la matematica con un insieme
finito e non contraddittorio di assiomi. Nel 1931, invece, Gödel dimostrò che l'informatica
non potrà mai rappresentare tutta la matematica, perché ci saranno sempre proposizioni
vere ma non dimostrabili tramite il calcolo. Ci si iniziò a chiedere se esistessero
modelli di calcolo meccanici in grado di risolvere tutti i problemi. Nel 1936, Turing
propose la macchina di Turing, una \textbf{sola} macchina programmabile in grado di
risolvere tutti i problemi risolvibili:
\[
  Int(P,x) = \begin{cases}
    P(x) & \text{se } P(x) \text{ termina}\\
    \uparrow & \text{se } P(x) \text{ non termina}
  \end{cases}
\] 
dove $P$ è un programma e $x$ è un input. La macchina di Turing è un modello teorico
di calcolatore, che non esiste fisicamente, ma è in grado di simulare qualsiasi altro
calcolatore. Da questo modello deriva la concezione di calcolabilità, cioè se un problema
è intuitivamente calcolabile, allora esiste un programma in grado di risolverlo.

Altri modelli di calcolo che sono stati proposti sono:
\begin{itemize}
  \item Lambda-calcolo
  \item Funzioni ricorsive
  \item Linguaggi di programmazione (Turing-completi)
\end{itemize}

\begin{define}
  La Turing-completezza è la proprietà di un linguaggio di programmazione di essere
  in grado di simulare una macchina di Turing, cioè di poter risolvere qualsiasi problema
  risolvibile.
\end{define}

\subsubsection{Calcolabilità}
Un programma è calcolabile se termina, ma non è detto che termini in un tempo ragionevole.
Non esistono algoritmi che possono dire se un programma termina o meno. Questo è un esempio
di problema non calcolabile.

I problemi non calcolabili sono infinitamente più numerosi di quelli calcolabili

\end{document}
