\documentclass[a4paper]{article}

\usepackage[utf8]{inputenc}
\usepackage[T1]{fontenc}
\usepackage{textcomp}
\usepackage[italian]{babel}
\usepackage{amsmath, amssymb}
\usepackage{booktabs,xltabular}
\usepackage{amsfonts}
\usepackage{cancel}
\usepackage{mdframed}
\usepackage{makecell}
\usepackage{float}
\usepackage{xcolor}
\usepackage{listings}
\usepackage{graphicx}
\usepackage{tikz}
\usetikzlibrary{shapes, arrows, automata, petri, decorations.pathreplacing, positioning, calc}
\usepackage{circuitikz}
\usepackage[label=corner]{karnaugh-map}
\graphicspath{{./figures/}}

\usepackage{ntheorem}
\newtheorem{theorem}{Teorema}

\usepackage{import}
\usepackage{pdfpages}
\usepackage{transparent}
\usepackage{xcolor}

\usepackage{hyperref}
\hypersetup{
    colorlinks=false,
}

% Code blocks
\definecolor{codegreen}{rgb}{0,0.6,0}
\definecolor{codegray}{rgb}{0.5,0.5,0.5}
\definecolor{codepurple}{rgb}{0.58,0,0.82}
\definecolor{backcolour}{rgb}{0.95,0.95,0.95}

\lstdefinestyle{mystyle}{
	backgroundcolor=\color{backcolour},
	commentstyle=\color{codegreen},
	keywordstyle=\color{magenta},
	numberstyle=\tiny\color{codegray},
	stringstyle=\color{codepurple},
	basicstyle=\ttfamily\footnotesize,
	breakatwhitespace=false,
	breaklines=true,
	captionpos=b,
	keepspaces=true,
	numbers=left,
	numbersep=5pt,
	showspaces=false,
	showstringspaces=false,
	showtabs=false,
	tabsize=2
}

\lstset{style=mystyle}

\usepackage{color}

\definecolor{dkgreen}{rgb}{0,0.6,0}
\definecolor{gray}{rgb}{0.5,0.5,0.5}
\definecolor{mauve}{rgb}{0.58,0,0.82}

\lstset{frame=tb,
	aboveskip=3mm,
	belowskip=3mm,
	showstringspaces=false,
	columns=flexible,
	basicstyle={\small\ttfamily},
	numbers=none,
	numberstyle=\tiny\color{gray},
	keywordstyle=\color{blue},
	commentstyle=\color{dkgreen},
	stringstyle=\color{mauve},
	breaklines=true,
	breakatwhitespace=true,
	tabsize=3
}

\usepackage{import}
\usepackage{pdfpages}
\usepackage{transparent}
\usepackage{xcolor}



% Useful definitions frame
\theoremstyle{break}
\theoremheaderfont{\bfseries}
\newmdtheoremenv[%
	linecolor=gray,leftmargin=0,%
	rightmargin=0,
	innertopmargin=8pt,%
	innerbottommargin=8pt,
	ntheorem]{define}{Definizioni utili}[section]

% Example frame
\theoremstyle{break}
\theoremheaderfont{\bfseries}
\newmdtheoremenv[%
	linecolor=gray,leftmargin=0,%
	rightmargin=0,
	innertopmargin=8pt,%
	innerbottommargin=8pt,
	ntheorem]{example}{Esempio}[section]

% Important definition frame
\theoremstyle{break}
\theoremheaderfont{\bfseries}
\newmdtheoremenv[%
	linecolor=gray,leftmargin=0,%
	rightmargin=0,
	backgroundcolor=gray!40,%
	innertopmargin=8pt,%
	innerbottommargin=8pt,
	ntheorem]{definition}{Definizione}[section]

% Exercise frame
\theoremstyle{break}
\theoremheaderfont{\bfseries}
\newmdtheoremenv[%
	linecolor=gray,leftmargin=0,%
	rightmargin=0,
	innertopmargin=8pt,%
	innerbottommargin=8pt,
	ntheorem]{exercise}{Esercizio}[section]

% figure support
\usepackage{import}
\usepackage{xifthen}
\pdfminorversion=7
\usepackage{pdfpages}
\usepackage{transparent}
\newcommand{\incfig}[1]{%
	\def\svgwidth{\columnwidth}
	\import{./figures/}{#1.pdf_tex}
}

% FSM tikz
\tikzset{
    place/.style={
        circle,
        thick,
        draw=black,
        minimum size=6mm,
    },
        state/.style={
        circle,
        thick,
        draw=blue!75,
        fill=blue!20,
        minimum size=6mm,
    },
}

\pdfsuppresswarningpagegroup=1

\begin{document}

\begin{titlepage}
	\begin{center}
		\vspace*{1cm}

		\Huge
		\textbf{Analisi 1}

		\vspace{0.5cm}
		\LARGE
		UniVR - Dipartimento di Informatica

		\vspace{1.5cm}

		\textbf{Fabio Irimie}

		\vfill


		\vspace{0.8cm}

    Corso di Giacomo Canevari

		1° Semestre 2023/2024

	\end{center}
\end{titlepage}


\tableofcontents
\pagebreak

% Laboratorio
\section{Esercizi svolti}
\subsection{Esercizio 1}
Si consideri una CPU con una pipeline a 5 stadi (F, D, E, M, S). Si riporti nel seguente 
diagramma, per ogni istruzione, lo stadio della pipeline coinvolto in ogni istante di 
clock. Si ipotizzi la pipeline vuota al tempo 1.

\begin{table}[H]
  \centering
  \resizebox{\columnwidth}{!}{%
    \begin{tabular}{|l|c|c|c|c|c|c|c|c|c|c|c|}
      \hline
      Istruzione & 1 & 2 & 3 & 4 & 5 & 6 & 7 & 8 & 9 & 10 & 11 \\
      \hline
      addl \%eax, \%ebx & F & D & E & M & S & & & &  &  &   \\
      \hline
      movl \$4, \%ecx & & F & D & E & M & S & & & &  &  \\
      \hline 
      subl \%ebx, \%ecx & & & F & D & D & D & D & M & S & &  \\
      \hline
      movl \$4, \%edx & & & & F & F & F & F & D & E & M & S \\
      \hline
    \end{tabular}%
  }
\end{table}

\subsection{Esercizio 2}
Si consideri una CPU con una pipeline a 5 stadi (F, D, E, M, S). Si riporti nel seguente diagramma,
per ogni istruzione, lo stadio della pipeline coinvolto in ogni istante di clock. Si ipotizzi la pipeline
vuota al tempo 1.

\begin{table}[H]
  \centering
  \resizebox{\columnwidth}{!}{%
    \begin{tabular}{|l|c|c|c|c|c|c|c|c|c|c|c|}
      \hline
      Istruzione               & 1 & 2 & 3 & 4 & 5 & 6 & 7 & 8 & 9 & 10 & 11  \\
      \hline
      ciclo: addl \%eax, \%ebx & F & D & E & M & S &   & F & D & D & E  & M  \\
      \hline
      movl \$4, \%ecx          &   & F & D & E & M & S &   & F & F & D  & E  \\
      \hline 
      subl \%eax, \%edx        &   &   & F & D & E & M & S &   &   & F  & D  \\
      \hline
      movl \$6, \%ebx          &   &   &   & F & D & E & M & S &   &    & F  \\
      \hline
      jmp ciclo                &   &   &   &   & F & D & E & M & S &    &    \\
      \hline
    \end{tabular}%
  }
\end{table}

\subsection{Esercizio 3}
Si consideri una CPU con una pipeline a 5 stadi (F, D, E, M, S). Si riporti nel seguente diagramma,
per ogni istruzione, lo stadio della pipeline coinvolto in ogni istante di clock. Si ipotizzi la pipeline
vuota al tempo 1. Si ipotizzi che il salto avvenga. Si ignorino le tecniche del Delay Slot e della
Branch Prediction. I commenti \#yes e \#no indicano se il salto avviene o meno.

\begin{table}[H]
  \centering
  \resizebox{\columnwidth}{!}{%
    \begin{tabular}{|l|c|c|c|c|c|c|c|c|c|c|c|c|c|}
      \hline
      Istruzione         & 1 & 2 & 3 & 4 & 5 & 6 & 7 & 8 & 9 & 10 & 11 & 12 & 13  \\
      \hline
      inizio: inc \%ebx  & F & D & E & M & S &   &   &   & F & D  & E  & M  & S   \\
      \hline
      movl \%ecx, \%edx  &   & F & D & E & M & S &   &   &   & F  & D  & E  & M   \\
      \hline 
      cmpl \%eax, 0x86FF &   &   & F & D & E & M & S &   &   &    & F  & D  & E   \\
      \hline
      jne inizio \#yes   &   &   &   & F & D & D & D & D & E & M  & S  & F  & D   \\
      \hline
      movl \%ecx, \%edx  &   &   &   &   & F & F & F & F &   &    &    &    & F   \\
      \hline
    \end{tabular}%
  }
\end{table}

\subsection{Esercizio 4}
Si consideri una CPU con una pipeline a 5 stadi (F, D, E, M, S). Si riporti nel seguente diagramma,
per ogni istruzione, lo stadio della pipeline coinvolto in ogni istante di clock. Si ipotizzi la pipeline
vuota al tempo 1 e che il salto non avvenga.

\begin{table}[H]
  \centering
  \resizebox{\columnwidth}{!}{%
    \begin{tabular}{|l|c|c|c|c|c|c|c|c|c|c|c|}
      \hline
      Istruzione               & 1 & 2 & 3 & 4 & 5 & 6 & 7 & 8 & 9 & 10 & 11 \\
      \hline
      START: subl \%eax, \%ebx & F & D & E & M & S &   &   &   &   &    &    \\
      \hline
      jz START \#no            &   & F & D & D & D & D & E & M & S &    &    \\
      \hline 
      subl \%ebx, \%ecx        &   &   & F & F & F & F & D & E & M & S  &    \\
      \hline
      movl \%edx, \%eax        &   &   &   &   &   &   & F & D & E & M  & S  \\
      \hline
    \end{tabular}%
  }
\end{table}

\subsection{Esercizio 5}
Si consideri una CPU con una pipeline a 5 stadi (F, D, E, M, S). Si riporti nel seguente diagramma,
per ogni istruzione, lo stadio della pipeline coinvolto in ogni istante di clock. Si ipotizzi la pipeline
vuota al tempo 1 e che il salto non avvenga.

\begin{table}[H]
  \centering
  \resizebox{\columnwidth}{!}{%
    \begin{tabular}{|l|c|c|c|c|c|c|c|c|c|c|c|c|c|c|c|}
      \hline
      Istruzione               & 1 & 2 & 3 & 4 & 5 & 6 & 7 & 8 & 9 & 10 & 11 & 12 & 13 & 14 & 15 \\
      \hline
      ciclo: addl \%eax, \%ebx & F & D & E & M & S &   &   &   &   &    &    &    &    &    &    \\
      \hline
      movl \%edx, \%ecx        &   & F & D & E & M & S &   &   &   &    &    &    &    &    &    \\
      \hline 
      subl \%ebx, \%ecx        &   &   & F & D & D & D & D & E & M & S  &    &    &    &    &    \\
      \hline
      jz ciclo \#no            &   &   &   & F & F & F & F & D & D & D  & D  & E  & M  & S  &    \\
      \hline
      movl \%ecx, \%edx        &   &   &   &   &   &   &   & F & F & F  & F  & D  & E  & M  & S  \\
      \hline
    \end{tabular}%
  }
\end{table}

\subsection{Esercizio 6}
Si consideri una CPU con una pipeline a 4 stadi (F, D, E, W). Si riporti nel seguente diagramma, per
ogni istruzione, lo stadio della pipeline coinvolto in ogni istante di clock. Si ipotizzi che la pipeline
sia vuota al tempo 1 e che jz faccia riferimento all’istruzione subl.

\begin{table}[H]
  \centering
  \resizebox{\columnwidth}{!}{%
    \begin{tabular}{|l|c|c|c|c|c|c|c|c|c|c|c|c|c|}
      \hline
      Istruzione               & 1 & 2 & 3 & 4 & 5 & 6 & 7 & 8 & 9 & 10 & 11 & 12 & 13 \\
      \hline
      ciclo: addl \%eax, \%ebx & F & D & E & W &   &   &   &   &   &    &    &    &    \\
      \hline
      movl \%ebx, \%ecx        &   & F & D & D & D & E & W &   &   &    &    &    &    \\
      \hline 
      subl \%eax, \%ecx        &   &   & F & F & F & D & D & D & E & W  &    &    &    \\
      \hline
      jz ciclo \#no            &   &   &   &   &   & F & F & F & D & D  & D  & E  & W  \\
      \hline
    \end{tabular}%
  }
\end{table}

\subsection{Esercizio 7}
Si consideri una CPU con una pipeline a 5 stadi (F, D, E, M, S). Si riporti nel seguente diagramma,
per ogni istruzione, lo stadio della pipeline coinvolto in ogni istante di clock. Si ipotizzi la pipeline
vuota al tempo 1 e si facciano le opportune ipotesi sul salto condizionale.

\noindent Nel caso in cui il salto non viene effettuato il diagramma sarà il seguente:
\begin{table}[H]
  \centering
  \resizebox{\columnwidth}{!}{%
    \begin{tabular}{|l|c|c|c|c|c|c|c|c|c|c|c|c|c|c|c|}
      \hline
      Istruzione              & 1 & 2 & 3 & 4 & 5 & 6 & 7 & 8 & 9 & 10 & 11 & 12 & 13 & 14 & 15 \\
      \hline
      init: movl \%ecx, \%edx & F & D & E & M & S &   &   &   &   &    &    &    &    &    &    \\
      \hline
      addl \$4, \%ebx         &   & F & D & E & M & S &   &   &   &    &    &    &    &    &    \\
      \hline 
      cmpl 0x319FA, \%ebx     &   &   & F & D & D & D & D & E & M & S  &    &    &    &    &    \\
      \hline
      jnz init \#no           &   &   &   & F & F & F & F & D & D & D  & D  & E  & M  & S  &    \\
      \hline
      addl \%eax, \%ecx       &   &   &   &   &   &   &   & F & F & F  & F  & D  & E  & M  & S  \\
      \hline
    \end{tabular}%
  }
\end{table}

\noindent Nel caso in cui il salto viene effettuato il diagramma sarà il seguente:
\begin{table}[H]
  \centering
  \resizebox{\columnwidth}{!}{%
    \begin{tabular}{|l|c|c|c|c|c|c|c|c|c|c|c|c|c|c|c|c|c|c|}
      \hline
      Istruzione              & 1 & 2 & 3 & 4 & 5 & 6 & 7 & 8 & 9 & 10 & 11 & 12 & 13 & 14 & 15 & 16 & 17 & 18 \\
      \hline
      init: movl \%ecx, \%edx & F & D & E & M & S &   &   &   &   &    &    & F  & D  & E  & M  & S  &    &    \\
      \hline
      addl \$4, \%ebx         &   & F & D & E & M & S &   &   &   &    &    &    & F  & D  & E  & M  & S  &    \\
      \hline 
      cmpl 0x319FA, \%ebx     &   &   & F & D & D & D & D & E & M & S  &    &    &    & F  & D  & D  & D  & D  \\
      \hline
      jnz init \#yes          &   &   &   & F & F & F & F & D & D & D  & D  & E  & M  & S  & F  & F  & F  & F  \\
      \hline
      addl \%eax, \%ecx       &   &   &   &   &   &   &   & F & F & F  & F  &    &    &    &    &    &    &    \\
      \hline
    \end{tabular}%
  }
\end{table}

\section{Esercizi moodle}
\subsection{Esercizio 1}
Una CPU con una pipeline a 2 stadi viene sostituita con una CPU con una pipeline a 4 stadi.
Se il tempo totale di esecuzione di una singola istruzione è rimasto invariato, qual è il minimo ed il massimo incremento delle prestazioni che si può attendere?

\vspace{1em}
\noindent Il massimo di incremento delle prestazioni è di 2 perchè raddoppiando il numero
di stadi si raddoppiano le prestazioni. Il minimo di incremento delle prestazioni è di 1,
quindi non si hanno miglioramenti.

\subsection{Esercizio 2}

Per il seguente esercizio che coinvolge 4 istruzioni successive eseguite da una CPU con pipeline a 4 stadi, individuare lo stadio in cui si trova ogni istruzione ad ogni ciclo macchina (dove qui consideriamo un massimo di 11 cicli coinvolti).
Dopodichè ripetere gli esercizi introducendo le migliorie architetturali discusse a lezione al fine di ridurre gli stalli nella pipeline.
Per migliorie architetturali parliamo di:

\begin{itemize}
  \item 
    Inoltro di operandi (sia per operazioni logico-aritmetiche che di memoria)
  \item 
    Branch prediction (o delay slot)
\end{itemize}


\noindent L'inoltro può avvenire su istruzioni a distanza maggiore di 1, naturalmente. E' evidente che la dipendenza di dati si può avere, in questa pipeline a 4 stadi, tra istruzioni fino a distanza 4.
Ognuno di questi casi va identificato (in software o hardware) e prevenuto/risolto onde evitare esecuzione scorretta.

\begin{table}[H]
  \centering
  \resizebox{\columnwidth}{!}{%
    \begin{tabular}{|l|c|c|c|c|c|c|c|c|c|c|c|}
      \hline
      Istruzione               & 1 & 2 & 3 & 4 & 5 & 6 & 7 & 8 & 9 & 10 & 11 \\
      \hline
      loop: addl \%eax, \%ebx  & F & D & E & W &   &   &   &   &   &    &    \\
      \hline
      movl ind \%ecx           &   & F & D & E & W &   &   &   &   &    &    \\
      \hline 
      subl \%ebx, \%ecx        &   &   & F & D & D & D & E & W &   &    &    \\
      \hline
      jz loop                  &   &   &   & F & F & F & D & D & D & E  & W  \\
      \hline
    \end{tabular}%
  }
\end{table}

\vspace{1em}
\noindent Ottimizzando con la branch prediction si ha:
\begin{table}[H]
  \centering
  \resizebox{\columnwidth}{!}{%
    \begin{tabular}{|l|c|c|c|c|c|c|c|c|c|c|c|}
      \hline
      Istruzione               & 1 & 2 & 3 & 4 & 5 & 6 & 7 & 8 & 9 & 10 & 11 \\
      \hline
      loop: addl \%eax, \%ebx  & F & D & E & W &   &   &   &   & F & D  & E  \\
      \hline
      movl ind \%ecx           &   & F & D & E & W &   &   &   &   & F  & D  \\
      \hline 
      subl \%ebx, \%ecx        &   &   & F & D & D & D & E & W &   &    & F  \\
      \hline
      jz loop                  &   &   &   & F & F & F & D & D & D & E  & W  \\
      \hline
    \end{tabular}%
  }
\end{table}

\subsection{Esercizio 3}
Per il seguente esercizio che coinvolge 4 istruzioni successive eseguite da una CPU con pipeline a 4 stadi, individuare lo stadio in cui si trova ogni istruzione ad ogni ciclo macchina (consideriamo un massimo di 11 cicli coinvolti).
Dopodichè ripetere gli esercizi introducendo le migliorie architetturali discusse a lezione al fine di ridurre gli stalli nella pipeline.


\vspace{1em}
\noindent Nel caso in cui il ciclo non viene effettuato si ha:
\begin{table}[H]
  \centering
  \resizebox{\columnwidth}{!}{%
    \begin{tabular}{|l|c|c|c|c|c|c|c|c|c|c|c|}
      \hline
      Istruzione                & 1 & 2 & 3 & 4 & 5 & 6 & 7 & 8 & 9 & 10 & 11 \\
      \hline
      inizio: mull \%eax, \%ebx & F & D & E & W &   &   &   &   &   &    & F  \\
      \hline
      movl \%ecx ind            &   & F & D & E & W &   &   &   &   &    &    \\
      \hline 
      cmpl \%ebx, 0h            &   &   & F & D & D & E & W &   &   &    &    \\
      \hline
      jnz inizio                &   &   &   & F & F & D & D & D & E & W  &    \\
      \hline
    \end{tabular}%
  }
\end{table}

\vspace{1em}
\noindent Ottimizzando con branch prediction si ha:
\begin{table}[H]
  \centering
  \resizebox{\columnwidth}{!}{%
    \begin{tabular}{|l|c|c|c|c|c|c|c|c|c|c|c|}
      \hline
      Istruzione                & 1 & 2 & 3 & 4 & 5 & 6 & 7 & 8 & 9 & 10 & 11 \\
      \hline
      inizio: mull \%eax, \%ebx & F & D & E & W &   &   &   & F & D & E  & W  \\
      \hline
      movl \%ecx ind            &   & F & D & E & W &   &   &   & F & D  & E  \\
      \hline 
      cmpl \%ebx, 0h            &   &   & F & D & D & E & W &   &   & F  & D  \\
      \hline
      jnz inizio                &   &   &   & F & F & D & D & D & E & W  & F  \\
      \hline
    \end{tabular}%
  }
\end{table}


\subsection{Esercizio 4}
Per il seguente esercizio che coinvolge 5 istruzioni successive eseguite da una CPU con pipeline a 4 stadi, individuare lo stadio in cui si trova ogni istruzione ad ogni ciclo macchina.
Dopodichè ripetere l'esercizio introducendo le migliorie a livello hardware o di compilatore discusse a lezione al fine di ridurre gli stalli nella pipeline.

\begin{table}[H]
  \centering
  \resizebox{\columnwidth}{!}{%
    \begin{tabular}{|l|c|c|c|c|c|c|c|c|c|c|c|c|c|c|}
      \hline
      Istruzione        & 1 & 2 & 3 & 4 & 5 & 6 & 7 & 8 & 9 & 10 & 11 & 12 & 13 & 14 \\
      \hline
      subl \$2, \%ebx   & F & D & E & W &   &   &   &   &   &    &    &    &    &   \\
      \hline
      movl ind, \%ecx   &   & F & D & E & W &   &   &   &   &    &    &    &    &   \\
      \hline 
      addl \%eax, \%ecx &   &   & F & D & D & D & E & W &   &    &    &    &    &   \\
      \hline
      movl ind, \%ebx   &   &   &   & F & F & F & D & E & W &    &    &    &    &   \\
      \hline
      movl \%ecx, ind   &   &   &   &   &   &   & F & D & D & E  & W  &    &    &   \\
      \hline
    \end{tabular}%
  }
\end{table}

\subsection{Esercizio 5}
Per il seguente esercizio che coinvolge 6 istruzioni eseguite da una CPU con pipeline a 4 stadi, individuare lo stadio in cui si trova ogni istruzione ad ogni ciclo macchina ipotizzando di predire che il salto non avvenga e che tale predizione sia confermata. Si assume che l'istruzione successiva al salto sia conosciuta al ciclo immediatamente successivo (perchè usiamo predizione statica oppure perchè l'istruzione di salto si trova nel buffer di predizione), nonchè che la predizione risulti corretta.
Dopodichè ripetere l'esercizio introducendo le migliorie a livello hardware o di compilatore discusse a lezione al fine di ridurre gli stalli nella pipeline.

\begin{table}[H]
  \centering
  \resizebox{\columnwidth}{!}{%
    \begin{tabular}{|l|c|c|c|c|c|c|c|c|c|c|c|c|c|c|}
      \hline
      Istruzione            & 1 & 2 & 3 & 4 & 5 & 6 & 7 & 8 & 9 & 10 & 11 & 12 & 13 & 14 \\
      \hline
      j1: addl \%ebx, \%ecx & F & D & E & W &   &   &   &   &   &    &    &    &    &   \\
      \hline
      movl ind, \%ebx       &   & F & D & E & W &   &   &   &   &    &    &    &    &   \\
      \hline 
      cmpl \%ebx, \%ecx     &   &   & F & D & D & D & E & W &   &    &    &    &    &   \\
      \hline
      movl ind, \%eax       &   &   &   & F & F & F & D & E & W &    &    &    &    &   \\
      \hline
      jnz j1 \#no           &   &   &   &   &   &   & F & D & D & E  & W  &    &    &   \\
      \hline
      subl \%ebx, \%eax     &   &   &   &   &   &   &   & F & F & D  & E  & W  &    &   \\
      \hline
    \end{tabular}%
  }
\end{table}

\vspace{1em}
\noindent Ottimizzando con branch prediction si ha che viene effettuata la fetch
dell'istruzione all'etichetta j1, ma poi viene scartata perchè il salto non viene
effettuato. Si ha quindi:
\begin{table}[H]
  \centering
  \resizebox{\columnwidth}{!}{%
    \begin{tabular}{|l|c|c|c|c|c|c|c|c|c|c|c|c|c|c|}
      \hline
      Istruzione            & 1 & 2 & 3 & 4 & 5 & 6 & 7 & 8 & 9 & 10 & 11 & 12 & 13 & 14 \\
      \hline
      j1: addl \%ebx, \%ecx & F & D & E & W &   &   &   &   & F & F  & F  &    &    &   \\
      \hline
      movl ind, \%ebx       &   & F & D & E & W &   &   &   &   &    &    &    &    &   \\
      \hline 
      cmpl \%ebx, \%ecx     &   &   & F & D & D & D & E & W &   &    &    &    &    &   \\
      \hline
      movl ind, \%eax       &   &   &   & F & F & F & D & E & W &    &    &    &    &   \\
      \hline
      jnz j1 \#no           &   &   &   &   &   &   & F & D & D & E  & W  &    &    &   \\
      \hline
      subl \%ebx, \%eax     &   &   &   &   &   &   &   & F & F & D  & E  & W  &    &   \\
      \hline
    \end{tabular}%
  }
\end{table}


\end{document}
