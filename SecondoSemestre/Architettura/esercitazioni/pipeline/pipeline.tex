\documentclass[a4paper]{article}

\usepackage[utf8]{inputenc}
\usepackage[T1]{fontenc}
\usepackage{textcomp}
\usepackage[italian]{babel}
\usepackage{amsmath, amssymb}
\usepackage{booktabs,xltabular}
\usepackage{amsfonts}
\usepackage{cancel}
\usepackage{mdframed}
\usepackage{makecell}
\usepackage{float}
\usepackage{xcolor}
\usepackage{listings}
\usepackage{graphicx}
\usepackage{tikz}
\usetikzlibrary{shapes, arrows, automata, petri, decorations.pathreplacing, positioning, calc}
\usepackage{circuitikz}
\usepackage[label=corner]{karnaugh-map}
\graphicspath{{./figures/}}

\usepackage{ntheorem}
\newtheorem{theorem}{Teorema}

\usepackage{import}
\usepackage{pdfpages}
\usepackage{transparent}
\usepackage{xcolor}

\usepackage{hyperref}
\hypersetup{
    colorlinks=false,
}

% Code blocks
\definecolor{codegreen}{rgb}{0,0.6,0}
\definecolor{codegray}{rgb}{0.5,0.5,0.5}
\definecolor{codepurple}{rgb}{0.58,0,0.82}
\definecolor{backcolour}{rgb}{0.95,0.95,0.95}

\lstdefinestyle{mystyle}{
	backgroundcolor=\color{backcolour},
	commentstyle=\color{codegreen},
	keywordstyle=\color{magenta},
	numberstyle=\tiny\color{codegray},
	stringstyle=\color{codepurple},
	basicstyle=\ttfamily\footnotesize,
	breakatwhitespace=false,
	breaklines=true,
	captionpos=b,
	keepspaces=true,
	numbers=left,
	numbersep=5pt,
	showspaces=false,
	showstringspaces=false,
	showtabs=false,
	tabsize=2
}

\lstset{style=mystyle}

\usepackage{color}

\definecolor{dkgreen}{rgb}{0,0.6,0}
\definecolor{gray}{rgb}{0.5,0.5,0.5}
\definecolor{mauve}{rgb}{0.58,0,0.82}

\lstset{frame=tb,
	aboveskip=3mm,
	belowskip=3mm,
	showstringspaces=false,
	columns=flexible,
	basicstyle={\small\ttfamily},
	numbers=none,
	numberstyle=\tiny\color{gray},
	keywordstyle=\color{blue},
	commentstyle=\color{dkgreen},
	stringstyle=\color{mauve},
	breaklines=true,
	breakatwhitespace=true,
	tabsize=3
}

\usepackage{import}
\usepackage{pdfpages}
\usepackage{transparent}
\usepackage{xcolor}



% Useful definitions frame
\theoremstyle{break}
\theoremheaderfont{\bfseries}
\newmdtheoremenv[%
	linecolor=gray,leftmargin=0,%
	rightmargin=0,
	innertopmargin=8pt,%
	innerbottommargin=8pt,
	ntheorem]{define}{Definizioni utili}[section]

% Example frame
\theoremstyle{break}
\theoremheaderfont{\bfseries}
\newmdtheoremenv[%
	linecolor=gray,leftmargin=0,%
	rightmargin=0,
	innertopmargin=8pt,%
	innerbottommargin=8pt,
	ntheorem]{example}{Esempio}[section]

% Important definition frame
\theoremstyle{break}
\theoremheaderfont{\bfseries}
\newmdtheoremenv[%
	linecolor=gray,leftmargin=0,%
	rightmargin=0,
	backgroundcolor=gray!40,%
	innertopmargin=8pt,%
	innerbottommargin=8pt,
	ntheorem]{definition}{Definizione}[section]

% Exercise frame
\theoremstyle{break}
\theoremheaderfont{\bfseries}
\newmdtheoremenv[%
	linecolor=gray,leftmargin=0,%
	rightmargin=0,
	innertopmargin=8pt,%
	innerbottommargin=8pt,
	ntheorem]{exercise}{Esercizio}[section]

% figure support
\usepackage{import}
\usepackage{xifthen}
\pdfminorversion=7
\usepackage{pdfpages}
\usepackage{transparent}
\newcommand{\incfig}[1]{%
	\def\svgwidth{\columnwidth}
	\import{./figures/}{#1.pdf_tex}
}

% FSM tikz
\tikzset{
    place/.style={
        circle,
        thick,
        draw=black,
        minimum size=6mm,
    },
        state/.style={
        circle,
        thick,
        draw=blue!75,
        fill=blue!20,
        minimum size=6mm,
    },
}

\pdfsuppresswarningpagegroup=1

\begin{document}

\begin{titlepage}
	\begin{center}
		\vspace*{1cm}

		\Huge
		\textbf{Probabilità e Statistica\\Esercizi}

		\vspace{0.5cm}
		\LARGE
		UniVR - Dipartimento di Informatica

		\vspace{1.5cm}

		\textbf{Fabio Irimie}

		\vfill


		\vspace{0.8cm}


		2° Semestre 2023/2024

	\end{center}
\end{titlepage}


\tableofcontents
\pagebreak

% Laboratorio
\section{Pipeline}
\subsection{Esercizio 1}
Si consideri una CPU con una pipeline a 5 stadi (F, D, E, M, S). Si riporti nel seguente 
diagramma, per ogni istruzione, lo stadio della pipeline coinvolto in ogni istante di 
clock. Si ipotizzi la pipeline vuota al tempo 1.

\begin{table}[H]
  \centering
  \resizebox{\columnwidth}{!}{%
    \begin{tabular}{|l|c|c|c|c|c|c|c|c|c|c|c|}
      \hline
      Istruzione & 1 & 2 & 3 & 4 & 5 & 6 & 7 & 8 & 9 & 10 & 11 \\
      \hline
      addl \%eax, \%ebx & F & D & E & M & S & & & &  &  &   \\
      \hline
      movl \$4, \%ecx & & F & D & E & M & S & & & &  &  \\
      \hline 
      subl \%ebx, \%ecx & & & F & D & D & D & D & M & S & &  \\
      \hline
      movl \$4, \%edx & & & & F & F & F & F & D & E & M & S \\
      \hline
    \end{tabular}%
  }
\end{table}

\subsection{Esercizio 2}
Si consideri una CPU con una pipeline a 5 stadi (F, D, E, M, S). Si riporti nel seguente diagramma,
per ogni istruzione, lo stadio della pipeline coinvolto in ogni istante di clock. Si ipotizzi la pipeline
vuota al tempo 1.

\begin{table}[H]
  \centering
  \resizebox{\columnwidth}{!}{%
    \begin{tabular}{|l|c|c|c|c|c|c|c|c|c|c|c|}
      \hline
      Istruzione               & 1 & 2 & 3 & 4 & 5 & 6 & 7 & 8 & 9 & 10 & 11  \\
      \hline
      ciclo: addl \%eax, \%ebx & F & D & E & M & S &   & F & D & D & E  & M  \\
      \hline
      movl \$4, \%ecx          &   & F & D & E & M & S &   & F & F & D  & E  \\
      \hline 
      subl \%eax, \%edx        &   &   & F & D & E & M & S &   &   & F  & D  \\
      \hline
      movl \$6, \%ebx          &   &   &   & F & D & E & M & S &   &    & F  \\
      \hline
      jmp ciclo                &   &   &   &   & F & D & E & M & S &    &    \\
      \hline
    \end{tabular}%
  }
\end{table}

\subsection{Esercizio 3}
Si consideri una CPU con una pipeline a 5 stadi (F, D, E, M, S). Si riporti nel seguente diagramma,
per ogni istruzione, lo stadio della pipeline coinvolto in ogni istante di clock. Si ipotizzi la pipeline
vuota al tempo 1. Si ipotizzi che il salto avvenga. Si ignorino le tecniche del Delay Slot e della
Branch Prediction. I commenti \#yes e \#no indicano se il salto avviene o meno.

\begin{table}[H]
  \centering
  \resizebox{\columnwidth}{!}{%
    \begin{tabular}{|l|c|c|c|c|c|c|c|c|c|c|c|c|c|}
      \hline
      Istruzione         & 1 & 2 & 3 & 4 & 5 & 6 & 7 & 8 & 9 & 10 & 11 & 12 & 13  \\
      \hline
      inizio: inc \%ebx  & F & D & E & M & S &   &   &   & F & D  & E  & M  & S   \\
      \hline
      movl \%ecx, \%edx  &   & F & D & E & M & S &   &   &   & F  & D  & E  & M   \\
      \hline 
      cmpl \%eax, 0x86FF &   &   & F & D & E & M & S &   &   &    & F  & D  & E   \\
      \hline
      jne inizio \#yes   &   &   &   & F & D & D & D & D & E & M  & S  & F  & D   \\
      \hline
      movl \%ecx, \%edx  &   &   &   &   & F & F & F & F &   &    &    &    & F   \\
      \hline
    \end{tabular}%
  }
\end{table}

\subsection{Esercizio 4}
Si consideri una CPU con una pipeline a 5 stadi (F, D, E, M, S). Si riporti nel seguente diagramma,
per ogni istruzione, lo stadio della pipeline coinvolto in ogni istante di clock. Si ipotizzi la pipeline
vuota al tempo 1 e che il salto non avvenga.

\begin{table}[H]
  \centering
  \resizebox{\columnwidth}{!}{%
    \begin{tabular}{|l|c|c|c|c|c|c|c|c|c|c|c|}
      \hline
      Istruzione               & 1 & 2 & 3 & 4 & 5 & 6 & 7 & 8 & 9 & 10 & 11 \\
      \hline
      START: subl \%eax, \%ebx & F & D & E & M & S &   &   &   &   &    &    \\
      \hline
      jz START \#no            &   & F & D & D & D & D & E & M & S &    &    \\
      \hline 
      subl \%ebx, \%ecx        &   &   & F & F & F & F & D & E & M & S  &    \\
      \hline
      movl \%edx, \%eax        &   &   &   &   &   &   & F & D & E & M  & S  \\
      \hline
    \end{tabular}%
  }
\end{table}

\subsection{Esercizio 5}
Si consideri una CPU con una pipeline a 5 stadi (F, D, E, M, S). Si riporti nel seguente diagramma,
per ogni istruzione, lo stadio della pipeline coinvolto in ogni istante di clock. Si ipotizzi la pipeline
vuota al tempo 1 e che il salto non avvenga.

\begin{table}[H]
  \centering
  \resizebox{\columnwidth}{!}{%
    \begin{tabular}{|l|c|c|c|c|c|c|c|c|c|c|c|c|c|c|c|}
      \hline
      Istruzione               & 1 & 2 & 3 & 4 & 5 & 6 & 7 & 8 & 9 & 10 & 11 & 12 & 13 & 14 & 15 \\
      \hline
      ciclo: addl \%eax, \%ebx & F & D & E & M & S &   &   &   &   &    &    &    &    &    &    \\
      \hline
      movl \%edx, \%ecx        &   & F & D & E & M & S &   &   &   &    &    &    &    &    &    \\
      \hline 
      subl \%ebx, \%ecx        &   &   & F & D & D & D & D & E & M & S  &    &    &    &    &    \\
      \hline
      jz ciclo \#no            &   &   &   & F & F & F & F & D & D & D  & D  & E  & M  & S  &    \\
      \hline
      movl \%ecx, \%edx        &   &   &   &   &   &   &   & F & F & F  & F  & D  & E  & M  & S  \\
      \hline
    \end{tabular}%
  }
\end{table}

\subsection{Esercizio 6}
Si consideri una CPU con una pipeline a 4 stadi (F, D, E, W). Si riporti nel seguente diagramma, per
ogni istruzione, lo stadio della pipeline coinvolto in ogni istante di clock. Si ipotizzi che la pipeline
sia vuota al tempo 1 e che jz faccia riferimento all’istruzione subl.

\begin{table}[H]
  \centering
  \resizebox{\columnwidth}{!}{%
    \begin{tabular}{|l|c|c|c|c|c|c|c|c|c|c|c|c|c|}
      \hline
      Istruzione               & 1 & 2 & 3 & 4 & 5 & 6 & 7 & 8 & 9 & 10 & 11 & 12 & 13 \\
      \hline
      ciclo: addl \%eax, \%ebx & F & D & E & W &   &   &   &   &   &    &    &    &    \\
      \hline
      movl \%ebx, \%ecx        &   & F & D & D & D & E & W &   &   &    &    &    &    \\
      \hline 
      subl \%eax, \%ecx        &   &   & F & F & F & D & D & D & E & W  &    &    &    \\
      \hline
      jz ciclo \#no            &   &   &   &   &   & F & F & F & D & D  & D  & E  & W  \\
      \hline
    \end{tabular}%
  }
\end{table}

\subsection{Esercizio 7}
Si consideri una CPU con una pipeline a 5 stadi (F, D, E, M, S). Si riporti nel seguente diagramma,
per ogni istruzione, lo stadio della pipeline coinvolto in ogni istante di clock. Si ipotizzi la pipeline
vuota al tempo 1 e si facciano le opportune ipotesi sul salto condizionale.

\noindent Nel caso in cui il salto non viene effettuato il diagramma sarà il seguente:
\begin{table}[H]
  \centering
  \resizebox{\columnwidth}{!}{%
    \begin{tabular}{|l|c|c|c|c|c|c|c|c|c|c|c|c|c|c|c|}
      \hline
      Istruzione              & 1 & 2 & 3 & 4 & 5 & 6 & 7 & 8 & 9 & 10 & 11 & 12 & 13 & 14 & 15 \\
      \hline
      init: movl \%ecx, \%edx & F & D & E & M & S &   &   &   &   &    &    &    &    &    &    \\
      \hline
      addl \$4, \%ebx         &   & F & D & E & M & S &   &   &   &    &    &    &    &    &    \\
      \hline 
      cmpl 0x319FA, \%ebx     &   &   & F & D & D & D & D & E & M & S  &    &    &    &    &    \\
      \hline
      jnz init \#no           &   &   &   & F & F & F & F & D & D & D  & D  & E  & M  & S  &    \\
      \hline
      addl \%eax, \%ecx       &   &   &   &   &   &   &   & F & F & F  & F  & D  & E  & M  & S  \\
      \hline
    \end{tabular}%
  }
\end{table}

\noindent Nel caso in cui il salto viene effettuato il diagramma sarà il seguente:
\begin{table}[H]
  \centering
  \resizebox{\columnwidth}{!}{%
    \begin{tabular}{|l|c|c|c|c|c|c|c|c|c|c|c|c|c|c|c|c|c|c|}
      \hline
      Istruzione              & 1 & 2 & 3 & 4 & 5 & 6 & 7 & 8 & 9 & 10 & 11 & 12 & 13 & 14 & 15 & 16 & 17 & 18 \\
      \hline
      init: movl \%ecx, \%edx & F & D & E & M & S &   &   &   &   &    &    & F  & D  & E  & M  & S  &    &    \\
      \hline
      addl \$4, \%ebx         &   & F & D & E & M & S &   &   &   &    &    &    & F  & D  & E  & M  & S  &    \\
      \hline 
      cmpl 0x319FA, \%ebx     &   &   & F & D & D & D & D & E & M & S  &    &    &    & F  & D  & D  & D  & D  \\
      \hline
      jnz init \#yes          &   &   &   & F & F & F & F & D & D & D  & D  & E  & M  & S  & F  & F  & F  & F  \\
      \hline
      addl \%eax, \%ecx       &   &   &   &   &   &   &   & F & F & F  & F  &    &    &    &    &    &    &    \\
      \hline
    \end{tabular}%
  }
\end{table}

\end{document}
