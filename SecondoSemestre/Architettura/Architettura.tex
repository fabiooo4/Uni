\documentclass[a4paper]{article}

\usepackage[utf8]{inputenc}
\usepackage[T1]{fontenc}
\usepackage{textcomp}
\usepackage[italian]{babel}
\usepackage{amsmath, amssymb}
\usepackage{amsfonts}
\usepackage{mdframed}
\usepackage{float}
\usepackage{xcolor}
\usepackage{listings}
\usepackage{graphicx}
\usepackage{tikz}
\usetikzlibrary{shapes, arrows, automata, petri, positioning, calc}
\usepackage{circuitikz}
\usepackage[label=corner]{karnaugh-map}
\graphicspath{{./figures/}}

\usepackage{ntheorem}
\newtheorem{theorem}{Teorema}

\usepackage{import}
\usepackage{pdfpages}
\usepackage{transparent}
\usepackage{xcolor}

\usepackage{hyperref}
\hypersetup{
    colorlinks=false,
}

% Code blocks
\definecolor{codegreen}{rgb}{0,0.6,0}
\definecolor{codegray}{rgb}{0.5,0.5,0.5}
\definecolor{codepurple}{rgb}{0.58,0,0.82}
\definecolor{backcolour}{rgb}{0.95,0.95,0.95}

\lstdefinestyle{mystyle}{
	backgroundcolor=\color{backcolour},
	commentstyle=\color{codegreen},
	keywordstyle=\color{magenta},
	numberstyle=\tiny\color{codegray},
	stringstyle=\color{codepurple},
	basicstyle=\ttfamily\footnotesize,
	breakatwhitespace=false,
	breaklines=true,
	captionpos=b,
	keepspaces=true,
	numbers=left,
	numbersep=5pt,
	showspaces=false,
	showstringspaces=false,
	showtabs=false,
	tabsize=2
}

\lstset{style=mystyle}

\usepackage{color}

\definecolor{dkgreen}{rgb}{0,0.6,0}
\definecolor{gray}{rgb}{0.5,0.5,0.5}
\definecolor{mauve}{rgb}{0.58,0,0.82}

\lstset{frame=tb,
	aboveskip=3mm,
	belowskip=3mm,
	showstringspaces=false,
	columns=flexible,
	basicstyle={\small\ttfamily},
	numbers=none,
	numberstyle=\tiny\color{gray},
	keywordstyle=\color{blue},
	commentstyle=\color{dkgreen},
	stringstyle=\color{mauve},
	breaklines=true,
	breakatwhitespace=true,
	tabsize=3
}

\usepackage{import}
\usepackage{pdfpages}
\usepackage{transparent}
\usepackage{xcolor}



% Useful definitions frame
\theoremstyle{break}
\theoremheaderfont{\bfseries}
\newmdtheoremenv[%
	linecolor=gray,leftmargin=0,%
	rightmargin=0,
	innertopmargin=8pt,%
	innerbottommargin=8pt,
	ntheorem]{define}{Definizioni utili}[section]

% Example frame
\theoremstyle{break}
\theoremheaderfont{\bfseries}
\newmdtheoremenv[%
	linecolor=gray,leftmargin=0,%
	rightmargin=0,
	innertopmargin=8pt,%
	innerbottommargin=8pt,
	ntheorem]{example}{Esempio}[section]

% Important definition frame
\theoremstyle{break}
\theoremheaderfont{\bfseries}
\newmdtheoremenv[%
	linecolor=gray,leftmargin=0,%
	rightmargin=0,
	backgroundcolor=gray!40,%
	innertopmargin=8pt,%
	innerbottommargin=8pt,
	ntheorem]{definition}{Definizione}[section]

% Exercise frame
\theoremstyle{break}
\theoremheaderfont{\bfseries}
\newmdtheoremenv[%
	linecolor=gray,leftmargin=0,%
	rightmargin=0,
	innertopmargin=8pt,%
	innerbottommargin=8pt,
	ntheorem]{exercise}{Esercizio}[section]

% figure support
\usepackage{import}
\usepackage{xifthen}
\pdfminorversion=7
\usepackage{pdfpages}
\usepackage{transparent}
\newcommand{\incfig}[1]{%
	\def\svgwidth{\columnwidth}
	\import{./figures/}{#1.pdf_tex}
}

% FSM tikz
\tikzset{
    place/.style={
        circle,
        thick,
        draw=black,
        minimum size=6mm,
    },
        state/.style={
        circle,
        thick,
        draw=blue!75,
        fill=blue!20,
        minimum size=6mm,
    },
}

\pdfsuppresswarningpagegroup=1

\begin{document}
\begin{titlepage}
	\begin{center}
		\vspace*{1cm}

		\Huge
		\textbf{Probabilità e Statistica\\Esercizi}

		\vspace{0.5cm}
		\LARGE
		UniVR - Dipartimento di Informatica

		\vspace{1.5cm}

		\textbf{Fabio Irimie}

		\vfill


		\vspace{0.8cm}


		2° Semestre 2023/2024

	\end{center}
\end{titlepage}


\tableofcontents
\pagebreak

\section{Architettura di Von Neumann}
L'esigenza era quella di avere un'architettura che permettesse di eseguire programmi
in modo automatico, senza dover cambiare il cablaggio del calcolatore, quindi il circuito
deve essere abbastanza generale per poter eseguire programmi diversi.

\subsection{Struttura}
L'architettura di Von Neumann è composta da 5 parti principali:
\begin{itemize}
	\item \textbf{Unità aritmetico-logica}: si occupa di eseguire le operazioni aritmetiche e logiche
	\item \textbf{Unità di controllo}: si occupa di controllare il flusso delle istruzioni
	\item \textbf{Memoria}: contiene i dati e le istruzioni
	\item \textbf{Input/Output}: permette di comunicare con l'esterno
	\item \textbf{Bus}: permette di trasferire i dati tra la memoria e l'unità aritmetico-logica (generalmente in oro)
\end{itemize}

\begin{figure}[H]
	\centering
	\begin{tikzpicture}
		% CPU
		\node[draw, rectangle, minimum width=2cm, minimum height=1.5cm] (cpu) at (0, 0) {\textbf{CPU}};
		% Memoria
		\node[draw, rectangle, align=center, minimum width=2cm, minimum height=1.5cm] (memory) at (4, 0) {\textbf{Memoria}\\\footnotesize Dati\\\footnotesize Istruzioni};

		% Bus
		\draw (-2,-2) -- (6.5,-2) node[right] {Bus};
		\draw[<->] (cpu) -- ++(0,-2);
		\draw[<->] (memory) -- ++(0,-2);

		% Input/Output below bus
		\node[draw, rectangle, minimum width=0.7cm, minimum height=0.7cm] (io) at (2, -3) {\textbf{I/O}};
		\draw[<->] (io) -- (2,-2);

		% Keyboard and mouse below I/O
		\node (keyboard) at (1, -4) {Tastiera};
		\node (mouse) at (3, -4) {Mouse};
		\draw[->] (keyboard) -- (io);
		\draw[->] (mouse) -- (io);


	\end{tikzpicture}
\end{figure}

\subsection{Caratteristiche}
Le istruzioni hanno bisogno di un'operazione che permetta di effettuare dei salti,
in modo da poter implementare i cicli e le strutture di controllo. Inoltre le istruzioni
devono essere eseugite in sequenza (un'istruzione alla volta).

\subsection{CPU}
Ogni processore ha un'set di istruzioni diverso in base all'architettura,
questo set di istruzioni è chiamato \textbf{ISA} (Instruction Set Architecture).
\textbf{Assembly} è un linguaggio di programmazione che permette di scrivere programmi
in base all'ISA del processore e questo linguaggio viene tradotto in linguaggio binario
attraverso un \textbf{assembler}. In questo corso viene usata l'architettura x86 (80x86).

\subsubsection{Modello semplificato}
Per rappresentare il funzionamento di un processore si può usare un modello semplificato
rappresentato ad alto livello. Questo modello è composto da:
\begin{itemize}
	\item \textbf{Central Processing Unit (CPU)}: esegue le istruzioni
	\item \textbf{Control Unit (CU)}: controlla il flusso delle istruzioni
	\item \textbf{Bus Dati (BD)}: trasferisce i dati alla CPU
	\item \textbf{Bus Istruzioni (BI)}: trasferisce le istruzioni alla CPU
	\item \textbf{Bus di Controllo (BC)}: trasferisce i segnali di controllo
	\item \textbf{Memory Address Register (MAR)}: contiene l'indirizzo di memoria da leggere
	\item \textbf{Memory Data Register (MDR)}: contiene i dati letti dalla memoria
	\item \textbf{Program Counter (PC)}: tiene conto dell'indirizzo dell'istruzione da eseguire
	\item \textbf{Instruction Register (IR)}: contiene l'istruzione corrente
	\item \textbf{Program Status Word (PSW)}: contiene i flag del processore (es. zero, carry, overflow)
	\item \textbf{Register File}: contiene i registri del processore (es. EAX, EBX, ECX, EDX, ESI, EDI, ESP, EBP)
	\item \textbf{Arithmetic Logic Unit (ALU)}: esegue le operazioni aritmetiche e logiche
\end{itemize}
\begin{figure}[H]
	\centering
	\begin{tikzpicture}[scale=1.5]
		\draw (0,0) rectangle (4,4);
		\node[above left, scale=0.7] at (4,0) {CPU};

		\draw (2,2) rectangle (4,4);
		\node[above left, scale=0.7] at (3.4,2) {CU};

		\node[draw, rectangle, align=center, scale=0.8, minimum width=1cm, minimum height=0.8cm] (pc) at (2.4, 2.35) {PC};
		\node[draw, rectangle, align=center, scale=0.8, minimum width=1cm, minimum height=0.8cm] (ir) at (2.4, 2.95) {IR};
		\node[draw, rectangle, align=center, scale=0.8, minimum width=1cm, minimum height=0.5cm] (psw) at (3.4, 3.55) {PSW};

		% Bus Istruzioni
		\draw (-0.5,0) -- (-0.5,4) node[above, scale=0.7] {BI};

		% Bus Dati
		\draw (-1,0) -- (-1,4) node[above, scale=0.7] {BD};

		% Bus di controllo
		\draw (4.5,0) -- (4.5,4) node[above, scale=0.7] {BC};

		% MAR (Memory Address Register)
		\node[draw, rectangle, align=center, scale=0.8, minimum width=0.5cm, minimum height=0.5cm] (mar) at (0.5, 3) {M\\A\\R};
		\draw[latex-latex] (mar) -- (-0.5,3);

		% MDR (Memory Data Register)
		\node[draw, rectangle, align=center, scale=0.8, minimum width=0.5cm, minimum height=0.5cm] (mdr) at (0.5, 1) {M\\D\\R};
		\draw[latex-latex] (mdr) -- (-1,1);

		\draw[latex-latex] (4,3) -- (4.5,3);

		% Register file
		\draw (1,0.4) rectangle (2,1.6);
		\draw (1,0.6) -- (2,0.6);
		\draw (1,0.8) -- (2,0.8);
		\draw (1,1) -- (2,1);
		\draw (1,1.2) -- (2,1.2);
		\draw (1,1.4) -- (2,1.4);

		\node[below left, xshift=8, scale=0.7] at (1.82,0.4) {Register File};

		% ALU
		\draw (2.7,0.5) -- (3.3,0.5) -- (3.5,1.5) -- (3.2,1.5) -- (3,1.2) -- (2.8,1.5) -- (2.5,1.5) -- cycle;
		\node[above, scale=0.6] at (3,0.5) {ALU};

		% Collegamenti
		\draw[latex-] (2.6,1.5) -- ++(0,0.2) -- ++(-0.8,0) -- ++(0,-0.1);
		\draw[latex-] (3.4, 1.5) -- ++(0,0.4) -- ++(-2.2,0) -- ++(0,-0.3);
		\draw[-latex] (3.4,1) -- ++(0.4,0) -- ++(0,2) -- ++(-0.2,0) -| (psw);
		\draw[-latex] (3,0.5) -- ++(0,-0.3) -- ++(-0.5,0) -- ++(0,0.3) -- ++(-0.5,0);
		\draw[-latex] (2.3,2) -- ++(0,-1) -- ++(0.3,0);
	\end{tikzpicture}
	\caption{Struttura di un processore}
\end{figure}

\noindent Il flusso di esecuzione delle istruzioni è il seguente:
\begin{itemize}
	\item \textbf{Fetch}: CU legge l'istruzione dalla memoria
	\item \textbf{Decode}: CU decodifica l'istruzione
	\item \textbf{Execute}: ALU esegue l'istruzione
\end{itemize}

\begin{figure}[H]
	\centering
	\begin{tikzpicture}[->,>=stealth',shorten >=1pt,auto,node distance=3cm,
			thick,main node/.style={circle,draw,font=\sffamily\Large\bfseries}]

		\node[main node] (1) {\footnotesize Fetch};
		\node[main node] (2) [right of=1] {\footnotesize Decode};
		\node[main node] (3) [right of=2] {\footnotesize Execute};

		\path[every node/.style={font=\sffamily\small}]
		(1) edge [bend left] node {} (2)
		(2) edge [bend left] node {} (3)
		(3) edge [bend left] node {} (1);
	\end{tikzpicture}
	\caption{Flusso di esecuzione delle istruzioni}
\end{figure}

\subsubsection{Istruzioni}
Ogni istruzione è codificata nel seguente modo:

\begin{figure}[H]
	\centering
	\begin{tikzpicture}
		% Opcode
		\draw (0,0) rectangle (1.5,1);
		\node at (0.75,0.5) {Opcode};

		% Metodi di indirizzamento
		\draw (1.8,0) -- (1.8,1);
		\node[align=center, scale=0.7] at (1.65,0.5) {M\\I};

		% Source
		\draw (1.5,0) rectangle (3.6,1);
		\node at (2.55,0.5) {Source};

		% Metodi di indirizzamento
		\draw (3.3,0) -- (3.3,1);
		\node[align=center, scale=0.7] at (3.45,0.5) {M\\I};

		% Destination
		\draw (3.6,0) rectangle (5.6,1);
		\node at (4.6,0.5) {Destination};

	\end{tikzpicture}
\end{figure}

Le istruzioni più comuni sono:
\begin{itemize}
	\item \textbf{MOV} (o MOVL): copia il contenuto di un registro in un altro
\end{itemize}


\subsubsection{Metodi di indirizzamento}
I metodi di indirizzamento (MI) sono diversi modi per accedere ai dati in memoria, i
più comuni sono:
\begin{itemize}
	\item \textbf{Registro}: Un'istruzione può accedere direttamente ai registri
	      ad esempio: \texttt{MOVL \%EAX, \%EBX}
	\item \textbf{Immediato}: Un'istruzione può accedere direttamente ai dati
	      ad esempio: \texttt{MOVL \$10, \%EBX}
	\item \textbf{Assoluto}: Un'istruzione può accedere direttamente ai dati
	      ad esempio: \texttt{MOVL DATO, \%EBX}\\
	      dove \texttt{DATO} è un'etichetta che punta ad un'indirizzo di memoria
	\item \textbf{Indiretto Registro}: Un'istruzione può contenere un registro che punta ad un altro registro
	      ad esempio: \texttt{MOVL (\%EAX), \%EBX}
	\item \textbf{Indiretto Registro con Spiazzamento}: Un'istruzione può mettere un offset
	      rispetto al registro contenuto nell'istruzione
	      ad esempio: \texttt{MOVL \$8(\%EAX), \%EBX}
\end{itemize}

\end{document}
