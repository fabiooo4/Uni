\documentclass[a4paper]{article}

\usepackage[utf8]{inputenc}
\usepackage[T1]{fontenc}
\usepackage{textcomp}
\usepackage[italian]{babel}
\usepackage{amsmath, amssymb}
\usepackage[makeroom]{cancel}
\usepackage{amsfonts}
\usepackage{mdframed}
\usepackage{xcolor}
\usepackage{float}
\usepackage{tikz}
\usepackage{nicematrix}
\usepackage{pgfplots}
\usetikzlibrary{fit}
\usetikzlibrary{pgfplots.fillbetween}
\pgfplotsset{compat=newest, ticks=none}
\usepackage{graphicx}
\graphicspath{{./figures/}}

\pgfdeclarelayer{ft}
\pgfdeclarelayer{bg}
\pgfsetlayers{bg,main,ft}

\usepackage{import}
\usepackage{pdfpages}
\usepackage{transparent}
\usepackage{xcolor}

\usepackage{hyperref}
\hypersetup{
  colorlinks=false,
}

\usepackage{ntheorem}
\newtheorem{theorem}{Teorema}

% Useful definitions frame
\theoremstyle{break}
\theoremheaderfont{\bfseries}
\newmdtheoremenv[%
linecolor=gray,leftmargin=0,%
rightmargin=0,
innertopmargin=8pt,%
ntheorem]{define}{Definizioni utili}[section]

% Example frame
\theoremstyle{break}
\theoremheaderfont{\bfseries}
\newmdtheoremenv[%
linecolor=gray,leftmargin=0,%
rightmargin=0,
innertopmargin=8pt,%
ntheorem]{example}{Esempio}[section]

% Important definition frame
\theoremstyle{break}
\theoremheaderfont{\bfseries}
\newmdtheoremenv[%
linecolor=gray,leftmargin=0,%
rightmargin=0,
backgroundcolor=gray!40,%
innertopmargin=8pt,%
ntheorem]{definition}{Definizione}[section]

% Exercise frame
\theoremstyle{break}
\theoremheaderfont{\bfseries}
\newmdtheoremenv[%
linecolor=gray,leftmargin=0,%
rightmargin=0,
innertopmargin=8pt,%
ntheorem]{exercise}{Esercizio}[section]


% figure support
\usepackage{import}
\usepackage{xifthen}
\pdfminorversion=7
\usepackage{pdfpages}
\usepackage{transparent}
\newcommand{\incfig}[1]{%
  \def\svgwidth{\columnwidth}
  \import{./figures/}{#1.pdf_tex}
}

\pdfsuppresswarningpagegroup=1

% Matrices
\makeatletter
\renewcommand*\env@matrix[1][*\c@MaxMatrixCols c]{%
  \hskip -\arraycolsep
  \let\@ifnextchar\new@ifnextchar
\array{#1}}
\makeatother

\begin{document}
\begin{titlepage}
	\begin{center}
		\vspace*{1cm}

		\Huge
		\textbf{Probabilità e Statistica\\Esercizi}

		\vspace{0.5cm}
		\LARGE
		UniVR - Dipartimento di Informatica

		\vspace{1.5cm}

		\textbf{Fabio Irimie}

		\vfill


		\vspace{0.8cm}


		2° Semestre 2023/2024

	\end{center}
\end{titlepage}


\tableofcontents
\pagebreak

% Info sul corso
% Programma su moodle
% Lezioni sia in presenza che su online
% Ogni terza ora del lunedì si discutono gli esercizi votati nel sondaggio su moodle
% Esercizi da fare a gruppi con un punto extra per il voto finale
% Esame esame molto simile agli esercizi assegnati
% Esame di 2 ore con 4 esercizi

\section{Numeri complessi}
\subsection{Insiemi di numeri}
I numeri sono divisi in insiemi in base alle operazioni che si possono fare con essi:
\begin{itemize}
  \item
    I numeri sono stati pensati per contare e per farlo è stato definito l'insieme dei numeri naturali
    che è definito come \[  \mathbb{N} = \{0, 1, 2, 3, \ldots\} \]
  \item
    Per fare operazioni di sottrazione è stato definito l'insieme dei numeri interi che è
    definito come \[ \mathbb{Z} = \{\ldots, -3, -2, -1, 0, 1, 2, 3, \ldots\} \]
  \item
    Per fare operazioni di divisione è stato definito l'insieme dei numeri razionali che è
    definito come \[ \mathbb{Q} = \left\{ \frac{p}{q} \mid p, q \in \mathbb{Z}, q \neq 0 \right\} \]
  \item
    Per fare operazioni di radice quadrata è stato definito l'insieme dei numeri reali che è
    definito come \[ \mathbb{R} = \left\{ x \mid x \in \mathbb{Q} \right\} \]
  \item
    Infine, per fare operazioni di radice quadrata di numeri negativi è stato definito
    l'insieme dei numeri complessi che è definito come
    \[ \mathbb{C} = \left\{ z \mid z = a + bi,\;\;\; a, b \in \mathbb{R}, i^2 = -1 \right\} \]
\end{itemize}

\noindent Ognuno di questi insiemi è un sottoinsieme dell'insieme successivo, ovvero
\[ \mathbb{N} \subset \mathbb{Z} \subset \mathbb{Q} \subset \mathbb{R} \subset \mathbb{C} \]

\noindent Le equazioni non risolvibili in un insieme vengono risolte in un insieme successivo, ad esempio
\[ x^2 + 1 = 0 \] non ha soluzioni in $\mathbb{R}$, ma ha soluzioni in $\mathbb{C}$.

\begin{figure}[H]
  \begin{theorem}
    \textbf{(Teorema fondamentale dell'algebra)}\\
    Qualsiasi equazione di forma:
    \[
      a_nx^n + a_{n-1}x^{n-1} + \ldots + a_1x + a_0 = 0
    \]
    dove
    \[
      n \in \mathbb{N},\;\; a_0, a_1, \ldots, a_n \in \mathbb{C},\;\; a_n \neq 0
    \]
    ed \( x \) è un incognita, ammette \( n \) soluzioni

  \end{theorem}

  \begin{define}
    \[
      a_nx^n + a_{n-1}x^{n-1} + \ldots + a_1x + a_0 \;\; \text{con} \;\; a_n \neq 0
    \]
    è detto \textbf{polinomio di grado \( n \)} con \textbf{coefficienti} \( a_0, \ldots, a_n \in \mathbb{C} \)
  \end{define}
\end{figure}

\subsection{Numeri immaginari}
Aggiungiamo ai numeri reali un "nuovo" numero $i$ che è definito come $i^2 = -1$. Questo numero
è detto: \textbf{unità immaginaria}. Per agevolare le operazioni con i numeri immaginari si
definisce l'insieme dei \textbf{numeri complessi} in modo da poter moltiplicare e sommare un
numero reale con un numero immaginario:
\[
  \mathbb{C} = \{ a + bi \;|\; a,b \in \mathbb{R} \}
\]

\( z = a + bi \) è detta \textbf{forma algebrica} di un numero complesso \( z \in \mathbb{C} \).
\[
  a = \Re(z) \quad \text{è detta parte reale di } z
\]
\[
  b = \Im(z) \quad \text{è detta parte immaginaria di } z
\]

\begin{figure}[H]
  \begin{define}
    Per agevolare la scrittura, al posto di scrivere:
    \[
      a + (-b)i
    \]
    si scrive:
    \[
      a - bi
    \]
  \end{define}
\end{figure}

\subsubsection{Esempi}
\begin{figure}[H]
  \begin{example}
    \begin{itemize}
      \item $3 + 2i$
      \item $-12 + \frac{1}{2}i$
      \item $3-\sqrt{2}i $
      \item $1+0 \cdot i = 1\;\; \in \mathbb{R}$
    \end{itemize}
  \end{example}
\end{figure}

\subsection{Operazioni tra i numeri complessi}
\subsubsection{Somma}
\begin{figure}[H]
  \begin{definition}
    L'addizione tra due numeri complessi è definita come:
    \[
      z_1 = a + bi \quad z_2 = c + di \quad \in \mathbb{C}
    \]
    \[
      z_1 + z_2 = (a + bi) + (c + di) = (a + c) + (b + d)i
    \]
  \end{definition}
\end{figure}

\begin{figure}[H]
  \begin{example}
    \[
      z_1 = 6 + 7i \quad z_2 = -12 + 1732i
    \]
    \[
      z_1 + z_2 = (6 + 7i) + (-12 + 1732i) = -6 + 1739i
    \]
  \end{example}
\end{figure}

\subsubsection{Prodotto}
\begin{figure}[H]
  \begin{definition}
    Il prodotto tra due numeri complessi è definito come:
    \[
      z_1 = a + bi \quad z_2 = c + di \quad \in \mathbb{C}
    \]
    \[
      z_1 \cdot z_2 = (a + bi) \cdot (c + di) = ac + adi + bci + bdi^2
    \]
    visto che $i^2 = -1$ si ha che $bdi^2 = -bd$ quindi
    \[
      z_1 \cdot z_2 = ac + adi + bci - bd = (ac - bd) + (ad + bc)i
    \]
  \end{definition}

  \begin{example}
    \[
      z_1 = 3 + 2i \quad z_2 = 10 - i
    \]
    \[
      z_1 \cdot z_2 = (3 + 2i) \cdot (10 - i) = 30 - 3i + 20i - 2i^2 = 32 + 17i
    \]
  \end{example}
\end{figure}

\subsubsection{Sottrazione}
Notiamo che per ogni numero complesso \( z = a + bi \; \in \mathbb{C} \), il numero complesso
\( -a -bi \) è l'unico numero complesso tale che \( z + (-z) = 0 \). Questo numero complesso è
detto \textbf{opposto} di \( z \) e si indica con \( -z \).

\begin{figure}[H]
  \begin{definition}
    La sottrazione tra due numeri complessi è definita come:
    \[
      z_1 = a + bi \quad z_2 = c + di \quad \in \mathbb{C}
    \]
    \[
      z_1 - z_2 = z_1 + (-z_2) = (a + bi) - (c + di) = (a - c) + (b - d)i
    \]
  \end{definition}

  \begin{example}
    \[
      z_1 = 3 + 2i \quad z_2 = 10 - i
    \]
    \[
      z_1 - z_2 = (3 + 2i) - (10 - i) = -7 + 3i
    \]
  \end{example}
\end{figure}

\subsubsection{Divisione}
\begin{definition}
  La divisione tra due numeri complessi è definita come:
  \[
    z_1, z_2, z_2 \neq 0 \quad \in \mathbb{C}
  \]
  Definiamo \( \frac{1}{z_2} \) come l'unico numero complesso tale che:
  \[
    z_2 \cdot \frac{1}{z_2} = 1
  \]
  \[
    \frac{z_1}{z_2} = z_1 \cdot \frac{1}{z_2}
  \]
  Sia \( z = a + bi\;\; \in \mathbb{C} \) e \( z \neq 0 \). Supponiamo che \( z'= c + di \) sia
  un numero complesso tale che \( z \cdot z' = 1 \), cioè:
  \[
    1 = z \cdot z' = (a + bi) \cdot (c + di) = (ac - bd) + (ad + bc)i
  \]
  Abbiamo \( ac-bd=1 \) e \( ad+bc=0 \).

  \noindent Possiamo trovare \( c \) sostituendo \( d = \frac{-1-ac}{b} \) nella prima equazione:
  \[
    c = -\frac{ad}{b} \quad d = \frac{-(1-ac)}{b} = \frac{1-ac}{b}
  \]
  \[
    c = \frac{-a (\frac{-1+ac}{b})}{b} = \frac{-a (\frac{-1+ac}{b})}{b} \cdot \frac{b}{b} = \frac{-a (-1 + ac)}{b^2}
  \]
  \[
    cb^2 = a - a^2c
  \]
  \[
    c(a^2 + b^2) = a
  \]
  \[
    c = \frac{a}{a^2 + b^2}
  \]
  \noindent Possiamo trovare \( d \) sostituendo \( c = \frac{-ad}{b} \) nella seconda equazione:
  \[
    d = \frac{-bc}{a} \quad c = \frac{-(1-bd)}{a} = \frac{1-bd}{a}
  \]
  \[
    d = \frac{-b (\frac{1-bd}{a})}{a} = \frac{-b (\frac{1-bd}{a})}{a} \cdot \frac{a}{a} = \frac{-b (1 - bd)}{a^2}
  \]
  \[
    ad^2 = b - b^2d
  \]
  \[
    d(a^2 + b^2) = b
  \]
  \[
    d = \frac{b}{a^2 + b^2}
  \]
  Quindi:
  \[
    z' = \frac{a}{a^2 + b^2} - \frac{b}{a^2 + b^2}i = \frac{a - bi}{a^2 + b^2}
  \]
  di conseguenza \[ \frac{1}{z} = \frac{a-bi}{a^2+b^2} \]

  \noindent Siano \( z_1 = a+bi, z_2 = c+di \neq 0 \in \mathbb{C} \). Definiamo:
  \[
    \frac{z_1}{z_2} = z_1 \cdot \frac{1}{z_2} = z_1 \cdot \frac{c-di}{c^2+d^2} = \frac{ac+bd}{c^2+d^2} + \frac{bc-ad}{c^2+d^2}i
  \]
\end{definition}

\begin{figure}[H]
  \begin{example}
    \[
      \frac{1+2i}{2-i} = \left(1+2i\right)\left(\frac{2}{5}+\frac{1}{5}i\right) = \left( \frac{2}{5}-\frac{2}{5} \right) + \left( \frac{1}{5} + \frac{4}{5} \right) i = i
    \]
  \end{example}
\end{figure}

\noindent Un trucco per dividere i numeri complessi è moltiplicare per \( 1 \) la frazione:
\[
  (a+bi)(a-bi) = a^2 \cancel{+abi} \cancel{- abi} + b^2 = a^2 + b^2 \quad \in \mathbb{R}
\]
In questo modo si arriva ad ottenere un numero reale al denominatore facilitando la divisione.
\begin{figure}[H]
  \begin{example}
    \[
      \frac{1+2i}{2-i}
    \]
    \[
      \left( \frac{1+2i}{2-i} \right) \left( \frac{2+i}{2+i} \right) = \frac{(1+2i)(2+i)}{2^2+(-1)^2} =
    \]
    \[
      = \frac{(1+2i)(2+i)}{5} = \frac{2+4i+i+2i^2}{5} = \frac{2+5i-2}{5} = \frac{5i}{5} = i
    \]
  \end{example}
\end{figure}

\subsection{Coniugato e modulo}
\subsubsection{Coniugato}
Sia \( z = a +bi \in \mathbb{C} \). Il numero complesso \( \overline{z} = a - bi \) è detto
\textbf{coniugato} di \( z \).

\subsubsection{Modulo}
Il \textbf{modulo} di \( z \) è definito come:
\[
  |z| = \sqrt{a^2 + b^2} \quad \in \mathbb{R}
\]

\subsubsection{Proprietà}
Siano \( z_1 = a + bi, z_2 = c + di \quad \in \mathbb{C} \)
\begin{enumerate}
  \item \( z_1 \overline{z_1} = a^2 + b^2 = |z_1|^2 \)
  \item \( \overline{z_1 + z_2} = \overline{(a+c) + (b+d)i} = (a-bi) + (c-di) = \overline{z_1} + \overline{z_2} \)
  \item \( \overline{z_1 z_2} = \overline{z_1} \cdot \overline{z_2} \)
  \item Se \[ z_1 \neq 0, \; \overline{\frac{1}{z_1}} = \frac{1}{\overline{z_1}} \]
    Infatti:
    \[
      \overline{z_1} \cdot \left( \overline{\frac{1}{z_1}} \right) = \left(\overline{ z_1 \cdot \frac{1}{z_1}} \right) = \overline{1+0i} = 1 - 0i = 1
    \]
  \item Se \( z_2 \neq 0 \) allora:
    \[
      \left( \overline{\frac{z_1}{z_2}} \right) = \left( \overline{z_1 \cdot \frac{1}{z_2}} \right) = \overline{z_1} \cdot \overline{\frac{1}{z_2}} = \overline{z_1} \cdot \frac{1}{\overline{z_2}} = \frac{\overline{z_1}}{\overline{z_2}}
    \]
  \item Se \( z_1 \neq 0 \), allora
    \[
      \frac{1}{z_1} \stackrel{def}{=} \frac{a-bi}{a^2+b^2}= \frac{\overline{z_1}}{|z_1|^2}
    \]
\end{enumerate}

\begin{example}
  \[
    z = \frac{1 + i}{2 - i} = \left( 1+i \right) \left( \frac{1}{2-i} \right)
  \]
  \[
    \frac{1}{2-i} = \frac{2+i}{5} = \frac{2+i}{5} = \frac{2}{5} + \frac{1}{5}i
  \]
  \[
    z = \left( 1+i \right) \left( \frac{2}{5} + \frac{1}{5}i \right) = \left( \frac{2}{5}-\frac{1}{5} \right) + \left( \frac{2}{5} + \frac{1}{5} \right)i = \frac{1}{5} + \frac{3}{5}i
  \]
  \[
    \overline{z} = \frac{1}{5} - \frac{3}{5}i
  \]
\end{example}

\subsection{Coordinate polari}
Per ogni numero complesso si ha una coppia di coordinate:
\[
  z = a + bi \quad \in \mathbb{C}
\]
\[ (a,b) = (\Re(z), \Im(z)) \in \mathbb{R}^2 \]


\begin{figure}[H]
  \begin{center}
    \begin{tikzpicture}[scale=0.8, domain=0:13]
      \draw[->] (-0.5,0) -- (6,0) node[right] {$\Re$};
      \draw[->] (0,-0.5) -- (0,5) node[above] {$\Im$};

      \draw[fill, blue] (5, 4) circle (4pt) node[above right] {$z = (a,b)$};

      \draw[blue, thick] (0, 0) -- (5, 4);

      \draw[red, thick] (3, 0) arc (0:38:3) node[right] {$\alpha$};
    \end{tikzpicture}
  \end{center}

  \caption{Rappresentazione di un numero complesso}
\end{figure}
\noindent Possiamo esprimere \( z \) in coordinate polari \( (r, \alpha) \) dove \( r \) è la lunghezza del
segmento \( OZ \), detto \textbf{raggio polare}, ed \( \alpha \) è l'angolo compreso tra l'asse delle x
e \( OZ \) in senso antiorario. \( \alpha \) viene misurato in radianti

\begin{figure}[H]
  \begin{example}
    \[
      z_1 = (1,0) \to 1
    \]
    \[
      z_2 = (1, \frac{\pi}{2}) \to i
    \]
    \[
      z_3 = (1, \pi) \to -1
    \]
    \[
      z_4 = (1, \frac{3\pi}{2}) \to -i
    \]

    \begin{center}
      \begin{tikzpicture}[scale=0.5, domain=0:13]
        \draw[->] (-4,0) -- (4,0) node[right] {$\Re$};
        \draw[->] (0,-4) -- (0,4) node[above] {$\Im$};

        \draw[fill, blue] (3, 0) circle (4pt) node[below right] {$z_1 = (1,0)$};
        \draw[fill, blue] (0, 3) circle (4pt) node[above left] {$z_2 = (1, \frac{\pi}{2})$};
        \draw[fill, blue] (-3, 0) circle (4pt) node[below left] {$z_3 = (1, \pi)$};
        \draw[fill, blue] (0, -3) circle (4pt) node[below left] {$z_4 = (1, \frac{3\pi}{2})$};

      \end{tikzpicture}
    \end{center}
    \caption{Esempi di numeri complessi in coordinate polari}
  \end{example}
\end{figure}

\subsection{Forma trigonometrica di un numero complesso}
Dato un \( z = (r, \alpha) \) in coordinate polari, vogliamo ricavare la forma algebrica. Per
fare ciò usiamo il seno e il coseno:
\[
  \cos(\alpha) = \frac{a}{r} \quad \sin(\alpha) = \frac{b}{r}
\]
\begin{figure}[H]
  \begin{center}
    \begin{tikzpicture}[scale=0.8, domain=0:13]
      \draw[->] (-0.5,0) -- (6,0) node[right] {$\Re$};
      \draw[->] (0,-0.5) -- (0,5) node[above] {$\Im$};

      \draw[fill, blue] (5, 4) circle (4pt);

      \draw[blue, thick] (0, 0) -- (5, 4);

      \draw[red, thick] (3, 0) arc (0:38:3) node[right] {$\alpha$};

      \draw[green, dashed] (5, 0) -- (5, 4) node[midway, right] {$b$};
      \draw[green, dashed] (0, 4) -- (5, 4) node[midway, above] {$a$};
    \end{tikzpicture}
  \end{center}

  \caption{Forma trigonometrica di un numero complesso}
\end{figure}
\begin{figure}[H]
  \begin{definition}
    La \textbf{forma trigonometrica} di un numero complesso è definita come:
    \[
      z = (r \cdot \cos(\alpha)) + (r \cdot \sin(\alpha)i) = r \cdot (\cos(\alpha) + i \cdot \sin(\alpha))
    \]
    \[
      r = |z| = \sqrt{a^2 + b^2}
    \]
    \[
      \alpha = \begin{cases}
        \frac{\pi}{2} \quad \text{se} \; a = 0,\; b > 0                                   \\
        \frac{3\pi}{2} \quad \text{se} \; a = 0,\; b < 0                                  \\
        \text{non definito} \quad \text{se} \; a = 0,\; b = 0                             \\
        \arctan\left(\frac{b}{a}\right) \quad \text{se} \; a > 0,\; b \ge 0\\
        \arctan\left(\frac{b}{a}\right) + 2\pi \quad \text{se} \; a > 0,\; b < 0         \\
        \arctan\left(\frac{b}{a}\right) + \pi \quad \text{se} \; a < 0,\; b \;\text{qualsiasi}
      \end{cases}
    \]
  \end{definition}
\end{figure}

\begin{figure}[H]
  \begin{example}
    \[
      1 = \cos(0) + i \cdot \sin(0)
    \]
    \[
      i = \cos(\frac{\pi}{2}) + i \cdot \sin(\frac{\pi}{2})
    \]
    \[
      -1 = \cos(\pi) + i \cdot \sin(\pi)
    \]
    \[
      -i = \cos(\frac{3\pi}{2}) + i \cdot \sin(\frac{3\pi}{2})
    \]

  \end{example}
\end{figure}

\subsection{Prodotto di numeri complessi in forma trigonometrica}
\begin{figure}[H]
  \begin{definition}
    \[
      z_1 = r\left(\cos(\alpha) + i \sin(\alpha)\right), \quad z_2 = s\left(\cos(\beta) + i \sin(\beta)\right) \quad \in \mathbb{C}
    \]
    \vspace{0.1cm}
    \[
      z_1  z_2 = r  s  (\cos(\alpha) + i \sin(\alpha))  (\cos(\beta) + i \sin(\beta)) =
    \]
    \[
      = r  s  \left( (\cos{\alpha} \cos(\beta) - \sin(\alpha) \sin(\beta)) + (\cos(\alpha) \sin(\beta) + \sin(\alpha) \cos(\beta))i \right) =
    \]
    \[
      = r  s  \left( \cos(\alpha + \beta) + i \sin(\alpha + \beta) \right)
    \]
  \end{definition}
\end{figure}
\subsection{Formula di de Moivre}
Dati \( n \in \mathbb{N}, \quad z = r(\cos(\alpha) + i \sin(\alpha)) \in \mathbb{C} \)
\[
  z^n = r^n \cdot (\cos(n\alpha) + i \sin(n\alpha))
\]

\begin{figure}[H]
  \begin{example}
    \[
      z = \sqrt{3} + i = 2 \cdot \left( \cos(\frac{\pi}{6}) + i \sin(\frac{\pi}{6}) \right)
    \]
    \[
      z^6 = 2^6 \cdot \left( \cos(\frac{\pi}{6} \cdot 6) + i \sin(\frac{\pi}{6} \cdot 6) \right) = 64 \cdot \left( \cos(\pi) + i \sin(\pi) \right) = -64
    \]
  \end{example}
\end{figure}

\subsection{Definizione di radice n-esima}
\[
  y \in  \mathbb{C}, \quad n \in \mathbb{N}
\]
Si dicono \textbf{radici n-esime} di \( y \) le soluzioni dell'equazione \( x^n = y \).

\subsection{Teorema delle radici n-esime}
\begin{figure}[H]
  \begin{theorem}
    Siano \( y \in \mathbb{C} \) e \( n \in \mathbb{N} \). Esistono precisamente \( n \) radici n-esime
    complesse distinte \( z_0, z_1, \ldots, z_{n-1} \) di \( y \). Se \( y = r(\cos(\alpha)+i\sin(\alpha)) \),
    allora per \( k = 0, \ldots, n-1 \) :
    \[
      z_k = \sqrt[n]{r} \left( \cos\left(\frac{\alpha + 2k\pi}{n}\right) + i \sin\left(\frac{\alpha + 2k\pi}{n}\right) \right)
    \]
    Si somma \( 2k \pi  \) per ottenere tutte le radici n-esime, siccome \( \sin \) e \( \cos \) sono periodiche.
    \label{th:radici_n-esime}
  \end{theorem}
\end{figure}

\subsubsection{Dimostrazione}
Per la formula di de Moivre sappiamo che:
\[
  z_k^n = \left( \sqrt[n]{r} \right)^n \left( \cos{\alpha + (2 \pi )k} + i \sin{\alpha + (2 \pi )k} \right)  =
\]
\[
  = r \left( \cos{\alpha} + i \sin{\alpha} \right) = y
\]

\begin{figure}[H]
  \centering
  \begin{tikzpicture}
    \draw[->] (-2,0) -- (2,0) node[right] {$\Re$};
    \draw[->] (0,-2) -- (0,2) node[above] {$\Im$};
    \draw (0,0) circle (1.5cm);

    \draw[red, thick] (0,0) -- (1.05,1.05);
    \draw[blue, thick] (0.5,0) arc (0:45:0.5) node[right] {$\alpha$};

    \draw[green, dashed] (1.05,1.05) -- (1.05,0) node[below] {$\cos{x}$};

    \draw[green, dashed] (1.05,1.05) -- (0,1.05) node[left] {$\sin{x}$};

  \end{tikzpicture}
  \caption{Circonferenza goinometrica}
\end{figure}

\begin{figure}[H]
  \centering
  \begin{tikzpicture}
    % Draw the axes
    \draw[->] (-0.5,0) -- (7,0) node[right] {$\alpha$};
    \draw[->] (0,-2) -- (0,2) node[above] {$\sin{x}$};

    % plot sine
    \draw[domain=0:7, smooth, variable=\x, blue] plot ({\x}, {sin(\x r)});

    % radians labels
    \foreach \x/\xtext in {
      1.5708/\frac{\pi}{2},
      3.14159/\pi,
      4.71239/\frac{3\pi}{2},
      6.28319/2\pi,
    }
    \draw (\x,0.1) -- (\x,-0.1) node[below] {$\xtext$};

  \end{tikzpicture}
  \caption{Funzione seno}
\end{figure}

\begin{figure}[H]
  \centering
  \begin{tikzpicture}
    % Draw the axes
    \draw[->] (-0.5,0) -- (7,0) node[right] {$\alpha$};
    \draw[->] (0,-2) -- (0,2) node[above] {$\cos{x}$};

    % plot cosine
    \draw[domain=0:7, smooth, variable=\x, red] plot ({\x}, {cos(\x r)});

    % radians labels
    \foreach \x/\xtext in {
      1.5708/\frac{\pi}{2},
      3.14159/\pi,
      4.71239/\frac{3\pi}{2},
      6.28319/2\pi,
    }
    \draw (\x,0.1) -- (\x,-0.1) node[below] {$\xtext$};
  \end{tikzpicture}
  \caption{Funzione coseno}
\end{figure}

Quindi \( z_0, \ldots, z_{n-1} \) sono soluzioni di \( y = x^n \) , cioè sono radici
n-esime di \( y \).

Siccome il periodo di \( \sin{} \) e \( \cos{} \) è \( 2 \pi \), le radici n-esime sono
tutte distinte.

\subsection{Radici quadrate di numeri reali negativi}
Sia \( a \in \mathbb{R} \; \subseteq \mathbb{C} \) tale che \( a < 0 \). Esistono
precisamente due radici quadrate di \( a \) in \( \mathbb{C} \). Infatti, abbiamo:

\begin{figure}[H]
  \centering
  \begin{tikzpicture}
    \draw[->] (-2,0) -- (2,0) node[right] {$\Re$};
    \draw[->] (0,-2) -- (0,2) node[above] {$\Im$};

    \draw[fill, blue] (-1, 0) circle (2pt) node[below, align=center, yshift=-2] {$-a$\\$\alpha = \pi $};
    \draw[fill, blue] (1, 0) circle (2pt) node[below, align=center, yshift=-2] {$a$\\$\alpha = 0$};

    \draw[red, thick, ->] (1, 0) arc (0:175:1) node[midway, above right] {$\alpha$};

    \draw[fill, green] (0, 1) circle (2pt) node[above left, align=center] {$\frac{\pi }{2}$};
    \draw[fill, green] (0, -1) circle (2pt) node[below left, align=center] {$\frac{3 \pi }{2}$};
  \end{tikzpicture}
  \caption{Radici quadrate di numeri reali negativi}
\end{figure}
\[
  a = (-a) (\cos{\pi } + i \sin{\pi })
\]
Per il teorema \ref{th:radici_n-esime}:
\[
  z_0 = \sqrt{-a} \left( \cos{\frac{\pi}{2}} + i \sin{\frac{\pi}{2}} \right) = i \sqrt{-a}
\]
\[
  z_1 = \sqrt{-a} \left( \cos{\frac{3\pi}{2}} + i \sin{\frac{3\pi}{2}} \right) = -i \sqrt{-a}
\]

\begin{figure}[H]
  \begin{define}
    Se abbiamo un polinomio della forma:
    \[
      ax^2 + bx + c, \quad a,\;b,\;c \in \mathbb{R}
    \]
    Le soluzioni sono:
    \[
      \frac{-b \pm \sqrt{b^2 -4ac}}{2a}
    \]
    \[
      \Delta = b^2 - 4ac
    \]
    In \( \mathbb{C} \) esistono 2 soluzioni anche se \( \Delta < 0 \).
  \end{define}
\end{figure}

\section{Sistemi lineari e matrici}
\subsection{Sistemi lineari}
Un \textbf{sistema lineare} è un insieme di \( m \) equazioni in \( n \) incognite
che può essere scritto nella forma:
\[
  \begin{cases}
    a_{11}x_1 + a_{12}x_2 + \ldots + a_{1n}x_n = b_1 \\
    a_{21}x_1 + a_{22}x_2 + \ldots + a_{2n}x_n = b_2 \\
    \vdots                                           \\
    a_{m1}x_1 + a_{m2}x_2 + \ldots + a_{mn}x_n = b_m \\
  \end{cases}
\]
dove \( b_k,\; a_{ij} \in \mathbb{C} \) oppure \( \mathbb{R} \) per \( 1 \le i \le m,\;
1 \le j \le n,\; 1 \le k \le m\). Se i \textbf{termini noti} sono tutti nulli il sistema è detto
\textbf{omogeneo}. Una n-upla \( (x_1, \ldots, x_n) \) di numeri complessi (o reali) è
una soluzione se soddisfa tutte le \( m \) equazioni.

\begin{example}
  Presa in considerazione la seguente tabella nutrizionale di cereali (per porzione):
  \begin{center}
    \begin{tabular}{c|c|c}
                      & Cheerios & Quakers \\
                      \hline
      Proteine (g)    & 4        & 3       \\
      Carboidrati (g) & 20       & 18      \\
      Grassi (g)      & 2        & 5       \\
    \end{tabular}
  \end{center}
  Quante porzioni di Cheerios e Quakers dobbiamo mangiare per ottenere \( 9g \) di
  proteine, \( 48g \) di carboidrati e \( 8g \) di grassi?
  \[
    \begin{cases}
      4C + 3Q = 9 \quad \text{(P)}    \\
      20C + 18Q = 48 \quad \text{(C)} \\
      2C + 5Q = 8 \quad \text{(G)}
    \end{cases}
  \]

  Per risolvere il sistema lineare:
  \begin{itemize}
    \item
      Moltiplichiamo le per \( \frac{1}{4} \)
      e otteniamo un sistema lineare \textbf{equivalente} (cioè con
      \textbf{esattamente} le stesse soluzioni):
      \[
        (P') \quad C + \frac{3}{4}Q = \frac{9}{4}
      \]
      \[
        (C) \quad 20C + 18Q = 48
      \]
      \[
        (G) \quad 2C + 5Q = 8
      \]
    \item Calcoliamo \( (C)-20(P') \) e \( (G)-2(P') \) e otteniamo:
      \[
        (P') \quad C + \frac{3}{4}Q = \frac{9}{4}
      \]
      \[
        (C') \quad 0C + 15Q = 18
      \]
      \[
        (G') \quad 0C + \frac{7}{2}Q = \frac{7}{2}
      \]
    \item Moltiplichiamo \( (C') \) per \( \frac{1}{3} \) e otteniamo:
      \[
        (P') \quad C + \frac{3}{4}Q = \frac{9}{4}
      \]
      \[
        (C') \quad 0C + Q = 1
      \]
      \[
        (G') \quad 0C + \frac{7}{2}Q = \frac{7}{2}
      \]
    \item Calcoliamo \( (G') - \frac{7}{2} (C") \) e otteniamo:
      \[
        (P') \quad C + \frac{3}{4}Q = \frac{9}{4}
      \]
      \[
        (C') \quad 0C + Q = 1
      \]
      \[
        (G') \quad 0C + 0Q = 0
      \]
  \end{itemize}

  \noindent Otteniamo dunque che \( Q = 1 \) e \( C = \frac{9}{4} - \frac{3}{4} = \frac{7}{4} \)

  \vspace{0.5cm}

  \noindent Per agevolare la risoluzione del sistema lineare si può utilizzare una matrice:
  \begin{itemize}
    \item \textbf{R1} = Riga 1
    \item \textbf{R2} = Riga 2
    \item \textbf{R3} = Riga 3
  \end{itemize}
  \[
    \begin{pmatrix}[cc|c]
      4  & 3  & 9  \\
      20 & 18 & 48 \\
      2  & 5  & 8
    \end{pmatrix}
  \]
  \[\downarrow \frac{1}{4} \cdot R1 \]
  \[
    \begin{pmatrix}[cc|c]
      1  & \frac{3}{4} & \frac{9}{4} \\
      20 & 18          & 48          \\
      2  & 5           & 8
    \end{pmatrix}
  \]
  \[
    \downarrow R2 - 20 \cdot R1
  \]
  \[
    \downarrow R3 - 2 \cdot R1
  \]
  \[
    \begin{pmatrix} [cc|c]
      1 & \frac{3}{4} & \frac{9}{4} \\
      0 & 3           & 3           \\
      0 & \frac{7}{2} & \frac{7}{2}
    \end{pmatrix}
  \]
  \[
    \downarrow \frac{1}{3} \cdot R2
  \]
  \[
    \begin{pmatrix}[cc|c]
      1 & \frac{3}{4} & \frac{9}{4} \\
      0 & 1           & 1           \\
      0 & \frac{7}{2} & \frac{7}{2}
    \end{pmatrix}
  \]
  \[
    \downarrow R3 - \frac{7}{2} \cdot R2
  \]
  \[
    \begin{pmatrix}[cc|c]
      1 & \frac{3}{4} & \frac{9}{4} \\
      0 & 1           & 1           \\
      0 & 0           & 0
    \end{pmatrix}
  \]
  \noindent Otteniamo dunque che \( Q = 1 \) e \( C = \frac{9}{4} - \frac{3}{4} = \frac{7}{4} \)
\end{example}

\subsection{Definizione}
\begin{figure}[H]
  \begin{definition}
    Siano \( m\;,n,;\ < 1 \). Una tabella \( A \) tale che:
    \[
      A = \begin{pmatrix}
        a_{11} & a_{12} & \ldots & a_{1n} \\
        a_{21} & a_{22} & \ldots & a_{2n} \\
        \vdots & \vdots & \ddots & \vdots \\
        a_{m1} & a_{m2} & \ldots & a_{mn} \\
      \end{pmatrix}
      = (a_{ij})_{m \times n}
    \]
    di \( m \times n \) elementi di \( \mathbb{C} \) disposti in \( m \) righe e \( n \) colonne
    si chiama una \textbf{matrice} di \textbf{dimensione} \( m \times n \).
    Gli elementi si chiamano \textbf{coefficienti} (o entrate) della matrice e sono
    contrassegnati con un doppio indice \( ij \) dove \( i \) indica la riga e \( j \)
    la colonna di appartenenza.

    \vspace{0.5cm}
    L'insieme di tutte le matrici di dimensione \( m \times n \) con entrate in \( \mathbb{C} \)
    si indica con \( M_{m \times n}(\mathbb{C}) \).

    \vspace{0.5cm}
    L'insieme di tutte le matrici di dimensione \( m \times n \) con entrate in \( \mathbb{R} \)
    si indica con \( M_{m \times n}(\mathbb{R}) \).
  \end{definition}
\end{figure}

\begin{figure}[H]
  \begin{example}
    \[
      \begin{pmatrix} 3 & i & 2+7i \\
        0 & 1 & \pi
      \end{pmatrix} \in M_{2 \times 3}(\mathbb{C})
      \]
      \[
        \begin{pmatrix}
          0 & 1  \\
          1 & -1
        \end{pmatrix} \in M_{2 \times 2}(\mathbb{R}) \subseteq M_{2 \times 2}(\mathbb{C})
      \]
    \end{example}
  \end{figure}

\subsection{Definizione}
Un sistema lineare di \( n \) incognite e \( m \) equazioni:
\[
a_{11}x_1 + a_{12}x_2 + \ldots + a_{1n}x_n = b_1
\]
\[
\vdots
\]
\[
a_{m1}x_1 + a_{m2}x_2 + \ldots + a_{mn}x_n = b_m
\]
può essere rappresentato nella forma \textbf{matriciale}:
\[
Ax = b
\]
\[
\underbrace{
  A = \begin{pmatrix}
    a_{11} & a_{12} & \ldots & a_{1n} \\
    a_{21} & a_{22} & \ldots & a_{2n} \\
    \vdots & \vdots & \ddots & \vdots \\
    a_{m1} & a_{m2} & \ldots & a_{mn} \\
    \end{pmatrix}}_{\text{Matrice dei coefficienti}} \quad \underbrace{x = \begin{pmatrix}
    x_1    \\
    x_2    \\
    \vdots \\
    x_n
    \end{pmatrix}}_{\text{Vettore delle incognite}} \quad \underbrace{b = \begin{pmatrix}
    b_1    \\
    b_2    \\
    \vdots \\
    b_m
\end{pmatrix}}_{\text{Vettore dei termini noti}}
\]
La matrice \[ (A \;|\; B) = \begin{pmatrix}[cccc|c]
a_{11} & a_{12} & \ldots & a_{1n} & b_1    \\
\vdots & \vdots & \ddots & \vdots & \vdots \\
a_{m1} & a_{m2} & \ldots & a_{mn} & b_n
\end{pmatrix}  \] è detta \textbf{matrice aumentata}.

\begin{example}
\[
  \begin{cases}
    2x_1 + 6x_2 + 3x_3 + 2x_4 = 4                    \\
    x_1 - 2x_2 + \frac{1}{2}x_3 + \frac{9}{4}x_4 = 1 \\
    -x_1 + x_2 - \frac{1}{2}x_3 - x_4 = \frac{2}{5}
  \end{cases}
\]
Scritto come matrice aumentata diventa:
\[
  \begin{pmatrix}[cccc|c]
    2  & 6  & 3            & 2           & 4           \\
    1  & -2 & \frac{1}{2}  & \frac{9}{4} & 1           \\
    -1 & 1  & -\frac{1}{2} & -1          & \frac{2}{5}
  \end{pmatrix}
\]
\[
  \frac{1}{2}R1
\]
\[
  \begin{pmatrix}[cccc|c]
    1  & 3  & \frac{3}{2}  & 1           & 2           \\
    1  & -2 & \frac{1}{2}  & \frac{9}{4} & 1           \\
    -1 & 1  & -\frac{1}{2} & -1          & \frac{2}{5}
  \end{pmatrix}
\]
\[
  R2 - R1 \quad R3 + R1
\]
\[
  \begin{pmatrix}[cccc|c]
    1 & 3  & \frac{3}{2} & 1           & 2            \\
    0 & -5 & -1          & \frac{5}{4} & -1           \\
    0 & 4  & 1           & 0           & \frac{12}{5}
  \end{pmatrix}
\]
\[
  \frac{-1}{5}R2
\]
\[
  \begin{pmatrix}[cccc|c]
    1 & 3 & \frac{3}{2} & 1            & 2            \\
    0 & 1 & \frac{1}{5} & -\frac{1}{4} & \frac{1}{5}  \\
    0 & 4 & 1           & 0            & \frac{12}{5}
  \end{pmatrix}
\]
\[
  R3 - 4R2
\]
\[
  \begin{pmatrix}[cccc|c]
    1 & 3 & \frac{3}{2} & 1            & 2           \\
    0 & 1 & \frac{1}{5} & -\frac{1}{4} & \frac{1}{5} \\
    0 & 0 & \frac{1}{5} & 1            & \frac{4}{5}
  \end{pmatrix}
\]
\[
  5R3
\]
\[
  \begin{pmatrix}[cccc|c]
    1 & 3 & \frac{3}{2} & 1            & 2           \\
    0 & 1 & \frac{1}{5} & -\frac{1}{4} & \frac{1}{5} \\
    0 & 0 & 1           & 5            & 8
  \end{pmatrix}
\]
\end{example}

\noindent Si ottiene il sistema lineare equivalente:
\[
\begin{cases}
  x_1 + 3x_2 + \frac{3}{2}x_3 + x_4 = 2                                  \\
  \quad \quad \;\;\, x_2 + \frac{1}{5}x_3 - \frac{1}{4}x_4 = \frac{1}{5} \\
  \quad \quad \quad \quad \quad x_3 + 5x_4 = 8
\end{cases}
\]
Assegiamo un parametro alla \textbf{variabile libera} \( x_4 \) :
\[
t = x_4 \quad x_4 = t
\]
\[
x_3 = 8 - 5t
\]
\[
x_2 = \frac{1}{5}- \frac{1}{5}(8-5t) + \frac{1}{4}t = \frac{-7}{5}+ t + \frac{1}{4}t = \frac{-7}{5} + \frac{5}{4}t
\]
\[
x_1 = 2 - 3(\frac{-7}{5} + \frac{5}{4}t) \frac{-3}{2}(8-5t)-t = 2 + \frac{21}{5} - 12 - \frac{15}{4}t - \frac{15}{2}t - t =
\]
\[
\frac{10 + 21 - 60}{5} + \frac{15+30}{4}t - t = \frac{-29}{5} + \frac{15}{4}t - \frac{4}{4}t = \frac{-29}{5} + \frac{11}{4}t
\]
Il sistema ha infinite soluzioni, una per ogni \( t \in \mathbb{C} \).

\subsection{Operazioni elementari}
Attraverso le seguenti operazioni sulla matrice aumenta \( (A | b) \), si ottiene un sistema
equivalente di forma più semplice:
\begin{itemize}
\item Moltiplicare una riga \( (R_i) \) per uno scalare \( \alpha \in \mathbb{C} \) \textbf{non nullo}:
  \[
    \alpha R_i
  \]
\item Sommare una riga \( (R_i) \) con un multiplo di un'altra riga \( (R_j) \):
  \[
    R_i + \alpha R_j
  \]
\item Scambiare riga \( R_i \) con riga \( R_j \):
  \[
    R_i \leftrightarrow R_j
  \]
\end{itemize}

\begin{example}
Prendiamo il seguente sistema lineare:
\[
  \begin{cases}
    2x_1 + 6x_2 + 3x_3 = 4          \\
    x_1 - 2x_2 + \frac{1}{2}x_3 = 1 \\
    -x_1 + x_2 - \frac{7}{10}x_3 = \frac{2}{5}
  \end{cases}
\]
Scritto come matrice aumentata diventa:
\[
  \begin{pmatrix}[ccc|c]
    2  & 6  & 3             & 4           \\
    1  & -2 & \frac{1}{2}   & 1           \\
    -1 & 1  & -\frac{7}{10} & \frac{2}{5}
  \end{pmatrix}
  \stackrel{\frac{1}{2}R1}{\sim}
  \begin{pmatrix}[ccc|c]
    1  & 3  & \frac{3}{2}   & 2           \\
    1  & -2 & \frac{1}{2}   & 1           \\
    -1 & 1  & -\frac{7}{10} & \frac{2}{5}
  \end{pmatrix}
\]
\[
  \stackrel{R2 - R1}{\underset{R3+R1}{\sim}}
  \begin{pmatrix} [ccc|c]
    1 & 3  & \frac{3}{2} & 2            \\
    0 & -5 & -1          & -1           \\
    0 & 4  & \frac{4}{5} & \frac{12}{5}
  \end{pmatrix}
  \stackrel{\frac{-1}{5}R2}{\sim}
  \begin{pmatrix}[ccc|c]
    1 & 3 & \frac{3}{2} & 2            \\
    0 & 1 & \frac{1}{5} & \frac{1}{5}  \\
    0 & 4 & \frac{4}{5} & \frac{12}{5}
  \end{pmatrix}
\]
\[
  \stackrel{R3 - 4R2}{\sim}
  \begin{pmatrix}[ccc|c]
    1 & 3 & \frac{3}{2} & 2           \\
    0 & 1 & \frac{1}{5} & \frac{1}{5} \\
    0 & 0 & 0           & \frac{8}{5}
  \end{pmatrix}
  \stackrel{\frac{5}{8}R3}{\sim}
  \begin{pmatrix}[ccc|c]
    1 & 3 & \frac{3}{2} & 2           \\
    0 & 1 & \frac{1}{5} & \frac{1}{5} \\
    0 & 0 & 0           & 1
  \end{pmatrix}
\]
Otteniamo un sistema lineare equivalente:
\[
  \begin{cases}
    x_1 + 3x_2 + \frac{3}{2}x_3 = 2                       \\
    \quad \quad \;\;\, x_2 + \frac{1}{5}x_3 = \frac{1}{5} \\
    \quad \quad \quad \quad \quad \quad  0 = 1
  \end{cases}
\]
Il sistema è impossibile, non ha soluzioni.
\end{example}

\subsection{Linee in \texorpdfstring{\( \mathbb{R}^2 \)}{R²} }
2 equazioni a 2 incognite con coefficienti in \( \mathbb{R} \):
\[
\begin{cases}
  a_{11}x + a_{12}y = b_1 \quad (I) \\
  a_{21}x + a_{22}y = b_2
\end{cases}
\]
\[
a_{11},\;a_{12},\;a_{21},\;a_{22},\;b_1,\;b_2 \in \mathbb{R}
\]
Questo sistema lineare può essere rappresentato come:
\[
y = \frac{-a_{11}}{a_{12}}x + \frac{b_1}{a_{12}} \quad (I)
\]
\[
y = \frac{-a_{21}}{a_{22}}x + \frac{b_2}{a_{22}} \quad (II)
\]

\noindent Il sistema può essere rappresentato come un sistema di rette nel piano cartesiano in cui
la soluzione è l'intersezione delle rette.

\begin{figure}[H]
\centering
\begin{tikzpicture}
  \draw[->] (-2,0) -- (2,0) node[right] {\( x \)};
  \draw[->] (0,-2) -- (0,2) node[above] {\( y \)};
  \draw[red, thick] (-2,1) -- (1.5,-1.5) node[right] {(I)};
  \draw[blue, thick] (-1.5,-1) -- (2,1) node[right] {(II)};
\end{tikzpicture}
\caption{Intersezione di due rette}
\end{figure}

\noindent Può anche succedere che le rette siano parallele, in questo caso il sistema è impossibile:
\begin{figure}[H]
\centering
\begin{tikzpicture}
  \draw[->] (-2,0) -- (2,0) node[right] {\( x \)};
  \draw[->] (0,-2) -- (0,2) node[above] {\( y \)};
  \draw[red, thick] (-2,1) -- (1.5,-1.5) node[right] {(I)};
  \draw[blue, thick] (-2,2) -- (1.5,-0.5) node[right] {(II)};
\end{tikzpicture}
\caption{Retta parallela}
\end{figure}

\noindent Oppure che le rette siano coincidenti, in questo caso il sistema è indeterminato,
cioè con infinite soluzioni:
\begin{figure}[H]
\centering
\begin{tikzpicture}
  \draw[->] (-2,0) -- (2,0) node[right] {\( x \)};
  \draw[->] (0,-2) -- (0,2) node[above] {\( y \)};
  \draw[red, thick] (-2,1) -- (1.5,-1.5) node[right] {\( (I) = (II) \) };
\end{tikzpicture}
\caption{Retta coincidente}
\end{figure}

\subsection{Metodo di eliminazione di Gauss (EG)}
Data una matrice \( M = (a_{ij}) \quad 1 \le i \le m \;\; 1 \le j \le n \) in \( M_{m \times n}(\mathbb{C}) \)
(oppure in \( M_{m \times n}(\mathbb{R}) \)) con righe \( R1, \ldots, Rn \), eseguiamo le
seguenti opreazioni elementari:
\begin{enumerate}
\item Scegliamo la prima colonna non nulla \( j \) di \( M \) (partendo da sinistra).
  Dopo aver eventualmente scambiato 2 righe di \( M \), otteniamo una matrice
  della forma:
  \[
    \begin{pmatrix}
      0      & \ldots & 0      & a_{1j} & \ldots & a_{1n} \\
      \vdots & \ddots & \vdots & \vdots & \ddots & \vdots \\
      0      & \ldots & 0      & a_{mj} & \ldots & a_{mn} \\
    \end{pmatrix} \quad \text{con \( a_{ij} \neq 0 \) }
  \]
  Moltiplicando \( R1 \) per \( \frac{1}{a_{ij}} \), si ottiene:
  \[
    \begin{pmatrix}
      0      & \ldots & 0      & 1      & *        & \ldots & *      \\
      \vdots & \ddots & \vdots & \vdots & \vdots   & \ddots & \vdots \\
      0      & \ldots & 0      & a_{mj} & a_{mj+1} & \ldots & a_{mn} \\
    \end{pmatrix}
  \]
  Adesso, per ogni \( 2 \le i \le m \), eseguiamo l'operazione elementare \( Ri - a_{ij}R1 \).
  Otteniamo una matrice della forma:
  \begin{figure}[H]
    \centering
    \begin{tikzpicture}
      \node at (0,0) {
          \(
          \begin{pmatrix}
            0      & \ldots & 0      & 1      & *      & \ldots & *      \\
            \vdots & \ddots & \vdots & 0      & *      & \ldots & *      \\
            \vdots & \ddots & \vdots & \vdots & \vdots &        & \vdots \\
            0      & \ldots & 0      & 0      & *      & \ldots & *      \\
          \end{pmatrix}
          \)
        };

      \draw[red, thick] (0.3,-1.1) rectangle (2,0.4) node[midway, xshift=2] {\( M' \) };
      \draw[<-] (0,-1.1) -- (0,-1.5) node[below] {Colonna \( j \) };
    \end{tikzpicture}
  \end{figure}
\item Ripetiamo il procedimento 1. su \( M' \) per ottenere:
  \begin{figure}[H]
    \centering
    \begin{tikzpicture}
      \node at (0,0) {
          \(
          \begin{pmatrix}
            0      & \ldots & 0      & 1      & *      & \ldots & \ldots & \ldots & \ldots & *      \\
            \vdots & \ddots & \vdots & 0      & \ldots & 0      & 1      & *      & \ldots & *      \\
            \vdots & \ddots & \vdots & \vdots & \ddots & \vdots & 0      & *      & \ldots & *      \\
            \vdots & \ddots & \vdots & \vdots & \ddots & \vdots & \vdots & \vdots &        & \vdots \\
            0      & \ldots & 0      & 0      & \ldots & 0      & 0      & 0      & \ldots & 0
          \end{pmatrix}
          \)
        };

      \draw[red, thick] (1.45,-1.45) rectangle (3.25,0.1) node[midway, xshift=3] {\( M'' \) };
    \end{tikzpicture}
  \end{figure}
  e così via...
\item Dopo un numero finito di passi, si ottiene una matrice che si chiama
  \textbf{matrice a scala}:
  \begin{figure}[H]
    \centering
    \begin{tikzpicture}
      \node at (0,0) {
          \(
          r
          \left(
            \begin{matrix}
              0      & \ldots & 0      & 1      & *      & \ldots & *      & *      & *      & \ldots \\
              \vdots & \ddots & \vdots & 0      & 0      & \ldots & 0      & 1      & *      & \ldots \\
              \vdots & \ddots & \vdots & \vdots & \vdots & \ddots & \vdots & 0      & 0      & \ldots \\
              \vdots & \ddots & \vdots & \vdots & \vdots & \ddots & \vdots & \vdots & \vdots & \ddots \\
              \vdots & \ddots & \vdots & \vdots & \vdots & \ddots & \vdots & \vdots & \vdots & \ddots \\
              \vdots & \ddots & \vdots & \vdots & \vdots & \ddots & \vdots & \vdots & \vdots & \ddots \\
              0      & \ldots & 0      & 0      & 0      & \ldots & 0      & 0      & 0      & \ldots
            \end{matrix}
            \;\;\;
          \begin{matrix}
            *      & \ldots & \ldots & *      & *      & \ldots & *      \\
            *      & \ldots & \ddots & *      & *      & \ldots & *      \\
            0      & 1      & \ddots & \vdots & \vdots & \ddots & \vdots \\
            \vdots & 0      & \ddots & 1      & *      & \ldots & *      \\
            \vdots & \vdots & \ddots & 0      & 0      & \ldots & 0      \\
            \vdots & \vdots & \ddots & \vdots & \vdots & \ddots & \vdots \\
            0      & 0      & \ldots & 0      & 0      & \ldots & 0
          \end{matrix}
        \right)
        \)
      };
    \draw[green,thick] (-3.2, 2) -- (-3.2,1.6) -- ++(2.35,0) -- ++(0,-0.65) -- ++(2.35,0)
      -- ++(0,-0.65) -- ++(0.6,0) -- ++(0.8,-0.65) -- ++(2.35,0);

    \draw[green, thick] (-2.95,1.87) circle (0.21);
    \draw[<-] (-2.95,2.15) -- (-2.95,2.5) node[above] {Pivot};
    \draw[green, thick] (-0.6,1.24) circle (0.21);
    \draw[green, thick] (1.78,0.59) circle (0.21);
    \draw[green, thick] (3.18,-0.07) circle (0.21);

    \draw (-3.2,-1.9) -- ++(0,-0.2) -- ++(0.5,0) -- ++(0,0.2) node[below] (dom1) {};
    \draw[green, fill, opacity=0.2] (-3.2,-2.1) rectangle (-2.7,2.1);
    \draw (-0.88,-1.9) -- ++(0,-0.2) -- ++(0.5,0) -- ++(0,0.2) node[below left] (dom2) {};
    \draw[green, fill, opacity=0.2] (-0.88,-2.1) rectangle (-0.38,2.1);
    \draw (1.55,-1.9) -- ++(0,-0.2) -- ++(0.5,0) -- ++(0,0.2) node[below left, yshift=-2] (dom3) {};
    \draw[green, fill, opacity=0.2] (1.55,-2.1) rectangle (2.05,2.1);
    \draw (2.95,-1.9) -- ++(0,-0.2) -- ++(0.5,0) -- ++(0,0.2) node[below left, yshift=-3] (dom4) {};
    \draw[green, fill, opacity=0.2] (2.95,-2.1) rectangle (3.45,2.1);

    \node[align=center] at (0,-3.5) (col) {Colonne\\dominanti};

    \draw[->] (col) -- (dom1);
    \draw[->] (col) -- (dom2);
    \draw[->] (col) -- (dom3);
    \draw[->] (col) -- (dom4);

  \end{tikzpicture}
\end{figure}
cioè esiste un numero \( 1 \le r \le m \) tale che:
\begin{enumerate}
  \item Le righe \( 1 \le i \le r \) non sono nulle.
  \item Ogni riga \( 2 \le i \le m \) ha un numero di zeri iniziali superiore alla
    riga precedente.
  \item le righe \( r+1 \le i \le m \) sono tutte nulle.
\end{enumerate}
Inoltre il primo coefficiente non nullo di ogni riga \( i \) è uguale a \( 1 \) e si chiama \textbf{pivot}.
La matrice è detta \textbf{forma ridotta} di \( M \). Le colonne che contengono pivot
sono dette \textbf{dominanti}.
\end{enumerate}

\begin{example}
Prendiamo in considerazione la matrice:
\[
M = \begin{pmatrix}
  0 & 0  & 0 & 5  & 4 \\
  0 & 10 & 0 & 30 & 2 \\
  0 & -i & 0 & 6  & 7
\end{pmatrix} \in M_{3 \times 5}(\mathbb{C})
\]
\[
\stackrel{R1 \leftrightarrow R2}{\leadsto}
\begin{pmatrix}
  0 & 10 & 0 & 30 & 2 \\
  0 & 0  & 0 & 5  & 4 \\
  0 & -i & 0 & 6  & 7
\end{pmatrix}
\stackrel{\frac{1}{10}R1}{\leadsto}
\begin{pmatrix}
  0 & 1  & 0 & 3 & \frac{1}{5} \\
  0 & 0  & 0 & 5 & 4           \\
  0 & -i & 0 & 6 & 7
\end{pmatrix}
\]
\begin{tikzpicture}
\node at (0,0) {
    \(
    \stackrel{R3 + iR1}{\leadsto}
    \begin{pmatrix}
      0 & 1 & 0 & 3      & \frac{1}{5}      \\
      0 & 0 & 0 & 5      & 4                \\
      0 & 0 & 0 & 6 + 3i & 7 + \frac{1}{5}i
    \end{pmatrix}
    \stackrel{\frac{1}{5}R2}{\leadsto}
    \begin{pmatrix}
      0 & 1 & 0 & 3      & \frac{1}{5}      \\
      0 & 0 & 0 & 1      & \frac{4}{5}      \\
      0 & 0 & 0 & 6 + 3i & 7 + \frac{1}{5}i
    \end{pmatrix}
    \)
  };

\draw[green, fill, opacity=0.2] (-3.5,-0.7) rectangle (-3.1,0.7);
\draw[green, thick] (-3.29,0.45) circle (0.2);

\draw[red, thick] (-3,-0.7) rectangle (-0.07,0.25);

\draw[green, fill, opacity=0.2] (1.65,-0.7) rectangle (2.05,0.7);
\draw[green, thick] (1.85,0.45) circle (0.2);

\draw[red, thick] (2.2,-0.7) rectangle (5.05,0.2);

\end{tikzpicture}
\\
\begin{tikzpicture}
\node at (0,0) {
    \(
    \stackrel{R3 - (6 + 3i)R2}{\leadsto}
    \begin{pmatrix}
      0 & 1 & 0 & 3 & \frac{1}{5}                  \\
      0 & 0 & 0 & 1 & \frac{4}{5}                  \\
      0 & 0 & 0 & 0 & \frac{11}{5} - \frac{11}{5}i
    \end{pmatrix}
    \stackrel{\frac{R3}{\frac{11}{5}-\frac{11}{5}i}}{\leadsto}
    \begin{pmatrix}
      0 & 1 & 0 & 3 & \frac{1}{5} \\
      0 & 0 & 0 & 1 & \frac{4}{5} \\
      0 & 0 & 0 & 0 & 1
    \end{pmatrix}
    \)
  };

\draw[green, fill, opacity=0.2] (-1.4,-0.7) rectangle (-1.0,0.7);
\draw[green, thick] (-1.19,0.45) circle (0.2);

\draw[green, fill, opacity=0.2] (-2.45,-0.7) rectangle (-2.05,0.7);
\draw[green, thick] (-2.25,0.45) circle (0.2);

\draw[red, thick] (-0.8,-0.7) rectangle (0.6,-0.19);

\draw[green, fill, opacity=0.2] (2.75,-0.7) rectangle (3.2,0.7);
\draw[green, thick] (2.97,0.45) circle (0.2);

\draw[green, fill, opacity=0.2] (4.35,-0.7) rectangle (4.78,0.7);
\draw[green, thick] (4.57,0.43) circle (0.24);

\draw[green, fill, opacity=0.2] (3.8,-0.7) rectangle (4.24,0.7);
\draw[green, thick] (4.02,0.45) circle (0.2);

\draw[green] (2.65,0.7) -- ++(0,-0.5) -- ++(1.08,0) -- ++(0,-0.35) -- ++(0.6,0) -- ++(0,-0.45) -- ++(0.4,0);
\end{tikzpicture}
\end{example}

\subsection{Risoluzione di un sistema lineare}
Dato un sistema lineare
\[
(*)\quad Ax=b
\]
con \( A \in M_{m \times n}(\mathbb{C}),\; b \in M_{m \times 1}(\mathbb{C}) \) procediamo con
il metodo di eliminazione di Gauss sulla matrice aumentata \( (A | b) \) fino ad ottenere la
forma ridotta \( (U | c) \) e un sistema lineare corrispondente
\[
Ux = c
\]
che è equivalente a \( (*) \). Chiamiamo \textbf{variabili dominanti} le \( r \) variabili
che corrispondono alle colonne dominanti e \textbf{variabili libere} le rimanenti.

\begin{example}
Prendiamo in considerazione il seguente sistema lineare:
\[
\begin{cases}
  10x_1 + 10x_2 + 30x_3 = 2 \\
  \hspace{1.5cm} 5x_3 = 4   \\
  -x_1 - x_2 + 6x_3 = 7
\end{cases}
\]
Scritto come matrice aumentata diventa:
\[
\begin{pmatrix}[ccc|c]
  10 & 10 & 30 & 2 \\
  0  & 0  & 5  & 4 \\
  -1 & -1 & 6  & 7
\end{pmatrix}
\stackrel{EG}{\leadsto}
\begin{pmatrix}[ccc|c]
  1 & 1 & 3 & \frac{1}{5} \\
  0 & 0 & 1 & \frac{4}{5} \\
  0 & 0 & 0 & 0
\end{pmatrix}
\]
\[
\hspace{3.3cm} x_1 \;\, x_2 \;\, x_3
\]
\( x_1 \) e \( x_3 \) sono variabili dominanti e \( x_2 \) è variabile libera.
\end{example}

\noindent Si ha uno dei seguenti casi:
\begin{enumerate}
\item[1)] Tutte le colonne di \( (U|c) \) tranne \( c \) sono dominanti. In questo caso il sistema
ha una soluzione unica. Ad esempio:
\begin{figure}[H]
  \centering
  \begin{tikzpicture}
    \node at (0,0) {
        \(
        \begin{pmatrix}[cc|c]
          1 & \frac{3}{4} & \frac{9}{4} \\
          0 & 1           & 1           \\
          0 & 0           & 0
        \end{pmatrix}
        \)
      };

    \draw[green, fill, opacity=0.2] (-0.76,-0.7) rectangle (-0.37,0.7);
    \draw[green, thick] (-0.57,0.45) circle (0.2);

    \draw[green, fill, opacity=0.2] (-0.22,-0.7) rectangle (0.17,0.7);
    \draw[green, thick] (-0.02,0.02) circle (0.2);
  \end{tikzpicture}
\end{figure}
\item[\( \infty \))] L'ultima colonna e almeno una colonna di \( U \) \textbf{non} sono dominanti. In
tal caso il sistema ha infinite soluzioni che si ottengono assegnando parametri alle
\( n-r \) variabili libere. Ad esempio:
\begin{figure}[H]
  \centering
  \begin{tikzpicture}
    \node at (0,0) {
        \(
        \begin{pmatrix}[cccc|c]
          1 & 3 & \frac{3}{2} & 1            & 2           \\
          0 & 1 & \frac{1}{5} & -\frac{1}{4} & \frac{1}{5} \\
          0 & 0 & 1           & 5            & 8
        \end{pmatrix}
        \)
      };

    \draw[green, fill, opacity=0.2] (-1.45,-0.7) rectangle (-1.05,0.7);
    \draw[green, thick] (-1.25,0.45) circle (0.2);

    \draw[green, fill, opacity=0.2] (-0.94,-0.7) rectangle (-0.54,0.7);
    \draw[green, thick] (-0.74,0) circle (0.2);

    \draw[green, fill, opacity=0.2] (-0.4,-0.7) rectangle (0,0.7);
    \draw[green, thick] (-0.2,-0.42) circle (0.2);
  \end{tikzpicture}
\end{figure}
\item[0)] L'ultima colonna \( c \) è dominante. In questo tal caso il sistema non ammette
soluzioni. Ad esempio:
\begin{figure}[H]
  \centering
  \begin{tikzpicture}
    \node at (0,0) {
        \(
        \begin{pmatrix}[ccc|c]
          1 & 3 & \frac{3}{2} & 2           \\
          0 & 1 & \frac{1}{5} & \frac{1}{5} \\
          0 & 0 & 0           & 1
        \end{pmatrix}
        \)
      };

    % green highlight for first column
    \draw[green, fill, opacity=0.2] (-1.05,-0.7) rectangle (-0.64,0.7);
    \draw[green, thick] (-0.85,0.45) circle (0.2);

    % green highlight for second column
    \draw[green, fill, opacity=0.2] (-0.53,-0.7) rectangle (-0.12,0.7);
    \draw[green, thick] (-0.33,0) circle (0.2);

    % green highlight for last column
    \draw[green, fill, opacity=0.2] (0.6,-0.7) rectangle (1,0.7);
    \draw[green, thick] (0.8,-0.42) circle (0.2);
  \end{tikzpicture}
\end{figure}
\end{enumerate}

\noindent \textbf{Attenzione:} la forma ridotta di una matrice \textbf{non} è unicovamente
determinata, ma le colonne dominanti sono univocamente determinate.

\subsection{Definizione di rango di una matrice}
\begin{definition}
Sia \( A \in M_{m \times n}(\mathbb{C}) \) con forma ridotta \( U \). Il numero \( r \) di righe non
nulle, pari al numero di colonne dominanti, è detto \textbf{rango} di \( U \) e si indica
con \( rk(U) \).
\end{definition}
Verrà dimostrato più avanti che ogni forma ridotta di \( A \) ha lo stesso rango, quindi
definiamo il rango di \( A \) come \( rk(A) = rk(U) \).

\noindent Si ha \( rk(A) \le min(m,n) \).

\subsection{Osservazione}
Possiamo ricavare le condizioni \( [1],\;[\infty],\;[0] \) usando il rango:

\begin{figure}[H]
  \begin{theorem}[Teprema di Rouchè-Capelli]
    Sia \( A \in M_{m \times n}(\mathbb{C}) \), sia \( b \in M_{m \times 1}(\mathbb(C)) \).
    \[
      [1] \Leftrightarrow rk(A) = rk(A|b) = n
    \]
    \[
      \hspace{0.3cm} "rk(U) = rk(U|c)"
    \]
    \vspace{0.05cm}
    \[
      [\infty] \Leftrightarrow rk(A) = rk(A|b) < n
    \]
    \[
      \hspace{1.1cm} "rk(U) = rk(U|c) < n"
    \]
    \vspace{0.05cm}
    \[
      [0] \Leftrightarrow rk(A) < rk(A|b)
    \]
    \[
      \hspace{1cm} "rk(U) < rk(U|c)"
    \]
  \end{theorem}
\end{figure}

\section{Matrici e le loro operazioni}
\subsection{Definizione di somma}
\begin{figure}[H]
\begin{definition}
Siano \( A=(a_{ij}) \quad 1 \le i \le m \; , \; 1 \le j \le n \) e 
\( B=(b_{ij}) \quad 1 \le i \le m \; , \; 1 \le j \le n \) due matrici in \( M_{m \times n}(\mathbb{C}) \).
La \textbf{somma} di \( A \) e \( B \) è la matrice \[
  A + B (a_{ij} + b_{ij}) \quad 1 \le i \le m\;,\;1 \le j \le n  = 
\] 
\[
  = \begin{pmatrix} 
    a_{11} + b_{11} & \ldots & a_{1n} + b_{1n} \\
    \vdots          & \ddots & \vdots          \\
    a_{m1} + b_{m1} & \ldots & a_{mn} + b_{mn}
  \end{pmatrix} 
\] 
in \( M_{m \times n}(\mathbb{C}) \) 
\end{definition}
\end{figure}
\begin{figure}[H]
\begin{example}
\[
  \begin{pmatrix} 
    1 & 0 & i \\
    -3 & 1 & 4
  \end{pmatrix} 
  +
  \begin{pmatrix} 
    2 & 4 & 1 \\
    2 & -i & 1+i 
  \end{pmatrix} 
  =
  \begin{pmatrix} 
    3 & 4 & 1+i \\
    -1 & 1-i & 5+i
  \end{pmatrix}
\]
\end{example}
\end{figure}

\subsubsection{Proprietà}
L'addizione di matrici è:
\begin{itemize}
\item \textbf{Associativa}, cioè:
\[
  A + (B + C) = (A + B) + C
\] 
\item \textbf{Commutativa}, cioè:
\[
  A + B = B + A
\]
\end{itemize}

\subsection{Definizione di prodotto per uno scalare}
\begin{figure}[H]
\begin{definition}
Data una matrice \( A = (a_{ij})_{1 \le i \le m\;,\;1 \le j \le n} \in M_{m \times n}(\mathbb{C}) \) e
\( \alpha \in \mathbb{C} \), il \textbf{prodotto} della matrice \( A \) per lo scalare
\( \alpha \) è la matrice:
\[
  \alpha A = (\alpha a_{ij})_{1 \le i \le m\;,\;1 \le j \le n} \in M_{m \times n}(\mathbb{C})
\] 
\end{definition}
\end{figure}
\begin{figure}[H]
\begin{example}
\[
  \frac{1}{2}\begin{pmatrix} 
    2+i & 5\\
    i & 1-2i
  \end{pmatrix} 
  =
  \begin{pmatrix} 
    1+\frac{1}{2}i & \frac{5}{2} \\
    \frac{1}{2}i & \frac{1}{2}-i
  \end{pmatrix}
\] 
\end{example}
\end{figure}

\subsubsection{Proprietà}
Il prodotto di una matrice per uno scalare gode delle seguenti proprietà:
\begin{itemize}
\item \textbf{Distributiva rispetto all'addizione}, cioè:
\[
  \alpha(A+B) = \alpha A + \alpha B
\] 
\[
  (\alpha + \beta)A = \alpha A + \beta A
\] 
per \( A,b \in M_{m \times n}(\mathbb{C})\;,\; \alpha,\beta \in \mathbb{C} \) 
\end{itemize}

\subsection{Definizione di matrice trasposta}
\begin{figure}[H]
\begin{definition}
Accanto a una matrice \( A = (a_{ij}) \in M_{m \times n}(\mathbb{C}) \), consideriamo
la matrice \( A^T \) ottenuta da \( A \) scambiando le righe con le colonne,
è detta \textbf{trasposta} di \( A \).
\end{definition}
\end{figure}
\begin{figure}[H]
\begin{example}
\[
  A = \begin{pmatrix} 
    1 & i & 7\\
    \pi & \frac{1}{12} & 0
  \end{pmatrix} 
  \quad
  A^T = \begin{pmatrix} 
    1 & \pi \\
    i & \frac{1}{12} \\
    7 & 0
  \end{pmatrix}
\] 
\end{example}
\end{figure}

\subsection{Definizione di prodotto di matrici}
\begin{itemize}
\item Una matrice di dimensione \( m \times 1 \) è detta \textbf{vettore} (colonna) e
si usa la notazione \( v = \begin{pmatrix} v_1 \\ \vdots \\ v_m \end{pmatrix} \in M_{m \times 1}(\mathbb{C})  \).

Una matrice di dimensione \( 1 \times n \) è detta \textbf{vettore riga} e si usa la notazione
\( v^T = \begin{pmatrix} v_1 & \ldots & v_n \end{pmatrix} \in M_{1 \times n}(\mathbb{C}) \).

Sia \( v^T = \begin{pmatrix} v_1 & \ldots & v_n \end{pmatrix} \)  un vettore riga
in \( M_{1 \times n}(\mathbb{C}) \) e \( u = \begin{pmatrix} v_1 \\ \vdots \\ v_n \end{pmatrix} \)
un vettore colonna in \( M_{n \times 1}(\mathbb{C}) \). Il \textbf{prodotto} di \( v^T \) per \( u \) è
il numero complesso: \( v^Tu = v_1u_1 + v_2u_2 + \ldots + v_nu_n \in \mathbb{C} \) 

\begin{example}
  \[
    v^T=\begin{pmatrix} 1 & 2 & 3 \end{pmatrix} \quad u=\begin{pmatrix} 1 \\ 0 \\ 3 \end{pmatrix}
  \] 
  \[
    v^Tu = 1 \cdot 1 + 2 \cdot 0 + 3 \cdot 3 = 1 + 0 + 9 = 10
  \] 
\end{example}
\item Possiamo vedere una matrice \( A=(a_{ij})_{1 \le i \le m\;,\;1 \le j \le n} \)  come \( m \)
vettori riga \( Ri=(a_{i1} \ldots a_{in})_{1 \le i \le m} \) detti \textbf{righe di \( A \)}
oppure \( n \) vettori colonna \( Cj=\begin{pmatrix} a_{1j} \\ \vdots \\ a_{mj} \end{pmatrix} _{1 \le j \le n} \) detti \textbf{colonne di \( A \)}.

Siano \[ A = (a_{ij})_{1 \le i \le m\;,\;1 \le j \le n} \in M_{m \times n}(\mathbb{C}) \]
\[ B = (b_{ij})_{1 \le i \le s\;,\;1 \le j \le t} \in M_{n \times t}(\mathbb{C}) \]
Se \( n=s \), allora possiamo formare il prodotto di \( A \) e \( B \):
\[
  AB = (c_{ij})_{1 \le i \le m\;,\;1 \le j \le t}
\] 
dove
\[
  c_{ij} = RiCj = \begin{pmatrix} a_{i1} & \ldots & a_{in} \end{pmatrix} 
  \begin{pmatrix} b_{1j} \\ \vdots \\ b_{nj} \end{pmatrix} = a_{i1}b_{1j} + \ldots + a_{in}b_{nj}
\] 
è il prodotto della riga \( i \) di \( A \) e la colonna \( j \) di \( B \).

\begin{example}
  \[
    \begin{pmatrix} 1 & 2 & 3\\
    0 & 1 & 4\end{pmatrix} 
      \begin{pmatrix} 1 & 4 & 0\\
        0 & 1 & 5\\
      1 & 2 & 4\end{pmatrix} 
      =
        \begin{pmatrix} R1C1 & R1C2 & R1C3\\
        R2C1 & R2C2 & R2C3\end{pmatrix} =
        \] 
        \[
          =
          \begin{pmatrix} 
            4 & 12 & 22\\
            4 & 9 & 21
          \end{pmatrix}
        \] 
      \end{example}
  \end{itemize}

\subsubsection{Proprietà}
Il prodotto di matrici gode delle seguenti proprietà:
\begin{itemize}
  \item \textbf{Associativa}, cioè:
    \[
      A(BC) = (AB)C
    \] 
  \item \textbf{Distributiva rispetto all'addizione}, cioè:
    \[
      (A+B)C = AC + BC
    \] 
    Con \( A,B \in M_{m \times n}(\mathbb{C}) \) e \( C \in M_{n \times t}(\mathbb{C}) \)
    \[
      A(B+C) = AB + AC
    \] 
    In sostanza le matrici devono avere il numero di colonne uguale al numero di righe.
  \item Scriviamo \( I_n \in M_{m \times n}(\mathbb{C}) \) per la matrice:
    \[
      \begin{pmatrix} 
        1 & 0 & \ldots & 0\\
        0 & 1 & \ldots & 0\\
        \vdots & \vdots & \ddots & \vdots\\
        0 & 0 & \ldots & 1
      \end{pmatrix}
    \] 
    Questa matrice viene detta \textbf{matrice identità}.

    Per ogni matrice \( M \in M_{m \times n}(\mathbb{C}) \), abbiamo che:
    \[ M \cdot I_m = I_m \cdot M =  M \]
    \begin{figure}[H]
      \begin{example}
        \[
          I_3 = \begin{pmatrix} 
            1 & 0 & 0\\
            0 & 1 & 0\\
            0 & 0 & 1
          \end{pmatrix}
          \quad
          I_2 = \begin{pmatrix} 
            1 & 0\\
            0 & 1
          \end{pmatrix}
        \] 
      \end{example}
    \end{figure}
    \begin{figure}[H]
      \begin{example}
        \[
          M = \begin{pmatrix} 
            1 & 2\\
            3 & 4
          \end{pmatrix}
          \quad
          M \cdot I_2 = \begin{pmatrix} 
            1 & 2\\
            3 & 4
          \end{pmatrix}
          \begin{pmatrix} 
            1 & 0 \\
            0 & 1
          \end{pmatrix} 
          =
          \begin{pmatrix} 
            1 & 2\\
            3 & 4
          \end{pmatrix}
          = M
        \] 
      \end{example}
    \end{figure}
  \item \( (AB)^T = B^T A^T \) con \[ A \in M_{m \times n}(\mathbb{C}) \quad
    B \in M_{n \times t}(\mathbb{C}) \]
    \[
      A^T \in M_{n \times m}(\mathbb{C}) \quad
      B^T \in M_{t \times n}(\mathbb{C})
    \] 
    \begin{figure}[H]
      \begin{example}
        \[
          A = \begin{pmatrix}
            1 & 0\\
            2 & 1
          \end{pmatrix} 
          \quad
          B = \begin{pmatrix} 
            1 & 2 & 4\\
            0 & 1 & 5
          \end{pmatrix} 
        \] 
        \[
          AB = \begin{pmatrix} 
            1 & 2 & 4\\
            2 & 5 & 13
          \end{pmatrix} 
        \] 
        \[
          A^T = \begin{pmatrix}
            1 & 2\\
            0 & 1
          \end{pmatrix} 
          \quad
          B^T = \begin{pmatrix} 
            1 & 0\\
            2 & 1\\
            4 & 5
          \end{pmatrix} 
        \] 
        \[
          (AB)^T = \begin{pmatrix} 
            1 & 2\\
            2 & 5\\
            4 & 13
          \end{pmatrix} 
        \] 
        \[
          B^T A^T = \begin{pmatrix} 
            1 & 0\\
            2 & 1\\
            4 & 5 
          \end{pmatrix}
          \begin{pmatrix} 
            1 & 2\\
            0 & 1
          \end{pmatrix}
          =
          \begin{pmatrix} 
            1 & 2\\
            2 & 5\\
            4 & 13
          \end{pmatrix}
        \] 
      \end{example}
    \end{figure}

  \item Il prodotto di matrici \textbf{non} è commutativo:
    \[
      AB \neq BA
    \] 
    Infatti:
    \[
      AB = \begin{pmatrix} 
        0 & 1\\
        0 & 0
      \end{pmatrix}
      \begin{pmatrix} 
        0 & 2\\
        0 & 4
      \end{pmatrix} 
      =
      \begin{pmatrix} 
        0 & 4\\
        0 & 0
      \end{pmatrix} 
    \] 
    \[
      BA = \begin{pmatrix} 
        0 & 2\\
        0 & 4
      \end{pmatrix} 
      \begin{pmatrix} 
        0 & 1\\
        0 & 0
      \end{pmatrix} 
      =
      \begin{pmatrix} 
        0 & 0\\
        0 & 0
      \end{pmatrix} 
    \] 
\end{itemize}

\subsection{Osservazione}
Siano \( A = (a_{ij})_{1 \le i \le m\;,\;1 \le j \le n} \in M_{m \times n}(\mathbb{C}) \) e
\( b = \begin{pmatrix} b_1 \\ \vdots \\ b_m \end{pmatrix} \in M_{m \times i}(\mathbb{C}) \),
\( x = \begin{pmatrix} x_1 \\ \vdots \\ x_n \end{pmatrix}  \).
Consideriamo \( Ax=b \) in forma matriciale. Abbiamo
\[
  Ax = \underbrace{\begin{pmatrix} 
      a_{11} & \ldots & a_{1n}\\
      \vdots & \ddots & \vdots\\
      a_{m1} & \ldots & a_{mn}
  \end{pmatrix}}_{\in M_{m \times n}(\mathbb{C})}
  \underbrace{\begin{pmatrix} 
      x_1\\
      \vdots\\
      x_n
  \end{pmatrix}}_{\in M_{n \times 1}(\mathbb{C})}
  =
  \underbrace{\begin{pmatrix} 
      a_{11}x_1 + \ldots + a_{1n}x_n\\
      \vdots\\
      a_{m1}x_1 + \ldots + a_{mn}x_n
  \end{pmatrix}}_{\in M_{m \times 1}(\mathbb{C})}
\]
che è uguale a \( b = \begin{pmatrix} b_1 \\ \vdots \\ b_m \end{pmatrix}  \) 
\[
  \begin{pmatrix} 
    b_1\\
    \vdots\\
    b_m
  \end{pmatrix} 
  =
  \begin{pmatrix} 
    a_{11}x_1 + \ldots + a_{1n}x_n\\
    \vdots\\
    a_{m1}x_1 + \ldots + a_{mn}x_n
  \end{pmatrix} 
\] 
\[
  \leadsto \begin{cases}
    a_{11}x_1 + \ldots + a_{1n}x_n = b_1\\
    \vdots\\
    a_{m1}x_1 + \ldots + a_{mn}x_n = b_m
  \end{cases}
\] 
\begin{figure}[H]
  \begin{example}
    \[
      \begin{cases}
        2x_1 + 6x_2 = 4\\
        x_1 - 2x_2 = 1\\
        -x_1 + x_2 = \frac{2}{5}
      \end{cases}
    \] 
    \[
      A = 
      \begin{pmatrix} 
        2 & 6 \\
        1 & -2 \\
        -1 & 1
      \end{pmatrix} 
      \quad
      x = \begin{pmatrix} 
        x_1\\
        x_2
      \end{pmatrix} 
      \quad
      b = \begin{pmatrix} 
        4\\
        1\\
        \frac{2}{5}
      \end{pmatrix} 
    \] 
    \[
      Ax = \begin{pmatrix} 
        2 & 6\\
        1 & -2\\
        -1 & 1
      \end{pmatrix} 
      \begin{pmatrix} 
        x_1\\
        x_2
      \end{pmatrix} 
      =
      \begin{pmatrix} 
        2x_1 + 6x_2\\
        x_1 - 2x_2\\
        -x_1 + x_2
      \end{pmatrix}
      =
      \begin{pmatrix} 
        4\\
        1\\
        \frac{2}{5}
      \end{pmatrix} 
    \] 
  \end{example}
\end{figure}

\subsection{Definizione}
Una matrice \( A = (a_{ij})_{1 \le i,j \le n} \in M_{n \times n}(\mathbb{C}) \) di dimensione
\( n \times n \) si dice \textbf{matrice quadrata} di ordine \( n \).
Gli elementi di \( A \): \( a_{ii}\quad 1 \le i \le n \) formano la \textbf{diagonale} di \( A \).
\begin{figure}[H]
  \begin{example}
    \[
      \begin{pNiceArray}{>{\strut}ccc}[margin,extra-margin = 1pt]
        0 & -10 & i\\
        7 & 8 & 0\\
        100 & \frac{1}{2} & -i
        \CodeAfter
        \begin{tikzpicture}
          \node [draw=green, fill=green, opacity=0.3, rounded corners=2pt, inner ysep = 0pt,
          rotate fit=-22, fit = (1-1) (3-3) ] {} ;
        \end{tikzpicture}
      \end{pNiceArray}
    \] 
  \end{example}
\end{figure}
\noindent Se tutti gli elementi fuorri dalla diagonale sono nulli, la matrice è detta
\textbf{matrice diagonale}.
\begin{figure}[H]
  \begin{example}
    \[
      \begin{pNiceArray}{>{\strut}ccc}[margin,extra-margin = 1pt]
        0 & 0 & 0\\
        0 & 8 & 0\\
        0 & 0 & -i
        \CodeAfter
        \begin{tikzpicture}
          \node [draw=green, fill=green, opacity=0.3, rounded corners=2pt, inner ysep = 0pt,
          rotate fit=-33, fit = (1-1) (3-3) ] {} ;
        \end{tikzpicture}
      \end{pNiceArray}
    \] 
  \end{example}
\end{figure}

\noindent Se tutti i coefficienti al di sotto della diagonale sono nulli, allora la
matrice è detta \textbf{matrice triangolare superiore}.
\begin{figure}[H]
  \begin{example}
    \[
      \begin{pmatrix} 
        0 & -10 & i\\
        0 & 8 & 0\\
        0 & 0 & -i 
      \end{pmatrix} 
    \] 
    \[
      \begin{pNiceArray}{>{\strut}ccc}[margin,extra-margin = 1pt]
        0 & -10 & i\\
        0 & 8 & 0\\
        0 & 0 & -i
        \CodeAfter
        \begin{tikzpicture}
          \node [draw=green, fill=green, opacity=0.3, rounded corners=2pt, inner ysep = 0pt,
          rotate fit=-28, fit = (2-1) (3-2) ] {} ;
          \node [draw=green, fill=green, opacity=0.3, rounded corners=2pt, inner ysep = 0pt,
          rotate fit=-28, fit = (3-1) ] {} ;
        \end{tikzpicture}
      \end{pNiceArray}
    \] 
  \end{example}
\end{figure}
\noindent Se tutti i coefficienti al di sopra della diagonale sono nulli, allora la matrice è detta
\textbf{matrice triangolare inferiore}.
\begin{figure}[H]
  \begin{example}
    \[
      \begin{pmatrix} 
        0 & 0 & 0\\
        7 & 8 & 0\\
        100 & \frac{1}{2} &-i 
      \end{pmatrix} 
    \] 
    \[
      \begin{pNiceArray}{>{\strut}ccc}[margin,extra-margin = 1pt]
        0 & 0 & 0\\
        7 & 8 & 0\\
        100 & \frac{1}{2} & -i
        \CodeAfter
        \begin{tikzpicture}
          \node [draw=green, fill=green, opacity=0.3, rounded corners=2pt, inner ysep = 2pt,
          rotate fit=-33, fit = (1-2) (2-3) ] {} ;
          \node [draw=green, fill=green, opacity=0.3, rounded corners=2pt, inner ysep = 2pt,
          rotate fit=-33, fit = (1-3) ] {} ;
        \end{tikzpicture}
      \end{pNiceArray}
    \]
  \end{example}
\end{figure}

\subsection{Matrici elementari}
Prendiamo la matrice identità:
\[
  I_n = \begin{pmatrix} 
    1 & 0 & \ldots & 0\\
    0 & 1 & \ldots & 0\\
    \vdots & \vdots & \ddots & \vdots\\
    0 & 0 & \ldots & 1
  \end{pmatrix}
\] 
Applichiamo le operazioni elementari alla matrice identità \( I_n \) per ottenere le
matrici elementari che denotiamo come segue:
\begin{itemize}
  \item 
    \(E_{ij}\) la matrice ottenuta da \( I_n \) scambiando la riga \( i \)
    con la riga \( j \) 
    \begin{example}
      \[
        n = 3 \quad I_3 = \begin{pmatrix} 
          1 & 0 & 0\\
          0 & 1 & 0\\
          0 & 0 & 1
        \end{pmatrix}
      \] 
      \[
        E_{12} = \begin{pmatrix} 
          0 & 1 & 0\\
          1 & 0 & 0\\
          0 & 0 & 1
        \end{pmatrix}
      \] 
    \end{example}
  \item \( E_i(\alpha) \) ottenuta da \( I_n \) moltiplicando la riga \( i \) per lo 
    scalare \( 0 \neq \alpha \in \mathbb{C} \) 
    \begin{example}
      \[
        n = 3 \quad \alpha = i + 5 \in \mathbb{C}
      \] 
      \[
        E_3(i+5) = \begin{pmatrix} 
          1 & 0 & 0\\
          0 & 1 & 0\\
          0 & 0 & i+5 
        \end{pmatrix}
      \] 
    \end{example}

  \item \( E_{ij}(\alpha) \) ottenuta da \( I_n \) sommando la riga \( i \) con la 
    riga \( j \) moltiplicata per lo scalare \( \alpha \in \mathbb{C} \) 
    \begin{example}
      \[
        n = 3 \quad \alpha = \frac{-5}{6} \in \mathbb{C}
      \] 
      \[
        E_{13} = \begin{pmatrix} 
          1 & 0 & \frac{-5}{6}\\
          0 & 1 & 0\\
          0 & 0 & 1
        \end{pmatrix}
      \] 
    \end{example}
\end{itemize}

\subsection{Moltiplicazione con matrici elementari}
\begin{figure}[H]
  \begin{example}
    \[
      A = \begin{pmatrix} 
        1 & 0\\
        0 & 3\\
        -1 & 5
      \end{pmatrix} 
    \] 
    \[
      E_{23}A = \begin{pmatrix} 
        1 & 0 & 0\\
        0 & 0 & 1\\
        0 & 1 & 0
      \end{pmatrix}
      \begin{pmatrix} 
        1 & 0\\
        0 & 3\\
        -1 & 5
      \end{pmatrix}
      =
      \begin{pmatrix} 
        1 & 0\\
        -1 & 5\\
        0 & 3
      \end{pmatrix}
    \] 
    \[
      E_3(i+5)A = \begin{pmatrix} 
        1 & 0 & 0\\
        0 & 1 & 0\\
        0 & 0 & i+5
      \end{pmatrix}
      \begin{pmatrix} 
        1 & 0\\
        0 & 3\\
        -1 & 5
      \end{pmatrix}
      =
      \begin{pmatrix} 
        1 & 0\\
        0 & 3\\
        -i-5 & 5(i+5)
      \end{pmatrix}
    \] 
    \[
      E_{13}A = \begin{pmatrix} 
        1 & 0 & \frac{-5}{6}\\
        0 & 1 & 0\\
        0 & 0 & 1
      \end{pmatrix}
      \begin{pmatrix} 
        1 & 0\\
        0 & 3\\
        -1 & 5
      \end{pmatrix}
      =
      \begin{pmatrix} 
        \frac{11}{6} & \frac{-25}{6}\\
        0 & 3\\
        -1 & 5
      \end{pmatrix}
    \] 
  \end{example}
\end{figure}
\noindent Osserviamo che ogni operazione elementare su una matrice \( A \in M_{m \times n} (\mathbb{C}) \)
corrisponde alla (pre)moltiplicazione di \( A \) con la matrice elementare ottenuta da
\( I_m \) effettuando la medesima operazione elementare.
\begin{figure}[H]
  \begin{define}
    \[
      AE_1(-\pi ) = \begin{pmatrix} 
        1 & 0\\
        0 & 3\\
        -1 & 5
      \end{pmatrix}
      \begin{pmatrix} 
        -\pi & 0\\
        0 & 1\\
      \end{pmatrix} 
      =
      \begin{pmatrix} 
        -\pi & 0\\
        0 & 3\\
        \pi & 5
      \end{pmatrix}
    \] 
  \end{define}
\end{figure}
\begin{figure}[H]
  \begin{example}
    \[
      A = \begin{pmatrix} 
        1 & -1 & 0\\
        3 & 2 & 15
      \end{pmatrix} 
      \stackrel{EG}{\leadsto}
    \] 
    \[
      \stackrel{R2-3R1}{\underset{E_{21}(-3)}{\leadsto}}
      \underbrace{
        \begin{pmatrix} 
          1 & -1 & 0\\
          0 & 5 & 15
      \end{pmatrix}}_{\equiv E_{21}A}
      \stackrel{\frac{1}{5}R2}{\underset{E_2(\frac{1}{5})}{\leadsto}}
      \underbrace{
        \begin{pmatrix} 
          1 & -1 & 0\\
          0 & 1 & 3
      \end{pmatrix}}_{\equiv E_2(\frac{1}{5})(E_{21}(-3)A)}
      =U
    \] 
    Otteniamo una matrice con 2 pivot e 2 colonne dominanti. Questa matrice viene chiamata
    \textbf{forma ridotta di \( A \) }. Quindi il calcolo può essere anche fatto in questo modo:
    \[
      U = E_2\left(\frac{1}{5}\right)(E_{21}(-3)A)=
    \] 
    \[
      = \begin{pmatrix}
        1 & 0\\
        0 & \frac{1}{5}
      \end{pmatrix} 
      \begin{pmatrix} 
        1 & 0\\
        -3 & 1
      \end{pmatrix} 
      \begin{pmatrix} 
        1 & -1 & 0\\
        3 & 2 & 15
      \end{pmatrix} =
    \] 
    \[
      = \underbrace{\begin{pmatrix} 
          1 & 0\\
          -\frac{3}{5} & \frac{1}{5}
      \end{pmatrix} }_{E}
      \begin{pmatrix} 
        1 & -1 & 0\\
        3 & 2 & 15
      \end{pmatrix} 
    \] 
    \vspace{0.5cm}
    \[
      \begin{pmatrix}[ccc|cc]
        1 & -1 & 0 & 1 & 0\\
        3 & 2 & 15 & 0 & 1
      \end{pmatrix} 
      \stackrel{R2-3R1}{\leadsto}
      \begin{pmatrix}[ccc|cc] 
        1 & -1 & 0 & 1 & 0\\
        0 & 5 & 15 & -3 & 1
      \end{pmatrix} 
    \] 
    \[
      \stackrel{\frac{1}{5}R2}{\leadsto}
      \begin{pmatrix}[ccc|cc] 
        1 & -1 & 0 & 1 & 0\\
        0 & 1 & 3 & -\frac{3}{5} & \frac{1}{5}
      \end{pmatrix}
    \] 
    A sinistra della barra abbiamo la matrice \( U \) e a destra la matrice \( E \).
  \end{example}
\end{figure}

\subsection{Definizione di matrice invertibile}
Una matrice \( A \in M_{n \times n}(\mathbb{C}) \) si dice \textbf{invertibile} se esiste
\( C \in M_{n \times n}(\mathbb{C}) \) tale che:
\[
  CA = I_n \quad \text{e} \quad AC = I_n
\]
In tal caso, \( C \) è detta \textbf{inversa} di \( A \). L'inversa di \( A \), quando
esiste, è univocamente determinata e si denota con \( A^{-1} \). Infatti, se \( C \) e \( C' \) 
sono due matrici inverse di \( A \) , allora:
\[
  C = I_nC = (C'A)C = C'(AC) = C'I_n = C'
\] 
\begin{figure}[H]
  \begin{example}
    \[
    A = \begin{pmatrix}
      2 & 5\\
      -3 & -7
    \end{pmatrix} 
    \quad
    C = \begin{pmatrix} 
      -7 & -5\\
      3 & 2
    \end{pmatrix} 
    \] 
    \[
    AC = \begin{pmatrix} 
      2 & 5\\
      -3 & -7
    \end{pmatrix}
    \begin{pmatrix} 
      -7 & -5\\
      3 & 2
    \end{pmatrix}
    =
    \begin{pmatrix} 
      1 & 0\\
      0 & 1
    \end{pmatrix}
    \] 
    \[
    CA = \begin{pmatrix} 
      -7 & -5\\
      3 & 2
    \end{pmatrix}
    \begin{pmatrix} 
      2 & 5\\
      -3 & -7
    \end{pmatrix}
    =
    \begin{pmatrix} 
      1 & 0\\
      0 & 1
    \end{pmatrix}
    \] 
    \[
      \leadsto C = A^{-1}
    \] 
  \end{example}
\end{figure}

\noindent Se \( A,B \in M_{m \times n}(\mathbb{C}) \) sono invertibili, allora lo è anche
il loro prodotto \( AB \). Infatti l'inversa di \( AB \) è \( B^{-1}A^{-1} \).
Infatti:
\[
  (AB)(B^{-1}A^{-1}) = A(BB^{-1})A^{-1} = A I_n A^{-1} = AA^{-1} = I_n
\] 
\begin{center}
  oppure
\end{center}
\[
  (B^{-1}A^{-1})(AB) = B^{-1}(A^{-1}A)B = (B^{-1}I_n)B = B^{-1}B = I_n
\] 
Quindi \( (AB)^{-1} = B^{-1}A^{-1} \).

\subsection{Inverse di matrici elementari}
Le matrici elementari sono tutte invertibili con inverse:
\[
  E_{ij}^{-1} = E_{ij}
\] 
\begin{figure}[H]
  \begin{example}
    \[
      E_{23} = \begin{pmatrix} 
        1 & 0 & 0\\
        0 & 0 & 1\\
        0 & 1 & 0
      \end{pmatrix}
    \] 
    \[
      \begin{pmatrix} 
        1 & 0 & 0\\
        0 & 0 & 1\\
        0 & 1 & 0
      \end{pmatrix} 
      \begin{pmatrix} 
        1 & 0 & 0\\
        0 & 0 & 1\\
        0 & 1 & 0
      \end{pmatrix}
      =
      \begin{pmatrix} 
        1 & 0 & 0\\
        0 & 1 & 0\\
        0 & 0 & 1
      \end{pmatrix}
    \] 
  \end{example}
\end{figure}
\[
  E_i(\alpha)^{-1} = E_i(\frac{1}{\alpha})
\] 
\begin{figure}[H]
  \begin{example}
    \[
      E_3(i+5) = \begin{pmatrix} 
        1 & 0 & 0\\
        0 & 1 & 0\\
        0 & 0 & i+5
      \end{pmatrix}
    \] 
    \[
      E_3(\frac{1}{i+5})E_3(i+5) = I_3
    \]
    \[
      \begin{pmatrix} 
        1 & 0 & 0\\
        0 & 1 & 0\\
        0 & 0 & i+5
      \end{pmatrix}
      \begin{pmatrix} 
        1 & 0 & 0\\
        0 & 1 & 0\\
        0 & 0 & \frac{1}{1+5} 
      \end{pmatrix}
      =
      \begin{pmatrix} 
        1 & 0 & 0\\
        0 & 1 & 0\\
        0 & 0 & 1
      \end{pmatrix}
    \] 
  \end{example}
\end{figure}
\[
  E_{ij}(\alpha)^{-1} = E_{ij}(-\alpha)
\] 
\begin{figure}[H]
  \begin{example}
    \[
      E_{23}(-\frac{5}{6}) = \begin{pmatrix} 
        1 & 0 & \frac{-5}{6}\\
        0 & 1 & 0\\
        0 & 0 & 1
      \end{pmatrix}
    \] 
    \[
      E_{23}(\frac{5}{6})E_{23}(-\frac{5}{6}) = I_3
    \] 
    \[
      \begin{pmatrix} 
        1 & 0 & 0\\
        0 & 1 & 0\\
        0 & 0 & \frac{5}{6}
      \end{pmatrix}
      \begin{pmatrix} 
        1 & 0 & \frac{-5}{6}\\
        0 & 1 & 0\\
        0 & 0 & 1
      \end{pmatrix}
      =
      \begin{pmatrix} 
        1 & 0 & 0\\
        0 & 1 & 0\\
        0 & 0 & 1
      \end{pmatrix}
    \] 
  \end{example}
\end{figure}

\subsection{Proposizione}
Sia \( Ax = b \) un sistema lineare in forma matriciale, cioè \( A \in M_{m \times n}(\mathbb{C}) \) 
e \( b \in M_{m \times 1}(\mathbb{C}) \). Se \( (U|c) \) è una forma ridotta della matrice
aumentata \( (A|b) \), allora i sistemi lineari \( Ax = b \) e \( Ux = c \) hanno le
stesse soluzioni, cioè sono equivalenti.

\subsubsection{Dimostrazione}
Siano \( E_1, \ldots, E_s \) le matrici elementari che trasformano \( (A|b) \) nella forma
ridotta \( (U|c) \). Allora:
\[
  (A|b) \underset{E_1}{\sim} (A'|b') \underset{E_2}{\sim} \ldots \underset{E_s}{\sim} (U|c)
\] 
Allora abbiamo:
\[
  (U|c) = E_s \ldots \underbrace{E_1(A|b)}_{(A'|b')}
\] 
Per 3.10, le matrici elementari \( E_1, \ldots, E_s \) sono invertibili. Dunque anche il
prodotto \( E = E_s \ldots E_1 \) è invertibile con \( E^{-1} = E_1^{-1} \ldots E_s^{-1} \).
Abbiamo che \( E(A|b) = (U|c) \), ovvero \( EA = U \) e \( Eb = c \).
Pertanto, se \( v \in M_{n \times 1}(\mathbb{C}) \) è una soluzione di \( Ax = b \),
cioè \( Av = b \), allora:
\[
  Uv = (EA)v = E(Av) = Eb = c
\] 
Quindi \( v \) è soluzione di \( Ux = c \).

Se \( v \in M_{a \times  1})\mathbb{C} \) è soluzione di \( Ux = c \), cioè \( Uv = c \),
allora:
\[
  Av = \underbrace{(E^{-1}E)}_{I_m}Av = E^{-1}(EA)v = E^{-1}(Uv) = E^{-1}c =
\] 
\[
  = E^{-1}(Eb) = \underbrace{(E^{-1}E)}_{I_m}b = b
\] 
Quindi \( v \) è soluzione di \( Ax = b \quad \square \). 

\subsection{Proposizione}
Sono equivalenti i seguenti enunciati per \( A \in  M_{m \times n}(\mathbb{C}) \):
\begin{enumerate}
  \item Il sistema lineare \( Ax = b \) ammette soluzione per qualsiasi \( b \in M_{m \times 1}(\mathbb{C}) \).
  \item Il rango \( rk(A) \) di \( A \) è pari al numero di righe di \( A \).
\end{enumerate}

\subsubsection{Dimostrazione}
Dimostriamo che 1. implica 2. Supponiamo (1.)

\vspace{1em}
\noindent Sia \( U \) una forma ridotta di \( A \):
\begin{itemize}
  \item[] 
  \begin{figure}[H]
    \centering
    \begin{tikzpicture}
      \node at (0,0) {
          \(
          r
          \left(
            \begin{matrix}
              0      & \ldots & 0      & 1      & *      & \ldots & *      & *      & *      & \ldots \\
              \vdots & \ddots & \vdots & 0      & 0      & \ldots & 0      & 1      & *      & \ldots \\
              \vdots & \ddots & \vdots & \vdots & \vdots & \ddots & \vdots & 0      & 0      & \ldots \\
              \vdots & \ddots & \vdots & \vdots & \vdots & \ddots & \vdots & \vdots & \vdots & \ddots \\
              \vdots & \ddots & \vdots & \vdots & \vdots & \ddots & \vdots & \vdots & \vdots & \ddots \\
              \vdots & \ddots & \vdots & \vdots & \vdots & \ddots & \vdots & \vdots & \vdots & \ddots \\
              0      & \ldots & 0      & 0      & 0      & \ldots & 0      & 0      & 0      & \ldots
            \end{matrix}
            \;\;\;
          \begin{matrix}
            *      & \ldots & \ldots & *      & *      & \ldots & *      \\
            *      & \ldots & \ddots & *      & *      & \ldots & *      \\
            0      & 1      & \ddots & \vdots & \vdots & \ddots & \vdots \\
            \vdots & 0      & \ddots & 1      & *      & \ldots & *      \\
            \vdots & \vdots & \ddots & 0      & 0      & \ldots & 0      \\
            \vdots & \vdots & \ddots & \vdots & \vdots & \ddots & \vdots \\
            0      & 0      & \ldots & 0      & 0      & \ldots & 0
          \end{matrix}
        \right)
        \)
      };
    \draw[green,thick] (-3.2, 2) -- (-3.2,1.6) -- ++(2.35,0) -- ++(0,-0.65) -- ++(2.35,0)
      -- ++(0,-0.65) -- ++(0.6,0) -- ++(0.8,-0.65) -- ++(2.35,0);

    \draw[green, thick] (-2.95,1.87) circle (0.21);
    \draw[<-] (-2.95,2.15) -- (-2.95,2.5) node[above] {Pivot};
    \draw[green, thick] (-0.6,1.24) circle (0.21);
    \draw[green, thick] (1.78,0.59) circle (0.21);
    \draw[green, thick] (3.18,-0.07) circle (0.21);

    \draw (-3.2,-1.9) -- ++(0,-0.2) -- ++(0.5,0) -- ++(0,0.2) node[below] (dom1) {};
    \draw[green, fill, opacity=0.2] (-3.2,-2.1) rectangle (-2.7,2.1);
    \draw (-0.88,-1.9) -- ++(0,-0.2) -- ++(0.5,0) -- ++(0,0.2) node[below left] (dom2) {};
    \draw[green, fill, opacity=0.2] (-0.88,-2.1) rectangle (-0.38,2.1);
    \draw (1.55,-1.9) -- ++(0,-0.2) -- ++(0.5,0) -- ++(0,0.2) node[below left, yshift=-2] (dom3) {};
    \draw[green, fill, opacity=0.2] (1.55,-2.1) rectangle (2.05,2.1);
    \draw (2.95,-1.9) -- ++(0,-0.2) -- ++(0.5,0) -- ++(0,0.2) node[below left, yshift=-3] (dom4) {};
    \draw[green, fill, opacity=0.2] (2.95,-2.1) rectangle (3.45,2.1);

    \node[align=center] at (0,-3.5) (col) {Colonne\\dominanti};

    \draw[->] (col) -- (dom1);
    \draw[->] (col) -- (dom2);
    \draw[->] (col) -- (dom3);
    \draw[->] (col) -- (dom4);

  \end{tikzpicture}
\end{figure}
\end{itemize}
Queste righe esistono se e solo se \( rk(U) < \) numero di righe di \( U \). 

\vspace{1em}
\noindent Esiste una matrice invertibile \( E \) tale che \( U = EA \) (\( E = \) prodotto
delle matrici elementari dell'Eliminaizone di Gauss). Consideriamo il vettore
\( C = \begin{pmatrix} 0 \\ \vdots \\ 0 \\ 1 \end{pmatrix} \) e mettiamo \( b = E^{-1}C \).
Allora il sistema lineare \( Ax=b \) ammette una soluzione \( v \) per (1.), cioè
\( Av = b \). Allora \( Uv = Eb = E(E^{-1}C) = C \) per (3.11). Per il teorema di
\textbf{Rouché-Capelli}, \( rk(U) = rk(U|c) \), cioè:
\[
  (U|c) = \begin{pmatrix}[ccccccc|c]
    1 & * & \ldots & * & * & \ldots & * & 0\\
    0 & 0 & \ldots & 0 & 1 & \ldots & * & 0\\
    \vdots & \vdots & \ddots & \vdots & \vdots & \ddots & \vdots & \vdots\\
    * & * & \ldots & * & * & \ldots & * & 1
  \end{pmatrix} 
\] 
L'ultima riga non può essere nulla, altrimenti l'ultima colonna di \( (U|c) \) sarebbe una
colonna dominante.

\vspace{1em}
\noindent Dunque \( rk(A) = rk(U) = \) numero di righe di \( U \) \( = \) numero di righe di \( A \).

\vspace{2em}
\noindent Dimostriamo che 2. implica 1. Supponiamo (2.)

\vspace{1em}
\noindent Sia \( b \in  M_{m \times 1}(\mathbb{C}) \) e consideriamo \( Ax = b \). Eseguendo l'Eliminazione
di Gauss sulla matrice \( (A|b) \), otteniamo una forma ridotta \( (U|c) \).
Siccome \( rk(U) = \) numero di righe di \( U \), ogni riga di \( U \) contiene un pivot.
Perciò \( rk(U) = rk(U|c) \) e quindi \( rk(A) = rk(A|b) \). Quindi siamo nel caso di
una soluzione unica, oppure nel caso di infinite soluzioni del teorema di \textbf{Rouché-Capelli}.
\( \quad \square \) 

\section{Matrici inverse e determinante}
\begin{figure}[H]
  \begin{example}
    \[
      A = \begin{pmatrix} 
        1 & 2 & 0\\
        5 & 11 & -1\\
        -4 & -10 & -2
      \end{pmatrix} 
    \] 
    Eseguiamo l'Eliminazione di Gauss e calcoliamo il prodotto delle matrici elementari
    contemporaneamente:
    \[
      \begin{pmatrix}[ccc|ccc] 
        1 & 2 & 0 & 1 & 0 & 0\\
      5 & 11 & -1 & 0 & 1 & 0\\
      -4 & -10 & -2 & 0 & 0 & 1
    \end{pmatrix} 
    \stackrel{E_{21}(-5)}{\underset{E_{31}4}{\leadsto}}
    \begin{pmatrix}[ccc|ccc] 
      1 & 2 & 0 & 1 & 0 & 0\\
      0 & 1 & -1 & -5 & 1 & 0\\
      0 & -2 & -2 & 4 & 0 & 1
    \end{pmatrix}
   \]
   \[
     \stackrel{E_{32}(2)}{\leadsto}
     \begin{pmatrix}[ccc|ccc]
       1 & 2 & 0 & 1 & 0 & 0\\
       0 & 1 & -1 & -5 & 1 & 0\\
       0 & 0 & -4 & -6 & 2 & 1
     \end{pmatrix}
     \stackrel{E_3(-\frac{1}{4})}{\leadsto}
     \begin{pmatrix}[ccc|ccc]
       1 & 2 & 0 & 1 & 0 & 0\\
       0 & 1 & -1 & -5 & 1 & 0\\
       0 & 0 & 1 & \frac{3}{2} & -\frac{1}{2} & -\frac{1}{4}
      \end{pmatrix}
   \] 
   A sinistra della barra abbiamo la matrice ridotta \( U \) , a destra abbiamo il prodotto delle
   matrici elementari. Cioè: 
   \[
     E_3(-\frac{1}{4})E_{32}(2)E_{31}(4)E_{21}(-5)
   \] 
   Siccome \( rk(U) = 3 \), possiamo continuare per ottenere la matrice identità:
   \[
     (U|E) \stackrel{E_{23}(1)}{\leadsto} \begin{pmatrix}[ccc|ccc]
       1 & 2 & 0 & 1 & 0 & 0\\
       0 & 1 & 0 & -\frac{3}{2} & \frac{1}{2} & -\frac{1}{4}\\
       0 & 0 & 1 & \frac{3}{2} & -\frac{1}{2} & -\frac{1}{4}
     \end{pmatrix} 
   \] 
   \[
     \stackrel{E_{12}(-2)}{\leadsto}
     \begin{pmatrix}[ccc|ccc]
       1 & 0 & 0 & 8 & -1 & \frac{1}{2}\\
       0 & 1 & 0 & -\frac{3}{2} & \frac{1}{2} & -\frac{1}{4}\\
       0 & 0 & 1 & \frac{3}{2} & -\frac{1}{2} & -\frac{1}{4}
     \end{pmatrix} 
   \] 
   A sinistra della barra abbiamo la matrice identità, a destra abbiamo la matrice 
   \( E' = E_{12}(-2)E_{23}(1)E \). Allora:
   \[
     I_3 = E_{12}(-2)E_{23}(1)U = E_{12}(-2)E_{23}(1)E \cdot A =
   \] 
   \[
     = E'A
   \] 
   Osserviamo che (24:58)
  \end{example}
\end{figure}

\end{document}
