\documentclass[a4paper]{article}

\usepackage[utf8]{inputenc}
\usepackage[T1]{fontenc}
\usepackage{textcomp}
\usepackage[italian]{babel}
\usepackage{amsmath, amssymb}
\usepackage[makeroom]{cancel}
\usepackage{amsfonts}
\usepackage{mdframed}
\usepackage{xcolor}
\usepackage{float}
\usepackage{tikz}
\usepackage{pgfplots}
\usetikzlibrary{pgfplots.fillbetween}
\pgfplotsset{compat=newest, ticks=none}
\usepackage{graphicx}
\graphicspath{{./figures/}}

\pgfdeclarelayer{ft}
\pgfdeclarelayer{bg}
\pgfsetlayers{bg,main,ft}

\usepackage{import}
\usepackage{pdfpages}
\usepackage{transparent}
\usepackage{xcolor}

\usepackage{hyperref}
\hypersetup{
    colorlinks=false,
}

\usepackage{ntheorem}
\newtheorem{theorem}{Teorema}

% Useful definitions frame
\theoremstyle{break}
\theoremheaderfont{\bfseries}
\newmdtheoremenv[%
	linecolor=gray,leftmargin=0,%
	rightmargin=0,
	innertopmargin=8pt,%
	ntheorem]{define}{Definizioni utili}[section]

% Example frame
\theoremstyle{break}
\theoremheaderfont{\bfseries}
\newmdtheoremenv[%
	linecolor=gray,leftmargin=0,%
	rightmargin=0,
	innertopmargin=8pt,%
	ntheorem]{example}{Esempio}[section]

% Important definition frame
\theoremstyle{break}
\theoremheaderfont{\bfseries}
\newmdtheoremenv[%
	linecolor=gray,leftmargin=0,%
	rightmargin=0,
	backgroundcolor=gray!40,%
	innertopmargin=8pt,%
	ntheorem]{definition}{Definizione}[section]

% Exercise frame
\theoremstyle{break}
\theoremheaderfont{\bfseries}
\newmdtheoremenv[%
	linecolor=gray,leftmargin=0,%
	rightmargin=0,
	innertopmargin=8pt,%
	ntheorem]{exercise}{Esercizio}[section]


% figure support
\usepackage{import}
\usepackage{xifthen}
\pdfminorversion=7
\usepackage{pdfpages}
\usepackage{transparent}
\newcommand{\incfig}[1]{%
	\def\svgwidth{\columnwidth}
	\import{./figures/}{#1.pdf_tex}
}

\pdfsuppresswarningpagegroup=1

% Matrices
\makeatletter
\renewcommand*\env@matrix[1][*\c@MaxMatrixCols c]{%
  \hskip -\arraycolsep
  \let\@ifnextchar\new@ifnextchar
  \array{#1}}
\makeatother

\begin{document}
\begin{titlepage}
	\begin{center}
		\vspace*{1cm}

		\Huge
		\textbf{Probabilità e Statistica\\Esercizi}

		\vspace{0.5cm}
		\LARGE
		UniVR - Dipartimento di Informatica

		\vspace{1.5cm}

		\textbf{Fabio Irimie}

		\vfill


		\vspace{0.8cm}


		2° Semestre 2023/2024

	\end{center}
\end{titlepage}


\tableofcontents
\pagebreak

% Info sul corso
% Programma su moodle
% Lezioni sia in presenza che su online
% Ogni terza ora del lunedì si discutono gli esercizi votati nel sondaggio su moodle
% Esercizi da fare a gruppi con un punto extra per il voto finale
% Esame esame molto simile agli esercizi assegnati
% Esame di 2 ore con 4 esercizi

\section{Numeri complessi}
\subsection{Insiemi di numeri}
I numeri sono divisi in insiemi in base alle operazioni che si possono fare con essi:
\begin{itemize}
	\item
	      I numeri sono stati pensati per contare e per farlo è stato definito l'insieme dei numeri naturali
	      che è definito come \[  \mathbb{N} = \{0, 1, 2, 3, \ldots\} \]
	\item
	      Per fare operazioni di sottrazione è stato definito l'insieme dei numeri interi che è
	      definito come \[ \mathbb{Z} = \{\ldots, -3, -2, -1, 0, 1, 2, 3, \ldots\} \]
	\item
	      Per fare operazioni di divisione è stato definito l'insieme dei numeri razionali che è
	      definito come \[ \mathbb{Q} = \left\{ \frac{p}{q} \mid p, q \in \mathbb{Z}, q \neq 0 \right\} \]
	\item
	      Per fare operazioni di radice quadrata è stato definito l'insieme dei numeri reali che è
	      definito come \[ \mathbb{R} = \left\{ x \mid x \in \mathbb{Q} \right\} \]
	\item
	      Infine, per fare operazioni di radice quadrata di numeri negativi è stato definito
	      l'insieme dei numeri complessi che è definito come
	      \[ \mathbb{C} = \left\{ z \mid z = a + bi,\;\;\; a, b \in \mathbb{R}, i^2 = -1 \right\} \]
\end{itemize}

\noindent Ognuno di questi insiemi è un sottoinsieme dell'insieme successivo, ovvero
\[ \mathbb{N} \subset \mathbb{Z} \subset \mathbb{Q} \subset \mathbb{R} \subset \mathbb{C} \]

\noindent Le equazioni non risolvibili in un insieme vengono risolte in un insieme successivo, ad esempio
\[ x^2 + 1 = 0 \] non ha soluzioni in $\mathbb{R}$, ma ha soluzioni in $\mathbb{C}$.

\begin{figure}[H]
	\begin{theorem}
		\textbf{(Teorema fondamentale dell'algebra)}\\
		Qualsiasi equazione di forma:
		\[
			a_nx^n + a_{n-1}x^{n-1} + \ldots + a_1x + a_0 = 0
		\]
		dove
		\[
			n \in \mathbb{N},\;\; a_0, a_1, \ldots, a_n \in \mathbb{C},\;\; a_n \neq 0
		\]
		ed \( x \) è un incognita, ammette \( n \) soluzioni

	\end{theorem}

	\begin{define}
		\[
			a_nx^n + a_{n-1}x^{n-1} + \ldots + a_1x + a_0 \;\; \text{con} \;\; a_n \neq 0
		\]
		è detto \textbf{polinomio di grado \( n \)} con \textbf{coefficienti} \( a_0, \ldots, a_n \in \mathbb{C} \)
	\end{define}
\end{figure}

\subsection{Numeri immaginari}
Aggiungiamo ai numeri reali un "nuovo" numero $i$ che è definito come $i^2 = -1$. Questo numero
è detto: \textbf{unità immaginaria}. Per agevolare le operazioni con i numeri immaginari si
definisce l'insieme dei \textbf{numeri complessi} in modo da poter moltiplicare e sommare un
numero reale con un numero immaginario:
\[
	\mathbb{C} = \{ a + bi \;|\; a,b \in \mathbb{R} \}
\]

\( z = a + bi \) è detta \textbf{forma algebrica} di un numero complesso \( z \in \mathbb{C} \).
\[
	a = \Re(z) \quad \text{è detta parte reale di } z
\]
\[
	b = \Im(z) \quad \text{è detta parte immaginaria di } z
\]

\begin{figure}[H]
	\begin{define}
		Per agevolare la scrittura, al posto di scrivere:
		\[
			a + (-b)i
		\]
		si scrive:
		\[
			a - bi
		\]
	\end{define}
\end{figure}

\subsubsection{Esempi}
\begin{figure}[H]
	\begin{example}
		\begin{itemize}
			\item $3 + 2i$
			\item $-12 + \frac{1}{2}i$
			\item $3-\sqrt{2}i $
			\item $1+0 \cdot i = 1\;\; \in \mathbb{R}$
		\end{itemize}
	\end{example}
\end{figure}

\subsection{Operazioni tra i numeri complessi}
\subsubsection{Somma}
\begin{figure}[H]
	\begin{definition}
		L'addizione tra due numeri complessi è definita come:
		\[
			z_1 = a + bi \quad z_2 = c + di \quad \in \mathbb{C}
		\]
		\[
			z_1 + z_2 = (a + bi) + (c + di) = (a + c) + (b + d)i
		\]
	\end{definition}
\end{figure}

\begin{figure}[H]
	\begin{example}
		\[
			z_1 = 6 + 7i \quad z_2 = -12 + 1732i
		\]
		\[
			z_1 + z_2 = (6 + 7i) + (-12 + 1732i) = -6 + 1739i
		\]
	\end{example}
\end{figure}

\subsubsection{Prodotto}
\begin{figure}[H]
	\begin{definition}
		Il prodotto tra due numeri complessi è definito come:
		\[
			z_1 = a + bi \quad z_2 = c + di \quad \in \mathbb{C}
		\]
		\[
			z_1 \cdot z_2 = (a + bi) \cdot (c + di) = ac + adi + bci + bdi^2
		\]
		visto che $i^2 = -1$ si ha che $bdi^2 = -bd$ quindi
		\[
			z_1 \cdot z_2 = ac + adi + bci - bd = (ac - bd) + (ad + bc)i
		\]
	\end{definition}

	\begin{example}
		\[
			z_1 = 3 + 2i \quad z_2 = 10 - i
		\]
		\[
			z_1 \cdot z_2 = (3 + 2i) \cdot (10 - i) = 30 - 3i + 20i - 2i^2 = 32 + 17i
		\]
	\end{example}
\end{figure}

\subsubsection{Sottrazione}
Notiamo che per ogni numero complesso \( z = a + bi \; \in \mathbb{C} \), il numero complesso
\( -a -bi \) è l'unico numero complesso tale che \( z + (-z) = 0 \). Questo numero complesso è
detto \textbf{opposto} di \( z \) e si indica con \( -z \).

\begin{figure}[H]
	\begin{definition}
		La sottrazione tra due numeri complessi è definita come:
		\[
			z_1 = a + bi \quad z_2 = c + di \quad \in \mathbb{C}
		\]
		\[
			z_1 - z_2 = z_1 + (-z_2) = (a + bi) - (c + di) = (a - c) + (b - d)i
		\]
	\end{definition}

	\begin{example}
		\[
			z_1 = 3 + 2i \quad z_2 = 10 - i
		\]
		\[
			z_1 - z_2 = (3 + 2i) - (10 - i) = -7 + 3i
		\]
	\end{example}
\end{figure}

\subsubsection{Divisione}
\begin{definition}
	La divisione tra due numeri complessi è definita come:
	\[
		z_1, z_2, z_2 \neq 0 \quad \in \mathbb{C}
	\]
	Definiamo \( \frac{1}{z_2} \) come l'unico numero complesso tale che:
	\[
		z_2 \cdot \frac{1}{z_2} = 1
	\]
	\[
		\frac{z_1}{z_2} = z_1 \cdot \frac{1}{z_2}
	\]
	Sia \( z = a + bi\;\; \in \mathbb{C} \) e \( z \neq 0 \). Supponiamo che \( z'= c + di \) sia
	un numero complesso tale che \( z \cdot z' = 1 \), cioè:
	\[
		1 = z \cdot z' = (a + bi) \cdot (c + di) = (ac - bd) + (ad + bc)i
	\]
	Abbiamo \( ac-bd=1 \) e \( ad+bc=0 \).

	\noindent Possiamo trovare \( c \) sostituendo \( d = \frac{-1-ac}{b} \) nella prima equazione:
	\[
		c = -\frac{ad}{b} \quad d = \frac{-(1-ac)}{b} = \frac{1-ac}{b}
	\]
	\[
		c = \frac{-a (\frac{-1+ac}{b})}{b} = \frac{-a (\frac{-1+ac}{b})}{b} \cdot \frac{b}{b} = \frac{-a (-1 + ac)}{b^2}
	\]
	\[
		cb^2 = a - a^2c
	\]
	\[
		c(a^2 + b^2) = a
	\]
	\[
		c = \frac{a}{a^2 + b^2}
	\]
	\noindent Possiamo trovare \( d \) sostituendo \( c = \frac{-ad}{b} \) nella seconda equazione:
	\[
		d = \frac{-bc}{a} \quad c = \frac{-(1-bd)}{a} = \frac{1-bd}{a}
	\]
	\[
		d = \frac{-b (\frac{1-bd}{a})}{a} = \frac{-b (\frac{1-bd}{a})}{a} \cdot \frac{a}{a} = \frac{-b (1 - bd)}{a^2}
	\]
	\[
		ad^2 = b - b^2d
	\]
	\[
		d(a^2 + b^2) = b
	\]
	\[
		d = \frac{b}{a^2 + b^2}
	\]
	Quindi:
	\[
		z' = \frac{a}{a^2 + b^2} - \frac{b}{a^2 + b^2}i = \frac{a - bi}{a^2 + b^2}
	\]
	di conseguenza \[ \frac{1}{z} = \frac{a-bi}{a^2+b^2} \]

	\noindent Siano \( z_1 = a+bi, z_2 = c+di \neq 0 \in \mathbb{C} \). Definiamo:
	\[
		\frac{z_1}{z_2} = z_1 \cdot \frac{1}{z_2} = z_1 \cdot \frac{c-di}{c^2+d^2} = \frac{ac+bd}{c^2+d^2} + \frac{bc-ad}{c^2+d^2}i
	\]
\end{definition}

\begin{figure}[H]
	\begin{example}
		\[
			\frac{1+2i}{2-i} = \left(1+2i\right)\left(\frac{2}{5}+\frac{1}{5}i\right) = \left( \frac{2}{5}-\frac{2}{5} \right) + \left( \frac{1}{5} + \frac{4}{5} \right) i = i
		\]
	\end{example}
\end{figure}

\noindent Un trucco per dividere i numeri complessi è moltiplicare per \( 1 \) la frazione:
\[
	(a+bi)(a-bi) = a^2 \cancel{+abi} \cancel{- abi} + b^2 = a^2 + b^2 \quad \in \mathbb{R}
\]
In questo modo si arriva ad ottenere un numero reale al denominatore facilitando la divisione.
\begin{figure}[H]
	\begin{example}
		\[
			\frac{1+2i}{2-i}
		\]
		\[
			\left( \frac{1+2i}{2-i} \right) \left( \frac{2+i}{2+i} \right) = \frac{(1+2i)(2+i)}{2^2+(-1)^2} =
		\]
		\[
			= \frac{(1+2i)(2+i)}{5} = \frac{2+4i+i+2i^2}{5} = \frac{2+5i-2}{5} = \frac{5i}{5} = i
		\]
	\end{example}
\end{figure}

\subsection{Coniugato e modulo}
\subsubsection{Coniugato}
Sia \( z = a +bi \in \mathbb{C} \). Il numero complesso \( \overline{z} = a - bi \) è detto
\textbf{coniugato} di \( z \).

\subsubsection{Modulo}
Il \textbf{modulo} di \( z \) è definito come:
\[
	|z| = \sqrt{a^2 + b^2} \quad \in \mathbb{R}
\]

\subsubsection{Proprietà}
Siano \( z_1 = a + bi, z_2 = c + di \quad \in \mathbb{C} \)
\begin{enumerate}
	\item \( z_1 \overline{z_1} = a^2 + b^2 = |z_1|^2 \)
	\item \( \overline{z_1 + z_2} = \overline{(a+c) + (b+d)i} = (a-bi) + (c-di) = \overline{z_1} + \overline{z_2} \)
	\item \( \overline{z_1 z_2} = \overline{z_1} \cdot \overline{z_2} \)
	\item Se \[ z_1 \neq 0, \; \overline{\frac{1}{z_1}} = \frac{1}{\overline{z_1}} \]
	      Infatti:
	      \[
		      \overline{z_1} \cdot \left( \overline{\frac{1}{z_1}} \right) = \left(\overline{ z_1 \cdot \frac{1}{z_1}} \right) = \overline{1+0i} = 1 - 0i = 1
	      \]
	\item Se \( z_2 \neq 0 \) allora:
	      \[
		      \left( \overline{\frac{z_1}{z_2}} \right) = \left( \overline{z_1 \cdot \frac{1}{z_2}} \right) = \overline{z_1} \cdot \overline{\frac{1}{z_2}} = \overline{z_1} \cdot \frac{1}{\overline{z_2}} = \frac{\overline{z_1}}{\overline{z_2}}
	      \]
	\item Se \( z_1 \neq 0 \), allora
	      \[
		      \frac{1}{z_1} \stackrel{def}{=} \frac{a-bi}{a^2+b^2}= \frac{\overline{z_1}}{|z_1|^2}
	      \]
\end{enumerate}

\begin{example}
	\[
		z = \frac{1 + i}{2 - i} = \left( 1+i \right) \left( \frac{1}{2-i} \right)
	\]
	\[
		\frac{1}{2-i} = \frac{2+i}{5} = \frac{2+i}{5} = \frac{2}{5} + \frac{1}{5}i
	\]
	\[
		z = \left( 1+i \right) \left( \frac{2}{5} + \frac{1}{5}i \right) = \left( \frac{2}{5}-\frac{1}{5} \right) + \left( \frac{2}{5} + \frac{1}{5} \right)i = \frac{1}{5} + \frac{3}{5}i
	\]
	\[
		\overline{z} = \frac{1}{5} - \frac{3}{5}i
	\]
\end{example}

\subsection{Coordinate polari}
Per ogni numero complesso si ha una coppia di coordinate:
\[
	z = a + bi \quad \in \mathbb{C}
\]
\[ (a,b) = (\Re(z), \Im(z)) \in \mathbb{R}^2 \]


\begin{figure}[H]
	\begin{center}
		\begin{tikzpicture}[scale=0.8, domain=0:13]
			\draw[->] (-0.5,0) -- (6,0) node[right] {$\Re$};
			\draw[->] (0,-0.5) -- (0,5) node[above] {$\Im$};

			\draw[fill, blue] (5, 4) circle (4pt) node[above right] {$z = (a,b)$};

			\draw[blue, thick] (0, 0) -- (5, 4);

			\draw[red, thick] (3, 0) arc (0:38:3) node[right] {$\alpha$};
		\end{tikzpicture}
	\end{center}

	\caption{Rappresentazione di un numero complesso}
\end{figure}
\noindent Possiamo esprimere \( z \) in coordinate polari \( (r, \alpha) \) dove \( r \) è la lunghezza del
segmento \( OZ \), detto \textbf{raggio polare}, ed \( \alpha \) è l'angolo compreso tra l'asse delle x
e \( OZ \) in senso antiorario. \( \alpha \) viene misurato in radianti

\begin{figure}[H]
	\begin{example}
		\[
			z_1 = (1,0) \to 1
		\]
		\[
			z_2 = (1, \frac{\pi}{2}) \to i
		\]
		\[
			z_3 = (1, \pi) \to -1
		\]
		\[
			z_4 = (1, \frac{3\pi}{2}) \to -i
		\]

		\begin{center}
			\begin{tikzpicture}[scale=0.5, domain=0:13]
				\draw[->] (-4,0) -- (4,0) node[right] {$\Re$};
				\draw[->] (0,-4) -- (0,4) node[above] {$\Im$};

				\draw[fill, blue] (3, 0) circle (4pt) node[below right] {$z_1 = (1,0)$};
				\draw[fill, blue] (0, 3) circle (4pt) node[above left] {$z_2 = (1, \frac{\pi}{2})$};
				\draw[fill, blue] (-3, 0) circle (4pt) node[below left] {$z_3 = (1, \pi)$};
				\draw[fill, blue] (0, -3) circle (4pt) node[below left] {$z_4 = (1, \frac{3\pi}{2})$};

			\end{tikzpicture}
		\end{center}
		\caption{Esempi di numeri complessi in coordinate polari}
	\end{example}
\end{figure}

\subsection{Forma trigonometrica di un numero complesso}
Dato un \( z = (r, \alpha) \) in coordinate polari, vogliamo ricavare la forma algebrica. Per
fare ciò usiamo il seno e il coseno:
\[
	\cos(\alpha) = \frac{a}{r} \quad \sin(\alpha) = \frac{b}{r}
\]
\begin{figure}[H]
	\begin{center}
		\begin{tikzpicture}[scale=0.8, domain=0:13]
			\draw[->] (-0.5,0) -- (6,0) node[right] {$\Re$};
			\draw[->] (0,-0.5) -- (0,5) node[above] {$\Im$};

			\draw[fill, blue] (5, 4) circle (4pt);

			\draw[blue, thick] (0, 0) -- (5, 4);

			\draw[red, thick] (3, 0) arc (0:38:3) node[right] {$\alpha$};

			\draw[green, dashed] (5, 0) -- (5, 4) node[midway, right] {$b$};
			\draw[green, dashed] (0, 4) -- (5, 4) node[midway, above] {$a$};
		\end{tikzpicture}
	\end{center}

	\caption{Forma trigonometrica di un numero complesso}
\end{figure}
\begin{figure}[H]
	\begin{definition}
		La \textbf{forma trigonometrica} di un numero complesso è definita come:
		\[
			z = (r \cdot \cos(\alpha)) + (r \cdot \sin(\alpha)i) = r \cdot (\cos(\alpha) + i \cdot \sin(\alpha))
		\]
		\[
			r = |z| = \sqrt{a^2 + b^2}
		\]
        \[
        \alpha = \begin{cases}
            \frac{\pi}{2} \quad \text{se} \; a = 0,\; b > 0\\
            -\frac{\pi}{2} \quad \text{se} \; a = 0,\; b < 0\\
            \text{non definito} \quad \text{se} \; a = 0,\; b = 0\\
            \arctan\left(\frac{b}{a}\right) \quad \text{se} \; a > 0,\; b \; \text{qualsiasi}\\
            \arctan\left(\frac{b}{a}\right) + \pi \quad \text{se} \; a < 0,\; b \ge 0\\
            \arctan\left(\frac{b}{a}\right) - \pi \quad \text{se} \; a < 0,\; b < 0\\
        \end{cases}
        \] 
	\end{definition}
\end{figure}

\begin{figure}[H]
	\begin{example}
		\[
			1 = \cos(0) + i \cdot \sin(0)
		\]
		\[
			i = \cos(\frac{\pi}{2}) + i \cdot \sin(\frac{\pi}{2})
		\]
		\[
			-1 = \cos(\pi) + i \cdot \sin(\pi)
		\]
		\[
			-i = \cos(\frac{3\pi}{2}) + i \cdot \sin(\frac{3\pi}{2})
		\]

	\end{example}
\end{figure}

\subsection{Prodotto di numeri complessi in forma trigonometrica}
\begin{figure}[H]
	\begin{definition}
		\[
			z_1 = r\left(\cos(\alpha) + i \sin(\alpha)\right), \quad z_2 = s\left(\cos(\beta) + i \sin(\beta)\right) \quad \in \mathbb{C}
		\]
		\vspace{0.1cm}
		\[
			z_1  z_2 = r  s  (\cos(\alpha) + i \sin(\alpha))  (\cos(\beta) + i \sin(\beta)) =
		\]
		\[
			= r  s  \left( (\cos{\alpha} \cos(\beta) - \sin(\alpha) \sin(\beta)) + (\cos(\alpha) \sin(\beta) + \sin(\alpha) \cos(\beta))i \right) =
		\]
		\[
			= r  s  \left( \cos(\alpha + \beta) + i \sin(\alpha + \beta) \right)
		\]
	\end{definition}
\end{figure}
\subsection{Formula di de Moivre}
Dati \( n \in \mathbb{N}, \quad z = r(\cos(\alpha) + i \sin(\alpha)) \in \mathbb{C} \)
\[
	z^n = r^n \cdot (\cos(n\alpha) + i \sin(n\alpha))
\]

\begin{figure}[H]
	\begin{example}
		\[
			z = \sqrt{3} + i = 2 \cdot \left( \cos(\frac{\pi}{6}) + i \sin(\frac{\pi}{6}) \right)
		\]
		\[
			z^6 = 2^6 \cdot \left( \cos(\frac{\pi}{6} \cdot 6) + i \sin(\frac{\pi}{6} \cdot 6) \right) = 64 \cdot \left( \cos(\pi) + i \sin(\pi) \right) = -64
		\]
	\end{example}
\end{figure}

\subsection{Definizione di radice n-esima}
\[
	y \in  \mathbb{C}, \quad n \in \mathbb{N}
\]
Si dicono \textbf{radici n-esime} di \( y \) le soluzioni dell'equazione \( x^n = y \).

\subsection{Teorema delle radici n-esime}
\begin{figure}[H]
	\begin{theorem}
		Siano \( y \in \mathbb{C} \) e \( n \in \mathbb{N} \). Esistono precisamente \( n \) radici n-esime
		complesse distinte \( z_0, z_1, \ldots, z_{n-1} \) di \( y \). Se \( y = r(\cos(\alpha)+i\sin(\alpha)) \),
		allora per \( k = 0, \ldots, n-1 \) :
		\[
			z_k = \sqrt[n]{r} \left( \cos\left(\frac{\alpha + 2k\pi}{n}\right) + i \sin\left(\frac{\alpha + 2k\pi}{n}\right) \right)
		\]
		Si somma \( 2k \pi  \) per ottenere tutte le radici n-esime, siccome \( \sin \) e \( \cos \) sono periodiche.
		\label{th:radici_n-esime}
	\end{theorem}
\end{figure}

\subsubsection{Dimostrazione}
Per la formula di de Moivre sappiamo che:
\[
	z_k^n = \left( \sqrt[n]{r} \right)^n \left( \cos{\alpha + (2 \pi )k} + i \sin{\alpha + (2 \pi )k} \right)  =
\]
\[
	= r \left( \cos{\alpha} + i \sin{\alpha} \right) = y
\]

\begin{figure}[H]
	\centering
	\begin{tikzpicture}
		\draw[->] (-2,0) -- (2,0) node[right] {$\Re$};
		\draw[->] (0,-2) -- (0,2) node[above] {$\Im$};
		\draw (0,0) circle (1.5cm);

		\draw[red, thick] (0,0) -- (1.05,1.05);
		\draw[blue, thick] (0.5,0) arc (0:45:0.5) node[right] {$\alpha$};

		\draw[green, dashed] (1.05,1.05) -- (1.05,0) node[below] {$\cos{x}$};

		\draw[green, dashed] (1.05,1.05) -- (0,1.05) node[left] {$\sin{x}$};

	\end{tikzpicture}
	\caption{Circonferenza goinometrica}
\end{figure}

\begin{figure}[H]
	\centering
	\begin{tikzpicture}
		% Draw the axes
		\draw[->] (-0.5,0) -- (7,0) node[right] {$\alpha$};
		\draw[->] (0,-2) -- (0,2) node[above] {$\sin{x}$};

		% plot sine
		\draw[domain=0:7, smooth, variable=\x, blue] plot ({\x}, {sin(\x r)});

		% radians labels
		\foreach \x/\xtext in {
				1.5708/\frac{\pi}{2},
				3.14159/\pi,
				4.71239/\frac{3\pi}{2},
				6.28319/2\pi,
			}
		\draw (\x,0.1) -- (\x,-0.1) node[below] {$\xtext$};

	\end{tikzpicture}
	\caption{Funzione seno}
\end{figure}

\begin{figure}[H]
	\centering
	\begin{tikzpicture}
		% Draw the axes
		\draw[->] (-0.5,0) -- (7,0) node[right] {$\alpha$};
		\draw[->] (0,-2) -- (0,2) node[above] {$\cos{x}$};

		% plot cosine
		\draw[domain=0:7, smooth, variable=\x, red] plot ({\x}, {cos(\x r)});

		% radians labels
		\foreach \x/\xtext in {
				1.5708/\frac{\pi}{2},
				3.14159/\pi,
				4.71239/\frac{3\pi}{2},
				6.28319/2\pi,
			}
		\draw (\x,0.1) -- (\x,-0.1) node[below] {$\xtext$};
	\end{tikzpicture}
	\caption{Funzione coseno}
\end{figure}

Quindi \( z_0, \ldots, z_{n-1} \) sono soluzioni di \( y = x^n \) , cioè sono radici
n-esime di \( y \).

Siccome il periodo di \( \sin{} \) e \( \cos{} \) è \( 2 \pi \), le radici n-esime sono
tutte distinte.

\subsection{Radici quadrate di numeri reali negativi}
Sia \( a \in \mathbb{R} \; \subseteq \mathbb{C} \) tale che \( a < 0 \). Esistono
precisamente due radici quadrate di \( a \) in \( \mathbb{C} \). Infatti, abbiamo:

\begin{figure}[H]
	\centering
	\begin{tikzpicture}
		\draw[->] (-2,0) -- (2,0) node[right] {$\Re$};
		\draw[->] (0,-2) -- (0,2) node[above] {$\Im$};

		\draw[fill, blue] (-1, 0) circle (2pt) node[below, align=center, yshift=-2] {$-a$\\$\alpha = \pi $};
		\draw[fill, blue] (1, 0) circle (2pt) node[below, align=center, yshift=-2] {$a$\\$\alpha = 0$};

		\draw[red, thick, ->] (1, 0) arc (0:175:1) node[midway, above right] {$\alpha$};

		\draw[fill, green] (0, 1) circle (2pt) node[above left, align=center] {$\frac{\pi }{2}$};
		\draw[fill, green] (0, -1) circle (2pt) node[below left, align=center] {$\frac{3 \pi }{2}$};
	\end{tikzpicture}
	\caption{Radici quadrate di numeri reali negativi}
\end{figure}
\[
	a = (-a) (\cos{\pi } + i \sin{\pi })
\]
Per il teorema \ref{th:radici_n-esime}:
\[
	z_0 = \sqrt{a} \left( \cos{\frac{\pi}{2}} + i \sin{\frac{\pi}{2}} \right) = i \sqrt{-a}
\]
\[
	z_1 = \sqrt{-a} \left( \cos{\frac{3\pi}{2}} + i \sin{\frac{3\pi}{2}} \right) = -i \sqrt{-a}
\]

\begin{figure}[H]
	\begin{define}
		Se abbiamo un polinomio della forma:
		\[
			ax^2 + bx + c, \quad a,\;b,\;c \in \mathbb{R}
		\]
		Le soluzioni sono:
		\[
			\frac{-b \pm \sqrt{b^2 -4ac}}{2a}
		\]
		\[
			\Delta = b^2 - 4ac
		\]
		In \( \mathbb{C} \) esistono 2 soluzioni anche se \( \Delta < 0 \).
	\end{define}
\end{figure}

\section{Sistemi lineari e matrici}
\subsection{Sistemi lineari}
Un \textbf{sistema lineare} è un insieme di \( m \) equazioni in \( n \) incognite
che può essere scritto nella forma:
\[
	\begin{cases}
		a_{11}x_1 + a_{12}x_2 + \ldots + a_{1n}x_n = b_1 \\
		a_{21}x_1 + a_{22}x_2 + \ldots + a_{2n}x_n = b_2 \\
		\vdots                                           \\
		a_{m1}x_1 + a_{m2}x_2 + \ldots + a_{mn}x_n = b_m \\
	\end{cases}
\]
dove \( b_k,\; a_{ij} \in \mathbb{C} \) oppure \( \mathbb{R} \) per \( 1 \le i \le m,\;
1 \le j \le n,\; 1 \le k \le m\). Se i \textbf{termini noti} sono tutti nulli il sistema è detto
\textbf{omogeneo}. Una n-upla \( (x_1, \ldots, x_n) \) di numeri complessi (o reali) è
una soluzione se soddisfa tutte le \( m \) equazioni.

\begin{example}
	Presa in considerazione la seguente tabella nutrizionale di cereali (per porzione):
	\begin{center}
		\begin{tabular}{c|c|c}
			                & Cheerios & Quakers \\
			\hline
			Proteine (g)    & 4        & 3       \\
			Carboidrati (g) & 20       & 18      \\
			Grassi (g)      & 2        & 5       \\
		\end{tabular}
	\end{center}
	Quante porzioni di Cheerios e Quakers dobbiamo mangiare per ottenere \( 9g \) di
	proteine, \( 48g \) di carboidrati e \( 89g \) di grassi?
	\[
		\begin{cases}
			4C + 3Q = 9 \quad \text{(P)}    \\
			20C + 18Q = 48 \quad \text{(C)} \\
			2C + 5Q = 8 \quad \text{(G)}
		\end{cases}
	\]

	Per risolvere il sistema lineare:
	\begin{itemize}
		\item
		      Moltiplichiamo le per \( \frac{1}{4} \)
		      e otteniamo un sistema lineare \textbf{equivalente} (cioè con
		      \textbf{esattamente} le stesse soluzioni):
		      \[
			      (P') \quad C + \frac{3}{4}Q = \frac{9}{4}
		      \]
		      \[
			      (C) \quad 20C + 18Q = 48
		      \]
		      \[
			      (G) \quad 2C + 5Q = 8
		      \]
		\item Calcoliamo \( (C)-20(P') \) e \( (G)-2(P') \) e otteniamo:
		      \[
			      (P') \quad C + \frac{3}{4}Q = \frac{9}{4}
		      \]
		      \[
			      (C') \quad 0C + 15Q = 18
		      \]
		      \[
			      (G') \quad 0C + \frac{7}{2}Q = \frac{7}{2}
		      \]
		\item Moltiplichiamo \( (C') \) per \( \frac{1}{3} \) e otteniamo:
		      \[
			      (P') \quad C + \frac{3}{4}Q = \frac{9}{4}
		      \]
		      \[
			      (C') \quad 0C + Q = 1
		      \]
		      \[
			      (G') \quad 0C + \frac{7}{2}Q = \frac{7}{2}
		      \]
		\item Calcoliamo \( (G') - \frac{7}{2} (C") \) e otteniamo:
		      \[
			      (P') \quad C + \frac{3}{4}Q = \frac{9}{4}
		      \]
		      \[
			      (C') \quad 0C + Q = 1
		      \]
		      \[
			      (G') \quad 0C + 0Q = 0
		      \]
	\end{itemize}

  \noindent Otteniamo dunque che \( Q = 1 \) e \( C = \frac{9}{4} - \frac{3}{4} = \frac{7}{4} \) 

  \vspace{0.5cm}

  \noindent Per agevolare la risoluzione del sistema lineare si può utilizzare una matrice:
  \begin{itemize}
    \item \textbf{R1} = Riga 1
    \item \textbf{R2} = Riga 2
    \item \textbf{R3} = Riga 3
  \end{itemize}
	\[
		\begin{pmatrix}[cc|c]
			4  & 3  & 9  \\
			20 & 18 & 48 \\
			2  & 5  & 8
		\end{pmatrix}
	\]
  \[\downarrow \frac{1}{4} \cdot R1 \]
	\[
		\begin{pmatrix}[cc|c]
			1  & \frac{3}{4} & \frac{9}{4} \\
			20 & 18          & 48          \\
			2  & 5           & 8
		\end{pmatrix}
	\]
  \[
  \downarrow R2 - 20 \cdot R1
  \] 
  \[
  \downarrow R3 - 2 \cdot R1
  \] 
	\[
		\begin{pmatrix} [cc|c]
			1 & \frac{3}{4} & \frac{9}{4} \\
			0 & 3           & 3           \\
			0 & \frac{7}{2} & \frac{7}{2}
		\end{pmatrix}
	\]
  \[
  \downarrow \frac{1}{3} \cdot R2
  \] 
	\[
		\begin{pmatrix}[cc|c]
			1 & \frac{3}{4} & \frac{9}{4} \\
			0 & 1           & 1           \\
			0 & \frac{7}{2}           &\frac{7}{2} 
		\end{pmatrix}
	\]
  \[
  \downarrow R3 - \frac{7}{2} \cdot R2
  \] 
  \[
    \begin{pmatrix}[ccc|c]
      1 & \frac{3}{4} & \frac{9}{4} & \frac{9}{4}\\
      0 & 1           & 1           & 1 \\
      0 & 0           & 0 & 0 
  \end{pmatrix} 
  \] 
  \noindent Otteniamo dunque che \( Q = 1 \) e \( C = \frac{9}{4} - \frac{3}{4} = \frac{7}{4} \)
\end{example}

\subsection{Definizione}
\begin{figure}[H]
	\begin{definition}
		Siano \( m\;,n,;\ < 1 \). Una tabella \( A \) tale che:
		\[
			A = \begin{pmatrix}
				a_{11} & a_{12} & \ldots & a_{1n} \\
				a_{21} & a_{22} & \ldots & a_{2n} \\
				\vdots & \vdots & \ddots & \vdots \\
				a_{m1} & a_{m2} & \ldots & a_{mn} \\
			\end{pmatrix}
			= (a_{ij})_{m \times n}
		\]
		di \( m \times n \) elementi di \( \mathbb{C} \) disposti in \( m \) righe e \( n \) colonne
		si chiama una \textbf{matrice} di \textbf{dimensione} \( m \times n \).
		Gli elementi si chiamano \textbf{coefficienti} (o entrate) della matrice e sono
		contrassegnati con un doppio indice \( ij \) dove \( i \) indica la riga e \( j \)
		la colonna di appartenenza.

		\vspace{0.5cm}
		L'insieme di tutte le matrici di dimensione \( m \times n \) con entrate in \( \mathbb{C} \)
		si indica con \( M_{m \times n}(\mathbb{C}) \).

		\vspace{0.5cm}
		L'insieme di tutte le matrici di dimensione \( m \times n \) con entrate in \( \mathbb{R} \)
		si indica con \( M_{m \times n}(\mathbb{R}) \).
	\end{definition}
\end{figure}

\begin{figure}[H]
	\begin{example}
		\[
			\begin{pmatrix} 3 & i & 2+7i \\
                0 & 1 & \pi
			\end{pmatrix} \in M_{2 \times 3}(\mathbb{C})
		\]
		\[
			\begin{pmatrix}
				0 & 1  \\
				1 & -1
			\end{pmatrix} \in M_{2 \times 2}(\mathbb{R}) \subseteq M_{2 \times 2}(\mathbb{C})
		\]
	\end{example}
\end{figure}

\subsection{Definizione}
Un sistema lineare di \( n \) incognite e \( m \) equazioni:
\[
	a_{11}x_1 + a_{12}x_2 + \ldots + a_{1n}x_n = b_1
\]
\[
	\vdots
\]
\[
	a_{m1}x_1 + a_{m2}x_2 + \ldots + a_{mn}x_n = b_m
\]
può essere rappresentato nella forma \textbf{matriciale}:
\[
	Ax = b
\]
\[
	\underbrace{
		A = \begin{pmatrix}
			a_{11} & a_{12} & \ldots & a_{1n} \\
			a_{21} & a_{22} & \ldots & a_{2n} \\
			\vdots & \vdots & \ddots & \vdots \\
			a_{m1} & a_{m2} & \ldots & a_{mn} \\
		\end{pmatrix}}_{\text{Matrice dei coefficienti}} \quad \underbrace{x = \begin{pmatrix}
			x_1    \\
			x_2    \\
			\vdots \\
			x_n
		\end{pmatrix}}_{\text{Vettore delle incognite}} \quad \underbrace{b = \begin{pmatrix}
			b_1    \\
			b_2    \\
			\vdots \\
			b_m
		\end{pmatrix}}_{\text{Vettore dei termini noti}}
\]
La matrice \[ (A \;|\; B) = \begin{pmatrix}[cccc|c]
		a_{1n} & a_{12} & \ldots & a_{1n} & b_1    \\
		\vdots & \vdots & \ddots & \vdots & \vdots \\
		a_{m1} & a_{m2} & \ldots & a_{mn} & b_n
	\end{pmatrix}  \] è detta \textbf{matrice aumentata}.

\begin{figure}[H]
	\begin{example}
		\[
			\begin{cases}
				2x_1 + 6x_2 + 3x_3 + 2x_4 = 4                    \\
				x_1 - 2x_2 + \frac{1}{2}x_3 + \frac{9}{4}x_4 = 1 \\
				-x_1 + x_2 - \frac{1}{2}x_3 - x_4 = \frac{2}{5}
			\end{cases}
		\]
		Scritto come matrice aumentata diventa:
		\[
			\begin{pmatrix}[cccc|c]
				2  & 6  & 3            & 2           & 4           \\
				1  & -2 & \frac{1}{2}  & \frac{9}{4} & 1           \\
				-1 & 1  & -\frac{1}{2} & -1          & \frac{2}{5}
			\end{pmatrix}
		\]
		\[
			\frac{1}{2}R1
		\]
		\[
      \begin{pmatrix}[cccc|c]
        1  & 3  & \frac{3}{2}  & 1 & 2            \\
        1  & -2 & \frac{1}{2}  & \frac{9}{4} & 1  \\
        -1 & 1  & -\frac{1}{2} & -1 & \frac{2}{5}
			\end{pmatrix}
		\]
		\[
			R2 - R1 \quad R3 + R1
		\]
		\[
      \begin{pmatrix}[cccc|c]
        1 & 3  & \frac{3}{2} & 1 & 2            \\
        0 & -5 & -1          & \frac{5}{4} & -1 \\
        0 & 4  & 1           & 0 & \frac{12}{5}
			\end{pmatrix}
		\]
		\[
			\frac{-1}{5}R2
		\]
		\[
			\begin{pmatrix}[cccc|c]
				1 & 3 & \frac{3}{2} & 1            & 2            \\
				0 & 1 & \frac{1}{5} & -\frac{1}{4} & \frac{1}{5}  \\
				0 & 4 & 1           & 0            & \frac{12}{5}
			\end{pmatrix}
		\]
		\[
			R3 - 4R2
		\]
		\[
			\begin{pmatrix}[cccc|c]
				1 & 3 & \frac{3}{2} & 1            & 2           \\
				0 & 1 & \frac{1}{5} & -\frac{1}{4} & \frac{1}{5} \\
				0 & 0 & \frac{1}{5} & 1            & \frac{4}{5}
			\end{pmatrix}
		\]
		\[
			5R3
		\]
		\[
			\begin{pmatrix}[cccc|c]
				1 & 3 & \frac{3}{2} & 1            & 2           \\
				0 & 1 & \frac{1}{5} & -\frac{1}{4} & \frac{1}{5} \\
				0 & 0 & 1           & 5            & 8
			\end{pmatrix}
		\]
	\end{example}
\end{figure}
\end{document}
