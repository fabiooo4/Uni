\documentclass[a4paper]{article}

\usepackage[utf8]{inputenc}
\usepackage[T1]{fontenc}
\usepackage{textcomp}
\usepackage[italian]{babel}
\usepackage{amsmath, amssymb}
\usepackage[makeroom]{cancel}
\usepackage{amsfonts}
\usepackage{mdframed}
\usepackage{xcolor}
\usepackage{float}
\usepackage{tikz}
\usepackage{tikz-cd}
\usepackage{nicematrix}
\usepackage{pgfplots}
\usetikzlibrary{fit}
\usetikzlibrary{pgfplots.fillbetween}
\pgfplotsset{compat=newest, ticks=none}
\usepackage{graphicx}
\graphicspath{{./figures/}}

\pgfdeclarelayer{ft}
\pgfdeclarelayer{bg}
\pgfsetlayers{bg,main,ft}

\usepackage{import}
\usepackage{pdfpages}
\usepackage{transparent}
\usepackage{xcolor}

\usepackage{hyperref}
\hypersetup{
  colorlinks=false,
}

\usepackage{ntheorem}
\newtheorem{theorem}{Teorema}

% Useful definitions frame
\theoremstyle{break}
\theoremheaderfont{\bfseries}
\newmdtheoremenv[%
linecolor=gray,leftmargin=0,%
rightmargin=0,
innertopmargin=8pt,%
ntheorem]{define}{Definizioni utili}[section]

% Example frame
\theoremstyle{break}
\theoremheaderfont{\bfseries}
\newmdtheoremenv[%
linecolor=gray,leftmargin=0,%
rightmargin=0,
innertopmargin=8pt,%
ntheorem]{example}{Esempio}[section]

% Important definition frame
\theoremstyle{break}
\theoremheaderfont{\bfseries}
\newmdtheoremenv[%
linecolor=gray,leftmargin=0,%
rightmargin=0,
backgroundcolor=gray!40,%
innertopmargin=8pt,%
ntheorem]{definition}{Definizione}[section]

% Exercise frame
\theoremstyle{break}
\theoremheaderfont{\bfseries}
\newmdtheoremenv[%
linecolor=gray,leftmargin=0,%
rightmargin=0,
innertopmargin=8pt,%
ntheorem]{exercise}{Esercizio}[section]


% figure support
\usepackage{import}
\usepackage{xifthen}
\pdfminorversion=7
\usepackage{pdfpages}
\usepackage{transparent}
\newcommand{\incfig}[1]{%
  \def\svgwidth{\columnwidth}
  \import{./figures/}{#1.pdf_tex}
}

\pdfsuppresswarningpagegroup=1

% Matrices
\makeatletter
\renewcommand*\env@matrix[1][*\c@MaxMatrixCols c]{%
  \hskip -\arraycolsep
  \let\@ifnextchar\new@ifnextchar
\array{#1}}
\makeatother

\begin{document}
\begin{titlepage}
	\begin{center}
		\vspace*{1cm}

		\Huge
		\textbf{Probabilità e Statistica\\Esercizi}

		\vspace{0.5cm}
		\LARGE
		UniVR - Dipartimento di Informatica

		\vspace{1.5cm}

		\textbf{Fabio Irimie}

		\vfill


		\vspace{0.8cm}


		2° Semestre 2023/2024

	\end{center}
\end{titlepage}


\tableofcontents
\pagebreak

\section{Scheda 2}
\subsection{Esercizio 1}
Si consideri
\[
	f: \mathbb{C}^3 \to \mathbb{C}^3, \quad f \left( \begin{pmatrix}
			x \\
			y \\
			z \\
		\end{pmatrix}  \right) =
	\begin{pmatrix}
		ix + y \\
		x + iy \\
		x + 3z
	\end{pmatrix}
\]

\begin{enumerate}
	\item[(a)] Verificare che \( f \) è lineare.

	      \vspace{1em}
	      \noindent Verifico se valgono le seguenti proprietà con \( v, w \in \mathbb{C}^3 \) e \( \alpha \in \mathbb{C} \):
	      \begin{itemize}
		      \item
		            \(
		            f(v + w) = f(v) + f(w)
		            \)
		            \[
			            f \left( \begin{pmatrix} v_1\\v_2\\v_3 \end{pmatrix} +
			            \begin{pmatrix} w_1\\w_2\\w_3 \end{pmatrix}  \right)
			            =
			            f \left( \begin{pmatrix} v_1\\v_2\\v_3 \end{pmatrix}  \right)
			            +
			            f \left( \begin{pmatrix} w_1\\w_2\\w_3 \end{pmatrix}  \right)
		            \]
		            \[
			            f \left( \begin{pmatrix} v_1 + w_1\\ v_2 + w_2\\ v_3 + w_3 \end{pmatrix}  \right)
			            =
			            \begin{pmatrix}
				            iv_1 + v_2 \\
				            v_1 + iv_2 \\
				            v_1 + 3v_3
			            \end{pmatrix}
			            +
			            \begin{pmatrix}
				            iw_1 + w_2 \\
				            w_1 + iw_2 \\
				            w_1 + 3w_3
			            \end{pmatrix}
		            \]
		            \[
			            \begin{pmatrix}
				            i(v_1 + w_1) + (v_2 + w_2) \\
				            (v_1 + w_1) + i(v_2 + w_2) \\
				            (v_1 + w_1) + 3(v_3 + w_3)
			            \end{pmatrix}
			            =
			            \begin{pmatrix}
				            iv_1 + v_2 + iw_1 + w_2 \\
				            v_1 + iv_2 + w_1 + iw_2 \\
				            v_1 + 3v_3 + w_1 + 3w_3
			            \end{pmatrix}
		            \]
		            \[
			            \begin{pmatrix}
				            i(v_1 + w_1) + (v_2 + w_2) \\
				            (v_1 + w_1) + i(v_2 + w_2) \\
				            (v_1 + w_1) + 3(v_3 + w_3)
			            \end{pmatrix}
			            =
			            \begin{pmatrix}
				            i(v_1 + w_1) + (v_2 + w_2) \\
				            (v_1 + w_1) + i(v_2 + w_2) \\
				            (v_1 + w_1) + 3(v_3 + w_3)
			            \end{pmatrix}
		            \]
		            \begin{center}
			            Quindi vale la proprietà
		            \end{center}

		      \item
		            \(
		            f(\alpha v) = \alpha f(v)
		            \)
		            \[
			            f \left( \alpha \begin{pmatrix} v_1\\v_2\\v_3 \end{pmatrix}  \right)
			            =
			            \alpha f \left( \begin{pmatrix} v_1\\v_2\\v_3 \end{pmatrix}  \right)
		            \]
		            \[
			            f \left( \begin{pmatrix} \alpha v_1\\\alpha v_2\\\alpha v_3 \end{pmatrix}  \right)
			            =
			            \alpha \begin{pmatrix}
				            iv_1 + v_2 \\
				            v_1 + iv_2 \\
				            v_1 + 3v_3
			            \end{pmatrix}
		            \]
		            \[
			            \begin{pmatrix}
				            i(\alpha v_1) + (\alpha v_2) \\
				            (\alpha v_1) + i(\alpha v_2) \\
				            (\alpha v_1) + 3(\alpha v_3)
			            \end{pmatrix}
			            =
			            \begin{pmatrix}
				            \alpha (iv_1 + v_2) \\
				            \alpha (v_1 + iv_2) \\
				            \alpha (v_1 + 3v_3)
			            \end{pmatrix}
		            \]
		            \[
			            \begin{pmatrix}
				            i\alpha v_1 + \alpha v_2 \\
				            \alpha v_1 + i\alpha v_2 \\
				            \alpha v_1 + 3\alpha v_3
			            \end{pmatrix}
			            =
			            \begin{pmatrix}
				            i\alpha v_1 + \alpha v_2  \\
				            \alpha v_1 + i \alpha v_2 \\
				            \alpha v_1 + 3 \alpha v_3
			            \end{pmatrix}
		            \]
		            \begin{center}
			            Quindi vale la proprietà
		            \end{center}
	      \end{itemize}
	      Entrambe le proprietà sono verificate, quindi \( f \) è lineare.


	\item[(b)] Determinare la matrice \( A \) associata ad \( f \) rispetto alla base canonica

	      \vspace{1em}
	      \[
		      A = \begin{pmatrix}
			      i & 1 & 0 \\
			      1 & i & 0 \\
			      1 & 0 & 3
		      \end{pmatrix}
	      \]

	      \[
		      f_A = A \begin{pmatrix}
			      x \\
			      y \\
			      z
		      \end{pmatrix}
		      =
		      \begin{pmatrix}
			      i & 1 & 0 \\
			      1 & i & 0 \\
			      1 & 0 & 3
		      \end{pmatrix}
		      \begin{pmatrix}
			      x \\
			      y \\
			      z
		      \end{pmatrix}
		      =
		      \begin{pmatrix}
			      ix + y \\
			      x + iy \\
			      x + 3z
		      \end{pmatrix}
	      \]

	\item[(c)] Stabilire se \( f \) è un isomorfismo

	      \vspace{1em}
	      \noindent \( f \) è un isomorfismo se e solo se \( A \) è invertibile. \( A \) è
	      invertibile se \( det(A) \neq 0 \).
	      \[
		      det(A) = det \begin{pmatrix}
			      i & 1 & 0 \\
			      1 & i & 0 \\
			      1 & 0 & 3
		      \end{pmatrix}
		      = 3 \cdot det \begin{pmatrix}
			      i & 1 \\
			      1 & i
		      \end{pmatrix}
		      =
	      \]
	      \[
		      = 3 \cdot (i \cdot i - (1 \cdot 1)) = 3 \cdot (-1 -1) = 3 \cdot (-2) = -6
	      \]
	      \( det(A) \neq 0 \) quindi \( f \) è un isomorfismo.
	\item[(d)] Verificare che \( \mathcal{B} = \left\{
	      b_1 = \begin{pmatrix}
		      1 \\
		      0 \\
		      i
	      \end{pmatrix} ,
	      b_2 = \begin{pmatrix}
		      0 \\
		      1 \\
		      0
	      \end{pmatrix} ,
	      b_3 = \begin{pmatrix}
		      5 \\
		      4 \\
		      3
	      \end{pmatrix}
	      \right\}  \)
	      è una base di \( \mathbb{C}^3 \)

	      \vspace{1em}
	      \[
		      \begin{pmatrix}
			      1 & 0 & 5 \\
			      0 & 1 & 4 \\
			      i & 0 & 3
		      \end{pmatrix}
		      \underset{E_{31}(-i)}{\stackrel{R_3 - iR_1}{\sim}}
		      \begin{pmatrix}
			      1 & 0 & 5      \\
			      0 & 1 & 4      \\
			      0 & 0 & 3 - i5
		      \end{pmatrix}
		      \underset{E_{3}(\frac{1}{3-i5})}{\stackrel{\frac{1}{3-i5}R_3}{\sim}}
		      \begin{pmatrix}
			      1 & 0 & 5 \\
			      0 & 1 & 4 \\
			      0 & 0 & 1
		      \end{pmatrix}
	      \]
	      La matrice ridotta ha tutte le colonne dominanti, cioè tutti gli elementi
	      di \( \mathcal{B} \) sono linearmente indipendenti, quindi \( \mathcal{B} \) è una base di \( \mathbb{C}^3 \).

	\item[(e)] Verificare che \( \mathcal{C} = \left\{
	      c_1 = \begin{pmatrix}
		      1 \\
		      1 \\
		      1
	      \end{pmatrix} ,
	      c_2 = \begin{pmatrix}
		      1 \\
		      1 \\
		      0
	      \end{pmatrix} ,
	      c_3 = \begin{pmatrix}
		      i \\
		      0 \\
		      0
	      \end{pmatrix}
	      \right\}  \)
	      è una base di \( \mathbb{C}^3 \)

	      \vspace{1em}
	      \[
		      \begin{pmatrix}
			      1 & 1 & i \\
			      1 & 1 & 0 \\
			      1 & 0 & 0
		      \end{pmatrix}
		      \underset{E_{21}(-1)}{\stackrel{R_2 - R_1}{\sim}}
		      \begin{pmatrix}
			      1 & 1 & i  \\
			      0 & 0 & -i \\
			      1 & 0 & 0
		      \end{pmatrix}
		      \underset{E_{31}(-1)}{\stackrel{R_3 - R_1}{\sim}}
		      \begin{pmatrix}
			      1 & 1  & i  \\
			      0 & 0  & -i \\
			      0 & -1 & -i
		      \end{pmatrix}
		      \underset{E_{32}}{\stackrel{R_3 \leftrightarrow R_2}{\sim}}
	      \]
	      \[
		      \begin{pmatrix}
			      1 & 1  & i  \\
			      0 & -1 & -i \\
			      0 & 0  & -i
		      \end{pmatrix}
		      \underset{E_{2}(-1)}{\stackrel{(-1)R_2}{\sim}}
		      \begin{pmatrix}
			      1 & 1 & i  \\
			      0 & 1 & i  \\
			      0 & 0 & -i
		      \end{pmatrix}
		      \underset{E_{3}(i)}{\stackrel{iR_3}{\sim}}
		      \begin{pmatrix}
			      1 & 1 & i \\
			      0 & 1 & i \\
			      0 & 0 & 1
		      \end{pmatrix}
	      \]
	      La matrice ridotta ha tutte le colonne dominanti, cioè tutti gli elementi
	      di \( \mathcal{C} \) sono linearmente indipendenti, quindi \( \mathcal{C} \) è una base di \( \mathbb{C}^3 \).


	\item[(f)] Determinare la matrice \( D \) associata ad \( f \) rispetto alla base
	      \( \mathcal{B} \) del dominio e alla base \( \mathcal{C} \) del codominio

	      \vspace{1em}
	      La matrice \( D \) avrà come colonne \( [f(b_i)]_{\mathcal{C}} \) con \( i = 1, 2, 3 \).
	      \[
		      f(b_1) = f \left( \begin{pmatrix} 1 \\ 0 \\ i \end{pmatrix} \right)
		      = \begin{pmatrix} i \\ 1 \\ 1+3i \end{pmatrix}
		      \Rightarrow
		      [f(b_1)]_{\mathcal{C}} = \alpha_1 c_1 + \alpha_2 c_2 + \alpha_3 c_3
	      \]
	      \[
		      = \alpha_1 \begin{pmatrix} 1 \\ 1 \\ 1 \end{pmatrix} + \alpha_2 \begin{pmatrix} 1 \\ 1 \\ 0 \end{pmatrix} + \alpha_3 \begin{pmatrix} i \\ 0 \\ 0 \end{pmatrix}
	      \]
	      \[
		      = \begin{pmatrix}
			      \alpha_1 + \alpha_2 + i\alpha_3 \\
			      \alpha_1 + \alpha_2             \\
			      \alpha_1
		      \end{pmatrix}
	      \]
	      \[
		      \begin{cases}
			      \alpha_1 + \alpha_2 + i\alpha_3 = i \\
			      \alpha_1 + \alpha_2 = 1             \\
			      \alpha_1 = 1+3i
		      \end{cases}
		      \leadsto
		      \begin{cases}
			      \alpha_1 = 1+3i \\
			      \alpha_2 = -3i  \\
			      \alpha_3 = \frac{i-1}{i}
		      \end{cases}
	      \]

	      \vspace{1em}
	      \[
		      f(b_2) = f \left( \begin{pmatrix} 0 \\ 1 \\ 0 \end{pmatrix} \right)
		      = \begin{pmatrix}
			      1 \\
			      i \\
			      0
		      \end{pmatrix}
		      \Rightarrow
		      [f(b_2)]_{\mathcal{C}} = \beta_1 c_1 + \beta_2 c_2 + \beta_3 c_3
	      \]
	      \[
		      = \beta_1 \begin{pmatrix} 1\\1\\1 \end{pmatrix}
		      + \beta_2 \begin{pmatrix} 1\\1\\0 \end{pmatrix}
		      + \beta_3 \begin{pmatrix} i\\0\\0 \end{pmatrix}
	      \]
	      \[
		      = \begin{pmatrix}
			      \beta_1 + \beta_2 + i\beta_3 \\
			      \beta_1 + \beta_2            \\
			      \beta_1
		      \end{pmatrix}
	      \]
	      \[
		      \begin{cases}
			      \beta_1 + \beta_2 + i\beta_3 = 1 \\
			      \beta_1 + \beta_2 = i            \\
			      \beta_1 = 0
		      \end{cases}
		      \leadsto
		      \begin{cases}
			      \beta_1 = 0 \\
			      \beta_2 = i \\
			      \beta_3 = \frac{1-i}{i}
		      \end{cases}
	      \]

	      \vspace{1em}
	      \[
		      f(b_3) = f \left( \begin{pmatrix} 5\\4\\3 \end{pmatrix}  \right) =
		      \begin{pmatrix}
			      4 + 5i \\
			      5 + 4i \\
			      14
		      \end{pmatrix}
		      \Rightarrow
		      [f(b_3)]_{\mathcal{C}} = \gamma_1 c_1 + \gamma_2 c_2 + \gamma_3 c_3
	      \]
	      \[
		      = \gamma_1 \begin{pmatrix} 1\\1\\1 \end{pmatrix}
		      + \gamma_2 \begin{pmatrix} 1\\1\\0 \end{pmatrix}
		      + \gamma_3 \begin{pmatrix} i\\0\\0 \end{pmatrix}
	      \]
	      \[
		      = \begin{pmatrix}
			      \gamma_1 + \gamma_2 + i\gamma_3 \\
			      \gamma_1 + \gamma_2             \\
			      \gamma_1
		      \end{pmatrix}
	      \]
	      \[
		      \begin{cases}
			      \gamma_1 + \gamma_2 + i\gamma_3 = 4+5i \\
			      \gamma_1 + \gamma_2 = 5+4i             \\
			      \gamma_1 = 14
		      \end{cases}
		      \leadsto
		      \begin{cases}
			      \gamma_1 = 14      \\
			      \gamma_2 = -9 + 4i \\
			      \gamma_3 = \frac{i-1}{i}
		      \end{cases}
	      \]

	      \vspace{1em}
	      \[
		      D = \begin{pmatrix}
			      \alpha_1 & \beta_1 & \gamma_1 \\
			      \alpha_2 & \beta_2 & \gamma_2 \\
			      \alpha_3 & \beta_3 & \gamma_3
		      \end{pmatrix}
		      =
		      \begin{pmatrix}
			      1+3i          & 0             & 14            \\
			      -3i           & i             & -9+4i         \\
			      \frac{i-1}{i} & \frac{1-i}{i} & \frac{i-1}{i}
		      \end{pmatrix}
	      \]

	\item[(g)] Calcolare la matrice \( A_{\mathcal{B} \to \mathcal{C}} \) del cambio di base
	      da \( \mathcal{B} \) a \( \mathcal{C} \)

	      \vspace{1em}
	      La matrice \( A_{\mathcal{B} \to \mathcal{C}} \) del cambio di base ha come
	      colonne \( [b_i]_{\mathcal{C}} \) con \( i = 1, 2, 3 \).
	      \[
		      b_1 = \begin{pmatrix} 1 \\ 0 \\ i \end{pmatrix}
		      \Rightarrow
		      [b_1]_{\mathcal{C}} = \alpha_1 c_1 + \alpha_2 c_2 + \alpha_3 c_3
	      \]
	      \[
		      = \alpha_1 \begin{pmatrix} 1 \\ 1 \\ 1 \end{pmatrix} + \alpha_2 \begin{pmatrix} 1 \\ 1 \\ 0 \end{pmatrix} + \alpha_3 \begin{pmatrix} i \\ 0 \\ 0 \end{pmatrix}
	      \]
	      \[
		      = \begin{pmatrix}
			      \alpha_1 + \alpha_2 + i\alpha_3 \\
			      \alpha_1 + \alpha_2             \\
			      \alpha_1
		      \end{pmatrix}
	      \]
	      \[
		      \begin{cases}
			      \alpha_1 + \alpha_2 + i\alpha_3 = 1 \\
			      \alpha_1 + \alpha_2 = 0             \\
			      \alpha_1 = i
		      \end{cases}
		      \leadsto
		      \begin{cases}
			      \alpha_1 = i  \\
			      \alpha_2 = -i \\
			      \alpha_3 = \frac{1}{i}
		      \end{cases}
	      \]

	      \vspace{1em}
	      \[
		      b_2 = \begin{pmatrix} 0 \\ 1 \\ 0 \end{pmatrix}
		      \Rightarrow
		      [b_2]_{\mathcal{C}} = \beta_1 c_1 + \beta_2 c_2 + \beta_3 c_3
	      \]
	      \[
		      = \beta_1 \begin{pmatrix} 1 \\ 1 \\ 1 \end{pmatrix}
		      + \beta_2 \begin{pmatrix} 1 \\ 1 \\ 0 \end{pmatrix}
		      + \beta_3 \begin{pmatrix} i \\ 0 \\ 0 \end{pmatrix}
	      \]
	      \[
		      = \begin{pmatrix}
			      \beta_1 + \beta_2 + i\beta_3 \\
			      \beta_1 + \beta_2            \\
			      \beta_1
		      \end{pmatrix}
	      \]
	      \[
		      \begin{cases}
			      \beta_1 + \beta_2 + i\beta_3 = 0 \\
			      \beta_1 + \beta_2 = 1            \\
			      \beta_1 = 0
		      \end{cases}
		      \leadsto
		      \begin{cases}
			      \beta_1 = 0 \\
			      \beta_2 = 1 \\
			      \beta_3 = -\frac{1}{i}
		      \end{cases}
	      \]

	      \vspace{1em}
	      \[
		      b_3 = \begin{pmatrix} 5 \\ 4 \\ 3 \end{pmatrix}
		      \Rightarrow
		      [b_3]_{\mathcal{C}} = \gamma_1 c_1 + \gamma_2 c_2 + \gamma_3 c_3
	      \]
	      \[
		      = \gamma_1 \begin{pmatrix} 1 \\ 1 \\ 1 \end{pmatrix}
		      + \gamma_2 \begin{pmatrix} 1 \\ 1 \\ 0 \end{pmatrix}
		      + \gamma_3 \begin{pmatrix} i \\ 0 \\ 0 \end{pmatrix}
	      \]
	      \[
		      = \begin{pmatrix}
			      \gamma_1 + \gamma_2 + i\gamma_3 \\
			      \gamma_1 + \gamma_2             \\
			      \gamma_1
		      \end{pmatrix}
	      \]
	      \[
		      \begin{cases}
			      \gamma_1 + \gamma_2 + i\gamma_3 = 5 \\
			      \gamma_1 + \gamma_2 = 4             \\
			      \gamma_1 = 3
		      \end{cases}
		      \leadsto
		      \begin{cases}
			      \gamma_1 = 3 \\
			      \gamma_2 = 1 \\
			      \gamma_3 = \frac{1}{i}
		      \end{cases}
	      \]

	      \vspace{1em}
	      \[
		      A_{\mathcal{B} \to \mathcal{C}} = \begin{pmatrix}
			      i           & 0            & 3           \\
			      -i          & 1            & 1           \\
			      \frac{1}{i} & -\frac{1}{i} & \frac{1}{i}
		      \end{pmatrix}
	      \]
\end{enumerate}

\subsection{Esercizio 2}
Si consideri
\[
	g: M_{2 \times 2}(\mathbb{R}) \to M_{2 \times 2}(\mathbb{R}), \quad g(A) = AB - BA
\]
dove \( B = \begin{pmatrix}
	-1 & 1 \\
	0  & 1
\end{pmatrix} \in M_{2 \times 2}(\mathbb{R}) \).

\begin{enumerate}
	\item[(a)] Verificare che
	      \[
		      \mathcal{D} = \left\{
		      D_1 = \begin{pmatrix}
			      1 & 0 \\
			      0 & 0
		      \end{pmatrix} ,
		      D_2 = \begin{pmatrix}
			      1 & 1 \\
			      0 & 0
		      \end{pmatrix} ,
		      D_3 = \begin{pmatrix}
			      1 & 1 \\
			      1 & 0
		      \end{pmatrix} ,
		      D_4 = \begin{pmatrix}
			      1 & 1 \\
			      1 & 1
		      \end{pmatrix}
		      \right\}
	      \]
	      è una base di \( M_{2 \times 2}(\mathbb{R}) \)

	\item[(b)] Verificare che
	      \[
		      \mathcal{E} = \left\{
		      E_1 = \begin{pmatrix}
			      1 & 0 \\
			      0 & 0
		      \end{pmatrix} ,
		      E_2 = \begin{pmatrix}
			      0 & 1 \\
			      0 & 0
		      \end{pmatrix} ,
		      E_3 = \begin{pmatrix}
			      0 & 0 \\
			      1 & 0
		      \end{pmatrix} ,
		      E_4 = \begin{pmatrix}
			      0 & 0 \\
			      0 & 1
		      \end{pmatrix}
		      \right\}
	      \]
	      è una base di \( M_{2 \times 2}(\mathbb{R}) \)

	\item[(c)] Calcolare la matrice \( M \) associata a \( g \) rispetto alla base \( \mathcal{D} \)
	      del dominio e rispetto alla base \( \mathcal{E} \) del codominio.

	\item[(d)] Calcolare il rango e la nullità di \( g \)

	\item[(e)] Calcolare una base di \( N(f_M) \) ed una base di \( Im(f_M) \)
\end{enumerate}

\subsection{Esercizio 3}
Data la matrice
\[
	N = \begin{pmatrix}
		5  & 1 & 0  \\
		-1 & 4 & -1 \\
		0  & 1 & 3
	\end{pmatrix}
	\in M_{3 \times 3}(\mathbb{C})
\]

\begin{enumerate}
	\item[(a)] Calcolare il polinomio caratteristico \( P_N \) di \( N \)

	      \vspace{1em}
	      \[
		      P_N = det(N - \lambda I_3) = det \begin{pmatrix}
			      5 - \lambda & 1           & 0           \\
			      -1          & 4 - \lambda & -1          \\
			      0           & 1           & 3 - \lambda
		      \end{pmatrix}
	      \]
	      \[
		      = - det \begin{pmatrix}
			      5 - \lambda & 0  \\
			      -1          & -1
		      \end{pmatrix}
		      + (3-\lambda) det \begin{pmatrix}
			      5 - \lambda & 1           \\
			      -1          & 4 - \lambda
		      \end{pmatrix}
	      \]
	      \[
		      = -(-5+\lambda) + (3-\lambda)((5-\lambda)(4-\lambda) + 1)
	      \]
	      \[
		      = 5 - \lambda + (3 - \lambda) (20 - 5 \lambda - 4 \lambda + \lambda^2 + 1)
	      \]
	      \[
		      = 5 - \lambda + (3 - \lambda) (21 - 9 \lambda + \lambda^2)
	      \]
	      \[
		      = 5 - \lambda + 63 - 27 \lambda + 3 \lambda^2 - 21 \lambda + 9 \lambda^2 - \lambda^3
	      \]
	      \[
		      = -\lambda^3 + 12 \lambda^2 - 49 \lambda + 68
	      \]

	\item[(b)] Calcolare tutti gli autovalori di \( N \) in \( \mathbb{C} \)

	      \vspace{1em}
	      Utilizzo Ruffini per trovare gli zeri del polinomio caratteristico:
	      \[
		      \begin{array}{c|ccc|c}
			        & -1 & 12 & -49 & 68  \\
			      \\
			      4 &    & -4 & 32  & -68 \\
			      \hline
			        & -1 & 8  & -17 & 0
		      \end{array}
	      \]
	      \[
		      (\lambda-4)(-\lambda^2 + 8\lambda - 17) = 0
	      \]
	      \[
		      \lambda_1 = 4
	      \]
	      \[
		      \lambda_{1,2} = \frac{-8 \pm \sqrt{64 - 68}}{-2} = \frac{-8 \pm \sqrt{-4}}{-2} = \frac{-8 \pm 2i}{-2} = 4 \pm i
	      \]
	      \[
		      \lambda_2 = 4 + i
	      \]
	      \[
		      \lambda_3 = 4 - i
	      \]

	\item[(c)] Stabilire se la matrice \( N \) è diagonalizzabile

	      \vspace{1em}
	      Siccome la matrice \( N \in M_{3 \times 3}(\mathbb{C}) \) ha 3 autovalori distinti, allora è diagonalizzabile.
\end{enumerate}

\end{document}
