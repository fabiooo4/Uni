\documentclass[a4paper]{article}

\usepackage[utf8]{inputenc}
\usepackage[T1]{fontenc}
\usepackage{textcomp}
\usepackage[italian]{babel}
\usepackage{amsmath, amssymb}
\usepackage[makeroom]{cancel}
\usepackage{amsfonts}
\usepackage{mdframed}
\usepackage{xcolor}
\usepackage{float}
\usepackage{tikz}
\usepackage{tikz-cd}
\usepackage{nicematrix}
\usepackage{pgfplots}
\usetikzlibrary{fit}
\usetikzlibrary{pgfplots.fillbetween}
\pgfplotsset{compat=newest, ticks=none}
\usepackage{graphicx}
\graphicspath{{./figures/}}

\pgfdeclarelayer{ft}
\pgfdeclarelayer{bg}
\pgfsetlayers{bg,main,ft}

\usepackage{import}
\usepackage{pdfpages}
\usepackage{transparent}
\usepackage{xcolor}

\usepackage{hyperref}
\hypersetup{
  colorlinks=false,
}

\usepackage{ntheorem}
\newtheorem{theorem}{Teorema}

% Useful definitions frame
\theoremstyle{break}
\theoremheaderfont{\bfseries}
\newmdtheoremenv[%
linecolor=gray,leftmargin=0,%
rightmargin=0,
innertopmargin=8pt,%
ntheorem]{define}{Definizioni utili}[section]

% Example frame
\theoremstyle{break}
\theoremheaderfont{\bfseries}
\newmdtheoremenv[%
linecolor=gray,leftmargin=0,%
rightmargin=0,
innertopmargin=8pt,%
ntheorem]{example}{Esempio}[section]

% Important definition frame
\theoremstyle{break}
\theoremheaderfont{\bfseries}
\newmdtheoremenv[%
linecolor=gray,leftmargin=0,%
rightmargin=0,
backgroundcolor=gray!40,%
innertopmargin=8pt,%
ntheorem]{definition}{Definizione}[section]

% Exercise frame
\theoremstyle{break}
\theoremheaderfont{\bfseries}
\newmdtheoremenv[%
linecolor=gray,leftmargin=0,%
rightmargin=0,
innertopmargin=8pt,%
ntheorem]{exercise}{Esercizio}[section]


% figure support
\usepackage{import}
\usepackage{xifthen}
\pdfminorversion=7
\usepackage{pdfpages}
\usepackage{transparent}
\newcommand{\incfig}[1]{%
  \def\svgwidth{\columnwidth}
  \import{./figures/}{#1.pdf_tex}
}

\pdfsuppresswarningpagegroup=1

% Matrices
\makeatletter
\renewcommand*\env@matrix[1][*\c@MaxMatrixCols c]{%
  \hskip -\arraycolsep
  \let\@ifnextchar\new@ifnextchar
\array{#1}}
\makeatother

\begin{document}
\begin{titlepage}
	\begin{center}
		\vspace*{1cm}

		\Huge
		\textbf{Probabilità e Statistica\\Esercizi}

		\vspace{0.5cm}
		\LARGE
		UniVR - Dipartimento di Informatica

		\vspace{1.5cm}

		\textbf{Fabio Irimie}

		\vfill


		\vspace{0.8cm}


		2° Semestre 2023/2024

	\end{center}
\end{titlepage}


\tableofcontents
\pagebreak

\section{Scheda 1}
\subsection{Esercizio 1}
Date le seguenti matrici a coefficienti complessi:
\[
	A = \begin{pmatrix}
		i  & 0   \\
		-1 & 1+i
	\end{pmatrix}
	\quad
	B = \begin{pmatrix}
		0 & -i \\
		i & 0
	\end{pmatrix}
	\quad
	C = \begin{pmatrix}
		1 & 1 \\
		1 & i
	\end{pmatrix}
	\quad
	D = \begin{pmatrix}
		2           & 0  \\
		\frac{1}{2} & -1
	\end{pmatrix}
\]
calcolare
\begin{enumerate}
	\item[(a)] \( (CD)A \)
	      \[
		      (CD)A =
		      \left(
		      \begin{pmatrix}
				      1 & 1 \\
				      1 & i
			      \end{pmatrix}
		      \begin{pmatrix}
				      2           & 0  \\
				      \frac{1}{2} & -1
			      \end{pmatrix}
		      \right)
		      \begin{pmatrix}
			      i  & 0   \\
			      -1 & 1+i
		      \end{pmatrix}
	      \]
	      \[
		      =
		      \begin{pmatrix}
			      \frac{5}{2}      & -1 \\
			      2 + \frac{1}{2}i & -i
		      \end{pmatrix}
		      \begin{pmatrix}
			      i  & 0   \\
			      -1 & 1+i
		      \end{pmatrix}
	      \]
	      \[
		      = \begin{pmatrix}
			      1 + \frac{5}{2}i  & -1 -i \\
			      -\frac{1}{2} + 3i & 1 - i
		      \end{pmatrix}
	      \]
	\item[(b)] \( B^TB \)
	      \[
		      B^T = \begin{pmatrix}
			      0  & i \\
			      -i & 0
		      \end{pmatrix}
	      \]
	      \[
		      B^TB = \begin{pmatrix}
			      0  & i \\
			      -i & 0
		      \end{pmatrix}
		      \begin{pmatrix}
			      0 & -i \\
			      i & 0
		      \end{pmatrix}
		      = \begin{pmatrix}
			      -1 & 0  \\
			      0  & -1
		      \end{pmatrix}
	      \]
	\item[(c)] \( 3A\left(B + 4D^T\right) \)
	      \[
		      D^T = \begin{pmatrix}
			      2 & \frac{1}{2} \\
			      0 & -1
		      \end{pmatrix}
	      \]
	      \[
		      3A\left(B + 4D^T\right) =
		      3\begin{pmatrix}
			      i  & 0   \\
			      -1 & 1+i
		      \end{pmatrix}
		      \left( \begin{pmatrix}
				      0 & -i \\
				      i & 0
			      \end{pmatrix}
		      + 4
		      \begin{pmatrix}
				      2 & \frac{1}{2} \\
				      0 & -1
			      \end{pmatrix} \right)
	      \]
	      \[
		      = \begin{pmatrix}
			      3i & 0      \\
			      -3 & 3 + 3i
		      \end{pmatrix}
		      \left( \begin{pmatrix}
				      0 & -i \\
				      i & 0
			      \end{pmatrix}
		      +
		      \begin{pmatrix}
				      8 & 2  \\
				      0 & -4
			      \end{pmatrix} \right)
	      \]
	      \[
		      = \begin{pmatrix}
			      3i & 0      \\
			      -3 & 3 + 3i
		      \end{pmatrix}
		      \begin{pmatrix}
			      8 & 2 - i \\
			      i & -4
		      \end{pmatrix}
	      \]
	      \[
		      = \begin{pmatrix}
			      24i      & 3 + 6i  \\
			      -27 + 3i & -18 -9i
		      \end{pmatrix}
	      \]
	\item[(d)] \( C^2A^T \)
	      \[
		      A^T = \begin{pmatrix}
			      i & -1  \\
			      0 & 1+i
		      \end{pmatrix}
	      \]
	      \[
		      C^2A^T =
		      \left( \begin{pmatrix}
				      1 & 1 \\
				      1 & i
			      \end{pmatrix}
		      \begin{pmatrix}
				      1 & 1 \\
				      1 & i
			      \end{pmatrix} \right)
		      \begin{pmatrix}
			      i & -1  \\
			      0 & 1+i
		      \end{pmatrix}
	      \]
	      \[
		      = \begin{pmatrix}
			      2     & 1+i \\
			      1 + i & 0
		      \end{pmatrix}
		      \begin{pmatrix}
			      i & -1    \\
			      0 & 1 + i
		      \end{pmatrix}
	      \]
	      \[
		      = \begin{pmatrix}
			      2i     & -2 + 2i \\
			      -1 + i & -1 -i
		      \end{pmatrix}
	      \]
	\item[(e)] \( \frac{1}{2} \left( B^2 - 3D^TC \right)  \)
	      \[
		      D^T = \begin{pmatrix}
			      2 & \frac{1}{2} \\
			      0 & -1
		      \end{pmatrix}
	      \]
	      \[
		      \frac{1}{2} \left( B^2 - 3D^TC \right) =
		      \frac{1}{2} \left( \begin{pmatrix}
				      0 & -i \\
				      i & 0
			      \end{pmatrix}
		      \begin{pmatrix}
				      0 & -i \\
				      i & 0
			      \end{pmatrix}
		      - 3
		      \begin{pmatrix}
				      2 & \frac{1}{2} \\
				      0 & -1
			      \end{pmatrix}
		      \begin{pmatrix}
				      1 & 1 \\
				      1 & i
			      \end{pmatrix} \right)
	      \]
	      \[
		      = \frac{1}{2} \left(
		      \begin{pmatrix}
				      1 & 0 \\
				      0 & 1
			      \end{pmatrix}
		      -
		      \begin{pmatrix}
				      6 & \frac{3}{2} \\
				      0 & -3
			      \end{pmatrix}
		      \begin{pmatrix}
				      1 & 1 \\
				      1 & i
			      \end{pmatrix} \right)
	      \]
	      \[
		      = \frac{1}{2} \left(
		      \begin{pmatrix}
				      1 & 0 \\
				      0 & 1
			      \end{pmatrix}
		      -
		      \begin{pmatrix}
				      \frac{15}{2} & 6 + \frac{3}{2}i \\
				      -3           & -3i
			      \end{pmatrix} \right)
	      \]
	      \[
		      = \frac{1}{2}
		      \begin{pmatrix}
			      -\frac{13}{2} & -6 - \frac{3}{2}i \\
			      3             & 1 + 3i
		      \end{pmatrix}
	      \]
	      \[
		      = \begin{pmatrix}
			      -\frac{13}{4} & -3 - \frac{3}{4}i          \\
			      \frac{3}{2}   & \frac{1}{2} + \frac{3}{2}i
		      \end{pmatrix}
	      \]
\end{enumerate}

\subsection{Esercizio 2}
Date le seguenti matrici a coefficienti complessi:
\[
	A = \begin{pmatrix}
		1     & -1    & -1   & 7       \\
		3 + i & -3 -i & -2-i & 11 + 7i \\
		3     & -3    & -2   & 11
	\end{pmatrix}
	\quad
	B = \begin{pmatrix}
		5 & -5   \\
		i & -1   \\
		5 & -5+i
	\end{pmatrix}
\]
\[
	C = \begin{pmatrix}
		i & 0 & -1      \\
		7 & i & 0       \\
		6 & 1 & -7 + 6i
	\end{pmatrix}
	\quad
	D = \begin{pmatrix}
		1 & -2 & \frac{1}{2} & 4      & 0 \\
		1 & -1 & \frac{1}{2} & 4+i    & 1 \\
		3 & -5 & \frac{3}{2} & 12 + i & 1
	\end{pmatrix}
\]

\begin{enumerate}
	\item[(a)] Usare l’algoritmo di Eliminazione di Gauss per determinare una forma ridotta
	      di ognuna delle matrici:
	      \begin{enumerate}
		      \item[(i)] \( A \)
		            \[
			            \begin{pmatrix}
				            1     & -1    & -1   & 7       \\
				            3 + i & -3 -i & -2-i & 11 + 7i \\
				            3     & -3    & -2   & 11
			            \end{pmatrix}
			            \stackrel{R_2 - (3+i)R_1}{\leadsto}
		            \]
		            \[
			            \begin{pmatrix}
				            1 & -1 & -1 & 7   \\
				            0 & 0  & 1  & -10 \\
				            3 & -3 & -2 & 11
			            \end{pmatrix}
			            \stackrel{R_3 - 3R_1}{\leadsto}
		            \]
		            \[
			            \begin{pmatrix}
				            1 & -1 & -1 & 7   \\
				            0 & 0  & 1  & -10 \\
				            0 & 0  & 1  & -10
			            \end{pmatrix}
			            \stackrel{R_3 - R_2}{\leadsto}
		            \]
		            \[
			            \begin{pmatrix}
				            1 & -1 & -1 & 7   \\
				            0 & 0  & 1  & -10 \\
				            0 & 0  & 0  & 0
			            \end{pmatrix}
		            \]
		      \item[(ii)] \( B \)
		            \[
			            \begin{pmatrix}
				            5 & -5   \\
				            i & -1   \\
				            5 & -5+i
			            \end{pmatrix}
			            \stackrel{\frac{1}{5}R_1}{\leadsto}
			            \begin{pmatrix}
				            1 & -1    \\
				            i & -1    \\
				            5 & -5 +i
			            \end{pmatrix}
			            \stackrel{R_2 - iR_1}{\leadsto}
		            \]
		            \[
			            \begin{pmatrix}
				            1 & -1     \\
				            0 & -1 + i \\
				            5 & -5 +i
			            \end{pmatrix}
			            \stackrel{R_3 - 5R_1}{\leadsto}
			            \begin{pmatrix}
				            1 & -1    \\
				            0 & -1 +i \\
				            0 & i
			            \end{pmatrix}
			            \stackrel{\frac{1}{-1+i}R_2}{\leadsto}
		            \]
		            \[
			            \begin{pmatrix}
				            1 & -1 \\
				            0 & 1  \\
				            0 & i
			            \end{pmatrix}
			            \stackrel{R_3 - iR_2}{\leadsto}
			            \begin{pmatrix}
				            1 & -1 \\
				            0 & 1  \\
				            0 & 0
			            \end{pmatrix}
		            \]
		      \item[(iii)] \( C \)
		            \[
			            \begin{pmatrix}
				            i & 0 & -1      \\
				            7 & i & 0       \\
				            6 & 1 & -7 + 6i
			            \end{pmatrix}
			            \stackrel{-iR_1}{\leadsto}
			            \begin{pmatrix}
				            1 & 0 & i       \\
				            7 & i & 0       \\
				            6 & 1 & -7 + 6i
			            \end{pmatrix}
			            \stackrel{R_2 -7R_1}{\leadsto}
		            \]
		            \[
			            \begin{pmatrix}
				            1 & 0 & i       \\
				            0 & i & -7i     \\
				            6 & 1 & -7 + 6i
			            \end{pmatrix}
			            \stackrel{R_3 - 6R_1}{\leadsto}
			            \begin{pmatrix}
				            1 & 0 & i   \\
				            0 & i & -7i \\
				            0 & 1 & -7
			            \end{pmatrix}
			            \stackrel{-iR_2}{\leadsto}
		            \]
		            \[
			            \begin{pmatrix}
				            1 & 0 & i  \\
				            0 & 1 & -7 \\
				            0 & 1 & -7
			            \end{pmatrix}
			            \stackrel{R_3 - R_2}{\leadsto}
			            \begin{pmatrix}
				            1 & 0 & i  \\
				            0 & 1 & -7 \\
				            0 & 0 & 0
			            \end{pmatrix}
		            \]
		      \item[(iv)] \( D \)
		            \[
			            \begin{pmatrix}
				            1 & -2 & \frac{1}{2} & 4      & 0 \\
				            1 & -1 & \frac{1}{2} & 4+i    & 1 \\
				            3 & -5 & \frac{3}{2} & 12 + i & 1
			            \end{pmatrix}
			            \stackrel{R_2 - R_1}{\leadsto}
			            \begin{pmatrix}
				            1 & -2 & \frac{1}{2} & 4      & 0 \\
				            0 & 1  & 0           & i      & 1 \\
				            3 & -5 & \frac{3}{2} & 12 + i & 1
			            \end{pmatrix}
		            \]
		            \[
			            \stackrel{R_3 - 3R_1}{\leadsto}
			            \begin{pmatrix}
				            1 & -2 & \frac{1}{2} & 4 & 0 \\
				            0 & 1  & 0           & i & 1 \\
				            0 & 1  & 0           & i & 1
			            \end{pmatrix}
			            \stackrel{R_3 - R_2}{\leadsto}
			            \begin{pmatrix}
				            1 & -2 & \frac{1}{2} & 4 & 0 \\
				            0 & 1  & 0           & i & 1 \\
				            0 & 0  & 0           & 0 & 0
			            \end{pmatrix}
		            \]
	      \end{enumerate}
	\item[(b)] Calcolare il rango di ognuna delle matrici:
	      \begin{enumerate}
		      \item[(i)] \( rk(A) = 2 \)
		      \item[(ii)] \( rk(B) = 2 \)
		      \item[(iii)] \( rk(C) = 2 \)
		      \item[(iv)] \( rk(D) = 2 \)
	      \end{enumerate}
	\item[(c)] Scrivere i sistemi lineari per cui le matrici \( A,B,C,D \) sono le
	      corrispondenti matrici aumentate ed usare il Teorema di Rouchè-Capelli per
	      stabilire se tali sistemi lineari ammettono o non ammettono soluzioni
	      \begin{enumerate}
		      \item[(i)]
		            \[
			            A = \begin{cases}
				            x - y - z = 7 \\
				            z = -10       \\
			            \end{cases}
			            \quad
			            \text{Ha infinite soluzioni
			            }
		            \]
		      \item[(ii)]
		            \[
			            B = \begin{cases}
				            x = -1 \\
				            0 = 1  \\
				            0 = 0
			            \end{cases}
			            \quad
			            \text{Non ha soluzioni}
		            \]
		      \item[(iii)]
		            \[
			            C = \begin{cases}
				            x = i  \\
				            y = -7 \\
				            0 = 0
			            \end{cases}
			            \quad
			            \text{Ha una sola soluzione}
		            \]


		      \item[(iv)]
		            \[
			            D = \begin{cases}
				            x -2y + \frac{1}{2}z + 4w = 0 \\
				            y + iw = 1                    \\
				            0 = 0
			            \end{cases}
			            \quad
			            \text{Ha infinite soluzioni}
		            \]
	      \end{enumerate}
	\item[(d)] Trovare tutte le soluzioni del sistema lineare \( A \begin{pmatrix}
		      x_1 \\
		      x_2 \\
		      x_3 \\
		      x_4
	      \end{pmatrix} = \begin{pmatrix}
		      -1    \\
		      -1 -i \\
		      -1
	      \end{pmatrix}   \)
	      \[
		      \begin{cases}
			      x_1 - x_2 - x_3 + 7x_4 = -1 \\
			      x_3 -10x_4 = -1 -i          \\
			      0 = -1
		      \end{cases}
	      \]
	      Il sistema lineare non ha soluzioni
	\item[(e)] Trovare tutte le soluzioni del sistema lineare omogeneo associato alla
	      matrice \( B \)
	      \[
		      \begin{cases}
			      x_1 - x_2 = 0 \\
			      x_2 = 0       \\
		      \end{cases}
	      \]
	      \[
		      \begin{cases}
			      x_1 = 0 \\
			      x_2 = 0
		      \end{cases}
	      \]
\end{enumerate}

\subsection{Esercizio 3}
Si consideri la matrice
\[
	A_t = \begin{pmatrix}
		t-1 & 2t-2 & t-1 \\
		t+1 & 2t+2 & t+1 \\
		1   & 2    & t+1
	\end{pmatrix}
	\quad
	\text{con} \; t \in \mathbb{R}
\]
\begin{enumerate}
	\item[(a)] Calcolare il rango \( rk(A_t) \) per ogni valore di \( t \in \mathbb{R} \)
	      \[
		      \begin{pmatrix}
			      t-1 & 2t-2 & t-1 \\
			      t+1 & 2t+2 & t+1 \\
			      1   & 2    & t+1
		      \end{pmatrix}
		      \stackrel{\frac{1}{t-1}R_1}{\leadsto}
		      \begin{pmatrix}
			      1   & 2    & 1   \\
			      t+1 & 2t+2 & t+1 \\
			      1   & 2    & t+1
		      \end{pmatrix}
	      \]
	      \[
		      \stackrel{R_2 - (t+1)R_1}{\leadsto}
		      \begin{pmatrix}
			      1 & 2 & 1   \\
			      0 & 0 & 0   \\
			      1 & 2 & t+1
		      \end{pmatrix}
		      \stackrel{R_3 - R_1}{\leadsto}
		      \begin{pmatrix}
			      1 & 2 & 1 \\
			      0 & 0 & 0 \\
			      0 & 0 & t
		      \end{pmatrix}
	      \]
	      \[
		      \stackrel{R_3 \leftrightarrow R_2}{\leadsto}
		      \begin{pmatrix}
			      1 & 2 & 1 \\
			      0 & 0 & t \\
			      0 & 0 & 0
		      \end{pmatrix}
		      \stackrel{\frac{1}{t}R_2}{\leadsto}
		      \begin{pmatrix}
			      1 & 2 & 1 \\
			      0 & 0 & 1 \\
			      0 & 0 & 0
		      \end{pmatrix}
	      \]
	      \[
		      rk(U) = rk(A_t) = 2 \quad \forall t \in \mathbb{R}
	      \]
	\item[(b)] Se \( A_t \) è la matrice aumentata di un sistema lineare, per quali valori
	      di \( t \in \mathbb{R} \)  tale sistema ammette soluzioni?
        \[
        A_t = \begin{cases}
          x_1 + 2x_2 = 1 \\
          0 = 1
        \end{cases}
        \] 
        Il sistema lineare non ammette soluzioni per nessun valore di \( t \in \mathbb{R} \)
\end{enumerate}

\subsection{Esercizio 4}
Trovare tutte le soluzioni complesse di \( x^3 - 1 = 0 \) 

\[
  x^3 - 1 = 0
\]
\[
  x^3 = 1
\] 
\[
  x = \sqrt[3]{1}  
\] 

\[
  z = a + bi \quad a = 1,\; b = 0
\] 
\[
  r = \sqrt{a^2 + b^2} = 1
\] 
\[
  \alpha = \arctan{\frac{b}{a}} = \arctan{\frac{0}{1}} = 0
\] 

\vspace{1em}
\[
  \sqrt[n]{z} = \sqrt[n]{r} \left( \cos{\frac{\alpha + 2k\pi}{n}} + i \sin{\frac{\alpha + 2k\pi}{n}} \right)
\] 
\[
  x = \sqrt[3]{1} = \cos\left(\frac{0 + 2k\pi}{3}\right) + i \sin\left(\frac{0 + 2k\pi}{3}\right) \quad k = 0,1,2
\] 

\vspace{1em}
\[
  x_0 = \cos\left(\frac{0}{3}\right) + i \sin\left(\frac{0}{3}\right) = 1
\] 
\[
  x_1 = \cos\left(\frac{2\pi}{3}\right) + i \sin\left(\frac{2\pi}{3}\right) = -\frac{1}{2} + i \frac{\sqrt{3}}{2}
\] 
\[
  x_2 = \cos\left(\frac{4\pi}{3}\right) + i \sin\left(\frac{4\pi}{3}\right) = -\frac{1}{2} - i \frac{\sqrt{3}}{2}
\] 

\end{document}
