\documentclass[a4paper]{article}

\usepackage[utf8]{inputenc}
\usepackage[T1]{fontenc}
\usepackage{textcomp}
\usepackage[italian]{babel}
\usepackage{amsmath, amssymb}
\usepackage[makeroom]{cancel}
\usepackage{amsfonts}
\usepackage{mdframed}
\usepackage{xcolor}
\usepackage{float}
\usepackage{tikz}
\usepackage{pgfplots}
\usetikzlibrary{pgfplots.fillbetween}
\pgfplotsset{compat=newest, ticks=none}
\usepackage{graphicx}
\graphicspath{{./figures/}}

\pgfdeclarelayer{ft}
\pgfdeclarelayer{bg}
\pgfsetlayers{bg,main,ft}

\usepackage{import}
\usepackage{pdfpages}
\usepackage{transparent}
\usepackage{xcolor}

\usepackage{hyperref}
\hypersetup{
    colorlinks=false,
}

\usepackage{ntheorem}
\newtheorem{theorem}{Teorema}

% Useful definitions frame
\theoremstyle{break}
\theoremheaderfont{\bfseries}
\newmdtheoremenv[%
	linecolor=gray,leftmargin=0,%
	rightmargin=0,
	innertopmargin=8pt,%
	ntheorem]{define}{Definizioni utili}[section]

% Example frame
\theoremstyle{break}
\theoremheaderfont{\bfseries}
\newmdtheoremenv[%
	linecolor=gray,leftmargin=0,%
	rightmargin=0,
	innertopmargin=8pt,%
	ntheorem]{example}{Esempio}[section]

% Important definition frame
\theoremstyle{break}
\theoremheaderfont{\bfseries}
\newmdtheoremenv[%
	linecolor=gray,leftmargin=0,%
	rightmargin=0,
	backgroundcolor=gray!40,%
	innertopmargin=8pt,%
	ntheorem]{definition}{Definizione}[section]

% Exercise frame
\theoremstyle{break}
\theoremheaderfont{\bfseries}
\newmdtheoremenv[%
	linecolor=gray,leftmargin=0,%
	rightmargin=0,
	innertopmargin=8pt,%
	ntheorem]{exercise}{Esercizio}[section]


% figure support
\usepackage{import}
\usepackage{xifthen}
\pdfminorversion=7
\usepackage{pdfpages}
\usepackage{transparent}
\newcommand{\incfig}[1]{%
	\def\svgwidth{\columnwidth}
	\import{./figures/}{#1.pdf_tex}
}

\pdfsuppresswarningpagegroup=1

\begin{document}
\begin{titlepage}
	\begin{center}
		\vspace*{1cm}

		\Huge
		\textbf{Probabilità e Statistica\\Esercizi}

		\vspace{0.5cm}
		\LARGE
		UniVR - Dipartimento di Informatica

		\vspace{1.5cm}

		\textbf{Fabio Irimie}

		\vfill


		\vspace{0.8cm}


		2° Semestre 2023/2024

	\end{center}
\end{titlepage}


\tableofcontents
\pagebreak

% Parte fatta con il prof di esercitazione
\section{Sistema di riferimento}
È un \textbf{sistema di coordinate} rispetto al quale vengono misurate le grandezze coinvolte in un problema.
Per fissare un sistema di riferimento si devono fissare:
\begin{itemize}
	\item Un punto di origine $O$
	\item Un insieme di assi lungo determinate direzioni
\end{itemize}

\subsection{Spazio cartesiano}
È il sistema di riferimento più comune, individuato da 2 o 3 rette mutuamente perpendicolari,
dette \textbf{assi cartesiani}, avendo in comune un unico punto chiamato \textbf{origine}.
\label{D1}
\begin{definition}[Coordinate cartesiane]
	Le coordinate cartesiane di un punto $P$ nello spazio vengono determinate tracciando il segmento
	di perpendicolare da \( P \) ad ognuno degli assi. La lunghezza di ciascun segmento da \( O \) fino al
	piede della perpendicolare determina il valore della coordinata cartesiana.
	\label{D2}
\end{definition}

\section{Grandezze}
\subsection{Grandezze scalari}
Sono grandezze che si possono rappresentare con un numero reale, ad esempio la massa, la temperatura, ecc.
Per definire una grandezza scalare è necessario specificare:
\begin{itemize}
	\item Il valore numerico
	\item L'unità di misura
\end{itemize}

\subsection{Grandezze vettoriali}
Sono grandezze che si possono rappresentare con un \textbf{vettore}, ad esempio la forza, la velocità, ecc.
Per definire una grandezza vettoriale è necessario utilizzare un vettore, cioè un segmento orientato
definito da:
\begin{itemize}
	\item Intensità
	\item Direzione
	\item Verso
\end{itemize}
\label{D3}

\noindent Si può moltiplicare un vettore per uno scalare, ottenendo un vettore con direzione del primo vettore
e intensità uguale al prodotto del modulo del primo vettore per lo scalare. Il verso resterà lo stesso
del primo vettore in caso di scalare positivo, sarà opposto in caso di scalare negativo.

\subsubsection{Scomposizione di un vettore}
Un vettore può essere scomposto in due vettori, detti \textbf{componenti}, lungo due direzioni ortogonali.
\[
	\vec{v} = \vec{v_x} + \vec{v_y} \quad \text{\textbf{somma vettoriale}}
\]
\label{D4}
Il modulo del vettore \( \vec{v} \) si trova applicando il teorema di Pitagora al modulo delle componenti:
\[
	|\vec{v}| = \sqrt{|\vec{v_x}|^2 + |\vec{v_y}|^2}
\]
\subsubsection{Versori}
Sono \textbf{vettori unitari} (con \( modulo=1 \)) diretti come gli assi, in genere indicati come
\( \vec{i},\vec{j},\vec{k} \).

\noindent Un vettore può essere indicato come somma dei versori, ciascuno moltiplicato per il modulo
della rispettiva componente del vettore:
\[
	\vec{v} = v_x\hat{i} + v_y\hat{j}
\]
Ad esempio:
\[
	\vec{a} = 2\hat{i} + 3\hat{j} \quad \text{o} \;\; \vec{a}(2,3)
\]

\subsubsection{Somma di vettori}
La somma di due vettori si ottiene sommando le rispettive componenti:
\label{D5}
\[
	\vec{C} = \vec{A} + \vec{B}
\]
\[
	A_x+B_x = C_x \quad A_y+B_y = C_y
\]
\[
	\vec{C} = C_x\hat{i} + C_y\hat{j}
\]

\subsubsection{Differenza di vettori}
La differenza di due vettori si ottiene sottraendo le rispettive componenti:
\label{D6}
\[
	\vec{C} = \vec{A} - \vec{B} = \vec{A} + (-\vec{B})
\]
\[
	A_x-B_x = C_x \quad A_y-B_y = C_y
\]
\[
	\vec{C} = C_x\hat{i} + C_y\hat{j}
\]

\subsection{Rapporti trigonometrici}
\[
	\sin(\alpha) = \frac{opposto}{ipotenusa}
\]
\[
	\cos(\alpha) = \frac{adiacente}{ipotenusa}
\]
\[
	\tan(\alpha) = \frac{opposto}{adiacente}
\]

\subsection{Prodotto scalare}
Il prodotto scalare tra due vettori \( \vec{A} \) e \( \vec{B} \) è definito come:
\[
	\vec{A} \cdot \vec{B} = |\vec{A}| |\vec{B}|\cos(\theta)
\]
Dove \( \theta \) è l'angolo tra i due vettori.
\[
	\theta = 0 \quad \vec{A} \cdot \vec{B} = AB
\]
\[
	\theta = 90\quad \vec{A} \cdot \vec{B} = 0
\]
\[
	\theta = 180\quad \vec{A} \cdot \vec{B} = -AB
\]
Il prodotto scalare è quindi il numero che si ottiene moltiplicando il modulo di un vettore per
l'intensità del vettore componente del secondo lungo il primo. Un altro modo per scriverlo è:
\[
	\vec{a} \cdot \vec{b} = ab\cos(\theta) = ab_a
\]
Dove \( b_a = b\cos(\theta) \)

\subsubsection{Prodotto scalare tramite componenti}
Presi 2 vettori:
\[
	\vec{A} = a_x\hat{i} + a_y\hat{j} + a_z\hat{k} \quad \vec{A}(a_x,a_y,a_z)
\]
\[
	\vec{B} = b_x\hat{i} + b_y\hat{j} + b_z\hat{k} \quad \vec{B}(b_x,b_y,b_z)
\]
Il prodotto scalare si può calcolare come:
\[
	C = \vec{A} \cdot \vec{B} = a_xb_x + a_yb_y + a_zb_z
\]
Perchè:
\[
	\hat{i} \cdot \hat{i} = \hat{j} \cdot \hat{j} = \hat{k} \cdot \hat{k} = 1
\]
\[
	\hat{i} \cdot \hat{j} = \hat{i} \cdot \hat{k} = \hat{j} \cdot \hat{k} = 0
\]

\begin{figure}[H]
	\begin{example}
		\[
			\vec{A} = 3\hat{i} + 2\hat{j} \quad \vec{B} = \hat{i} + 2\hat{j}
		\]
		\[
			\vec{A} \cdot \vec{B} = 2\cdot3 + 3\cdot4 = 6 + 12 = 18
		\]
		\[
			C = \vec{A} \cdot \vec{B} = (3\hat{i} + 2\hat{j}) \cdot (\hat{i} + 2\hat{j}) = 3\cdot1 + 2\cdot2 = 3 + 4 = 7
		\]
	\end{example}
\end{figure}

\subsection{Prodotto vettoriale}
Il prodotto vettoriale tra 2 vettori, è un vettore avente modulo uguale al prodotto dei loro
moduli per il seno dell'angolo compreso tra essi:
\[
	|\vec{a} \times \vec{b}| = ab \sin(\theta)
\]
La direzione si individuano con la regola della mano destra.
\label{D7}
Il modulo del vettore risultante è uguale all'area del parallelogramma generato dai vettori \( \vec{a} \) e \( \vec{b} \).
\end{document}
