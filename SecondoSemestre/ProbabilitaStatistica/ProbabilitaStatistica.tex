\documentclass[a4paper]{article}

\usepackage[utf8]{inputenc}
\usepackage[T1]{fontenc}
\usepackage{textcomp}
\usepackage[italian]{babel}
\usepackage{amsmath, amssymb}
\usepackage[makeroom]{cancel}
\usepackage{amsfonts}
\usepackage{mdframed}
\usepackage{xcolor}
\usepackage{float}
\usepackage{tikz}
\usepackage{pgfplots}
\usetikzlibrary{pgfplots.fillbetween}
\pgfplotsset{compat=newest, ticks=none}
\usepackage{graphicx}
\graphicspath{{./figures/}}

\pgfdeclarelayer{ft}
\pgfdeclarelayer{bg}
\pgfsetlayers{bg,main,ft}

\usepackage{import}
\usepackage{pdfpages}
\usepackage{transparent}
\usepackage{xcolor}

\usepackage{hyperref}
\hypersetup{
    colorlinks=false,
}

\usepackage{ntheorem}
\newtheorem{theorem}{Teorema}

% Useful definitions frame
\theoremstyle{break}
\theoremheaderfont{\bfseries}
\newmdtheoremenv[%
	linecolor=gray,leftmargin=0,%
	rightmargin=0,
	innertopmargin=8pt,%
	ntheorem]{define}{Definizioni utili}[section]

% Example frame
\theoremstyle{break}
\theoremheaderfont{\bfseries}
\newmdtheoremenv[%
	linecolor=gray,leftmargin=0,%
	rightmargin=0,
	innertopmargin=8pt,%
	ntheorem]{example}{Esempio}[section]

% Important definition frame
\theoremstyle{break}
\theoremheaderfont{\bfseries}
\newmdtheoremenv[%
	linecolor=gray,leftmargin=0,%
	rightmargin=0,
	backgroundcolor=gray!40,%
	innertopmargin=8pt,%
	ntheorem]{definition}{Definizione}[section]

% Exercise frame
\theoremstyle{break}
\theoremheaderfont{\bfseries}
\newmdtheoremenv[%
	linecolor=gray,leftmargin=0,%
	rightmargin=0,
	innertopmargin=8pt,%
	ntheorem]{exercise}{Esercizio}[section]


% figure support
\usepackage{import}
\usepackage{xifthen}
\pdfminorversion=7
\usepackage{pdfpages}
\usepackage{transparent}
\newcommand{\incfig}[1]{%
	\def\svgwidth{\columnwidth}
	\import{./figures/}{#1.pdf_tex}
}

\pdfsuppresswarningpagegroup=1

\begin{document}
\begin{titlepage}
	\begin{center}
		\vspace*{1cm}

		\Huge
		\textbf{Analisi 1}

		\vspace{0.5cm}
		\LARGE
		UniVR - Dipartimento di Informatica

		\vspace{1.5cm}

		\textbf{Fabio Irimie}

		\vfill


		\vspace{0.8cm}

    Corso di Giacomo Canevari

		1° Semestre 2023/2024

	\end{center}
\end{titlepage}


\tableofcontents
\pagebreak

% Libro: S.M ROSS, Probabilità e Statistica per l'ingegneria e le scienze Apogeo, 2015 (ed. 3)

\section{Cos'è la probabilità e la statistica?}
La statistica è una scienza che si occupa di raccogliere, organizzare, analizzare
e interpretare i dati. Nella statistica si cerca di estrapolare informazioni da
esperimenti \textbf{aleatori} (esperimenti che non si possono ripetere esattamente allo stesso
modo) e di prendere decisioni basate su queste informazioni. Ogni esperimento aleatorio
ha bisogno di un \textbf{modello probabilistico} che ne descriva le caratteristiche principali.

\subsection{Popolazione, variabili e campione}
\begin{itemize}
	\item \textbf{Popolazione}: tutti i possibili oggetti di un'indagine statistica
	\item \textbf{Individuo}: un singolo oggetto della popolazione
	\item \textbf{Variabile}: una qualsiasi caratteristica di un individuo della
	      popolazione soggetta a possibili variazioni (es. altezza, peso, sesso, ecc.)
	\item \textbf{Range della variabile}: \( R_x \) è l'insieme di tutti i possibili
	      valori che la variabile \( x \) può assumere
	\item \textbf{Campione}: un sottoinsieme rappresentativo della popolazione
	      composto dalle variabili relative ad un sottoinsieme di indibidui
	\item \textbf{TODO}
\end{itemize}

\subsection{Parametro e Stima}
\begin{itemize}
	\item \textbf{Parametro}:una misura che descrive una proprietà dell'intera popolazione
	\item \textbf{Stima}: una misura che descrive una proprietà del campione che
	      fornisce informazioni sul parametro
\end{itemize}

\section{Variabili}
Le variabili possono essere di diverso tipo:
\begin{itemize}
	\item \textbf{Variabili qualitative nominali}:
	      \begin{itemize}
		      \item \textbf{Ordinali}: possono essere ordinate
		      \item \textbf{Non ordinali}: non possono essere ordinate
	      \end{itemize}
	\item \textbf{Variabili quantitative}:
	      \begin{itemize}
		      \item \textbf{Aleatorie continue}: derivano da processi di misura e assumono
		            i loro range. Sono sottoinsiemi reali
		      \item \textbf{Aleatorie discrete}: assumono valori interi
	      \end{itemize}
\end{itemize}

\section{Statistica descrittiva}
Consiste nella raccolta, organizzazione, rappresentazione e analisi dei \textbf{dati}.

\subsection{Strumenti di sintesi}
\subsubsection{Tabelle di frequenza}
Sono tabelle di individui con una certa caratteristica o aventi una caratteristica
appartenenta ad un certo intervallo.
\begin{itemize}
	\item \textbf{Frequenza assoluta}: conteggio del numero di individui
	\item \textbf{Frequenza relativa}: percentuale del numero di individui
	\item \textbf{Frequenza cumulativa}: conteggio o percentuale del numero di individui
	      fino ad un certo punto
\end{itemize}

\subsubsection{Distribuzioni}
Sono rappresentazioni grafiche delle frequenze di una variabile. Possono essere:
\begin{itemize}
	\item \textbf{Caso discreto}: valore variabile \( \to  \) frequenza relativa
	\item \textbf{Caso continuo o numerabile}: intervallo di valori variabile \( \to  \) frequenza relativa
\end{itemize}

\subsubsection{Distribuzioni cumulative}
Sono distribuzioni che rappresentano la frequenza cumulativa di una variabile. Possono essere:
\begin{itemize}
	\item \textbf{Caso discreto}: valore variabile \( \to  \) frequenza cumulativa relativa
	\item \textbf{Caso continuo o numerabile}: intervallo \( \to  \) frequenza cumulativa relativa
\end{itemize}

\subsubsection{Grafici}
Sono rappresentazioni grafiche delle distribuzioni. Possono essere:
\begin{itemize}
	\item \textbf{Istogrammi}: è costituito da rettangoli, insistenti sulle classi
	      della partizione, attigui le cui aree sono confrontabili con le probabilità.
	\item \textbf{Poligoni di frequenza}: rappresentano le frequenze di una variabile
	\item \textbf{Diagrammi a torta}: rappresentano le frequenze relative di una variabile
	\item \textbf{Boxplot}: rappresentano le frequenze di una variabile
\end{itemize}

\subsubsection{Indici statistici-stime}
Sono misure quantitative che fornicono informazioni sulla distribuzione di una certa
caratteristica.

\section{Frequenze campionarie}
Siano \( \underline{x} = (\tilde{x}_1, \ldots, \tilde{x}_n) \) i valori assunti da una variabile

\section{Indici statistici}
\subsection{Indici di posizione}
Forniscono informazioni del valore attorno al quale si posizionano i dati. Sono:
\begin{itemize}
	\item \textbf{Media campionaria}: valore medio dei dati
	\item \textbf{Moda campionaria}: valore che si ripete più frequentemente. Ci possono
	      essere più valori modali. Sia \( \underline{y} = (y_1, \ldots, y_n) \) il campione
	      ordinato (\( y_i \in \{\tilde{x_1}, \ldots, \tilde{x_n}\}  \) e \( y_i \le y_{i+1} \) )
	\item \textbf{Mediana campionaria}: è il valore centrale del campione, una volta ordinato.
	      \[
		      M = \begin{cases}
			      y_{\frac{n+1}{2}}                             & \text{se } n \text{ è dispari} \\
			      \frac{y_{\frac{n}{2}} + y_{\frac{n}{2}+1}}{2} & \text{se } n \text{ è pari}
		      \end{cases}
	      \]
\end{itemize}

\subsection{Indici di dispersione}
Forniscono informazioni su quanto i dati si disperdono attorno ad un valore centrale. Sono:
\begin{itemize}
	\item \textbf{Scarto Quadratico Medio}: misura la dispersione dei dati attorno alla media
	      \[
		      s'^2 = \frac{1}{n} \sum_{i=1}^{n} (\tilde{x_i} - \bar{x})^2
	      \]
	\item \textbf{Varianza campionaria}: misura la dispersione dei dati attorno alla media
	      \[
		      s^2 = \frac{1}{n-1} \sum_{i=1}^{n} (\tilde{x_i} - \bar{x})^2
	      \]
	\item \textbf{Deviazione standard campionaria}: misura la distanza dei dati attorno alla media
	      \[
		      s = \sqrt{s^2} = \sqrt{\frac{1}{n-1} \sum_{i=1}^{n} (\tilde{x_i} - \bar{x})^2}
	      \]
	\item \textbf{Range}: Sia \( \underline{x} = (\tilde{x_1}, \ldots, \tilde{x_n}) \) un campione
	      di dimensione \( n \). Il range è definito come:
	      \[
		      R = \max(\underline{x}) - \min(\underline{x})
	      \]
\end{itemize}

\subsection{Valori usuali}
Si possono definire \textbf{valori usuali} di una variabile i valori del campione compresi
tra:
\begin{itemize}
	\item \textbf{Minimo valore "usuale"}: media campionaria - 2 deviazioni standard
	\item \textbf{Massimo valore "usuale"}: media campionaria + 2 deviazioni standard
\end{itemize}


\end{document}
