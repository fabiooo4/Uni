\documentclass[a4paper]{article}

\usepackage[utf8]{inputenc}
\usepackage[T1]{fontenc}
\usepackage{textcomp}
\usepackage[italian]{babel}
\usepackage{amsmath, amssymb}
\usepackage{amsfonts}
\usepackage{mdframed}
\usepackage{ntheorem}
\usepackage{xcolor}
\usepackage{listings}
\usepackage{graphicx}
\graphicspath{{./figures/}}

\usepackage{import}
\usepackage{pdfpages}
\usepackage{transparent}
\usepackage{xcolor}

% Useful definitions frame
\theoremstyle{break}
\theoremheaderfont{\bfseries}
\newmdtheoremenv[%
linecolor=gray,leftmargin=0,%
rightmargin=0,
innertopmargin=8pt,%
ntheorem]{define}{Definizioni utili}[section]

% Example frame
\theoremstyle{break}
\theoremheaderfont{\bfseries}
\newmdtheoremenv[%
linecolor=gray,leftmargin=0,%
rightmargin=0,
innertopmargin=8pt,%
ntheorem]{example}{Esempio}[section]

% Important definition frame
\theoremstyle{break}
\theoremheaderfont{\bfseries}
\newmdtheoremenv[%
linecolor=gray,leftmargin=0,%
rightmargin=0,
backgroundcolor=gray!40,%
innertopmargin=8pt,%
ntheorem]{definition}{Definizione}[section]

% Exercise frame
\theoremstyle{break}
\theoremheaderfont{\bfseries}
\newmdtheoremenv[%
linecolor=gray,leftmargin=0,%
rightmargin=0,
innertopmargin=8pt,%
ntheorem]{exercise}{Domanda}[section]


% figure support
\usepackage{import}
\usepackage{xifthen}
\pdfminorversion=7
\usepackage{pdfpages}
\usepackage{transparent}
\newcommand{\incfig}[1]{%
  \def\svgwidth{\columnwidth}
  \import{./figures/}{#1.pdf_tex}
}

\pdfsuppresswarningpagegroup=1

% Code blocks
\definecolor{codegreen}{rgb}{0,0.6,0}
\definecolor{codegray}{rgb}{0.5,0.5,0.5}
\definecolor{codepurple}{rgb}{0.58,0,0.82}
\definecolor{backcolour}{rgb}{0.95,0.95,0.95}

\lstdefinestyle{mystyle}{
	backgroundcolor=\color{backcolour},
	commentstyle=\color{codegreen},
	keywordstyle=\color{magenta},
	numberstyle=\tiny\color{codegray},
	stringstyle=\color{codepurple},
	basicstyle=\ttfamily\footnotesize,
	breakatwhitespace=false,
	breaklines=true,
	captionpos=b,
	keepspaces=true,
	numbers=left,
	numbersep=5pt,
	showspaces=false,
	showstringspaces=false,
	showtabs=false,
	tabsize=2
}

\lstset{style=mystyle}

\usepackage{color}

\definecolor{dkgreen}{rgb}{0,0.6,0}
\definecolor{gray}{rgb}{0.5,0.5,0.5}
\definecolor{mauve}{rgb}{0.58,0,0.82}

\lstset{frame=tb,
	aboveskip=3mm,
	belowskip=3mm,
	showstringspaces=false,
	columns=flexible,
	basicstyle={\small\ttfamily},
	numbers=none,
	numberstyle=\tiny\color{gray},
	keywordstyle=\color{blue},
	commentstyle=\color{dkgreen},
	stringstyle=\color{mauve},
	breaklines=true,
	breakatwhitespace=true,
	tabsize=3
}

\usepackage{import}
\usepackage{pdfpages}
\usepackage{transparent}
\usepackage{xcolor}

\begin{document}
\title{Esame Probabilità e Statistica}
\author{Università di Verona - Irimie Fabio VR501504}
\date{4 Luglio 2024}
\maketitle

\tableofcontents
\pagebreak

\section{Domanda 1}
Nei test effettuati sui bambini dagli 8 ai 10 anni (popolazione
di riferimento) è risultato che la loro soglia di attenzione ha
legge normale di media \( \mu = 40 \)  minuti e deviazione standard  
\( \sigma = 8 \)  minuti.

\subsection{Parte 1}
Calcolare la probabilità che un bambino scelto a caso nella
popolazione abbia una soglia di attenzione maggiore di 45 minuti
(usare la seconda cifra decimale)

\subsubsection{Soluzione senza R}
Standardizzo la normale:
\[
  Z = \frac{\bar{X} - \mu}{\sigma} = \frac{45 - 40}{8} = 0.625
\] 
Vogliamo calcolare:
\[
P(Z > 0.625) = 1 - P(Z \le 0.625) = 1 - \phi(0.625) = 1 - 0.73237 \approx 0.27
\] 

\subsubsection{Soluzione con R}
\begin{lstlisting}[language=R]
  pnorm(45, mean = 40, sd = 8, lower.tail = FALSE)
  # 0.2659855
\end{lstlisting}

\subsection{Parte 2}
Scelti a caso \( n=500 \) bambini dalla popolazione, si consideri la
media campionaria delle soglie di attenzione. Calcolare la 
probabilità che la media  delle soglie di attenzione del 
campione sia inferiore di 40 minuti (usa 1 cifra decimale).

\subsubsection{Soluzione senza R}
Essendo che la media della normale e la media campionaria coincidono,
allora la probabilità che la media campionaria sia inferiore a 40
è la stessa di \( P(Z < 0) \) che è 0.5.

\subsubsection{Soluzione con R}
\begin{lstlisting}[language=R]
  pnorm(40, mean = 40, sd = 8)
  # 0.5
\end{lstlisting}

\subsection{Parte 3}
Scelti a caso \( n=10 \) dalla popolazione, calcolare la probabilità 
che al massimo due di loro abbiano una soglia di attenzione 
maggiore di 45 minuti (usa 2 cifre decimali).

\subsubsection{Soluzione senza R}
\[
  p = 1 - P(Z \le 0.625) = 1 - 0.73237 = 0.26763
\] 
\[
P(X \le 2) = P(X = 0) + P(X = 1) + P(X = 2) = \sum_{k=0}^{2}
\binom{10}{k} p^k (1-p)^{10-k} =
\] 
\[
  = \frac{10!}{0!10!} (0.73237)^{10} + \frac{10!}{1!9!} (0.26763)^1 (0.73237)^9 + \frac{10!}{2!8!} (0.26763)^2 (0.73237)^8=
\] 
\[
 P(X \le 2) \approx 0.48
\] 

\subsubsection{Soluzione con R}
\begin{lstlisting}[language=R]
  pnorm(45, mean = 40, sd = 8, lower.tail = FALSE)
  # 0.26598
  pbinom(2, size = 10, prob = 0.26598)
  # 0.478
\end{lstlisting}

\section{Domanda 2}
In un sondaggio tra gli italiani su una certa questione il \( 56\% \)
degli intervistati si è detto favorevole ed il restante \( 44\% \)
contrario. Il \( 53\% \) degli intervistati erano maschi e di
questi il \( 60\% \) era favorevole. Scelto a caso un 
intervistato calcolare la probabilità che sia femmina sapendo 
che è favorevole.

\subsection{Soluzione}
\begin{itemize}
  \item \textbf{S} = "essere favorevole"
  \item \textbf{N} = "essere contrario"
  \item \textbf{M} = "essere maschio"
  \item \textbf{F} = "essere femmina"
\end{itemize}
\[
P(S) = 0.56 \quad P(N) = 0.44
\] 
\[
P(M) = 0.53 \quad P(F) = 0.47
\] 
\[
P(S|M) = 0.6
\] 
Vogliamo calcolare \( P(F|S) \):
\[
P(F|S) = 1 - P(M|S)
\] 

Usando la formula di Bayes otteniamo:
\[
P(M|S) = \frac{P(S|M) \cdot P(M)}{P(S)} = \frac{0.6 \cdot 0.53}{0.56} = 0.5678
\] 
\[
P(F|S) = 1 - 0.5678 = 0.4322
\] 

\section{Domande 3-7}
Vogliamo testare le luminosità di picco degli schermi di un
certo modello di computer portatili. Testiamo la luminosità di 
picco per \( n = 7 \)  portatili con lo stesso tipo di schermo.
Le luminosità misurate, in candele-per-metro-quadro (nits), sono:
\[
770, \quad 800, \quad 760, \quad 780, \quad 790, \quad 800, \quad 760
\] 
Il produttore pubblica una luminosità media pari a \( \mu_0 = 800\,\text{nits} \) 

\noindent con deviazione standard \( \sigma = 50\,\text{nits} \) 

\vspace{1em}
\noindent Si dia per buona la varianza fornita dal produttore.
Vogliamo studiare la luminosità media.

\subsection{Domanda 3}
Si stimi la media del campione misurato.

\subsubsection{Soluzione senza R}
La media del campione è:
\[
  \bar{x} = \frac{770 + 800 + 760 + 780 + 790 + 800 + 760}{7} = 780
\] 
\subsubsection{Soluzione con R}
\begin{lstlisting}[language=R]
  x <- c(770, 800, 760, 780, 790, 800, 760)
  mean(x)
  # 780
\end{lstlisting}

\subsection{Domanda 4}
Si trovi, per la luminosita' media degli schermi, l'intervallo 
bilaterale di confidenza uguale a 0.98.

\subsubsection{Soluzione con R}
Per trovare l'intervallo di confidenza bilaterale bisogna usare
la seguente formula:
\[
  \left( \bar{x} - z_{\frac{\alpha}{2}} \frac{\sigma}{\sqrt{n}} \,,\,
    \bar{x} + z_{\frac{\alpha}{2}} \frac{\sigma}{\sqrt{n}}
  \right) 
\] 
Dove \( z_{\frac{\alpha}{2}} \) è il quantile della normale standard
corrispondente a \( \frac{\alpha}{2} \) e \( \alpha = 0.02 \).
\begin{lstlisting}[language=R]
  alpha <- 0.02
  z <- qnorm(1 - alpha / 2)
  sigma <- 50
  n <- 7
  x <- 780
  upper <- round(x + z * sigma / sqrt(n))
  lower <- round(x - z * sigma / sqrt(n))
  cat("(", lower, ",", upper, ")\n")
  # ( 736 , 824 )
\end{lstlisting}

\subsection{Domanda 5}
Con riferimento alla domanda precedente, quanti schermi si dovrebbero
testare per avere un intervallo con confidenza di 0.98 non più
largo di 10 nits?

\subsubsection{Soluzione con R}
\begin{lstlisting}[language=R]
  alpha <- 0.02
  z <- qnorm(1 - alpha / 2)
  sigma <- 50
  n <- 542
  x <- 780
  upper <- x + z * sigma / sqrt(n)
  lower <- x - z * sigma / sqrt(n)
  cat("(", round(lower), ",", round(upper), ")\n")
  # ( 775 , 785 )

  diff <- upper - lower
  cat("Differenza:" , diff , "\n")
  # Differenza: 9.99252
\end{lstlisting}
La risposta è \( n \ge 156 \) 

\subsection{Domanda 6}
Calcolare la statistica-test per la media della popolazione.

\subsubsection{Soluzione}
La statistica-test a varianza nota è data da:
\[
  \frac{\bar{x} - \mu_0}{\frac{\sigma }{\sqrt{n} }} =
\frac{780 - 800}{\frac{50}{\sqrt{7} }} \approx -1.058
\] 

\subsection{Domanda 7}
Ora, si assuma di modificare l'ipotesi alternativa: \( H_1: \mu < 800\,\text{nits} \).
Si calcoli l'intervallo critico sinistro \( C \) con livello di
significatività pari ad \( \alpha = 0.05 \).

\subsubsection{Soluzione}
L'intervallo critico sinistro è dato da:
\[
  C = (-\infty, z_{\alpha})
\]
Dove \( z_{\alpha} \) è il quantile della normale standard corrispondente
a \( \alpha = 0.05 \).
\begin{lstlisting}[language=R]
  alpha <- 0.05
  z <- qnorm(alpha)
  cat("(", -Inf, ",", z, ")\n")
  # ( -Inf , -1.644854 )
\end{lstlisting}

\section{Domanda 12}
Da una rilevazione statistica risulta che il numero di incidenti
mensili nel tratto di autostrada A4 Venezia-Trieste segue una 
distribuzione di Poisson con media 16. 

\subsection{Parte 1}
Calcolare la probabilità che in un mese ci siano più di 20 incidenti
(usa 3 cifre decimali).

\subsubsection{Soluzione}
\[
P(X > 20)
\] 
\begin{lstlisting}[language=R]
  ppois(20,16, lower.tail = FALSE) 
  # 0.132
\end{lstlisting}

\subsection{Parte 2}
La probabilità che in un mese ci siano esattamente dai 10 ai
15 incidenti (usa 3 cifre decimali).

\subsubsection{Soluzione}
\[
P(10 < X < 15) = P(X \le 15) - P(X < 10)
\] 
\begin{lstlisting}[language=R]
  ppois(15,16) - ppois(9,16)
  # 0.423
\end{lstlisting}

\subsection{Parte 3}
Durante un certo anno si sono verificati i seguenti incidenti:
\[
13, \quad 20, \quad 9, \quad 16, \quad 22, \quad 17, \quad 10, \quad 9, \quad 20, \quad 17, \quad 17, \quad 16
\] 
Calcola la media.

\subsubsection{Soluzione}
\[
  \bar{x} = \frac{13 + 20 + 9 + 16 + 22 + 17 + 10 + 9 + 20 + 17 + 17 + 16}{12} = 15.5
\] 

\subsection{Parte 3}
Durante un certo anno si sono verificati i seguenti incidenti:
\[
13, \quad 20, \quad 9, \quad 16, \quad 22, \quad 17, \quad 10, \quad 9, \quad 20, \quad 17, \quad 17, \quad 16
\] 
Calcola il 90° percentile.

\subsubsection{Soluzione}
\begin{lstlisting}[language=R]
  data <- c(13, 20, 9, 16, 22, 17, 10, 9, 20, 17, 17, 16)
  quantile(data, 0.9)
  # 20
\end{lstlisting}

\end{document}
