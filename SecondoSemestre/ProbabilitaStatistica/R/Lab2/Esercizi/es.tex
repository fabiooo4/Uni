% Options for packages loaded elsewhere
\PassOptionsToPackage{unicode}{hyperref}
\PassOptionsToPackage{hyphens}{url}
%
\documentclass[
]{article}
\usepackage{amsmath,amssymb}
\usepackage{iftex}
\ifPDFTeX
  \usepackage[T1]{fontenc}
  \usepackage[utf8]{inputenc}
  \usepackage{textcomp} % provide euro and other symbols
\else % if luatex or xetex
  \usepackage{unicode-math} % this also loads fontspec
  \defaultfontfeatures{Scale=MatchLowercase}
  \defaultfontfeatures[\rmfamily]{Ligatures=TeX,Scale=1}
\fi
\usepackage{lmodern}
\ifPDFTeX\else
  % xetex/luatex font selection
\fi
% Use upquote if available, for straight quotes in verbatim environments
\IfFileExists{upquote.sty}{\usepackage{upquote}}{}
\IfFileExists{microtype.sty}{% use microtype if available
  \usepackage[]{microtype}
  \UseMicrotypeSet[protrusion]{basicmath} % disable protrusion for tt fonts
}{}
\makeatletter
\@ifundefined{KOMAClassName}{% if non-KOMA class
  \IfFileExists{parskip.sty}{%
    \usepackage{parskip}
  }{% else
    \setlength{\parindent}{0pt}
    \setlength{\parskip}{6pt plus 2pt minus 1pt}}
}{% if KOMA class
  \KOMAoptions{parskip=half}}
\makeatother
\usepackage{xcolor}
\usepackage[margin=1in]{geometry}
\usepackage{color}
\usepackage{fancyvrb}
\newcommand{\VerbBar}{|}
\newcommand{\VERB}{\Verb[commandchars=\\\{\}]}
\DefineVerbatimEnvironment{Highlighting}{Verbatim}{commandchars=\\\{\}}
% Add ',fontsize=\small' for more characters per line
\usepackage{framed}
\definecolor{shadecolor}{RGB}{248,248,248}
\newenvironment{Shaded}{\begin{snugshade}}{\end{snugshade}}
\newcommand{\AlertTok}[1]{\textcolor[rgb]{0.94,0.16,0.16}{#1}}
\newcommand{\AnnotationTok}[1]{\textcolor[rgb]{0.56,0.35,0.01}{\textbf{\textit{#1}}}}
\newcommand{\AttributeTok}[1]{\textcolor[rgb]{0.13,0.29,0.53}{#1}}
\newcommand{\BaseNTok}[1]{\textcolor[rgb]{0.00,0.00,0.81}{#1}}
\newcommand{\BuiltInTok}[1]{#1}
\newcommand{\CharTok}[1]{\textcolor[rgb]{0.31,0.60,0.02}{#1}}
\newcommand{\CommentTok}[1]{\textcolor[rgb]{0.56,0.35,0.01}{\textit{#1}}}
\newcommand{\CommentVarTok}[1]{\textcolor[rgb]{0.56,0.35,0.01}{\textbf{\textit{#1}}}}
\newcommand{\ConstantTok}[1]{\textcolor[rgb]{0.56,0.35,0.01}{#1}}
\newcommand{\ControlFlowTok}[1]{\textcolor[rgb]{0.13,0.29,0.53}{\textbf{#1}}}
\newcommand{\DataTypeTok}[1]{\textcolor[rgb]{0.13,0.29,0.53}{#1}}
\newcommand{\DecValTok}[1]{\textcolor[rgb]{0.00,0.00,0.81}{#1}}
\newcommand{\DocumentationTok}[1]{\textcolor[rgb]{0.56,0.35,0.01}{\textbf{\textit{#1}}}}
\newcommand{\ErrorTok}[1]{\textcolor[rgb]{0.64,0.00,0.00}{\textbf{#1}}}
\newcommand{\ExtensionTok}[1]{#1}
\newcommand{\FloatTok}[1]{\textcolor[rgb]{0.00,0.00,0.81}{#1}}
\newcommand{\FunctionTok}[1]{\textcolor[rgb]{0.13,0.29,0.53}{\textbf{#1}}}
\newcommand{\ImportTok}[1]{#1}
\newcommand{\InformationTok}[1]{\textcolor[rgb]{0.56,0.35,0.01}{\textbf{\textit{#1}}}}
\newcommand{\KeywordTok}[1]{\textcolor[rgb]{0.13,0.29,0.53}{\textbf{#1}}}
\newcommand{\NormalTok}[1]{#1}
\newcommand{\OperatorTok}[1]{\textcolor[rgb]{0.81,0.36,0.00}{\textbf{#1}}}
\newcommand{\OtherTok}[1]{\textcolor[rgb]{0.56,0.35,0.01}{#1}}
\newcommand{\PreprocessorTok}[1]{\textcolor[rgb]{0.56,0.35,0.01}{\textit{#1}}}
\newcommand{\RegionMarkerTok}[1]{#1}
\newcommand{\SpecialCharTok}[1]{\textcolor[rgb]{0.81,0.36,0.00}{\textbf{#1}}}
\newcommand{\SpecialStringTok}[1]{\textcolor[rgb]{0.31,0.60,0.02}{#1}}
\newcommand{\StringTok}[1]{\textcolor[rgb]{0.31,0.60,0.02}{#1}}
\newcommand{\VariableTok}[1]{\textcolor[rgb]{0.00,0.00,0.00}{#1}}
\newcommand{\VerbatimStringTok}[1]{\textcolor[rgb]{0.31,0.60,0.02}{#1}}
\newcommand{\WarningTok}[1]{\textcolor[rgb]{0.56,0.35,0.01}{\textbf{\textit{#1}}}}
\usepackage{graphicx}
\makeatletter
\def\maxwidth{\ifdim\Gin@nat@width>\linewidth\linewidth\else\Gin@nat@width\fi}
\def\maxheight{\ifdim\Gin@nat@height>\textheight\textheight\else\Gin@nat@height\fi}
\makeatother
% Scale images if necessary, so that they will not overflow the page
% margins by default, and it is still possible to overwrite the defaults
% using explicit options in \includegraphics[width, height, ...]{}
\setkeys{Gin}{width=\maxwidth,height=\maxheight,keepaspectratio}
% Set default figure placement to htbp
\makeatletter
\def\fps@figure{htbp}
\makeatother
\setlength{\emergencystretch}{3em} % prevent overfull lines
\providecommand{\tightlist}{%
  \setlength{\itemsep}{0pt}\setlength{\parskip}{0pt}}
\setcounter{secnumdepth}{-\maxdimen} % remove section numbering
\ifLuaTeX
  \usepackage{selnolig}  % disable illegal ligatures
\fi
\IfFileExists{bookmark.sty}{\usepackage{bookmark}}{\usepackage{hyperref}}
\IfFileExists{xurl.sty}{\usepackage{xurl}}{} % add URL line breaks if available
\urlstyle{same}
\hypersetup{
  pdftitle={Lab2},
  pdfauthor={Irimie Fabio},
  hidelinks,
  pdfcreator={LaTeX via pandoc}}

\title{Lab2}
\usepackage{etoolbox}
\makeatletter
\providecommand{\subtitle}[1]{% add subtitle to \maketitle
  \apptocmd{\@title}{\par {\large #1 \par}}{}{}
}
\makeatother
\subtitle{Exercises}
\author{Irimie Fabio}
\date{}

\begin{document}
\maketitle

{
\setcounter{tocdepth}{2}
\tableofcontents
}
\hypertarget{exercise-1---create-a-new-mean-and-sd-function}{%
\section{Exercise 1 - Create a new mean and sd
function}\label{exercise-1---create-a-new-mean-and-sd-function}}

\begin{itemize}
\item
  A:

  Create the Lab2 project. Use the same structure used for Lab1:

  \begin{itemize}
  \tightlist
  \item
    scripts,
  \item
    plots,
  \item
    data
  \end{itemize}
\item
  B:

  Install the palmerpenguins package, load the penguins dataset or,
  alternatively, download the .RData object from moodle and import it
  after placing it inside the data directory of the project (hint: use
  the load() function).
\item
  C:

  Compute the mean, the standard deviation, and the median for the
  numeric variables of the dataset.
\item
  D:

  Create a function called stat\_auto that simultaneously returns both
  the mean and the standard deviation of a given vector (hint: return an
  object of type list or simply a vector). Then try it on the same
  numeric variables in C. to check the results (hint: if you obtain NA
  maybe you forgot to remove NA terms in the vector).
\item
  E:

  Create a function called stat\_manual that simultaneously returns both
  the mean and the standard deviation of a given vector without using
  the mean() and the sd() functions (hint: you can use length(), sum(),
  and na.omit() functions). Then try it on the same numeric variables in
  C. to check the results.
\end{itemize}

\hypertarget{b}{%
\subsection{B}\label{b}}

\begin{Shaded}
\begin{Highlighting}[]
\FunctionTok{library}\NormalTok{(palmerpenguins)}
\FunctionTok{data}\NormalTok{(penguins)}
\end{Highlighting}
\end{Shaded}

\hypertarget{c}{%
\subsection{C}\label{c}}

\begin{Shaded}
\begin{Highlighting}[]
\CommentTok{\# Means}
\FunctionTok{cat}\NormalTok{(}\StringTok{"Means: }\SpecialCharTok{\textbackslash{}n}\StringTok{"}\NormalTok{)}
\DocumentationTok{\#\# Means:}
\FunctionTok{colMeans}\NormalTok{(penguins[, }\FunctionTok{c}\NormalTok{(}\DecValTok{3}\SpecialCharTok{:}\DecValTok{6}\NormalTok{, }\DecValTok{8}\NormalTok{)], }\AttributeTok{na.rm =} \ConstantTok{TRUE}\NormalTok{)}
\DocumentationTok{\#\#    bill\_length\_mm     bill\_depth\_mm flipper\_length\_mm       body\_mass\_g }
\DocumentationTok{\#\#          43.92193          17.15117         200.91520        4201.75439 }
\DocumentationTok{\#\#              year }
\DocumentationTok{\#\#        2008.02907}
\FunctionTok{cat}\NormalTok{(}\StringTok{"}\SpecialCharTok{\textbackslash{}n}\StringTok{"}\NormalTok{)}

\CommentTok{\# Medians}
\FunctionTok{cat}\NormalTok{(}\StringTok{"Medians: }\SpecialCharTok{\textbackslash{}n}\StringTok{"}\NormalTok{)}
\DocumentationTok{\#\# Medians:}
\FunctionTok{sapply}\NormalTok{(penguins[, }\FunctionTok{c}\NormalTok{(}\DecValTok{3}\SpecialCharTok{:}\DecValTok{6}\NormalTok{, }\DecValTok{8}\NormalTok{)], median, }\AttributeTok{na.rm =} \ConstantTok{TRUE}\NormalTok{)}
\DocumentationTok{\#\#    bill\_length\_mm     bill\_depth\_mm flipper\_length\_mm       body\_mass\_g }
\DocumentationTok{\#\#             44.45             17.30            197.00           4050.00 }
\DocumentationTok{\#\#              year }
\DocumentationTok{\#\#           2008.00}
\FunctionTok{cat}\NormalTok{(}\StringTok{"}\SpecialCharTok{\textbackslash{}n}\StringTok{"}\NormalTok{)}

\CommentTok{\# Standard deviations}
\FunctionTok{cat}\NormalTok{(}\StringTok{"Standard deviations: }\SpecialCharTok{\textbackslash{}n}\StringTok{"}\NormalTok{)}
\DocumentationTok{\#\# Standard deviations:}
\FunctionTok{sapply}\NormalTok{(penguins[, }\FunctionTok{c}\NormalTok{(}\DecValTok{3}\SpecialCharTok{:}\DecValTok{6}\NormalTok{, }\DecValTok{8}\NormalTok{)], sd, }\AttributeTok{na.rm =} \ConstantTok{TRUE}\NormalTok{)}
\DocumentationTok{\#\#    bill\_length\_mm     bill\_depth\_mm flipper\_length\_mm       body\_mass\_g }
\DocumentationTok{\#\#         5.4595837         1.9747932        14.0617137       801.9545357 }
\DocumentationTok{\#\#              year }
\DocumentationTok{\#\#         0.8183559}
\FunctionTok{cat}\NormalTok{(}\StringTok{"}\SpecialCharTok{\textbackslash{}n}\StringTok{"}\NormalTok{)}
\end{Highlighting}
\end{Shaded}

\hypertarget{d}{%
\subsection{D}\label{d}}

\begin{Shaded}
\begin{Highlighting}[]
\NormalTok{stat\_auto }\OtherTok{\textless{}{-}} \ControlFlowTok{function}\NormalTok{(vec, }\AttributeTok{na.rm =} \ConstantTok{FALSE}\NormalTok{) \{}
  \ControlFlowTok{if}\NormalTok{ (na.rm) \{}
\NormalTok{    mean }\OtherTok{\textless{}{-}} \FunctionTok{mean}\NormalTok{(vec, }\AttributeTok{na.rm =} \ConstantTok{TRUE}\NormalTok{)}
\NormalTok{    sd }\OtherTok{\textless{}{-}} \FunctionTok{sd}\NormalTok{(vec, }\AttributeTok{na.rm =} \ConstantTok{TRUE}\NormalTok{)}

    \FunctionTok{return}\NormalTok{(}\FunctionTok{list}\NormalTok{(}\StringTok{"mean"} \OtherTok{=}\NormalTok{ mean, }\StringTok{"sd"} \OtherTok{=}\NormalTok{ sd))}
\NormalTok{  \}}

\NormalTok{  mean }\OtherTok{\textless{}{-}} \FunctionTok{mean}\NormalTok{(vec)}
\NormalTok{  sd }\OtherTok{\textless{}{-}} \FunctionTok{sd}\NormalTok{(vec)}

  \FunctionTok{return}\NormalTok{(}\FunctionTok{list}\NormalTok{(}\StringTok{"mean"} \OtherTok{=}\NormalTok{ mean, }\StringTok{"sd"} \OtherTok{=}\NormalTok{ sd))}
\NormalTok{\}}

\FunctionTok{sapply}\NormalTok{(penguins[, }\FunctionTok{c}\NormalTok{(}\DecValTok{3}\SpecialCharTok{:}\DecValTok{6}\NormalTok{, }\DecValTok{8}\NormalTok{)], stat\_auto, }\AttributeTok{na.rm =} \ConstantTok{TRUE}\NormalTok{)}
\end{Highlighting}
\end{Shaded}

\begin{verbatim}
##      bill_length_mm bill_depth_mm flipper_length_mm body_mass_g year     
## mean 43.92193       17.15117      200.9152          4201.754    2008.029 
## sd   5.459584       1.974793      14.06171          801.9545    0.8183559
\end{verbatim}

\hypertarget{e}{%
\subsection{E}\label{e}}

\begin{Shaded}
\begin{Highlighting}[]
\NormalTok{stat\_manual }\OtherTok{\textless{}{-}} \ControlFlowTok{function}\NormalTok{(vec, }\AttributeTok{na.rm =} \ConstantTok{FALSE}\NormalTok{) \{}
  \ControlFlowTok{if}\NormalTok{ (na.rm) \{}
\NormalTok{    sum }\OtherTok{\textless{}{-}} \FunctionTok{sum}\NormalTok{(vec, }\AttributeTok{na.rm =} \ConstantTok{TRUE}\NormalTok{)}
\NormalTok{    mean }\OtherTok{\textless{}{-}}\NormalTok{ sum }\SpecialCharTok{/} \FunctionTok{na.omit}\NormalTok{(}\FunctionTok{length}\NormalTok{(vec))}

\NormalTok{    sum }\OtherTok{\textless{}{-}} \FunctionTok{sum}\NormalTok{((vec }\SpecialCharTok{{-}}\NormalTok{ mean)}\SpecialCharTok{\^{}}\DecValTok{2}\NormalTok{, }\AttributeTok{na.rm =} \ConstantTok{TRUE}\NormalTok{)}
\NormalTok{    denom }\OtherTok{\textless{}{-}} \FunctionTok{na.omit}\NormalTok{(}\FunctionTok{length}\NormalTok{(vec)) }\SpecialCharTok{{-}} \DecValTok{1}
\NormalTok{    varianza }\OtherTok{\textless{}{-}}\NormalTok{ sum }\SpecialCharTok{/}\NormalTok{ denom}

\NormalTok{    sd }\OtherTok{\textless{}{-}} \FunctionTok{sqrt}\NormalTok{(varianza)}

    \FunctionTok{return}\NormalTok{(}\FunctionTok{list}\NormalTok{(}\StringTok{"mean"} \OtherTok{=}\NormalTok{ mean, }\StringTok{"sd"} \OtherTok{=}\NormalTok{ sd))}
\NormalTok{  \}}

\NormalTok{  sum }\OtherTok{\textless{}{-}} \FunctionTok{sum}\NormalTok{(vec)}
\NormalTok{  mean }\OtherTok{\textless{}{-}}\NormalTok{ sum }\SpecialCharTok{/} \FunctionTok{length}\NormalTok{(vec)}

\NormalTok{  sum }\OtherTok{\textless{}{-}} \FunctionTok{sum}\NormalTok{((vec }\SpecialCharTok{{-}}\NormalTok{ mean)}\SpecialCharTok{\^{}}\DecValTok{2}\NormalTok{)}
\NormalTok{  denom }\OtherTok{\textless{}{-}} \FunctionTok{length}\NormalTok{(vec) }\SpecialCharTok{{-}} \DecValTok{1}
\NormalTok{  varianza }\OtherTok{\textless{}{-}}\NormalTok{ sum }\SpecialCharTok{/}\NormalTok{ denom}

\NormalTok{  sd }\OtherTok{\textless{}{-}} \FunctionTok{sqrt}\NormalTok{(varianza)}

  \FunctionTok{return}\NormalTok{(}\FunctionTok{list}\NormalTok{(}\StringTok{"mean"} \OtherTok{=}\NormalTok{ mean, }\StringTok{"sd"} \OtherTok{=}\NormalTok{ sd))}
\NormalTok{\}}

\FunctionTok{sapply}\NormalTok{(penguins[, }\FunctionTok{c}\NormalTok{(}\DecValTok{3}\SpecialCharTok{:}\DecValTok{6}\NormalTok{, }\DecValTok{8}\NormalTok{)], stat\_manual, }\AttributeTok{na.rm =} \ConstantTok{TRUE}\NormalTok{)}
\end{Highlighting}
\end{Shaded}

\begin{verbatim}
##      bill_length_mm bill_depth_mm flipper_length_mm body_mass_g year     
## mean 43.66657       17.05145      199.7471          4177.326    2008.029 
## sd   5.449612       1.971543      14.06909          799.985     0.8183559
\end{verbatim}

\hypertarget{exercise-2---table-of-frequencies}{%
\section{Exercise 2 - Table of
frequencies}\label{exercise-2---table-of-frequencies}}

\begin{itemize}
\item
  A:

  In the penguins dataset, transform a numeric variable to a categorical
  one by aggregating values into classes. Consider the flipper length
  variable and create 10mm wide classes using the cut() function (hint:
  use the range() function to determine the min and max values of the
  variable, then define a sequence for the cuts).
\item
  B:

  Use the table() function on the new variable generated by cut(). Then
  transform it into a data.frame object. Rename the columns accordingly
  using the colnames() function (hint: the second column correspond to
  the absolute frequencies).
\item
  C:

  Add the the columns for: relative frequencies, cumulative absolute
  frequencies, and cumulative relative frequencies.
\item
  D:

  Use the geom\_col() function to plot the frequence of each class.
  Then, using the geom\_text(aes(label = \ldots)) function, add the
  relative frequence as a percentage above each column (hint: substitute
  the \ldots{} with the relative frequency values. Use the round()
  function to choose the appropriate number of digits).
\end{itemize}

\hypertarget{a}{%
\subsection{A}\label{a}}

\begin{Shaded}
\begin{Highlighting}[]
\NormalTok{r }\OtherTok{\textless{}{-}} \FunctionTok{range}\NormalTok{(penguins}\SpecialCharTok{$}\NormalTok{flipper\_length\_mm, }\AttributeTok{na.rm =} \ConstantTok{TRUE}\NormalTok{)}
\NormalTok{splits }\OtherTok{\textless{}{-}} \FunctionTok{seq}\NormalTok{(r[}\DecValTok{1}\NormalTok{], r[}\DecValTok{2}\NormalTok{], }\DecValTok{10}\NormalTok{)}
\NormalTok{splits }\OtherTok{\textless{}{-}} \FunctionTok{append}\NormalTok{(splits, r[}\DecValTok{2}\NormalTok{])}

\NormalTok{classes }\OtherTok{\textless{}{-}} \FunctionTok{cut}\NormalTok{(penguins}\SpecialCharTok{$}\NormalTok{flipper\_length\_mm, splits, }\AttributeTok{ordered\_result =} \ConstantTok{TRUE}\NormalTok{)}

\FunctionTok{cat}\NormalTok{(}\StringTok{"Splits: "}\NormalTok{, splits, }\StringTok{"}\SpecialCharTok{\textbackslash{}n}\StringTok{"}\NormalTok{)}
\end{Highlighting}
\end{Shaded}

\begin{verbatim}
## Splits:  172 182 192 202 212 222 231
\end{verbatim}

\begin{Shaded}
\begin{Highlighting}[]
\FunctionTok{cat}\NormalTok{(}\StringTok{"Classes: }\SpecialCharTok{\textbackslash{}n}\StringTok{"}\NormalTok{)}
\end{Highlighting}
\end{Shaded}

\begin{verbatim}
## Classes:
\end{verbatim}

\begin{Shaded}
\begin{Highlighting}[]
\NormalTok{classes}
\end{Highlighting}
\end{Shaded}

\begin{verbatim}
##   [1] (172,182] (182,192] (192,202] <NA>      (192,202] (182,192]
##   [7] (172,182] (192,202] (192,202] (182,192] (182,192] (172,182]
##  [13] (172,182] (182,192] (192,202] (182,192] (192,202] (192,202]
##  [19] (182,192] (192,202] (172,182] (172,182] (182,192] (182,192]
##  [25] (172,182] (182,192] (182,192] (182,192] <NA>      (172,182]
##  [31] (172,182] (172,182] (182,192] (182,192] (192,202] (192,202]
##  [37] (182,192] (172,182] (172,182] (182,192] (172,182] (192,202]
##  [43] (182,192] (192,202] (182,192] (182,192] (172,182] (172,182]
##  [49] (182,192] (182,192] (182,192] (182,192] (182,192] (192,202]
##  [55] (182,192] (182,192] (182,192] (192,202] (172,182] (192,202]
##  [61] (182,192] (192,202] (182,192] (182,192] (182,192] (182,192]
##  [67] (192,202] (182,192] (182,192] (192,202] (182,192] (182,192]
##  [73] (192,202] (192,202] (182,192] (192,202] (182,192] (182,192]
##  [79] (182,192] (192,202] (182,192] (192,202] (182,192] (192,202]
##  [85] (182,192] (192,202] (182,192] (182,192] (182,192] (182,192]
##  [91] (192,202] (202,212] (182,192] (182,192] (182,192] (202,212]
##  [97] (182,192] (192,202] (172,182] (182,192] (182,192] (202,212]
## [103] (182,192] (182,192] (192,202] (182,192] (192,202] (182,192]
## [109] (172,182] (192,202] (192,202] (182,192] (192,202] (192,202]
## [115] (182,192] (192,202] (182,192] (192,202] (182,192] (182,192]
## [121] (182,192] (192,202] (172,182] (192,202] (182,192] (192,202]
## [127] (182,192] (192,202] (182,192] (202,212] (182,192] (192,202]
## [133] (192,202] (192,202] (182,192] (182,192] (182,192] (192,202]
## [139] (182,192] (192,202] (192,202] (182,192] (182,192] (182,192]
## [145] (182,192] (182,192] (182,192] (182,192] (192,202] (192,202]
## [151] (182,192] (192,202] (202,212] (222,231] (202,212] (212,222]
## [157] (212,222] (202,212] (202,212] (212,222] (202,212] (212,222]
## [163] (212,222] (212,222] (212,222] (212,222] (202,212] (212,222]
## [169] (202,212] (212,222] (202,212] (212,222] (212,222] (212,222]
## [175] (212,222] (212,222] (212,222] (212,222] (212,222] (212,222]
## [181] (202,212] (212,222] (212,222] (202,212] (202,212] (222,231]
## [187] (212,222] (212,222] (212,222] (212,222] (202,212] (202,212]
## [193] (202,212] (222,231] (202,212] (212,222] (212,222] (212,222]
## [199] (202,212] (222,231] (212,222] (212,222] (202,212] (212,222]
## [205] (202,212] (222,231] (212,222] (212,222] (202,212] (212,222]
## [211] (202,212] (222,231] (202,212] (212,222] (212,222] (222,231]
## [217] (212,222] (222,231] (212,222] (222,231] (212,222] (222,231]
## [223] (212,222] (212,222] (212,222] (212,222] (212,222] (222,231]
## [229] (202,212] (212,222] (212,222] (222,231] (202,212] (212,222]
## [235] (202,212] (222,231] (202,212] (222,231] (212,222] (212,222]
## [241] (202,212] (222,231] (212,222] (222,231] (202,212] (222,231]
## [247] (212,222] (222,231] (212,222] (212,222] (202,212] (222,231]
## [253] (212,222] (222,231] (212,222] (222,231] (212,222] (212,222]
## [259] (202,212] (212,222] (202,212] (202,212] (212,222] (222,231]
## [265] (212,222] (222,231] (212,222] (222,231] (212,222] (212,222]
## [271] (212,222] <NA>      (212,222] (212,222] (202,212] (212,222]
## [277] (182,192] (192,202] (192,202] (182,192] (192,202] (192,202]
## [283] (172,182] (192,202] (192,202] (192,202] (192,202] (192,202]
## [289] (182,192] (192,202] (182,192] (192,202] (192,202] (172,182]
## [295] (182,192] (192,202] (172,182] (182,192] (182,192] (192,202]
## [301] (192,202] (192,202] (192,202] (192,202] (182,192] (202,212]
## [307] (182,192] (192,202] (182,192] (202,212] (192,202] (192,202]
## [313] (192,202] (202,212] (182,192] (202,212] (202,212] (182,192]
## [319] (192,202] (192,202] (192,202] (192,202] (182,192] (202,212]
## [325] (182,192] (192,202] (192,202] (192,202] (192,202] (202,212]
## [331] (182,192] (192,202] (182,192] (202,212] (192,202] (192,202]
## [337] (202,212] (182,192] (192,202] (202,212] (192,202] (192,202]
## [343] (202,212] (192,202]
## 6 Levels: (172,182] < (182,192] < (192,202] < ... < (222,231]
\end{verbatim}

\hypertarget{b-1}{%
\subsection{B}\label{b-1}}

\begin{Shaded}
\begin{Highlighting}[]
\FunctionTok{table}\NormalTok{(classes)}
\end{Highlighting}
\end{Shaded}

\begin{verbatim}
## classes
## (172,182] (182,192] (192,202] (202,212] (212,222] (222,231] 
##        22        96        85        47        67        24
\end{verbatim}

\hypertarget{exercise-3---histogram-boxplot-and-quartiles}{%
\section{Exercise 3 - Histogram, Boxplot and
quartiles}\label{exercise-3---histogram-boxplot-and-quartiles}}

\begin{itemize}
\item
  A:

  Using the geom\_histogram() function of the ggplot2 package plot the
  flipper length distribution coloring each species with a different
  color (hint: use the fill argument of the aes() function to fill the\\
  histogram area and the position = ``identity'' argument of the
  geom\_histogram()). Play with the binwidth argument. Try to insert y =
  ..density.. in aes(). Do you notice any change?
\item
  B:

  About the flipper length, for each species of penguins compute the:

  \begin{enumerate}
  \def\labelenumi{\arabic{enumi}.}
  \tightlist
  \item
    Sample mean
  \item
    Sample median
  \item
    Sample standard deviation (use a division by \(n-1\))
  \item
    Sample variance
  \end{enumerate}

  (hint: to choose only a specific species use
  penguins{[}penguins\$species == ``Gentoo'',{]})
\item
  C:

  Using the geom\_boxplot() function of the ggplot2 package plot the
  boxplot for the flipper length variable coloring each species with a
  different color (hint: use the color argument of the aes() function).
\item
  D:

  Compute the flipper length quartiles for the ``Gentoo'' penguins (Q1,
  Q2, Q3).
\item
  E:

  Calculate the flipper length 40th percentile for the ``Adelie''
  penguins.
\end{itemize}

\hypertarget{exercise-4---multiple-boxplots-from-scratch}{%
\section{Exercise 4 - Multiple boxplots from
scratch}\label{exercise-4---multiple-boxplots-from-scratch}}

\begin{itemize}
\item
  A:

  Generate random data with some structure, and create one data set for
  each day of the week (hint: use the for() cycle, data should have 7
  columns). At the end you should obtain a matrix with N rows (N =
  number of random number to generate each time) and 7 columns (one for
  each day of the week).
\item
  B:

  Go from a wide to a long data format. You should create a data.frame
  object with exactly two columns. One contains the values created in
  A., the other contains the corresponding day of the week.
\item
  C:

  Plot the seven boxplots (one for each day of the week) in one graph,
  horizontally oriented (hint: coord\_flip() function translates the
  axes, the ``limits'' argument of scale\_x\_discrete() allows you to
  reorder the axis labels).
\end{itemize}

\hypertarget{exercise-5---exploratory-analysis-of-data}{%
\section{Exercise 5 - Exploratory analysis of
data}\label{exercise-5---exploratory-analysis-of-data}}

Penguins dataset does not contain their weights and flipper lengths
only. Many other variables are available. Let's explore it a little
more:

\begin{itemize}
\item
  A:

  How many islands are there? And how many penguins are present in each
  isle? Are the 3 species of penguins living together? (hint: use the
  table() function).
\item
  B:

  Try to use the geom\_bar() or geom\_col() functions to graphically
  represent the population of each island, colored by species (hint:
  islands in the x-axis, number of penguins in the y-axis).
\item
  C:

  Use a scatter plot to represent flipper length vs.~body mass. Color
  the point according to the ``sex'' variable. Try to use facets to see
  if there are differences across species (hint: use
  facet\_grid(\textasciitilde{} species) function to add facets for
  species).
\item
  D:

  The numeric variables shows some interesting relationships. Are they
  correlated? Use the cor() and the corrplot() functions to study
  correlations between numeric variables (hint: try to google corrplot()
  to see which package you have to install to use it).
\item
  E:

  Choose a pair of numeric variables, compute the correlation between
  them without using the cor() function (hint: remember the formula).
\item
  F:

  Plot the scatter plot for bill length vs.~bill depth. Color the points
  by species. Use the function geom\_smooth(formula = ``y
  \textasciitilde{} x'') to add a line to represent the linear
  relationship between the two variables. Then, again, use
  geom\_smooth(formula = ``y \textasciitilde{} x'') colored by species.
  What are you noticing?
\end{itemize}

\end{document}
