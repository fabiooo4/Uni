% Options for packages loaded elsewhere
\PassOptionsToPackage{unicode}{hyperref}
\PassOptionsToPackage{hyphens}{url}
%
\documentclass[
]{article}
\usepackage{amsmath,amssymb}
\usepackage{iftex}
\ifPDFTeX
  \usepackage[T1]{fontenc}
  \usepackage[utf8]{inputenc}
  \usepackage{textcomp} % provide euro and other symbols
\else % if luatex or xetex
  \usepackage{unicode-math} % this also loads fontspec
  \defaultfontfeatures{Scale=MatchLowercase}
  \defaultfontfeatures[\rmfamily]{Ligatures=TeX,Scale=1}
\fi
\usepackage{lmodern}
\ifPDFTeX\else
  % xetex/luatex font selection
\fi
% Use upquote if available, for straight quotes in verbatim environments
\IfFileExists{upquote.sty}{\usepackage{upquote}}{}
\IfFileExists{microtype.sty}{% use microtype if available
  \usepackage[]{microtype}
  \UseMicrotypeSet[protrusion]{basicmath} % disable protrusion for tt fonts
}{}
\makeatletter
\@ifundefined{KOMAClassName}{% if non-KOMA class
  \IfFileExists{parskip.sty}{%
    \usepackage{parskip}
  }{% else
    \setlength{\parindent}{0pt}
    \setlength{\parskip}{6pt plus 2pt minus 1pt}}
}{% if KOMA class
  \KOMAoptions{parskip=half}}
\makeatother
\usepackage{xcolor}
\usepackage[margin=1in]{geometry}
\usepackage{color}
\usepackage{fancyvrb}
\newcommand{\VerbBar}{|}
\newcommand{\VERB}{\Verb[commandchars=\\\{\}]}
\DefineVerbatimEnvironment{Highlighting}{Verbatim}{commandchars=\\\{\}}
% Add ',fontsize=\small' for more characters per line
\usepackage{framed}
\definecolor{shadecolor}{RGB}{248,248,248}
\newenvironment{Shaded}{\begin{snugshade}}{\end{snugshade}}
\newcommand{\AlertTok}[1]{\textcolor[rgb]{0.94,0.16,0.16}{#1}}
\newcommand{\AnnotationTok}[1]{\textcolor[rgb]{0.56,0.35,0.01}{\textbf{\textit{#1}}}}
\newcommand{\AttributeTok}[1]{\textcolor[rgb]{0.13,0.29,0.53}{#1}}
\newcommand{\BaseNTok}[1]{\textcolor[rgb]{0.00,0.00,0.81}{#1}}
\newcommand{\BuiltInTok}[1]{#1}
\newcommand{\CharTok}[1]{\textcolor[rgb]{0.31,0.60,0.02}{#1}}
\newcommand{\CommentTok}[1]{\textcolor[rgb]{0.56,0.35,0.01}{\textit{#1}}}
\newcommand{\CommentVarTok}[1]{\textcolor[rgb]{0.56,0.35,0.01}{\textbf{\textit{#1}}}}
\newcommand{\ConstantTok}[1]{\textcolor[rgb]{0.56,0.35,0.01}{#1}}
\newcommand{\ControlFlowTok}[1]{\textcolor[rgb]{0.13,0.29,0.53}{\textbf{#1}}}
\newcommand{\DataTypeTok}[1]{\textcolor[rgb]{0.13,0.29,0.53}{#1}}
\newcommand{\DecValTok}[1]{\textcolor[rgb]{0.00,0.00,0.81}{#1}}
\newcommand{\DocumentationTok}[1]{\textcolor[rgb]{0.56,0.35,0.01}{\textbf{\textit{#1}}}}
\newcommand{\ErrorTok}[1]{\textcolor[rgb]{0.64,0.00,0.00}{\textbf{#1}}}
\newcommand{\ExtensionTok}[1]{#1}
\newcommand{\FloatTok}[1]{\textcolor[rgb]{0.00,0.00,0.81}{#1}}
\newcommand{\FunctionTok}[1]{\textcolor[rgb]{0.13,0.29,0.53}{\textbf{#1}}}
\newcommand{\ImportTok}[1]{#1}
\newcommand{\InformationTok}[1]{\textcolor[rgb]{0.56,0.35,0.01}{\textbf{\textit{#1}}}}
\newcommand{\KeywordTok}[1]{\textcolor[rgb]{0.13,0.29,0.53}{\textbf{#1}}}
\newcommand{\NormalTok}[1]{#1}
\newcommand{\OperatorTok}[1]{\textcolor[rgb]{0.81,0.36,0.00}{\textbf{#1}}}
\newcommand{\OtherTok}[1]{\textcolor[rgb]{0.56,0.35,0.01}{#1}}
\newcommand{\PreprocessorTok}[1]{\textcolor[rgb]{0.56,0.35,0.01}{\textit{#1}}}
\newcommand{\RegionMarkerTok}[1]{#1}
\newcommand{\SpecialCharTok}[1]{\textcolor[rgb]{0.81,0.36,0.00}{\textbf{#1}}}
\newcommand{\SpecialStringTok}[1]{\textcolor[rgb]{0.31,0.60,0.02}{#1}}
\newcommand{\StringTok}[1]{\textcolor[rgb]{0.31,0.60,0.02}{#1}}
\newcommand{\VariableTok}[1]{\textcolor[rgb]{0.00,0.00,0.00}{#1}}
\newcommand{\VerbatimStringTok}[1]{\textcolor[rgb]{0.31,0.60,0.02}{#1}}
\newcommand{\WarningTok}[1]{\textcolor[rgb]{0.56,0.35,0.01}{\textbf{\textit{#1}}}}
\usepackage{graphicx}
\makeatletter
\def\maxwidth{\ifdim\Gin@nat@width>\linewidth\linewidth\else\Gin@nat@width\fi}
\def\maxheight{\ifdim\Gin@nat@height>\textheight\textheight\else\Gin@nat@height\fi}
\makeatother
% Scale images if necessary, so that they will not overflow the page
% margins by default, and it is still possible to overwrite the defaults
% using explicit options in \includegraphics[width, height, ...]{}
\setkeys{Gin}{width=\maxwidth,height=\maxheight,keepaspectratio}
% Set default figure placement to htbp
\makeatletter
\def\fps@figure{htbp}
\makeatother
\setlength{\emergencystretch}{3em} % prevent overfull lines
\providecommand{\tightlist}{%
  \setlength{\itemsep}{0pt}\setlength{\parskip}{0pt}}
\setcounter{secnumdepth}{-\maxdimen} % remove section numbering
\ifLuaTeX
  \usepackage{selnolig}  % disable illegal ligatures
\fi
\IfFileExists{bookmark.sty}{\usepackage{bookmark}}{\usepackage{hyperref}}
\IfFileExists{xurl.sty}{\usepackage{xurl}}{} % add URL line breaks if available
\urlstyle{same}
\hypersetup{
  pdftitle={Lab2},
  pdfauthor={Irimie Fabio},
  hidelinks,
  pdfcreator={LaTeX via pandoc}}

\title{Lab2}
\usepackage{etoolbox}
\makeatletter
\providecommand{\subtitle}[1]{% add subtitle to \maketitle
  \apptocmd{\@title}{\par {\large #1 \par}}{}{}
}
\makeatother
\subtitle{Exercises}
\author{Irimie Fabio}
\date{}

\begin{document}
\maketitle

{
\setcounter{tocdepth}{2}
\tableofcontents
}
\hypertarget{exercise-1}{%
\section{Exercise 1}\label{exercise-1}}

\begin{itemize}
\item
  A:

  Create the Lab2 project. Use the same structure used for Lab1:

  \begin{itemize}
  \tightlist
  \item
    scripts,
  \item
    plots,
  \item
    data
  \end{itemize}
\item
  B:

  Install the palmerpenguins package, load the penguins dataset or,
  alternatively, download the .RData object from moodle and import it
  after placing it inside the data directory of the project (hint: use
  the load() function).
\item
  C:

  Compute the mean, the standard deviation, and the median for the
  numeric variables of the dataset.
\item
  D:

  Create a function called stat\_auto that simultaneously returns both
  the mean and the standard deviation of a given vector (hint: return an
  object of type list or simply a vector). Then try it on the same
  numeric variables in C. to check the results (hint: if you obtain NA
  maybe you forgot to remove NA terms in the vector).
\item
  E:

  Create a function called stat\_manual that simultaneously returns both
  the mean and the standard deviation of a given vector without using
  the mean() and the sd() functions (hint: you can use length(), sum(),
  and na.omit() functions). Then try it on the same numeric variables in
  C. to check the results.
\end{itemize}

\hypertarget{b}{%
\subsection{B}\label{b}}

\begin{Shaded}
\begin{Highlighting}[]
\FunctionTok{library}\NormalTok{(palmerpenguins)}
\FunctionTok{data}\NormalTok{(penguins)}
\end{Highlighting}
\end{Shaded}

\hypertarget{c}{%
\subsection{C}\label{c}}

\begin{Shaded}
\begin{Highlighting}[]
\CommentTok{\# Means}
\FunctionTok{cat}\NormalTok{(}\StringTok{"Means: }\SpecialCharTok{\textbackslash{}n}\StringTok{"}\NormalTok{)}
\DocumentationTok{\#\# Means:}
\FunctionTok{colMeans}\NormalTok{(penguins[, }\FunctionTok{c}\NormalTok{(}\DecValTok{3}\SpecialCharTok{:}\DecValTok{6}\NormalTok{, }\DecValTok{8}\NormalTok{)], }\AttributeTok{na.rm =} \ConstantTok{TRUE}\NormalTok{)}
\DocumentationTok{\#\#    bill\_length\_mm     bill\_depth\_mm flipper\_length\_mm       body\_mass\_g }
\DocumentationTok{\#\#          43.92193          17.15117         200.91520        4201.75439 }
\DocumentationTok{\#\#              year }
\DocumentationTok{\#\#        2008.02907}
\FunctionTok{cat}\NormalTok{(}\StringTok{"}\SpecialCharTok{\textbackslash{}n}\StringTok{"}\NormalTok{)}

\CommentTok{\# Medians}
\FunctionTok{cat}\NormalTok{(}\StringTok{"Medians: }\SpecialCharTok{\textbackslash{}n}\StringTok{"}\NormalTok{)}
\DocumentationTok{\#\# Medians:}
\FunctionTok{sapply}\NormalTok{(penguins[, }\FunctionTok{c}\NormalTok{(}\DecValTok{3}\SpecialCharTok{:}\DecValTok{6}\NormalTok{, }\DecValTok{8}\NormalTok{)], median, }\AttributeTok{na.rm =} \ConstantTok{TRUE}\NormalTok{)}
\DocumentationTok{\#\#    bill\_length\_mm     bill\_depth\_mm flipper\_length\_mm       body\_mass\_g }
\DocumentationTok{\#\#             44.45             17.30            197.00           4050.00 }
\DocumentationTok{\#\#              year }
\DocumentationTok{\#\#           2008.00}
\FunctionTok{cat}\NormalTok{(}\StringTok{"}\SpecialCharTok{\textbackslash{}n}\StringTok{"}\NormalTok{)}

\CommentTok{\# Standard deviations}
\FunctionTok{cat}\NormalTok{(}\StringTok{"Standard deviations: }\SpecialCharTok{\textbackslash{}n}\StringTok{"}\NormalTok{)}
\DocumentationTok{\#\# Standard deviations:}
\FunctionTok{sapply}\NormalTok{(penguins[, }\FunctionTok{c}\NormalTok{(}\DecValTok{3}\SpecialCharTok{:}\DecValTok{6}\NormalTok{, }\DecValTok{8}\NormalTok{)], sd, }\AttributeTok{na.rm =} \ConstantTok{TRUE}\NormalTok{)}
\DocumentationTok{\#\#    bill\_length\_mm     bill\_depth\_mm flipper\_length\_mm       body\_mass\_g }
\DocumentationTok{\#\#         5.4595837         1.9747932        14.0617137       801.9545357 }
\DocumentationTok{\#\#              year }
\DocumentationTok{\#\#         0.8183559}
\FunctionTok{cat}\NormalTok{(}\StringTok{"}\SpecialCharTok{\textbackslash{}n}\StringTok{"}\NormalTok{)}
\end{Highlighting}
\end{Shaded}

\hypertarget{d}{%
\subsection{D}\label{d}}

\begin{Shaded}
\begin{Highlighting}[]
\NormalTok{stat\_auto }\OtherTok{\textless{}{-}} \ControlFlowTok{function}\NormalTok{(vec, }\AttributeTok{na.rm =} \ConstantTok{FALSE}\NormalTok{) \{}
  \ControlFlowTok{if}\NormalTok{ (na.rm }\SpecialCharTok{==} \ConstantTok{TRUE}\NormalTok{) \{}
\NormalTok{    mean }\OtherTok{\textless{}{-}} \FunctionTok{mean}\NormalTok{(vec, }\AttributeTok{na.rm =} \ConstantTok{TRUE}\NormalTok{)}
\NormalTok{    sd }\OtherTok{\textless{}{-}} \FunctionTok{sd}\NormalTok{(vec, }\AttributeTok{na.rm =} \ConstantTok{TRUE}\NormalTok{)}
\NormalTok{  \} }\ControlFlowTok{else}\NormalTok{ \{}
\NormalTok{    mean }\OtherTok{\textless{}{-}} \FunctionTok{mean}\NormalTok{(vec)}
\NormalTok{    sd }\OtherTok{\textless{}{-}} \FunctionTok{sd}\NormalTok{(vec)}
\NormalTok{  \}}

  \FunctionTok{return}\NormalTok{(}\FunctionTok{list}\NormalTok{(}\StringTok{"mean"} \OtherTok{=}\NormalTok{ mean, }\StringTok{"sd"} \OtherTok{=}\NormalTok{ sd))}
\NormalTok{\}}

\FunctionTok{sapply}\NormalTok{(penguins[, }\FunctionTok{c}\NormalTok{(}\DecValTok{3}\SpecialCharTok{:}\DecValTok{6}\NormalTok{, }\DecValTok{8}\NormalTok{)], stat\_auto, }\AttributeTok{na.rm =} \ConstantTok{TRUE}\NormalTok{)}
\end{Highlighting}
\end{Shaded}

\begin{verbatim}
##      bill_length_mm bill_depth_mm flipper_length_mm body_mass_g
## mean 43.92193       17.15117      200.9152          4201.754   
## sd   5.459584       1.974793      14.06171          801.9545   
##      year     
## mean 2008.029 
## sd   0.8183559
\end{verbatim}

\hypertarget{e}{%
\subsection{E}\label{e}}

\end{document}
