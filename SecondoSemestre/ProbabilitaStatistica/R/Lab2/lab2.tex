% Options for packages loaded elsewhere
\PassOptionsToPackage{unicode}{hyperref}
\PassOptionsToPackage{hyphens}{url}
%
\documentclass[
]{article}
\usepackage{amsmath,amssymb}
\usepackage{iftex}
\ifPDFTeX
  \usepackage[T1]{fontenc}
  \usepackage[utf8]{inputenc}
  \usepackage{textcomp} % provide euro and other symbols
\else % if luatex or xetex
  \usepackage{unicode-math} % this also loads fontspec
  \defaultfontfeatures{Scale=MatchLowercase}
  \defaultfontfeatures[\rmfamily]{Ligatures=TeX,Scale=1}
\fi
\usepackage{lmodern}
\ifPDFTeX\else
  % xetex/luatex font selection
\fi
% Use upquote if available, for straight quotes in verbatim environments
\IfFileExists{upquote.sty}{\usepackage{upquote}}{}
\IfFileExists{microtype.sty}{% use microtype if available
  \usepackage[]{microtype}
  \UseMicrotypeSet[protrusion]{basicmath} % disable protrusion for tt fonts
}{}
\makeatletter
\@ifundefined{KOMAClassName}{% if non-KOMA class
  \IfFileExists{parskip.sty}{%
    \usepackage{parskip}
  }{% else
    \setlength{\parindent}{0pt}
    \setlength{\parskip}{6pt plus 2pt minus 1pt}}
}{% if KOMA class
  \KOMAoptions{parskip=half}}
\makeatother
\usepackage{xcolor}
\usepackage[margin=1in]{geometry}
\usepackage{color}
\usepackage{fancyvrb}
\newcommand{\VerbBar}{|}
\newcommand{\VERB}{\Verb[commandchars=\\\{\}]}
\DefineVerbatimEnvironment{Highlighting}{Verbatim}{commandchars=\\\{\}}
% Add ',fontsize=\small' for more characters per line
\usepackage{framed}
\definecolor{shadecolor}{RGB}{248,248,248}
\newenvironment{Shaded}{\begin{snugshade}}{\end{snugshade}}
\newcommand{\AlertTok}[1]{\textcolor[rgb]{0.94,0.16,0.16}{#1}}
\newcommand{\AnnotationTok}[1]{\textcolor[rgb]{0.56,0.35,0.01}{\textbf{\textit{#1}}}}
\newcommand{\AttributeTok}[1]{\textcolor[rgb]{0.13,0.29,0.53}{#1}}
\newcommand{\BaseNTok}[1]{\textcolor[rgb]{0.00,0.00,0.81}{#1}}
\newcommand{\BuiltInTok}[1]{#1}
\newcommand{\CharTok}[1]{\textcolor[rgb]{0.31,0.60,0.02}{#1}}
\newcommand{\CommentTok}[1]{\textcolor[rgb]{0.56,0.35,0.01}{\textit{#1}}}
\newcommand{\CommentVarTok}[1]{\textcolor[rgb]{0.56,0.35,0.01}{\textbf{\textit{#1}}}}
\newcommand{\ConstantTok}[1]{\textcolor[rgb]{0.56,0.35,0.01}{#1}}
\newcommand{\ControlFlowTok}[1]{\textcolor[rgb]{0.13,0.29,0.53}{\textbf{#1}}}
\newcommand{\DataTypeTok}[1]{\textcolor[rgb]{0.13,0.29,0.53}{#1}}
\newcommand{\DecValTok}[1]{\textcolor[rgb]{0.00,0.00,0.81}{#1}}
\newcommand{\DocumentationTok}[1]{\textcolor[rgb]{0.56,0.35,0.01}{\textbf{\textit{#1}}}}
\newcommand{\ErrorTok}[1]{\textcolor[rgb]{0.64,0.00,0.00}{\textbf{#1}}}
\newcommand{\ExtensionTok}[1]{#1}
\newcommand{\FloatTok}[1]{\textcolor[rgb]{0.00,0.00,0.81}{#1}}
\newcommand{\FunctionTok}[1]{\textcolor[rgb]{0.13,0.29,0.53}{\textbf{#1}}}
\newcommand{\ImportTok}[1]{#1}
\newcommand{\InformationTok}[1]{\textcolor[rgb]{0.56,0.35,0.01}{\textbf{\textit{#1}}}}
\newcommand{\KeywordTok}[1]{\textcolor[rgb]{0.13,0.29,0.53}{\textbf{#1}}}
\newcommand{\NormalTok}[1]{#1}
\newcommand{\OperatorTok}[1]{\textcolor[rgb]{0.81,0.36,0.00}{\textbf{#1}}}
\newcommand{\OtherTok}[1]{\textcolor[rgb]{0.56,0.35,0.01}{#1}}
\newcommand{\PreprocessorTok}[1]{\textcolor[rgb]{0.56,0.35,0.01}{\textit{#1}}}
\newcommand{\RegionMarkerTok}[1]{#1}
\newcommand{\SpecialCharTok}[1]{\textcolor[rgb]{0.81,0.36,0.00}{\textbf{#1}}}
\newcommand{\SpecialStringTok}[1]{\textcolor[rgb]{0.31,0.60,0.02}{#1}}
\newcommand{\StringTok}[1]{\textcolor[rgb]{0.31,0.60,0.02}{#1}}
\newcommand{\VariableTok}[1]{\textcolor[rgb]{0.00,0.00,0.00}{#1}}
\newcommand{\VerbatimStringTok}[1]{\textcolor[rgb]{0.31,0.60,0.02}{#1}}
\newcommand{\WarningTok}[1]{\textcolor[rgb]{0.56,0.35,0.01}{\textbf{\textit{#1}}}}
\usepackage{graphicx}
\makeatletter
\def\maxwidth{\ifdim\Gin@nat@width>\linewidth\linewidth\else\Gin@nat@width\fi}
\def\maxheight{\ifdim\Gin@nat@height>\textheight\textheight\else\Gin@nat@height\fi}
\makeatother
% Scale images if necessary, so that they will not overflow the page
% margins by default, and it is still possible to overwrite the defaults
% using explicit options in \includegraphics[width, height, ...]{}
\setkeys{Gin}{width=\maxwidth,height=\maxheight,keepaspectratio}
% Set default figure placement to htbp
\makeatletter
\def\fps@figure{htbp}
\makeatother
\setlength{\emergencystretch}{3em} % prevent overfull lines
\providecommand{\tightlist}{%
  \setlength{\itemsep}{0pt}\setlength{\parskip}{0pt}}
\setcounter{secnumdepth}{-\maxdimen} % remove section numbering
\ifLuaTeX
  \usepackage{selnolig}  % disable illegal ligatures
\fi
\IfFileExists{bookmark.sty}{\usepackage{bookmark}}{\usepackage{hyperref}}
\IfFileExists{xurl.sty}{\usepackage{xurl}}{} % add URL line breaks if available
\urlstyle{same}
\hypersetup{
  pdftitle={Lab1},
  pdfauthor={Irimie Fabio},
  hidelinks,
  pdfcreator={LaTeX via pandoc}}

\title{Lab1}
\usepackage{etoolbox}
\makeatletter
\providecommand{\subtitle}[1]{% add subtitle to \maketitle
  \apptocmd{\@title}{\par {\large #1 \par}}{}{}
}
\makeatother
\subtitle{Exercise 3}
\author{Irimie Fabio}
\date{}

\begin{document}
\maketitle

{
\setcounter{tocdepth}{2}
\tableofcontents
}
\hypertarget{statistica-descrittiva}{%
\section{Statistica descrittiva}\label{statistica-descrittiva}}

\hypertarget{dataset}{%
\subsection{Dataset}\label{dataset}}

Il dataset \emph{palmerpenguins} è un dataset che contiene i dati su 3
specie di pinguini: \emph{Adelie}, \emph{Gentoo} e \emph{Chinstrap}.

\hypertarget{installazione}{%
\subsubsection{Installazione}\label{installazione}}

Per installare il pacchetto \emph{palmerpenguins} è necessario eseguire
i seguenti comandi:

\begin{Shaded}
\begin{Highlighting}[]
\CommentTok{\# install.packages("palmerpenguins")}
\CommentTok{\# library(palmerpenguins)}
\end{Highlighting}
\end{Shaded}

\hypertarget{utilizzo}{%
\subsubsection{Utilizzo}\label{utilizzo}}

Per utilizzare il dataset \emph{palmerpenguins} è necessario caricare in
memoria i dati con il comando:

\begin{Shaded}
\begin{Highlighting}[]
\FunctionTok{data}\NormalTok{(penguins)}
\end{Highlighting}
\end{Shaded}

Per visualizzare i dati è possibile utilizzare i comandi:

\begin{Shaded}
\begin{Highlighting}[]
\NormalTok{penguins}
\DocumentationTok{\#\# \# A tibble: 344 x 8}
\DocumentationTok{\#\#    species island    bill\_length\_mm bill\_depth\_mm flipper\_length\_mm}
\DocumentationTok{\#\#    \textless{}fct\textgreater{}   \textless{}fct\textgreater{}              \textless{}dbl\textgreater{}         \textless{}dbl\textgreater{}             \textless{}int\textgreater{}}
\DocumentationTok{\#\#  1 Adelie  Torgersen           39.1          18.7               181}
\DocumentationTok{\#\#  2 Adelie  Torgersen           39.5          17.4               186}
\DocumentationTok{\#\#  3 Adelie  Torgersen           40.3          18                 195}
\DocumentationTok{\#\#  4 Adelie  Torgersen           NA            NA                  NA}
\DocumentationTok{\#\#  5 Adelie  Torgersen           36.7          19.3               193}
\DocumentationTok{\#\#  6 Adelie  Torgersen           39.3          20.6               190}
\DocumentationTok{\#\#  7 Adelie  Torgersen           38.9          17.8               181}
\DocumentationTok{\#\#  8 Adelie  Torgersen           39.2          19.6               195}
\DocumentationTok{\#\#  9 Adelie  Torgersen           34.1          18.1               193}
\DocumentationTok{\#\# 10 Adelie  Torgersen           42            20.2               190}
\DocumentationTok{\#\# \# i 334 more rows}
\DocumentationTok{\#\# \# i 3 more variables: body\_mass\_g \textless{}int\textgreater{}, sex \textless{}fct\textgreater{}, year \textless{}int\textgreater{}}

\FunctionTok{head}\NormalTok{(penguins)}
\DocumentationTok{\#\# \# A tibble: 6 x 8}
\DocumentationTok{\#\#   species island    bill\_length\_mm bill\_depth\_mm flipper\_length\_mm}
\DocumentationTok{\#\#   \textless{}fct\textgreater{}   \textless{}fct\textgreater{}              \textless{}dbl\textgreater{}         \textless{}dbl\textgreater{}             \textless{}int\textgreater{}}
\DocumentationTok{\#\# 1 Adelie  Torgersen           39.1          18.7               181}
\DocumentationTok{\#\# 2 Adelie  Torgersen           39.5          17.4               186}
\DocumentationTok{\#\# 3 Adelie  Torgersen           40.3          18                 195}
\DocumentationTok{\#\# 4 Adelie  Torgersen           NA            NA                  NA}
\DocumentationTok{\#\# 5 Adelie  Torgersen           36.7          19.3               193}
\DocumentationTok{\#\# 6 Adelie  Torgersen           39.3          20.6               190}
\DocumentationTok{\#\# \# i 3 more variables: body\_mass\_g \textless{}int\textgreater{}, sex \textless{}fct\textgreater{}, year \textless{}int\textgreater{}}

\FunctionTok{str}\NormalTok{(penguins)}
\DocumentationTok{\#\# tibble [344 x 8] (S3: tbl\_df/tbl/data.frame)}
\DocumentationTok{\#\#  $ species          : Factor w/ 3 levels "Adelie","Chinstrap",..: 1 1 1 1 1 1 1 1 1 1 ...}
\DocumentationTok{\#\#  $ island           : Factor w/ 3 levels "Biscoe","Dream",..: 3 3 3 3 3 3 3 3 3 3 ...}
\DocumentationTok{\#\#  $ bill\_length\_mm   : num [1:344] 39.1 39.5 40.3 NA 36.7 39.3 38.9 39.2 34.1 42 ...}
\DocumentationTok{\#\#  $ bill\_depth\_mm    : num [1:344] 18.7 17.4 18 NA 19.3 20.6 17.8 19.6 18.1 20.2 ...}
\DocumentationTok{\#\#  $ flipper\_length\_mm: int [1:344] 181 186 195 NA 193 190 181 195 193 190 ...}
\DocumentationTok{\#\#  $ body\_mass\_g      : int [1:344] 3750 3800 3250 NA 3450 3650 3625 4675 3475 4250 ...}
\DocumentationTok{\#\#  $ sex              : Factor w/ 2 levels "female","male": 2 1 1 NA 1 2 1 2 NA NA ...}
\DocumentationTok{\#\#  $ year             : int [1:344] 2007 2007 2007 2007 2007 2007 2007 2007 2007 2007 ...}

\FunctionTok{View}\NormalTok{(penguins)}
\end{Highlighting}
\end{Shaded}


\end{document}
