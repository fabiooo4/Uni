\documentclass[a4paper]{article}

\usepackage[utf8]{inputenc}
\usepackage[T1]{fontenc}
\usepackage{textcomp}
\usepackage[italian]{babel}
\usepackage{amsmath, amssymb}
\usepackage[makeroom]{cancel}
\usepackage{amsfonts}
\usepackage{mdframed}
\usepackage{xcolor}
\usepackage{float}
\usepackage{tikz}
\usepackage{pgfplots}
\pgfplotsset{width=10cm,compat=1.9}
\usetikzlibrary{pgfplots.fillbetween, pgfplots.statistics}
\pgfplotsset{compat=newest, ticks=none}
\usepackage{graphicx}
\graphicspath{{./figures/}}

\usepackage{import}
\usepackage{pdfpages}
\usepackage{transparent}
\usepackage{xcolor}

\usepackage{hyperref}
\hypersetup{
    colorlinks=false,
}

\usepackage{ntheorem}
\newtheorem{theorem}{Teorema}

% Useful definitions frame
\theoremstyle{break}
\theoremheaderfont{\bfseries}
\newmdtheoremenv[%
	linecolor=gray,leftmargin=0,%
	rightmargin=0,
	innertopmargin=8pt,%
	ntheorem]{define}{Definizioni utili}[section]

% Example frame
\theoremstyle{break}
\theoremheaderfont{\bfseries}
\newmdtheoremenv[%
	linecolor=gray,leftmargin=0,%
	rightmargin=0,
	innertopmargin=8pt,%
	ntheorem]{example}{Esempio}[section]

% Important definition frame
\theoremstyle{break}
\theoremheaderfont{\bfseries}
\newmdtheoremenv[%
	linecolor=gray,leftmargin=0,%
	rightmargin=0,
	backgroundcolor=gray!40,%
	innertopmargin=8pt,%
	ntheorem]{definition}{Definizione}[section]

% Exercise frame
\theoremstyle{break}
\theoremheaderfont{\bfseries}
\newmdtheoremenv[%
	linecolor=gray,leftmargin=0,%
	rightmargin=0,
	innertopmargin=8pt,%
	ntheorem]{exercise}{Esercizio}[section]


% figure support
\usepackage{import}
\usepackage{xifthen}
\pdfminorversion=7
\usepackage{pdfpages}
\usepackage{transparent}
\newcommand{\incfig}[1]{%
	\def\svgwidth{\columnwidth}
	\import{./figures/}{#1.pdf_tex}
}

\pdfsuppresswarningpagegroup=1

\begin{document}
\begin{titlepage}
	\begin{center}
		\vspace*{1cm}

		\Huge
		\textbf{Probabilità e Statistica\\Esercizi}

		\vspace{0.5cm}
		\LARGE
		UniVR - Dipartimento di Informatica

		\vspace{1.5cm}

		\textbf{Fabio Irimie}

		\vfill


		\vspace{0.8cm}


		2° Semestre 2023/2024

	\end{center}
\end{titlepage}


\tableofcontents
\pagebreak

\section{Probabilità elementari e probabilità condizionate}
\subsection{Esercizio 1}
Un corso è frequentato da 10 studenti: 6 maschi e 4 femmine. Viene effettuato un
esame ed i punteggi degli studenti sono tutti diversi. Si suppone ciascuna classifica
equiprobabile.
\begin{enumerate}
	\item[a.] Qual è la cardinalità dello spazio dei campioni costituito da tutte le
	      possibili classifiche? Qual è una possibile misura di probabilità associata?
	      \[
		      \Omega \subseteq \mathbb{R}
	      \]
	      Tutte le possibili classifiche equivalgono a tutti i modi in cui si possono
	      ordinare i punteggi degli studenti. Quindi, la cardinalità dello spazio degli
	      eventi è $10!$:
	      \[
		      card(\Omega) = 10!
	      \]
	      La probabilità associata è calcolata come il rapporto tra il numero di eventi
	      favorevoli e il numero di eventi possibili:
	      \[
		      P(\omega) = \frac{card(\omega)}{card(\Omega)} = \frac{1}{10!}
	      \]
	\item[b.] Qual è la probabilità che le quattro studentesse ottengano punteggi migliori?
	      \[
		      E = \text{"Studentesse ottengono punteggi migliori"}
	      \]
	      La probabilità che le studentesse ottengano punteggi migliori è data dal rapporto
	      tra il numero di eventi favorevoli e il numero di eventi possibili:
	      \[
		      P(E) = \frac{4!}{10!}
	      \]
\end{enumerate}

\subsection{Esercizio 2}
In uno stock di 100 prodotti, 20 sono difettosi.
\begin{enumerate}
	\item[a.] Dieci vengono scelti a caso, senza rimpiazzo.

	      \noindent Qual è la probabilità che esattamente la metà siano difettosi?
	\item[b.] Dieci vengono scelti a caso, con rimpiazzo.

	      \noindent Qual è la probabilità che esattamente la metà siano difettosi?
\end{enumerate}

\end{document}
