\documentclass[a4paper]{article}

\usepackage[utf8]{inputenc}
\usepackage[T1]{fontenc}
\usepackage{textcomp}
\usepackage[italian]{babel}
\usepackage{amsmath, amssymb}
\usepackage[makeroom]{cancel}
\usepackage{amsfonts}
\usepackage{mdframed}
\usepackage{xcolor}
\usepackage{listings}
\usepackage{float}
\usepackage{tikz}
\usepackage{pgfplots}
\pgfplotsset{width=10cm,compat=1.9}
\usetikzlibrary{pgfplots.fillbetween, pgfplots.statistics}
\pgfplotsset{compat=newest, ticks=none}
\usepackage{graphicx}
\graphicspath{{./figures/}}

\usepackage{import}
\usepackage{pdfpages}
\usepackage{transparent}
\usepackage{xcolor}

\usepackage{hyperref}
\hypersetup{
    colorlinks=false,
}

\usepackage{ntheorem}
\newtheorem{theorem}{Teorema}

% Code blocks
\definecolor{codegreen}{rgb}{0,0.6,0}
\definecolor{codegray}{rgb}{0.5,0.5,0.5}
\definecolor{codepurple}{rgb}{0.58,0,0.82}
\definecolor{backcolour}{rgb}{0.95,0.95,0.95}

\lstdefinestyle{mystyle}{
	backgroundcolor=\color{backcolour},
	commentstyle=\color{codegreen},
	keywordstyle=\color{magenta},
	numberstyle=\tiny\color{codegray},
	stringstyle=\color{codepurple},
	basicstyle=\ttfamily\footnotesize,
	breakatwhitespace=false,
	breaklines=true,
	captionpos=b,
	keepspaces=true,
	numbers=left,
	numbersep=5pt,
	showspaces=false,
	showstringspaces=false,
	showtabs=false,
	tabsize=2
}

\lstset{style=mystyle}

\usepackage{color}

\definecolor{dkgreen}{rgb}{0,0.6,0}
\definecolor{gray}{rgb}{0.5,0.5,0.5}
\definecolor{mauve}{rgb}{0.58,0,0.82}

\lstset{frame=tb,
	aboveskip=3mm,
	belowskip=3mm,
	showstringspaces=false,
	columns=flexible,
	basicstyle={\small\ttfamily},
	numbers=none,
	numberstyle=\tiny\color{gray},
	keywordstyle=\color{blue},
	commentstyle=\color{dkgreen},
	stringstyle=\color{mauve},
	breaklines=true,
	breakatwhitespace=true,
	tabsize=3
}

\usepackage{import}
\usepackage{pdfpages}
\usepackage{transparent}
\usepackage{xcolor}



% Useful definitions frame
\theoremstyle{break}
\theoremheaderfont{\bfseries}
\newmdtheoremenv[%
	linecolor=gray,leftmargin=0,%
	rightmargin=0,
	innertopmargin=8pt,%
	ntheorem]{define}{Definizioni utili}[section]

% Example frame
\theoremstyle{break}
\theoremheaderfont{\bfseries}
\newmdtheoremenv[%
	linecolor=gray,leftmargin=0,%
	rightmargin=0,
	innertopmargin=8pt,%
	ntheorem]{example}{Esempio}[section]

% Important definition frame
\theoremstyle{break}
\theoremheaderfont{\bfseries}
\newmdtheoremenv[%
	linecolor=gray,leftmargin=0,%
	rightmargin=0,
	backgroundcolor=gray!40,%
	innertopmargin=8pt,%
	ntheorem]{definition}{Definizione}[section]

% Exercise frame
\theoremstyle{break}
\theoremheaderfont{\bfseries}
\newmdtheoremenv[%
	linecolor=gray,leftmargin=0,%
	rightmargin=0,
	innertopmargin=8pt,%
	ntheorem]{exercise}{Esercizio}[section]


% figure support
\usepackage{import}
\usepackage{xifthen}
\pdfminorversion=7
\usepackage{pdfpages}
\usepackage{transparent}
\newcommand{\incfig}[1]{%
	\def\svgwidth{\columnwidth}
	\import{./figures/}{#1.pdf_tex}
}

\pdfsuppresswarningpagegroup=1

\begin{document}
\begin{titlepage}
	\begin{center}
		\vspace*{1cm}

		\Huge
		\textbf{Probabilità e Statistica\\Esercizi}

		\vspace{0.5cm}
		\LARGE
		UniVR - Dipartimento di Informatica

		\vspace{1.5cm}

		\textbf{Fabio Irimie}

		\vfill


		\vspace{0.8cm}


		2° Semestre 2023/2024

	\end{center}
\end{titlepage}


\tableofcontents
\pagebreak

\section{Probabilità elementari e probabilità condizionate}
\subsection{Esercizio 1}
Un corso è frequentato da 10 studenti: 6 maschi e 4 femmine. Viene effettuato un
esame ed i punteggi degli studenti sono tutti diversi. Si suppone ciascuna classifica
equiprobabile.
\begin{enumerate}
	\item Qual è la cardinalità dello spazio dei campioni costituito da tutte le
	      possibili classifiche? Qual è una possibile misura di probabilità associata?
	      \[
		      \Omega \subseteq \mathbb{R}
	      \]
	      Tutte le possibili classifiche equivalgono a tutti i modi in cui si possono
	      ordinare i punteggi degli studenti. Quindi, la cardinalità dello spazio degli
	      eventi è $10!$:
	      \[
		      card(\Omega) = 10!
	      \]
	      La probabilità associata è calcolata come il rapporto tra il numero di eventi
	      favorevoli e il numero di eventi possibili:
	      \[
		      P(\omega) = \frac{card(\omega)}{card(\Omega)} = \frac{1}{10!}
	      \]
	\item Qual è la probabilità che le quattro studentesse ottengano punteggi migliori?
	      \[
		      E = \text{"Studentesse ottengono punteggi migliori"}
	      \]
	      La probabilità che le studentesse ottengano punteggi migliori è data dal rapporto
	      tra il numero di eventi favorevoli e il numero di eventi possibili:
	      \[
		      P(E) = \frac{4!}{10!}
	      \]
\end{enumerate}

\subsection{Esercizio 2}
In uno stock di 100 prodotti, 20 sono difettosi.
\begin{enumerate}
	\item Dieci vengono scelti a caso, senza rimpiazzo.

	      \noindent Qual è la probabilità che esattamente la metà siano difettosi?

	      \vspace{1em}
	      Per calcolare la probabilità che esattamente la metà dei prodotti siano difettosi
	      si deve fare il rapporto tra la moltiplicazione del numero di modi in cui si possono
	      scegliere 5 prodotti difettosi e 5 prodotti non difettosi e il numero di modi in
	      cui si possono scegliere 10 prodotti:
	      \[
		      P(E) = \frac{\binom{20}{5} \cdot \binom{80}{5}}{\binom{100}{10}} =
		      \frac{\frac{20!}{5! \cdot 15!}\cdot \frac{80!}{5! \cdot 75!}}{\frac{100!}{10! \cdot 90!}} =
		      \frac{15504 \cdot 24040016}{1.731030946 \cdot 10^{13}} \approx 0.214 = 21.4\%
	      \]
	\item Dieci vengono scelti a caso, con rimpiazzo.

	      \noindent Qual è la probabilità che esattamente la metà siano difettosi?
	      \vspace{1em}
	      Si può considerare una variabile di Bernoulli:
	      \[
		      X_i = \begin{cases}
			      1 & \text{se il prodotto è difettoso } p = \frac{20}{100} = \frac{1}{5} \\
			      0 & \text{altrimenti } p = \frac{80}{100} = \frac{4}{5}
		      \end{cases}
	      \]
	      La probabilità che esattamente la metà dei prodotti siano difettosi è data dalla
	      distribuzione binomiale:
	      \[
		      P(\omega) = \binom{n}{k} p^k (1-p)^{n-k}
	      \]
	      \[
		      P(E) = \binom{10}{5} \left( \frac{1}{5} \right)^5 \left( \frac{4}{5} \right)^5 =
		      \frac{10!}{5! \cdot 5!} \left( \frac{1}{5} \right)^5 \left( \frac{4}{5} \right)^5 \approx
		      0.0264 = 2.64\%
	      \]
\end{enumerate}

\subsection{Esercizio 3}
Consideriamo l’estrazione di una carta da una mazzo di 40 carte napoletane (4 semi; per
ciascun seme 10 carte dall’asso al 7 più tre figure).
\begin{enumerate}
	\item Qual'è la probabilità di estrarre un asso?

	      \vspace{1em}
	      \noindent Il numero di semi in un mazzo di carte napoletane è 4. Quindi ci sono 4
	      assi, di conseguenza la probabilità di trovare un asso è:
	      \[
		      P(A) = \frac{4}{40} = \frac{1}{10} = 0.1 = 10\%
	      \]
\end{enumerate}
Estraiamo ora 2 carte.
\begin{enumerate}
	\item[2.] Qual'è la probabilità di estrarre un asso e un re, rispettivamente?

	      \noindent Calcolare le probabilità nel caso di:
	      \begin{itemize}
		      \item Estrazione con reinserimento

		            \vspace{1em}
		            \noindent La probabilità di estrarre un asso e un re con reinserimento è data dal
		            prodotto delle probabilità di estrarre un asso e un re:
		            \[
			            P(A) = \frac{4}{40} = \frac{1}{10} = 0.1 = 10\% \quad \text{Estrarre un asso}
		            \]
		            \[
			            P(R) = \frac{4}{40} = \frac{1}{10} = 0.1 = 10\% \quad \text{Estrarre un re}
		            \]
		            \[
			            P(A \cap R) = P(A) \cdot P(R) = \frac{4}{40} \cdot \frac{4}{40} = \frac{1}{10} \cdot \frac{1}{10} = \frac{1}{100} = 0.01 = 1\%
		            \]
		      \item Estrazione senza reinserimento

		            \vspace{1em}
		            La probabilità di estrarre un re dopo aver estratto un asso è data dalla
		            probabilità condizionata:
		            \[
			            P(R\,|\,A) = \frac{P(R \cap A)}{P(A)} = \frac{\frac{1}{10} \cdot \frac{1}{10}}{\frac{1}{10}} = \frac{1}{10} = 0.1 = 10\%
		            \]
	      \end{itemize}
\end{enumerate}

\subsection{Esercizio 4}
Una popolazione si compone per un 40 percento di fumatori (F) e per il restante 60 per
cento di non fumatori (N). Si sa che il 25 per cento dei fumatori e il 7 per cento dei
non fumatori ha una malattia respiratoria cronica (M).
\begin{enumerate}
	\item Calcolare la probabilità che un individuo scelto a caso sia effetto dalla
	      malattia respiratoria.

	      \vspace{1em}
	      \noindent Un individuo scelto a caso ha una probabilità di essere fumatore malato:
	      \[
		      P(M\,|\,F) = 0.25
	      \]
	      e una probabilità di essere non fumatore malato:
	      \[
		      P(M\,|\,N) = 0.07
	      \]
	      La probabilità che un individuo scelto a caso sia affetto dalla malattia respiratoria
	      è data dalla probabilità totale:
	      \[
		      P(M) = P(M\,|\,F) \cdot P(F) + P(M\,|\,N) \cdot P(N)
	      \]
	      \[
		      P(M) = 0.25 \cdot 0.4 + 0.07 \cdot 0.6 = 0.1 + 0.042 = 0.142 = 14.2\%
	      \]
	\item Se l’individuo scelto è affetto dalla malattia, calcolare la probabilità che
	      sia un fumatore.

	      \vspace{1em}
	      \noindent La probabilità che un individuo affetto dalla malattia sia un fumatore è
	      data dalla probabilità condizionata:
	      \[
		      P(F\,|\,M) = \frac{P(M\,|\,F) \cdot P(F)}{P(M)} = \frac{0.25 \cdot 0.4}{0.142}
		      = \frac{0.1}{0.142} = 0.704 = 70.4\%
	      \]
\end{enumerate}

\subsection{Esercizio 5}
Una particolare analisi del sangue è efficace al 99\% nell’individuare una determinata
malattia quando essa è presente. Si possono però verificare dei falsi positivi
(ovvero una persona sana che si sottopone al test ha una probabilità pari a 0.01 di
risultare erroneamente positiva al test). Se l’incidenza della malattia è del 0.5\%,
calcolare la probabilità che un individuo risultato positivo al test sia effettivamente
malato.

\vspace{1em}
\[
	T = \text{Test positivo}
\]
\[
	M = \text{Persona malata} \quad \overline{M} = \text{Persona sana}
\]
La probabilità che il test sia efficace è:
\[
	P(T\,|\,M) = 0.99
\]
La probabilità che il test non sia efficace è:
\[
	P(T\,|\,\overline{M}) = 0.01
\]
La probabilità di essere malato è:
\[
	P(M) = 0.005 = 0.5\%
\]
La probabilità che il test sia positivo è data dalla probabilità totale:
\[
	P(T) = P(T\,|\,M) \cdot P(M) + P(T\,|\,\overline{M}) \cdot P(\overline{M})
\]
\[
	P(T) = 0.99 \cdot 0.005 + 0.01 \cdot 0.995 = 0.00495 + 0.00995 = 0.0149 = 1.49\%
\]
La probabilità che un individuo risultato positivo al test sia effettivamente malato è
data dalla probabilità condizionata:
\[
	P(T\,|\,P) = \frac{P(T\,|\,M) \cdot P(M)}{P(T)}
\]
\[
	P(T\,|\,P) = \frac{0.99 \cdot 0.005}{0.0149} = \frac{0.00495}{0.0149} \approx
	0.332 = 33.2\%
\]

\subsection{Esercizio 6}
In un vivaio si vendono dei sacchetti con 30 bulbi di dalie. I sacchetti sono
di due tipi \( S_1 \)  e \( S_2 \) . Tre quarti di tali sacchetti sono di tipo \( S_1 \) e contengono
10 bulbi di dalie rosse (R) e 20 bulbi di dalie gialle (G), mentre il restante
quarto dei sacchetti è di tipo \( S_2 \) e contiene 5 bulbi di dalie rosse e 25 bulbi di
dalie gialle.

\noindent Mario compra un sacchetto a caso e pianta un bulbo e vede di
che colore è.

\vspace{1em}
\noindent I sacchetti \( S_1 \)  sono:
\[
	P(S_1) = \frac{3}{4} = 0.75 = 75\%
\]
\[
	P(R\,|\,S_1) = \frac{10}{30} = \frac{1}{3} \approx 0.333 = 33.3\%
\]
\[
	P(G\,|\,S_1) = \frac{20}{30} = \frac{2}{3} \approx 0.666 = 66.6\%
\]
I sacchetti \( S_2 \)  sono:
\[
	P(S_2) = \frac{1}{4} = 0.25 = 25\%
\]
\[
	P(R\,|\,S_2) = \frac{5}{30} = \frac{1}{6} \approx 0.166 = 16.6\%
\]
\[
	P(G\,|\,S_2) = \frac{25}{30} = \frac{5}{6} \approx 0.833 = 83.3\%
\]

\begin{itemize}
	\item Determinare la probabilità che il bulbo produca una dalia rossa.

	      \vspace{1em}
	      La probabilità che il bulbo produca una dalia rossa è data dalla probabilità totale:
	      \[
		      P(R) = P(R\,|\,S_1) \cdot P(S_1) + P(R\,|\,S_2) \cdot P(S_2)
	      \]
	      \[
		      P(R) = \frac{1}{3} \cdot \frac{3}{4} + \frac{1}{6} \cdot \frac{1}{4} = \frac{1}{4} + \frac{1}{24} = \frac{7}{24} \approx 0.291 = 29.1\%
	      \]
	\item Se la dalia nata è rossa, qual è la probabilità che provenga da un sacchetto di
	      tipo \( S_1 \)?

	      \vspace{1em}
	      La probabilità che una dalia rossa provenga da \( S_1 \) è data dalla probabilità
	      condizionata:
	      \[
		      P(S_1\,|\,R) = \frac{P(R\,|\,S_1) \cdot P(S_1)}{P(R)}
	      \]
	      \[
		      P(S_1\,|\,R) = \frac{\frac{1}{3} \cdot \frac{3}{4}}{\frac{7}{24}} = \frac{1}{4} \cdot \frac{24}{7} = \frac{6}{7} \approx 0.857 = 85.7\%
	      \]
\end{itemize}

\subsection{Esercizio 7}
In un test clinico, un individuo viene sottoposto ad un esame di laboratorio, per
stabilire se ha o non ha una data malattia. Il test può avere esito positivo o negativo.
C’è però sempre una possibilità di errore: può darsi che alcuni degli individui
risultati positivi siano in realtà sani (falsi positivi), e che qualcuno degli individui
risultati negativi siano in realtà malati (falsi negativi). Prima di applicare su larga
scala un test nei laboratori, è quindi indispensabile valutarne la bontà, sottoponendo
al test un campione di persone che sappiamo già se sono sane o malate. Uno dei parametri
che definiscono la qualità diagnostica del test è la Sensibilità = 1 - P(falsi negativi).

\noindent In Italia c’è un malato di HIV ogni 40.000 persone. Un paziente si sottopone
ad un test con una procedura che fornisce statisticamente lo 0,7\% di falsi negativi e lo
0,01\% di falsi positivi. Calcolare la probabilità a posteriori di essere ammalato, a
test effettuato con esito positivo. Come cambia questa probabilità se però paziente e
medico si convincono che, in base ai sintomi ed alle circostanze del possibile contagio,
la probabilità a priori sia ad esempio 10 volte più alta della media nazionale?

\vspace{1em}
\[
	M = \text{Persona malata} \quad \overline{M} = \text{Persona sana}
\]
\[
	N = \text{Test negativo} \quad T = \text{Test positivo}
\]
\noindent La probabilità di avere l'HIV in Italia è:
\[
	P(M) = \frac{1}{40000} = 0.000025 = 0.0025\%
\]
La probabilità di non averlo è:
\[
	P(\overline{M}) = 1 - P(M) = 1 - 0.000025 = 0.999975 = 99.9975\%
\]
La probabilità che il test sia falso negativo è:
\[
	P(N\,|\,M) = 0.007 = 0.7\%
\]
La probabilità che il test sia falso positivo è:
\[
	P(T\,|\,\overline{M}) = 0.0001 = 0.01\%
\]
La probabilità a posteriori di essere ammalato a test effettuato con esito positivo è
data dalla probabilità condizionata:
\[
	P(M\,|\,T) = \frac{P(T\,|\,M) \cdot P(M)}{P(T)}
\]
La probabilità che il test risulti positivo è data dalla probabilità totale:
\[
	P(T) = P(T\,|\,M) \cdot P(M) + P(T\,|\,\overline{M}) \cdot P(\overline{M})
\]
\[
	P(T\,|\,M) = 1 - P(N\,|\,M) = 1 - 0.007 = 0.993 \quad \text{Sensibilità}
\]
\[
	P(T) = 0.993 \cdot 0.000025 + 0.0001 \cdot 0.999975 \approx 0.000124822 = 0.0125\%
\]

\vspace{1em}
\noindent Se la probabilità a priori è 10 volte più alta della media nazionale:
\[
	P(M) = 10 \cdot \frac{1}{40000} = 0.00025 = 0.025\%
\]
Le nuove probabilità che il test sia positivo è:
\[
	P(T) = 0.993 \cdot 0.00025 + 0.0001 \cdot 0.99975 \approx 0.000249822 = 0.0249\%
\]

\subsection{Esercizio 8}
Si considerano due dadi apparentemente uguali, di cui uno equo (ossia fornisce un numero
a caso con equiprobabilità) mentre l’altro fornisce un numero pari con probabilità doppia
rispetto a quella dei numeri dispari.

\noindent Viene scelto a caso un dado e lo si lancia.
\begin{enumerate}
	\item Calcolare la probabilità che esca un numero pari
	      \[
		      Par = \text{Esce un numero pari}
	      \]
	      \[
		      Dis = \text{Esce un numero dispari}
	      \]
	      \[
		      P(D_1) = \frac{1}{2} = 0.5 = 50\% \quad \text{Dado equo}
	      \]
	      \[
		      P(D_2) = \frac{1}{2} = 0.5 = 50\% \quad \text{Dado non equo}
	      \]
	      \[
		      P(Par\,|\,D_1) = \frac{1}{2} = 0.5 = 50\%
	      \]
	      \[
		      P(Dis\,|\,D_1) = \frac{1}{2} = 0.5 = 50\%
	      \]
	      \[
		      P(Par\,|\,D_2) = \frac{6}{9} = \frac{2}{3} \approx 0.666 = 66.6\%
	      \]
	      La probabilità che esca un numero pari è data dalla probabilità totale:
	      \[
		      P(Par) = P(Par\,|\,D_1) \cdot P(D_1) + P(Par\,|\,D_2) \cdot P(D_2)
	      \]
	      \[
		      P(Par) = \frac{1}{2} \cdot \frac{1}{2} + \frac{1}{2} \cdot \frac{2}{3} = \frac{1}{4} + \frac{1}{3} = \frac{3}{12} + \frac{4}{12} = \frac{7}{12} \approx 0.583 = 58.3\%
	      \]
	\item Sapendo che è uscito un numero dispari, quanto vale la probabilità che il dado
	      lanciato sia quello equo?

	      \vspace{1em}
	      \noindent La probabilità che il dado lanciato sia quello equo è data dalla probabilità
	      condizionata:
	      \[
		      P(D_1\,|\,Dis) = \frac{P(Dis\,|\,D_1)\cdot P(D_1)}{P(Dis)}
	      \]
	      \[
		      P(D_1\,|\,Dis) = \frac{\frac{1}{2}*\frac{1}{2}}{1 - 0.583} = \frac{\frac{1}{4}}{0.417} \approx 0.6 = 60\%
	      \]
\end{enumerate}

\subsection{Esercizio 9}
Consideriamo una popolazione di \( N = 400 \) piantine in cui si presentano i seguenti
genotipi AA, Aa, aa. Supponiamo che una piantina venga estratta con uguale probabilità.
Supponiamo inoltre che vi sono 196 piantine del genotipo AA, 168 del genotipo Aa, e 36
del genotipo aa.
\[
	N = 400
\]
\[
	AA = 196
\]
\[
	Aa = 168
\]
\[
	aa = 36
\]
\begin{enumerate}
	\item Qual è la distribuzione dei genotipi, cioè se estraggo a caso una piantina qual
	      è la probabilità di ciascun genotipo?
	      \[
		      P(AA) = \frac{196}{400} = 0.49 = 49\%
	      \]
	      \[
		      P(Aa) = \frac{168}{400} = 0.42 = 42\%
	      \]
	      \[
		      P(aa) = \frac{36}{400} = 0.09 = 9\%
	      \]
\end{enumerate}
Supponiamo che l’allele a sia un letale recessivo, che però si esprime solo nella pianta
allo stadio adulto.
\begin{enumerate}
	\item[2.] La distribuzione precedente è ancora valida per la fase adulta (sì, no
	      e perchè)?

	      \vspace{1em}
	      \noindent La distribuzione precedente non è valida per la fase adulta. Infatti,
	      le piante con genotipo aa non sopravvivono allo stadio adulto, quindi la probabilità
	      di trovare una pianta adulta con genotipo aa è nulla. Di conseguenza le probabilità
	      in età adulta sono:
	      \[
		      N = 400 - 36 = 364
	      \]
	      \[
		      P(AA) = \frac{196}{364} \approx 0.538 = 53.8\%
	      \]
	      \[
		      P(Aa) = \frac{168}{364} \approx 0.462 = 46.2\%
	      \]
	      \[
		      P(aa) = 0
	      \]
	\item[3.] Qual è la probabilità di trovare una pianta in vita dopo un certo periodo di
	      tempo, quindi adulta?

	      \vspace{1em}
	      \noindent La probabilità di trovare una pianta in vita dopo un certo periodo di tempo
	      è data dalla somma delle probabilità di trovare una pianta adulta con genotipo AA e
	      Aa:
	      \[
		      P(AA) + P(Aa) = 0.49 + 0.42 = 0.91 = 91\%
	      \]
	\item[4.] Qual è la probabilità di estrarre una pianta adulta del genotipo AA? E del
	      tipo Aa?
	      \[
		      P(AA) = 0.538 = 53.8\%
	      \]
	      \[
		      P(Aa) = 0.462 = 46.2\%
	      \]
\end{enumerate}

\subsection{Esercizio 10}
Consideriamo la tabella che riporta i risultati di uno studio in cui sono stati osservati il numero di
querce e lecci infestati da un certo insetto
\begin{table}[H]
	\centering
	\begin{tabular}{c|ccc}
		        & Infestato & Non infestato & Totali \\
		\hline
		Quercia & 40        & 30            & 70     \\
		Leccio  & 52        & 10            & 62     \\
		\hline
		Totali  & 92        & 40            & 132
	\end{tabular}
\end{table}
\begin{enumerate}
	\item Calcolare la probabilità che un albero sia infestato.
	      \[
		      P(I) = \frac{92}{132} \approx 0.697 = 69.7\%
	      \]
	\item Calcolare la probabilità che una quercia sia infestata.
	      \[
		      P(Q_I) = \frac{40}{70} \approx 0.571 = 57.1\%
	      \]
	\item Calcolare la probabilità che un leccio sia infestato.
	      \[
		      P(L_I) = \frac{52}{62} \approx 0.839 = 83.9\%
	      \]
	\item Quale definizione di probabilità si è utilizzata nei punti precedenti?

	      \vspace{1em}
	      La definizione utilizzata nei punti precedenti è: La probabilità è calcolata come
	      il rapporto tra i casi favorevoli e i casi possibili.
\end{enumerate}

\subsection{Esercizio 11}
Tre scrivanie tra loro indistinguibili contengono ciascuna due cassetti. La prima
contiene una moneta d'oro in ciascun cassetto. La seconda una monta d'argento in un
cassetto ed una moneta d'oro nell'altro, la terza due monte d'argento. Si apre un
cassetto e si trova una moneta d'oro. Qual è la probabilità che anche nell'altro cassetto
della stessa scrivania ci sia una moneta d'oro?

\vspace{1em}
\noindent Avendo trovato una moneta d'oro, le uniche possibilità sono che la moneta sia stata
trovata nella scrivania 1 o nella 2. La probabilità che la seconda moneta sia d'oro riduce
le possibilità alla scrivania 1, quindi:
\[
	O = \text{Moneta d'oro}
\]
\[
	P(S_1) = \frac{1}{3} = 0.333 = 33.3\%
\]
\[
	P(O\,|\,S_1) = 1
\]
\[
	P(O) = \left( 1 + \frac{1}{2} + 0 \right) \cdot \frac{1}{3} = \frac{1}{2} = 0.5 = 50\%
\]
La probabilità che la seconda moneta sia d'oro è data dalla probabilità condizionata:
\[
	P(S_1\,|\,O) = \frac{P(O\,|\,S_1) \cdot P(S_1)}{P(O)}
\]
\[
	P(S_1\,|\,O) = \frac{1 \cdot \frac{1}{3}}{\frac{1}{2}} = \frac{2}{3} = 0.666 = 66.6\%
\]

\subsection{Esercizio 12}
Da un'indagine nelle scuole, risulta che la percentuale degli alunni che portano gli
occhiali è il 10\% nelle scuole elementari, il 25\% nelle scuole medie e il 40\% nelle
superiori.
\[
	O = \text{Porta gli occhiali}
\]
\[
	E = \text{Frequenta le elementari}
\]
\[
	M = \text{Frequenta le medie}
\]
\[
	S = \text{Frequenta le superiori}
\]

\[
	P(O\,|\,E) = 0.1 = 10\%
\]
\[
	P(O\,|\,M) = 0.25 = 25\%
\]
\[
	P(O\,|\,S) = 0.4 = 40\%
\]
\begin{enumerate}
	\item Calcolare la probabilità che scegliendo a caso 3 studenti, uno per fascia, almeno
	      uno porti gli occhiali.

	      \vspace{1em}
	      La probabilità che almeno uno porti gli occhiali è data dal complementare dell'evento
	      che nessuno porti gli occhiali:
	      \[
		      P(O \ge 1) = 1 - P(O = 0)
	      \]
	      \[
		      P(O \ge 1) = 1 - \left( P(\overline{O}\,|\,E) \cdot P(\overline{O}\,|\,M) \cdot P(\overline{O}\,|\,S) \right) =
	      \]
	      \[
		      = 1 - \left( 0.9 \cdot 0.75 \cdot 0.6 \right) = 1 - 0.405 = 0.595 = 59.5\%
	      \]
	\item Scegliendo uno studente a caso fra tutti (e supponendo che la scelta di ogni
	      fascia sia equiprobabile) calcolare la probabilità che questo abbia gli occhiali.

	      \vspace{1em}
	      La probabilità che uno studente scelto a caso è data dalla probabilità totale:
	      \[
		      P(O) = P(O\,|\,E) \cdot P(E) + P(O\,|\,M) \cdot P(M) + P(O\,|\,S) \cdot P(S)
	      \]
	      \[
		      P(O) = 0.1 \cdot \frac{1}{3} + 0.25 \cdot \frac{1}{3} + 0.4 \cdot \frac{1}{3} = \frac{1}{4} = 0.25 = 25\%
	      \]
	\item Sapendo che lo studente porti gli occhiali, calcolare la probabilità che
	      frequenti le elementari.

	      \vspace{1em}
	      \noindent La probabilità che uno studente frequenti le elementari sapendo che porta
	      gli occhiali è data dalla probabilità condizionata:
	      \[
		      P(E\,|\,O) = \frac{P(O\,|\,E) \cdot P(E)}{P(O)}
	      \]
	      \[
		      P(E\,|\,O) = \frac{0.1 \cdot \frac{1}{3}}{\frac{1}{4}} \approx 0.2667 = 26.67\%
	      \]
\end{enumerate}

\subsection{Esercizio 13}
È stata studiata la relazione tra forma fisica e malattie cardiovascolari in un gruppo di
impiegati delle ferrovie di sesso maschile, con questi risultati:
\begin{table}[H]
	\centering
	\begin{tabular}{p{1.3in}|p{1.3in}|p{1.3in}}
		Frequenza cardiaca\newline sotto esercizio\newline (battiti/minuto) &
		Percentuale dei lavoratori                                          &
		Mortalità per malattie cardiovascolari in 20 anni                                                               \\
		\hline
		\centering{\( \le 105 \)}                                           & \centering{20\%} & \centering{9.2\%} \cr
		\centering{106-115}                                                 & \centering{30\%} & \centering{8.7\%} \cr
		\centering{116-127}                                                 & \centering{30\%} & \centering{11.6\%} \cr
		\centering{\( > 127 \) }                                            & \centering{20\%} & \centering{13.2\%}
	\end{tabular}
\end{table}
\noindent Supponiamo che un certo test sia positivo se la frequenza cardiaca sotto esercizio è
maggiore di 127 battiti al minuto e negativo altrimenti.
\begin{enumerate}
	\item Qual è la probabilità di mortalità per malattie cardiovascolari in 20 anni?

	      \vspace{1em}
	      \[
		      T = \text{Test positivo}
	      \]
	      \[
		      \overline{T} = \text{Test negativo}
	      \]
	      \[
		      M = \text{Mortalità per malattie cardiovascolari in 20 anni}
	      \]
	      \[
		      F_1 = \text{Frequenza cardiaca} \le 105
	      \]
	      \[
		      F_2 = \text{Frequenza cardiaca} \in [106, 115]
	      \]
	      \[
		      F_3 = \text{Frequenza cardiaca} \in [116, 127]
	      \]
	      \[
		      F_4 = \text{Frequenza cardiaca} > 127
	      \]
	      La probabilità di mortalità per malattie cardiovascolari in 20 anni è data dalla probabilità
	      totale:
	      \[
		      P(M) = P(M\,|\,F_1) \cdot P(F_1) + P(M\,|\,F_2) \cdot P(F_2) +
	      \]
	      \[
		      P(M\,|\,F_3) \cdot P(F_3) + P(M\,|\,F_4) \cdot P(F_4)
	      \]
	      \[
		      P(M) = 0.092 \cdot 0.2 + 0.087 \cdot 0.3 + 0.116 \cdot 0.3 + 0.132 \cdot 0.2 = 0.1056 = 10.56\%
	      \]
	\item Qual è la probabilità di avere avuto un test positivo tra gli uomini che sono
	      morti nel periodo di 20 anni?

	      \vspace{1em}
	      La probabilità di aver avuto un test positivo è data dalla probabilità condizionata:
	      \[
		      P(F_4\,|\,M) = \frac{P(M\,|\,F_4) \cdot P(F_4)}{P(M)}
	      \]
	      \[
		      P(F_4\,|\,M) = \frac{0.132 \cdot 0.2}{0.1056} \approx 0.25 = 25\%
	      \]
	\item Qual è la probabilità di avere avuto un test positivo tra gli uomini che sono
	      sopravvissuti nel periodo di 20 anni?

	      \vspace{1em}
	      La probabilità di avere avuto un test positivo tra gli uomini che sono sopravvissuti
	      è data dalla probabilità condizionata:
	      \[
		      P(F_4\,|\,\overline{M}) = \frac{P(\overline{M}\,|\,F_4) \cdot P(F_4)}{P(\overline{M})}
	      \]
	      \[
		      P(F_4\,|\, \overline{M}) = \frac{(1-0.132) \cdot 0.2}{1-0.1056} \approx 0.1941 = 19.41\%
	      \]
	\item Qual è la probabilità di morte tra gli uomini con un test negativo?

	      \vspace{1em}
	      La probabilità di morte tra gli uomini con un test negativo è data dalla probabilità
	      totale dei casi in cui il test è negativo:
	      \[
		      P(M\,|\,\overline{T}) = P(M\,|\,F_1) \cdot P(F_1) + P(M\,|\,F_2) \cdot P(F_2) +P(M\,|\,F_3) \cdot P(F_3)
	      \]
	      \[
		      P(M\,|\,\overline{T}) = 0.092 \cdot 0.2 + 0.087 \cdot 0.3 + 0.116 \cdot 0.3 = 0.0793 = 7.93\%
	      \]
\end{enumerate}

\section{Variabili aleatorie: Binomiale, Poisson, Normale, Uniforme}
\subsection{Esercizio 1}
Una persona arriva all'ascensore dell'edificio un minuto prima dell'orario di inizio del
suo lavoro. Il tempo di attesa dell'ascensore, misurato in minuti, è una variabile
aleatoria che ha una distribuzione uniforme in \( [0,3] \text{ (minuti)} \). Considerando
trascurabile il tempo impiegato dall'ascensore:
\begin{enumerate}
	\item Qual'è la probabilità che la persona non arrivi in ritardo al lavoro?

	      \vspace{1em}
	      La probabilità che la persona arrivi in ritardo è:
	      \[
		      P(X > 1) = \frac{[1,3]}{[0,3]} = \frac{2}{3}
	      \]
	      La probabilità che la persona non arrivi in ritardo è:
	      \[
		      P(X \le 1) = 1 - P(X > 1) = 1 - \frac{2}{3} = \frac{1}{3}
	      \]
	\item La probabilità precedente viene considerata bassa. Con quanto anticipo la persona
	      deve arrivare prima all'ascensore in modo tale che la probabilità di arrivare in
	      ritardo sia al massimo 0.05?

	      \vspace{1em}
	      La probabilità che la persona arrivi in ritardo è:
	      \[
		      P(X > x) = 0.05
	      \]
	      \[
		      P(X > x) = 1 - P(X \le x) = 1 - \frac{x}{3} = 0.05
	      \]
	      \[
		      x = (1 - 0.05) \cdot 3 = 2.85 \text{ minuti}
	      \]
\end{enumerate}

\subsection{Esercizio 2}
Nel gioco del tiro al bersaglio si vincono 10 punti se si colpisce il bersaglio entro
\( 2cm \) dal centro, 5 punti se si colpisce tra \( 2cm \)  e \( 3cm \) e 2 punti se si
colpisce tra \( 3cm \) e \( 5cm \). Oltre \( 5cm \) non si vince nulla.

\noindent Si supponga che la distanza del punto in cui si colpisce ed il centro del
bersaglio abbia distribuzione uniforme sull'intervallo \( (0,10) \).

\[
	10 \text{ punti} = (0,2]
\]
\[
	5 \text{ punti} = (2,3]
\]
\[
	2 \text{ punti} = (3, 5]
\]
\[
	0 \text{ punti} = (5,10)
\]

\begin{enumerate}
	\item Qual'è il numero atteso di punti per lancio?

	      \vspace{1em}
	      Il numero atteso di punti per lancio è la media dei punti pesati per la probabilità
	      di ottenere quel punteggio:
	      \[
		      \mathbb{E} = 10 \cdot \frac{2}{10} + 5 \cdot \frac{1}{10} + 2 \cdot \frac{2}{10} = 2 + 0.5 + 0.4 = 2.9
	      \]
	\item Qual'è la probabilità di avere 10 punti in un lancio?

	      \vspace{1em}
	      La probabilità di avere 10 punti in un lancio è:
	      \[
		      P(10) = \frac{2}{10} = \frac{1}{5} = 0.2 = 20\%
	      \]
	\item Assumendo i lanci indipendenti, qual'è la probabilità di ottenere 40 punti in 4
	      lanci?

	      \vspace{1em}
	      La probabilità di ottenere 40 punti in 4 lanci è data dalla probabilità di avere
	      10 punti in 1 lancio ripetuta 4 volte:
	      \[
		      P(40) = P(10)^4 = 0.2^4 = 0.0016
	      \]
\end{enumerate}

\subsection{Esercizio 3}
Arturo partecipa ad un gioco in cui la probabilità di vittoria è \( p = \frac{2}{3} \).
\begin{enumerate}
	\item Calcolare il numero minimo \( n \) di partite diverse che Arturo deve giocare,
	      se vuole avere una probabilità superiore a \( \frac{9}{10} \) di vincere almeno una
	      volta.

	      \vspace{1em}
	      La probabilità di perdere una partita è \( 1 - p = \frac{1}{3} \). La probabilità
	      di vincere almeno una partita è il complementare di una sconfitta ripetuta \( n \)
	      volte:
	      \[
		      P(X \ge 1) = 1- P(X = 0) = 1 - (1 - p)^n = \frac{9}{10}
	      \]
	      \[
		      (1 - p)^n = \frac{1}{10}
	      \]
	      \[
		      n = \log_{1-p} \frac{1}{10} = \log_{\frac{1}{3}} \frac{1}{10} = \log_{3}10 = 2.096
	      \]
	      Approssimando per eccesso all'intero più vicino, siccome non si possono giocare
	      frazioni di partite, il numero minimo di partite da giocare è 3.
	      \[
		      n = 3 \text{ partite}
	      \]
	\item Supponendo che giochi \( n = 5 \) partite diverse, determinare la probabilità che
	      in queste partite Arturo abbia ottenuto esattamente 3 vittorie.

	      \vspace{1em}
	      La probabilità che Arturo abbia ottenuto esattamente 3 vittorie in 5 partite è data
	      dalla distribuzione binomiale:
	      \[
		      P(X = 3) = \binom{5}{3} \cdot p^3 \cdot (1-p)^2 =  10 \cdot \left( \frac{2}{3} \right)^3
		      \cdot \left( \frac{1}{3} \right)^2 \approx 0.33 = 33\%
	      \]
	\item Supponendo che giochi \( n = 180 \) partite diverse, stimare la probabilità che
	      in queste partite Arturo abbia ottenuto un numero di vittorie compreso tra 108 e 121,
	      inclusi.

	      \vspace{1em}
	      La probabilità che Arturo abbia ottenuto un numero di vittorie compreso tra 108 e 121
	      in 180 partite è data dalla distribuzione binomiale:
	      \[
		      P(108 \le X \le 121) = \sum_{k=108}^{121} \binom{180}{k} \cdot p^k \cdot (1-p)^{180-k}
	      \]
	      Questa probabilità è stata calcolata attraverso il seguente codice in linguaggio R:
	      \begin{lstlisting}[language=R]

  sum(dbinom(108:121, 180, 2 / 3)) 
  # 0.5651441
    \end{lstlisting}
	      \[
		      P(108 \le X \le 121) = 0.565 = 56.5\%
	      \]
\end{enumerate}

\subsection{Esercizio 4}
Un esperimento consiste nel ripetere 10 volte l'estrazione (con reinserimento) di una
biglia da un urna contenente 50 biglie, 15 delle quali sono bianche. Sia \( Y_{10} \) la
variabile aleatoria che descrive il numero totale di biglie bianche estratte.
\[
	\text{Biglie totali} = 50
\]
\[
	\text{Biglie bianche} = 15
\]
\[
	p = \frac{15}{50} = 0.3
\]
\begin{enumerate}
	\item Determinare la probabilità che il numero di biglie bianche estratte siano almeno 2.

	      \vspace{1em}
	      La probabilità che il numero di biglie bianche estratte sia almeno 2 è data dalla
	      distribuzione binomiale:
	      \[
		      P(Y_{10} \ge 2) = 1 - P(Y_{10} < 2) = 1 - (P(Y_{10} = 0) + P(Y_{10} = 1))
	      \]
	      \[
		      P(Y_{10} \ge 2) = 1 - \left(\binom{10}{0} \cdot \left( 0.3 \right)^0
		      \cdot \left( 0.7 \right)^{10} + \binom{10}{1} \cdot
		      \left( 0.3 \right)^1 \cdot \left( 0.7 \right)^9\right) =
	      \]
	      \[
		      = 1 - \left( \frac{10!}{10!} \cdot 1 \cdot \left(0.7\right)^{10} +
		      \frac{10!}{9!} \cdot 0.3 \cdot \left( 0.7 \right)^9 \right)  =
	      \]
	      \[
		      = 1 - \left( \left( 0.7 \right)^{10} + 3 \cdot \left( 0.7 \right)^9 \right)
	      \]
	      \[
		      = 1 - \left( 0.149308346 \right) \approx 0.85 = 85\%
	      \]
	\item Come si calcola la probabilità che il numero di biglie estratte sia maggiore
	      strettamente di 3 e minore o uguale di 6? (\( P(3 < Y_{10} \le 6) \))

	      \vspace{1em}
	      La probabilità si calcola sommando le probabilità di avere 4 biglie
	      bianche, 5 biglie bianche e 6 biglie bianche:
	      \[
		      P(3 < Y_{10} \le 6) P(Y_{10} = 4) + P(Y_{10} = 5) + P(Y_{10} = 6)
	      \]
	      \[
		      P(3 < Y_{10} \le 6) = \left( \binom{10}{4} \cdot 0.3^4 \cdot 0.7^6 \right) +
	      \]
	      \[
		      \left( \binom{10}{5} \cdot 0.3^5 \cdot 0.7^5 \right) +
	      \]
	      \[
		      \left( \binom{10}{6} \cdot 0.3^6 \cdot 0.7^4 \right) =
	      \]
	      \[
		      = 0.200120949 + 0.102919345 + 0.036756909 \approx 0.34 = 34\%
	      \]
\end{enumerate}

\subsection{Esercizio 5}
Un calcolatore è collegato ad una rete che permette l'accesso ad un massimo di 20 persone.
Collegati a questa rete vi sono i terminali di 24 operatori, ognuno dei quali, ad un dato
istante, richiede con probabilità \( p = 0.6 \) di essere connesso al calcolatore centrale.

\noindent Qual'è la probabilità che ad un dato istante la rete sia satura (cioè che tutti
e 20 gli accessi siano utilizzati)?

\vspace{1em}
\noindent La probabilità che la rete sia satura è data dalla distribuzione binomiale:
\[
  P(X \ge 20) = \sum_{k=20}^{24} \binom{24}{k} \cdot p^k \cdot (1-p)^{24-k}
\] 
\[
	P(X = 20) = \binom{24}{20} \cdot 0.6^{20} \cdot 0.4^4 = \frac{24!}{20! \cdot 4!} \cdot
	0.000036562 \cdot 0.0256 =
\]
\[
	= 10626 \cdot 0.000036562 \cdot 0.0256 \approx 0.0099
\]
\[
  P(X = 21) = \binom{24}{21} \cdot 0.6^{21} \cdot 0.4^3 = \frac{24!}{21! \cdot 3!} \cdot
  0.000021937 \cdot 0.064 =
\] 
\[
  = 2024 \cdot 0.000021937 \cdot 0.064 \approx 0.0028
\] 
\[
  P(X = 22) = \binom{24}{22} \cdot 0.6^{22} \cdot 0.4^2 = \frac{24!}{22! \cdot 2!} \cdot
  0.000013162 \cdot 0.16 =
\] 
\[
  = 276 \cdot 0.000013162 \cdot 0.16 \approx 0.00058
\] 
\[
  P(X = 23) = \binom{24}{23} \cdot 0.6^{23} \cdot 0.4^1 = \frac{24!}{23! \cdot 1!} \cdot
  0.000007897 \cdot 0.4 =
\] 
\[
  = 24 \cdot 0.000007897 \cdot 0.4 \approx 0.00008
\] 
\[
  P(X = 24) = \binom{24}{24} \cdot 0.6^{24} \cdot 0.4^0 = 0.6^{24} \approx 0.000004
\] 
\[
  P(X \ge 20) = 0.0099 + 0.0028 + 0.00058 + 0.00008 + 0.000004 \approx 0.0135 = 1.35\%
\] 

\subsection{Esercizio 6}
In una linea produttiva ogni pezzo ha probabilità \( p = 0.03 \) di essere difettoso.
L'essere difettoso è indipendente da pezzo a pezzo.

\noindent Calcolare la probabilità che su 30 pezzi non più di 2 siano difettosi.

\vspace{1em}
\noindent La probabilità che su 30 pezzi non più di 2 siano difettosi è data dalla distribuzione
binomiale:
\[
	P(X \le 2) = P(X = 0) + P(X = 1) + P(X = 2)
\]
Per semplicità di calcolo uso una funzione di R:
\begin{lstlisting}[language=R]
  pbinom(2, 30, 0.03)
  # 0.9399309
\end{lstlisting}
La probabilità che su 30 pezzi non più di 2 siano difettosi è:
\[
	P(X \le 2) = 0.9399 = 93.99\%
\]

\subsection{Esercizio 7}
Una segreteria di un'azienda di statistica immette i dati in una banca dati con un ritmo
medio di 200 immissioni all'ora. Si suppone indipendenza tra le immissioni.

\noindent Quale modello probabilistico si può utilizzare per calcolare la probabilità che
ci siano esattamente 5 immissioni nel successivo minuto? Si può ipotizzare che sia una
Poisson? Determinare tale probabilità.

\vspace{1em}
\noindent Il modello probabilistico più adatto è la distribuzione di Poisson perchè
abbiamo degli eventi indipendenti che si verificano con una certa frequenza in un
certo intervallo.

\noindent La probabilità che ci siano esattamente 5 immissioni nel successivo minuto è
data dalla distribuzione di Poisson:
\[
	\mathcal{P}_{\lambda}(n) = \frac{\lambda^n}{n!}e^{-\lambda}
\]
\[
	\lambda = 200 \text{ immissioni/ora} = \frac{200}{60} \text{ immissioni/minuto} = 3.33 \text{ immissioni/minuto}
\]
\[
	n = 5 \text{ immissioni}
\]
\[
	\mathcal{P}(5) = \frac{3.33^5}{5!}e^{-3.33} \approx 0.122
\]

\subsection{Esercizio 8}
Un esperimento consiste nel ripetere 60 volte l'estrazione (con reinserimento) di una
biglia da un'urna contenente 50 biglie, 15 delle quali sono bianche. Sia \( Y_{60} \) la
variabile aleatoria che descrive il numero totale di biglie bianche estratte. Determinare
la probabilità che il numero di bilglie estratte sia maggiore strettamente di 10 e minore
o uguale di 30. (\( P(10 < Y_{60} \le 30) \))

\vspace{1em}
\noindent La probabilità che il numero di biglie estratte sia maggiore strettamente di 10 e
minore o uguale di 30 è data dalla distribuzione binomiale:
\[
	P(10 < Y_{60} \le 30) = \sum_{k=11}^{30} \binom{60}{k} \cdot p^k \cdot (1-p)^{60-k}
\]
\[
	P(10 < Y_{60} \le 30) = \sum_{k=11}^{30} \binom{60}{k} \cdot 0.3^k \cdot 0.7^{60-k}
\]
\begin{lstlisting}[language=R]
  sum(dbinom(11:30, 60, 15 / 50))
  # 0.9857581
\end{lstlisting}
\[
	P(10 < Y_{60} \le 30) \approx 0.9857 = 98.57\%
\]

\subsection{Esercizio 9}
In un corso di laurea a numero chiuso si ritiene che il numero di studenti non debba
essere superiore a 80. Inoltre è noto, però, che il 25\% degli studenti che superano
l'esame di ammissione cambiano idea e non confermano l'iscrizione.

\noindent Se 100 studenti superano l'esame di ammissione, qualè la probabilità che più di
80 decidano di iscriversi?

\vspace{1em}
\noindent La probabilità che uno studente che ha superato l'esame si iscriva è:
\[
	P(I) = 0.75
\]
La probabilità che più di 80 studenti si iscrivano è data dalla distribuzione binomiale:
\[
\]
\[
	P(X > 80) = \sum_{k=81}^{100} \binom{100}{k} \cdot 0.75^k \cdot 0.25^{100-k}
\]
\begin{lstlisting}[language=R]
  pbinom(81, 100, 0.75, lower.tail=FALSE)
  # 0.06301142
\end{lstlisting}
\[
	P(X > 80) \approx 0.063 = 6.3\%
\]

\subsection{Esercizio 10}
Da una rilevazione risulta che il numero di incidenti sul lavoro avvenuti in un'azienda
in un mese è una variabile distribuita secondo Poisson con valor medio \( \lambda = 1.5 \).
Calcolare:
\[
	\mathcal{P}_{\lambda}(n) = \frac{\lambda^n}{n!}e^{-\lambda}
\]
\begin{enumerate}
	\item La probabilità che in un mese non ci siano incidenti.

	      \vspace{1em}
	      La probabilità che in un mese non ci siano incidenti è data dalla distribuzione
	      di Poisson:
	      \[
		      \mathcal{P}(0) = \frac{1.5^0}{0!}e^{-1.5} = e^{-1.5} \approx 0.2231 = 22.31\%
	      \]
	\item La probabilità che in un mese ci siano più di 2 incidenti.

	      \vspace{1em}
	      La probabilità che in un mese ci siano più di 2 incidenti è data dal complementare
	      della probabilità che in un mese ci siano al massimo 2 incidenti:
	      \[
		      P(X > 2) = 1 - P(X \le 2) = 1 - \left( P(X = 0) + P(X = 1) + P(X = 2) \right)
	      \]
	      \[
		      P(X > 2) = 1 - \left( 0.2231 + \frac{1.5^1}{1!}e^{-1.5} +
		      \frac{1.5^2}{2!}e^{-1.5} \right) =
	      \]
	      \[
		      = 1 - \left( 0.2231 + 1.5e^{-1.5} + 1.125e^{-1.5} \right) \approx 0.1912 = 19.12\%
	      \]
\end{enumerate}

\subsection{Esercizio 11}
Da rilevazioni statistiche si è valutato che una persona su 100.000 è allergica ad un
tipo di polline. In una popolazione di 300.000 abitanti calcola la probabilità che:
\begin{enumerate}
	\item Nessuna persona sia allergica.

    \vspace{1em}
    \noindent Questo problema può essere risolto con la distribuzione di Poisson:
    \[
      \lambda = 300000 \cdot \frac{1}{100000} = 3
    \] 
    La probabilità che nessuno sia allergico è:
    \[
      \mathcal{P}(0) = \frac{3^0 \cdot e^{-3}}{0!} = e^{-3} \approx 0.0498 = 4.98\%
    \] 
	\item Al massimo 3 persone siano allergiche.

    \vspace{1em}
    \noindent La probabilità che al massimo 3 persone siano allergiche è data dalla 
    distribuzione di Poisson:
    \[
      \mathcal{P}(X \le 3) = \sum_{k=0}^{3} \frac{3^k \cdot e^{-3}}{k!}
    \] 
    Il risultato della sommatoria è dato dal seguente codice in R:
    \begin{lstlisting}[language=R]

  ppois(3, 3)
  # 0.6472319
    \end{lstlisting}
    \[
    \mathcal{P}(X \le 3) \approx 0.6472 = 64.72\%
    \] 
\end{enumerate}

\subsection{Esercizio 12}
La probabilità che un individuo contragga una malattia rara è dello 0.3 per mille individui.

\noindent Qual è la probabilità che in una città dove vivono 20000 persone vi siano meno 
di 4 persone che contraggono questa malattia?

\[
  p = \frac{0.3}{1000} = 0.0003
\] 
Questo problema è modellabile con la distribuzione di Poisson:
\[
  \lambda = 20000 \cdot 0.0003 = 6
\]
La probabilità che vi siano meno di 4 persone che contraggono la malattia è:
\[
  \mathcal{P}(X < 4) = \sum_{k=0}^{3} \frac{6^k \cdot e^{-6}}{k!}
\]
Il risultato della sommatoria è dato dal seguente codice in R:
\begin{lstlisting}[language=R]
  ppois(3, 6)
  # 0.1512039
\end{lstlisting}
\[
  \mathcal{P}(X < 4) \approx 0.1512 = 15.12\%
\]

\subsection{Esercizio 13}
Alla focacceria arrivano mediamente 30 clienti l'ora. Calcolare la probabilità che in
due minuti:
\begin{enumerate}
  \item Non arrivi nessuno.

    \vspace{1em}
    Questo problema è modellabile con la distribuzione di Poisson:
    \[
    \lambda = 30 \cdot \frac{2}{60} = 1
    \] 
    La probabilità che non arrivi nessuno è data da:
    \[
    \mathcal{P}(0) = \frac{1^0 \cdot e^{-1}}{0!} = e^{-1} \approx 0.3679 = 36.79\%
    \] 
  \item Arrivino almeno 3 clienti.

    \vspace{1em}
    La probabilità che arrivino almeno 3 clienti è data dal complementare della probabilità
    che arrivino al massimo 2 clienti:
    \[
    \mathcal{P}(X \ge 3) = 1 - \mathcal{P}(X < 3) = 1 - \sum_{k=0}^{2} \frac{1^k \cdot e^{-1}}{k!}
    \]
    Il risultato della sommatoria è dato dal seguente codice in R:
    \begin{lstlisting}[language=R]

  1 - ppois(2, 1)
  # 0.0803014
    \end{lstlisting}
    \[
    \mathcal{P}(X \ge 3) \approx 0.0803 = 8.03\%
    \]
\end{enumerate}

\subsection{Esercizio 14}
Ad un esame universitario, il voto \( X \) medio è stato \( \mu = \mathbb{E}[X] = 24 \),
con \( \sigma^2 = var(X) = 4^2 \). Supponendo i voti normalmente distribuiti, calcolare
la probabilità che uno studente abbia riportato:
\begin{enumerate}
  \item Un voto superiore a 27.

    \vspace{1em}
    Questo problema è modellabile con la distribuzione normale standard, bisogna quindi
    standardizzare il voto:
    \[
    Z = \frac{X - \mu}{\sigma} = \frac{27 - 24}{4} = 0.75
    \]
    Per trovare la probbailità di trovare un voto superiore a 27 bisogna cercare nelle
    tabelle dei valori della distribuzione normale standard e prendere il complementare:
    \[
    P(Z > 0.75) = 1 - 0.7734 = 0.2266 = 22.66\%
    \] 
  \item Un voto non inferiore a 22.

    \vspace{1em}
    Per trovare la probabilità di trovare un voto non inferiore a 22 bisogna standardizzare
    il voto:
    \[
    Z = \frac{X - \mu}{\sigma} = \frac{22 - 24}{4} = -0.5
    \]
    Successivamente bisogna cercare il valore di probabilità nella tabella e prendere il
    complementare:
    \[
    P(Z \ge -0.5) = 1 - 0.3085 = 0.6915 = 69.15\%
    \] 
\end{enumerate}

\subsection{Esercizio 15}
L'altezza X di un gruppo di 20.000 individui è distribuita normalmente con media
\( \mu = \mathbb{E}[X] = 170cm \) e con deviazione standard \( \sigma = \sqrt{var(X)} = 10cm  \).
\begin{enumerate}
  \item Qual è la probabilità che l'altezza sia compresa fra \( 155cm \) e \( 180cm \)?

    \vspace{1em}
    Questo problema è modellabile con la distribuzione normale standard, bisogna quindi
    standardizzare le altezze:
    \[
    Z_1 = \frac{155 - 170}{10} = -1.5
    \] 
    \[
    Z_2 = \frac{180 - 170}{10} = 1
    \] 
    La probabilità che l'altezza sia compresa fra \( 155cm \) e \( 180cm \) è data dalla
    differenza delle probabilità trovate dalle tabelle della distribuzione normale standard:
    \[
    P(-1.5 < Z < 1) = P(Z < 1) - P(Z < -1.5) 
    \] 
    \[
    P(-1.5 < Z < 1) = 0.8413 - 0.0668 = 0.7745 = 77.45\%
    \] 
  \item Quante persone ci si aspetta che siano alte almeno 2 metri?

    \vspace{1em}
    Standardizzando l'altezza:
    \[
    Z = \frac{200 - 170}{10} = 3
    \]
    La probabilità che una persona sia alta almeno 2 metri è:
    \[
    P(Z > 3) = 1 - 0.9987 = 0.0013 = 0.13\%
    \]
    Per sapere quante persone ci si aspetta che siano alte almeno 2 metri bisogna moltiplicare
    la probabilità per il numero di persone:
    \[
    0.0013 \cdot 20000 = 26
    \]
    Ci si aspetta, quindi, che 26 persone siano alte almeno 2 metri in un gruppo di 20.000
    individui
  \item Quante persone ci si aspetta che siano alte non più \( 160cm \)?

    \vspace{1em}
    Standardizzando l'altezza:
    \[
      Z = \frac{160 - 170}{10} = -1
    \] 
    La probabilità che una persona sia alta non più di \( 160cm \) è:
    \[
      P(Z < -1) = 0.1587
    \]
    Il numero di persone che ci si aspetta siano alte non più di \( 160cm \) è:
    \[
      0.1587 \cdot 20000 = 3174
    \]
\end{enumerate}

\end{document}
