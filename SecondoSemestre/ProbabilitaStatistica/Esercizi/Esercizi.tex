\documentclass[a4paper]{article}

\usepackage[utf8]{inputenc}
\usepackage[T1]{fontenc}
\usepackage{textcomp}
\usepackage[italian]{babel}
\usepackage{amsmath, amssymb}
\usepackage[makeroom]{cancel}
\usepackage{amsfonts}
\usepackage{mdframed}
\usepackage{xcolor}
\usepackage{float}
\usepackage{tikz}
\usepackage{pgfplots}
\pgfplotsset{width=10cm,compat=1.9}
\usetikzlibrary{pgfplots.fillbetween, pgfplots.statistics}
\pgfplotsset{compat=newest, ticks=none}
\usepackage{graphicx}
\graphicspath{{./figures/}}

\usepackage{import}
\usepackage{pdfpages}
\usepackage{transparent}
\usepackage{xcolor}

\usepackage{hyperref}
\hypersetup{
    colorlinks=false,
}

\usepackage{ntheorem}
\newtheorem{theorem}{Teorema}

% Useful definitions frame
\theoremstyle{break}
\theoremheaderfont{\bfseries}
\newmdtheoremenv[%
	linecolor=gray,leftmargin=0,%
	rightmargin=0,
	innertopmargin=8pt,%
	ntheorem]{define}{Definizioni utili}[section]

% Example frame
\theoremstyle{break}
\theoremheaderfont{\bfseries}
\newmdtheoremenv[%
	linecolor=gray,leftmargin=0,%
	rightmargin=0,
	innertopmargin=8pt,%
	ntheorem]{example}{Esempio}[section]

% Important definition frame
\theoremstyle{break}
\theoremheaderfont{\bfseries}
\newmdtheoremenv[%
	linecolor=gray,leftmargin=0,%
	rightmargin=0,
	backgroundcolor=gray!40,%
	innertopmargin=8pt,%
	ntheorem]{definition}{Definizione}[section]

% Exercise frame
\theoremstyle{break}
\theoremheaderfont{\bfseries}
\newmdtheoremenv[%
	linecolor=gray,leftmargin=0,%
	rightmargin=0,
	innertopmargin=8pt,%
	ntheorem]{exercise}{Esercizio}[section]


% figure support
\usepackage{import}
\usepackage{xifthen}
\pdfminorversion=7
\usepackage{pdfpages}
\usepackage{transparent}
\newcommand{\incfig}[1]{%
	\def\svgwidth{\columnwidth}
	\import{./figures/}{#1.pdf_tex}
}

\pdfsuppresswarningpagegroup=1

\begin{document}
\begin{titlepage}
	\begin{center}
		\vspace*{1cm}

		\Huge
		\textbf{Probabilità e Statistica\\Esercizi}

		\vspace{0.5cm}
		\LARGE
		UniVR - Dipartimento di Informatica

		\vspace{1.5cm}

		\textbf{Fabio Irimie}

		\vfill


		\vspace{0.8cm}


		2° Semestre 2023/2024

	\end{center}
\end{titlepage}


\tableofcontents
\pagebreak

\section{Probabilità elementari e probabilità condizionate}
\subsection{Esercizio 1}
Un corso è frequentato da 10 studenti: 6 maschi e 4 femmine. Viene effettuato un
esame ed i punteggi degli studenti sono tutti diversi. Si suppone ciascuna classifica
equiprobabile.
\begin{enumerate}
	\item[a.] Qual è la cardinalità dello spazio dei campioni costituito da tutte le
	      possibili classifiche? Qual è una possibile misura di probabilità associata?
	      \[
		      \Omega \subseteq \mathbb{R}
	      \]
	      Tutte le possibili classifiche equivalgono a tutti i modi in cui si possono
	      ordinare i punteggi degli studenti. Quindi, la cardinalità dello spazio degli
	      eventi è $10!$:
	      \[
		      card(\Omega) = 10!
	      \]
	      La probabilità associata è calcolata come il rapporto tra il numero di eventi
	      favorevoli e il numero di eventi possibili:
	      \[
		      P(\omega) = \frac{card(\omega)}{card(\Omega)} = \frac{1}{10!}
	      \]
	\item[b.] Qual è la probabilità che le quattro studentesse ottengano punteggi migliori?
	      \[
		      E = \text{"Studentesse ottengono punteggi migliori"}
	      \]
	      La probabilità che le studentesse ottengano punteggi migliori è data dal rapporto
	      tra il numero di eventi favorevoli e il numero di eventi possibili:
	      \[
		      P(E) = \frac{4!}{10!}
	      \]
\end{enumerate}

\subsection{Esercizio 2}
In uno stock di 100 prodotti, 20 sono difettosi.
\begin{enumerate}
	\item[a.] Dieci vengono scelti a caso, senza rimpiazzo.

	      \noindent Qual è la probabilità che esattamente la metà siano difettosi?

	      \vspace{1em}
	      Per calcolare la probabilità che esattamente la metà dei prodotti siano difettosi
	      si deve fare il rapporto tra la moltiplicazione del numero di modi in cui si possono
	      scegliere 5 prodotti difettosi e 5 prodotti non difettosi e il numero di modi in
	      cui si possono scegliere 10 prodotti:
	      \[
		      P(E) = \frac{\binom{20}{5} \cdot \binom{80}{5}}{\binom{100}{10}} =
		      \frac{\frac{20!}{5! \cdot 15!}\cdot \frac{80!}{5! \cdot 75!}}{\frac{100!}{10! \cdot 90!}} =
		      \frac{15504 \cdot 24040016}{1.731030946 \cdot 10^{13}} \approx 0.214 = 21.4\%
	      \]
	\item[b.] Dieci vengono scelti a caso, con rimpiazzo.

	      \noindent Qual è la probabilità che esattamente la metà siano difettosi?
	      \vspace{1em}
	      Si può considerare una variabile di Bernoulli:
	      \[
		      X_i = \begin{cases}
			      1 & \text{se il prodotto è difettoso} p = \frac{20}{100} = \frac{1}{5} \\
			      0 & \text{altrimenti} p = \frac{80}{100} = \frac{4}{5}
		      \end{cases}
	      \]
	      La probabilità che esattamente la metà dei prodotti siano difettosi è data dalla
	      distribuzione binomiale:
	      \[
		      P(\omega) = \binom{k}{n} p^n (1-p)^{k-n}
	      \]
	      \[
		      P(E) = \binom{10}{5} \left( \frac{1}{5} \right)^5 \left( \frac{4}{5} \right)^5 =
		      \frac{10!}{5! \cdot 5!} \left( \frac{1}{5} \right)^5 \left( \frac{4}{5} \right)^5 \approx
		      0.0264 = 2.64\%
	      \]
\end{enumerate}

\subsection{Esercizio 3}
Consideriamo l’estrazione di una carta da una mazzo di 40 carte napoletane (4 semi; per
ciascun seme 10 carte dall’asso al 7 più tre figure).
\begin{enumerate}
	\item[a.] Qual'è la probabilità di estrarre un asso?

	      \vspace{1em}
	      \noindent Il numero di semi in un mazzo di carte napoletane è 4. Quindi ci sono 4
	      assi, di conseguenza la probabilità di trovare un asso è:
	      \[
		      P(A) = \frac{4}{40} = \frac{1}{10} = 0.1 = 10\%
	      \]
\end{enumerate}
Estraiamo ora 2 carte.
\begin{enumerate}
	\item[b.] Qual'è la probabilità di estrarre un asso e un re, rispettivamente?

	      \noindent Calcolare le probabilità nel caso di:
	      \begin{itemize}
		      \item Estrazione con reinserimento

		            \vspace{1em}
		            \noindent La probabilità di estrarre un asso e un re con reinserimento è data dal
		            prodotto delle probabilità di estrarre un asso e un re:
		            \[
			            P(A) = \frac{4}{40} = \frac{1}{10} = 0.1 = 10\% \quad \text{Estrarre un asso}
		            \]
		            \[
			            P(R) = \frac{4}{40} = \frac{1}{10} = 0.1 = 10\% \quad \text{Estrarre un re}
		            \]
		            \[
			            P(A \cap R) = P(A) \cdot P(R) = \frac{4}{40} \cdot \frac{4}{40} = \frac{1}{10} \cdot \frac{1}{10} = \frac{1}{100} = 0.01 = 1\%
		            \]
		      \item Estrazione senza reinserimento

		            \vspace{1em}
		            La probabilità di estrarre un re dopo aver estratto un asso è data dalla
		            probabilità condizionata:
		            \[
			            P(R\,|\,A) = \frac{P(R \cap A)}{P(A)} = \frac{\frac{1}{10} \cdot \frac{1}{10}}{\frac{1}{10}} = \frac{1}{10} = 0.1 = 10\%
		            \]
	      \end{itemize}
\end{enumerate}

\subsection{Esercizio 4}
Una popolazione si compone per un 40 percento di fumatori (F) e per il restante 60 per
cento di non fumatori (N). Si sa che il 25 per cento dei fumatori e il 7 per cento dei
non fumatori ha una malattia respiratoria cronica (M).
\begin{enumerate}
	\item Calcolare la probabilità che un individuo scelto a caso sia effetto dalla
	      malattia respiratoria.

	      \vspace{1em}
	      \noindent Un individuo scelto a caso ha una probabilità di essere fumatore malato:
	      \[
		      P(M\,|\,F) = 0.25
	      \]
	      e una probabilità di essere non fumatore malato:
	      \[
		      P(M\,|\,N) = 0.07
	      \]
	      La probabilità che un individuo scelto a caso sia affetto dalla malattia respiratoria
	      è data dalla probabilità totale:
	      \[
		      P(M) = P(M\,|\,F) \cdot P(F) + P(M\,|\,N) \cdot P(N)
	      \]
	      \[
		      P(M) = 0.25 \cdot 0.4 + 0.07 \cdot 0.6 = 0.1 + 0.042 = 0.142 = 14.2\%
	      \]
	\item Se l’individuo scelto è affetto dalla malattia, calcolare la probabilità che
	      sia un fumatore.

	      \vspace{1em}
	      \noindent La probabilità che un individuo affetto dalla malattia sia un fumatore è
	      data dalla probabilità condizionata:
	      \[
		      P(F\,|\,M) = \frac{P(M\,|\,F) \cdot P(F)}{P(M)} = \frac{0.25 \cdot 0.4}{0.142}
		      = \frac{0.1}{0.142} = 0.704 = 70.4\%
	      \]
\end{enumerate}

\subsection{Esercizio 5}
Una particolare analisi del sangue è efficace al 99\% nell’individuare una determinata
malattia quando essa è presente. Si possono però verificare dei falsi positivi
(ovvero una persona sana che si sottopone al test ha una probabilità pari a 0.01 di
risultare erroneamente positiva al test). Se l’incidenza della malattia è del 0.5\%,
calcolare la probabilità che un individuo risultato positivo al test sia effettivamente
malato.

\vspace{1em}
\[
	T = \text{Test positivo}
\]
\[
	M = \text{Persona malata} \quad \overline{M} = \text{Persona sana}
\]
La probabilità che il test sia efficace è:
\[
	P(T\,|\,M) = 0.99
\]
La probabilità che il test non sia efficace è:
\[
	P(T\,|\,\overline{M}) = 0.01
\]
La probabilità di essere malato è:
\[
	P(M) = 0.005 = 0.5\%
\]
La probabilità che il test sia positivo è data dalla probabilità totale:
\[
	P(T) = P(T\,|\,M) \cdot P(M) + P(T\,|\,\overline{M}) \cdot P(\overline{M})
\]
\[
	P(T) = 0.99 \cdot 0.005 + 0.01 \cdot 0.995 = 0.00495 + 0.00995 = 0.0149 = 1.49\%
\]
La probabilità che un individuo risultato positivo al test sia effettivamente malato è
data dalla probabilità condizionata:
\[
	P(T\,|\,P) = \frac{P(T\,|\,M) \cdot P(M)}{P(T)}
\]
\[
	P(T\,|\,P) = \frac{0.99 \cdot 0.005}{0.0149} = \frac{0.00495}{0.0149} \approx
	0.332 = 33.2\%
\]

\subsection{Esercizio 6}
In un vivaio si vendono dei sacchetti con 30 bulbi di dalie. I sacchetti sono
di due tipi \( S_1 \)  e \( S_2 \) . Tre quarti di tali sacchetti sono di tipo \( S_1 \) e contengono
10 bulbi di dalie rosse (R) e 20 bulbi di dalie gialle (G), mentre il restante
quarto dei sacchetti è di tipo \( S_2 \) e contiene 5 bulbi di dalie rosse e 25 bulbi di
dalie gialle.

\noindent Mario compra un sacchetto a caso e pianta un bulbo e vede di
che colore è.

\vspace{1em}
\noindent I sacchetti \( S_1 \)  sono:
\[
	P(S_1) = \frac{3}{4} = 0.75 = 75\%
\]
\[
	P(R\,|\,S_1) = \frac{10}{30} = \frac{1}{3} \approx 0.333 = 33.3\%
\]
\[
	P(G\,|\,S_1) = \frac{20}{30} = \frac{2}{3} \approx 0.666 = 66.6\%
\]
I sacchetti \( S_2 \)  sono:
\[
	P(S_2) = \frac{1}{4} = 0.25 = 25\%
\]
\[
	P(R\,|\,S_2) = \frac{5}{30} = \frac{1}{6} \approx 0.166 = 16.6\%
\]
\[
	P(G\,|\,S_2) = \frac{25}{30} = \frac{5}{6} \approx 0.833 = 83.3\%
\]

\begin{itemize}
	\item Determinare la probabilità che il bulbo produca una dalia rossa.

	      \vspace{1em}
	      La probabilità che il bulbo produca una dalia rossa è data dalla probabilità totale:
	      \[
		      P(R) = P(R\,|\,S_1) \cdot P(S_1) + P(R\,|\,S_2) \cdot P(S_2)
	      \]
	      \[
		      P(R) = \frac{1}{3} \cdot \frac{3}{4} + \frac{1}{6} \cdot \frac{1}{4} = \frac{1}{4} + \frac{1}{24} = \frac{7}{24} \approx 0.291 = 29.1\%
	      \]
	\item Se la dalia nata è rossa, qual è la probabilità che provenga da un sacchetto di
	      tipo \( S_1 \)?

	      \vspace{1em}
	      La probabilità che una dalia rossa provenga da \( S_1 \) è data dalla probabilità
	      condizionata:
	      \[
		      P(S_1\,|\,R) = \frac{P(R\,|\,S_1) \cdot P(S_1)}{P(R)}
	      \]
	      \[
		      P(S_1\,|\,R) = \frac{\frac{1}{3} \cdot \frac{3}{4}}{\frac{7}{24}} = \frac{1}{4} \cdot \frac{24}{7} = \frac{6}{7} \approx 0.857 = 85.7\%
	      \]
\end{itemize}

\subsection{Esercizio 7}
In un test clinico, un individuo viene sottoposto ad un esame di laboratorio, per
stabilire se ha o non ha una data malattia. Il test può avere esito positivo o negativo.
C’è però sempre una possibilità di errore: può darsi che alcuni degli individui
risultati positivi siano in realtà sani (falsi positivi), e che qualcuno degli individui
risultati negativi siano in realtà malati (falsi negativi). Prima di applicare su larga
scala un test nei laboratori, è quindi indispensabile valutarne la bontà, sottoponendo
al test un campione di persone che sappiamo già se sono sane o malate. Uno dei parametri
che definiscono la qualità diagnostica del test è la Sensibilità = 1 - P(falsi negativi).

\noindent In Italia c’è un malato di HIV ogni 40.000 persone. Un paziente si sottopone
ad un test con una procedura che fornisce statisticamente lo 0,7\% di falsi negativi e lo
0,01\% di falsi positivi. Calcolare la probabilità a posteriori di essere ammalato, a
test effettuato con esito positivo. Come cambia questa probabilità se però paziente e
medico si convincono che, in base ai sintomi ed alle circostanze del possibile contagio,
la probabilità a priori sia ad esempio 10 volte pi`u alta della media nazionale?
\end{document}
