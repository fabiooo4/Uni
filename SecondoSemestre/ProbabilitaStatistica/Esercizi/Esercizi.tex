\documentclass[a4paper]{article}

\usepackage[utf8]{inputenc}
\usepackage[T1]{fontenc}
\usepackage{textcomp}
\usepackage[italian]{babel}
\usepackage{amsmath, amssymb}
\usepackage[makeroom]{cancel}
\usepackage{amsfonts}
\usepackage{mdframed}
\usepackage{xcolor}
\usepackage{float}
\usepackage{tikz}
\usepackage{pgfplots}
\pgfplotsset{width=10cm,compat=1.9}
\usetikzlibrary{pgfplots.fillbetween, pgfplots.statistics}
\pgfplotsset{compat=newest, ticks=none}
\usepackage{graphicx}
\graphicspath{{./figures/}}

\usepackage{import}
\usepackage{pdfpages}
\usepackage{transparent}
\usepackage{xcolor}

\usepackage{hyperref}
\hypersetup{
    colorlinks=false,
}

\usepackage{ntheorem}
\newtheorem{theorem}{Teorema}

% Useful definitions frame
\theoremstyle{break}
\theoremheaderfont{\bfseries}
\newmdtheoremenv[%
	linecolor=gray,leftmargin=0,%
	rightmargin=0,
	innertopmargin=8pt,%
	ntheorem]{define}{Definizioni utili}[section]

% Example frame
\theoremstyle{break}
\theoremheaderfont{\bfseries}
\newmdtheoremenv[%
	linecolor=gray,leftmargin=0,%
	rightmargin=0,
	innertopmargin=8pt,%
	ntheorem]{example}{Esempio}[section]

% Important definition frame
\theoremstyle{break}
\theoremheaderfont{\bfseries}
\newmdtheoremenv[%
	linecolor=gray,leftmargin=0,%
	rightmargin=0,
	backgroundcolor=gray!40,%
	innertopmargin=8pt,%
	ntheorem]{definition}{Definizione}[section]

% Exercise frame
\theoremstyle{break}
\theoremheaderfont{\bfseries}
\newmdtheoremenv[%
	linecolor=gray,leftmargin=0,%
	rightmargin=0,
	innertopmargin=8pt,%
	ntheorem]{exercise}{Esercizio}[section]


% figure support
\usepackage{import}
\usepackage{xifthen}
\pdfminorversion=7
\usepackage{pdfpages}
\usepackage{transparent}
\newcommand{\incfig}[1]{%
	\def\svgwidth{\columnwidth}
	\import{./figures/}{#1.pdf_tex}
}

\pdfsuppresswarningpagegroup=1

\begin{document}
\begin{titlepage}
	\begin{center}
		\vspace*{1cm}

		\Huge
		\textbf{Probabilità e Statistica\\Esercizi}

		\vspace{0.5cm}
		\LARGE
		UniVR - Dipartimento di Informatica

		\vspace{1.5cm}

		\textbf{Fabio Irimie}

		\vfill


		\vspace{0.8cm}


		2° Semestre 2023/2024

	\end{center}
\end{titlepage}


\tableofcontents
\pagebreak

\section{Probabilità elementari e probabilità condizionate}
\subsection{Esercizio 1}
Un corso è frequentato da 10 studenti: 6 maschi e 4 femmine. Viene effettuato un
esame ed i punteggi degli studenti sono tutti diversi. Si suppone ciascuna classifica
equiprobabile.
\begin{enumerate}
	\item Qual è la cardinalità dello spazio dei campioni costituito da tutte le
	      possibili classifiche? Qual è una possibile misura di probabilità associata?
	      \[
		      \Omega \subseteq \mathbb{R}
	      \]
	      Tutte le possibili classifiche equivalgono a tutti i modi in cui si possono
	      ordinare i punteggi degli studenti. Quindi, la cardinalità dello spazio degli
	      eventi è $10!$:
	      \[
		      card(\Omega) = 10!
	      \]
	      La probabilità associata è calcolata come il rapporto tra il numero di eventi
	      favorevoli e il numero di eventi possibili:
	      \[
		      P(\omega) = \frac{card(\omega)}{card(\Omega)} = \frac{1}{10!}
	      \]
	\item Qual è la probabilità che le quattro studentesse ottengano punteggi migliori?
	      \[
		      E = \text{"Studentesse ottengono punteggi migliori"}
	      \]
	      La probabilità che le studentesse ottengano punteggi migliori è data dal rapporto
	      tra il numero di eventi favorevoli e il numero di eventi possibili:
	      \[
		      P(E) = \frac{4!}{10!}
	      \]
\end{enumerate}

\subsection{Esercizio 2}
In uno stock di 100 prodotti, 20 sono difettosi.
\begin{enumerate}
	\item Dieci vengono scelti a caso, senza rimpiazzo.

	      \noindent Qual è la probabilità che esattamente la metà siano difettosi?

	      \vspace{1em}
	      Per calcolare la probabilità che esattamente la metà dei prodotti siano difettosi
	      si deve fare il rapporto tra la moltiplicazione del numero di modi in cui si possono
	      scegliere 5 prodotti difettosi e 5 prodotti non difettosi e il numero di modi in
	      cui si possono scegliere 10 prodotti:
	      \[
		      P(E) = \frac{\binom{20}{5} \cdot \binom{80}{5}}{\binom{100}{10}} =
		      \frac{\frac{20!}{5! \cdot 15!}\cdot \frac{80!}{5! \cdot 75!}}{\frac{100!}{10! \cdot 90!}} =
		      \frac{15504 \cdot 24040016}{1.731030946 \cdot 10^{13}} \approx 0.214 = 21.4\%
	      \]
	\item Dieci vengono scelti a caso, con rimpiazzo.

	      \noindent Qual è la probabilità che esattamente la metà siano difettosi?
	      \vspace{1em}
	      Si può considerare una variabile di Bernoulli:
	      \[
		      X_i = \begin{cases}
			      1 & \text{se il prodotto è difettoso} p = \frac{20}{100} = \frac{1}{5} \\
			      0 & \text{altrimenti} p = \frac{80}{100} = \frac{4}{5}
		      \end{cases}
	      \]
	      La probabilità che esattamente la metà dei prodotti siano difettosi è data dalla
	      distribuzione binomiale:
	      \[
		      P(\omega) = \binom{k}{n} p^n (1-p)^{k-n}
	      \]
	      \[
		      P(E) = \binom{10}{5} \left( \frac{1}{5} \right)^5 \left( \frac{4}{5} \right)^5 =
		      \frac{10!}{5! \cdot 5!} \left( \frac{1}{5} \right)^5 \left( \frac{4}{5} \right)^5 \approx
		      0.0264 = 2.64\%
	      \]
\end{enumerate}

\subsection{Esercizio 3}
Consideriamo l’estrazione di una carta da una mazzo di 40 carte napoletane (4 semi; per
ciascun seme 10 carte dall’asso al 7 più tre figure).
\begin{enumerate}
	\item Qual'è la probabilità di estrarre un asso?

	      \vspace{1em}
	      \noindent Il numero di semi in un mazzo di carte napoletane è 4. Quindi ci sono 4
	      assi, di conseguenza la probabilità di trovare un asso è:
	      \[
		      P(A) = \frac{4}{40} = \frac{1}{10} = 0.1 = 10\%
	      \]
\end{enumerate}
Estraiamo ora 2 carte.
\begin{enumerate}
	\item[2.] Qual'è la probabilità di estrarre un asso e un re, rispettivamente?

	      \noindent Calcolare le probabilità nel caso di:
	      \begin{itemize}
		      \item Estrazione con reinserimento

		            \vspace{1em}
		            \noindent La probabilità di estrarre un asso e un re con reinserimento è data dal
		            prodotto delle probabilità di estrarre un asso e un re:
		            \[
			            P(A) = \frac{4}{40} = \frac{1}{10} = 0.1 = 10\% \quad \text{Estrarre un asso}
		            \]
		            \[
			            P(R) = \frac{4}{40} = \frac{1}{10} = 0.1 = 10\% \quad \text{Estrarre un re}
		            \]
		            \[
			            P(A \cap R) = P(A) \cdot P(R) = \frac{4}{40} \cdot \frac{4}{40} = \frac{1}{10} \cdot \frac{1}{10} = \frac{1}{100} = 0.01 = 1\%
		            \]
		      \item Estrazione senza reinserimento

		            \vspace{1em}
		            La probabilità di estrarre un re dopo aver estratto un asso è data dalla
		            probabilità condizionata:
		            \[
			            P(R\,|\,A) = \frac{P(R \cap A)}{P(A)} = \frac{\frac{1}{10} \cdot \frac{1}{10}}{\frac{1}{10}} = \frac{1}{10} = 0.1 = 10\%
		            \]
	      \end{itemize}
\end{enumerate}

\subsection{Esercizio 4}
Una popolazione si compone per un 40 percento di fumatori (F) e per il restante 60 per
cento di non fumatori (N). Si sa che il 25 per cento dei fumatori e il 7 per cento dei
non fumatori ha una malattia respiratoria cronica (M).
\begin{enumerate}
	\item Calcolare la probabilità che un individuo scelto a caso sia effetto dalla
	      malattia respiratoria.

	      \vspace{1em}
	      \noindent Un individuo scelto a caso ha una probabilità di essere fumatore malato:
	      \[
		      P(M\,|\,F) = 0.25
	      \]
	      e una probabilità di essere non fumatore malato:
	      \[
		      P(M\,|\,N) = 0.07
	      \]
	      La probabilità che un individuo scelto a caso sia affetto dalla malattia respiratoria
	      è data dalla probabilità totale:
	      \[
		      P(M) = P(M\,|\,F) \cdot P(F) + P(M\,|\,N) \cdot P(N)
	      \]
	      \[
		      P(M) = 0.25 \cdot 0.4 + 0.07 \cdot 0.6 = 0.1 + 0.042 = 0.142 = 14.2\%
	      \]
	\item Se l’individuo scelto è affetto dalla malattia, calcolare la probabilità che
	      sia un fumatore.

	      \vspace{1em}
	      \noindent La probabilità che un individuo affetto dalla malattia sia un fumatore è
	      data dalla probabilità condizionata:
	      \[
		      P(F\,|\,M) = \frac{P(M\,|\,F) \cdot P(F)}{P(M)} = \frac{0.25 \cdot 0.4}{0.142}
		      = \frac{0.1}{0.142} = 0.704 = 70.4\%
	      \]
\end{enumerate}

\subsection{Esercizio 5}
Una particolare analisi del sangue è efficace al 99\% nell’individuare una determinata
malattia quando essa è presente. Si possono però verificare dei falsi positivi
(ovvero una persona sana che si sottopone al test ha una probabilità pari a 0.01 di
risultare erroneamente positiva al test). Se l’incidenza della malattia è del 0.5\%,
calcolare la probabilità che un individuo risultato positivo al test sia effettivamente
malato.

\vspace{1em}
\[
	T = \text{Test positivo}
\]
\[
	M = \text{Persona malata} \quad \overline{M} = \text{Persona sana}
\]
La probabilità che il test sia efficace è:
\[
	P(T\,|\,M) = 0.99
\]
La probabilità che il test non sia efficace è:
\[
	P(T\,|\,\overline{M}) = 0.01
\]
La probabilità di essere malato è:
\[
	P(M) = 0.005 = 0.5\%
\]
La probabilità che il test sia positivo è data dalla probabilità totale:
\[
	P(T) = P(T\,|\,M) \cdot P(M) + P(T\,|\,\overline{M}) \cdot P(\overline{M})
\]
\[
	P(T) = 0.99 \cdot 0.005 + 0.01 \cdot 0.995 = 0.00495 + 0.00995 = 0.0149 = 1.49\%
\]
La probabilità che un individuo risultato positivo al test sia effettivamente malato è
data dalla probabilità condizionata:
\[
	P(T\,|\,P) = \frac{P(T\,|\,M) \cdot P(M)}{P(T)}
\]
\[
	P(T\,|\,P) = \frac{0.99 \cdot 0.005}{0.0149} = \frac{0.00495}{0.0149} \approx
	0.332 = 33.2\%
\]

\subsection{Esercizio 6}
In un vivaio si vendono dei sacchetti con 30 bulbi di dalie. I sacchetti sono
di due tipi \( S_1 \)  e \( S_2 \) . Tre quarti di tali sacchetti sono di tipo \( S_1 \) e contengono
10 bulbi di dalie rosse (R) e 20 bulbi di dalie gialle (G), mentre il restante
quarto dei sacchetti è di tipo \( S_2 \) e contiene 5 bulbi di dalie rosse e 25 bulbi di
dalie gialle.

\noindent Mario compra un sacchetto a caso e pianta un bulbo e vede di
che colore è.

\vspace{1em}
\noindent I sacchetti \( S_1 \)  sono:
\[
	P(S_1) = \frac{3}{4} = 0.75 = 75\%
\]
\[
	P(R\,|\,S_1) = \frac{10}{30} = \frac{1}{3} \approx 0.333 = 33.3\%
\]
\[
	P(G\,|\,S_1) = \frac{20}{30} = \frac{2}{3} \approx 0.666 = 66.6\%
\]
I sacchetti \( S_2 \)  sono:
\[
	P(S_2) = \frac{1}{4} = 0.25 = 25\%
\]
\[
	P(R\,|\,S_2) = \frac{5}{30} = \frac{1}{6} \approx 0.166 = 16.6\%
\]
\[
	P(G\,|\,S_2) = \frac{25}{30} = \frac{5}{6} \approx 0.833 = 83.3\%
\]

\begin{itemize}
	\item Determinare la probabilità che il bulbo produca una dalia rossa.

	      \vspace{1em}
	      La probabilità che il bulbo produca una dalia rossa è data dalla probabilità totale:
	      \[
		      P(R) = P(R\,|\,S_1) \cdot P(S_1) + P(R\,|\,S_2) \cdot P(S_2)
	      \]
	      \[
		      P(R) = \frac{1}{3} \cdot \frac{3}{4} + \frac{1}{6} \cdot \frac{1}{4} = \frac{1}{4} + \frac{1}{24} = \frac{7}{24} \approx 0.291 = 29.1\%
	      \]
	\item Se la dalia nata è rossa, qual è la probabilità che provenga da un sacchetto di
	      tipo \( S_1 \)?

	      \vspace{1em}
	      La probabilità che una dalia rossa provenga da \( S_1 \) è data dalla probabilità
	      condizionata:
	      \[
		      P(S_1\,|\,R) = \frac{P(R\,|\,S_1) \cdot P(S_1)}{P(R)}
	      \]
	      \[
		      P(S_1\,|\,R) = \frac{\frac{1}{3} \cdot \frac{3}{4}}{\frac{7}{24}} = \frac{1}{4} \cdot \frac{24}{7} = \frac{6}{7} \approx 0.857 = 85.7\%
	      \]
\end{itemize}

\subsection{Esercizio 7}
In un test clinico, un individuo viene sottoposto ad un esame di laboratorio, per
stabilire se ha o non ha una data malattia. Il test può avere esito positivo o negativo.
C’è però sempre una possibilità di errore: può darsi che alcuni degli individui
risultati positivi siano in realtà sani (falsi positivi), e che qualcuno degli individui
risultati negativi siano in realtà malati (falsi negativi). Prima di applicare su larga
scala un test nei laboratori, è quindi indispensabile valutarne la bontà, sottoponendo
al test un campione di persone che sappiamo già se sono sane o malate. Uno dei parametri
che definiscono la qualità diagnostica del test è la Sensibilità = 1 - P(falsi negativi).

\noindent In Italia c’è un malato di HIV ogni 40.000 persone. Un paziente si sottopone
ad un test con una procedura che fornisce statisticamente lo 0,7\% di falsi negativi e lo
0,01\% di falsi positivi. Calcolare la probabilità a posteriori di essere ammalato, a
test effettuato con esito positivo. Come cambia questa probabilità se però paziente e
medico si convincono che, in base ai sintomi ed alle circostanze del possibile contagio,
la probabilità a priori sia ad esempio 10 volte più alta della media nazionale?

\vspace{1em}
\[
	M = \text{Persona malata} \quad \overline{M} = \text{Persona sana}
\]
\[
	N = \text{Test negativo} \quad T = \text{Test positivo}
\]
\noindent La probabilità di avere l'HIV in Italia è:
\[
	P(M) = \frac{1}{40000} = 0.000025 = 0.0025\%
\]
La probabilità di non averlo è:
\[
	P(\overline{M}) = 1 - P(M) = 1 - 0.000025 = 0.999975 = 99.9975\%
\]
La probabilità che il test sia falso negativo è:
\[
	P(N\,|\,M) = 0.007 = 0.7\%
\]
La probabilità che il test sia falso positivo è:
\[
	P(T\,|\,\overline{M}) = 0.0001 = 0.01\%
\]
La probabilità a posteriori di essere ammalato a test effettuato con esito positivo è
data dalla probabilità condizionata:
\[
	P(M\,|\,T) = \frac{P(T\,|\,M) \cdot P(M)}{P(T)}
\]
La probabilità che il test risulti positivo è data dalla probabilità totale:
\[
	P(T) = P(T\,|\,M) \cdot P(M) + P(T\,|\,\overline{M}) \cdot P(\overline{M})
\]
\[
	P(T\,|\,M) = 1 - P(N\,|\,M) = 1 - 0.007 = 0.993 \quad \text{Sensibilità}
\]
\[
	P(T) = 0.993 \cdot 0.000025 + 0.0001 \cdot 0.999975 \approx 0.000124822 = 0.0125\%
\]

\vspace{1em}
\noindent Se la probabilità a priori è 10 volte più alta della media nazionale:
\[
	P(M) = 10 \cdot \frac{1}{40000} = 0.00025 = 0.025\%
\]
Le nuove probabilità che il test sia positivo è:
\[
	P(T) = 0.993 \cdot 0.00025 + 0.0001 \cdot 0.99975 \approx 0.000249822 = 0.0249\%
\]

\subsection{Esercizio 8}
Si considerano due dadi apparentemente uguali, di cui uno equo (ossia fornisce un numero
a caso con equiprobabilità) mentre l’altro fornisce un numero pari con probabilità doppia
rispetto a quella dei numeri dispari.

\noindent Viene scelto a caso un dado e lo si lancia.
\begin{enumerate}
	\item Calcolare la probabilità che esca un numero pari
	      \[
		      Par = \text{Esce un numero pari}
	      \]
        \[
          Dis = \text{Esce un numero dispari}
        \] 
	      \[
		      P(D_1) = \frac{1}{2} = 0.5 = 50\% \quad \text{Dado equo}
	      \]
	      \[
		      P(D_2) = \frac{1}{2} = 0.5 = 50\% \quad \text{Dado non equo}
	      \]
	      \[
		      P(Par\,|\,D_1) = \frac{1}{2} = 0.5 = 50\%
	      \]
        \[
          P(Dis\,|\,D_1) = \frac{1}{2} = 0.5 = 50\%
        \] 
	      \[
		      P(Par\,|\,D_2) = \frac{6}{9} = \frac{2}{3} \approx 0.666 = 66.6\%
	      \]
        La probabilità che esca un numero pari è data dalla probabilità totale:
        \[
          P(Par) = P(Par\,|\,D_1) \cdot P(D_1) + P(Par\,|\,D_2) \cdot P(D_2)
        \] 
        \[
          P(Par) = \frac{1}{2} \cdot \frac{1}{2} + \frac{1}{2} \cdot \frac{2}{3} = \frac{1}{4} + \frac{1}{3} = \frac{3}{12} + \frac{4}{12} = \frac{7}{12} \approx 0.583 = 58.3\% 
        \] 
	\item Sapendo che è uscito un numero dispari, quanto vale la probabilità che il dado
	      lanciato sia quello equo?

        \vspace{1em}
        \noindent La probabilità che il dado lanciato sia quello equo è data dalla probabilità
        condizionata:
        \[
        P(D_1\,|\,Dis) = \frac{P(Dis\,|\,D_1)\cdot P(D_1)}{P(Dis)}
        \] 
        \[
        P(D_1\,|\,Dis) = \frac{\frac{1}{2}*\frac{1}{2}}{1 - 0.583} = \frac{\frac{1}{4}}{0.417} \approx 0.6 = 60\%
        \] 
\end{enumerate}

\subsection{Esercizio 9}
Consideriamo una popolazione di \( N = 400 \) piantine in cui si presentano i seguenti 
genotipi AA, Aa, aa. Supponiamo che una piantina venga estratta con uguale probabilità.
Supponiamo inoltre che vi sono 196 piantine del genotipo AA, 168 del genotipo Aa, e 36 
del genotipo aa.
\[
  N = 400
\] 
\[
  AA = 196
\] 
\[
  Aa = 168
\] 
\[
  aa = 36
\] 
\begin{enumerate}
  \item Qual è la distribuzione dei genotipi, cioè se estraggo a caso una piantina qual 
    è la probabilità di ciascun genotipo?
    \[
    P(AA) = \frac{196}{400} = 0.49 = 49\%
    \] 
    \[
    P(Aa) = \frac{168}{400} = 0.42 = 42\%
    \] 
    \[
    P(aa) = \frac{36}{400} = 0.09 = 9\%
    \] 
\end{enumerate}
Supponiamo che l’allele a sia un letale recessivo, che però si esprime solo nella pianta 
allo stadio adulto.
\begin{enumerate}
  \item[2.] La distribuzione precedente è ancora valida per la fase adulta (sì, no 
    e perchè)?

    \vspace{1em}
    \noindent La distribuzione precedente non è valida per la fase adulta. Infatti,
    le piante con genotipo aa non sopravvivono allo stadio adulto, quindi la probabilità
    di trovare una pianta adulta con genotipo aa è nulla. Di conseguenza le probabilità
    in età adulta sono:
    \[
    N = 400 - 36 = 364
    \] 
    \[
    P(AA) = \frac{196}{364} \approx 0.538 = 53.8\%
    \] 
    \[
    P(Aa) = \frac{168}{364} \approx 0.462 = 46.2\%
    \] 
    \[
    P(aa) = 0
    \] 
  \item[3.] Qual è la probabilità di trovare una pianta in vita dopo un certo periodo di 
    tempo, quindi adulta?
    
    \vspace{1em}
    \noindent La probabilità di trovare una pianta in vita dopo un certo periodo di tempo
    è data dalla somma delle probabilità di trovare una pianta adulta con genotipo AA e
    Aa:
    \[
    P(AA) + P(Aa) = 0.49 + 0.42 = 0.91 = 91\%
    \] 
  \item[4.] Qual è la probabilità di estrarre una pianta adulta del genotipo AA? E del 
    tipo Aa?
    \[
      P(AA) = 0.538 = 53.8\%
    \] 
    \[
      P(Aa) = 0.462 = 46.2\%
    \]
\end{enumerate}

\subsection{Esercizio 10}
Consideriamo la tabella che riporta i risultati di uno studio in cui sono stati osservati il numero di
querce e lecci infestati da un certo insetto
\begin{table}[H]
  \centering
  \begin{tabular}{c|ccc}
    & Infestato & Non infestato & Totali \\
    \hline
    Quercia & 40 & 30 & 70 \\
    Leccio & 52 & 10 & 62 \\
    \hline
    Totali & 92 & 40 & 132 
  \end{tabular}
\end{table}
\begin{enumerate}
  \item Calcolare la probabilità che un albero sia infestato.
    \[
    P(I) = \frac{92}{132} \approx 0.697 = 69.7\%
    \] 
  \item Calcolare la probabilità che una quercia sia infestata.
    \[
    P(Q_I) = \frac{40}{70} \approx 0.571 = 57.1\%
    \] 
  \item Calcolare la probabilità che un leccio sia infestato.
    \[
    P(L_I) = \frac{52}{62} \approx 0.839 = 83.9\%
    \] 
  \item Quale definizione di probabilità si è utilizzata nei punti precedenti?
    
    \vspace{1em}
    La definizione utilizzata nei punti precedenti è: La probabilità è calcolata come
    il rapporto tra i casi favorevoli e i casi possibili.
\end{enumerate}

\subsection{Esercizio 11}
Tre scrivanie tra loro indistinguibili contengono ciascuna due cassetti. La prima 
contiene una moneta d'oro in ciascun cassetto. La seconda una monta d'argento in un
cassetto ed una moneta d'oro nell'altro, la terza due monte d'argento. Si apre un 
cassetto e si trova una moneta d'oro. Qual è la probabilità che anche nell'altro cassetto
della stessa scrivania ci sia una moneta d'oro?

\vspace{1em}
\noindent Avendo trovato una moneta d'oro, le uniche possibilità sono che la moneta sia stata
trovata nella scrivania 1 o nella 2. La probabilità che la seconda moneta sia d'oro riduce
le possibilità alla scrivania 1, quindi:
\[
  O = \text{Moneta d'oro}
\]
\[
  P(S_1) = \frac{1}{3} = 0.333 = 33.3\%
\] 
\[
  P(O\,|\,S_1) = 1
\] 
\[
  P(O) = \left( 1 + \frac{1}{2} + 0 \right) \cdot \frac{1}{3} = \frac{1}{2} = 0.5 = 50\%
\] 
La probabilità che la seconda moneta sia d'oro è data dalla probabilità condizionata:
\[
  P(S_1\,|\,O) = \frac{P(O\,|\,S_1) \cdot P(S_1)}{P(O)}
\] 
\[
  P(S_1\,|\,O) = \frac{1 \cdot \frac{1}{3}}{\frac{1}{2}} = \frac{2}{3} = 0.666 = 66.6\%
\] 

\subsection{Esercizio 12}
Da un'indagine nelle scuole, risulta che la percentuale degli alunni che portano gli
occhiali è il 10\% nelle scuole elementari, il 25\% nelle scuole medie e il 40\% nelle 
superiori.
\[
  O = \text{Porta gli occhiali}
\] 
\[
  E = \text{Frequenta le elementari}
\] 
\[
  M = \text{Frequenta le medie}
\] 
\[
  S = \text{Frequenta le superiori}
\] 

\[
  P(O\,|\,E) = 0.1 = 10\% 
\] 
\[
  P(O\,|\,M) = 0.25 = 25\%
\] 
\[
  P(O\,|\,S) = 0.4 = 40\%
\] 
\begin{enumerate}
  \item Calcolare la probabilità che scegliendo a caso 3 studenti, uno per fascia, almeno
    uno porti gli occhiali.

    \vspace{1em}
    La probabilità che almeno uno porti gli occhiali è data dal complementare dell'evento
    che nessuno porti gli occhiali:
    \[
      P(O \ge 1) = 1 - P(O = 0)
    \] 
    \[
      P(O \ge 1) = 1 - \left( P(\overline{O}\,|\,E) \cdot P(\overline{O}\,|\,M) \cdot P(\overline{O}\,|\,S) \right) =
    \] 
    \[
      = 1 - \left( 0.9 \cdot 0.75 \cdot 0.6 \right) = 1 - 0.405 = 0.595 = 59.5\%
    \] 
  \item Scegliendo uno studente a caso fra tutti (e supponendo che la scelta di ogni 
    fascia sia equiprobabile) calcolare la probabilità che questo abbia gli occhiali.
    
    \vspace{1em}
    La probabilità che uno studente scelto a caso è data dalla probabilità totale:
    \[
      P(O) = P(O\,|\,E) \cdot P(E) + P(O\,|\,M) \cdot P(M) + P(O\,|\,S) \cdot P(S)
    \] 
    \[
      P(O) = 0.1 \cdot \frac{1}{3} + 0.25 \cdot \frac{1}{3} + 0.4 \cdot \frac{1}{3} = \frac{1}{4} = 0.25 = 25\%
    \] 
  \item Sapendo che lo studente porti gli occhiali, calcolare la probabilità che 
    frequenti le elementari.

    \vspace{1em}
    \noindent La probabilità che uno studente frequenti le elementari sapendo che porta
    gli occhiali è data dalla probabilità condizionata:
    \[
    P(E\,|\,O) = \frac{P(O\,|\,E) \cdot P(E)}{P(O)}
    \] 
    \[
      P(E\,|\,O) = \frac{0.1 \cdot \frac{1}{3}}{\frac{1}{4}} \approx 0.2667 = 26.67\%
    \] 
\end{enumerate}

\subsection{Esercizio 13}
È stata studiata la relazione tra forma fisica e malattie cardiovascolari in un gruppo di
impiegati delle ferrovie di sesso maschile, con questi risultati:
\begin{table}[H]
  \centering
  \begin{tabular}{p{1.3in}|p{1.3in}|p{1.3in}}
    Frequenza cardiaca\newline sotto esercizio\newline (battiti/minuto) &
    Percentuale dei lavoratori &
    Mortalità per malattie cardiovascolari in 20 anni\\
    \hline
    \centering{\( \le 105 \)} & \centering{20\%} & \centering{9.2\%} \cr
    \centering{106-115} & \centering{30\%} & \centering{8.7\%} \cr
    \centering{116-127} & \centering{30\%} & \centering{11.6\%} \cr
    \centering{\( > 127 \) } & \centering{20\%} & \centering{13.2\%} 
  \end{tabular}
\end{table}
\noindent Supponiamo che un certo test sia positivo se la frequenza cardiaca sotto esercizio è
maggiore di 127 battiti al minuto e negativo altrimenti.
\begin{enumerate}
  \item Qual è la probabilità di mortalità per malattie cardiovascolari in 20 anni?
  \item Qual è la probabilità di avere avuto un test positivo tra gli uomini che sono 
    morti nel periodo di 20 anni?
  \item Qual è la probabilità di avere avuto un test positivo tra gli uomini che sono 
    sopravvissuti nel periodo di 20 anni?
  \item Qual è la probabilità di morte tra gli uomini con un test negativo?
\end{enumerate}
\end{document}
