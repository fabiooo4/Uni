\documentclass{article}
\usepackage[italian]{babel}
\usepackage{amsfonts}
\usepackage{mdframed}
\usepackage{ntheorem}
\usepackage{xcolor}
\usepackage{graphicx}
\graphicspath{ {./figures/} }

\usepackage{import}
\usepackage{pdfpages}
\usepackage{transparent}
\usepackage{xcolor}

% Inkscape figures
\newcommand{\incfig}[2][1]{%
	\def\svgwidth{#1\columnwidth}
	\import{./figures/}{#2.pdf_tex}
}

\pdfsuppresswarningpagegroup=1

% Useful definitions frame
\theoremstyle{break}
\theoremheaderfont{\bfseries}
\newmdtheoremenv[%
	linecolor=gray,leftmargin=0,%
	rightmargin=0,
	innertopmargin=8pt,%
	ntheorem]{define}{Definizioni utili}[section]

% Example frame
\theoremstyle{break}
\theoremheaderfont{\bfseries}
\newmdtheoremenv[%
	linecolor=gray,leftmargin=0,%
	rightmargin=0,
	innertopmargin=8pt,%
	ntheorem]{example}{Esempio}[section]

% Important definition frame
\theoremstyle{break}
\theoremheaderfont{\bfseries}
\newmdtheoremenv[%
	linecolor=gray,leftmargin=0,%
	rightmargin=0,
	backgroundcolor=gray!40,%
	innertopmargin=8pt,%
	ntheorem]{definition}{Definizione}[section]

% Exercise frame
\theoremstyle{break}
\theoremheaderfont{\bfseries}
\newmdtheoremenv[%
	linecolor=gray,leftmargin=0,%
	rightmargin=0,
	innertopmargin=8pt,%
	ntheorem]{exercise}{Esercizio}[section]



\begin{document}
\begin{titlepage}
	\begin{center}
		\vspace*{1cm}

		\Huge
		\textbf{Analisi 1}

		\vspace{0.5cm}
		\LARGE
		UniVR - Dipartimento di Informatica

		\vspace{1.5cm}

		\textbf{Fabio Irimie}

		\vfill


		\vspace{0.8cm}

    Corso di Giacomo Canevari

		1° Semestre 2023/2024

	\end{center}
\end{titlepage}


\tableofcontents
\pagebreak

\section{Introduzione}
La logica ha lo scopo di formalizzare il ragionamento matematico
che è caratterizzato dal concetto di dimostrazione senza ambiguità

\section{Sintassi della logica proposizionale}
La logica proposizionale è formata da simboli formali ben definiti
e sono divisi in:
\subsection{Connettivi}
\begin{itemize}
	\item \( \vee \) Congiunzione, And logico
	\item \( \wedge \) Disgiunzione, Or logico
	\item \( \neg \) Negazione, Not logico (non connette niente, è solo una costante logica
	      che equivale a 0 nella logica booleana)
	\item \( \bot \) Falso, Bottom, Assurdo
	\item \( \to  \) Implicazione, If-then
\end{itemize}

\subsection{Ausiliari}
\begin{itemize}
	\item () Le parentesi non fanno parte della proposizione,
	      ma servono solo a costruire il linguaggio
\end{itemize}

\subsection{Simboli proposizionali}
\begin{itemize}
	\item \( p_n, q_n, \psi_n, \ldots  \) Le lettere minuscole indicizzate vengono usate per indicare
	      una proposizione (sono infiniti simboli numerabili)
\end{itemize}

\subsection{Altri simboli}
\begin{itemize}
	\item \( | \) Tale che
	\item \( \leftrightarrow \) Se e solo se
\end{itemize}

\begin{define}
	\begin{enumerate}
		\item \textbf{Stringa}: Una sequenza finita di simboli o caratteri
		\item \textbf{Infinito numerabile}: Un insieme è infinito numerabile se è il
		      più piccolo infinito possibile, cioè se è in corrispondenza
		      biunivoca con l'insieme \( \mathbb{N} \)
	\end{enumerate}
\end{define}

\section{Principio di induzione}
Il principio di induzione è un principio logico che permette di dimostrare che una proprietà è vera
per tutti gli elementi di un insieme infinito numerabile.

Una prima definizione induttiva fatta in modo non formale, ma con frasi in italiano è la seguente:

L'insieme di proposizioni \( PROP \) è così definito \emph{induttivamente}:
\begin{enumerate}
	\item \( \bot \to PROP \)
	\item se \( p \) è un simbolo proposizionale allora \( p \in PROP \)
	\item \textbf{(Caso induttivo)} se \( \alpha, \beta \in PROP \) allora
	      \( (\alpha \wedge \beta ) \in PROP, (\alpha \vee \beta ) \in PROP,
	      (\alpha \to \beta ) \in PROP, (\neg \alpha ) \in PROP \)
	\item nient'altro appartiene a \( PROP \)
\end{enumerate}

In questo modo è stato creato l'insieme \( PROP \) che contiene tutte le proposizioni che possono essere create
usando gli unici simboli che abbiamo definito \( (\wedge, \vee, \to, \neg) \).
\vspace{0.5cm}
\\
Esempi di proposizioni corrette e scorrette:
\begin{itemize}
	\item \( (p_{7} \to p_{0}) \in PROP \)
	\item \( p_{7} \to p_{0} \notin PROP \) (mancano le parentesi)
	\item \( ((\bot \vee p_{32}) \wedge (\neg p_{2})) \in PROP \)
	\item \( ((\to \wedge \notin PROP \)
	\item \( \neg \neg \bot \notin PROP \)
\end{itemize}

\subsection{Definizione induttiva formale dell'insieme \( PROP \)}
Adesso l'insieme \( PROP \) viene definito in modo formale usando i simboli proposizionali.
\begin{definition}
	L'insieme \( PROP \) è il più piccolo insieme \( X \) di stringhe tale che:
	\begin{enumerate}
		\item \( \bot \in X \)
		\item \( p \in X \) (Perchè è un simbolo proposizionale)
		\item se \( \alpha, \beta \in X \) allora \( (\alpha \to \beta ) \in X, (\alpha \vee \beta ) \in X,
		      (\alpha \wedge \beta ) \in X, (\neg \alpha ) \in X \)
	\end{enumerate}
	\( p, \alpha , \beta, \ldots  \) sono elementi proposizionali generici
\end{definition}
\underline{\( AT  = \) simboli proposizionali + \( \bot \)}
è l'insieme di tutte le proposizioni atomiche,
cioè quelle che non contengono connettivi, sono quindi la più piccola parte
non ulteriormente scomponibile

\section{Proprietà su un insieme}
Definito \( P \) un insieme di proprietà assunte da un insieme \( A \) si ha che:
\begin{itemize}
	\item \( P \subseteq A \)
	\item \( a \in A \) dove \( a \) è un elemento generico dell'insieme \( A \)
\end{itemize}
Si dice che \( a \) gode della proprietà \( P \) se \( a \in P \).

Altri modi per dire che \( a \) gode della proprietà \( P \) sono:
\begin{itemize}
	\item \( P(a) \)
	\item \( P[a] \) (per non creare confusione con le parentesi tonde che sono
	      usate come simboli ausiliari per costruire il linguaggio)
\end{itemize}

\( P \subseteq PROP \)\hspace{5mm} \( \forall \alpha \in PROP \) \( . \) \( P(\alpha ) \)\\
(il punto mette in evidenza ciè che viene dopo di esso e può anche essere omesso)

\begin{example}
	Esempio di una proprietà sull'insieme \( \mathbb{N} \):

	\( P = \{n | n \in \mathbb{N} \) ed è pari \( \} \) essendo \( n \) un numero generico
	indica la proprietà di essere pari.
	\\
	\( P[5] \) \( \times  \)\\
	\( P[4] \) \( \surd \)
\end{example}

\subsection{Principio di induzione sui numeri naturali \( \mathbb{N} \)}
\( P \subseteq \mathbb{N} \)
\begin{enumerate}
	\item \textbf{Caso base}: se \( P[0] \) e
	\item \textbf{Passo induttivo}: se \( \forall n \in \mathbb{N} (P[n] \Rightarrow P[n+1]) \)
	      allora \( \forall n \in \mathbb{N} \hspace{2mm} . \hspace{2mm} P[n] \)
\end{enumerate}

Se si dimostra la proprietà per \( n \) e per il successivo \( (n+1) \), allora
si dimostra che la proprietà è vera per tutti i numeri naturali. Si sfrutta
il fatto che esiste un minimo a cui prima o poi si arriva.
\begin{exercise}
	Dimostra per induzione che: \( TODO \)
	\[\sum_{i=0}^{n} i = \frac{n(n+1)}{2}\]

\end{exercise}
\section{Teorema del principio di induzione su \( PROP \)}
\begin{definition}
	\( P \subseteq PROP \)
	\begin{enumerate}
		\item Se \( P[\alpha ], \alpha \in AT \) e
		\item Se \( P[\alpha ] \Rightarrow P[(\neg \alpha )] \) e
		\item se \( P[\alpha ]\) e \(P[\beta ] \Rightarrow P[(\alpha \wedge \beta )],
		      P[(\alpha \vee \beta) P[(\alpha \to \beta )]\)\\
		      allora \( \forall \psi \in PROP \hspace{2mm} . \hspace{2mm} P[\psi]\)
	\end{enumerate}
\end{definition}

Con questo teorema si possono dimostrare intere proposizioni complesse dimostrando i pezzi
più piccoli (\textbf{\emph{sottoformule}}) come mostrato nella figura \ref{fig:dimostrazionecomplessa}.

%\begin{example}
\label{ex:dimostrazioneComplessa}
\begin{figure}[ht]
	\centering
	\incfig[1]{dimostrazionecomplessa}
	\caption{Dimostrazione di una formula complessa}
	\label{fig:dimostrazionecomplessa}
\end{figure}
%\end{example}

\begin{exercise}
	Dimostra che ogni \( \psi \in PROP \) ha un numero pari di parentesi usando
	il principio di induzione per dimostrare proprietà sintattiche sulla struttura
	delle formule.

	\( P[\psi] \equiv \) \( \psi \) ha un numero pari di parentesi
	\begin{enumerate}
		\item \textbf{Caso base} \( \psi \in AT \) quindi \( \psi \) ha 0 parentesi
		      e quindi è pari: \( P[\psi] \) \( \surd \)
		\item \textbf{Ipotesi induttiva} \( \alpha , \beta \in PROP, \) \( P[\alpha ], P[\beta ] \)
		      ? \( P[(\alpha \to \beta )] \) (per \( \alpha  \) vale e per \( \beta  \) vale, si sono
		      aggiunte due parentesi, quindi la formula è ancora pari)
		\item \textbf{Passo induttivo} \( P[\alpha], P[\beta] \Rightarrow P[(\alpha \to \beta )], P[(\alpha \vee \beta )],
		      P[(\alpha \wedge \beta)]\) allora \( \forall \psi \in PROP \hspace{2mm}.\hspace{2mm} P[\psi] \)

	\end{enumerate}
\end{exercise}

\end{document}
