\documentclass{article}
\title{Logica}
\author{Fabio Irimie}
\date{1° semestre 2023}
\begin{document}
\maketitle
\tableofcontents
\section{Introduzione}
\subsection{Logica Proposizionale}
$<\N, 0, succ>$
\paragraph{succ} è una funzione che associa un numero naturale ad un altro numero naturale: $succ(n) = n + 1$

I naturali soddisfano i seguenti assiomi:
\begin{itemize}
\item \textbf{assioma 1} 0 è un elemento privilegiato di $N$ detto zero
\item \textbf{assioma 2} $succ: \N \to \N$ è un operazione unaria iniettiva su A
\item \textbf{assioma 3} $0 \notin Im(succ)$
\item \textbf{assioma 4} Se $P \subseteq \N$ e valgono le seguenti proprietà
\begin{enumerize}
\item $0 \in P$
\item $\forall n \in \N . (n \in P \Rightarrow succ(n) \in P$
allora $P = N$
\end{enumerize}
\end{itemize}
\end{document}
