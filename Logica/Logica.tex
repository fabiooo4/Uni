\documentclass{article}
\usepackage[italian]{babel}
\usepackage{amsfonts}
\usepackage{mdframed}
\usepackage{ntheorem}
\usepackage{xcolor}
\usepackage{graphicx}
\graphicspath{ {./figures/} }

\usepackage{import}
\usepackage{pdfpages}
\usepackage{transparent}
\usepackage{xcolor}

% Inkscape figures
\newcommand{\incfig}[2][1]{%
	\def\svgwidth{#1\columnwidth}
	\import{./figures/}{#2.pdf_tex}
}

\pdfsuppresswarningpagegroup=1

% Useful definitions frame
\theoremstyle{break}
\theoremheaderfont{\bfseries}
\newmdtheoremenv[%
	linecolor=gray,leftmargin=0,%
	rightmargin=0,
	innertopmargin=8pt,%
	ntheorem]{define}{Definizioni utili}[section]

% Example frame
\theoremstyle{break}
\theoremheaderfont{\bfseries}
\newmdtheoremenv[%
	linecolor=white,leftmargin=0,%
	rightmargin=0,
	innertopmargin=8pt,%
	ntheorem]{example}{Esempio}[section]

% Important definition frame
\theoremstyle{break}
\theoremheaderfont{\bfseries}
\newmdtheoremenv[%
	linecolor=gray,leftmargin=0,%
	rightmargin=0,
	backgroundcolor=gray!40,%
	innertopmargin=8pt,%
	ntheorem]{definition}{Definizione}[section]

% Exercise frame
\theoremstyle{break}
\theoremheaderfont{\bfseries}
\newmdtheoremenv[%
	linecolor=gray,leftmargin=0,%
	rightmargin=0,
	innertopmargin=8pt,%
	ntheorem]{exercise}{Esercizio}[section]



\begin{document}
\begin{titlepage}
	\begin{center}
		\vspace*{1cm}

		\Huge
		\textbf{Probabilità e Statistica\\Esercizi}

		\vspace{0.5cm}
		\LARGE
		UniVR - Dipartimento di Informatica

		\vspace{1.5cm}

		\textbf{Fabio Irimie}

		\vfill


		\vspace{0.8cm}


		2° Semestre 2023/2024

	\end{center}
\end{titlepage}


\tableofcontents
\pagebreak

\section{Ripasso di matematica}
\subsection{Relazioni}
Prendendo in considerazione 2 insiemi \( A, B \) e una relazione \( f \subseteq A\times B \)
si definisce \textbf{dominio} l'insieme \( A \) e \textbf{codominio} l'insieme \( B \).
Il prodotto cartesiano è definito nel seguente modo:
\[
	A \times B = \{(a,b) | a \in A, b \in B\}
\]
Ciò significa che si prende in considerazione una coppia ordinata di elementi formata da
un elemento di \( A \) e uno di \( B \).
La relazione \( f \) è una funzione sse (se e solo se) \( \forall a \in A \) \( \exists \) unico \( b \in B \)
si dice che: \( (a,b) \in f \), oppure \( f(a) = b \).

\subsection{Sottoinsieme delle parti}
Dato un insieme \( A \) si definisce \textbf{sottoinsieme delle parti} (scritto \( \mathcal{P}(A) \) o \( 2^{A} \))
l'insieme di tutti i sottoinsiemi di \( A \), cioè \( 2^{A} = {x|x \subseteq A} \).

Un esempio è il seguente:
\[
	A = \{3, 5\}
\]
\[
	2^{A} = \{ \emptyset, \{3\}, \{5\}, \{3,5\} \}
\]

\( \emptyset \) è l'insieme vuoto, cioè l'insieme che non contiene nessun elemento.

\section{Introduzione}
La logica ha lo scopo di formalizzare il ragionamento matematico
che è caratterizzato dal concetto di dimostrazione senza ambiguità

\section{Sintassi della logica proposizionale}
La logica proposizionale è formata da simboli formali ben definiti
e sono divisi in:
\subsection{Connettivi}
\begin{itemize}
	\item \( \vee \) Congiunzione, And logico
	\item \( \wedge \) Disgiunzione, Or logico
	\item \( \neg \) Negazione, Not logico (non connette niente, è solo una costante logica
	      che equivale a 0 nella logica booleana)
	\item \( \bot \) Falso, Bottom, Assurdo
	\item \( \to  \) Implicazione, If-then
\end{itemize}

\subsection{Ausiliari}
\begin{itemize}
	\item () Le parentesi non fanno parte della proposizione,
	      ma servono solo a costruire il linguaggio
\end{itemize}

\subsection{Simboli proposizionali}
\begin{itemize}
	\item \( p_n, q_n, \psi_n, \ldots  \) Le lettere minuscole indicizzate vengono usate per indicare
	      una proposizione (sono infiniti simboli numerabili)
\end{itemize}

\subsection{Altri simboli}
\begin{itemize}
	\item \( | \) Tale che
	\item \( \leftrightarrow \) Se e solo se
\end{itemize}

\begin{define}
	\begin{enumerate}
		\item \textbf{Stringa}: Una sequenza finita di simboli o caratteri
		\item \textbf{Infinito numerabile}: Un insieme è infinito numerabile se è il
		      più piccolo infinito possibile, cioè se è in corrispondenza
		      biunivoca con l'insieme \( \mathbb{N} \)
	\end{enumerate}
\end{define}

\section{Principio di induzione}
Il principio di induzione è un principio logico che permette di dimostrare che una proprietà è vera
per tutti gli elementi di un insieme infinito numerabile.

Una prima definizione induttiva fatta in modo non formale, ma con frasi in italiano è la seguente:

L'insieme di proposizioni \( PROP \) è così definito \emph{induttivamente}:
\begin{enumerate}
	\item \( \bot \to PROP \)
	\item se \( p \) è un simbolo proposizionale allora \( p \in PROP \)
	\item \textbf{(Caso induttivo)} se \( \alpha, \beta \in PROP \) allora
	      \( (\alpha \wedge \beta ) \in PROP, (\alpha \vee \beta ) \in PROP,
	      (\alpha \to \beta ) \in PROP, (\neg \alpha ) \in PROP \)
	\item nient'altro appartiene a \( PROP \)
\end{enumerate}

In questo modo è stato creato l'insieme \( PROP \) che contiene tutte le proposizioni che possono essere create
usando gli unici simboli che abbiamo definito \( (\wedge, \vee, \to, \neg) \).
\vspace{0.5cm}
\\
Esempi di proposizioni corrette e scorrette:
\begin{itemize}
	\item \( (p_{7} \to p_{0}) \in PROP \)
	\item \( p_{7} \to p_{0} \notin PROP \) (mancano le parentesi)
	\item \( ((\bot \vee p_{32}) \wedge (\neg p_{2})) \in PROP \)
	\item \( ((\to \wedge \notin PROP \)
	\item \( \neg \neg \bot \notin PROP \)
\end{itemize}

\subsection{Definizione induttiva formale dell'insieme \( PROP \)}
Adesso l'insieme \( PROP \) viene definito in modo formale usando i simboli proposizionali.
\begin{definition}
	L'insieme \( PROP \) è il più piccolo insieme \( X \) di stringhe tale che:
	\begin{enumerate}
		\item \( \bot \in X \)
		\item \( p \in X \) (Perchè è un simbolo proposizionale)
		\item se \( \alpha, \beta \in X \) allora \( (\alpha \to \beta ) \in X, (\alpha \vee \beta ) \in X,
		      (\alpha \wedge \beta ) \in X, (\neg \alpha ) \in X \)
	\end{enumerate}
	\( p, \alpha , \beta, \ldots  \) sono elementi proposizionali generici
\end{definition}
\underline{\( AT  = \) simboli proposizionali + \( \bot \)}
è l'insieme di tutte le proposizioni atomiche,
cioè quelle che non contengono connettivi, sono quindi la più piccola parte
non ulteriormente scomponibile

\section{Proprietà su un insieme}
Definito \( P \) un insieme di proprietà assunte da un insieme \( A \) si ha che:
\begin{itemize}
	\item \( P \subseteq A \)
	\item \( a \in A \) dove \( a \) è un elemento generico dell'insieme \( A \)
\end{itemize}
Si dice che \( a \) gode della proprietà \( P \) se \( a \in P \).

Altri modi per dire che \( a \) gode della proprietà \( P \) sono:
\begin{itemize}
	\item \( P(a) \)
	\item \( P[a] \) (per non creare confusione con le parentesi tonde che sono
	      usate come simboli ausiliari per costruire il linguaggio)
\end{itemize}

\( P \subseteq PROP \)\hspace{5mm} \( \forall \alpha \in PROP \) \( . \) \( P(\alpha ) \)\\
(il punto mette in evidenza ciè che viene dopo di esso e può anche essere omesso)

\begin{example}
	Esempio di una proprietà sull'insieme \( \mathbb{N} \):

	\( P = \{n | n \in \mathbb{N} \) ed è pari \( \} \) essendo \( n \) un numero generico
	indica la proprietà di essere pari.
	\\
	\( P[5] \) \( \times  \)\\
	\( P[4] \) \( \surd \)
\end{example}

\subsection{Principio di induzione sui numeri naturali \( \mathbb{N} \)}
\( P \subseteq \mathbb{N} \)
\begin{enumerate}
	\item \textbf{Caso base}: se \( P[0] \) e
	\item \textbf{Passo induttivo}: se \( \forall n \in \mathbb{N} (P[n] \Rightarrow P[n+1]) \)
	      allora \( \forall n \in \mathbb{N} \hspace{2mm} . \hspace{2mm} P[n] \)
\end{enumerate}

Se si dimostra la proprietà per \( n \) e per il successivo \( (n+1) \), allora
si dimostra che la proprietà è vera per tutti i numeri naturali. Si sfrutta
il fatto che esiste un minimo a cui prima o poi si arriva.
\begin{exercise}
	Dimostra per induzione che:
	\[\sum_{i=0}^{n} i = \frac{n(n+1)}{2}\]

\end{exercise}
\section{Teorema del principio di induzione su \( PROP \)}
\begin{definition}
	\( P \subseteq PROP \)
	\begin{enumerate}
		\item Se \( P[\alpha ], \alpha \in AT \) e
		\item Se \( P[\alpha ] \Rightarrow P[(\neg \alpha )] \) e
		\item se \( P[\alpha ]\) e \(P[\beta ] \Rightarrow P[(\alpha \wedge \beta )],
		      P[(\alpha \vee \beta) P[(\alpha \to \beta )]\)\\
		      allora \( \forall \psi \in PROP \hspace{2mm} . \hspace{2mm} P[\psi]\)
	\end{enumerate}
\end{definition}

Con questo teorema si possono dimostrare intere proposizioni complesse dimostrando i pezzi
più piccoli (\textbf{\emph{sottoformule}}) come mostrato nella figura \ref{fig:dimostrazionecomplessa}.

\label{ex:dimostrazioneComplessa}
\begin{figure}[h]
	\centering
	\incfig[1]{dimostrazionecomplessa}
	\caption{Dimostrazione di una formula complessa}
	\label{fig:dimostrazionecomplessa}
\end{figure}

\pagebreak

\begin{exercise}
	\label{ex:parentesiPari}
	Dimostra che ogni \( \psi \in PROP \) ha un numero pari di parentesi usando
	il principio di induzione per dimostrare proprietà sintattiche sulla struttura
	delle formule.

	\( P[\psi] \equiv \) \( \psi \) ha un numero pari di parentesi
	\begin{enumerate}
		\item \textbf{Caso base} \( \psi \in AT \) quindi \( \psi \) ha 0 parentesi
		      e quindi è pari: \( P[\psi] \) \( \surd \)
		\item \textbf{Ipotesi induttiva} \( \alpha , \beta \in PROP, \) \( P[\alpha ], P[\beta ] \)
		      ? \( P[(\alpha \to \beta )] \) (per \( \alpha  \) vale e per \( \beta  \) vale, si sono
		      aggiunte due parentesi, quindi la formula è ancora pari)
		\item \textbf{Passo induttivo} \( P[\alpha], P[\beta] \Rightarrow P[(\alpha \to \beta )], P[(\alpha \vee \beta )],
		      P[(\alpha \wedge \beta)]\) allora \( \forall \psi \in PROP \hspace{2mm}.\hspace{2mm} P[\psi] \)

	\end{enumerate}
\end{exercise}

\section{Definizione ricorsiva di funzioni su PROP}
\begin{definition}
	Riprendendo l'esercizio \ref{ex:parentesiPari} si definisce la funzione \( \pi \) che associa ad ogni formula
	proposizionale (equivalente di un input nell'informatica) un numero naturale
	(equivalente di un output nell'informatica). La funzione \( \pi \) quindi dopo aver dato in input
	un argomento (qualsiasi formula proposizionale atomica o complessa) restituisce in output
	il numero di parentesi che contiene la formula in input.
	\[ \pi : PROP \to \mathbb{N} \]
	\begin{enumerate}
		\item \textbf{Caso base} \( \pi [\alpha] = 0 \) se \( \alpha \in AT \)
		\item \textbf{Ipotesi induttiva} \( \pi[(\neg \alpha)] = \pi[\alpha ] + 2 \) In questo passaggio viene
		      chiamata la funzione \( \pi \) dentro la funzione \( \pi \) stessa, quindi è una definizione
		      ricorsiva. In questo caso si aggiungono 2 parentesi al numero di parentesi di \( \alpha \) \(\pi[\alpha] \)
		\item \textbf{Passo induttivo} \( \pi[(\alpha \to \beta )] = \pi[(\alpha \vee \beta)] =
		      \pi[(\alpha \wedge \beta )] = \pi[\alpha] + \pi[\beta] + 2\) dove \( \pi[\alpha ] \) e \( \pi[\beta ] \)
		      sono il numero di parentesi di \( \alpha \) e \( \beta \) e si aggiungono 2 parentesi per
		      il connettivo.
	\end{enumerate}
\end{definition}

Di seguito ci sono 2 esempi in cui viene messa in pratica la funzione \( \pi \) definita sopra in
modo da capire meglio come funziona.

\begin{example}
	\[
		\pi[(p_{2}\to p_{1})]\stackrel{caso \:3}{=}\pi[p_{2}]+\pi[p_{1}]+2\stackrel{caso \:1}{=}0+0+2=2
	\]
\end{example}
\begin{example}
	\[
		\pi[(p_{1}\vee (p_{2}\vee p_{1}))] = (\pi[p_{2}] + \pi[p_{1}] + 2) + \pi[p_{1}] + 2 = (0 + 0 + 2) + 0 + 2 = 4
	\]
\end{example}

Tutte le funzioni definite ricorsivamente sono funzioni, e non tutte le funzioni possono essere definite ricorsivamente.

\subsection{Definizione più precisa dell'esercizio \ref{ex:parentesiPari}}
Ogni \( \alpha \in PROP \) ha un numero pari di parentesi: \( \forall \alpha \in PROP \) \( P[\alpha] \stackrel{sse}{\Leftrightarrow} \pi[\alpha] \) è pari
\begin{enumerate}
	\item \( P[\alpha ] \) \( \alpha \in AT \)

	      se \( \alpha \in AT \! \) \( \pi[\alpha]\stackrel{def}{=}0 \) quindi \( \surd \)
	\item Suppongo che valga \( P[\alpha ],\; P[(\neg \alpha )] \: ? \)

	      \( P[\alpha ] \Leftrightarrow \pi[\alpha ] pari \) è pari perchè lo abbiamo supposto prima (consideriamo 0 come pari)


	      \( \pi[(\neg \alpha )] = \pi [\alpha ] + 2 \) è pari quindi \( P[(\neg \alpha )] \; \surd \)

	      Si può definire un simbolo nuovo che non vuole dire niente nel linguaggio proposizionale e
	      gli si assegnano i connettivi possibili per non doverli più scrivere ogni volta.
	      Per questo esercizio prendiamo in considerazione \[ \circ \in \{\to , \vee, \wedge\} \]

	\item \( (\alpha \circ \beta) \)

	      suppongo \( P[\alpha ], P[\beta ] \)

	      allora \( \pi[\alpha ] \) e \( \pi[\beta ] \) sono pari

	      quindi \( \pi[(\alpha \circ \beta )] = \pi[\alpha ] + \pi[\beta ] + 2 \;\) (è pari)

\end{enumerate}

Ho dimostrato per induzione che \( \forall \psi \in PROP \; P[\psi] \; \) \( \Box \)

(\( \Box \) è un simbolo che indica la fine della dimostrazione.)

\section{Dimostrazione ricorsiva di rango e sottoformula}
Il rango di una formula è il numero di connettivi che contiene.
\begin{definition}
	Considerato \( r \) il rango di una proposizione

	\( r: PROP \to \mathbb{N} \)

	\begin{enumerate}
		\item \( r[\psi] = 0 \) se \( \psi \in AT \)
		\item \( r[(\neg \psi)] = 1 + r[\psi] \)
		\item \( r[(\psi \circ \gamma )] = 1 + max(r[\psi], r[\gamma]) \hspace{5mm} \) \( \circ \in \{\vee, \wedge, \to \} \)

	\end{enumerate}
\end{definition}

\end{document}
